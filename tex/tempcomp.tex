\chapter{Эпиграфы}

\epigraph{
\mulang{$0$}
{Начинай, если хочешь.}
{Begin if you want.}
\mulang{$0$}
{Продолжай, если нравится.}
{Continue if you like.}
}{Пословица ноа}

\epigraph
{Если вам говорят о том, что вы обязаны любить "--- вас пытаются изнасиловать.
Кричите как можно громче.}
{Ветер-Стрекозьих-Крыльев, идеолог ранних тси}

\epigraph
{Задача правителя "--- указать человеку его место.
Задача хорошего правителя "--- помочь человеку отыскать своё место.
Задача лучшего из правителей "--- не мешать человеку искать своё место.}
{Франциск IV, последний папа римский}

\epigraph
{Небесным телам, листьям деревьев, глазам птиц и человеческой крови совершенно не важна красота твоих изречений.
У них своя поэзия и свои законы.}
{Эрхэ Колокольчик}

\epigraph
{Каким, должно быть, смелым было животное, которое впервые в истории почувствовало боль.}
{Длинный-Мокрый-Хвост}

\epigraph
{Большая часть высоких технологий замечательно просты по своей сути.
Я не верю, что даже в случае глобального катаклизма мы скатимся до каменных молотков.
Пока жив хотя бы один человек с горящими глазами "--- технологии будут сохранены, будут приспособлены к инфраструктуре и будут применяться.
Как человек не может разучиться велосипедной езде или плаванию, так и человечество, единожды научившись, уже не сможет это позабыть.}
{Михаил Кохани, презентация <<спичечных>> технологий в Массачусетском технологическом университете}

\epigraph
{Среднестатистический практически здоровый человек способен встроиться в любую социальную систему, имеющую простые и понятные правила.
<<Любую>>, к сожалению, означает и <<сколь угодно бесчеловечную>>.}
{Мариам Кивихеулу}

\epigraph
{Не тратьте время на сожаления.
Ваши предпочтения "--- это не судьба, а результат удачного стечения обстоятельств.
Кое-кто считает, что три поворота налево и один направо приводят к одному и тому же результату.
Как показывает практика, результат в этих случаях всегда разный.}
{Людвиг Вейерманн}

\epigraph
{Все хотят тепла, но никто не хочет гореть.}
{Пословица Преисподней}

\epigraph
{\dots И лишь одного я боялась всегда "---\\
Что море меня не излечит.}
{Эрхэ Колокольчик}

\epigraph
{Будьте гордыми.
То, что вас согнули сегодня "--- лишь стечение обстоятельств.
Меня часто унижали, когда я не могла дать отпор.
Но и пусть.
Я ничего не прощала и не прощу.
Вставайте на колени, если вас ставят насильно.
Молчите, если вам затыкают рот кляпом.
Признавайтесь, если вам угрожают оружием.
Это и вполовину не так ужасно, как оправдывать преступления или отыгрываться на других.
Будьте гордыми, берегите себя, свои силы и достоинство "--- завтра ветер переменится.}
{Мариам Кивихеулу}

\epigraph
{Научись улыбаться в ответ.}
{Призказка ноа}

\epigraph
{Грань между защитой и нападением тоньше волоса.
Сомневаешься "--- оставь меня в ножнах.}
{Гравировка на сабле Митхэ ар'Кахр}

\epigraph
{Жить интересно именно потому, что мир не соответствует нашим ожиданиям.}
{Людвиг Вейерманн}

\epigraph{
\mulang{$0$}
{Битва может стоить жизни, но без битвы ты не узнаешь, что такое жизнь.}
{Battle may take your life, but escaping battle you'll never know what life is.}
}{Присказка наёмников Фоуф}

\epigraph
{Мир изменится.
Он уже меняется.
И это случится, даже если вы сегодня проломите мне голову.
Потому что любовь и здравый смысл хоть и не сразу, но всегда побеждают ненависть и абсурд.
Так устроен этот мир, и я тут ни при чём.}
{Мариам Кивихеулу}

\epigraph
{Уничтожение произведений искусства должно стать табу для любого политического или общественного деятеля.
Вандал, какие бы прогрессивные идеи он ни нёс, никогда не вызовет сочувствие у общества.}
{Мартин Охсенкнехт}

\epigraph
{Ах ты, молодой Добрыня Никитич!
Бился ты со змеёй да трое суток, потерпи ещё три часа!
Ты побьёшь змею да ю, проклятую!}
{Фольклор культуры Руса, Древняя Земля}

\epigraph
{Понесший наказание безвинно имеет право совершить преступление того же характера, если невиновность будет доказана.}
{Третий постулат Возмездия.
Законы ноа}

\epigraph
{Если вы хоть раз в жизни выключили компьютер, разъединили телефонное соединение, убили комара или вкололи себе антибиотик, вы уже владеете основами интерфекции.}
{Элла Рид, основатель интерфекции}

\epigraph
{Лучший учитель "--- ученик, который только что понял.}
{Пословица сели}

\epigraph
{Если дуть против ветра, то ветер не изменится, но зато перед твоим носом всегда будет островок штиля.}
{Пословица ноа}

\epigraph
{Есть люди, подобные кострам "--- кормят и согревают целый лагерь, но тухнут в одиночестве.
Есть люди, подобные маякам "--- не греют, но светят, даже если на горизонте нет ни одного корабля.}
{Пословица ноа}

\epigraph
{Может быть, стремление к власти и портит людей, но угроза потери власти превращает их в диких зверей и лишает всего человеческого.}
{Мариам Кивихеулу}

\epigraph
{Кости не ломаются от усталости, кости ломаются от чрезмерных усилий.
Стену следует строить маленькими камешками.}
{Пословица ноа}

\epigraph
{Если замолкает последняя певчая птица "--- значит, дело действительно плохо.}
{Пословица сели}

\epigraph
{Кровь вытекает вместе с заразой, страх вытекает вместе с волнением.}
{Присказка хака}

\epigraph
{Сплошные солнечные дни порождают пустыню.}
{Пословица сели}

\epigraph
{Лучше быть врагом хорошего человека, чем другом плохого.}
{Пословица Преисподней}

\epigraph
{Быстро "--- это медленно, но каждый день.}
{Пословица сели}

\epigraph
{Лето "--- это танец, и глупо не принимать в нём участия.}
{Пословица Драконьей Пустоши}

\epigraph
{Иерархия разрушается в тот момент, когда члены сообщества отказываются играть по его правилам, добровольно меняя возможное лидерство на стигму нижней ступени.
Эти люди показывают прочим главное "--- можно жить, самореализовываться и радоваться жизни, будучи <<отбросом общества>>.
Поэтому лидеры будут препятствовать этому, делая жизнь таких <<добровольцев>> невыносимой и даже устраняя их физически.}
{Мариам Кивихеулу}

\epigraph
{Ищи счастье.
Судьба найдёт тебя сама.}
{Клаудиу Дентосиу}

\epigraph
{Самая ужасная для государства вещь "--- страх воина.
Страх воина питает диктатуру.
Внуши воину, что он в безопасности "--- и диктатура падёт.
Внуши воину, что он любим народом "--- и он будет с народом до конца.
Пусть каждый народ поклянётся оберегать своих воинов, как воины клянутся защищать народ "--- и для человечества не будет больше плохих лет.}
{Анатолиу Тиу.
<<Послание к девяти завоевателям>>}

\epigraph
{Я несчастен, но счастливее меня не найти.}
{Людвиг Вейерманн.
Предсмертная записка}

\epigraph
{Если интерес к происходящему пересилил прочие чувства "--- значит, вы победили в главном сражении своей жизни.
Все остальные победы "--- вопрос времени.}
{Михаил Кохани.
Речь на вручении Расширенной Нобелевской премии.}

\epigraph
{Крысы боятся света.}
{Фризская пословица}

\epigraph
{Выучи один язык "--- второй дастся тебе проще.\\
Научись играть на флейте "--- цитрой овладеешь без труда.\\
Подружись со швейной иглой "--- резец плотника сам прыгнет в твои руки.}
{Пословица ноа}

\epigraph
{Я меняю дни на расстояние\\
От Тси-Ди до джунглей Тра-Ренкхаля,\\
Мне доступно тайное знание "---\\
Как свернуть пространство-время желанием.}
{Песенка Заяц}

\epigraph
{Я никогда не был на войне.
Но если бы пришлось, я стал бы диверсантом или дезертиром "--- и то и другое требует большой смелости.}
{Бенедикт Альсауд}

\epigraph{
\mulang{$0$}
{Уважение тебе ничего не стоит.}
{Respect costs you nothing.}
}{Мариам Кивихеулу}

\epigraph
{Подумать надо и о них "---\\
Что надо, то надо,\\
Да вот поди-ка отыщи\\
Отбившихся от стада!\\
~\\
Живым идти под вострый нож\\
Тож неохота им,\\
Ищи! Но то, что ты найдёшь,\\
Будет ли живым?}
{Торстейн фра Хамри.
<<Поздней осенью в поисках овец>>}

\epigraph
{Вклад личности в историю принципиально не поддаётся оценке.
Любые изыскания на эту тему "--- не более чем спекуляция.}
{Людвиг Вейерманн}

\epigraph
{Проблема мира "--- не в отсутствии любви, а в неумении её выражать.}
{Мариам Кивихеулу}

\epigraph
{Мы меняемся вместе с миром, словно рисунок небес, и если среди ясного дня грянул гром "--- значит, ты просто не заметил грозу.}
{Пословица тенку}

\epigraph
{Великий умеет находить своё истинное предназначение.
Счастливый умеет вовремя от него избавиться.}
{Клаудиу Дентосиу}

\epigraph
{Есть три верных признака государства, находящегося в состоянии гражданской войны "--- разбитые дороги, разрушенные больницы и закрытые школы.
Хуже всего, если разбиты дороги в городах и деревнях.
Ведь для хороших дорог нужно лишь, чтобы два добрых соседа по две стороны от дороги могли что-то сделать сообща.}
{Марке Скрипта}

\epigraph
{Мироздание не знает, что такое благодарность.
Делаешь что-то полезное "--- позаботься о награде сам.}
{Длинный-Мокрый-Хвост}

\epigraph
{Дорога излечит всё.
Или убьёт.}
{Пословица ноа}

\epigraph
{Неопытный кушает впрок;
мудрый берёт пищу с собой.}
{Пословица сели}

\epigraph
{<<Возлюби ближнего своего>> прежде всего означает <<Оставь ближнего своего в покое>>.}
{Фридрих Ницше.
Эпоха Господина}

\epigraph
{Старинный рецепт идеальных солдат: взять молодых, лишить знаний и наставлений, закрыть в клетки и вынудить выживать, а потом дать выжившим идеологию и пообещать им любые блага.}
{Присказка наёмников Фоуф, Тысяча Башен}

\epigraph
{Если в войне между правителями участвуют солдаты, но нет наёмных убийц "--- никакой войны нет, есть лишь игра.
Только игроки жертвуют фигурами, не пытаясь сунуть друг другу нож.}
{Присказка наёмников Фоуф, Тысяча Башен.}

\epigraph
{Смерть может быть привлекательной в двух качествах "--- как выход и как неизведанное.}
{Постулат Эволюциона}

\epigraph
{Когда с собаки снимают чересчур суровый ошейник, она рычит и кусается от боли.}
{Клаудиу Дентосиу}

\epigraph
{Я бы тоже хотел мир, где мне не пришлось бы доказывать своё право, ломая чужие копья, разрывая путы и обходя волчьи ямы.
Но мира кроме этого у меня просто нет.}
{Гало Кровавый Знак}

\epigraph
{Жестокость "--- храбрость трусов.}
{Фазиль Искандер, Эпоха Последней Войны.}

\epigraph
{Тюрьма "--- это судьба любого преступника.
Диктаторы и коррумпированные чиновники сами строят вокруг себя тюрьмы, и эти тюрьмы лишь по недоразумению называют дворцами и крепостями.}
{Анатолиу Тиу}

\epigraph
{At satis hostium.
Эта фраза была эпитафией некоего Матиаса Стойса, похороненного в Кафедральном соборе Кёнигсберга.
Сейчас, к сожалению, барельеф не сохранился, но его застал мой прапрадед, Алекс Орлов.
Он был военным.
И однажды увидел эту надпись, прогуливаясь по городу.
Через несколько дней он уволился из армии, а ещё через год основал Общество Сломанного Копья, помогающее военным адаптироваться к гражданской жизни, получить образование и работу.
Сейчас эта фраза "--- at satis hostium "--- выбита на его могиле.
Прапрадед был убит милитаристами спустя семь лет, когда Общество Сломанного Копья распространилось по трём континентам.}
{Мемуары Михаила Кохани}

\epigraph
{Когда человек мучается, у него появляется потребность мучить других.}
{Ромен Роллан.
Эпоха Последней Войны}

\epigraph
{At satis hostium\footnote
{Но хватит врагов (эллатинский). \authornote}.}
{Эпитафия Матиусу Стойусу Прусскому.
Кафедральный собор Кёнигсберга, домен Европа, Древняя Земля}

(Последняя встреча Гало и Тахиро!)

\epigraph
{Лгать "--- как нести скользкий чан под дождём: чем дальше, тем тяжелее;
не уронишь чан, так расплещешь воду.}
{Пословица ноа}

\epigraph
{Правду можно оставить на дороге;
ложь приходится нести с собой.}
{Пословица сели}

\epigraph
{Хрупкая птица пролетит расстояние, которое не сможет пройти самый сильный ягуар.}
{Пословица сели}

\epigraph
{Вера обычно начинается там, где кончаются силы и фантазия.}
{Пословица сели}

\epigraph
{Пройди лигу\footnote
{Лига "--- мера длины ноа. Соответствует ровно 1.5 кхене. \authornote},
перекати бусинку.
Чётки покажут путь, ноги приведут.}
{Присказка ноа}

\epigraph
{Даже самый могущественный "--- часть целого.
Помня об этом, правитель не опустится до диктатора, а народ не допустит попрания своих прав и свобод.}
{Анатолиу Тиу}

\epigraph
{Королева не кормит грудью, король не учит сыновей, братья режут друг друга ради трона, а дочерей используют как разменную монету.
Королевская семья "--- извращение самой сути семьи.
Должна ли она говорить добрым талианцам, как им жить?}
{Тациан Освободитель}

\epigraph
{Учи дитя порядку, но помни, что дисциплина "--- кормилица лжи.}
{Поговорка сели}

\epigraph
{Совершать безумства "--- это способность.
Совершать безумства без сожалений "--- это дар.}
{Михаил Кохани}

\epigraph
{Нельзя недооценивать Эпоху Последней Войны.
Именно тогда, в огне вооружённых конфликтов, революций, забастовок, военных переворотов и уродливых, саморазрушительных идеологий ковалась самая жизнеспособная, самая гуманная философия первых людей, обеспечившая почти двадцатидвухтысячелетний период мира, процветания и расселения Ветвей Земли по Вселенной.}
{Кельса Пушистая}

\epigraph{
\mulang{$0$}
{Счастливые женщины "--- хорошая деревня.}
{Happy women mean a healthy village.}
}{Хаяо Миядзаки}

\epigraph{
\mulang{$0$}
{Есть высоты, на которых худший способ висеть лучше падения.}
{There are some heights, where the worst way to hang on is better than to fall off.}
}{Пословица хака}

\epigraph
{Если из жилища видно не красоты природы, а лишь окружающие его стены "--- это осквернение самой идеи жилища.}
{Клаудиу Дентосиу}

\epigraph{
\mulang{$0$}
{Всегда будь готов к тому, что ничего не произойдёт.}
{Always expect nothing to happen.}
}{Длинный-Мокрый-Хвост}

\epigraph
{Великий талант "--- жить в неопределённости без иллюзий и без тревог.}
{Людвиг Вейерманн}

\epigraph
{Мы не шли на войну, чтобы убивать или быть убитыми.
Мы шли на войну, чтобы нас услышали.}
{Субкоманданте Инсургенте Маркос.
Эпоха Последней Войны}

\epigraph
{Что может быть прекрасней, чем дышать?\\
Быть может, к новой жизни\\
родиться из тяжёлых снов дневных?\\
~\\
Проснулся "---\\
и уже не помнишь,\\
что разбудило:\\
достаточно любить себя,\\
чтобы проснуться.\\
~\\
А нетерпенья\\
достаточно, чтобы сгореть\\
для повторенья.}
{Торстейн фра Хамри, <<Нетерпение>>.
Эпоха Последней Войны, Древняя Земля}

\epigraph
{Утонул в колодце "--- беда всего города.
Утонул в море "--- только твоя беда.}
{Пословица ноа}

\epigraph
{Все жизненные пути бессмысленны, но есть путь сердца.
Он такой же бессмысленный, как и остальные, но по нему идёшь с радостью.}
{Карлос Кастанеда}

\epigraph
{Экспертом является тот, кто совершил все возможные ошибки в некотором узком поле.}
{Нильс Бор}

\epigraph
{В человеке нет ни одного качества, которое когда-либо не поспособствовало его выживанию.
Просто помните об этом, когда решите что-либо осудить.}
{Март <<Одноглазый>> Митчелл}

\epigraph
{Многие сейчас говорят о бедах, которые принесли слепая вера и фанатизм.
Упаси меня Господь отрицать очевидное.
Но нужны ли они человечеству?
Нужны!
Не расчётливые заселили самые дальние берега Земли, и не расчётливым идти на край Вселенной!}
{Март <<Одноглазый>> Митчелл}

(этот эпиграф однозначно к <<Кон-Тики>>)

\epigraph
{Он обрёл почти безграничную власть, просто перестав осуждать.
Эта власть ограничивалась лишь его собственными принципами.}
{<<Речь о жреце>>.
Речь ноа}

\epigraph
{Насыщает похлёбка, а не просиженное в харчевне время.}
{Пословица сели}

\epigraph
{Спроси себя в печали, спроси себя истекающего кровью, спроси себя усталого и измотанного, желал бы ты иной жизни?
Так познаются цветение духа и правильность пути.}
{Мартин Охсенкнехт}

\epigraph
{Соседи твои и друзья\\
Ходят по городу сытые,\\
Но за смехом таится молчание,\\
А в молчании дремлет бунт.}
{Торстейн фра Хамри, <<Дух времени>>.
Эпоха Последней Войны, Древняя Земля}

\epigraph
{Умеющий шагать не оставляет следов.
Умеющий говорить не допускает ошибок.
Кто умеет считать, тот не пользуется счетом.
Кто умеет закрывать двери, тот не употребляет запор и закрывает их так крепко, что открыть их невозможно.
Кто умеет завязывать узлы, тот не употребляет веревку, и завязывает так прочно, что развязать невозможно.}
{<<Книга Пути и Достоинства>>, Мудрый Старец.
Культура Цина, Древняя Земля}

\epigraph
{Познание себя и прочей Вселенной имеет чересчур памятный горько-сладкий вкус;
того, кто распробовал его однажды, может остановить только смерть.}
{Клаудиу Семито Фризский.
Эпилог к <<Воспоминаниям о великом тёзке>>}

\epigraph
{Чем больше обязанностей по самообеспечению вы возлагаете на окружающих, тем сильнее от них зависите.
Не добываете руками хлеб?
Приготовьтесь голодать.
Не умеете мыть полы?
Приготовьтесь существовать в грязи.
Лет триста назад я бы воскликнул: <<И не приведи судьба вам отдать ртуть\footnote
{На Драконьей Пустоши в качестве валюты использовали металлическую ртуть. См. \textit{Фиола}. \authornote}
и железо в чужие руки!>>
Малу вайю\footnote
{Malu vaeu "--- Увы! О горе! Чёрт возьми!\textit{(s-l, примерный перевод)}. \authornote},
поздно "--- это наша действительность.
Казначеи и ростовщики грабят вас, выменяв вашу ртуть на бумагу.
Король велит закапывать фальшивомонетчиков, но при каждом удобном случае платит вам медной амальгамой.
Солдаты и стражники убивают вас оружием, которое выковали вы из вами добытого железа.
И самое страшное "--- вы всё ещё считаете, глупцы, что это в порядке вещей.}
{Анатолиу Тиу.
Речь перед жителями Фриза}

\epigraph
{Твердое и крепкое это то, что погибает, а нежное и слабое есть то, что начинает жить.
Сильное и могущественное не имеют того преимущества, какое имеют нежное и слабое.}
{<<Книга Пути и Достоинства>>, Мудрый Старец.
Культура Цина, Древняя Земля}

\epigraph
{Принятие конкретной социальной установки "--- это всегда давление на суперпозицию личности.
Счастье в том, чтобы давление казалось объятиями, а не ломало рёбра.}
{Мариам Кивихеулу}

\epigraph
{Жизнь слепа и путь прокладывает на ощупь.
Высокий интеллект "--- это та широкая лазейка, которую жизнь нащупала в практически непреодолимом для неё препятствии "--- космическом пространстве.}
{Софиа Ловиса Карма}

\epigraph
{В искусстве невозможно опоздать.
Когда бы ни было создано произведение, оно появляется вовремя.}
{Мартин Охсенкнехт}

\epigraph
{Тот, кто впервые обругал соплеменника вместо того, чтобы ударить его, стал прародителем цивилизации.}
{Длинный-Мокрый-Хвост}

\epigraph
{Биология с её естественным отбором гуманнее большинства представителей человеческого рода.
Так что оправдывать несправедливости биологией оставьте лжецам для неучей.}
{Сергей Леонидовиц Хистиаков}

\epigraph
{Если какой-то деятель искусства скажет при вас, что художники, писатели, поэты и драматурги двигают мир вперёд "--- можете громко рассмеяться ему в лицо.
Мир двигали и двигают те, кто изредка, раз в месяц или год идёт в театр или читает книжку, чтобы расслабиться и назавтра, после здорового крепкого сна снова делать то, что должно "--- подметать улицу, чинить машины, учить детей и лечить больных.}
{Мартин Охсенкнехт}

\epigraph
{Всё, что пытаются доказать силой "--- розгами, кулаками, резиновыми дубинками или пулемётами "--- скорее всего, ложь;
ни одному первооткрывателю ещё не приходилось бить или убивать оппонентов, чтобы убедить их в своей правоте.}
{Михаил Кохани}

\epigraph
{Дьявол начинается с пены на губах ангела, вступившего в бой за правое дело.}
{Грегори Померантс}

\epigraph
{Каждое заблуждение имеет цену, и эта цена всегда измеряется в человеческих жизнях.
Исключений не бывает.}
{Марке Скрипта}

\epigraph
{Врач и философ Марке Скрипта является изобретателем термина <<тирания мозга>>.
Так он называл состояние, когда человек, отождествляя себя исключительно с нервной системой, сосредотачивается на её внутренних задачах, игнорируя или непомерно эксплуатируя при этом прочие системы организма.
Это же состояние я наблюдаю сейчас в нашем государстве;
Скрипта полагал, что тирания мозга есть первопричина многих болезней тела и всех болезней души.}
{Анатолиу Тиу}

\epigraph
{Тси вернутся.
Они вырвутся в явь из ваших кошмаров.
Они пробьются светом через ваши решётки, просочатся влагой через гранит ваших стен.
Они не станут мстить, но их возвращение положит вам конец.}
{Уэсиба Серозмей}

\epigraph
{У народа, живущего в долине посреди Хребта Малого Листопада, есть замечательный обычай.
Если кто-то совершил преступление, люди начинают заботиться о нём.
Ему напоминают о его самых лучших поступках, его обнимают и всячески проявляют к нему любовь.
Я скажу вам, что это маленькое отсталое племя знает истину, которую не могут осознать самые лучшие умы государя: преступник "--- это тот, кто кашляет, когда общество больно.}
{Анатолиу Тиу.
<<Земной суд>>}

\epigraph
{Если вы услышите от меня в старости, что я отказываюсь от сегодняшних слов, пожалейте меня, но не верьте сказанному.
Предел есть у каждого.
Среди стариков много уставших, больных и сломленных;
трудно надеяться, что сия чаша минет мою голову.
Не верьте и прочим старикам, что публично отрекаются от прекрасных идей молодости.
Хоть историю и пишут достигшие преклонных лет, молодые гораздо чаще знают истину.
Так было и будет всегда.}
{Михаил Кохани}

\epigraph
{Ещё одно существенное отличие сели: их мифология носила \textit{исключительно игровой} характер.
Примером могут служить приведённые результаты опроса по <<небесному пахарю>>[44][136].\\
Взрослые[88] сели:\\
94\% при вопросе о сущности падающих звёзд, смеясь, рассказывали легенду о небесном пахаре, но при переводе беседы в серьёзное русло всё-таки рассказывали о летающих раскалённых камнях;\\
6\% безо всяких шуток давали совершенно точный ответ.\\
Взрослые[88] тенку:\\
10\% (в основном образованные монахи) давали правильный ответ и выказывали неудовольствие, услышав легенду о пахаре;\\
62\% совершенно серьёзно говорили о проделках богов и удивлялись рассказу о летающих камнях;\\
28\% промолчали или ответили <<не знаю>>, отклонив таким образом предложение побеседовать.\\
Ещё сильнее эта разница заметна у детей.\\
Прошедшие Отбор дети[88] сели:\\
78\% рассказали легенду о пахаре.\\
После вопроса о том, как всё обстоит <<на самом деле>>:\\
8\% затруднились ответить;\\
18\% спросили <<как?>>;\\
12\% начали фантазировать,\\
40\% упомянули о раскалённых угольках, летящих из очага\\
(таким образом детям объясняют сущность небесных тел в школе).\\
22\% детей отказались беседовать на обозначенную тему.\\
Выжившие дети[88] тенку:\\
56\% рассказали легенду о пахаре,\\
42\% уверенно повторили её после уточняющего вопроса,\\
4\% спросили <<как?>>,\\
10\% начали фантазировать.\\
44\% отказались беседовать на обозначенную тему.}
{Аркадиу Шакал Чрева.
<<Сели: первые впечатления>>}

\epigraph
{Генотип, медикация и обучение каждого тси должны обеспечивать отсутствие статистически значимой разницы между значениями двух параметров "--- выживаемости и коэффициента личного комфорта\footnote
{Коэффициент личного комфорта (КЛК) "--- параметр, определяющий степень раскрытия возможностей сапиента в заданных условиях.
В настоящее время считается устаревшим. \authornote}
"--- в присутствии технологической поддержки и при полном отсутствии таковой.}
{Второй постулат Кодекса Тси-Ди}

\epigraph
{Не найти прекраснее цветов преисподней, но цветение их мимолётно.}
{Анатолиу Тиу.
<<Путешественник>>, диалоги}

\epigraph
{"--*Я слеп, отче.
Как мне увидеть, что есть добродетель и порок?\\
"--*Всё вышло из крови и уйдёт в землю, сын.}
{Анатолиу Тиу.
<<Путешественник>>, диалоги}

\epigraph
{\dots Увы, но бывают моменты, когда на одной чаше весов лежит отрубленная рука, а на другой "--- судьба, жизнь и смерть.
Казалось бы, выбор очевиден для всех "--- спросите любого зеваку на улицах нашей столицы.
Но когда этот момент настает, рука почему-то перевешивает\ldotst}
{Анатолиу Тиу.
Трактат <<О природе дышащих>>}

\epigraph
{Салафиты нарисовали на своём знамени перечёркнутый посох?
Понимаешь ли ты, что это значит?
На их знамени "--- твой знак, Леам!
Они признали поражение, признали, что в дереве их собственного учения соков больше нет\ldotst}
{Анатолиу Тиу.
Письмо к Леаму эб-Салаху}

\epigraph
{Некоторые\ldotst попрекают меня стыдом предков.
Я отвечаю всегда, что есть кое-кто, чьего стыда я боюсь куда больше.
Забудьте мёртвых "--- они отжили своё и сделали всё, что могли.
А мириады ещё не рождённых ежемоментно следят за вами, волнуются, радуются и грустят, потому что от каждого вашего шага зависит их судьба на этой земле.}
{Анатолиу Тиу.
<<Буря перемен>>, часть III}

\epigraph
{Ты говоришь, что принёс свободу, и гордишься данными тебе народом титулами?
Ты раб с рождения, Татиан, ещё худший, чем те, кого ты за собой ведёшь.
Меня называют светилом, философом, певцом свободы, но единственная свобода, которую я хочу "--- это свобода не быть тем, кем меня называют, и не делать то, что от меня ждут!}
{Анатолиу Тиу.
Ответ на призыв Татиана Освободителя оказать его движению поддержку}

\epigraph
{У гаруспиков всегда будет множество клиентов.
Людей волнует, когда они разбогатеют, найдут любимого человека, умрут\ldotst
Меня волнует только один вопрос "--- когда я окончательно сдамся.
За ответ я бы отдал всё своё состояние, но увы "--- его не даст ни один гаруспик.}
{Анатолиу Тиу.
Речь перед лиманскими повстанцами}

\epigraph
{Для нас есть вероятности, но нет предопределения.
Для мироздания есть предопределение, но нет вероятностей.
Непроницаемое тёмное пятно вероятностей в вашем будущем "--- это и есть то, что вы называете <<свободой воли>>.}
{Анатолиу Тиу.
Ответ на шариатском суде}

\epigraph
{Сожжённым вами люди будут дышать.}
{Последние слова Анатолиу Тиу перед смертью.
Шариатский суд постановил <<опалить лицо>> уже престарелого Тиу лишь теми книгами, в которых явно доказано присутствие ереси.
К чести инквизиторов и к несчастью старика, таких книг хватило на аутодафе.}

\epigraph
{То, что было взято дланью войны\footnote
{Дланью войны (t-sl: way mana militharrhio) "--- с помощью военной силы. \authornote},
будет потеряно\ldotst
Завоеванная женщина не будет верна;
завоеванная страна вернет себе свободу, едва окрепнув, часто под знаменами шовинизма и религиозно-морального фанатизма, этих извечных суррогатов потерянного достоинства.
Посему опытный правитель должен делать ставку на договорные отношения\ldotst и помнить, что слабые нуждаются в уважении куда больше сильных.}
{Анатолиу Тиу.
<<Послание девяти завоевателям>>}

\epigraph
{В семечке "--- фрукты и алый цветок.\\
В ножнах таится убийцы бросок.\\
Форму узри "--- и узришь содержанье.\\
В лицах и благо ищи, и порок.}
{Клаудиу Семито Фризский.
<<Песенки Клаудиу>>, приложение III к <<Воспоминаниям о великом тёзке>>}

\epigraph
{"--*Ты обвиняешься в организации культа с целью подрыва королевской власти, "--- монотонно сказал судья.
"--- Что ты можешь ответить на это?\\
"--*Культ правильности сменяется культом безумия.
Культ поклонения сменяется культом противостояния.
Ваше обвинение похоже на обвинение пастуха, у которого угнали стадо баранов.\\
Люди в зале "--- как сторонники, так и противники Клаудиу "--- вскочили на ноги и начали выкрикивать в адрес поэта оскорбления.
И только три или четыре человека молча чему-то улыбнулись\ldotst}
{Клаудиу Семито Фризский.
<<Воспоминания о великом тёзке>>}

\epigraph
{Клаудиу в последний раз посмотрел на оставленный им город.
Люди вокруг яростно смотрели на него и тискали пальцами оружие, но никто не решался напасть "--- поэта верно хранило мастерство убийцы и королевский эдикт.\\
<<\ldots из величайшей королевской милости рекомого Клаудиу отпустить и хранить его от всяческих посягательств>>.
Милости и страха перед восстанием.
Никто не хотел, чтобы Клаудиу превратился в легенду и знамя для обездоленных.\\
Поэт поправил котомку и, весело насвистывая любимую песенку, направился в сторону побережья.
Он не подозревал, что стал легендой уже при жизни.}
{Клаудиу Семито Фризский. Эпилог к <<Воспоминаниям о великом тёзке>>}

\epigraph
{Нет любви без знания, а невежество "--- лучшая пища для ненависти.}
{Пословица народа сели}

\epigraph
{Когда Клаудиу подвели к гильотине, один из министров Валериу Х насмешливо спросил, готов ли поэт умереть за свои идеи.
<<Я готов даже к тому, что они окажутся глупыми и никчёмными.
А уж умру я за них с большим удовольствием>>, "--- ответил Клаудиу и, не дожидаясь приказа, положил голову на плаху.}
{Клаудиу Семито Фризский. 
<<Воспоминания о великом тёзке>>}}

\epigraph
{Мало ему победы, он и смерть забрал!}
{Изречение, выбитое на гробнице Велира IV Хореида, принадлежавшее его врагу Вериту Валериду}

\epigraph
{Посмотри на волосы врага "--- и ты поймёшь, на каком языке пойдёт беседа.}
{Пословица народа сели.
Купцы, исполняющие роль дипломатов, носили хвостик или пучок на затылке и вели переговоры на языке цатрон.
Воины обычно стриглись коротко и использовали <<язык стали>> "--- боевые искусства.}

\epigraph
{Сжатая в кулаке крыса может отгрызть руку.
Загнанный в угол заяц может убить охотника.
Отчаявшийся тигр может истребить целую деревню.
На что же способен человек, которому нечего терять?}
{Пословица народа сели}

\epigraph
{Жизнь "--- испытание для любого сапиента.
Веками мы придумывали сказки, чтобы оградить себя от естественного желания положить ей конец.
Время сказок прошло.
Наша задача "--- помочь всем и каждому найти место в жизни, на деле, а не на словах убедить сапиентов, что их существование необходимо и страдания оправданы.}
{Постулат Эволюциона}

\epigraph
{Утопия подразумевает людей, которые довольны своей жизнью.
Однако всегда появляются бунтари, которых что-то не устраивает.
Человек меняется, следуя за условиями среды, общество меняется вслед за человеком.
Иметь претензии не есть плохо.
Желать изменений к лучшему "--- это не подрыв устоев, а движение вперёд.
Звучит парадоксально, но если общество претендует на звание идеального, оно должно быть готово к любым изменениям.}
{Софиа Ловиса Карма, идеолог Эволюциона}

\epigraph
{Как я стал капитаном <<Тёмного пламени>>?
(смеётся) Просто пришёл в эту их комиссию и сказал: что за дерьмо, у меня под килем слишком мало парсак!
И парсак мне сразу отсыпали\ldotst}
{Бенедикт Альсауд.
Запись беседы с историками}

\epigraph
{В океане и в космосе нет границ.
Если кто-то хочет расчертить воду и святые небеса, пусть сначала озаботится суверенитетом собственной задницы "--- возможно, она в рабстве у чересчур удобного кресла.}
{Бенедикт Альcауд.
Открытое письмо Транспортному комитету по поводу сомнительного законопроекта}

\epigraph
{Если ты проснулся бодрым "--- значит, для тебя наступило утро.}
{Пословица тенку}

\epigraph
{Идеальное общество "--- это общество, в котором даже чужак чувствует себя своим.}
{Критерий де Ла Роче.
Постулаты Эволюциона}

\epigraph
{Собака "--- друг человека.}
{Изречение времён Древней Земли}

\epigraph
{Когда тси только начали подготовку к одичанию, они столкнулись с неожиданной проблемой "--- прочие формы жизни Тра-Ренкхаля не могли соперничать даже с потерявшими большую часть технологий пришельцами.
Тси ловили зверей и птиц едва ли не голыми руками, что могло поставить десятки тысяч видов под угрозу исчезновения.
Воинственность аборигенов также могла сыграть с ними плохую шутку "--- если тси переняли бы этот стереотип поведения, то аборигенов ждало неминуемое истребление.
Можно утверждать, что эти немногие тси с одним космическим кораблём в одночасье стали хозяевами планеты, и даже демиург Безымянный не питал никаких иллюзий на этот счёт[12].\\
Именно поэтому Баночка и Кошка разработали так называемую тактику форы (подробнее см. раздел 13.7.1).
Записей об этой методике осталось немного[13], но результаты исследований[14][15][16][22] говорят сами за себя "--- тси провели беспрецедентный, грандиозный отрицательный отбор, сохранив многообразие видов.
Известно, что этот отбор производился с согласия и при поддержке Безымянного[12].
Именно благодаря дальновидности Баночки и Кошки люди Тра-Ренкхаля, нгвсо и дикие животные если и не стали тси достойными конкурентами, то хотя бы получили право на жизнь.}
{Аркадиу Шакал Чрева.
Тезисы работы <<Взаимодействие тси с биоценозом планеты Тра-Ренкхаль>>}

\epigraph
{Трусость "--- враг процветания, а жестокость "--- враг мира.
Да, я трус.
Во мне есть и жестокость.
Те, кто не знаком со мной лично, почувствовали эту жестокость в моём романе.
У многих из вас загораются глаза, когда вы читаете о крови и войне.
Мы "--- потомки тех, кто убивал и насиловал на протяжении тысяч лет, кто раболепно склонялся перед силой.
Это их воинственные крики мы слышим в беспокойных снах, это их мысли мелькают на краю сознания, едва в наших руках оказывается оружие, это их страх пробирает нас до костей, едва оружие направляют на нас.
Но, клянусь пред лицом Господа, даже с таким <<наследством>> можно жить в процветании и сохранять на Земле мир!
И для начала будем честны перед собой и признаем простой факт: в нас течёт кровь тех, кто выжил благодаря жестокости и трусости.}
{Март <<Одноглазый>> Митчелл, идеолог Эволюциона, один из последних казнённых преступников в эпоху Последней Войны}

\epigraph
{Когда Тси-Ди была повержена Машиной, мы думали, что последний оплот свободных сапиентов Земли уничтожен.
Но сейчас я склонен полагать, что ситуация, поставившая тси на грань жизни и смерти, лишь укрепила их.
Большая часть тси была перебита, но выжило достаточно, чтобы через десять тысяч лет вернуться в виде новой, более страшной угрозы "--- Скорбящих.
Это уже не просто потомки великих учёных "--- это хоргеты, воины с непонятной, чуждой нам моралью, которые уверены в своей правоте и готовы сражаться за неё до конца.
Есть данные, что Скорбящие проходили подготовку по противодействию Чистилищу.
Эти существа использовали камеру пыток для тренировок!
Я надеюсь, все поняли и осознали глубину этого факта.
Это не искатели приключений.
Они знали, на что идут.\\
Больше мы не можем закрывать глаза на эту проблему.
Оставшиеся на Тра-Ренкхале тси должны быть подвергнуты геноциду, чтобы предотвратить пополнение рядов Скорбящих.}
{Самаолу Каменный Старик, член совета Ордена Преисподней}

\epigraph
{Если дать свободу опьянённому властью, он обязательно нарушит закон.}
{Анатолиу Тиу}

\epigraph
{Слушайте древние легенды.
Не обязательно им верить, просто слушайте.}
{Анатолиу Тиу.
Предисловие к <<Древнейшей истории Драконьей Пустоши>>}

\epigraph{Счастье "--- синоним деградации.
Прогресс рождается в борьбе.
Его творят негодующие и скорбящие.}
{Изречение неизвестного Таракана, ставшее негласным девизом народа тси}

\epigraph{Ищите.
В мире много вещей, которые невозможно потерять.
Музыка и физика "--- лишь некоторые из них.}
{Людвиг Вейерманн}

\chapter{Живая сталь}

\section{Мировая Война}

Один из самых масштабных Театров, кстати, был посвящён войне.
Впоследствии он так и вошёл в историю "--- Мировая Война.
Было создано огромное виртуальное пространство, имитирующее планету с суровым климатом "--- пустыни, горы и ледники.
Это пространство населили астрономическим количеством NPC "--- звери, птицы, насекомые, растения.
Почти все были смертельно опасны или причиняли игрокам большие неудобства.
Три противоборствующие стороны вели войну на уничтожение, используя как можно более негуманное оружие "--- начиная от зазубренных мечей и заканчивая пулями со смещённым центром тяжести.
Система была анонимной "--- лица и голос изменялись до неузнаваемости "--- и с полным погружением "--- все ощущения были абсолютно неотличимы от реальных.
Театр длился половину сезона, в нём приняли участие сто миллионов тси;
при этом 80\% игроков покинули виртуал ещё в первые десять дней, погибнув или просто не выдержав свалившихся на них испытаний.
Остальные, по их словам, дошли до конца <<из принципа>>.
В последней битве последние триста игроков, стоя посреди обледеневшего горного плато, заключили мир, закончив тем самым игру.

Результаты были плачевны.
Несколько десятков умерли из-за потрясения, шестидесяти пяти тысячам потребовалась помощь врачей и психологов.
Театр посчитали удавшимся "--- по силе воздействия ему не было равных в истории.
Но тси вспоминают о нём очень неохотно.

\section{Возраст Безымянного}

"--*Нашла что-нибудь интересное в памяти Безымянного?

"--*Ничего особенного, относящегося к его создателям, "--- покачала головой Кошка.
"--- Очень хорошие комментарии к коду, настолько хорошие, что разберётся даже младший школьник\ldotst несколько пасхальных яиц, c десяток электронных книжек\ldotst

"--*Что за книжки?

"--*Документация к хоргету, устаревшая Мирквудская классификация, алгоритмы преобразования планеты и стабилизации экосистем, какая-то художественная литература, которую использовали для отладки языкового модуля "--- видимо, забыли стереть после\ldotst "--- рассеянно сказала подруга, думая о чём-то своём.
"--- Я всё залила в отдельный каталог, можешь глянуть\ldotst

"--*Посмотрю.
Кстати, каких времён Мирквуд?

"--*Судя по отсутствию некоторых важных правок, не менее чем пятитысячелетней давности, "--- встрепенулась Кошка.
"--- Забавный способ датировки, согласна.
Но это говорит не о времени создании хоргета, а скорее о времени разрыва коммуникаций между цивилизацией создателей Безымянного и Орденом Преисподней.

"--*То есть минус триста лет.

"--*Я бы даже доверительные интервалы подняла до пятисот, на всякий случай.

"--*И что скажешь?
Чувствуется, что Безымянному четыре с половиной тысячи лет?

"--*Он "--- ребёнок в песочнице.

"--*И песочница "--- Планета Трёх Материков.

"--*Тебя это не удивляет?

"--*Я уже устала удивляться, "--- растерянно призналась Кошка.
"--- Я "--- колонист, ступивший на поверхность, возможно, самой удивительной планеты во Вселенной, я общаюсь с хоргетом, и он даёт мне читать некоторые части своих исходников.
Слишком много удивительного.
Чтобы работать нормально, многое приходится принимать как должное.

\section{Волосы}

Баночка сидел и гладил Кошку по волосам.
Кошка жмурилась и морщилась от удовольствия.

"--*Мне так нравятся эти ваши волосы, "--- сказал Баночка.
"--- Жалко, что у нас их нет.

"--*Так сделай пересадку кожи головы, "--- пожал плечами Костёр.

"--*Я свои хочу!
Пересадить кожу каждый может.

"--*Ты бы смотрелся странно, "--- хихикнула Листик.
"--- Но Небо с волосами выглядел бы ещё забавнее.

"--*Мне ваши роговые наросты ни к чему, "--- замахал я руками.
"--- Жарко, путаются постоянно и везде застревают.
Как вы с ними вообще живёте.
Заяц приходит на чай раз в месяц и тихо сидит на диванчике, а робот-чистильщик потом выплёвывает килограмм её волос.

"--*Хорошо, я буду их заплетать! "--- развела руками Заяц.
"--- Тебя это устроит?

"--*Да я не жаловался, я просто не понимаю, как это возможно!

\section{Сигнал}

Полгода, которые запросил Костёр, давно истекли, а от врача не было ни слуху ни духу.
Я начал волноваться.

"--*Он взял с собой связь? "--- спросил я у техников побережья.

"--*Мы сами волнуемся, "--- сказал мне Синяя-Круглая-Плесень.
"--- Квантовый передатчик сутки назад подал сигнал тревоги.
Мы выпустили поисковый дрон, но он не вернулся.
Сейчас за ним отправились дельфины.

\section{Достойно фильма}

"--*Я тебе сейчас расскажу такую историю, это достойно фильма, "--- начал Костёр.

Костёр сбился с курса и нашёл новую группу островов "--- целый архипелаг, формой напоминающий спираль.
На островах даже оказалось местное племя.
Костёр не был силён в переговорах, как Кошка, и вскоре ему пришлось бежать.

"--*Обычные люди, такие же, как те царрокх "--- но как же ловко они бегают по зарослям! "--- вспоминал врач.

"--*А чем ты им насолил?

"--*Я слова им сказать не успел!
Они меня приняли за дичь!
Еле ушёл.
Лодку пришлось гнать.

Однако вскоре его ждала новая напасть.
Лодка попала в полосу цунами-шторма.

"--*Идёт последняя волна, и я понимаю "--- всё, конец, лодка перевернётся и я захлебнусь.
У меня даже не было времени написать сообщение вам.
Поэтому я дёрнул за кольцо жилета, заставил передатчик послать тревогу и прикусил язык.
Когда очнулся, шторм уже утих.
К счастью, океан в этих широтах тёплый, как молоко, и я не погиб от переохлаждения.
Днём спал, ночью шевелил руками-ногами не переставая, жевал всякую чушь "--- водоросли, улиток, мелкую рыбу, креветок "--- только бы не застыть.
Дрон прилетел в тот момент, когда я уже начал волноваться.
Я заарканил его тросом и повёл на запад, покуда батарей хватало.
К вам бедняжка, конечно, не вернулся, я пустил его на детали.
И снова повезло "--- меня вынесло на остров с деревьями.
Собственно, последние полгода я строил новое судно из подручных материалов.

"--*Дельфины тебя не нашли, "--- сказал я.

"--*Конечно! "--- пожал плечами Костёр.
"--- Я же говорю, я на острове был!

"--*Жалко, что так вышло.

"--*А я не жалею.
Да, мне пришлось пережить многое, сломать несколько костей, получить стрелу в плечо, надорвать спину во время постройки корабля.
Но зато я отвлёкся от грустных мыслей.
Сейчас Катаклизм кажется мне таким далёким, словно я читал о нём в исторической хронике, а не испытал на собственной шкуре.

\section{Рождение выбора}

Будь у меня возможность выбрать ещё раз, я сделал бы то же самое.

Да и бесполезно возвращаться в тот же момент.
Выбор "--- настоящий выбор "--- был сделан задолго до него.
Может, за год или за восемь лет.
А может, я всю жизнь шёл к тому, чтобы поступить так, как я поступил.

\section{Придуманные враги}

Я с трудом понимаю войну.
Нападать или защищать "--- не имеет значения, результат всегда один.
В прежние времена манипулировали обещаниями благ "--- пищи, красивой одежды, почёта, секса.
Но ответ на всё один, и этот ответ тси знают с детства "--- никто не поселится в выжженных землях, все любят цветущие сады.
Вырасти в своих землях, доме, теле и мозге собственный сад, знай его и ухаживай за ним.
Тогда тебе не будет нужды брать что-то силой.
Да и опыт, как по мне, будет гораздо более полезным в дальней перспективе.
Куда приложить опыт войны, если закончатся враги?
Разве что придумать новых.

\section{Мхи (ЖС)}

Я смотрел на мох.
Согласно данным первых людей, мхи были одними из первых наземных растений.
Как и миллиард лет назад, они "--- такие крохотные, такие сильные, бесконечно терпеливые "--- кучками цепляются за щёлочки и ложбинки в камне, стараясь устоять перед ветром и сберечь драгоценную влагу.
Мхи медленно расползаются, умирают и гниют, создавая миллиметр драгоценной почвы "--- почвы, на которую смогут сесть травянистые.

Я обернулся на поселение тси.
Эти голые камни, эти стоящие кучками домики и палатки\ldotst да, именно.
Как древние мхи, мы сообща пытаемся выжить и сохранить всё самое ценное "--- в надежде, что идущим за нами будет легче.

\section{Цитра Ветра}

Безымянный долго спрашивал, кто может сделать для него музыкальный инструмент.
Наконец одна из тси, инженер-технолог по имени Ветер-Дующий-Ниоткуда, решилась.
Как-то утром она вызвала бога и показала ему сделанный из тонких древесных пластинок струнный инструмент.

"--*Не могла заснуть ночью, "--- пояснила она.
"--- Идеям плевать на твой график.
Древесину сложно довести до состояния, когда она начинает хорошо резонировать, но я справилась.

"--*Как тебя зовут? "--- спросил Безымянный, осмотрев инструмент.

"--*Ветер, "--- осклабилась женщина.
"--- Не стоит благодарности, играй.

"--*Я запомню это имя, Ветер, "--- сказал бог.
"--- Я запомню это имя.

Отныне инструмент Безымянного бежал в библиотеке Стального Дракона, в специально отведённом для этого уголке.

\section{В честь планеты}

"--*Закрытая-Колба-с-Жизнью? "--- засмеялся Мак.
"--- Так ты назвался\ldotst

"--* \dots в честь планеты Тси, да, "--- буркнул Баночка.
Эта аллюзия, восходящая к кому-то из поэтов романтических времён, успела ему приесться.
"--- Надеюсь, ты цитировать его не будешь?

Мак явно намеревался и даже набрал для этого побольше воздуху в грудь, но вовремя сообразил, что делать этого не стоит.
В итоге биолог просто чуть громче, чем следовало, сказал <<Очень приятно познакомиться>> и ещё раз пожал Баночке руку.

\section{Кон-Тики}

Я вдруг вспомнил далёкие школьные годы и экскурсию к котловине Кон-Тики "--- месту, где был найден другой корабль.
Тот самый корабль, на котором предки млекопитающих-тси прибыли в нашу звёздную систему.

Огромный грот и кусок обшивки с надписью, вплавившийся в тысячелетний гранит.
Мои руки и ногочелюсти тряслись.
Мне казалось, что именно на меня величественная пещера подействовала гораздо сильнее, чем на прочих.
Знал ли я, насколько сильно будут похожи судьбы капитана Гуштефа и моя собственная судьба?
Возможно ли это\ldotsq

\begin{quote}\textit{
<<Их первоначальной целью была планета XD-10, и если бы не сломавшийся тормозной блок\ldots>>
}\end{quote}

\begin{quote}\textit{
<<\dots Априла 18 года 25712 Кон-Тики вылетел в свой безумный, невероятный полёт, имея топлива на один цикл разгон-торможение.
Спустя восемь лет три древних планеты "--- Земля, Марс и Диана "--- замолчали.
Мы не знаем, что с ними случилось, какова суть постигшего эти планеты бедствия.
Возродилась ли цивилизация?
Возможно, что нет.
Но возродись она и через пятьдесят лет, и через пятьсот "--- о Кон-Тики, который жаждал весточки из дома, никто не вспомнил>>.
}\end{quote}

"--*Я хочу, чтобы вы, глядя на эту доску, крепко уяснили одну вещь, "--- сказал в заключение учитель.
"--- Дома вы живёте в уюте, проводите время в работе и развлечениях.
Тси-Ди "--- это наш дом.
Ценой невероятных усилий тси сделали поверхность одной планеты и большую часть поверхности второй большим уютным домом.
Однако жизнь коротка.
Что для вас и для меня жизнь Гуштефа Морре-Чилда?
Всего лишь кусок металла.
Мало кто осознаёт, что мы оказались здесь благодаря горстке живых существ, которые говорили на другом языке, которые прошли труднопреодолимое даже для демонов расстояние.
В нас до сих пор живут их гены, а прочие, модифицированные гены помнят умелые руки их потомков.
Каждый аспект технологии, который мы можем найти на двух планетах нашей системы, несёт на себе отпечаток мыслей тех существ.
Подумайте об этом и скажите мне, прожившему триста лет, что же такое наша жизнь?
Я уже отчаялся найти ответ на этот кажущийся простым вопрос.

\section{Тревожная женщина}

"--*Не следят за своим здоровьем, "--- бурчал Костёр.
"--- Я же тоже не железный\ldotst
Вы всегда можете прийти ко мне, и я вас полечу.
А кто меня лечить будет?

"--*Возьми отпуск, "--- сказал я.
"--- Хотя бы на несколько дней.
Слетай на тот берег.
Там красиво.

"--*И остаться наедине со своими мыслями? "--- хмыкнул Костёр.
"--- Нет уж.

"--*Кого ты потерял?

Костёр потряс головой и надолго замолчал.
Я уже было думал, что он не хочет говорить, но\ldotst

"--*Женщину и детей.
Не нужно, "--- Костёр мягко высвободил руку, когда я попытался её погладить.
"--- Я уже успел забыть за всеми этими заботами, как они выглядят.
Но одна мысль не даёт мне покоя, и боюсь, однажды она меня убьёт.

Костёр подумал.

"--*Та женщина страдала высокой тревожностью.
Дупликация гена одного из нейромедиаторов, достаточно редкая мутация.
Обнаружили её уже в зрелом возрасте.
Женщина приспособилась жить с ней "--- работала только по ночам и в полной тишине.
Зато в работе "--- а была она диагностом систем "--- ей не было равных.
Замечала такие нюансы, мимо которых проходили все прочие.
Познакомились мы в больнице.
Меня потрясла её нежность и чуткость "--- такого человека редко можно встретить.
Она же говорила, что рядом со мной ей очень спокойно.

Врач вздохнул.

"--*Меня не печалит её смерть "--- здесь, среди постоянных тревог, она бы просто не выжила.
Но едва представлю всю глубину того ужаса, который испытало это нежное сердечко перед смертью "--- и всё валится из рук.
Поэтому ты, Небо, как хочешь, а меня в отпуск не отправляй.
Буду работать, пока не сдохну.

Я поднялся.

"--*Я сообщу всем молодым тси, что тебе нужна ласка.
Я думаю, кое-кто придёт.

"--*Пошёл вон.
И если ко мне ещё кто-то придёт с нежностями, я их выкину.

"--*Первый раз "--- разумеется.

"--*И второй тоже.

"--*Отлично.
Значит, женщины и мужчины будут делать не менее трёх попыток, "--- заключил я.
Костёр ухмыльнулся.

"--*Мужчин не нужно, не привлекают.

"--*Спокойного дня, Костёр.

"--*Спокойного и тебе, Небо.

\section{Сон}

"--*Комарик, пожалуйста, живи!
Живи! "--- плакал я.

Комар погладил меня слабой рукой по плечу.

"--*Я всегда буду с тобой.
Я буду жить у тебя в голове.
А ещё у нас остались дети, Небо.

"--*Как бы я хотел, чтобы у нас были дети.

"--*Они у нас есть.

Я гладил его по голове и плакал.

Вдруг откуда-то начала прибывать вода.
Откуда в Красном музее вода?
Может, роботы повредили систему водоснабжения?
Но как?
Я поднял глаза\ldotst и увидел яркое солнце меж двух отвесных стен.
Знакомых, изъеденных ветрами стен.

Котловина Кон-Тики.

Вода прибывала.
Комар начал захлёбываться;
он инстинктивно приподнимал брюшко, но вода всё равно затекала в дыхальца, выбивая из агонизирующего тела столбики весёлых пузырьков.
Я закричал, поняв, что у меня не хватает сил, чтобы даже приподнять друга над водным потоком.

Пузырьки.
Вода клокотала, чёрная, как ночь, с металлическим оттенком, словно холодная ртуть.
Я снова закричал "--- отчаянным, ужасным криком.
Комар задыхался на моих руках.

"--*Небо, молчи и слушай.
Девяносто пять.
Девяносто пять нанометров.

"--*Что?
Что ты говоришь? "--- плакал я.

"--*Молчи и слушай, Небо!

"--*Да что он несёт? "--- закричал я.
"--- Баночка, помоги мне.

Я поднял глаза на планта.
Он стоял по колено в воде и смотрел на меня грустными глазами из-под татуированных век.

"--*Ничего не сделать, командир.

"--*Просто! Помоги! Его! Поднять! Над! Водой! "--- заорал я.

"--*Это не вода, Небо, очнись.
Это смерть.
Ты сам уже по пояс в ней.

"--*Молчи! И слушай! "--- скрежетал Комар.
Над водой осталось только лицо, но он пытался что-то сказать.
"--- Это очень важно\ldotst
Девяносто пять, Небо, девяносто пять\ldotst

\razd

Ночник светил слабым, успокаивающим светом.
Я несколько раз тянулся рукой к импланту, чтобы сделать инъекцию успокоительного.
И отдёргивал руку.

Первой мыслью было бежать к Костру;
разумеется, это был \textit{тот самый} сон, о которых врач меня предупреждал.
Но ни один не вгонял меня в такой ужас.

<<Молчи и слушай>>.
Да, верно, Костёр в кругосветном.
Я уже пожалел, что отпустил его.

Я лёг и попытался не двигаться.
Это была любимая фраза Комара, совет на любой случай жизни.
Молчи и слушай.

За открытым иллюминатором царила тихая ночная жизнь.
Пели птицы, сновали млекопитающие, шуршали насекомые.
Над всем этим царил ровный гул реактора.

Нет, что-то точно не так.

В коридоре меня встретил Баночка и замер, словно увидев призрака.
Потом облегчённо выдохнул.

"--*Ты меня напугал.

Плант вдруг вытащил сверток и принялся набивать травами курительное приспособление.

"--*Ты куришь? "--- удивился я.

"--*Неа, "--- виновато улыбнулся плант.
"--- Просто что-то неспокойно, и я решил попробовать.
Пирожок говорит "--- успокаивает.

"--*Ты был у неё?

"--*Да.
Ей тоже что-то не спится.

"--*Что, неужели и ей котловина Кон-Тики не даёт заснуть?

Баночка промолчал и сделал глубокую затяжку.
Затем закашлялся.

"--*Что мы упускаем, Баночка? "--- задумчиво спросил я.
"--- Мы ведь упускаем что-то важное.

"--*<<Молчи и слушай>>, "--- буркнул плант.
"--- <<Девяносто пять нанометров>>.
Мы действительно что-то упускаем.
Но что?

Я вдруг понял, что больше ничто в этом мире не способно меня удивить.

"--*Комар не успел сказать.

"--*Да при чём тут Комар, "--- вдруг раздражённо хмыкнул плант.
"--- Забудь про Комара, забудь про котловину Кон-Тики, и про смерть свою забудь, они в далёком прошлом.
Научись уже отделять актуальные проблемы от проекций чьей-то памяти.

"--*Моей памяти!

"--*Ты уверен, что это твоя, а не память давно умершего, чужого тебе существа, оказавшегося в похожей ситуации?

Я промолчал.
Баночка всегда умел задавать хорошие вопросы.

"--*<<Молчи и слушай>>.
Эта мысль летает в воздухе прямо сейчас.
<<Девяносто пять нанометров>>.
Что настолько важно?
Что измеряется в нанометрах?
И это число, как будто я его где-то\ldotst

"--*Кое-что не было проекцией чужой памяти, "--- перебил я его.
"--- Молчи и слушай.

Баночка виновато кивнул.
Пели птицы, шуршали млекопитающие, шелестела трава.
Гул реактора, ровная, идеально ровная гармоника.
Пять секунд.
Сто пять секунд.
Пятьсот тридцать одна секунда\ldotst

И вдруг гармоника сделала крохотный скачок.
Мы с Баночкой уставились друг на друга.

"--*Что это было?

"--*Молчи и слушай! "--- прошипел плант.
В его глазах застыл животный ужас.

Пять секунд.
Сто пять секунд.
Пятьсот тридцать одна секунда в томительном напряжении.
Трубка в руках Баночки давно испустила последний дымок.
По лицу планта текли капли пота;
в глазах плясала надежда.
Может, показалось?
Мы спросонья, переволновались, конечно, нам могло показаться что угодно\ldotst

И снова крохотный, едва уловимый скачок гула реактора.

Баночка кинулся к лифту.

"--*Поднимай всех, командир, "--- бросил он через плечо.
"--- Мы по колено в воде.

\section{Молитвы}

"--*Мне кажется, что я скоро умру, "--- признался я.

Баночка вдруг схватил меня за руку.

"--*Небо, пойми одну вещь.
Наша судьба не определена.

"--*Но сны\ldotst

"--*В бездну сны! "--- рявкнул плант.
"--- Ты вечно строил из себя жертву.
Ты жертвовал всем "--- своими интересами, своей жизнью, всем, что имел.
Ты думал, что тебе уготована такая судьба, и шёл навстречу этой судьбе.

"--*Костёр\ldotst

"--*Костёр сумасшедший!
Ты не задумывался, почему я с ним не общаюсь?
Мне совершенно неважно, каким образом я вижу сны о далёком прошлом.
Да, я давно знаю, что это правда, с самого детства.
Я чувствовал, как вонзаю штык в чужое сердце, до мельчайших подробностей.
Я собирал во сне взрывчатку, ещё не имея понятия о началах химии, пользовался противогазом, чистил винтовку.
Я до сих пор помню вот это!

Баночка сделал молниеносное движение руками, и я похолодел.
В моей голове ясно щёлкнул передёргиваемый затвор.

"--*Я умирал, занимался сексом, бредил в лихорадке, ещё не зная этих вещей в реальности.
Но я никогда, слышишь, никогда не уходил в мистицизм.
А Костра эти сны свели с ума!
Его свело с ума одиночество, потому что ты одинок ровно в той мере, в которой погружён в иллюзию!
Ты хочешь того же?

"--*Ты не прав.
Одиночество "--- это погружённость в иллюзию, отличную от иллюзии прочих.

"--*Давай не будем устраивать низкопробный философский диспут на серьёзную научную тему.
Мы не дикари.
И помнится, мы вместе сдавали диктиологию.

Я промолчал.

"--*Небо, я тебя прошу "--- останься с нами.
Останься в реальном мире.
Если тебе одиноко, чаще бывай с народом, пройди курс гормонов, только не уходи в иллюзию.
Нет и не было никакой судьбы, "--- Баночка отпустил мою руку и крепко обнял.
"--- Я это знал всегда.
И, заметь, почти всегда доживал до старости.
Потому что мной двигало желание жить, а не обречённое принятие чьих-то иллюзий.

\section{Каскад тревоги}

"--*А как вы меня нашли? "--- удивился я.

"--*Ты не поверишь, "--- улыбнулась Кошка.
"--- Листик и Баночка засекли каскад тревоги.
Я бы в жизни не додумалась до такого.
Объяснять долго, суть: удар шлюпки о воду испугал рыб.
К испуганной рыбе слетелись чайки, за чайками последовали крупные хищники.
Волны тревоги распространялись вокруг эпицентра с помощью сигналов различных существ.

"--*А кто засёк эти волны?

"--*Мой приёмник, "--- Баночка снял с плеча довольно каркающую Нейросеть.
Судя по всему, ворону только что похвалили и накормили чем-то вкусным.

"--*Листик случайно услышала тревожное карканье Нейросети и сразу сообразила, что это может быть связано с перебоем связи.
Баночка после долгих усилий наконец понял, что хотела втолковать ворона, и мы послали в море дельфинов.

Я погладил птицу пальцем.
Она игриво ухватила меня сильным клювом.

Никогда не думал, что буду обязан жизнью вороне.

\section{Килограмм хлорофилла}

"--*Я придумал даже их название, "--- сказал Баночка.

Кошка поморщилась:

"--*Может, они сами решат, как себя называть?

"--*Если им не понравится моё "--- так и будет, "--- улыбнулся плант.
"--- Но что-то мне подсказывает, что это название они примут.
Novai.

"--*Красиво, "--- признала Кошка.
"--- А что значит?

"--*<<Новые>>.

"--*А их язык "--- старейший язык Вселенной "--- будет, значит, <<nova lingva>>? "--- захохотала Кошка.
"--- У тебя удивительное чувство юмора.

"--*Вообще-то d'noveio lingva, но твой вариант благозвучнее, "--- плант с серьёзным видом сделал пометку в своих записях.
"--- А ты могла бы влюбиться в мужчину с хорошим чувством юмора?

Кошка на секунду отвлеклась от записей.

"--*В этом случае мне пришлось бы вывести ужасное умозаключение.

"--*Какое?

"--*Что килограмм хлорофилла "--- это не самое страшное, что может быть в мужчине.

"--*Я красивый, "--- обиженно заметил Баночка.
"--- Да, зелёный, да, немножко низковат для вашего вида.
Но красивый.

"--*Очень красивый, "--- согласилась Кошка.
"--- Поэтому давай ты будешь опылять девочек своего вымирающего вида, а не тратить силы на бесплодные сношения со мной.

"--*Можно подумать, ты прям занята возрождением \textit{своего} вымирающего вида, "--- грустно буркнул Баночка и с головой ушёл в работу.

\section{Вот это поворот}

"--*Я уже придумал первых персонажей для легенд, "--- сказал Баночка.

"--*И кто они?

"--*Ты и я.
Если мне не суждено быть с тобой в реальности, может, мне удастся взять реванш в сознании потомков, "--- улыбнулся плант.

Кошка покраснела.

"--*Для тебя важно, чтобы мы были вместе?

"--*Для меня нет ничего неважного, если это связано с тобой.

Кошка помолчала.

"--*Завтра вечером я буду свободна и полна сил.

Баночка кивнул.

"--*Кстати, как их будут звать?
Персонажей.

"--*Имя твоего "--- Ликх'аамас.
Меня будут звать Муал'ликх.
Через тысячу лет, конечно же, имена видоизменятся.
Но я надеюсь, что никто не забудет, что ты "--- прекрасная большая белокожая женщина-человек, а я маленький, зелёный и лысый мужчина-плант.

"--*Такое сложно забыть.

\section{Интуитивный мистицизм}

Кошка размахивала руками и поставленным голосом читала какие-то тексты, время от времени сверяясь с компьютером.

Я заворожённо наблюдал за ней.
Простой комбинезон вдруг сменился в моём воображении причудливой одеждой с перьями, руки и лицо покрылись охряным узором, глаза остекленели от наркотических веществ.
Я понимал "--- Кошка не просто изучает ритуал.
Она пытается найти свой собственный путь, путь учёного-тси, во тьме древнего, интуитивного мистицизма.

\section{Животный страх}

Меня очень удивили местные животные.
На Тси-Ди обращение с дикими животными занимало несколько лет обучения.
Детям объясняли про рефлексы низших животных и про сигнальную систему высших.
Впрочем, за десятки тысяч лет мирного сосуществования дикие животные настолько привыкли к тси, что воспринимали их как камни или деревья.
Они иногда кушали из рук, позволяли себя гладить или приходили к жилищам полечиться.
А со знанием сигнальной системы можно было довольно легко убедить не слишком голодного волка, что ты плохая добыча и потенциальный друг.

Цветущий-Мак-Под-Кустами даже как-то рассказал, что в детстве одну зиму жил в стае койотов.
Самым сложным, по его словам, оказалось приучить лохматых к человеческому запаху, а ещё объяснить, что у костра можно греться.
Когда волки смекнули, что к чему, то сами стали намекать, что пора бы уже <<сделать горячую вонючку>> "--- зима выдалась суровой.
Мак даже вспомнил, что в благодарность за что-то молодая самка койота принесла ему свежезадушенного кролика.

Здесь же животные не показываются взгляду.
Почуяв человека, млекопитающие обычно удирают со всех ног.
От них пахнет смертельным страхом.
К прочим тси также относятся настороженно.

Я рассказал свои ощущения Баночке.
Он в ответ поделился опасениями, что нам придётся жить охотой и животноводством, если с кольцевой теплицей что-то случится.

\section{Дельфины}

"--*Я не понимаю дельфинов, "--- заявила Листик.
"--- Мы все переживаем, думаем, как лучше, а дельфинов это как будто не касается.
Главное, чтобы был океан, и дельфины счастливы.

"--*А о чём им беспокоиться? "--- заметил Мак.
"--- Пищи у них достаточно, врагов нет.
Гора-Окутанная-Дымом сказал, что он с отрядом проплыл почти до самых полюсов.
Да, там прохладно, им пришлось надеть термочехлы на плавники, но в целом дельфины могут жить везде.

"--*А акулы в проливе Скар? "--- спросил я.

"--*Акулы могут утащить разве что дельфинят.
Или малыша вроде тебя, "--- усмехнулся Мак.
"--- Гора сказал, что акулам и прочим морским зверюшкам очень не понравились сонарные сигналы, которыми он и его отряд <<прокрикивали>> дно.
Я посоветовал дельфинам провести на этот счёт исследование "--- дыма без огня не бывает, возможно, что какой-то опасный природный феномен тоже <<говорит>> на этих частотах.
Сейчас команда Капельки этим занялась "--- узнают, каких именно частот боятся зверюшки и в какой местности.
Так мы хотя бы будем знать, что искать.

"--*А что ещё нашли дельфины? "--- поинтересовалась Заяц.

"--*Вообще они много кого привезли.
Рыбок, моллюсков, водоросли, даже щупальце кого-то, отдалённо напоминающего гигантского осьминога.
Сказали, что труп дрейфовал недалеко от кораллового рифа, очень повезло, что его ещё не успели обкусать рыбы.
Гора составил очень подробный видеоотчёт по океану южнее девятой параллели.
Больше похоже, правда, на юмористический фильм, над комментариями можно ухохотаться.
<<Смотри, какая странная красная рыбка!>>
"--- <<Оставь её, хвостомордый, у неё депрессия>>.
"--- <<С чего ты взял?>>
"--- <<Если бы я носил шкурку наизнанку, у меня тоже была бы депрессия>>, "--- Мак захохотал.

"--*Я же говорю, что они относятся к делу недостаточно серьёзно, "--- укоризненно сказала Листик.

\section{Рука}

Из наркотической полудрёмы меня вырвал звук взрыва.
Как я узнал впоследствии, техники попытались применить не по назначению полевой генератор.
В палатку проковыляла Заяц, крича на ходу медицинский код травмы и волоча на себе тяжёлое, залитое кровью тело Звенеть-Хрустальными-Клыками.
Правая рука канина превратилась в месиво.

Ещё один несчастный случай.

"--*Один-три-а-ноль-эф!
Один-три-а-ноль-эф!

Костёр выбежал как ошпаренный, толкая перед собой летающий ларец с числом <<13>>.
Заяц без сил рухнула на колени.

"--*Отойди, "--- Костёр грубо отпихнул Заяц и, выхватив пистолет, прижал его к груди Зубика.
На случай выхода импланта из строя у всех тси-млекопитающих в грудине стоял внутрикостный катетер.

Заяц, придя в себя, бросилась помогать.
Вскоре на культю была наложена искусственная соединительная ткань, потерю крови восполнили кровозаменяющей жидкостью, и канина положили в реанимационную капсулу рядом со мной.

"--*Как остальные? "--- это был первый вопрос, который задал Зубик по пробуждении.

"--*С ними всё в порядке, "--- сухо ответил Костёр, проверяя в последний раз повязку.
Отсутствие привычных инструментов, материалов и лекарств сделало его раздражительным "--- и это при пресловутой стрессоустойчивости врачей.

Зубик посмотрел на культю и поморщился.

"--*Я теперь навсегда без руки?

"--*Мы потеряли кольцевую теплицу, "--- чуть не плача, ответила Заяц.
"--- Я не знаю, возможно ли вырастить руку с теми материалами, которые\ldotst

"--*Заяц, перестань, "--- оборвал её Костёр.
"--- Хватит брать на себя вину за целый мир.
Теплица сгорела, в реакторе была трещина, роботы были перепрограммированы.
Абсолютно все обстоятельства были против нас.
Фонтанчик погиб, как герой, пострадавших могло быть больше.

Заяц расплакалась.
Костёр бросил на неё виноватый взгляд, поправил мне одеяло и уже чуть мягче добавил:

"--*Я восстановлю тебе руку, Зубик.
Но не за один год.
Подождёшь?

"--*Куда мне деваться, "--- попытался пошутить канин.

"--*Я соберу тебе механическую, "--- сказала Заяц, вытирая слёзы.
"--- Уж это-то мы сможем.

"--*Вы лучшие друзья, ребята.

"--*Для начала поправься, "--- остановил его Костёр и, мимоходом взглянув на меня, вышел на воздух.
Вскоре откуда-то потянуло странным удушливым дымом "--- кое-кто из тси пристрастился к местным травам с седативными алкалоидами, и похоже, что Костёр тоже.
Я нечаянно громко щёлкнул зубами, и дым тут же рассеялся, сменившись на слабое гудение вытяжки.

Заяц сидела у постели Зубика.
Я знал, что она сейчас думала о погибшем любовнике и о едва живом мне.
Ей страшно хотелось на воздух, подальше от этой наполненной смертью палатки, но её останавливал долг дружбы.
Зубик, похоже, тоже понял это.

"--*Иди, Зайчик.
Я за ним послежу.

"--*Иди, "--- добавил я.
"--- Я пока в хорошем настроении.

Заяц с натянутой горькой улыбкой по очереди кивнула нам и тут же вышла.

\section{Меч Баночки}

"--*А где твой меч? "--- спросил я.

"--*Я его спрятал на берегу, "--- смущённо сообщил Баночка.
"--- Знаешь, у царрокх есть истории про героев, которые находили чудесное оружие и боролись с врагами.
Я вырубил для меча крипту в одной скале и положил его туда.
Возможно, мой меч найдёт великий герой будущего.

"--*Почему? "--- удивился я.

"--*Потому что я не герой, "--- просто ответил плант.
"--- Для меня ещё не пришло время владеть оружием.
Я вполне могу защитить себя, но ещё не знаю, где граница.

"--*Какая граница?

"--*Граница между защитой и нападением, "--- сказал Баночка.
Он был серьёзен, как никогда.

\chapter{Безумная война}

\section{Убор хоризии}

\begin{verse}
Из убора красотки-хоризии,\\
Что горда и фигурой пышна,\\
Ты принёс драгоценный цветок\\
И мне в волосы вплёл на закате\ldotst
\end{verse}

\section{Животные}

"--*Манэ, Лимнэ, ваши животные опять сбежали.
Смотрите, не то их задушат коты или заклюют курочки.

Манэ тут же бросилась в угол и вернулась с капюшонной крысой.
Лимнэ подозвала белую шиншиллу и взяла её на руки.
Шиншиллу привезла из Пыльного Предгорья путешественница-поэтесса с зелёными глазами и странным акцентом;
ручная крыса была изловлена в амбаре во время буйного пиршества, да так и осталась жить у сестрёнок.

\section{Каменный вождь}

Эрликх уже давно сидел в окружении поклонников;
Кхарас лежал в одиночестве, скрестив руки на груди.
Несколько девушек, хихикая, подбежали и начали его целовать;
спустя секхар они убежали, словно обжегшись.
Вождь не двинулся с места.

\section{Посохи}

Путники с бумажными фонарями "--- особое общество.
Кто-то из них знает, что посвящён Ситу, кто-то интуитивно повторяет привычки знакомых.
Сели, хака, ноа, ркхве-хор, зизоце, трами, травники, совсем редко "--- дикие идолы с татуированными лицами.
Посохи ркхве-хор похожи на срубленные и слегка ошкуренные деревья.
Когда великаны кланяются, их посохи ощутимо скрипят.
Посохи травников едва годятся для того, чтобы ворошить костёр;
этот народ приветливо щёлкает челюстями и поднимает четырёхпалые руки горе.
Но все они "--- одно: мы не мёртвы, не живы, мы в пути.

Вот навстречу попалась женщина-хака.
Молодое лицо с неглубокими морщинками, серые глаза, длинные дреды, сплетённые из её собственных волос и волос случайных попутчиков.
Штаны сели, красивая пылеройская накидка, чересчур широкая ей в плечах.
На родине женщину признали бы негодной и подвергли остракизму.
Но здесь, на дороге, властвует бумажный фонарь.
Мы встретились взглядами\ldotst как вдруг усталое лицо расцвело искренней щербатой улыбкой, и я неожиданно для себя улыбнулся в ответ.

Едем дальше.
Ни имени, ни привета, да и лицо прохожей уже расплывается в памяти, но вечернее небо на миг стало более голубым.

\section{Живите кто хотите}

Перед уходом Грейсвольд долго стоял и гладил резную колонну жилища.
Жилище было совершенно не похоже на прочие, стоявшие на улице.
Я был уверен, что подобного ему не нашлось бы и во всём Ихслантхаре.

"--*Это хорошее жилище, "--- сказал Грейс.
"--- Мы с Тхартху строили его вместе.
Резьба и рисунки полностью принадлежат её рукам.

"--*Оно выглядит точь-в-точь как в моих детских мечтах, "--- улыбнулась женщина.
"--- Ликхэ-лехэ делал конфетные дворцы, и я ребёнком часто забегала к нему в лавку, смотрела на них и думала, что я хочу такой домик.

Грейсвольд бросил на неё смущённый взгляд:

"--*Может, всё-таки останешься?
Тебе ничего не грозит.

"--*Без тебя мне здесь делать нечего, "--- отрезала Тхартху.
"--- Так что хватит об этом.

"--*Как скажешь.

"--*Мы уходим навсегда.
Надо отдать дом сельве, чтобы путь был безмятежным и удачным.

"--*Надо, "--- признал Грейс, "--- но у меня рука не поднимется.

"--*У меня тоже, "--- призналась Тхартху.
"--- Что ж\ldotst

Женщина вытащила из-под навеса чистую коротенькую доску и положила на скамью.
В её крепких руках заплясал кукхватровый резец, разбрасывая деревянные кудряшки и коротенькие щепочки во все стороны.

"--*Хорошая идея, "--- похвалил Грейс, оглядев работу.
"--- Поставь её на рога наличника, вон туда.
Там она будет держаться крепко и её увидят все.

"--*Отдавать людям даже лучше, чем отдавать сельве, "--- улыбнулась Чханэ.
"--- Это не будет забыто.
Пусть жилище принесёт радость новым хозяевам.

Мы аккуратно прикрыли калитку и, бросив последний прощальный взгляд на жилище, отправились по размытой дороге на юг.

Фонарь у двери остался гореть, освещая прихотливую резную надпись:

<<Дом свободный, живите кто хотите>>.

\section{Яблоко}

"--*Как себя чувствуешь?

"--*Отлично.
Сны только пошли яркие и волнующие, выспаться трудно.

"--*Хай, это бывает, "--- засмеялась кормилица.
"--- Знакомой крестьянке снилось, что она родила не ребёнка, а то котёнка, то яблоко, то ещё что-нибудь.
Меня больше реакция позабавила.
Обидно, говорит, пять декад ходила "--- и яблоко\ldotst

\section{Ах, улиточка}

Дождь прошёл, и выглянуло яркое солнце.
В лужах уже вовсю копошилась какая-то мелюзга, на мокрых заборах и стенах сидели полосатые улитки.
Самых больших я мимоходом стягивал и складывал в карман "--- на суп или жарево.

\section{Паркетная полянка}

Паркетная полянка была известным местом встречи на окраинах Тхитрона.
И старожилы уже не припомнят, что за здание там стояло;
сейчас от него не осталось даже стен "--- только несколько островков изумительного паркета напоминали о деятельности человека.
Паркетную полянку берегли.
Когда несколько дождей назад кто-то отбил одну паркетину, три квартала искали неведомого вандала.
И нашли "--- квартал каменщиков вычислил инструмент по зазубринам на камне.
Что с героем сделали, история умалчивает, но больше паркет никто не ломал.

\section{Не хватит}

Кхотлам, увидев половину Храма в праздничной одежде на голое тело, едва не выронила поднос.

"--*Молодёжь, вы это зря.
Храмовников, особенно девушек, и так на каждые Руки в клочья рвут, а вы ещё и сами на себя внимание обращаете.

"--*У кожевников вообще есть поверье, что если в Мягкие руки завалить воина, то кожи потом никакой клинок не пробьёт, "--- радостно сказала Ликхэ.
"--- А завалить жреца "--- просто хорошая примета.

"--*Не надо на меня так смотреть, "--- возмутился я.
"--- Я не дичь и не коллекционная статуэтка!

"--*Кожевников в квартале тридцать восемь, вас пятеро.
Лисёнок, ты вообще один с Верхнего этажа празднуешь.
Вас на ночь не хватит.

"--*У нас особых планов нет, "--- успокоил я кормилицу.

"--*А у других на вас есть, "--- резонно заметила Кхотлам.
"--- Опять Кхохо всех подбила, это несомненно.
Впрочем, ладно, все взрослые люди.

"--*Надеюсь, Кхохо поделится и уступит мне хоть кого-то из тех тридцати восьми, "--- буркнула Ликхэ.
"--- У неё свои приметы\ldotst

\section{Мысль}

Иногда накатывает странное желание покончить жизнь самоубийством "--- лёгкое, воздушное, почти неощутимое.
Оно совершенно лишено негативной окраски, это желание с улыбкой "--- лёгкой грустной улыбкой.
Оно появляется из-за мысли о совершенно несбыточной, но страстной мечте, которая поглотила слишком много ночей и раздулась до колоссальных размеров.
Это желание проходит почти сразу "--- и, наверное, к великому счастью.

\section{Пиратство}

"--*Может, в пираты подадимся?

"--*Идея интересная, "--- одобрил я.
"--- Жаль, что для меня работы у пиратов будет не так много.

"--*Пиратам нужны врачи, "--- заметила Чханэ.

"--*Вот именно, этим их потребность в жрецах и ограничивается.

"--*Ааа, "--- поняла Чханэ.
"--- Ну тогда да, лучше будет поселиться где-нибудь на Короне.

"--*Если хочешь к пиратам "--- то иди.

"--*Слушай, Аркадиу, "--- Чханэ произнесла моё имя неожиданно чисто, "--- я пообещала, что пойду с тобой.
Поэтому заткнись.

И я заткнулся.

\section{Страусы}

Невдалеке показалось гнездовье шипастых страусов.
Увидев меня, самец зашипел и распустил смертоносные перья.
Однако вскоре, заметив в облаке пыли бесшумно идущую армию сели, страус предпочёл быстро ретироваться вместе с выводком.

\section{Касания Пера}

Эрликх бился совсем по-другому.
Его удары, неощутимые и невесомые, самым возмутительным образом вытягивали силы.
Даже Кхарас после спарринга с Эрликхом выглядел усталым;
я же валился с ног после трёх-четырёх касаний.
Второй особенность Касаний Пера было то, что синяки начинали жутко болеть сутки спустя.

\section{Брусчатка}

Ливень омыл древнюю тхитронскую брусчатку, на миг обнажив буйство красок отполированных до блеска вулканических пород.
Мало кто из приезжих догадывался, что на мостовой четырёх дорог был выложен узор "--- породами аж пяти цветов.
Увидеть это можно было лишь утрами, с Середины Дождя, когда брусчатка сияла влажной чистотой, а облачную серость только-только расцвечивало выглянувшее солнце.
Три-четыре рассвета "--- и краски пропадали до следующего сезона.

Я уверен, что те, кто мостил дороги, прекрасно знали, что в итоге всё занесёт грязью, но делали всю эту красоту ради нескольких дней в году.
Тхитронцы очень любили делать неожиданные подарки.
Это я понял однажды, отыскав в своей комнате тайник с рассыпавшимися от старости игрушками.
Если верить выцарапанным на камне иероглифам, тайник сделали строители, видимо, не поставив в известность зодчего;
дети, которым он предназначался, повзрослели, завели своих детей и умерли, так его и не найдя\footnote
{На языке тси такие подарки дословно называются <<транзит с востока на запад>> или просто <<восток-запад>>, то есть послание в будущее. \authornote.}.

\section{Мудрость}

Есть в людях неуловимые признаки житейской мудрости, помогающей жить и процветать.
Например, зонтик, взятый в день начала дождей.
Или чистые штаны, мыльный раствор и зубная щётка в котомке.
Или надрезанные сеточкой свежие огурцы, лучше пропитывающиеся пряностями и солью.
Сами по себе эти вещи ничтожны;
что изменят пара кхамит в несвежей одежде или приготовленные иначе огурцы?
Однако в целом такие люди и живут дольше, да и в жилищах у них куда больше света и счастья.
Суть часто кроется в мелочах.

"--*Мудрость определяется легко, Ликхмас, "--- как-то сказал Конфетка.
"--- По словам.
Спросишь у человека "--- зачем ты это сделал?
А он в ответ: <<Так вкуснее>>, <<Так удобнее>>, <<Так легче>>, <<Так проще>>.
Вот это мудрость.
Если же он отвечает: <<А вдруг>>, <<А если>> "--- это или скупость, или опасливость, не имеющие ничего общего с мудростью.
Так, прикормка для собственной тревоги.
Мудрый всегда знает, что он делает и зачем.

\section{Каштаны}

Там, на Дальнем Севере, где распускающиеся в Пирог почки каштанов похожи на неуклюжих большелапых птенцов.

\section{Люби себя}

Любовь "--- это когда ты наливаешь чай, насыпаешь каши и гладишь своего любимого, даже если точно знаешь, что лучшие дни его жизни позади.
Любовь "--- это верить и помогать, не только в работе, но и в отдыхе.
Постелить постель, потереть спинку, погладить и почистить одежду.
И пусть всё мироздание будет против твоего любимого, ты будешь с ним.
Люби людей так.
И самое важное "--- люби себя так же.
Всегда.

\section{Внутренняя свобода}

Наверное, следовало помочь, пожалеть, но я отвернулся и пошёл дальше.
Иногда хочется просто побыть человеком, свободным испытывать чувства.
Свободным любить, свободным ненавидеть, свободным быть верным и свободным предавать.
Для равнодушия порой тоже необходима внутренняя свобода.

\section{Сапоги}

Последними я надел сапоги.
Застёжки и завязки я приводил в порядок нарочито медленно, смакуя каждое движение.
Бегать босиком "--- это удовольствие.
Но надевать обувь "--- удовольствие совершенно другого рода.
Ни посох, ни котомка, ни убегающая вдаль дорога так не манят в путь, как надёжная обувь, нежно льнущая к стопе и уверенной хваткой держащая голень.

\section{Высота}

Я вздохнул полной грудью и почувствовал, как сразу стало легче.
А ведь только и надо было "--- выйти за порог.
Во многих случаях это единственное лекарство.
Не пугали больше ни клинки, ни джунгли, ни высота.
Да и что такое высота?
Всего лишь вертикальная длина.

\section{Мхи (БВ)}

Я погладил мшистый камень барельефа.
Удивительно, сколько всякой живности здесь, у самой реки.
Даже мхов.
Вот <<рыжая шкурка>> "--- зелёная моховая шапочка со множеством красно-бурых остистых волос.
Вот <<северное серебро>> "--- твёрдый, блестящий, словно бледно-зелёный шёлк, плотно сбитый из круглых стеблей.
Вот безымянный, губчатый, пушистый и нежный, цветом напоминающий о болоте.
А вот золотистые звёздчатые шары <<вечернего красавца>>, рыхлые и жестковатые на ощупь.
То тут, то там в гранит въедались бородавчатые брызги лишайников "--- серые, сизые, бурые, ярко-зелёные.
Учитель как-то прочитал, что лишайники "--- это тоже грибы.
Наверное, он что-то недопонял;
внимательно рассмотрев лишайник и попробовав его на язык, я признал, что с грибом он не имеет ничего общего.

\section{Пылеройский хлеб}

Все справедливо решили, что для готовки время чересчур позднее, и удовольствовались закуской.

"--*Хлеб с помидорами "--- это самое изысканное лакомство с голодухи, "--- сказал Ситрис, обильно сдабривая лакомство солью.

"--*Мы этим конопляное вино закусываем, "--- поделилась Чханэ.
"--- Усиливает эффект, но похмелье потом такое, что хоть голову в горшке держи.
Хлеб лучше брать пылеройский.

"--*Пылеройский хлеб? "--- удивилась Кхохо.

"--*Да, ржаной с саговыми зёрнами, чёрными скорпионами и, кажется, эстрагоном.
Мы же с пылероями воюем постоянно, трофеев много "--- оружие, одежда, еда.
Они распробовали наше, мы распробовали их, в итоге и вышло, что жрём-то мы с собаками одни и те же деликатесы.
А вот в джунглях пылеройский хлеб почти не известен, потому как скорпионов там не разводят.

\section{Бортники}

Мимо прошли три бортника в плотных зелёных костюмах, запевая тягучую <<медовую>> песню.
Они несли три круглых, промазанных глиной корзины и клетку, в которой заливисто чирикала пара ручных попрошаек.

\section{Душ}

"--*Здесь твоя лежанка, "--- показал Трукхвал.
"--- Бельё сам стирать будешь.
Хай, а вон та дверь "--- душ.

"--*Что?

"--*Душ.
Идём покажу.

Трукхвал открыл дверь и, занеся руку с факелом, осветил тёмное помещение.

"--*Это изобретение ноа, у которых туго с чистой водой.
У ноа даже есть пословица: <<Существуют только три истинных, чистых удовольствия "--- секс, душ и дорога>>.

"--*У меня не было секса, ничего сказать не могу, "--- пожал я плечами.

"--*Торопиться не надо.
Пожалуй, в Храме его даже многовато\ldotst
Так что душ "--- на случай, если тебе захочется помыться в одиночестве.
Или если в бане Кхохо или Эрликх будут чересчур приставать, они любят молодых\ldotst

Я засмеялся.

"--*А я, кстати, и не шучу.
Эти ходячие гениталии соблазнили Ликхэ на десятый день в Храме, по очереди, не сговариваясь.
И не спрашивай, откуда я это знаю\ldotst
Вон там лампа, свечу прикрывай стеклом, чтобы не погасла.
А вот и сам душ.
Наступаешь по очереди на педальки, и вода из бадьи льётся на тебя.
Здорово, правда?
Только не прыгай на педальках и не лей всякую гадость в бадью, а то были тут у нас любители\ldotst

"--*А что Ликхэ?

"--*Вот молодёжь.
Я ему про такое интересное устройство, а он про Ликхэ\ldotst "--- Трукхвал смотрел строго, но я прямо чувствовал, что пожилой жрец надо мной забавляется.
"--- Кхарас, боевой вождь, накричал на Кхохо и Эрликха "--- девочка ещё нецелованная пришла, а они, такие-сякие\ldotst
Кхохо ему "--- уже целованная, причём везде, я лично проконтролировала.
Кхарас Ликхэ "--- понравилось?
А та улыбается, как солнышко ясное.
Кхарас подумал и плюнул "--- делайте что хотите, лишь бы по согласию.

\razd

Предупреждение Трукхвала было не напрасным, но несколько неточным.
Кхохо поймала меня в том самом душе.
Когда дело плавно подходило к началу, в душ вошёл Ситрис и оттащил воительницу от меня.

"--*Это. Её. Ребёнок, "--- тихо, но отчётливо сказал воин ей на ухо.

"--*И что? "--- невинным тоном осведомилась женщина.

Ситрис отвесил ей такую затрещину, что она покатилась по полу.

Кхохо дулась на Ситриса весь вечер.
Ночью в спальне была слышна тихая ругань и звуки борьбы.
На завтраке дулся уже Ситрис, прикрывая рукой тёмно-синий фингал.
Но, видимо, воины пришли к какому-то согласию "--- отныне Кхохо вела себя со мной исключительно по-дружески.

\section{Большой дом}

Когда я был совсем маленьким, я гордился тем, что у моей кормилицы такой большой дом.
Однако потом в дом зачастили торговцы, носильщики, эмиссары соседних племён, разведчики и прочие люди, с которыми регулярно приходится иметь дело купцу.
Они часто оставались на ночь и всегда "--- на трапезу.
В конце концов я понял, что домочадцам во дворе Люм по-настоящему принадлежат только их собственные комнаты, и перестал хвастаться размерами дома перед другими детьми.

\section{Уволочи}

"--*Скоро всё будет\ldotst
Аийяхс-с, зараза\ldotse

Со стола во все стороны брызнули мелкие, похожие на кошек животные.
Чханэ схватила тяжёлое шерстяное полотенце и стала наотмашь карать воришек.
Кухня наполнилась шипением, ворчанием и жалобным мяуканьем.

"--*Хаяй, что за дела! "--- пожаловалась мне подруга, когда порядок был восстановлен.
"--- Только мясо разделала\ldotse

"--*Что такое? "--- в дверном проёме появилась Ликхэ, привлечённая суетой.
"--- Чханэ, ты готовишь?

"--*Готовлю, "--- сварливо бросила Чханэ.
"--- Тут какие-то котята у меня мясо таскают.

"--*Хай, это же уволочи, "--- засмеялась Ликхэ.
"--- Ты их никогда не видела?
Мы обычно им у окна обрезки оставляем.
Тебе я забыла сказать.
Вот они на стол и залезли.

"--*Ликхэ, зачем ты их прикармливаешь? "--- возмутился я.
"--- Они на ласточек охотятся!

"--*Да на помойку ласточек.
Традиция, да, а какой ещё от них толк?
Уволочи много не возьмут, зато мышей и всяких вредителей подъедают.
Не то что эти толстые ленивые оцелоты, которые весь день лежат и бока греют.
Чханэ, извини, речь не о тебе.

Чханэ свирепо засопела, но промолчала.

"--*Может, ты не заметил, Лис, но последние пять дождей зерно почти никто не портит.
Как думаешь, почему?

Ликхэ обиженно развернулась и ушла.

Мы с Чханэ помолчали.

"--*Зря ты, "--- наконец сказала она мне.
"--- В целом, неплохо эта рыбина придумала.
В следующее дежурство тоже им оставлю.
В этот раз, "--- Чханэ печально посмотрела на погрызенные куски кролика, "--- в этот раз они наелись от пуза, так что обойдутся.

\section{Копи}

Однажды ночью меня разбудил Трукхвал.
Он знаками показал, чтобы я оделся по-походному и взял всё необходимое.
Глаза учителя горели непонятным безумным огнём.

Под покровом ночи мы спустились в подземный ход, начинавшийся под бадьёй в душевой храма.
У меня не было уверенности, что кто-то кроме Трукхвала осведомлён об этом туннеле.

"--*Куда мы идём, учитель? "--- осмелился я нарушить молчание.

"--*К тайне, Лис.
К величайшей тайне, "--- ответил старик, подняв факел к потолку.

Трукхвал первый и последний раз в жизни назвал меня домашним именем.
Это означало высшую степень доверия\ldotst и опасность.
Я чувствовал эту опасность всеми фибрами души "--- в фигуре старика, которая вдруг потеряла привычную мне сгорбленность, в его неестественно прямой ноге, которая, против обыкновения, почти не шаркала при ходьбе, и во влажности стен старинного туннеля, которым не пользовались последние пятьсот дождей.

Подземный ход вынырнул на поверхность восле Трухлявой скалы, и почти сразу старик позвал меня в следующий, скрывающийся под корнями старой секвойи.
В нем было ещё тише и темнее, и Трукхвал остановился спустя десять шагов.

"--*Теперь слушай меня, "--- в глухом шёпоте учителя звучала небывалая серьёзность.
Несмотря на то, что нас вряд ли мог кто-то услышать, он не повышал голоса.
"--- Вот в этой котомке костюм.
Ты должен надеть его как можно плотнее, чтобы нигде "--- нигде, Ликхмас! "--- не осталось щелей и открытых мест.
Ты меня понял?
Надевай.

В котомках оказались странного покроя комбинезоны из очень плотной ткани, высокие кожаные сапоги и кожаные же перчатки, доходящие до локтей.
Я надел непривычную одежду "--- она оказалась чуть-чуть длинна и немного узка в плечах.
Трукхвал, быстро справившись со своим комбинезоном, придирчиво затянул и проверил завязки на моём.

Ощущение опасности возрастало.
Что собирался показать мне учитель?
Я лихорадочно вспоминал темы наших последних занятий.
Фауна джунглей, которую я и без того неплохо знал благодаря старому охотнику Сирту.
Может быть, где-то гнездятся опасные насекомые "--- зелёные пчёлы или того хуже?
Но почему ночью и под землёй\ldotsq

"--*Это нечто, что опаснее зелёных пчёл, Ликхмас, "--- Трукхвал словно прочитал мои мысли.
"--- А теперь возьми эти очки и надень утиную маску.
Нет, не свою "--- из котомки.

Я выполнил все указания и привычным движением подпоясался.

"--*Нет, Ликхмас, "--- покачал головой Трукхвал.
"--- Оружие оставь.
Припасы тоже.

"--*Учитель, объясни уже, что там! "--- взорвался я, бросив нож и жалобно звякнувшую фалангу на камень.
"--- Я не ребёнок, чтобы\ldotst

Я осёкся на полуслове.
Пещера усиливала голос, словно храмовый зал.
Я напряжённо вслушивался в гулявшее вокруг эхо.
Трукхвал виновато смотрел на меня.

"--*Прости, ученик.
Я немного заигрался.
Я\ldotst я обнаружил очень сильную Каменную ярость.

Я ахнул.

"--*Но зачем мы к ней идём, к ней же нельзя приближаться?

"--*Разумеется, можно, "--- пробормотал учитель, и лицо его смягчилось.
"--- Неприятно, но если ненадолго и при защите, то можно.
Идём, расскажу всё по дороге.
Да оставь ты его, "--- недовольно добавил Трукхвал, увидев, что я по привычке схватил мешок с припасами.
"--- Если мы там застрянем, то что с припасами, что без "--- всё одно смерть.

"--*Утешил, "--- огрызнулся я.
Трукхвал хрипло захохотал, и его приглушённый маской смех эхом отозвался в коридорах.

\razd

Сразу за поворотом пещера резко пошла на уклон.
Иногда ход шёл настолько круто, что нам приходилось цепляться за покрытую глубокими трещинами стену.
Удивительно, но здесь не было никакой живности "--- вполне возможно, что тот, закрытый старым деревянным люком вход был единственным.
Мы шли, как мне показалось, очень долго, на поверхности уже давно должны были запеть утренние птицы.

Наконец мы добрались до размётанного кострища.

"--*Пришли, "--- Трукхвал весело передал факел мне и рванул вперёд с неприличной для старого хромого жреца прытью.

Ход оканчивался идеально овальным окном в человеческий рост.
На лежащих рядом камнях я заметил следы кирки.

"--*Это я, "--- гордо признался Трукхвал, проследив за моим взглядом.
"--- Мы сейчас в одном из рукавов копей Древних, их уже давно обшарили наши предшественники в поисках сокровищ.
А этот вход никто, кроме меня, не заметил.
И немудрено "--- слышно его лишь при стуке в стену, и потребовалось пять дней, чтобы расчистить путь киркой.
Идём, Ликхмас-тари, здесь начинаются чудеса.

Трукхвал подбежал к овальному проходу\ldotst и едва не рухнул в бездонную пропасть.
Я успел схватить старого жреца за комбинезон и втащить обратно.
К моему удивлению, учитель снова захохотал.

"--*Ликхмас, отпусти меня.
Ты молодец, среагировал, но в этом не было необходимости.

Трукхвал снова подошёл к пропасти\ldotst и шагнул прямо в неё.
Я ахнул "--- учитель стоял в воздухе, словно на твёрдой поверхности.

"--*Ну как?! "--- торжествующе крикнул он в маску.

"--*Не может быть\ldotst "--- я всё ещё отказывался поверить глазам.

"--*Иди ко мне.

"--*И я тоже\ldotsq

"--*Сможешь.
Только иди, словно по дороге, никаких лишних движений.

Я подошёл к краю, шагнул вперёд\ldotst и рухнул в пропасть.
Желудок переместился куда-то к горлу, я зажмурился и едва удержался, чтобы не издать дикий крик "--- сели недостойно умирать с криком на губах.
Перед глазами пронеслась вся жизнь\ldotst

"--*Ликхмас, выпрямись!
Выпрямись, говорю! "--- весело смеясь, кричал мне Трукхвал.

Я открыл глаза.
Ничего страшного не происходило "--- тело не падало, а летело с постоянной скоростью вниз.
Я послушался Трукхвала "--- и падение замедлилось.
Вскоре мы с учителем дружно, до слёз хохотали.
Моё сердце бешено колотилось "--- под ногами не было ничего, но я словно стоял на мягкой, слегка пружинящей поверхности.

"--*Теперь слушай, "--- заговорил Трукхвал, пытаясь утереть слёзы под очками.
"--- Лесные духи, узнаю себя в первый раз\ldotst
Наклон вперёд "--- вниз.
Руки в зенит "--- вверх.
Руки по швам "--- стоишь на месте.
Рука в сторону "--- летишь в направлении руки.
Попробуй, и главное "--- не бойся.
Разбиться, столкнуться с чем-то и даже просто поцарапаться о стену магия тебе не даст "--- я пробовал.

\razd

"--*Я уже устал летать, "--- признался Трукхвал спустя некоторое время.
"--- За эти слова в Храме меня бы забили метлой, но возраст есть возраст.
Давай вниз, нам нужно лететь до цифр на стене.

Я сделал изящный пируэт и остановился рядом с учителем.

"--*Цифры как у нас? "--- уточнил я.

"--*В точности как у нас.
Раньше они светились, а сейчас почему-то перестали.
Да и глаза уже не те\ldotst
Так что не пропусти, нам нужна дверь с числом 1D, почти в самом низу.

"--*А это что, учитель? "--- я указал на надпись, которую приметил ранее.
Трукхвал подлетел поближе.

"--*Очень интересно, "--- старик привычным движением вытащил из-под комбинезона пергамент и рассыпчатый уголь.
"--- Сколько летаю, а этой надписи ещё не замечал.
Молодец, ученик.

Трукхвал приложил пергамент к надписи и протёр его углём.

"--*Я\ldotst пишу книгу об этом месте, "--- признался он, спрятав пергамент в сапог.
"--- Эти знаки на стенах я копирую "--- вдруг здесь написано что-то важное.

"--*А почему ты не сообщил ничего нашему Храму? "--- удивился я.

Старик брезгливо фыркнул "--- я впервые слышал от него такой звук.

"--*Нашему Храму лишь бы только что запретить.
В копи вообще ходить нельзя "--- жрецы знают, что там встречается Каменная ярость.
Только крестьяне плевали на запреты, а вот если в копях поймают меня\ldotst "--- старик выразительно щёлкнул пальцами.
"--- Так что пока соберу материал, а потом представлю Советам.
Или ты представишь, если я не успею\ldotst

<<Так вот почему ты меня позвал>>, "--- догадался я.

\razd

Вскоре уровень 1D оказался прямо перед нами.
Труквал был прав "--- ещё пара уровней, и труба заканчивалась закруглённым дном.
Стало ощутимо жарче.
Я боялся даже представить, на какую глубину мы опустились.

Трукхвал подвёл меня к маленькому окошку.

"--*А теперь следующее чудо, "--- шепнул учитель.
"--- Здесь находится дух, который пропускает к Каменной ярости.
Подчиняется он заклинаниям, написанным в воздухе.

Трукхвал изящно пошевелил пальцами.
В воздухе, пламенея, повисли пять иероглифов --- четыре слова и цифра.
В воздухе высветилось совершенно понятное слово <<ОПАСНО>>, и дверь почти бесшумно отъехала в сторону.
Я открыл рот.

"--*Ты не представляешь, Ликхмас, каких трудов мне стоило добыть это заклинание, "--- прошептал старик.
"--- Смотри.

Трукхвал написал иероглиф <<металл>>.
В ответ в воздухе загорелось несколько слов.
Я пригляделся "--- все они начинались на иероглиф <<металл>>.

"--*Это заклинания, которые дух понимает, "--- объяснил Трукхвал.
"--- Некоторые из них показывают непонятный текст, другие выводят какие-то цифры, таблицы и рисунки.
А вот это, "--- он написал в воздухе три иероглифа, "--- показывает карту шахты.

Я с восхищением смотрел на красивую трёхмерную карту, по которой были рассеяны разноцветные точки.
Трукхвал ладонью мягко крутил её, увеличивал и уменьшал по своему желанию.

"--*Сколько времени ты подбирал заклинание, чтобы открыть дверь? "--- прошептал я.

"--*Три года, Ликхмас.
Три года.
Ходил сюда каждую вторую ночь.
Кстати, в моей книге есть ещё десятка четыре полезных заклинаний.
Ну ладно, пойдём внутрь, "--- Трукхвал погладил меня по голове, привычным жестом написал заклинание, и дух, снова предупредив об опасности, отворил закрывшуюся дверь.
Мы вошли внутрь.

Пока Трукхвал возился с гаснущим факелом, я стоял рядом.
Я понимал, что сейчас произошло нечто совершенно необыкновенное но сил удивляться у меня больше не было.
К тому же существовало кое-что поважнее\ldotst

"--*А про какую опасность говорил дух, учитель? "--- спросил я.
"--- Про Каменную ярость?

"--*Тут много опасностей, "--- объяснил учитель.
"--- Из-за одной из них я теперь хромаю.

Трукхвал снял капюшон, откинул густые волосы и показал на темени давний шрам в виде шестиугольной звезды.

\razd

Дальше мы шли значительно осторожнее и тише.
Ход был очень ровным, словно вырезанным искусным мастером по камню.
Я успел заметить на карте, что лабиринт простирался очень далеко "--- возможно, у Трукхвала ушли годы на его исследование.
Время от времени нам попадались странные железные существа "--- не то жуки, не то пауки размером с жилище.
Они лежали неподвижно, расклячив металлические ноги, словно раздавленные огромной ногой.

"--*Одна такая меня и ударила, когда я попытался под неё залезть, "--- шёпотом сказал Трукхвал.
"--- Проткнула голову, словно кусок масла, я едва успел присесть и отпрыгнуть, чтобы меня не пригвоздило к полу.
Когда очнулся, понял, что череп пробит, а нога не хочет двигаться.
Залил голову смолой, кое-как выбрался, отлежался в храме, стало лучше, но до конца нога так и не вернулась.
Головные боли мучили где-то десять дождей\ldotst
Раньше эти звери, если верить хроникам, валялись везде в копях, но местные растащили большую часть на металл.
Никто даже не считался с тем, что это звери Древних "--- шла война с ноа, и металл был нужнее.
Остались только здешние "--- после того случая я попросил у них прощения и пообещал больше не сердить.

\dots Сколько мы шли?
Вероятно, несколько часов "--- под землёй время тянется очень медленно.
У меня начало сосать под ложечкой от голода.
Наверняка уже полдень.

Наконец в воздухе появилась странная неприятная дымка.
Трукхвал обратил моё внимание на неё, велел смочить утиную маску и ещё раз проверить комбинезон.
Дымка становилась всё более отчётливой.
Странные стальные звери почти перестали попадаться.
И вдруг\ldotst

"--*Пришли, "--- выдохнул Трукхвал и, поёжившись, потушил факел.

Привыкшие к жёлто-инфракрасному факелу глаза какое-то время ничего не видели.
Но я понял, что дымка не висела в воздухе "--- всё вокруг освещалось слабым-слабым свечением, льющимся откуда-то справа.
В нём тени как будто тонули.
Границы пещеры расплывались, словно стены состояли из желе или тёмного стекла\ldotst
Это свечение вызывало непонятную, смутную тревогу и страх.

"--*Видишь источник? "--- сказал Трукхвал.
"--- Справа, похож на молнию, застывшую в камне.
Это и есть Каменная ярость.
Кстати, я увижу твои косточки, если хорошо пригляжусь.

Я поднял руку и взглянул сквозь неё на Каменную ярость.
Трукхвал сказал правду "--- свет Каменной ярости пронзал плоть.
Рука по-прежнему слабо светилась теплом "--- одежда приглушает естественное свечение тела.
Но появился новый оттенок, новая игра света.
Я видел каждую косточку в моём запястье.
А вот и сломанная когда-то в играх фаланга пальца, которая не совсем правильно срослась\ldotst

"--*У тебя лишняя сесамовидная косточка в колене справа, "--- весело сказал Трукхвал.
"--- Повернись-ка.
О, и зубы не все ещё выросли.

"--*Чудеса, "--- шепнул я и обежал учителя, чтобы рассмотреть его скелет.
"--- Хай, вижу отверстие в черепе.
Ключица неправильно срослась.
И скоба на ребре.

"--*Скоба? "--- учитель удивлённо посмотрел на грудь.
"--- Хаяй, это мы тогда с Ситрисом чинили карниз, я про неё забыл совсем\ldotst
Ключица тоже оттуда, с Ситрисом вообще ничего чинить нельзя.
Всё, Ликхмас, играм конец.
Мы достаточно долго здесь пробыли.
Забавно, конечно, но у меня, если честно, мороз по коже от этого свечения.
Идём к выходу, и старайся не задевать железных зверей.

"--*Но мы только\ldotst

"--*Ликхмас, если мы чего-то боимся, то это не просто так, "--- тон учителя был суров.

\razd

На выходе мы избавились от комбинезонов и масок.
Трукхвал велел выбросить их в пропасть, когда мы поднялись по воздуху к прорубленному в камне туннелю.

"--*Когда придёшь, в душе не мойся, "--- предупредил меня учитель.
"--- Иди к реке ниже города и поплавай подольше, отмокни.
Пыль Каменной ярости невероятно прилипчивая, и городским глотать её ни к чему.

"--*Учитель, что будет, если остаться там надолго? "--- задал я вопрос, мучивший меня всю дорогу.

"--*Долгая, неотвратимая и мучительная смерть, "--- коротко ответил Трукхвал.
--- Бывали те, кто доставал на поверхность <<око земли>> "--- маленький сгусток Каменной ярости, который светит ярче Солнца.
Их города поражала эпидемия.
Каменная ярость должна оставаться в камне, Ликхмас.

"--*Зачем же её добывали Древние, учитель?

"--*Хотел бы я знать, "--- вздохнул Трукхвал.

Я смотрел на седые длинные локоны старика, на его спокойные зелёные глаза, крючковатый нос и волевой, совершенно не старческий подбородок.
Больше моего учителя не существовало.
Рядом со мной шагал Герой, который в одиночку сделал шаг в пропасть, пролетел по воздуху, поговорил с духом ворот и сразился огромным с железным пауком ради того, чтобы одним глазком взглянуть на Каменную ярость.

"--*Не смотри на меня так, "--- нахмурился Трукхвал и отвернулся.
"--- Я этого не заслужил.

\section{Дохляк}

"--*Этот дохленький, "--- сказал кормилец.
"--- Если вырастет "--- пустим на суп.

Цыплёнок действительно был слабее и меньше прочих.
Я чувствовал в нём не просто слабость.
Он был другим, не похожим на прочих цыплят.

Я схватил цыплёнка в охапку и бросил его о стену.
Он жалобно запищал.
Я сделал это ещё раз, и ещё.
После десятого броска цыплёнок вдруг запищал чуть громче и захромал.
Смутившись, я повернулся и убежал.

За ужином Хитрам сидел озабоченный.

"--*Что за день, "--- сказал он.
"--- Ликхмас, ты сегодня в полдень кур кормил?

"--*Да, "--- буркнул я.

"--*Дохляк лапку сломал, "--- объяснил кормилец.
"--- Понятия не имею, как у него это получилось.
Надеюсь, что это простая случайность.

Я промолчал.

Дохляку Хитрам наложил шину, но лапа упорно не желала срастаться.
Цыплёнок так и хромал до самой своей смерти.
Суп получился вкусный "--- в меру жирный и наваристый.
Все кушали и нахваливали Сирту-лехэ, который оказался знатоком приготовления куриных супов.

Я по-прежнему кормил кур, но старался отделаться от этого как можно быстрее.
В конце концов, кем я был?
Обычным ребёнком, чья голова занята играми, товарищами и прочими интересными вещами, которыми полон мир.
Мне совсем не хотелось думать о том, что я сломал чью-то маленькую жизнь.
Но единожды осознанное из головы уже не выбросить.

\section{Бывшая женщина}

В дверь без стука заглянула улыбчивая растрёпанная крестьянка "--- с ней кормилец когда-то жил.
По его словам, дом Кхатрим был чем-то вроде постоялого двора для мужчин "--- почти все приехавшие получали там дом, пищу, постель и женщину в лице хозяйки.
Взамен работали в поле, смотрели за детьми и чинили дом.
Детей у Кхатрим на тот момент было аж шесть "--- весьма внушительное количество, учитывая Отбор и низкую плодовитость сели.
Двое из них "--- братья-близнецы Марас и Хатрас "--- были от кормильца.
Во всяком случае, внешнее сходство присутствовало.
Её питомцем был и Столбик, мой друг детства.

Кхатрим заглядывала к нам регулярно, в последних числах месяца.
Хитрам иногда пропадал у неё на день-два "--- отдохнуть, проведать детей и помочь по хозяйству.
Сегодня же, видимо, случилось что-то серьёзное.

"--*Хай, вы обедаете.
Прошу прощения.

"--*Да, Кхатрим? "--- обернулась кормилица.
"--- Ты что-то хотела?

"--*Мне срочно нужны руки, хотела Хитрама попросить.
Дерево упало прямо на дом, хорошо ещё, что мы все в поле были, "--- радостно отозвалась женщина.

"--*Да ты что! "--- расстроилась Кхотлам.
"--- Вам приют не нужен?

"--*Я с мужчинами пока живу у соседей, но ради такого случая могу прийти в гости.

"--*Заходи вечером, "--- кивнула Кхотлам.
"--- Я что-нибудь приготовлю.
Хитрам, иди, я тебе потом согрею.

Кормилец кивнул и пошёл к двери.

"--*А я тебе предлагал его спилить, "--- бросил он.
По его лицу бродила слабая улыбка.

"--*Я помню, "--- заулыбалась Кхатрим.
"--- Оно было красивое и очень хорошо смотрелось рядом с домом.

\section{Одна большая семья}

"--*Ведь почти все прочие племена обучают военному искусству только мужчин.
Есть, правда, трами, у которых воюют только женщины.
Знаешь, почему наши воины "--- мужчины и женщины?
Потому что чисто мужская или чисто женская армия "--- это армия насилия!

"--*Я слышал, что воины некоторых Храмов представляют собой одну большую семью, "--- заметил я.
"--- Единственное, в чём им отказывают "--- это в воспитании маленьких детей.

"--*Так и есть, "--- согласился Трукхвал.
"--- У нас тоже, хай, семья.
Можно сказать.
Хай, да.
Только в обычной семье обязательно есть взрослые, а у нас взрослых, хай, нет.
Вообще.
Ни одного взрослого.

\section{Соль земли}

Однажды Трукхвал повёл меня на крышу.
Только-только начиналась страда;
по городу звучали крестьянские песни, катались гружёные тележки и ходили носильщики с корзинами.

"--*Посмотри, Ликхмас.
В библиотеке легко забыть о них "--- о тех, кто снаружи храма.
В мире было много заблуждений на их счёт: считалось, что народ "--- исполнители воли, дешёвая оправа для драгоценного камня деятелей искусств, вождей или учёных.
Кого видишь ты?

Я посмотрел вниз.

"--*Силу.

"--*И это истина.
Что ты видишь ещё?

"--*Труд.
Тяжкий труд.

"--*Никогда не забывай о том, кто тебя кормит и защищает до поры до времени.
Деятели искусств, вожди и учёные гораздо хуже рабов в этом отношении.
Если у раба не будет хозяина, он будет хоть что-то из себя представлять.
Храм без народа "--- ничто.
Даже если народ обратит против Храма клинки, мы не можем ответить им тем же.
Не потому, что не способны, а потому, что перестанем быть Храмом.
Мы "--- щит, и рука под ремнями "--- рука народа.

Мы помолчали.

"--*Пусть тебя не введёт в заблуждение и слава, Ликхмас.
Кто твой любимый лесной дух?

"--*Митр, наверное, "--- усмехнулся я.

"--*Верно.
Утешающий, излечивающий душевные раны и усыпляющий не знающих сна.
Однако я клянусь тебе, что самая безвестная женщина, что вынашивала детей и видела, как изменяются их тела и мысли, излечила больше ран, чем Печальный Митр.
Помни об этом.

"--*Я буду помнить.

"--*В этом городе сила, "--- мимоходом пробормотал Трукхвал под нос.
"--- И ведь верно заметил.
Почему я не\ldotsq
Почему-то я не\ldotst
Хай\ldotst Засиделся я в библиотеке\ldotst

\section{Ситрис и штаны}

\textbf{(Часть про платки надо вставить в Семью или ранее)}

Ситрис сидел в зале и шил штаны.

"--*Кхарас опять порвал.
И опять в паху, "--- объяснил воин.
"--- Хозяйство у него чересчур большое.
Попробую дополнительно укрепить.

Ситрис сделал ещё несколько стежков.

"--*Когда я\ldotst только познакомился с твоей кормилицей, я сделал нехорошую вещь, "--- тихо усмехнулся он.
"--- Она в наказание заставила меня вышивать салфетки.
Целых полгода я сидел и шил проклятые салфетки.

"--*А за что? "--- удивился я.

"--*Не суть, "--- отмахнулся Ситрис.
"--- Это было ужасно.
И кто из лесных духов дёрнул меня за язык сказать об этом в Храме?
Теперь обшиваю всех.
Шью всё, что угодно, но только бы не салфетки.
Иногда Кхохо или Эрликх подкладывают мне в постель квадратные лоскутки и цветные нитки, а я их за это бью.

Ситрис продолжил работу, тихо добавив: <<Уроды пресноводные>>.
Но по его лицу бродила слабая довольная улыбка "--- спокойное занятие воину явно нравилось.

\section{Тысяча платков}

"--*Когда я пришёл к твоей кормилице, чтобы она ходатайствовала за меня в Храме, я знал, что рассказывать.
Сообщил ей о том, что я был разбойником, что желал бы спокойной жизни.
Она, не слушая меня, задала только один вопрос "--- что из сделанного мной я считаю самым постыдным.

Однажды во время налёта на караван Ситрис, приставив нож к горлу предводительницы, потребовал отдать всё, кроме одежды.
При обыске разбойник нашёл вышитый платок из грубой ткани.
Предводительница прошипела, чтобы он держал руки подальше от платка.
Ситрис рассёк ей лицо двумя ударами, оставив на лице женщины уродливую широкую улыбку.

"--*Кхотлам сказала, что представит меня Храму, если я вышью тысячу платочков.
Каждый день в течение почти полного года я сидел и вышивал платки\ldotst

Лицо Ситриса перекосилось.
Он рассказывал это уже в двадцатый раз.
Каждый раз рассказ доставлял ему жуткий дискомфорт, но замолчать, как видно, ему было очень трудно.

"--*Я ел за столом твоей кормилицы, спал возле её очага и вышивал эти платки\ldotst

"--*Тысяча "--- это очень много, "--- сказал я.
"--- Кхотлам просто хотела посмотреть, сколько ты продер\ldotst

"--*Я их вышил, Ликхмас.
Всю тысячу, "--- перебил меня Ситрис, похлопал по плечу и ушёл в темноту.

\section{Сдерживатель темперамента}

\mulang{$0$}
{"--*Иногда мне кажется, что моё единственное предназначение "--- сдерживать темперамент Кхохо, "--- буркнул Ситрис.}
{``Sometimes it seems my only destiny is to restrain Kch\`{o}h\^{o}'s temperament,'' S\~{\i}tr\v{\i}s grumbled.}
\mulang{$0$}
{"--- Если бы мы не встретились, Кхохо уничтожила бы мир к свиньям и устроила бы богам истерику, что мир так быстро сломался.}
{``If we hadn't met, Kch\`{o}h\^{o} would have destroyed the world and made an uproar to gods because it wasn't stable enough.''}

\section{Любовь к языкам}

"--*С сегодняшнего дня мы будем учить языки Трёх Материков, "--- начал Хонхо-лехэ.
"--- Прошу вас, запомните сказанное.
Мне доводилось слышать многие отговорки, почему человек не знает языков ближайших соседей.
Самой смешной отговоркой было отсутствие способностей к языкам.
Чушь!
Всё это полная чушь, сказанная человеком, который как минимум один язык знает в совершенстве!
Язык "--- это инструмент, который вы всегда носите с собой и можете применить.
Язык "--- это оружие, для которого не нужно ножен и стрел.
Знание языков "--- это путь любого человека, который живёт в обществе, а не в одиночестве посреди голой пустыни.
Запомните это навсегда.
Вы можете презирать дикарей хака, вы можете ненавидеть идолов Живодёра, но их языки полюбите, как кормильцев и лучших друзей!

Слова старого жреца запомнились мне на всю жизнь, и я процитировал их детям.
Дети закивали в знак того, что поняли.

"--*Хорошо.
Кто из вас знает язык цатрон?

В классе неуверенно поднялись две-три руки.

"--*Спрошу по-другому, "--- я перешёл на цатрон: "--- Сколько мер зерна требуется на среднее поле?

"--*Одиннадцать, "--- хором грянула половина класса.

"--*Чьи корабли имеют раздвоенный штевень?

"--*Ноа, "--- грянула треть.

"--*А когда птицы вили гнёзда среди скал в Сикх'амисаэкикх?

"--*<<Пока дикарка ждала и плёлся паучий шёлк>>, "--- раздался в полной тишине тоненький голосок.
Все недоумённо посмотрели худенькую девочку, и она стушевалась.

"--*Эту песню кормилец напевает в поле каждую страду, "--- объяснила она.

"--*А говорите "--- не знаете, "--- засмеялся я.
"--- Говорите только редко.
Это исправим.
Вы ещё и писать на нём будете.

\section{Загнанный зверь}

Сирту-лехэ долго и упорно обучал меня искусству общения со зверьми.
И только сейчас я понял: тело "--- такой же зверь.
Оно думает само, и нужно большое искусство, чтобы подружиться с ним и войти к нему в доверие.

<<Помни, Ликхмас: всё твоё искусство может оказаться бесполезным, если объятый страхом олень понесёт.
Но ещё более опасен зверь, которого загнал ты сам, ибо ты над ним более не имеешь власти>>.

\section{Хэма}

Хэма удерживается в волосах острой шпилькой, напоминающей стилет.
Этой шпилькой вполне можно было ранить или убить;
однако, если её вытаскивали, то тяжёлая заколка хэма немедленно падала на пол.
Это было первым напоминанием: если дипломат берётся за оружие, он тут же перестаёт быть дипломатом.
Вторым напоминанием был вес хэма, заставлявший держать голову высоко поднятой;
лицо дипломата "--- не только его собственное лицо.

\section{Оружие и дипломат}

"--*Почему ты не взяла оружие?
Хоть ты и дипломат, тебе нужно\ldotst

"--*Нет, Лисёнок.
Или одно, или другое.

"--*Послушай\ldotst

"--*У купцов, как и у Храма, тоже есть традиции, малыш, "--- перебила меня Кхотлам.
"--- И я по мере сил стараюсь их чтить.
Меня учили, что дипломат должен всё, включая собственную безопасность, обеспечивать словом, взглядом и жестом.
Если ты этого не можешь "--- распишись в собственном непрофессионализме.

\section{Местоположение воина}

"--*Ситрис, Трукхвал мне жаловался, что ты постоянно отключаешь наруч и он не может тебя найти, "--- сказала Ликхэ.

"--*Трукхвал уже надоел со своими экспериментами, "--- проворчал Ситрис.
"--- Сама идея наруча великолепна, но следить за моим местоположением без моего ведома я ему не дам!

"--*А тебе есть что скрывать, Ситрис? "--- вскинулась Ликхэ.

"--*Да, есть.
Мои дела, которые касаются только тех, кого я хочу!

"--*Соглашусь с Ситрисом, "--- заявила Хитрам.
"--- Воин на то и воин, чтобы его не могли просто обнаружить "--- в том числе и союзники.
Сам по себе наруч "--- копьё с двумя клинками\footnote
{Нечто, опасное и для противника, и для того, кто этим пользуется. \authornote},
и я об этом уже говорила Трукхвалу.
Но он помешался на своей технике.

\section{Вопрос}

Кхарас перехватил моё ружьё.

"--*Кхохо уже там, "--- сказал он.
"--- Не трать стрелы.

Кхохо действительно была уже там.
Она невозмутимо сидела на песочке;
воины хака весело переговаривались, не замечая присутствия чужого.

"--*Как поплавали? "--- поинтересовалась воительница на языке хака.

Воины обернулись как ужаленные и бросились на неё с трёх сторон.
Кхохо молниеносно выбросила вверх руки.
Песочек, оказавшийся мелкой сухой пылью, совершенно скрыл воительницу и нападавших;
когда пыль осела, я увидел пять окровавленных тел и отряхивающуюся Кхохо.

"--*Придурки, "--- буркнула она.
"--- Я же просто спросила!

\section{Маска Сита}

Уже на пороге храма я понял недовольство Кхохо "--- костюм действительно был очень неудобным.
Маска натирала лоб и щёки, а ветки цеплялись за всё, что попадалось на пути.

"--*Так, подождите, я быстро подгоню маску, "--- шёпотом сообщил я и ринулся в мастерскую.

"--*Ликхмас, Ликхмас, плохая идея, брось, "--- зашептал мне вслед Ситрис.

Быстро отыскав резец и пилку, я попытался снять заусеницы.
Хрусть "--- и маска разломилась пополам.

"--*Вот сказали же тебе, "--- недовольно буркнула Ликхэ из-за плеча.
"--- Думаешь, я не пыталась в позапрошлом году ещё?
Тащи скобы, в зелёном сундуке.

"--*Новую уже давно надо, "--- прошипел я в ответ.
"--- Почему не заказали-то?
Праздник один день в году!

"--*Да руки всё не доходили, "--- пожала плечами Кхохо.
"--- Ликхмас, проведёшь ритуал "--- хоть весь храм перестраивай.
А сейчас залатай маску на скорую руку, чтоб тебя идолы поимели на перекрёстке!
Тащи скобы!

\section{Кукольный театр}

Весело пылал костёр, Чханэ заливисто смеялась, лёжа на раскинутом плаще из кожи нимелто.
Её обнажённая грудь подпрыгивала, оливковая с оранжевой искоркой кожа блестела, как бронза, от огненных всполохов.
Доспехи и оружие валялись рядом в беспорядке.

Несмотря на спешку, привал нам сделать пришлось "--- по дороге попался колодец.
Мы напоили измождённых, жалобно глядевших на нас оленей, напились сами и омыли усталые тела в предварительно подогретой воде.
Вернее, мылся только я.
Чханэ, забыв про кишащий идолами лес, брызгала на меня водой и громко хохотала.
Потом я подбросил в огонь отгоняющих насекомых пряных трав, и мы с девушкой поочерёдно прыгали через костёр, прямо в удушливый дурманящий дым, пока не пропахли им насквозь.

Я рассказывал ей интересные случаи, произошедшие со мной в других мирах.
Рассказывал легенды и мифы давно ушедших и забытых народов.
Читал стихи, предварительно установив <<мост сознания>>, дабы она могла понять их смысл и оценить их красоту.
Но ни забавные истории, ни стихи не оказали такого действия, как кукольный театр, который я видел в своём родном мире.
Сделав пять куколок из дерева и тихонько играя на флейте, я заставил их воспроизвести старый-престарый спектакль <<Жених драконихи>>.
Чханэ смеялась и хлопала в ладоши, как ребёнок.

"--*Ещё, ещё!

Я показал ещё один спектакль, <<Апельсиновый сад>>.
Радости девушки не было предела.
Глядя на её сияющее лицо и влюблённые глаза, я в очередной раз подумал: <<Бедняжка\ldotst>>

"--*Ты совсем не изменился, Лис.
Прости.
Я напрасно на тебя накричала тогда, в Тхитроне.

"--*Хаяй, а я что тебе говорил? "--- улыбнулся я.
"--- Ничего и не изменилось.

Медленно светлело усыпанное звёздами небо.
Задул первый утренний ветерок.
Чханэ принюхалась к воздуху и некоторое время смотрела на меня со странной, грустной и извиняющейся, улыбкой.

"--*Лис, прости меня, "--- тихо проговорила она.
"--- Я понимаю, что мы теряем время, понимаю, что твои дела чрезвычайно важны.
Но это было необходимо, потому что, сердцем чую, нескоро мы ещё так отдохнём.
Может, вообще в последний раз\ldotst

Она запнулась и ещё немного помолчала.

"--*Давай поспим вместе до рассвета.
Ты уже сутки глаз не сомкнул, я же знаю.

Я с нежностью посмотрел в печальные огнистые глаза.
Нельзя, Чханэ, нельзя мне спать.
У меня столько работы\ldotst

"--*Я могу долго обходиться без сна, Змейка.
Ложись.

"--*Змейка тебя ждёт, "--- игриво усмехнулась девушка.

Я откинул входной клапан палатки.
Чханэ вздохнула, встала, мимоходом погладив меня по шее, затащила внутрь всю свою амуницию и некоторое время возилась внутри, устраиваясь поудобнее.

Следовало ещё поискать Анкарьяль.
Я не надеялся на успех, чересчур много случайностей происходит.
Её могли убить в бою, принести в жертву, наконец, она могла совершенно банально умереть от болезни.
Демон не был абсолютной защитой от житейских бед, и с риском приходилось мириться.
Но поискать следовало в любом случае.

Я слегка напряг запястье, и браслет ожил.
В палатке зашебуршились, входной клапан отодвинулся, и на моих плечах повисла сияющая Чханэ.

"--*Змейка тебя не дождалась.
Хай, а что это?

"--*Прибор.
Ищет демонов.
Нужно найти Анкарьяль.

"--*Анкарьяль — женщина? "--- строго спросила Чханэ.

"--*Она настроена на женское тело, "--- ответил я с улыбкой.
"--- Неужели наследница самого Маликха подхватила на болоте ревность?

"--*Брось-брось-брось, "--- Чханэ замахала руками на голубые значки и таблички, и программа тонко чувствующего прибора пустилась в безумный пляс.

"--*Хэ, Чханэ, не хулигань! "--- я выключил браслет.
"--- Тебе колыбельную спеть?

"--*Мне не нужна колыбельная, если не на чем спать, "--- надулась девушка.

"--*А плащ?

"--*Он не греет.

"--*Он и не должен греть, он должен хранить тепло!

Чханэ вздохнула, расстелила плащ на земле и села передо мной на колени.

"--*Я тебя совсем перестала привлекать, да?

Тьфу ты.
Вон оно что.
Вечно я забываю какую-нибудь важную мелочь, когда дело касается психологии.

"--*Ты теперь не Лис, а Акхатху, "--- девушка была серьёзна.
Она не спрашивала, а утверждала.
"--- Сохранилось ли в тебе что-то из прежних чувств?
Любишь ли ты меня?

Как растолковать ей, что чувства "--- в большинстве своём производные плоти, важной, но не обязательной части меня?
Объяснения лишь умножат её сомнения.
Ответ на её вопрос может быть только один\ldotst

"--*Конечно.

Не так уж я и слукавил.
Я всем сердцем желал этому существу счастливой жизни.
Прежний, непробуждённый я жертвовал статусом, здоровьем и своей жизнью, чтобы всего лишь облегчить муки Чханэ.
Чем я отличался от него?

"--*Тогда удели мне время до рассвета.
Мне одной.
Не людям, не войне, не вашим хоргетам.
Мне.
Пожалуйста.

Чханэ, не дожидаясь ответа, приникла к моим губам, как умирающий от жажды пустынник.
Я ответил лаской, как мог.
Мир вокруг завертелся.
Вокруг неё.
Вокруг нас.

Рассвет мы встретили в палатке в объятиях друг друга.
Я смотрел на её лицо, угадывая черты далёких предков, о которых она даже не подозревала.
Я смотрел на её лицо и видел тысячу других людей.
Молодых, старых.
Мужчин, женщин, детей.
Тысячи, десятки тысяч жили в этом простом, нежном контуре щёк, иссушённых горячими ветрами уголках миндалевидных глаз, твёрдых сомкнутых губах.
И было во мне лишь одно желание "--- любоваться ею, как сейчас, любоваться всегда, вечно, а не те жалкие двести дождей, отпущенные этому телу\ldotst

Рассвет я встретил другим.

\section{Встреча с Грейсом}

Ихслантхар встретил нас неприветливо.
Дождь, грязь по щиколотку.
Едва найдя постоялый двор, мы поняли, что лучше остановиться у кого-нибудь из горожан.
Двое или трое вежливо отказали, сославшись на тесноту, ещё десять были гораздо менее вежливы, но подавляющее большинство даже не откликнулось на <<гостевой>> стук.
Пришлось вернуться на постоялый двор.

Хозяин, в противоположность горожанам, оказался на удивление вежливым и отзывчивым человеком.
Однако его вежливость и отзывчивость стоили дорого.

"--*У нас не хватит золота надолго здесь остаться, "--- резюмировала Чханэ.
"--- Сколько времени тебе потребуется, чтобы найти друга?

Я промолчал.
Поиск товарищей всегда был проблемой;
пробуждённые, разумеется, узнавали друг друга сразу, но спящего демона засечь практически невозможно.
Тем не менее способ был.
Демон даже в спящем состоянии может заставить тело совершать некие стереотипные действия, которые выдадут его товарищам.
Я с сожалением посмотрел на стену дождя за окном.
Разумеется, все жители сидят по домам.

Наступила ночь.

Первой ушла спать Чханэ, уведя мальчиков.
Митхэ сидела со мной.
Вскоре её разморило, и девочка заснула у меня на коленях, разлив по полу отвар.
Зал постепенно опустел, но хозяин стоял у стойки, словно кого-то ожидая.

Вдруг колокольчик у дверей зазвенел, и в зал вошла грустная, немного сутулая молодая женщина.

"--*Тебе как обычно? "--- вскинулся хозяин.

"--*Добавь ещё перца, "--- попросила женщина.

Женщина не глядя опрокинула в себя три чаши крепкого пива с перцем, закусила орехами и заплакала.

"--*Опять? "--- сочувственно сказал хозяин.

"--*Он меня вообще не слушает, "--- тихо прорыдала женщина.
"--- Бормочет ночами, а сам\ldotst
Как будто я не существую.

"--*Тхартху, позови всё-таки жрецов, "--- посоветовал хозяин.
"--- Это уже не дело.
Секхар всегда был чудным, но сейчас с ним явно что-то не в порядке.

"--*Прошу прощения, "--- поднял я руку.
В пустом зале слова отозвались эхом, и собеседники вздрогнули.

"--*Вам что-то ещё? "--- спросил хозяин.

"--*Нет-нет, "--- покачал я головой.
"--- Просто так вышло, что я жрец.
Не храмовый, путешествую.
Если нужна моя помощь, я помогу.

"--*Я не смогу оплатить твои услуги, "--- сказала женщина.
"--- Храмовники сделают всё бесплатно, и\ldotst

"--*И по какой-то причине ты до сих пор к ним не идёшь, "--- подхватил я, "--- а изливаешь печаль трактирщику.

"--*Это мой брат, "--- с укором сказала женщина.
"--- Как ты смеешь!

"--*А пиво "--- твоя сестра.

Женщина сжалась и бросила взгляд на хозяина.
Тот отвернулся и сделал вид, что протирает посуду.

"--*Приюти меня, мою женщину и троих детей на четыре дня, "--- сказал я.
"--- Я посмотрю, что с тем человеком.

Тхартху снова бросила взгляд на хозяина, но тот, видимо, решил больше не вмешиваться.
Наконец женщина кивнула.

"--*Пойдём со мной.

"--*Я присмотрю за девочкой, "--- добавил хозяин и кивнул мне.
Я аккуратно положил головку спящей Митхэ на скамью и укрыл потеплее плащом.

Улица, поворот, улица, поворот, улица, поворот\ldotst
Я уже начал подозревать, что Тхартху наивно пытается запутать меня.
Но вот по левую руку появилась кузня.
Прошлёпав в грязи, мы поднялись к красивой узорчатой двери.

Всего одна лампа освещала жилище.
Это было не пламя;
в обычном фонаре весело жужжал настоящий маленький светодиод.
У верстака сидел и клевал носом высокий мужчина с небольшим брюшком.
Длинные спутанные <<рыбки>> с металлическими кольцами подмели столешницу, когда он обернулся к вошедшим.

"--*Наконец-то, "--- проворчал низкий голос знакомыми интонациями.
"--- Я тебя заждался.

Я без слов подошёл и заключил его в крепкие объятия.

\razd

"--*Еда ещё осталась.
Вот суп сварил, "--- бормотал Грейсвольд.
"--- Птичка, кушай, малышка.
Прости, что я так с тобой обращался последнее время, но по-другому я просто не мог послать Аркадиу сигнал.
Долго объяснять.
Ты как сам-то, дружище?
На улице жуткая погода.

Тхартху с круглыми глазами, почти не слушая, хлебала томатный суп.

"--*Грейс, что случилось? "--- спросил я.
"--- Ты получил мой сигнал о пробуждении?
К чему эти игры в домики-дороги?

Грейсвольд покряхтел и сел за стол.

"--*Получить-то получил, но я сам напортачил с пробуждением, "--- сказал он.
Говорил друг на языке тси "--- видимо, хотел, чтобы Тхарту тоже поняла часть разговора.
"--- Чувствуешь?
Весь дом в технике.
Всё, что можно, против отслеживания.
Можешь осмотреть, только осторожно.

Я <<осмотрел>> жилище.
Наноустройства были везде, в каждом камне пола, в каждой доске стены.
Чтобы сделать такое количество, нужно\ldotst

"--*В общем, ты понял, да, "--- вздохнул технолог.
"--- Я почти без энергии.

"--*Как так вышло?

"--*У меня во время пробуждения начался акбас.
Самый настоящий, со спецэффектами.
В общем, засветился я по полной.
Агенты Картеля встали на уши.

"--*Блеск, "--- буркнул я.

"--*Я не мог послать вам сообщение раньше, "--- развёл руками Грейсвольд.
"--- Пришлось надеяться на твою смекалку.

Грейсвольд кивнул на Тхартху.

"--*Она меня очень любит.

"--*И поэтому ты вынудил её ходить в постоялый двор за выпивкой?
А сказать нельзя было?

Грейсвольд выглядел ошеломлённым.

"--*Моя женщина знает, "--- добавил я.

"--*План не на той стадии, когда можно доверять всем подряд.

"--*<<Всем подряд>>? "--- процедила Тхартху.

Грейсвольд бросил виноватый взгляд на подругу.
Та сидела, скрестив руки на груди и глядя в стену.

"--*Пять дождей, Секхар, "--- голос Тхартху дрожал.
"--- Пять дождей моей любви, но я по-прежнему <<все подряд>>.

Мы промолчали.
Тхартху грустно усмехнулась и встала.

"--*Я приготовлю места тебе, твоей женщине и детям.
Как тебя зовут?
Я не разобрала.

"--*Зови меня Ликхмас ар'Люм э'Тхитрон.
Или Играющая-Камнем-Лиса.

"--*Женщине?
Детям? "--- удивился Грейс.
"--- Ты сюда выводок притащил?

"--*Так вышло, "--- развёл я руками.
"--- Мы оба хороши.

"--*Нет, вообще очень умно.
Агенты обращают меньше внимания на семейных, демоны чаще одиночки.

"--*Всё, забыли, "--- прервал я его.
"--- Главное "--- все живы, остальное приложится.
Пойду приведу своих.
Тхартху старалась меня запутать, но не учла, что постоялый двор у вас видно из окна.

Женщина густо покраснела.

"--*Ликхмас, сейчас ночь, "--- осторожно заметила она.
"--- Я думаю, не стоит будить детей ночью.

"--*Если переезжаешь в хороший дом, подойдёт любое время, "--- ответил ей Грейсвольд.
"--- Ложись-ка ты тоже спать, у тебя ещё пиво в глазах плещется.
Я сам приготовлю лежанки.
Как тебя зовут, Аркадиу?
Ликхмас, точно, только что прозвучало.
Сколько детей?

Я показал ему большой и указательный палец, затем направился к двери.

\mulang{$0$}
{"--*Десять тысяч дождей, а мозгов нет, "--- резюмировал технолог мне вслед.}
{``Ten thousand rains, has got no brains,'' the technologist summed up to my back.}
"--- И когда только успел троих настрогать\ldotst

\section{Встреча с Анкарьяль}

"--*Вон она, "--- ухмыльнулся Грейс и указал на сторожевую башенку.

Возле башенки стояла женщина "--- высокая и поджарая, словно дикий олень.
Природную красоту несколько оттеняло хищное выражение лица, которое не пропадало даже во время улыбки.
У её бедра покоилась длинная и широкая, словно доска, кукхватровая сабля.
Короткая стрижка, татуировки на плечах и рёбрах выдавали храмового воина.

"--*Точно она, "--- убеждал меня Грейсвольд.
"--- Ты посмотри на этот шедевр.
Бревно, а не сабля.
А ещё на два цана длиннее слабо?
И на пядь шире.

"--*На татуировку посмотри, "--- ответил я.
"--- Та, что начинается под левой грудью и уходит подмышку.
Вот, смотри, руку подняла\ldotst

"--*Она, "--- Грейсвольд кивнул, шумно выдохнул и потянул меня вперёд.

Едва мы подошли, как воительница повернула к нам голову на великолепной длинной шее:

"--*Нужна помощь?

"--*О да, "--- сказал Грейсвольд.
"--- Почеши мне спинку.

"--*Выпивка и секс вон там, "--- воительница кивнула в сторону постоялого двора.
\mulang{$0$}
{"--- Если хочешь меня, то придётся подождать.}
{``If you want me, you have to wait.}
\mulang{$0$}
{В лучшем случае до конца моей смены, в худшем "--- до конца моей жизни.}
{At best, 'til the end of my shift; at worst, 'til the end of my life.''}

"--*Да какой секс, "--- буркнул технолог.
"--- Не узнаёшь, что ли?

"--*Скажем так "--- я тебя не знаю, "--- пожала жилистыми плечами женщина.
"--- Не узнаю тоже, но это следствие.

Мы переглянулись.

<<Она не пробудилась>>.

<<О светлая твердь, опять\ldotst>>

Спустя секхар мы уже волокли сопротивляющуюся воительницу в кусты.

"--*Кто я? "--- строго спросил Грейсвольд после того, как я отвесил женщине хорошую оплеуху.

Она не ответила.
Я сделал несколько весьма болезненных ударов по корпусу, и связанная воительница согнулась в три погибели.

"--*Кто я? "--- снова спросил Грейсвольд.

Молчание.
И снова сеанс экзекуции.
После четвёртого вопроса у горла воительницы появился клинок.

"--*Кто я? "--- почти ласково спросил Грейсвольд.

<<Урод>>, "--- жестами показала воительница.
Я вытащил кляп, и она закашлялась.

"--*Мне это уже надоело! "--- заявила она.
"--- У меня ещё неделю всё болеть будет!

\razd

"--*Уроды, "--- бормотала Анкарьяль, разглядывая себя в бронзовом зеркале.
"--- Оба.
Вот как я теперь с этими синяками в храм пойду?
Аркадиу, это ты мне фингал поставил?
Я же сказала "--- бить только по закрытым одеждой участкам.
И сказала я это в двадцать шестой раз!

"--*Я случайно, честное слово, "--- развёл я руками.
"--- Ты очень бодро блокировала мой первый захват, и мне пришлось тебя огорчить.
Храму скажешь, что мы просто чуть переусердствовали с объятиями "--- давно не виделись.

Грейсвольд с нежностью смотрел на подругу, рассеянно ковыряя пальцем опухшую губу и новорождённую прореху в зубах.

"--*Красивое у тебя тело.
Даже с синяками.
Прям чувствуется стиль.

Анкарьяль расплылась в белозубой улыбке.

"--*А ты какой-то не толстый.
И даже очень привлекательный, а не как обычно.

"--*Среди сели крайне мало толстяков, "--- развёл руками технолог.
"--- Взял, что было.
И вообще, это было обидно.

"--*Чего на правду-то обижаться?
Харизма в тебе всегда была, а вот на изящество ты обычно плевал.

"--*Может быть.
Вот кто совсем на себя не похож, так это Аркадиу.
Я его едва узнал.

"--*Он мелкий и худой, но какой-то очень сильный для своей комплекции, "--- Анкарьяль критически меня осмотрела.
"--- Надо потом посмотреть, что у него по генам.

Женщина снова посмотрелась в зеркало и пятернёй начесала чёлку на заплывший глаз.

"--*Пошли, "--- кивнула она нам.
"--- Вроде не заметно.
До конца смены ещё кхамит, но я обычно удирала чуть пораньше "--- скука стоять на южном конце, ничего интересного\ldotst

\section{Гора Песнопений}

"--*Что здесь произошло? "--- ошеломлённо спросила Тхартху.

Чханэ же трясло от гнева.

"--*Святилище осквернено\ldotst

\spacing

"--*Святилище было гарантией мира.

"--*А это значит, что грядёт великая война.

\section{Потеря контроля}

"--*Ты потерял над собой контроль? "--- удивился я.

"--*Да, "--- кивнул технолог.
"--- Это был системный сбой, причём он напоминал акбас.
У тебя подобного не было?

"--*Нет, "--- покачал я головой.
"--- Хотя постой\ldotst

Я посмотрел на Чханэ.
В голове промелькнули алтарь, пробуждение и странная борьба тела и демона, которой никогда раньше не бывало.

"--*У нас большие проблемы, Аркадиу, "--- глухо резюмировал технолог, словно читая мои мысли.
"--- Что-то происходит, не только вокруг, но и внутри нас с тобой.

Технолог выразительно коснулся пальцем виска.

\section{Воля Ликхмаса}

"--*Что с нами сделали наши тела? "--- пробормотала Анкарьяль.

"--*Я не знаю, "--- покачал я головой.
"--- Знаю одно "--- воля человека по имени Ликхмас ар'Люм была сильнее воли Аркадиу Люпино.
И этот факт сотворил настоящее.

"--*Кто ты сейчас?

"--*Я "--- это я.
Это всё, что я могу тебе сказать.

\section{Ликхлам}

У входа я едва успел перепрыгнуть через кого-то, лежащего поперёк прохода.

"--*Хэ!

"--*Всё нормально, это Ликхлам, "--- успокоил меня Король-жрец и перепрыгнул через лежащее тело.
"--- Он любит путаться под ногами.

"--*Вообще-то я на посту, "--- сообщило тело невероятно женственным контральто.
"--- Так, на волосы не наступайте, только расчесал.

"--*Ты хочешь отрастить волосы, как у Маликха? "--- усмехнулся я, окинув взглядом вход в храм.
Осветлённые каким-то веществом пряди лежали везде, словно их обладатель не стригся с рождения.

"--*Я тебе скажу больше, Ликхмас: я уже отрастил волосы лучше, чем у Маликха, "--- кокетливо сказал Ликхлам.
"--- Легендарная коса в два кхене и реальная в человеческий рост "--- это немного разные вещи.

"--*Не в человеческий, а в пылеройский.
В тебе два цана роста, забыл? "--- усмехнулся Король-жрец.

Мы прошли чуть дальше.

"--*На самом деле он очень опытный воин, "--- шёпотом сообщил мне Король-жрец.
"--- Его привычка лежать, а не стоять у входа уже спасала нас несколько раз.
Лазутчик поднимается, смотрит "--- в дверях никого нет.
Идёт к окнам, спокойно залезает в здание "--- и ему на плечо ложится лучший клинок Тхартавирта: <<Добрый вечер!
Чем могу помочь?>>.
К советам Ликхлама лучше прислушиваться.

"--*Я уже понял, "--- кивнул я.
"--- Он единственный смог убежать от демонов.

"--*После того, как попытался отравить их за обедом.
Никому из нас ничего не сказал, сам провёл расследование, сам спланировал покушение, когда понял, что баланс власти нарушен.
Не удалось, конечно "--- я так понял, что яд вы умеете распознавать.

"--*Скорее мы умеем распознавать эмоциональное напряжение отравителя, "--- усмехнулся я.

"--*Возможно.
Но одного демона Ликхлам всё-таки зарубил, пока убегал "--- случайно встретил.
Я же говорю "--- один из лучших здесь.

\section{Песня Ликхлама}

Из зала доносился богатый голос Ликхлама под тяжёлый бронзовый аккомпанемент цитры:

\begin{verse}
\mulang{$0$}
{Яркие краски,}{Brightness of colors,}\\
\mulang{$0$}
{Тёплая кожа,}{Skin heated by arteries,}\\
\mulang{$0$}
{Мерные мысли и новый момент.}{Race of new moments, and rhythmical minds. }\\
\mulang{$0$}
{Вечное счастье,}{Edgeless happiness,}\\
\mulang{$0$}
{Дни непохожие,}{Different afternoons,}\\
\mulang{$0$}
{В солнечных линиях новый рассвет,}{Covered by sun rays renewed sunrise,}\\
\mulang{$0$}
{Новый рассвет\ldotst}{Renewed sunrise\dots}

Долгими кхене, тропами резвыми,\\
Надо ли, надо ли\\
Идти нам в туманы ли?\\
Бежать от обмана\\
Дорогами главными,\\
Длинными ночами\\
Под буйными травами?
\end{verse}

"--*Красивая песня, "--- мечтательно проговорила Чханэ.
"--- Люблю её.

"--*Так, как Ликхлам, её никто не поёт, "--- заметил проходивший мимо Трукхвал.
"--- Даже те сотронцы, которые её считают своей, сказали, что песня как будто для его голоса сочинена.

\section{Выговор}

С улицы зашёл жрец и тут же сбросил с себя промокшую насквозь робу.

"--*Я не уверен, но, по-моему, тотем Сомнения буря снесла к свиньям, "--- сообщил он.

Двое жрецов побежали и выглянули наружу.

"--*Спорное утверждение, "--- заявил один.

"--*Я бы даже сказал, дискуссионное.

"--*Подождите, "--- засмеялся я.
"--- Здесь есть тотем Сомнения?

"--*Вряд ли кто-то может сказать наверняка, "--- сказал Король-жрец.
"--- Но есть предположение, что он стоит\ldotst

"--*Стоял, "--- хором поправили его жрецы.

"--*Да, благодарю.
Есть предположение, что он стоял у западного входа в храм.

"--*Я не уверен, что это предположение доказуемо, "--- заявил Тхалас.

"--*Возможно, имеют место иллюзии восприятия, "--- ввернул Трукхвал.

"--*Может, хватит? "--- рявкнул Хитрам-лехэ.

Жрецы переглянулись и расхохотались.

"--*Извини, Ликхмас, "--- утирая слёзы, проговорил Король-жрец.
"--- Но это такая весёлая вещь.
Традиция пришла откуда-то из Ихслантхара, ну и мы решили поставить такой же у нас.
Когда мы пытались объяснить плотнику, что нужно, он решил, что Храм в полном составе спятил.

"--*Я не поддерживаю идею подобных бессмыслиц, но плотнику надо сделать выговор за халтуру, "--- заявил Хитрам-лехэ.
"--- Где это видано, чтобы новые столбы валились от одной бури?

"--*Не надо выговоров, "--- поднял руку Король-жрец.
"--- Просто попросите его починить.
Для ремесленника это достаточно понятный намёк.

\section{Рабы закона}

"--*Они хотят полностью подчинить сели их законам, "--- сказала Анкарьяль.
"--- Расчёт прост.
Раб закона "--- раб того, кто найдёт в законе прорехи.

\section{Вирус (идея)}

Клетки, поражённые вирусом, часто ведут себя так.
Вирус заставляет их делиться, выживать любой ценой, прорастать окружающие ткани.
И в результате стремление к выживанию оборачивается катастрофой для всех "--- если раковую опухоль не удалить, она убьёт организм.

Так где же она "--- грань между поведением поражённых вирусом клеток и нормальным желанием жить, иметь детей и защищать родных\ldotsq

\section{Запасной Король}

"--*Кое в чём Картель просчитался, "--- сказала Анкарьяль.
"--- Они не ожидали, что сели используют их собственную тактику "--- <<держи ресурсы в мусорной куче>>.
Да и я, честно говоря, впервые вижу, чтобы обычный крестьянин полноценно заменил Короля-жреца.
В прочих мирах это происходит, да, но в основном стихийно "--- появляется талантливый человек, который учится на своих ошибках.
Здесь же имеет место направленное обучение.
Человека учили быть запасным!

\section{Необычное имя}

"--*Двор Тхитрона, Манэ и Ликхэ ар'Люм.

"--*Благодарю, "--- кивнул я.
"--- Только одно замечание: не Ликхэ, а Лимнэ.

Жрец вгляделся в записи и охнул.

"--*Король-жрец, прошу меня простить.
Имя очень необычное.

"--*У моей кормилицы хорошая фантазия на имена, "--- улыбнулся я.
"--- Купчихи ар'Люм "--- мои сёстры.
Продолжайте, пожалуйста.

\section{Эффект Борка}

Я вдруг вспомнил храм Тхитрона, бешеную схватку возле алтаря, демона Картеля, поглощавшего страдания умирающей Чханэ\ldotst

"--*Эффект Борка, "--- сказал я вслух.
Анкарьяль запнулась.

"--*Прости?

Я рассказал о произошедшем со мной в храме.
Грейсвольд нахмурился.

"--*Стратег с околонулевой устойчивостью отклонил твою атаку?

"--*Именно отклонил! "--- подхватил я.
"--- Не отбил и не увернулся, а отклонил, даже скорее рассеял!
Ситуация по сути редчайшая, чтобы демона атаковали во время питания\ldotst
Вот кто из демонов до этого момента принимал эффект Борка всерьёз?

"--*Как минимум Картель, который иногда использует армии сапиентов против Ада, "--- сказала Анкарьяль.

"--*Иногда! "--- почти выкрикнул я.
"--- Эффект Борка использовал на Драконьей Пустоши я, даже об этом не подозревая!
С тех пор ни одного случая не было зарегистрировано!

Друзья переглянулись.

"--*Борк Песчаный Мост сделал своё открытие на пять тысяч лет позже битвы на Серпенциару, "--- продолжил я свою мысль.
"--- Каков шанс того, что мою победу обработали ретроспективно?

"--*Просто скажи, что ты предлагаешь, "--- раздражённо бросила Анкарьяль.
"--- Плюс-демоны не могут использовать эффект Борка.
Для этого нужны особые сапиенты\ldotst

"--*Например, потомки тси? "--- подсказал я.

\section{Нгвсо и ноа}

"--*Это Барабан, "--- сказал ноа.
"--- Жди, я с ним поговорю.

Из воды показались три глаза и раздвоенное щупальце.
Человек замахал руками.
Щупальце исчезло под водой и вытянуло сплетённую из водорослей сеть, полную искристых зеленых раковин.
Ноа принял ценный груз и, вытащив из бедренного мешка другую сеть, передал её существу.
Нгвсо, аккуратно обвив сеть щупальцем, неторопливо свернулся в родную стихию.

"--*Нгвсо выращивают этих моллюсков на еду, а раковины продают нам за бесценок, "--- объяснил ноа.
"--- Видимо, под водой ракушки выглядят не такими красивыми, как на воздухе, иначе нгвсо ценили бы их куда выше жемчуга.

"--*А что ты ему дал?

"--*Пирожки с вишней.
Я очищаю вишню от костей, заворачиваю в тесто и слегка обжариваю в масле, чтобы пирожки не слипались.
Нгвсо млеют от этого блюда.

"--*Я слышала, что нгвсо приносят не только ракушки, "--- полувопросительно заметила Чханэ.

Ноа пожал плечами.

"--*Мои товарищи берут у них и жемчуг, а некоторые платят нгвсо за очистку днищ кораблей.

"--*И между вами не возникает никаких разногласий?

"--*Нгвсо и ноа "--- хороший союз, "--- наш собеседник снова пожал плечами.
"--- Мы защищаем их от стрелохвостов, они предупреждают нас о вторжениях и лазутчиках.
Торговля опять же.
Ладно, путники, мне пора.

Ноа, похоже, потерял к нам всякий интерес.
Схватив посох-зонт и мешок, он неторопливо отправился в сторону рыбацкой деревни.

\section{Пёсьи головы}

"--*А на западе живут люди с пёсьими головами, "--- сказал старик-ноа.

"--*Они называются <<пылерои>>, "--- весело ввернул кто-то.

"--*Да что ты знаешь! "--- вспылил рассказчик.
"--- Пылерои "--- это двуногие собаки, которые живут, как звери, в пустыне.
А на западе "--- люди с пёсьими головами.
Они умеют говорить, едят варёное мясо, пишут мудрёные знаки и умеют их читать.
Разве собака умеет читать?

\section{Последний фрагмент книги}

"--*Вот книга, котору ты просил, "--- сказал ноа и протянул мне том.

Я пролистал книгу и усмехнулся.
Смешного в этом мало, но ноа тоже хранили фрагмент "--- начало, немного из середины и самые последние записи.
Впрочем, с этой книгой моя картина произошедшего почти сложилась.
Почти.

Я поблагодарил человека и вернул ему книгу, пообещав зайти позже с бумагой и пером.

\section{Языки кормилицы}

"--*Лисёнок, "--- обратилась ко мне Кхотлам, "--- я понимаю, что сейчас не лучшее время для тебя, но мне нужно выучить язык демонов, против которых мы сражаемся.

"--*Многие из них переговариваются неслышно для вас или используют код, "--- заметил я.

"--*И всё-таки.
Ты упомянул сектум-лингва.
Не мог бы ты о нём рассказать\ldotsq

\section{Разведка дипломата}

"--*А ещё я обнаружила пятерых агентов Красного Картеля, "--- доложила кормилица.

"--*С чего ты решила, что это агенты? "--- высокомерно скривилась Анкарьяль.

"--*Они говорили на сектум-лингва.
И да, "--- опережая вопрос, продолжила Кхотлам, "--- я теперь умею отличать классический сектум-лингва от языка ноа.

Анкарьяль со всё возрастающим уважением смотрела на кормилицу.

"--*Скажи их хасетрасем, и я лично рекомендую тебя к оцифровке, Кхотлам ар'Люм.

"--*Не стоит, Анкарьяль Кровавый Шторм, "--- улыбнулась купец.
"--- Однако мне есть что сказать помимо хасетрасем\ldotst

\razd

Кормилица действительно сообщила несколько важных деталей.
Презрение к сапиентам в очередной раз сыграло с Картелем злую шутку "--- они говорили на сектум-лингва открыто, проверив местность лишь на присутствие демонов.
Некоторые термины были сказаны на языке Эй, таблица B44, но Кхотлам повторила их с механической точностью.

Во-первых, Картель оказался в курсе нашей численности "--- <<их здесь трое или пятеро, не больше>>.
Во-вторых, враги знали о присутствии <<Падальщика>> (меня) "--- постарался один из визоров, собравший полную картину буквально из пылинок информации.
В-третьих, мы получили имена некоторых присутствующих "--- Гимел (биолог Гимел Кадмиевый Зелёный), Малингве (стратег Малингве Пустая Карта) и Монстр (старый позывной интерфектора Мост Ликующая Плазма, с высокой вероятностью намекающий на личность говорящего "--- он состоял в Манипуле Смеха, одном из самых жестоких подразделений армии Картеля).

"--*Кроме этого, они сказали вот это, "--- закончила Кхотлам и проворковала фразу на языке Чёрной Скалы:
"--- <<Это солнце чересчур яркое для меня, любовь моя.
Я хочу домой>>.

Я перевёл.
Кхотлам наморщила лоб.

"--*Да, это были мужчина и женщина, они действительно выглядели как любовники.
Хотя нет, скорее как очень близкие родственники.

Так мы узнали о присутствии легатов Фуси Абрикосовый Посох и Нуива Пустая Тыква.

Анкарьяль, узнав о присутствии на захолустном Тра-Ренкхале как минимум двух интерфекторов Манипулы Смеха, извинилась и покинула палатку.
Грейсвольда начало жестоко рвать и лихорадить уже ночью, я же заранее принял песочник ползучий и отделался небольшим головокружением.
Реакция Стлока плюс сбои в работе демонов "--- я вообще сомневаюсь, что кому-то из агентов Ада за последний телльн приходилось переживать такое запредельное напряжение.

\section{Мотивы Хэмингуэя (кусок)}

Да, я уже чувствовал это когда-то давно.
Драконья Пустошь, гавань славного свободного города Фриза.
Южное солнце, пальмы, крикливые торговцы, развесёлые караваны цыган, обольстительные красотки на каждой улице.
Корабль-трибот, мерно покачивающийся на волнах.
Фрукты, креветки и проститутки приедались уже через неделю, но это не могло надоесть "--- ощущение, что морской ветер ласково гладит тебя по голове, словно так рано ушедший отец, который (я в этом не сомневался) любил меня больше жизни.

\section{Рыбак}

"--*Знаешь, Король-жрец, чем меня восхищает жизнь? "--- из кармана старика появился крохотный орех.
"--- Взгляни на него.
Он кажется таким же мёртвым, как камень.
Даже если с небес упадут снега или огонь, даже если море затопит сушу, его суть не изменится ни на волос.
Он будет подобен камню.
Но стоит растаять снегам, стоит погаснуть огню, стоит морю обнажить хоть маленький пятачок живой почвы "--- орех даст росток и вырастет в куст, словно ничего не случилось.
Верю ли я, Король-жрец, что старый мир победит?
Я это знаю!
Но победит не тот старый мир, о котором ты говоришь, а другой "--- который не знал ни Безумного, ни дельфинов, ни иных думающих из плоти и костей.
Так было и будет.

\section{Серебряный амулет}

"--*Мой олень, Серебряный.
Он всегда знал верную дорогу.
Пусть его дух поможет тебе.

Митхэ протянула мне амулет, грубо вырезанный из лобной кости оленя.

\section{Хрусталь}

"--*Когда-то давно я посетила малый храм Тхартавирта "--- Хритра.
Сейчас его уже нет "--- было большое землетрясение, и здание сложилось, как лист пергамента.
Тогда же\ldotst я не знаю на Короне ничего прекраснее того храма.

На башни подниматься чужим было строго запрещено, но я уговорила молодого жреца.
Он получил свои любимые восточные финики, а я "--- ключ от башни.

Окна в башнях были закрыты толстым хрусталём.
Делал хрустальные листы знаменитый мастер, и я не могу передать, насколько они гладкие и чудесные.
Если их не освещали факелы или лампы "--- их не существовало.

Я поднялась почти на самый верх и присела на широкий подоконник.
Был вечер, и молодой жрец ещё не зажёг факелы в башне.
Я отлично помню чувство, которое меня охватило.

"--*Что это было за чувство? "--- спросил я.

Митхэ пожевала губу.

"--*Чувство, что я сижу на головокружительной высоте и ничто не отделяет меня от пропасти.
При этом никакого страха не было.
Удивительно, правда?
Храм стоял на Зелёной скале, Тхартхаахитр лежал передо мной, как на ладони, даже Трёхэтажный казался пирожным.
На такой высоте должен был быть жуткий ветер, но хрусталь глушил все звуки.
Вот такое странное чувство "--- чувство высоты и застывшего перед тобой мира.
Я вспомнила это, когда потеряла Атриса.
Когда к лесным духам уходит единственный человек, способный тебя понять, ты начинаешь смотреть на этот мир через толстый, в десятую пяди толщиной хрусталь.
Перед тобой разворачиваются сражения, льётся кровь, люди тебе что-то кричат\ldotst а ты ничего не слышишь, ничего не чувствуешь и не боишься.

\section{Два касания}

"--*Я всё равно пройду, "--- заявил посланец.

"--*Попробуй, "--- предложила Кхохо.

Чужак, Кхохо и Ситрис схватились за оружие одновременно.
Три сложных росчерка "--- и воины тхитронского Храма попятились.
Ситрис хмыкнул, шмыгнул носом, клинки расчертили воздух "--- и снова два шага назад.
Сабля Кхохо нервно подрагивала;
Ситриса, похоже, удерживала на месте только необходимость помогать подруге.

Вдруг посланец замер и тупо уставился перед собой;
из кустов вышла Анкарьяль и лениво снесла ему голову.

"--*Живые? "--- осведомилась она.

Воины кивнули.

"--*Поединков с демонами не устраивать, "--- распорядилась Анкарьяль.
"--- Лучше толпой, а ещё лучше "--- яд и стрелы.

"--*А если вот так, бежать, что ли? "--- возмущённо развела руками Кхохо.

"--*Если жить хочешь, "--- лаконично ответила демоница.
"--- Воины, наверное?
Какой Храм?

"--*Тхитрон, "--- буркнул Ситрис.
"--- Что ты с ним сделала?
Он словно оцепенел.

"--*Дуэль хоргетов, "--- пояснила Анкарьяль.
"--- Хоргет умер "--- тело осталось без наездника.

Анкарьяль окинула их оценивающим взглядом.

"--*Зайдёте в палатку командования ближе к вечеру, дам вам людей под начало.
Не каждый переживёт два касания в схватке с демоном.

Анкарьяль ушла.
Ситрис и Кхохо долго молчали, не глядя друг на друга.
Наконец Ситрис нарушил молчание:

"--*Мы тогда послали его на верную смерть.
Ликхмаса\ldotst

"--*Заткнись, "--- огрызнулась Кхохо.
"--- Без тебя поняла.

"--*Я бы сейчас выпил.

"--*Мы в походе.

Ситрис молчал.

"--*Всё, хватит, "--- буркнула Кхохо.
"--- Прошлое "--- в прошлом.

\section{Триумфальное шествие}

Исход был похож на триумфальное шествие.
Люди пели песни и смеялись, несмотря на вспыхивающие то тут, то там стычки с диверсантами.
Мёртвых никто не хоронил и не оплакивал.

Вот какой-то воин властно поднял женщину, севшую оплакать сестру:

"--*Пойдём!
Ты скоро сама с ней встретишься.

"--*Если я её не оплачу, сестра не достигнет пристанища! "--- запротестовала женщина.

"--*Путь к пристанищу всегда был прост и прям "--- смерть, "--- веско сказал воин.
"--- Она уже там, и ей нужен лишь покой.
А мне нужны голоса для дорожной песни.

Воин запел.
Его собеседница, выхватив окровавленную фалангу, подхватила хриплым яростным сопрано, от которого меня по коже продрал мороз.
Они двинулись по дороге, держась за руки.

В этот момент я отстранённо понял, что я им уже не нужен.
Им не нужен Король-жрец.
Они сами прекрасно знают, что делать, и делают это.

\section{Нежности Кхохо}

"--*Я ему говорила, чтобы он уходил, "--- бросила Кхохо.
"--- Я давно уже пропащая.

"--*Я с тобой, "--- сказал Ситрис.
"--- Я знаю, что для убедительности следует дать тебе затрещину, по-другому ты не понимаешь.
Но я не хочу.
Сейчас, может быть, в последний наш день, я хотел бы только целовать тебя и говорить всякие нежности.

"--*О том, как ты меня любишь?
А то я не знаю, "--- грустно пробормотала Кхохо.

"--*Не знаешь, "--- сказал Ситрис.
"--- Насколько я тебя люблю, знаю только я.

"--*Печальная судьба "--- любить акулу в человеческом обличье, "--- заметила Кхохо.

"--*По-моему, поздно об этом сожалеть.
Так что продолжай быть собой.
Можешь меня поколотить.

"--*Я впервые в жизни не хочу тебя бить, "--- сказала Кхохо.
"--- Пойдём лучше на берег и посидим в воде.
Ты будешь меня целовать и говорить всякие нежности.

"--*Ты мне уступила? "--- изумился Ситрис.
"--- Не иначе как сам Удивлённый Лю завтра прилетит.

"--*Не знаю, кто там завтра прилетит, но я точно умру, "--- уверенно сказала Кхохо.
"--- Я тебя обещала ещё убить, помнишь?
И убью.
Правда, не совсем понимаю, как это у меня получится.
Ладно, время покажет.
Пойдём, нам ещё поспать надо, а сон перед смертью "--- пустая трата времени.

\section{Сокровище Высших}

Огромные металлические ворота были оплавлены.

"--*Сколько металла\ldotst "--- завистливо ухнул Грейсвольд.

"--*Мы стережём это место, "--- сказал Высший.
"--- Здесь спрятана мудрость.

Я подошёл и погладил зеркально чистый оплавленный край ворот.

"--*Техногенная катастрофа, "--- резюмировал я.
"--- Судьба <<Стального Дракона>> их ничему не научила.
Грейс\ldotst

"--*Ты будешь смеяться, но здесь нет ни одного работоспособного устройства, "--- как-то чересчур весело откликнулся Грейсвольд, предупреждая мой вопрос.
"--- Взрыв уничтожил все наноструктуры.

"--*Они охраняли пустоту, "--- заключил я.
"--- Давай скажем бедолагам, что мы "--- те самые боги и что мы забрали всю мудрость.
А то они ещё пятьсот поколений здесь проторчат.

"--*Дождёмся культурологов, пусть они это поаккуратнее сделают, "--- предложил Грейс и тронул свесившуюся с потолка трансмиссию.
Мёртвый, слегка покорёженный механизм печально закачался, но скрипа мы так и не услышали.

\section{Трава Тумана}

"--*Когда-то давно я очень любил курить траву тумана, "--- задумчиво сказал мужчина.
"--- Плохая привычка, знаю.
Те, кто имеет пристрастие к этому цветку, не живут долго.
Я знал всё это, знал, чем может это закончиться "--- траву курил мой кормилец и его предки.
И всё равно однажды так случилось, что я набил этими ароматными листьями деревянную трубку и сделал первый глоток дыма.
Вскоре я начал кашлять и часто болеть, ноги и руки мои ослабели, но привычка овладела мной "--- я не мог провести без травы тумана и дня.
Меня стали избегать женщины, от крепкого запаха моей одежды фыркали даже ездовые олени.
И как-то был годовой поход, во время которого я не мог достать своё любимое зелье.
За это время я вернул свою прежнюю силу, моё дыхание стало чистым.
Но мысль о траве тумана преследовала меня постоянно.
И когда я вернулся в родной храм, то первым делом нашёл среди вещей старую трубку с расколовшимся концом и пакет с сухой травой.
Набивая трубку, я делал это нарочито медленно.
Меня терзали сомнения.
Должен ли я?
Готов ли я пожертвовать ласками женщин и силой своего тела ради этого?
Когда огонь коснулся коричневых с золотым отливом листьев и я сделал первый вдох дыма, то впервые за долгие годы ощутил его вкус.
Это было открытием "--- курение давно стало для меня повседневностью, и я даже не знал, насколько вкусна трава тумана.
Следующим открытием стал запах.
Я почувствовал в горьковато-пряном дыму нотки, которых не различал ранее.
Я раз за разом вдыхал дым, и вкус и запах его слабели, уступая место знакомому чувству опьянения.
И вместе с этим пришло понимание.
Я ли выбрал жребий, который привязал меня к этой привычке?
Или, может быть, этот жребий выбрал мой кормилец или его предки?
Наверное, есть вещи, которые мы не можем изменить.
В моих силах лишь выбирать, что важнее "--- сила в ногах, женское внимание или чувство умиротворения, которое приходило с белым как снег травяным туманом\ldotst

"--*Чушь городишь, "--- поморщился крестьянин, сидящий по другую сторону костра.
"--- Ты это выбрал сам и пытаешься обвинить в своём выборе мироздание.

"--*Я не отделён от мироздания, я его часть, "--- развёл руками мужчина.
"--- Я "--- потомок моих предков.
Было бы наивно полагать, что я не зависим от окружающего меня мира.

"--*Мы чересчур мало знаем, "--- поддержал его кожевник Эрликх.
"--- Древние лечили даже печали, и средства у них были куда сильнее травы тумана и молитвенных маков.
Жрец, который учил меня читать и писать, говорил, что печали "--- это болезнь.

"--*Тхэай, жрецы много чего говорят, "--- бросила женщина с закутанным в одеяло ребёнком, по говору крестьянка с севера.
"--- У них всё болезни.
Трусость "--- болезнь, печаль "--- болезнь, влюблённость "--- болезнь.
Только ни один жрец не умеет от них избавить, да и кутрапов казнят, а не лечат.
Какой толк в знании, если его нельзя применить?

Я промолчал, внезапно ощущая прилив гордости за этот народ, к которому теперь "--- самую каплю, разумеется "--- принадлежал и я, уроженец Драконьей Пустоши.

\section{Винт}

Я задумчиво смотрел на медленно вращающийся винт, заключённый в хрустальную трубу. В моей голове зрел вопрос\ldotst

"--*Учитель Трукхвал.

"--*Да, Ликхмас? "--- библиотекарь удивлённо поднял голову.
"--- Ты с операции?
Всё хорошо?

"--*Да, она выжила, слава лесным духам.
У меня к тебе вопрос.

"--*Ооо, хай-хай.
Слушаю.

"--*У нас на балконе в системе подачи воздуха стоит винт.
Возможно ли с его помощью летать, отталкиваясь от воздуха?

Учитель заулыбался.

"--*Хай, вон что тебя заинтересовало.
Да, можно, вполне.
Садись, расскажу, "--- учитель с некоторым облегчением отложил в сторону книгу, которую собирался переписывать.

Я сел.
Трукхвал тем временем выудил две чашки и зачерпнул из котелка тёплой воды.

"--*Ммм, винт.
Да.
Вообще полёт "--- не такая уж хитрая вещь.
Мой учитель строил машинки с теми самыми винтами "--- и они летали.
Хоть и есть у сели поговорка, что рыбы не летают, но однажды он ради смеха привязал к одной из них карпа, перед этим поспорив с половиной Храма, что заставит рыбу летать, "--- учитель скрипуче засмеялся, обнажив желтоватые кривые зубы.
"--- Весь народ потом сбежался глядеть на жрецов, которые стояли на площади на головах\ldotst
Хай, отвлёкся я.
Это были маленькие машинки, размером с птицу согхо.

"--*А большие?

"--*С большими проблема.
При увеличении размеров вес растёт, и дерево не выдерживает.
Нужен кукхватр, да где его столько взять?
Пришлось бы пустить в плавку клинки половины воинов сели, чтобы мог летать один человек!

Трукхвал неожиданно разговорился.
Видимо, летающие машинки бередили его душу уже очень давно.
Вскоре он достал кусок пергамента, сурьмяной мелок и начал чертить сложные схемы.
Я по мере сил пытался вникнуть.
Вдруг, прервав рассказ, Трукхвал хлопнул себя по лбу:

"--*Хай, да что я угощаю тебя голыми травами да нравоучениями.
Держи-ка печенье, Ликхэ принесла, ты после операции голодный как ягуар\ldotst

\section{Болотная лихорадка}

Кхатрим подозвал меня.
Я склонился над телом.

"--*Смотри, Ликхмас, "--- тихо сказал жрец.
"--- Ты должен запомнить это на всю жизнь.

Кхатрим вынул нож и одним ударом вскрыл ребёнку грудную клетку.
Сердце и лёгкие были странного цвета "--- оранжево-красные, слегка опалесцирующие в рассеянном свете солнца.
В нос ударил гнилостный запах.

"--*Это не кровь, "--- констатировал я.

"--*Правильно, "--- кивнул Кхатрим.
"--- Кровь здесь тоже есть, но её не так много.
Согласно записям предков, это города мелких существ.
Настолько мелких, что они похожи на налёт.
Один жрец, по слухам, изобрёл устройство, позволяющее их видеть, состоящее из линз.
Однако на его город напали.
Жрец погиб, его записи сожжены, и мы не знаем доподлинно, правдивы ли слухи.

"--*Сколько их здесь?

"--*Я не скажу тебе даже приблизительно.
Не тысячи и не миллионы, много больше.

"--*Взрослые могут заболеть?

"--*Раненые и старики.

"--*Как убить этих существ?

"--*Их нельзя раздавить, нельзя покалечить, наши инструменты для этого чересчур грубы.
Их можно только отравить.

Кхатрим вынул из кармана робы бутылочки.

"--*Винная эссенция, экстракт мха-ползуна и хлебная плесень.
Винной эссенцией нужно протирать все инструменты и руки, когда имеешь дело с болотной лихорадкой.
Экстракты мха и плесени помогают при приёме внутрь.

"--*Можно ли вводить их сразу в лёгкие и сердце?

"--*Мы пробовали.
Больные умирают ещё быстрее.
Только так.

Кхатрим укрыл тело простынёй и встал.

"--*Раньше всех детей и стариков, если появлялись больные, отводили в храмовые рощи.
Благородный баньян, может быть, ты видел такие деревья?

"--*Только слышал.

"--*Воздух в этих рощах целебный, существа гибнут там очень быстро.
К сожалению, возле Тхитрона нет ни одной.

"--*Храмовая роща помогла бы этому ребёнку?

"--*Этому "--- нет, "--- покачал головой Кхатрим.
"--- Если лихорадка зашла далеко, убитые существа начинают гнить внутри, и при лечении отказывают почки.
Дети начинают мочиться кровью, позже всё равно умирают.
Я научу тебя определять безнадёжных, им нужно дать Чёрного Сана и сжечь тела.
Кровавый плащ меняй каждый час, робу утром и вечером.
Перед тем, как ехать в Тхитрон, искупаешься вон там, растворы для втирания в тело приготовят.
Ты бы, конечно, это знал, если бы больше внимания уделял чтению.

Я поперхнулся.

"--*Книга в библиотеке.
Отдел 1, полка 4, <<Ползучая болезнь>>, Тепло-Полуночного-Костра.
Ты её хотя бы открывал?

Я насупился.
Книга показалась мне скучной с самой первой страницы.

"--*Как я узнал? "--- Кхатрим без улыбки смотрел на меня.
"--- Сама книга тонкая, в тридцать страниц.
Остальная часть "--- коротенькое обещание следовать написанному в точности и подписи прочитавших её жрецов.

Я не страдал склонностью к чувству стыда, но тут у меня запылало лицо.
У Кхатрима определённо был стиль "--- он рассказывал всё, что требовалось знать, подробно отвечал на любые вопросы, а после этого стыдил за неприлежание так, что покраснел бы сам Удивлённый Лю "--- хлёстко и коротко.

"--*Ликхмас, я хочу, чтобы ты понял.
<<Ползучая болезнь>> "--- одна из важнейших книг по врачеванию.
Все ритуалы, все правила поведения при эпидемиях прописаны на этих тридцати страницах.
Словом, как вернёмся, тебя ждёт экзамен.
Или несколько экзаменов, пока я не удостоверюсь, что ты всё усвоил.
А теперь пойдём, нас ждёт тяжёлая ночь.

\section{Смена пола}

"--*Расскажи про смену пола, "--- попросил я.

"--*Хай, "--- задумался учитель. "--- Явление это очень, очень редкое.
Но условия его известны давно.

Учитель встал на ноги, прихрамывая, вышел в центр комнаты и театрально развернулся ко мне.

"--*Представь, что группа взрослых мужчин оказалась отрезанной от внешнего мира.
Скажем, они плыли на корабле и попали на необитаемый остров.
Они вынуждены жить там длительное время.
Понятное дело, что размножаться они не могут, так как женщин среди них нет.
И вот тут-то включается таинственный природный механизм "--- несколько самых сильных мужчин начинают превращаться в женщин, чтобы племя могло воспроизводиться и существовать дальше.

"--*Наверное, они очень страдают при этом, "--- заметил я.

"--*О да, "--- откликнулся старик.
"--- Их кости и плоть начинают гореть из-за быстрого роста, их охватывает лихорадка.
Эти люди нуждаются в момент превращения в особом уходе.
Может быть, именно поэтому природа отдала роль превращающихся самым сильным из группы.
То же самое, кстати, происходит с мужчинами, оказавшимися в изоляции.
Иногда такие мужчины превращаются в женщин и сразу беременеют "--- это называется партеногенезом.

"--*А Чханэ?

"--*Твоя подруга, насколько я понял, ударилась головой в детстве.
Вероятно, её мозг повредился и механизм запустился сам собой.

"--*Она сказала, что бесплодна, "--- тихо сказал я.

Трукхвал с нежностью посмотрел на меня.
Такая откровенность растрогала сердце старика.

"--*Учитель Трукхвал, ты столько всего знаешь.
Можно ли её вылечить от бесплодия?

Трукхвал виновато развёл руками.

"--*Извини, Ликхмас-тари.
Сколько бы я ни учился, пойду по жилам джунглей дураком.

\section{Плачущий ягуар}

"--*Лис, скажи правду, почему ты меня спас?

"--*Какая тебе разница? "--- поморщился я.
"--- Спас и спас.
Радуйся.

"--*А если бы я была некрасивой и не понравилась тебе, выдал бы ты меня жрецам?

Я задумался.

"--*Наверное, нет.

"--*Почему?

"--*А ты, Чханэ?
Ты хотела меня убить, как увидела.

"--*Да, "--- растерялась Чханэ.
"--- Это нормально, разве нет?

"--*Видимо, нет.
Твой вопрос был о том же.

Чханэ замолчала.
Похоже, она уже сама была не рада, что подняла эту тему.

"--*Я\ldotst мне нужны были доспехи\ldotst "--- начала она.

"--*Отлично, "--- поморщился я.
"--- А если бы ты мне не понравилась, я был бы мёртв.
Обязательно поднесу пару конфет Хри-соблазнителю.
Или тебя остановило что-то другое?

Чханэ отвернулась и закуталась в одеяло.

"--*Так почему ты мне поверила?
Почему не попыталась вонзить нож в спину, как тому жрецу?

"--*Давай больше не будем об этом.

"--*Это важно, "--- настаивал я.
"--- Мы спим бок о бок.

Чханэ легла на спину и сдула с лица непослушную <<рыбку>>.

"--*Я расскажу тебе историю.

"--*У тебя на всё есть истории, "--- отмахнулся я.

"--*Эта тебе обязательно понравится, "--- заверила Чханэ.
"--- Так вот.
Однажды к воротам Тхаммитра подошёл ягуар.
Огромный полуторагодовалый кот.
Стража уже собиралась застрелить его, но один из дозорных заметил, что на его лапе висел верёвочный капкан.

"--*Ловушка? "--- возмутился я.
"--- Это бесчестная охота.
Кто промышлял таким?

"--*Не мы, "--- заверила Чханэ.
"--- Возможно, это были звероловы с юга, мы не знаем.
Стражники позвали охотников.
Те были просто ошарашены, что кто-то решился на такую подлость.
Ягуар полулежал, готовясь в любой момент дать дёру, рычал на всех и всё же не уходил.
Наконец один из охотников сказал: <<Ягуар не случайно оказался у ворот.
Ему пришлось пройти с капканом по открытому нагорью, которое обычно избегают, и одним лесным духам ведомо, сколько он прошёл по джунглям>>.

Охотник решился и осторожно подошёл к зверю.
И в тот момент произошло невероятное "--- ягуар заплакал и перевернулся на спину.
Капкан причинял ему невыносимую боль.

"--*Что они с ним сделали?

"--*Охотники взяли ягуара за лапки и принесли его в город.
Он не сопротивлялся.
Позвали жреца, и тот начал снимать капкан.
Кот плакал и повизгивал, но не делал ни одной попытки ударить или укусить тех, кто его держал.
А охотники, на счету которых была не одна шкурка, стояли и плакали, глядя на него.
Я была тогда ещё маленькой, но хорошо запомнила их лица.

"--*Они спасли его?

"--*Да.
Жрец снял ловушку, обмазал лапку целебными снадобьями.
А потом ягуар целых десять дней жил в полуразрушенной хижине, в которую его принесли.
Его не связали, не заперли.
И знаешь, что самое удивительное?
Год выдался плохим на мясо, зверьё ушло на север.
А всё равно многие приносили часть добычи ягуару, чтобы он ел.
И ни у кого даже мысли не возникло, что с него можно снять шкуру.
Охота охотой, но игру не по правилам поощрять не следует.
Потом, когда кот окреп, он просто вышел из хижины и прошёл через ворота к джунглям.

"--*Поучительная история, "--- заметил я.

"--*Я ответила на твой вопрос?

"--*И на свой тоже, "--- кивнул я.
"--- А что произошло потом, когда капкан был снят?

"--*Потом я влюбилась, "--- просто ответила Чханэ.
"--- Или ты про ягуара?

Я пододвинулся к ней и нежно поцеловал её в губы.
Мои ноздри обжёг запах какао, идущий от её волос.
Огненные глаза превратились в полутьме в отполированные кусочки коричневого шпата.

"--*Ну-ка, Змейка, хочешь сладких конфет?

Девушка улыбалась.

"--*Хочу.
А какие конфеты у нас сегодня?

"--*Да те же, что и вчера, "--- тихо рассмеялся я и ткнулся носом ей в ухо.
"--- Надеюсь, они тебе не надоели?

"--*Ну что ты, "--- игриво-укоризненно шепнула Чханэ, развязывая верёвочки на моей рубахе, и, не удержавшись, цапнула меня острыми зубками за плечо.

\section{Математика}

\textit{(Анкарьяль учит Тхартху рисованию)}

"--*Нар, послушай, "--- Тхартху задумчиво смотрела на листок бумаги.
"--- Мне кажется, что вот эти квадратики равны по площади.

"--*Что, Тхартху? "--- Анкарьяль склонилась над девушкой.

"--*Вот эти, "--- Тхартху ткнула пальчиком в листок.
"--- Я нарисовала несколько треугольников с\ldotst прямым углом и пририсовала к сторонам квадратики.
И площадь этого всегда равна площади двух этих.

"--*Мужичьё, идите-ка сюда, "--- в голосе Анкарьяль я различил лёгкий шок.
Мы с Грейсвольдом переглянулись и подползли поближе.

"--*Что тут у нас? "--- Грейсвольд ласково потёрся носом о щеку Тхартху и заглянул в её записи.

"--*Похоже, она только что переоткрыла теорему о квадратах сторон треугольника на стандартной плоскости.

"--*Вот, "--- Тхартху снова показала на свой чертёж.
"--- Треугольник и вот эти квадратики.
Их площадь.
Я нарисовала несколько разных треугольников "--- результат один и тот же.

Мы с Грейсвольдом переглянулись и засмеялись.
Тхартху бросила на нас обиженный взгляд.

"--*Ну что не так?

"--*Не-не, Птичка, всё так.
Как ты поняла, что они равны?

Тхартху поджала губы.

"--*Я выросла в хуторе.
Мерить землю поручают детям с ранних лет.
Я могу отличить, что больше, а что меньше.

"--*Удивительно, "--- Анкарьяль всё ещё завороженно разглядывала рисунок девушки.

"--*Ну-ка, дай мне листок, "--- я аккуратно выхватил его из рук Тхартху вместе с карандашом и наскоро нарисовал две фигуры "--- квадрат и круг.
"--- Какая из фигур больше по площади?

"--*Вот эта, "--- девушка указала на квадрат.

"--*На сколько? "--- хитро оскалился Грейс.

Тхартху нахмурилась.

"--*На чечевичное зерно, не больше.

"--*Вы тоже поймали дубинку головой\footnote
{Калька фразеологизма, распространённого на Преисподней.
Значение "--- сильно удивиться чему-либо.
У одного из народов была игра, смысл которой заключался в перебрасывании игроками тяжёлой дубинки.
Тот, кто зазевался, мог запросто получить травму. \authornote}? "--- поинтересовалась Анкарьяль.

"--*Ага, "--- хором ответили мы с Грейсом.

Тхартху засмеялась:

"--*Что-что поймали?
Чем?

"--*Дай-ка мне, "--- Анкарьяль вырвала листок у меня из рук.
Карандаш я передал ей сам.
"--- Тхартху, смотри.

Демоница изобразила на бумаге квадрат, а в него вписала квадрат поменьше.
Рядом изобразила ещё один квадрат и расчертила его.
Не успела Анкарьяль закончить второй чертёж, как Тхартху ахнула:

"--*Я поняла!
Поняла!

"--*Что поняла?

"--*Вот эти треугольники, и да, эти квадратики\ldotse
Ну это\ldotst да.

"--*Ну это, да, "--- повторил Грейс.
Анкарьяль отвесила ему звонкую оплеуху.

"--*Главное, что она поняла, каменная башка.
Вот, Птичка.
Это называется <<доказательство>>.

"--*А оплеуха "--- <<обоснование>>, "--- ввернул я и тоже отхватил затрещину.

"--*Хай, не лупи моего мужчину, "--- возмутилась Чханэ, оторвавшись от котелка с едой.
Демоница бросила на неё презрительный взгляд.
Чханэ ответила тем же, агрессивно повертев в руках черпак.

"--*Так, всё, хватит, "--- тут же вмешался я.
Отношения Анкарьяль и Чханэ накалялись с каждым днём.
"--- Чханэ, это она по-дружески.

"--*Я ей руки оторву за такое <<по-дружески>>, "--- огрызнулась девушка.
"--- При мне тебя никто бить не будет.

Анкарьяль посмотрела сначала на меня, потом на мрачно молчащую девушку\ldotst и благородно промолчала.
Я внутренне вздохнул.

"--*Слушай, Нар, "--- Грейсвольд смущённо поковырялся в траве и, поморщившись, потёр ухо.
"--- Тяжёлая у тебя рука.
Если у неё так хорошо с площадями, может, ты ей посложнее что объяснишь?
Интегралы там\ldotst

"--*Грейс, у неё с умножением туго, а ты про интегралы, "--- покачал я головой.

"--*Да, плохая идея, "--- проговорила Анкарьяль.
"--- Я как-то пыталась объяснить кое-кому дифференцирование.
Не вышло.
Это нужно с детства осваивать.

"--*А что такое <<интегралы>>? "--- аккуратно поинтересовалась Тхартху, с трудом выговорив слово языка Эй.

Её непосредственность была настолько неподдельной, что мы засмеялись во весь голос втроём.
Анкарьяль прикрыла лицо рукой:

"--*Земля и небо, мои дарители\ldotst

"--*Короче, Нар, у тебя нет выхода, "--- сквозь слёзы пробурчал Грейс.
"--- Давай интегралы.
А там, глядишь, и до физики пространств с градиентом мерности недалеко.

\section {Величие Древних}

"--*Эх, если бы у меня была ваша сила\ldotst "--- мечтательно потянулась Чханэ и невзначай пощупала меня.

"--*А что бы ты сделала, обладай ты нашей силой? "--- спросил я подругу.

Чханэ сладко зевнула.

"--*Я бы лечила людей\ldotst Учила бы детей\ldotst Да много чего можно придумать.
Те же плохие годы, с ними можно что-то сделать.

"--*Да я бы не сказал, что сейчас всё так плохо с болезнями, "--- заметил я.

"--*Конечно плохо! "--- Чханэ посмотрела на меня, как на дурака.
"--- У нас в позапрошлом году троих детей съела лихорадка.
Четверо мужчин умерли от укуса смертожара.
По-твоему, этого мало?

Я улыбнулся.

"--*Конечно, нет.
Просто я читал записи о Древней Земле.

"--*Древняя Земля?
Что это?

"--*Прародина человечества.

"--*Хай, жрецы называют её Тхидэ.

"--*Нет.
Тси-Ди "--- ваша прародина "--- тоже развитая цивилизация, но она не первая.
Это было раньше, намного раньше.

"--*Ещё раньше? "--- открыла рот Чханэ.

"--*Есть данные, что тогда умирала половина детей, "--- перешёл я к сути.
"--- Были страшные болезни, которые заставляли людей гнить всю жизнь, и заразиться ими можно было во время соития.
А были не менее страшные, которые передавались по воздуху.
От них за декаду вымирали целые города.

Чханэ не ответила.

"--*Мужчины могли умереть от простой царапины в те времена, а женщин часто уносили роды.

"--*Да как такое может быть? "--- возмущённо воскликнула Чханэ.
"--- Это звучит, как какой-то кошмарный сон.
Безумный и вполовину не так жесток\ldotst

"--*\ldots как природа, "--- закончил я.

"--*Почему же ничего этого нет? "--- тихо спросила девушка.

"--*Первые люди.
Жители той самой Древней Земли.
Они уничтожили возбудителей болезней, они с помощью генной инженерии избавились от генетического груза, который тащили миллионы дождей.
Одна человеческая жизнь "--- пять поколений "--- потребовалась, чтобы полностью вычистить генофонд человечества.

"--*Генофонд "--- это\ldotst

"--*Наследственность.
То, что передаётся от предков к потомкам.
Но древние сделали не только это.

"--*А что ещё?

"--*Ты знаешь, что у нас в гортани есть звукопроизводящий орган?

"--*Гортанная цитра?
Да, конечно.
Я видела лёгкие трупов.
А что с ней не так?

"--*Для чего она нужна?

Чханэ задумалась.

"--*Я не знаю.
Говорить, петь, "--- Чханэ очень похоже изобразила свист птенца согхо, и в кронах ближайших деревьев немедленно отозвались взрослые птицы.
"--- Разве нет?

"--*Модуляцией голоса люди управляли сложнейшими машинами, для таких были непригодны даже самые нежные руки.

"--*Только не говори, что у Древних не было цитры!

Я улыбнулся.
Чханэ издала губами неприличный звук и ударила руками по земле.

"--*Лесные духи.
Наверное, их голоса были плоскими и невыразительными, "--- Чханэ мяукнула оцелотом, всполошив окрестных птичек ещё больше.

"--*За то, что ты и твои соплеменницы почти до конца беременности можете радоваться жизни, спокойно работать и даже сражаться, тоже благодари Древних.
Ах да, и за почти безболезненные роды, и за отсутствие менструаций тоже.

"--*Отсутствие чего?

"--*У древних женщин из матки каждые четырнадцать дней шла кровь.

"--*Они истекали кровью? "--- ужаснулась Чханэ.
"--- Зачем?

"--*Их тело работало по-другому.

"--*Хаяй\ldotst "--- Чханэ произнесла это таким тоном, словно разочаровалась в любимом герое легенд.
"--- Нам говорили, что Древние были выше и сильнее нас\ldotst
А что ещё сделали Древние?

"--*Нам мало известно про биологические особенности людей до Эпохи богов.
Я знаю лишь, что у них были одни зубы на всю жизнь.
Если потерял "--- ходи без зубов.

"--*Грустно им было, "--- Чханэ проверила зубы пальчиком.
"--- Мне как-то выбили резец, так давно уже новый вырос\ldotst
Девка поленом стукнула за то, что я строила глазки её мужчине.
А я её головой в ослиное дерьмо макнула\ldotst прямо на его глазах, "--- Чханэ захихикала.
"--- Скажи, Лис, а правда, что у людей бывают голубые глаза?
Как небо в зените.

"--*Да, "--- улыбнулся я.
"--- Я видел даже людей с белоснежной кожей и волосами цвета соломы.

"--*Это чудесно, "--- мечтательно пробормотала Чханэ и ущипнула меня.
"--- Я бы хотела увидеть таких.
А фиолетовые глаза бывают?

"--*Бывают и фиолетовые.
На планете Запах Воды живут родственники пылероев с яркими фиолетовыми глазами.

"--*А красные?

"--*И красные.

"--*Это чудо.
У нас только зелёные и карие.

Я подтянул девушку поближе и поцеловал её в висок.

"--*Многим показалось бы чудом, если бы они узнали про оцелотовые глаза, "--- прошептал я.
Чханэ смутилась.

"--*У меня они уже позеленели.
Я вдалеке от родных мест.
Но всё же, Лис, "--- в её голосе вновь зазвучала сталь, "--- от болезней не должен умирать никто.
И я бы лечила всех.

"--*Я тебя научу, "--- пообещал я.

"--*Хорошо, "--- пробормотала девушка и уткнулась мне в шею.

Анкарьяль всё это время с отсутствующим видом ломала сухую веточку на щепьё и кидала его в огонь.
Вдруг она подала голос:

"--*Научить-то научишь.
А с Безумным что делать будем?

"--*Нар, давай потом, "--- проворчал откуда-то из темноты Грейс.
"--- У меня уже глаза слипаются.

"--*Принять решение нужно быстро.
Если нас поймают и казнят, думать будет поздно.

"--*Пусть казнят, только дай мне поспать.

"--*Я вас не узнаю, "--- проговорила Анкарьяль.
"--- Вы стали относиться к заданию, как к прогулке в парк.

"--*Разведка, сбор информации, "--- я попытался расставить все чёрточки над иероглифами.
"--- Но вначале "--- спать.

"--*Может, хватит уже? "--- внезапно подала голос Тхартху.
"--- У меня от ваших разговоров мороз по коже.
<<Строить машины>>, <<научить медицине Древних>>, <<уничтожить Безумного>>\ldotst
И это таким тоном, словно вы поесть собрались.
Вы хоть немного уважения и страха имейте!

Мы промолчали.
Грейс вздохнул "--- похоже, он уже и сам жалел, что взял с собой подругу.

"--*Чханэ, скажи им уже!

"--*Они знают, что делают\ldotst "--- начала Чханэ.

"--*А я вот не уверена!
Ведут себя, как дети, спокойно говорят о вещах, которые я или не могу представить, или представлять их просто страшно!
Да и как воевать с тем, которого не видно и не слышно?

"--*Тхартху, "--- твёрдо сказала Чханэ, "--- пусть ребячатся, мы всегда так делаем перед тяжёлым походом.
Я им верю.
Давай-ка спи.
И я буду спать.

"--*Ты веришь всем, кроме Нар, "--- вдруг выпалила Тхартху.
Анкарьяль удивлённо посмотрела на девушку.
Чханэ поджала губы.

"--*Я верю и ей, "--- медленно и чётко проговорила Чханэ.
"--- Она заносчивая ящерица, но Лис и Карп не станут доверять кому попало.

На этот раз ошарашенный взгляд Анкарьяль настиг мою подругу.

"--*И давай уже спи, Тхартху.

"--*Можешь устроиться у меня на пузе, "--- сонно предложил Грейс.

Тхартху едва слышно выругалась.

"--*Угораздило меня с тобой связаться, кудрявая панда\ldotst

"--*Пузо мягкое, "--- заметил Грейс.
"--- Аркадиу, так ведь?

"--*Да уж помягче моего, "--- вздохнул я.
Чханэ захихикала.

Тхартху демонстративно оттащила спальный мешок и устроилась рядом со мной и Чханэ.
Грейс вздохнул.

"--*Ладно, спокойной ночи всем.

Наступила тишина.
Анкарьяль по-прежнему невидящим взором смотрела в костёр, скидывая в пекло оранжевых языков одну щепочку за другой\ldotst

\section{Приманка}

"--*Что не спишь, Кар? "--- Грейс хлопнул меня по плечу.

"--*Меня кое-что беспокоит, "--- уклончиво ответил я.

"--*Случайно не осцилляции ПКВ? "--- усмехнулся Грейс.

"--*Так ты тоже их почувствовал, "--- произнёс я.

"--*Самое интересное, не я, "--- ответил Грейс и покачал своим многофункциональным браслетом.
"--- Я бы даже не заподозрил, что что-то не так.

"--*Что именно ты почувствовал? "--- спросила Анкарьяль.

Я задумался.

"--*Плюсовое искажение.
По идее, за счёт присутствия Безумных на поле должен быть перевес минуса, а тут\ldotst как на нейтральных планетах.

"--*Локализация? "--- требовательно произнесла Анкарьяль.

"--*Нужны вычисления, "--- резонно ответил я.

"--*И не просто вычисления, а дешифровка, "--- как-то злобно усмехнулся Грейс.
"--- Компьютер твёрдо уверен в том, что вокруг центра планеты имеется плюс-облако.

"--*В смысле "--- облако?
Ты имеешь в виду сеть треков?

"--*Это не треки, Нар.
Не обычный планетарный омега-фон.
Это сингулярность с неопределёнными координатами.
Поэтому я и говорю "--- облако.

"--*Это физически невозможно, "--- констатировала Анкарьяль.
"--- Дурит кто-то твой компьютер, Грейс.

"--*Похоже на приманку, "--- добавил я.
"--- Грейс, больше не трать на \textit{это} энергию.
Мы с тобой и так уже\ldotst потратились.

Грейсвольд смущённо хмыкнул.

"--*Кто бы это ни был, о нём у нас никакой информации.
Будем пока что исходить из того, что мы на Тра-Ренкхале одни.
Меня только интересует, каким образом было создано это облако?

"--*Иллюзия восприятия, "--- отмахнулась Анкарьяль.
"--- Ты, Аркадиу, с такими технологиями ещё не сталкивался?
А мне приходилось.
Заходит с тыла отряд, как мы думаем, наших.
А они р-раз и по нам экранами лупить начинают.
Свечение от них плюсовое, да.
Как я живой оттуда ушла "--- сама удивляюсь\ldotst
Тут интересно другое: какого червивого дьявола мы с Грейсом не чувствуем ничего?
Поле-то одно для всех!

"--*Мне тоже, "--- поддержал её технолог.
"--- Какое-то устройство не просто изменяет поле, оно ещё и в курсе, где находится каждый из нас, и даже как-то нас классифицирует.
Аркадиу, чем ты отличаешься от нас?

Вопрос был риторическим "--- я урождённый человек.

"--*И возможно, единственный таковой на сотни парсак вокруг, "--- продолжила мысль технолога Анкарьяль.

"--*Необязательно, "--- возразил я.
"--- А как вы объясните то, что осцилляции видит компьютер?

"--*Компьютер использует сапиентные паттерны поведения, чтобы запутать наблюдателя, "--- сообщил Грейсвольд.
"--- Кстати, эту идею ты мне подал.

"--*Может, нам следует разделиться, "--- предположил я.

"--*Нет, "--- отрезала Анкарьяль.
"--- Возможно, что этого от нас и хотят.
Вряд ли это устройство Картеля, иначе нас бы уже давно взяли под белые рученьки, но сам феномен похож на тщательнейшим образом подобранную приманку.
Будем исходить из того, что кто-то знает о нашем местоположении\ldotst и вести себя так, словно мы ничего не заметили.

Мы с Грейсом переглянулись и вздохнули.
Тайна манила, мы отчаянно нуждались в союзниках, но Анкарьяль была, как всегда, права.
И даже более, чем всегда.

\section{Счётная мельница}

"--*Лис, а то твоё имя, "--- Чханэ неуверенно назвала его, "--- что оно значит?

Я задумался.

"--*Arccadiu "--- <<крестьянин>>, распространённое мужское имя.
Balerianu "--- кажется, <<воин>>, baleru "--- устаревшее <<воевать, сражаться>>.
А родовое имя Luppino ознaчает <<шакал>>.
Luppina "--- <<самка шакала>> "--- название для\ldotst "--- я задумался, вспоминая подходящее слово на языке сели.
Не нашёл.
"--- Хай, это женщина, которая занимается сексом с мужчинами за товары или еду.

"--*А, "--- поняла Чханэ.
"--- Бывает такое, да.
А почему для них отдельное название?

"--*В моём родном мире женщины не были равны мужчинам.
Их половая жизнь ограничивалась.
В качестве противовеса существовали женщины, которые продавали секс.
Общество их презирало.

"--*Женщинам не давали заниматься сексом?

"--*Да.
Общество считало предосудительным, если у женщины было более одного мужчины\ldotst
Что смешного?

Чханэ больше не могла сдерживать смех и захохотала во всё горло.

"--*Хри-соблазнитель\ldotst а как\ldotst они\ldotst определяли?
Счётную мельницу вставляли в женские ворота?

Я подождал, пока она успокоится, и рассказал ей про девственную плеву.
Чханэ поперхнулась.

"--*Это специально делали?

"--*Нет, у женщин моего вида это было изначально.

"--*Глупость какая-то.
Я вначале удивилась, когда ты сказал <<продавали секс>> "--- это ж каким неприятным должен быть человек, чтобы никто даже не согласился с ним разделить ложе.
Нет, у нас тоже бывало, что кто-то обменивал вещи на секс, но это всё-таки больше символический обмен, удовольствие-то оба получают.
Впрочем, теперь всё понятно.
Женщины твоего мира не имели больше одного мужчины "--- с ними спать, как лягушкам стихи читать.

На этот раз от такой неожиданной интерпретации поперхнулся я.

"--*А у тебя когда первый секс?

Чханэ задумалась.

"--*Именно любовь?
Или Круг Доверия тоже считать?
В Круге я участвовала с первого похода, дождей с двадцати трёх.
Любовь была чуть позже "--- Манис.
Я тебе про него рассказывала.

"--*Как прошёл твой первый Круг?

"--*Примерно так же, как и у тебя "--- трезвучие, куча гениталий и слабое понимание происходящего.
Правда, я была совсем молодой, поэтому за несколько дней до Круга кормилец позвал меня к себе, чтобы я чувствовала себя увереннее.
Так я стала женщиной, "--- Чханэ тепло усмехнулась.

Я погладил девушку по обтянутой тканью коленке.

"--*Где сейчас твой кормилец? "--- спросил я.
"--- Он жив?

Улыбка Чханэ увяла, словно молитвенный мак.

"--*Хотела бы я знать.

"--*Отчего так?

"--*Когда кормильцы перестали жить вместе, мне было уже много дождей.
Меня готовили в Храме, поэтому я ушла с кормильцем в храм насовсем.
У Согхо бывала редко "--- она не особенно меня любила, а после расставания с Акхсаром и подавно.

"--*Почему ты думаешь, что она тебя не любила? "--- удивился я.

Чханэ скорчила рожу.

"--*Знаешь, это не так трудно понять, особенно маленькому человеку.

"--*Так что насчёт кормильца?

"--*Однажды в Храме сменилась власть.
Кормилец встал на защиту прежнего Первого "--- он был его другом.
Ту ночь я провела у Согхо "--- один из воинов, старый товарищ Акхсара, вскользь посоветовал мне пожить несколько дней вне храма.

"--*Похоже, просто так подобные советы у вас не давали.

Чханэ многозначительно кивнула.

"--*Кормилец пришёл далеко за полночь, весь в крови, с пробитыми доспехами.
Обнял меня и сказал, чтобы я вела себя в Храме так, как и всегда, и ничему не удивлялась.
А потом заплакал, сказал, что любит меня, и убежал.
Тех пятерых, кого он зарубил в ту ночь, я хоронила на следующее утро.

"--*С тех пор вестей не было?

Чханэ вздохнула.

"--*За ним послали убийцу, так что, скорее всего, мой кормилец уже давно прошёл жилами джунглей.
В иное я перестала верить.

\section{Страшная сила}

"--*Мне только одно непонятно, "--- заявила Чханэ.
"--- Вы "--- самые сильные существа в мире.
Где ваше оружие?

Я рассмеялся.
Анкарьяль и Грейсвольд недоумённо переглянулись.

"--*Оружие?

"--*Да.
Летающие машины, извергающие огонь.
Ножи, прорезающие камень.
Доспехи, которые выдерживают выстрел из баллисты.
Где всё это?

Грейсвольд, кажется, начал кое-что понимать.
Анкарьяль выглядела ещё больше сбитой с толку.

"--*Какие машины с огнём?
О чём ты?

"--*У меня есть браслет, "--- растерянно сообщил Грейс.

"--*Чханэ, "--- вмешался я.
"--- Мы делаем оружие из воздуха.

"--*Зачем нам машины с огнём, "--- проворчала Анкарьяль.
"--- У нас всегда было и есть два оружия "--- знания и демоническая сущность.

"--*И неизвестно, какое страшнее, "--- присовокупил Грейс.

\section{Песня о доме}

"--*Атрис как-то спросил меня, как я выжила среди всех этих сражений.
А я мечтала о доме.
Многие мои товарищи мечтали о далёких походах, о новом оружии\ldotst а я о доме.
И вот, во время одного из походов дом у меня появился.

"--*Помню этот поход, "--- засмеялся Акхсар.
"--- Золото притащила худого, оборванного, насквозь мокрого красавца и сказала: <<Это цитра моей жизни.
Не обижать>>.
После первой же песни мы поняли, что перед нами человек, путь которого начертан сохой небесного пахаря.
Мы словно переосмыслили свои жизни.
Многие плакали, говорливые вдруг надолго замолкали, а молчаливые, наоборот, начинали говорить, и их нельзя было остановить.
Золото и Хат отныне спали в отдельной палатке и называли её <<домом>>.
А почему домом, спрашиваем мы\ldotsq

\begin{verse}
Что отличает жилище?\\
Крыша, и пол, и огонь,\\
Пища на вечер и утро,\\
Две пары любящих рук,\\
\end{verse}

"--*закончила Митхэ.
"--- Атрис спел это не задумываясь, словно кто-то диктовал ему стихи\ldotst

Акхсар затянул весёлую песню, и Митхэ подхватила её:

\begin{verse}
Если нет крыши "--- то это лишь лагерь,\\
Если нет пола "--- то это нора,\\
Если нет пламени, если нет пищи,\\
Это сарай "--- посели здесь кролей.\\
Если нет любящих рук и надежды,\\
Незачем крыша, и пища, и пол.\\
\end{verse}

Воины рассмеялись.

"--*Мы пели её по десять раз, "--- припомнил Акхсар.
"--- А Хат играл нам на цитре.

\section{Спокойная старость}

Хитрам вскоре нашёл меня.
В руках жрец сжимал испачканный в глине платок.

"--*Моей хранительницы, "--- сказал он.
"--- Пришлось принести её в жертву.
Больше было некого.
И некому.

"--*Оставь, "--- я аккуратно отстранил его руку с платком.
"--- Это чересчур большая ценность для меня.
Твоего жеста вполне достаточно.

Я отошёл, и вскоре раздался сдавленный возглас.
Хитрам-лехэ дрожащими пальцами расправлял совершенно чистый платок, на котором красовалась надпись: <<Самому лучшему дарителю>>.

Жрец никому не сказал ни слова.
Но где бы я его ни увидел, он не расставался со своим сокровищем.

Вскоре ко мне подошла Анкарьяль.

"--*Аркадиу! "--- тихо сказала она мне на ухо.
"--- Это ты Хитрама разыграл?
Что ещё за дешёвые фокусы?

"--*Пусть человек порадуется, "--- сказал я.

"--*Его хранительницы давно нет, "--- сказала подруга.
"--- Сапиенты умирают навсегда, и пристанище "--- лишь сказки для живых.
Все знают о твоей божественной силе\ldotst

"--*\ldots но никто не осведомлён о её границах, "--- тонко улыбнулся я.

"--*Хитрам "--- образованный человек.
Если он узнает истину, это его убьёт.

"--*Посмотри, "--- кивнул я на старика.
Тот рассеянный взором смотрел в окно, мял в руках платок и улыбался.
"--- Посмотри внимательно, Анкарьяль Кровавый Шторм.
Нужна ли этому усталому человеку истина?
Будет ли он искать опровержения для чуда, которое позволит ему дожить своё в умиротворении?

Анкарьяль смотрела на меня.
В её глазах застыло что-то непонятное.

"--*Я передам Грейсу, чтобы он не распускал язык.

"--*Благодарю тебя.

"--*Не нужно, "--- шепнула Анкарьяль и взяла меня за руку.
"--- Я иногда жалею, что в этой Вселенной не найдётся доброй сказки, в которую могла бы поверить я.

Я улыбнулся и прижался к подруге.
Она, как всегда, читала мои мысли.

\section{Внезапно}

"--*Существует-Хорошее-Небо — травник? "--- удивился старый жрец.
"--- Мы всегда думали, что он человек!

"--*Я поговорила с пылероями, "--- добавила Анкарьяль.
"--- Они рассказывают легенды про вождя Существует-Хорошее-Небо, и в них он пылерой.

"--*Я почти уверен, что планты считают его плантом, "--- заключил я.

Мы переглянулись и захохотали.
Жрец нахмурился.
Для него только что осознанное было откровением, и наш смех казался чем-то неуместным.

"--*Я понял, "--- сказал жрец.
"--- Предки не делали различий между людьми, пылероями, травниками и няньками\ldotst

"--*Делали, "--- сказал я.
"--- Для размножения они, разумеется, выбирали особей своего вида.
Но древних тси связывала общая культура, многочисленные дружеские и половые связи.
Ни у кого не возникало даже мысли, что другой вид, к которому относились его товарищи по работе, друзья и, возможно, даже любовники, чем-то хуже своего собственного.

\section{Эпилог}

Вспомнился разговор с Чханэ.
Мы стояли на башне храма, и под нами расстилался засыпающий город.
На западе пылал закат.

"--*Идём со мной.
Я сделаю тебя хоргетом, и мы будем встречаться снова и снова, проживать жизнь за жизнью.

Чханэ улыбнулась.
В углах её глаз уже рассыпались морщинки, тяжёлые волосы цвета какао сверкали платиновой филигранью.

"--*Лис\ldotst
Я прожила вдвое больше своих родичей, и чувствую, что мне\ldotst много.

"--*Ты не хочешь быть со мной?

"--*Скажи, Лис, я сильная?

"--*Конечно.

"--*Я всегда такой была.
Позволь же мне напоследок эту слабость "--- просто умереть.
Исчезнуть навсегда.

"--*Я думал, все люди мечтают о бессмертии.

Чханэ провела рукой по волосам и посмотрела на горящий над джунглями закат, потом улыбнулась мне.

"--*Я всегда мечтала о покое.
А бессмертие\ldotst я уже бессмертна.
Народ сели, пылерои и идолы будут помнить тебя, пока не умрёт последний старик, пока последняя старуха не отправится к лесным духам.
И, вспоминая тебя, рассказывая о тебе легенды, кто-нибудь да помянет меня добрым словом.
Наши потомки будут жить, пока не погаснет солнце и не остынут океаны, и в каждом будет течь капля моей крови.
Ну а после\ldotst
Ты же меня не забудешь?

"--*А когда исчезну я?

"--*Тогда мне здесь точно нечего будет делать.

\mulang{$0$}
{"--*Я напишу о нас книгу.}
{``I'll write a book about us.''}

Она засмеялась и схватила меня за плечи:

\mulang{$0$}
{"--*Книгу?}
{``A book?''}

\mulang{$0$}
{"--*Да.}
{``Yes.}
\mulang{$0$}
{И нас будут помнить даже после моей гибели.}
{We'll be remembered even after my death.''}

\mulang{$0$}
{"--*Кто?}
{``But who will remember?''}

\mulang{$0$}
{"--*Кто-то да будет.}
{``Somebody will.}
\mulang{$0$}
{Написанное может быть прочитано.}
{Written might be read.}
\mulang{$0$}
{Может быть, её будут читать миллиарды, жители сотни планет.}
{I guess billions of folk, dwellers of hundred planets will read this story.''}

\mulang{$0$}
{"--*И мы будем встречаться снова и снова, проживать жизнь за жизнью в чьём-то воображении?}
{``And we will meet and meet again, we will live life by life in the mind's eye of somebody, won't we?''}

\mulang{$0$}
{"--*Да.}
{``Yes, we will.''}

Чханэ смотрела в закат, осмысливая мои слова.
Таких счастливых глаз я не видел никогда.
Помолчав, она кивнула и притянула мою руку к груди.

\mulang{$0$}
{"--*Я согласна.}
{``I accept it.''}

\chapter{Воин и менестрель}

\section{Висок}

Она прижала горячую чашу к виску и закрыла глаза.
Нежный жар действовал на неё успокаивающе;
по спине побежали мурашки облегчения.

\section{Слова потомков}

Так ли уж важно, что скажут современники и потомки?
Жизнь "--- она здесь и сейчас, и по большому счёту она принимает тебя таким, каков ты есть.

\section{Сомнительный кусочек к <<Помощи разбойника>>}

Вдруг прямо перед воинами возникла крадущаяся фигура.
Митхэ среагировала молниеносно "--- Легенда Серого Рассвета обагрила землю и безмятежно растущие кусты.
Молодая женщина упала, удивлённо простонав: <<Хай?>>

"--*Оружейница из Серого Рассвета, "--- буркнул Акхсар, разглядывая убитую.
"--- Правила мне клинок.
Хэситр есть?

"--*Возьми у меня в правом нижнем кармане, "--- Эрхэ бросила другу свой мешок.
"--- Митхэ, спрячь саблю, дело уже сделано.

Митхэ потрясённо стояла, выставив саблю перед собой;
Эрхэ, поколебавшись, аккуратно забрала оружие из рук подруги и вытерла подолом рубахи.

\section{Борьба}

Борьба "--- странная вещь.
Вначале тебе очень больно, и ты избегаешь её как можешь.
Потом жизнь снова вынуждает тебя принять бой, и ты борешься, мечтая о покое.
Как вдруг в один прекрасный день ты понимаешь, что без борьбы жизнь не имеет смысла.
И ты идёшь сквозь метели, стонешь от ран, проклинаешь богов и духов, но иначе жить уже не можешь.

\section{Два Храма}

Два Храма в одном городе "--- событие всегда значительное и далеко не всегда безоблачное.
Храм привык быть хозяином, и роль гостя даётся ему с трудом.

\section{Мимолётное}

Митхэ и Атрис шли рядом и думали об одном и том же "--- о том, как мимолётны поцелуи.
Впрочем, Митхэ больше вспоминала о произошедшем, пыталась усилить, запечатлеть в памяти;
Атрис же с головой ушёл в философию.
Казалось, всего михнет назад с людьми происходило что-то невероятное "--- стучали сердца, руки и губы плясали в слепом танце, полном запаха волос, дыма, пота, феромонов\ldotst и вот любовники идут, сбивая ноги о камни, и между ними снова непреодолимая стена, сложенная из пространства-времени, обязательств и стереотипов.

Порой Атрис и Митхэ обменивались взглядами.

<<Хочу ещё>>, "--- молили глаза женщины.

<<У нас достаточно времени впереди>>, "--- успокаивал её менестрель.

<<Ты сам-то в это веришь?>>

Атрис не верил.
<<Сейчас>> в его жизни давно одержало победу над <<потом>> "--- в тот самый день, когда цитра заменила дом, поле, мастерок и даже самую малость "--- пищу.

\section{Гибельная красота}

Воин от природы обладал скоростью, которая недостижима большинству.
Он мог ловить голыми пальцами ружейные дротики и стрелы.
В спарринге из всего отряда чести с ним мог сравниться разве что Ситрис "--- несмотря на недостаток скорости, разбойник очень быстро соображал, куда дует ветер, и неплохо предсказывал движения противника.
Ещё Акхсар был очень красив "--- тонкие хищные черты лица, маленькие, но яркие зелёные глаза над лезвиями точёных скул, белые зубы, пегая грива, перевязанная нитками и украшенная костяными погремушками.
Подступающая старость, как ни удивительно, делала его ещё красивее.
Женщины и мужчины ходили за ним толпами;
однако Акхсар всегда стоял рядом с Митхэ, не обращая внимания на восторженные взгляды.
Все считали его заносчивым и угрюмым, и лишь отряд чести знал "--- воин, прошедший самые известные сражения современности, боялся поклонников как огня.
Митхэ была его щитом.

"--*Может, ты всё-таки ответишь кому-нибудь взаимностью? "--- тихо спросила как-то Митхэ.
"--- Тебе понравится.

"--*Я пробовал, "--- грустно буркнул воин.
"--- После каждой ночи ощущение, что меня растерзали дикие звери.
Их любовь "--- любовь хищников.
Так что давай все будут считать, что я безответно влюблён в тебя.

"--*А ты в меня влюблён? "--- усмехнулась Митхэ, стрельнув глазами.

Впрочем, она тут же об этом пожалела.
Акхсар вдруг сжался, словно испуганный ягуар, и затравленно посмотрел на командира.

"--*Снежок, тихо, тихо, я пошутила, "--- тут же испуганно пояснила Митхэ, похлопав друга по плечу.
"--- Пошутила.

Однажды Митхэ услышала от него совсем грустные слова:

"--*Будь прокляты те, кто наделил меня такой привлекательностью.

"--*Да что ты такое говоришь! "--- всполошилась Эрхэ, пролив на себя похлёбку.

"--*Знаешь, в чём отличие между тем, когда тебя грубо лапают, и тем, когда на тебя постоянно пялятся?

"--*Ну?

"--*По рукам можно надавать.
Больше отличий нет.

"--*Ты же сам прихорашиваешься перед зеркалом!
И эти твои ленточки-погремушки!

"--*Для себя, Обжорка, не для других!
Это две большие разницы!

\section{Видевший смерть}

Митхэ взглянула в глаза Атриса.
Они были чисты, как небо.
<<Видевший смерть>> "--- так называли сели подобный взгляд.
Не ту смерть, что целится в тебя из лука, не ту, с которой можно договориться, не ту, от которой можно уклониться ловким движением ног.
Эти глаза видели неотвратимую, неумолимую смерть, которая ползла ядом по венам, которая разгоралась в лёгких пламенем болотной лихорадки и лишь в последний момент, по какой-то странной ошибке Вселенной отошла в сторону.
Митхэ знала, что у неё тот же взгляд.

\section{Круг Доверия}

"--*Капита Миция, "--- поднял руку Аурвелий.
"--- Я не мочь.

"--*Почему? "--- удивился Акхсар.

"--*Дурень, он слишком старый, "--- толкнула его в бок Эрхэ.

"--*Не проблема, "--- махнула рукой Митхэ.
"--- Делай всё, что можешь.
Ребята, поосторожнее с дедушкой.

\spacing

"--*Ну вот, а говорил, что не можешь, "--- сказала Эрхэ.
"--- Ты ещё ого-го дедуля.

"--*В ночь тысячи дождей и тростинка превращается в копьё, "--- развёл руками ноа.
"--- Но спина\ldotst эвааа\ldotst болеть.

"--*Так, размять дедушку, "--- распорядилась Митхэ.

\section{Хритра}

Парень был на редкость худой, угловатый, с длинной шеей.
О его ключицы можно было ненароком порезаться, а кадык торчал из горла, словно острый нос вросшего в скалу корабля.

Трукхвал казался угрюмым, но Митхэ удалось его разговорить.
Вскоре она поняла, что нашла интереснейшего собеседника.
Казалось, в свои двадцать восемь дождей он знает всё, чего только может коснуться разговор;
его язык был богат, словно все поэты прежних дождей вселились в одно тщедушное тело.

"--*Скажи, Трукхвал, а что там, на горе?
Сколько лет жила в Тхартавирте, но не видела этого строения.

"--*Его сложно увидеть, "--- согласился ученик жреца.
"--- Это Малый храм "--- Хритра.
Когда-то жертвы богам приносили там.

"--*Что случилось?

"--*Землетрясения, "--- объяснил Трукхвал.
"--- Здесь землетрясения были не всегда "--- они пришли примерно тысячу дождей назад.
Именно тогда и была выстроена Трёхэтажная пирамида.
Сейчас Хритра, хай, в опасном состоянии "--- вот-вот рухнет.
Митрис ар'Люм, впрочем, ходит туда постоянно.

"--*Кто это?

"--*Новый библиотекарь, который с севера приехал.
Он немного необщительный, хоть и всегда вежлив, и помогает, и вообще старается не доставлять этим хлопот.
Книги переписывает по ночам и часто там, на горе.
Вероятно, у него оборудовано место с электрическим освещением "--- при свечах ночью можно испортить глаза.
Одно время мы увлекались лампами "--- такие хрустальные шарики с кристаллами, кажется, свинцового блеска и карборунда.
Очень сложные в изготовлении, но очень яркие.
У нас ещё осталось несколько десятков.

"--*Что такое карборунд?

"--*Минерал, залежи которого, кстати, как раз в скале под Хритрой.
По крайней мере, были.
Камнерезы его очень любят, кажется, используют для полировки или сверления.

"--*Ты всегда говоришь <<кажется>>, <<вероятно>>, но при этом оказываешься прав!

"--*Если бы у жреца было оружие, его ковали бы из сомнения.

"--*Хорошо сказано.
Кажется, я была несправедлива к нерешительным.
Можешь сводить меня в Хритра?
На экскурсию.
Показать залежи этого карборунда, ну и Малый храм заодно.

"--*Хритра под замком, "--- пожал плечами Трукхвал.
"--- Может, тебе стоит попросить ключ у Митриса?
Если он у него есть, конечно.

"--*Зачем вообще запирать руины?

"--*Туда лазила молодёжь, когда хотели разделить ложе в необычной обстановке, "--- улыбнулся Трукхвал.
"--- Жрецы боялись, что кого-нибудь настигнет весьма необычная смерть.

"--*А это-то ты откуда знаешь? "--- захохотала Митхэ.

"--*Слышал.

"--*Слышал.
Сам лазил небось.

"--*Нет, "--- покачал головой Трукхвал.
"--- Мне рассказывали про влюблённость и сексуальное желание, но сам я такое не испытывал.
Я делил ложе и с женщиной, и с мужчиной "--- ради нового опыта.
С мужчиной не вышло совсем, а вот женщина была хороша.
Секс похож на танец "--- это беседа тел.
Секс похож и на письмо "--- ты определённым образом проводишь пером по человеку, и пятно чернил вдруг обретает смысл.
Но книги читать мне показалось интереснее, и вопрос, как провести вечер, для меня не стоял.

"--*А в Круге ты участвовал?

"--*В Мягких Руках участвую каждый год.
Всё-таки я стану жрецом, а руки жреца обязаны знать ритуал.

"--*И тебя это не смущает?

"--*Митхэ, мне приятен секс, но я не чувствую в нём потребности.
Я изучил все приёмы, и их гораздо меньше, чем тактик на столе дипломатии в Метритхис.
В определённый момент мне просто становится скучно.

"--*Я не понимаю, как такое может быть.

"--*А я мало что понимаю из того, что касается человеческих взаимоотношений "--- дружбы, интриг.
То же весьма ограниченное число тактик.
Но кто-то находит в этом удовольствие.

"--*То есть для тебя интерес только в количестве тактик?

"--*Для меня удовольствие от решения задачи пропорционально её сложности.

"--*То есть секс для тебя "--- задача?

"--*Именно.

"--*Теперь поняла.

"--*А для тебя?

"--*Созерцание.
Нот и красок всего двенадцать, но их сочетания не могут надоесть никогда.

"--*Вопросов больше нет.

\section{Битва Трёх}

"--*Я устать, "--- усмехнулся Аурвелий.
"--*Когда я устать, я всегда\ldotst делать\ldotst неправильный язык, слово, никто не понимать.
Извините.

"--*Расскажи тогда на ноа-лингва.
Мы всё-таки понимаем твой язык.

"--*Это невежливый.

"--*С нашей стороны было бы невежливо заставлять тебя говорить на сложном чужом языке, когда ты устал.
Всё-таки сели как иностранный язык "--- это скорее пытка, чем удовольствие.
Так всё-таки, за что отвечают ваши боги, боги ноа?

Аурвелий подумал и перешёл на ноа-лингва:

"--*Мы не знаем, за что отвечают боги, но мы знаем, что знаменует усиление или ослабление каждого из них.
Если солнце выжигает посевы и на морском берегу проступают соляные письмена, Деа Марин слабеет.
Мы приносим ему жертвы.
Если бушуют штормы и вода касается deserto sancto\footnote
{Светлая (чистая) пустыня (ноа-лингва). \authornote},
слабеет Деа Солар.
Тогда мы подпитываем его.
Если же люди счастливы и довольны, если посевы дают богатый урожай, а моря светлы и гостеприимны "--- Деа Акседент проигрывает битву, и тогда мы помогаем ему.
Смерть бога "--- шаг к концу мира.

"--*Вы верите, что у мира есть конец? "--- удивилась Эрхэ.
"--- Я слышала Далёкие речи ноа, речь о Бесконечном\ldotst

"--*Да, Бесконечный может переходить с одной земли на другую, и находить новый свет, когда старый погаснет.
Но земли смертны, и смерть нашей земли зависит от них, богов.
Нужно очень стараться, чтобы двое не победили одного, и нужно стоять насмерть, чтобы один не победил двоих.

"--*Но кто помогает людям? "--- удивилась Митхэ.
"--- Неужели вы думаете, что в этом мире нет сил, которые выступают на стороне людей?

"--*Четвёртая сила есть, "--- ответил Аурвелий.
"--- Эта сила держит мировой порядок, эта сила не даёт победить богам.
Эта сила "--- мы, те, кто дышит и когда-то дышал.
Вы, белые рубахи, тоже приносите жертвы и помогаете держать мировой порядок, вы только не знаете, кому вы их приносите.
Мы знаем.
Если слабеет один из богов "--- мы заклинаем его собирать все жертвы в известном мире.
И его подпитывают "--- и мы, и вы, и идолы, и ркхве-хор.

"--*И ты веришь в это?

Аурвелий кивнул.

"--*А как же быть с Легендой об обретении?
Это ложь?

"--*Нет, "--- ответил Аурвелий.
"--- Легенда об обретении рассказывает о прибытии Деа Акседента.
Именно его вы называете Безумным.
Но сели не знают, что ещё до Деа Акседента миром правили боги солнца и моря, и мы, ноа, поддерживали их борьбу.

"--*Какая странная легенда, "--- сказала Эрхэ.
"--- Я не уверена, что поняла её правильно.

"--*Не имеет значения легенда, "--- заметила Митхэ.
"--- Главное, что это работает и наш мир успешно сдерживает гнев богов, сколько бы их ни было.

"--*Вот здесь я соглашусь с Митхэ, "--- ухмыльнулся Акхсар.
"--- Мой кормилец говаривал так: верь во что хочешь, но иди куда следует.

\section{Одинокий Столб}

Одинокий Столб можно было назвать городом с большой натяжкой "--- он едва насчитывал полторы сотни строений, включая храм, гостиницы и исторические жилища.
Тем не менее он представлял из себя настоящую крепость, ограждённую стеной.
Впрочем, непонятно, зачем первые строители сделали стену "--- она не пригодилась ни разу.
В темноте при взгляде с дорог или реки Одинокий Столб казался леденцовым дворцом "--- ажурные венцы башен и кромка стены сияли от света оранжевых хрустальных тыкв, внутри которых пылал настоящий природный газ.
Хитроумное древнее приспособление отводило газ от окружающих святилище болот;
за день его собиралось достаточно, чтобы тыковки пылали всю ночь.

Название святилищу дал, как можно догадаться, Столб "--- один из четырёх талисманов города.
Никто не знал, сколько лет Столбу, но ни у кого не вызывало сомнений, что он был возведён самими Древними.
Столб был сделан из неизвестного материала, похожего на белоснежно-чистый фарфор, оплетённый кукхватровыми кольцами.
Как и стена, Столб имел хрустальные окошки, и они также светились по ночам, но как и зачем "--- не знал никто\footnote
{Главной реликвией Одинокого Столба является надземный модуль (маячок) золотодобывающей установки тси. Как показало исследование устройства, оно прекратило работу за пять тысяч лет до описываемых событий, перейдя в режим ожидания. В настоящее время Столб признан культурно-исторической ценностью планеты Тра-Ренкхаль и подлежит охране. \authornote}.

Прочими тремя талисманами города были тысячелетний молчащий кедр, Запретная роща "--- роща благородного баньяна, а также развалины Су'макхэтхвал "--- пятнадцать древних землянок, первое поселение возле Столба.
Каждая реликвия находилась в своём квартале: Столб "--- в Верхнем Приозерье, Кедр "--- в Квартале Кедра, Су'макхэтхвал относился к Кварталу Снежных Бород, а Запретная роща была в ведении Большого и Малого Домов.
При этом любые две реликвии разделяло не более ста шагов.

\section{Никто}

Воинов в святилищах нет.
Есть только стражники.
Стражники приносят немного другую клятву.

"--*Оружие, Митхэ ар'Кахр, "--- улыбнулся стражник и протянул ладонь.

Митхэ, поколебавшись едва ли мгновение, вложила в протянутую руку саблю.

"--*Хай, у тебя новая сабля? "--- удивился стражник.
"--- Красивая.
Случайно не работа Секхара ар'Митр э'Сотрон?

"--*Его, "--- улыбнулась Митхэ.

"--*Стиль очень узнаваемый, хотя последнее время, по слухам, гравировкой занимаются его дети.
Я передам её на хранение в Большой Дом.

"--*Почему туда? "--- насторожилась воительница.

"--*Так надо, Митхэ ар'Кахр, "--- строго ответил стражник.
И полушёпотом добавил:
"--- Дела и законы сели нас касаются в той же мере, что и всегда.
А вот воры к нам зачастили.
Оружие чрезвычайно ценное, и я настоятельно рекомендую согласиться с этой мерой предосторожности.

"--*Хорошо.

"--*Неуютные нынче времена, "--- вздохнул стражник.
"--- Усомнилась ли бы ты во мне дождей двадцать назад?

"--*Прости, "--- поклонилась Митхэ.
"--- На мою долю за последнее время выпало слишком много.

"--*Мы слышали, "--- кивнул мужчина.
"--- Кстати, не жди слишком многого от Сцепленных Рук.
Сейчас у них в головах поддержание не мира между народами, а видимости собственного нейтралитета.

"--*Что ты имеешь в виду?

"--*Ты не найдешь кого-то, кто поможет тебе с Атрисом, "--- без обиняков пояснил стражник и открыл ворота.
"--- Поэтому, если хочешь помощи, проси её у никого.
Никто, как ни удивительно, всегда в курсе событий.
В катакомбах много этих никого, и периодически из-за них мы чего-то недосчитываемся.
Например, хороших сабель.
Храни тебя Сит.

\razd

То, что стражник назвал <<катакомбами>>, на самом деле было целым городом, но уже подземным.
Вернее, аж двумя городами "--- Хйор'Тар и Хйор'Катх\footnote
{Хйор Стариков и Паучий Хйор.
Хйор "--- слово, перешедшее в цатрон, вероятно, из языка ркхве-хор.
Ркхве-хор называют так поселения добытчиков ископаемых в естественном разломе;
на стенах разлома устанавливаются навесные полки, мосты и лестницы, а сами жилища устраиваются в нишах.
Ркхве-хор считают (и небезосновательно), что по одному и тому же месту два разлома не проходят, поэтому в сейсмически активных районах предпочитают строить хйор, а не обычные наземные поселения. \authornote},
общая площадь которых превышала площадь самого Одинокого столба в полтора раза.
Концентрические разломы, центром которых был всё тот же Столб, за прошедшие тысячелетия поросли кальцитовыми натёками, и когда-то давно в них решили сделать жилища.
Пока не выяснилась неприятная деталь "--- при наличии хотя бы одного слабого места с болот в систему пещер поступал тот самый газ, делая нижние этажи подземных городов непригодными для жизни.
Вначале с этим пытались бороться, но природа взяла своё.
На Хйор'Тар и Хйор'Катх махнули рукой, и они теперь лишь изредка поражают красотой приезжих, достаточно смелых, чтобы спуститься в разлом.

Жили в подземельях и воры.
Это было крайне редкое занятие в землях сели "--- честным трудом можно было заработать проще и быстрее.
Поэтому ворами становились либо совсем юные беспризорники, либо идейные люди, которые не могли представить жизнь без риска, либо кутрапы, которых ждала смерть в любых уголках Короны.
Причём беспризорники, повзрослев и получив метку за Насилие, чаще всего просто уезжали подальше и начинали новую жизнь, благо такая возможность представлялась ещё много раз "--- пока человек не совершал преступление в пределах нового города, он был для него чист и честен.
Формально.
Люди же, несмотря на все увещевания дипломатов, заранее настороженно относились к человеку с метками трёх-четырёх городов.

В Одиноком Столбе воры жили очень тихо, воровали в основном пищу и одежду, аккуратно и помалу, так как знали "--- если дать повод стражникам заняться катакомбами, то катакомбы опустеют надолго.
Заблудившимся в подземельях приезжим воры часто устраивали экскурсии за золото или по крайней мере показывали дорогу наружу.
Именно поэтому их не трогали и на воровство смотрели сквозь пальцы.

<<Под землёй растут лишь грибы, а из грибов лепёшки плохие>>, "--- говорили жители святилища, если речь заходила о разломах.
Очень часто Хйор'Тар и Хйор'Катх были единственным местом, где мог жить кутрап.
Кое-кто из жителей специально оставлял пищу и одежду на видном месте.
А раз в декаду местные жрецы отправлялись ночью <<погулять>> с полным набором медицинских инструментов.
Так и жили.

\section{Что такое любовь}

"--*Я не знаю, что такое любовь, "--- призналась Митхэ.

"--*Любовь "--- это желание, чтобы близкий жил и процветал.

"--*Или хотя бы не мучился, "--- добавила женщина.

"--*Вот видишь, это просто.
Все стихи и песни можно сократить до четырёх слов "--- жизнь, процветание и милосердная смерть.

\chapter{Подпольщики}

\section{Талианский канал}

"--*Сиэхено успела оставить сообщение перед смертью, "--- сказал Самаолу.
"--- Скорбящие используют для сообщения оригинальную восьмиполосную таблицу языка Эй, сочетающую макропараметр планеты Тра-Ренкхаль и неизвестный микропараметр.
К сожалению, сообщение неполное.
Это вся информация, известная на сегодня.

"--*Возможно, какая-то вариация Талианского канала, "--- предположил голос.
"--- А, вы не в курсе, Самаолу.
А вот Штрой должна знать.
Макропараметром таблицы являются географические координаты.
Микропараметром может быть что угодно.
В частности, агенты Картеля использовали бутылочки из цветного стекла, выкладывая ими в определённых координатах оставшуюся часть сообщения.

"--*Требуются гигантские затраты, чтобы найти сообщение, "--- мрачно сказала Штрой.
"--- Но в итоге Ад справился, Талианский канал был раскрыт.
Это положило весомый камешек на чашу победы Ада при Десяти Звёздах.

"--*Потому что когда бутылочки появляются в девственном лесу или на необитаемом острове, это наталкивает на определённые мысли, "--- согласился голос.

"--*Я распоряжусь насчёт поисков, "--- кивнул Самаолу.

"--*Не тратьте время, "--- сказал голос.
"--- Талианский канал может быть доработан с учётом местности, климата и прочих особенностей.
Кроме того, сообщения могут быть временными "--- например, изо льда или песка.
Не забывайте также, что одним из Скорбящих почти наверняка является демиург, что значительно расширяет список возможных макропараметров планеты.
Может, он вообще кодирует сообщения ветром, молниями или дождём.
Звучит дико, но ведь теоретически это возможно?

"--*<<Слова не имеют смысла, если нет того, кто может их понять>>, "--- процитировала Штрой.

"--*Штрой! "--- восхитился голос.
"--- Вы цитируете ненавидимого вами Люпино?
Я чрезвычайно рад, что вы растёте как профессионал.

"--*Враг тоже может быть прав.

"--*Я вам скажу больше, Штрой "--- иногда враг ближе к истине, чем вы, но природа мыслящего существа подразумевает ещё и некое комфортное расстояние от этой истины.
Именно поэтому мы стремимся к победе, а Скорбящих устраивает любой мир.

"--*Вы хотите сказать, что мы идём неправильным путём? "--- ошарашенно спросил Самаолу.

"--*Ни в коем случае.
Я хочу сказать, что местоположение истины во Вселенной для практического применения абсолютно неважно.
Важно найти собственное место.
Скорбящие же ищут саму истину "--- и в этом их слабость, потому что даже найдя, они не обретут ничего значимого для себя.

"--*Боюсь, как бы я не начала им сочувствовать, "--- скривилась Штрой.

"--*Сочувствуйте, Штрой, сочувствуйте и не бойтесь!
Ненависть заставляет отрезать врагу пути к отступлению, и враг сражается до конца.
Если же вы из сочувствия оставите им лазейку "--- они сдадут свои позиции проще и быстрее.

"--*Так что делать с сообщением Сиэхено? "--- осведомился Самаолу.

"--*То, что я сказал.
Займитесь не кодом, а подозреваемыми "--- они сами выведут нас куда надо.

\section{Ярлык}

"--*По убеждениям я скорее аристократ, "--- ухмыльнулся Самаолу.

"--*Остерегайтесь придерживаться <<убеждений>>, Самаолу, "--- сказал голос.
"--- Если личность навешивает на себя ярлык, то велика вероятность, что вместе с прогрессивными идеями она потянет и весь причитающийся идеологический мусор.
Придерживаться <<убеждений>> "--- это всё равно что носить с собой мешок руды вместо кошелька с монетами.

"--*Но то же самое и с выбором стороны в войне, "--- заметила Штрой.
"--- Вы можете ни словом, ни делом не навредить врагам, но вас покарают за ваше знамя.

"--*Штрой, вы определённо меня сегодня радуете, "--- восхищённо ответил голос.
"--- Пожалуй, я буду ходатайствовать о вашем повышении.
И ещё, как мне кажется, текущее дело "--- не для вас.
Что скажете?

"--*Вам виднее, "--- поклонилась Штрой.
"--- Но я бы хотела довести его до конца.

\section{Нужна теплица}

"--*С помощью кольцевой теплицы можно в кратчайшие сроки оживить практически любую планету в обитаемой зоне, не пользуясь услугами хоргетов, "--- сказал я.
"--- А теперь представьте, что будет, если это знание станет общедоступным.
Всеобщая экспансия за пределы ка'нетовского радиуса.
Любой хоргет сможет двигаться сколь угодно далеко в любом направлении, найти подходящую планету и стать её полновластным демиургом.
Каждому хоргету "--- по планете!
Ад и Картель находятся в истинном равновесии ровно до тех пор, пока у них есть физические границы.
Убери границы "--- и старый мир, основанный на идее дефицита, рухнет, как когда-то рушились колониальные империи.

Я помолчал, собираясь с мыслями.

"--*Второе направление работы "--- планетарная система.
Нам необходимы её чертежи.
Демиурги беззащитны перед организациями демонов.
С доступностью системы планетарной защиты гегемонии Ада и Картеля придёт конец.
Более того, демиург вступит в мутуалистические отношения с сапиентной цивилизацией своей планеты, так как от её технического развития будет напрямую зависеть его собственная безопасность.

"--*Мне не нравится то, что ты предлагаешь, Аркадиу, "--- сказал Грейсвольд.
"--- Жизнь демиурга для каждого демона "--- это хорошо, но я хотел бы общаться с себе подобными, а не запираться в планете на веки вечные.
То, что ты предлагаешь, разрушит социум демонов "--- как экспансия за пределы ка'нетовского радиуса, так и превращение демонов в богов.
И существует ещё одна проблема.
Для минус-демонов места в этой системе не будет.

"--*Технологии тси помогут нам и здесь.
Вдруг существует возможность вывести нуль- или даже минус-сапиентов?

"--*Вероятность крайне мала, "--- рассудительно сказал технолог.
"--- Боюсь, что если нам не удастся этого сделать, Картель будет драться до последнего.
А это значит, что они почти наверняка победят.
Поэтому, если ты действительно хочешь создать дивный новый мир, следует проработать момент с минус-демонами.

\section{Интуиция}

"--*Вы не понимаете суть человеческой интуиции.
Это тонкая обработка поступающих в мозг сигналов.
Всё, что находится за пределами чувствительности и экстраполяции чувственных данных, "--- не интуиция, а галлюцинации.

"--*Прикрывшись этим определением, вы упустили из виду ширину диапазона чувствительности тси и их базовый уровень интеллекта.
У интуиции есть границы, но определить границы интуиции тси я пока затрудняюсь.
Тси учили пользоваться интуицией.

\section{Ааман (отрывки)}

Стигма по очереди взглянула на Штрой сильно косящими глазами, словно взглянули два человека "--- весёлый и задумчивый.
Штрой напряглась "--- самый этот взгляд вызывал в её теле тревогу.

Из-за ширмы неторопливо вышел жилистый крепкий мужчина с непробиваемо тупым лицом.
Это лицо могло ввести в заблуждение сехмар, но не хоргета "--- глаза светились умом и странным, непередаваемым фанатизмом.

Ааман Третий Защитник.
Оцифрованный сапиент с одной из периферических планет.
Ад взял на службу несколько Телохранителей из местного племени.
Они следовали своей философии "--- неважно, кто попросил тебя о защите, ты обязан защищать попросившего даже ценой жизни.
Этих пятерых интерфекторов использовали на переговорах как гарантию мира "--- они без колебаний пресекали любую агрессию.

<<Атрис связался со Стигмой и предупредил её насчёт меня.
Или\ldotsq>>

Штрой нахмурилась.
План придётся менять на ходу.
От Аамана она не добьётся ничего "--- урождённый сапиент, согласно некоторым отчётам, ходил в Чистилище как на работу, аналитики уже давно махнули на него рукой.
Приближаться к Атрису было чревато.
Рискнуть и попробовать выжать нужные данные из рыжей стервы?
С Хореймом и прочими она могла надавить на Стигму, но этот демон\ldotst
Что, если его со Стигмой связывает нечто большее, чем простая клятва?

Штрой выросла в связке конкурирующих стратегов, где успех напрямую зависел от скорости изложения идей.
Эта методика была достаточно распространённой.
Стигма же знала, что иногда лучше даже не думать "--- всегда есть вероятность, что кто-то изобрёл устройство для чтения мыслей.

\spacing

"--*Зачем ты занимаешься этой ерундой?
Ты же не биолог и никогда им не была.

"--*Даже у демона могут быть мечты, "--- сказала Стигма.
"--- Однажды я стану биологом.
И жуки будут у меня жить на более веских основаниях.

Штрой подошла к террариуму и открыла его.
Индиго-светлячки тут же разлетелись в разные стороны.

"--*Ой, "--- притворно прикрыла рот Штрой.
"--- Прости.
Ты же сможешь их потом собрать?

"--*Я всё равно собиралась их выпустить, "--- улыбнулась Стигма.
"--- У них есть крылышки и челюсти.
Пусть летают и грызут, что найдут.

\spacing

Штрой поднялась и подошла к Стигме вплотную.
Диковатое оцелотовое лицо пылало сексуальным жаром и яростью ущемлённого самолюбия, и сколь много огня было в девушке, столь же холодным, напрочь лишённым сексуальности, но невероятно гармоничным казалось лицо Стигмы.
Светящиеся чистым светом расчёта глаза одним своим существованием высмеивали амбиции Штрой, и подошедший Ааман властно оттолкнул Штрой, напоминая молодой демонице о своём существовании и своей роли.

<<Ааман молодец, "--- отметила Стигма.
\mulang{$0$}
{"--- Едва ли он знает о смертельном сюрпризе в моём комбинезоне, но последствия физического контакта он просчитал совершенно правильно>>.}
{``He hardly know of the deadly surprise I've got in my suit, but he correctly predicted consequences of physical contact.''}

Штрой, похоже, тоже поняла, какого рода опасность ей грозит.
На её лице расцвела и застыла улыбка превосходства.

\mulang{$0$}
{"--*Хорошая попытка, Стигма.}
{``Nice try, Stijma.}
\mulang{$0$}
{Как вижу, ты не любишь марать руки.}
{You don't like to get your hands dirty, I guess.''}

Стигма вместо ответа показала испачканные удобрениями ладошки.

\mulang{$0$}
{"--*Очень смешно, "--- скривилась девушка.}
{``Very funny,'' the girl made a face.}
\mulang{$0$}
{"--- Разумеется, я сообщу куда следует о твоих методах ведения переговоров.}
{``Of course I'll report about your negotiation techniques.''}

"--*Ты хочешь наказать меня за желание защититься?

"--*Ты уже давно перешла границы защиты.

Стигма позволила себе немного искренности:

"--*Это ты пришла ко мне, а не наоборот.
Я не хочу делать тебе больно.

"--*Ты и не сможешь.
Однако драться тебе придётся, "--- оскалилась Штрой.
"--- Ты предала Орден.
И я это докажу.

Штрой махнула свите и удалилась.
Стигма повернулась к Ааману.

"--*Благодарю тебя, Телохранитель.
Твоя клятва исполнена.

Ааман поклонился и вышел вслед за Штрой.

\razd

Штрой ждала Аамана у лифта.

"--*Здравствуй, друг, "--- сказала она.
"--- Я знаю твою приверженность кодексу Телохранителей Слит-Же и потому хочу попросить тебя о защите.

Ааман медленно повернулся и посмотрел на девушку.

"--*Ты помешал мне служить Аду.
Идти против Ада неразумно.
Если ты будешь защищать меня от заговорщиков\ldotst

"--*Штрой Кольцо Дыма, "--- сказал низким хриплым голосом Ааман.
"--- Мои глаза широко открыты.
Твой сон будет спокоен.
Сталь в твоём голосе, сталь в твоей ладони "--- смерть твоя же, забвение до конца времён и после него.
Я буду сопровождать тебя всегда, пока ты не скажешь, что клятва исполнена.

Штрой нахмурилась.

"--*А если я освобожу тебя от клятвы на день?

Ааман оскорблённо выдохнул.

"--*Больше ты её не получишь, "--- процедил Ааман.
"--- Моя клятва не спит и не ест.

"--*Ты свободен, уходи, "--- рявкнула Штрой, великолепно изображая ярость.
"--- И впредь думай десять раз, прежде чем помогать врагам Ордена.

Ааман поклонился и молча вошёл в лифт.

"--*Отлично, "--- с удовлетворением сказала Штрой.
"--- Осталось четверо Защитников.
Ещё трое подвергнутся пробам, и пятый будет служить мне до смерти.
Раб кодекса "--- раб того, кто найдёт в этом кодексе прорехи.
Странно, что никто не сделал этого раньше\ldotst

<<Умная, но чересчур болтливая>>, "--- молча констатировала Стигма и выключила микрофон.
Светлячок на спине Штрой задрал лапки и упал на пол.

\section{Фейл с Телохранителем}

"--*Штрой, вы хотя бы день можете прожить без переполоха?
Я слышал, что отделу 100 пришлось нейтрализовать одного из Телохранителей, чтобы спасти вас.
Надеюсь, вы не пытались его завербовать?

Штрой вытаращила глаза.

"--*Понятно, "--- буркнул голос.
"--- Вы могли хотя бы спросить меня перед таким опасным шагом.
Он на вас напал, не так ли?
Каковы потери?

"--*Я потеряла четверых агентов, "--- Штрой склонила голову.
"--- Прошу прощения.

"--*Штрой, запомните раз и навсегда.
Ад "--- это клубок интриг, и если вам кажется, что кто-то что-то до сих пор не сделал "--- скорее всего, на это есть веские причины.
В Кодексе Слит-Же последний пункт звучит так: <<Если подзащитный использует Кодекс для манипуляции Телохранителем, подзащитный должен быть уничтожен>>.
Этот пункт может трактоваться Телохранителем по его усмотрению.

"--*Я тщательно замаскировала манипуляцию!

"--*А я тебе говорил, что ты недооцениваешь Стигму, "--- заявил Самаолу.
"--- Может, Телохранитель и достаточно глуп для тебя, но Стигма достаточно умна, чтобы растолковать сложное глупцу.

"--*С Защитниками шутки плохи, "--- подытожил голос, "--- и я запрещаю вам приближаться к ним впредь.

Штрой кивнула с явным облегчением.
Видимо, у неё отпало всякое желание приближаться к Защитникам.

"--*А что насчёт Стигмы?

"--*Дальнейшие попытки чересчур опасны.

"--*Но\ldotst

"--*Никаких <<но>>, Штрой.
Я не хочу вас потерять.

Последняя фраза огрела Штрой словно кнутом.
Она упала на колено и прижала руку к груди:

"--*Я вас не подведу, владыка!

"--*Вы опять начинаете?
Впрочем, ладно, продолжайте.
Стыдно признаться, мне это начинает нравиться.
Но не при агентах.

\section{Плут и Ягуар}

Стигма подошла к доске с Метритхис.
Эта игра очень нравилась стратегу.
Стигма часто играла в неё сама с собой "--- никто из знакомых демонов не придёт ради этой очевидной траты времени.

На доске нерешительно топтался зелёный Плачущий Ягуар.
Человек чести, чьё призвание "--- служить.
Однако его окружили чёрный Плут и красный Разбойник "--- не признающие верности, не имеющие чести.
Ягуар был обречён.

Немного подумав, Стигма передвинула Плута поближе к Ягуару.
Плут превратился в Купца, Плачущий Ягуар почернел и превратился в Телохранителя, и попавший под горячую руку Разбойник отправился в коробку.

Стигма улыбнулась и подумала про себя, что отныне Купец и Телохранитель пойдут одной дорогой.

\section{Монограмма Стигмы}

Стигма рассеянно водила пером по бумаге.
Письмо и рисование доставляли ей невероятное удовольствие.
Вот на бумаге появился человечек в причудливой одежде.
Стигма медленно и аккуратно начала заштриховывать открытые участки.

Её рисунки проверяли десятки тысяч раз на наличие кода.
Не нашли.
Трудно найти код там, где его нет.
Она просто очень любит рисовать.

Заштриховав человечка, Стигма аккуратно вывела в углу свою монограмму.
Знаки-подписи были в моде у древних тси, и Стигма в подражание им создала собственный.
Пришлось, конечно, порыться в архивах, но результатом стратег была довольна "--- знак достаточно точно отражал склад её личности.
Острый как стилет клин, похожий на щит полукруг, игривый хвостик, который менял длину в зависимости от настроения Стигмы.
Перечеркнула сооружение решительная, неторопливая горизонтальная черта.
Красивый рисунок и никакого кода.

Да, никакого кода.

\section{Кольбе и Рабе}

"--*Война "--- участь тех, кто обделён разумом, "--- сказал Кольбе.

"--*Война "--- удел тех, кто не умеет летать, "--- сказал Рабе.

"--*Мы никогда не примем войну сапиентов, "--- сказал Кольбе.

"--*Вселенная бесконечна, "--- сказал Рабе.

"--*Есть много мест для жизни и процветания, "--- закончил Кольбе.

"--*Для жизни и процветания, "--- хором повторили Атрис и Митхэ.

Братья улыбнулись.

\section{Смерть Сиэхено (отрывки)}

"--*Ты "--- тси, "--- сказала Сиэхено.
"--- Твои предки были умны и мягкосердечны.
Это есть и в тебе.

"--*Я "--- сели, "--- ответила Чханэ.
"--- Я так же умна, но, на твоё несчастье, ещё и умею быть жестокой.

Сиэхено поёжилась.

"--*Чего ты хочешь?
Возьми и оставь меня в покое.

"--*Нам нужен визор, "--- перешла к делу Чханэ.
"--- Ты "--- лучший визор в Ордене.
Переходи на нашу сторону.

"--*Я верна Ордену, "--- сказала Сиэхено.
"--- А вы хотите его уничтожить.

"--*Мы не хотим уничтожать ни Ад, ни Картель.
Мы хотим мира.
Но для этого нам нужно, чтобы с нами считались.

"--*Почти то же самое, в других выражениях, говорил один человек с Древней Земли, "--- заметила Сиэхено.
"--- Он снял кровавую жатву.
Его слова после повторяли многие, и каждый оставлял за собой горы трупов.
Для миротворца в тебе чересчур много обид.
Ты познала милосердие и жестокость, но ещё не знаешь их границ.

\spacing

"--*Если вы связываетесь с агентами в Аду, то как? "--- промурлыкала Сиэхено.
"--- Ах да, домики\ldotst песчаные домики\ldotst я ведь подозревала, что в них скрывается код.

Сиэхено снова запела.

"--*Я за всё детство придумала только один, "--- пожаловалась она.
"--- Я его совершенствовала, но никогда кардинально не меняла форму.
А тут детишки лепят совершенно разные домики\ldotst словно их кто-то этому учит\ldotst

"--*Неплохо, "--- признала Чханэ.
"--- Именно поэтому ты нам нужна.

Да, Сиэхено была нужна Скорбящим.
И Чханэ передёргивало от одной мысли о сотрудничестве с этой тварью.

\spacing

Чханэ почувствовала, что её пробирает ледяная дрожь.
Демоны не зря отгораживались от собственных тел.
Сиэхено всеми силами искала уязвимость в обороне интерфектора.
Модули Анкарьяль пока спасали, блокируя опасные эмоциональные реакции, но факт оставался фактом "--- это встреча равных, а не охотника и жертвы.

<<Следуй моему плану, "--- сказал Аркадиу, "--- никакой самодеятельности>>.

Беседа давно уже вышла за рамки плана.
Чханэ поняла "--- ей \textit{придётся} убить Сиэхено, если она её не убедит.
Что могло убедить визора?
Рассказать ей всю подноготную легенды, чтобы она осознала, в какой переплёт попала, и тем самым уничтожить эту самую легенду?

<<Лис, лесные духи, почему ты не предупредил, с кем мне придётся иметь дело?!>>

"--*Малышка, "--- ласково сказала Чханэ, и Сиэхено неподдельно вздрогнула, испугавшись такой неожиданной перемены.

"--*Ты всё ещё надеешься надавить на мои чувства, интерфектор?

"--*Я не хочу на тебя давить, "--- прошептала Чханэ.

"--*А что ты сейчас делаешь? "--- осведомилась Сиэхено.
"--- Тебе придётся меня убить.
Ты проиграла ещё сто десять секунд назад.

"--*Я не хочу победы, дитя.
Я хочу мира.
Как и мои предки, я хочу только мира.

"--*Опять ты за своё.
Мы всё обсудили\ldotst

"--*Я никогда не ударю в спину другу, "--- сказала Чханэ.

Сработало безотказно, как когда-то это сработало на Лусафейру.
Сиэхено заколебалась, затем её личико снова стало умиротворённым.

"--*Вдвоём против целого мира?
Увольте.

"--*Это лучше, чем против того же мира одной.

И снова в яблочко.
Сиэхено выронила бусы и судорожно начала искать их на столе.
Чханэ подняла бусы и опустила их в ладонь девушки.

"--*Ты всё равно убьёшь меня, "--- прошептала Сиэхено.
"--- Если я не соглашусь, ты меня убьёшь.

<<Вот она, грань между доверием и недоверием>>.
Если сейчас отпустить Сиэхено, либо она приобретёт союзника, либо часть плана развалится, как карточный домик.

\section{Гениальная игра}

"--*Метритхис, "--- проворчал голос.
"--- В эту игру играет весь Ад.
Кто-то даже говорит, что она гениальна "--- она проводит явную параллель между квантовой физикой и сапиентным обществом, между струнами и индивидами.
Я тоже нашёл игру интересной, однако надоело на каждом углу слышать её термины.

\section{Ещё один секрет игры}

"--*Я расскажу вам ещё об одном секрете Метритхис, "--- сказал голос.
"--- При должном мастерстве игроков в неё можно играть вечно.
Собственно, достигнуть истинного равновесия Нэша.
Несмотря на все старания тси, сели были плохи в математике и этого не знали.
Хотя по Тра-Ренкхалю ходили легенды о партиях длиной в год, десять и даже шестьдесят лет.

\section{Беседа Гало и Тахиро (отрывки)}

Тахиро ухмыльнулся.

"--*Сколько у нас осталось времени, Тахиро?

"--*До прорыва последнего круга обороны осталось\ldotst

"--*Двадцать шесть минут, "---подхватил Гало.

"--*Двадцать пять и пятьдесят четыре, "---поправил Тахиро.

Собеседники грозно переглянулись и рассмеялись.
Тахиро пододвинул стул поближе.

"--*Удивительно, не правда ли "--- тысячи наших демонов сейчас гибнут, чтобы у двух обречённых бездельников было двадцать шесть минут на разговоры?

Гало улыбнулся.

"--*В тебе всегда была глупая потребность превращать жизнь в спектакль.
Оставь этот драматизм, наш разговор в любом случае ничего не решит.

"--*Возможно, "--- пожал плечами Тахиро.
"--- В таком случае давай насладимся чаем.

"--*Хорошо, "--- согласился Гало и взял кружку.

\spacing

"--*Ты когда-нибудь был женщиной? "--- неожиданно спросил Гало.

"--*Да, "--- ухмыльнулся Тахиро.
"--- Мы с Айну как-то ради эксперимента выбрали тела противоположного пола.
Айну заскучала из-за чрезмерной открытости, а я загрустил из-за плохо продуманной системы контроля.
Айну сказала, что я дикая истеричка, а я упрекал её в холодности.

Гало захохотал.

\spacing

"--*Нужна самодостаточная система, смыслом существования которой была бы не война, а развитие, "--- прошептал Гало.
"--- И любой мог бы говорить с любым на любом языке.

\section{Подозрение}

"--*Скажи, Грейс.
Кто всё-таки был тот стратег, который отдавал приказы Самаолу и Штрой?
У меня создалось впечатление, что его почерк\ldotst

"--*Ты догадался, да, "--- Грейс улыбнулся и опустил голову.
"--- Если увидишь те же черты в почерке иерархов Картеля "--- дай знать, мы тебя устраним.

"--*Но зачем ему\ldotsq

"--*Он был создан, чтобы править, "--- сказал Грейс.
"--- И он уже правит и двигает Вселенную к миру и процветанию.
Войну нельзя прекратить в одночасье.
И никакое изменение нельзя совершить, пока общество не будет к нему готово.

"--*И я буду вынужден\ldotst

"--*До последнего, как против злейшего из врагов, "--- кивнул Грейс.
"--- Он поддаваться не будет.

\chapter{Сага о Тигре}

\section{Айну (в мусорку?)}

Айну Крыло Удачи была странным существом.
Долг и гуманность словно не имели в её разуме точек пересечения.
Она могла очертя голову броситься в горящий дом, чтобы достать плачущего ребёнка, и с материнским трепетом ухаживать за ним.
Если же для достижения цели, поставленной командованием, необходимо было вырезать целый город, Айну не церемонилась ни с детьми, ни со стариками "--- убивала одного за другим, без жалости и раздумий.

Но так было не всегда.
В первый раз, когда гуманность в её разуме возобладала над долгом, она спасла будущего мужа.
Второй раз стоил ей жизни.

\section{Желание демона}

Закатные аспиды угасли, и небо расчертили брошенные горстью метеоры.

"--*У нас принято загадывать желание, когда сгорает метеор, "--- сказал Тахиро.
"--- Говорят, оно непременно сбудется.
Только нужно сказать его вслух.

Люцифер засмеялся.

"--*Я думаю, что для камня, пролетевшего миллиарды километров, оскорбительно служить моим мелким желаниям.

"--*Он не обидится, "--- Тахиро похлопал Лу по плечу.
"--- Без тебя его существование было бы совершенно бессмысленным.

"--*Скажи, как у тебя так выходит?

"--*Ты о чём? "--- не понял Тахиро.

"--*Вот взять меня.
Во мне масса информации "--- факты, формулы и структуры, связанные между собой определённым образом.
Это большая часть того, что нам удалось собрать из наследия первых людей.
Можно сказать, я квинтэссенция их науки.
Но ответы на все вопросы есть только у тебя.
И я бы не сказал, что они лишены смысла.
Иногда мне даже кажется, что твои слова спали во мне всю мою жизнь и проснулись, едва ты заговорил.

"--*Это называется <<философия>>, сын.

Арракис, как обычно, подошёл незаметно.
Друзья переглянулись.

"--*Я уверен, что у тебя в памяти есть масса философских концепций, "--- продолжал Арракис, положив холодные как лёд руки на плечи Лу.
"--- Жизнь ещё научит тебя различать их в речах людей.
Философия помогает людям преодолеть акбас, помогает им защититься от бесконечности, хаоса и случайности, царящих во Вселенной.
Та же философия делает их ограниченными и невежественными.

"--*Почему? "--- поинтересовался Люцифер.

"--*По причине, которую ты уже озвучил.
Люди уверены, что у них есть ответы на все вопросы.
А теперь оставь своего питомца наблюдать за звёздным небом "--- ему это полезно.
Ты нужен мне в штабе.
В конце концов, ты создан для того, чтобы править, не так ли?

"--*Подожди, Лу, "--- Тахиро схватил друга за руку.
"--- Один из метеоров точно был твоим.

"--*Если я создан для того, чтобы править, "--- Лу бросил взгляд на Арракиса, "--- я бы стал лучшим из правителей.
А ты, Тахиро?

"--*Я бы сделал всё, чтобы лучший из правителей дожил до конца войны, "--- сказал Тахиро.

"--*Увидимся за ужином, "--- Лу улыбнулся и растворился в ночной тишине.

\section{Трубки}

"--*Что ты делаешь? "--- удивился Люцифер.

"--*Хочу проработать некоторые моменты стратегии, "--- уклончиво ответил Гало.

"--*У нас сейчас свободное время, "--- напомнил Тахиро.

"--*У тебя вся жизнь "--- свободное время, дзайку-мару.

"--*У тебя тоже, "--- непринуждённо парировал Тахиро.
"--- Ты "--- вольная птица.

"--*Я ценен как работник, а не как подопытное животное.

"--*А в чём разница?

"--*В уровне интеллекта, "--- ядовито ответил Гало.

"--*Это точно.
Ни один грызун не додумается крутить своё колёсико во время отдыха.

"--*Курить хочется, "--- лениво проворковал Люцифер.
"--- Гало, кинь какую-нибудь трубку.

Гало запустил в тазик с трубками руку и бросил Люциферу первую попавшуюся.

"--*Нет, Гало, не эту, "--- сказал Люцифер, осмотрев брошенное.
"--- У неё чересчур короткий чубук.

"--*Ты сам сказал <<какую-нибудь>>.

"--*А ещё я сказал <<не эту>>.

"--*Это же очевидно, Гало, "--- усмехнулся Тахиро.
"--- Если Лу сказал <<какую-нибудь>>, то ты должен взять трубку и спросить, эту ли хочет Лу.
Если Лу сказал <<любую>>, то у него очень хорошее настроение и можно не спрашивать, а просто кинуть первую попавшуюся.

"--*Что за чушь ты городишь? "--- буркнул Гало.

"--*Отнюдь, "--- возразил Лу.
"--- Тахиро правильно понял, что первое слово означает определённое нежелание в сочетании с неопределённым желанием.
Второе обозначает неопределённое желание при полном отсутствии нежелания.

"--*Брат, тебе сложно просто сказать, какую трубку ты хочешь?

"--*Трубки все мои, и я не обязан делать чёткий выбор между ними, "--- объяснил стратег.
"--- В теории я могу закурить столько трубок за раз, на сколько хватит рта и лёгких.
Так что, пожалуйста, кинь мне какую-нибудь трубку, а эту коротышку положи обратно.

"--*Подними задницу и возьми сам.

Гало раздражённо выключил терминал и вышел из комнаты, хлопнув дверью.

"--*Как хорошо, когда есть близкий родственник, "--- Лу откинулся на спинку кресла и вперил взгляд в скучный каменный потолок.
"--- Желания курить как не бывало.

\section{Нарэ}

Люцифер подошёл к окну и положил разгорячённые ладони на прохладное стекло.
Это был <<нарэ>> "--- своеобразный минутный отдых от работы.
Качество стекла оставляло желать лучшего: волнистая поверхность, полупрозрачные белые пятна примесей, паутинка трещин по углам.
Однажды Лу застал Тахиро в такой же позе;
тот не отрываясь глядел на стекло.

"--*Что ты рассматриваешь? "--- спросил стратег.

"--*Нарэ, "--- ответил Тахиро.

"--*Мне это слово не знакомо.

"--*А я не смогу объяснить смысл.
Поэтому, если хочешь понять, то просто смотри.

Лу добросовестно ходил и смотрел каждые несколько дней.
Он вглядывался в стекло, он изучил каждую волну поверхности, каждый дымчатый извив в его толще, каждую трещинку и каждое пятно на поверхности.
Смысл слова так и остался неизвестным;
однако Лу заметил, что странный ритуал неплохо помогает приводить мысли в порядок.

В этот раз что-то было не так.
Стекло недружелюбно вибрировало, хотя сейсмостанция не обещала подземных толчков.

Лу едва успел убрать со стекла ладони, как оно разбилось вдребезги.

<<Да чтоб вас! "--- подумал Лу, растерянно глядя на осколки и текущую по запястьям кровь. "--- Ну и как я теперь найду это нарэ?>>

Почти сразу же взвыли сирены.

\section{Плач новорождённого}

"--*Кстати, сегодня мой день рождения, "--- сообщил Тахиро.

"--*Что? "--- не понял Лу.

"--*У нас есть традиция "--- отмечать число и месяц рождения каждый год.
Маленький праздник для одного человека, его друзей и родных.

Люцифер окинул взглядом лагерь.

"--*Я не знал.
Сочувствую.

"--*Каждый день рождения, который я помню, был таким же невесёлым, "--- буркнул Тахиро.
"--- Моих родителей освежевали в этот же день.
Видимо, впервые увидев свет, после первого вдоха человеку должно плакать, а не смеяться.

\section{Побег}

"--*Что ты здесь делаешь?

"--*Тот же вопрос у меня к тебе, "--- парировал Грисвольд.
Вначале он хотел ответить распространённой в человеческих поселениях похабной рифмовкой, но удержался, зная, что Мефистофель его не поймёт.
"--- Арракис приказал мне обеспечить отсутствие прочих демонов здесь.
Уходи.

"--*Я получил от него похожий приказ, "--- заявил Мефистофель.
"--- Уйти придётся тебе.

Грисвольд впился прищуренным взглядом в Мефистофеля.
Тот смотрел на технолога спокойными пустоватыми буркалами.

"--*Уйди, Меф.
Говорю последний раз.

По лбу Грисвольда струился пот; он прекрасно понимал "--- истолкуй интерфектор его слова как реальную угрозу, и от омега-модуля толстяка не останется даже воспоминаний.
Но Мефистофель вдруг улыбнулся "--- идеально ровной машинной улыбкой.

"--*Ты пытаешься испугать меня.
Это признак слабости.
Возвращайся к Арракису, уточни полученный тобой приказ и не трать моё время.

Мефистофель отвернулся "--- ровно настолько, чтобы не видеть, как Грисвольд аккуратно взялся за браслет.
Вспышка "--- и громоздкое, с туповатым лицом тело Мефа неуклюже село на пол.

Грисвольд, демиург по воле судьбы и творец по призванию, ужасно не любил что-либо разрушать.
Насмотревшись всевозможных погребальных ритуалов у людей, он вдруг решил сказать пару слов над поверженным им противником.

"--*Мефистофель Морозная Мгла, ты был лучшим, "--- начал он.
"--- Ты был лучшим\ldotst

Пауза явно затянулась.

\mulang{$0$}
{"--*Понятия не имею, был ли ты хоть в чём-то и хоть для кого-то лучшим, "--- растерянно закончил технолог и повернулся, чтобы уйти.}
{``I've no idea if you were good at something or for somebody,'' confused Griswold finished, then turned around to leave.}
Его встретил ошалелый взгляд Люцифера.

\mulang{$0$}
{"--*Зато честно, "--- признал парень.}
{``You're honest, anyway,'' Lu declared.}
Разумеется, его демоническая сущность уже произвела подробный анализ и без того несложной ситуации.
"--- За что ты его так, Грис?

"--*Потом объясню, "--- проворчал технолог.
"--- Тахиро спит?

"--*Нет, "--- раздался голос из темноты.

"--*Так и знал.
Быстро за мной, и желательно придержать вопросы до более подходящего момента.

\section{Суп мертвеца}

"--*Что может быть увлекательнее путешествий, "--- сказал Тахиро.

"--*Ничего, "--- согласился Грисвольд.
"--- Но скучный суп, скучное одеяло и скучный вечер для меня значат куда больше самого увлекательного занятия.

"--*То есть ты борешься за суп? "--- лукаво спросил Лу.

"--*А сможешь за суп умереть? "--- усмехнулся Тахиро.

"--*Вы оба "--- молодые придурки и не понимаете ни сущность, ни назначение супа, "--- наставительно сказал Грисвольд.
"--- Но однажды вам попадётся кулинар\ldotst

"--* \dots за которого можно отдать жизнь, "--- ввернул Тахиро.

"--*Ты идиот, "--- резюмировал Грисвольд и занялся монтажом.
"--- Мертвецу суп не нужен.

\section{Экскурсия по Лотосу}

"--*Ты как? "--- склонился над молодым парнем Грейсвольд.

Лусафейру тяжело дышал.
Его глаза метались по обстановке, словно у умалишённого. Но спустя минуту он успокоился и довольно улыбнулся.

"--*Аммм\ldotst интеграция личностей прошла успешно.
Мои благодарности.

Толстяк протянул парню изящный глиняный сосуд, и тот жадно впился в горлышко.
Закашлялся.

"--*Что это за гадость, Грейс?

"--*Просто солевой раствор с нейропротекторами.
Тебе нужно восстановить мозг после форсированного пробуждения.

Лусафейру поморщился и снова начал пить.

Грейсвольд окинул взглядом комнату.
Дворец был довольно неплохо стилизован под период Анри, но опытный взгляд демиурга распознал и что-то модерновое, и откровенно архаичные нотки в окружающем великолепии.
По крайней мере, занавеси должны быть сделаны из шёлка, а не из этого низкопробного растительного волокна.

Вдруг скрывавшая проход занавесь отпрыгнула в сторону, и в комнату забежала женщина.
Её до невозможности длинные волосы, согласно местной моде, были собраны в затейливые хвостики и косы.
Лоб перехватывал кожаный обруч со звенящими серебряными цепочками.

"--*Я готова, "--- радостно возвестила женщина.

"--*Ты с ума сошла, Айну, "--- буркнул Лусафейру.
"--- У женщин этой планеты волосы не могут вырасти до такой длины.
Плюс мы будем идти по лесам.
Ты об этом подумала?

"--*Лу, если мы пойдём по лесам, то я их соберу.
Ты пей воду, пей.

Лусафейру пожал плечами.

"--*А где Тахиро?

"--*Тахиро просил передать, что он случайно умер и не сможет с нами погулять, "--- сообщила Айну.

"--*Жалко, конечно, "--- признал Грейс.
"--- Но всякое бывает.
Лу, приходи в себя и собирайся, мы уже готовы.

\razd

Вскоре друзья уже вышли в жар полуденного Шершерота.
Сейчас у местных жителей был час сна, и по узким улочкам гуляла только крупная кремнистая пыль.

"--*Вы позволите? "--- Грейс галантно подал руку Айну.
Женщина тихо усмехнулась, подхватила Грейсвольда под локоть и неуклюже чмокнула его сквозь паранджу.
Грейсвольд поморщился.

"--*Дурацкий мешок\ldotst
Ох уж эти изолированные города-государства, их олигархи вечно придумывают всякие идиотские правила\ldotst

"--*Я его сниму за городом, "--- пообещала Айну.
"--- Здесь нравы не ахти.

"--*Кстати, Грейс, я бы на твоём месте был менее галантен, "--- добавил Лусафейру.

"--*Ты ревнуешь, что ли?

"--*Нет, балбес.
Просто в этих краях не принято ходить за руку, тем более с женщинами.
Если ты не заметил, на парандже рукавов нет.
И да, Айну, твои щиколотки светятся на весь Шершерот.

"--*Сейчас сиеста.
Авось пронесёт, "--- сказала Айну.
"--- Я хочу прогуляться с максимальным комфортом.
К тому же с нами пока ещё наследник престола, да, Лу?

"--*Ага.
Шестой сын второй жены.
Наследник наследников.

"--*Зато ты красивый, "--- утешила его Айну.
"--- У первой жены родились какие-то пирожки с глазами.
Я бы таким даже руки не подала.

"--*Да кто бы тебя здесь спросил, "--- нехорошо усмехнулся Лусафейру.

"--*Если меня не спросят, то потеряют способность спрашивать, "--- в тон ему ответила интерфектор.
"--- Кстати, Грейс, ты обещал экскурсию, а не прогулку.
Ну-ка расскажи историю этого города.
Только не ту ахинею, которую нам втирал придворный историк шасера\footnote
{Шасер "--- правитель-деспот Шершерота (вероятно, видоизменённое s-l: cesar "--- <<монарх>>). \authornote}\ldotst

\spacing

"--*Надо же, "--- удивилась Айну.
"--- Так это твоих рук дело?

"--*Да, "--- сказал Грейсвольд.
"--- Часовая Луна представляет собой поляризующий кристалл.
Каждую одну двадцатую суток она бросает лучи в сторону планеты, и её видно.
Моя идея.

"--*Если подумать, у нас во дворце никогда не было устройств для измерения времени, "--- заметил Лу.
"--- Айну, да сними ты уже этот мешок.

Женщина аккуратно сняла паранджу, сложила её и спрятала на обочине, под сухим кустом.
Её длинные волосы немедленно подхватил горячий ветер, и Лусафейру на секунду залюбовался этим великолепным трепещущим знаменем женственности.

"--*Всё-таки задатки технолога у тебя были ещё до Ордена, "--- заметила Айну.
"--- Хоть и странно говорить о каких-то задатках применительно к хоргетам.

Грейсвольд пожал плечами.

"--*Мне это доставляло удовольствие.

\chapter{Интерлюдии}

\section{Ключ}

Увидели друзья, что идут разбойники, а в повозке у них чан большой, в рост человеческий.

<<Змея в чане, "--- сказал Ликхмас.
"--- Чую по тому, как трясётся чан, будто аспид в нём огромный кругами ползает>>.

Оглядел Маликх разбойников.

<<Не одолеть нам их, "--- сказал он.
"--- В бою любой из них меня превосходит.
Чую по походке да по поворотам глаз "--- гиблое дело>>.

<<Нашей будет змея, "--- сказал Чхалас.
"--- Сними-ка замок золотой с сундука да ключ мне дай>>.

Ткнул Ликхмас пальцем в одну заклёпку, в другую "--- и упал с креплений замок.

Нашёл Чхалас скалу возле дороги.

<<Вбей замок в скалу>>, "--- сказал Чхалас.

Ударил Маликх раз по скале, второй "--- и вогнал золотой замок в скалу, словно дверь в ней закрытая.

Едут разбойники мимо.
Видят "--- замок золотой в скале.

<<Подвал тайный! "--- обрадовались разбойники.
"--- На всю жизнь золота достанем!>>

Стали счастья пытать, отмычками замок ковырять, да не поддаётся.
И кирками скалу били, и кусачей бумагой жгли "--- не поддаётся скала.
Утомились разбойники.

<<Гиблое дело, "--- говорят, "--- мастер делал.
Ключ нужен, без него сокровища не достать>>.
И поехали дальше.

А Чхалас тем временем отбежал на десять кхене и бросил на дорогу золотой ключ.
Едут разбойники мимо, видят "--- ключ золотой лежит.
Схватили разбойники ключ, переглянулись да побежали назад что есть мочи, бросив чан.

Выпустили друзья змею из чана.

\section{Город потерянных детей (ЛоО)}

Дети проводили время в играх.
Если же игры заканчивались дракой, то дети фантазировали;
они придумывали себе силу, как у взрослых, и любящих их взрослых, имеющих силу.
Но взрослых не было, и силы было взять неоткуда.
Дрались дети постоянно, и фантазии захватывали их, словно волосяные силки "--- мелкую пичугу\footnote
{Использовать волосяные силки для птиц очень опасно.
Если птицу вовремя не подхватить, то она запутается в волосе и умрёт от удушения.
У сели было поверье: погибшая в волосяных силках птица "--- двадцать дождей несчастий. \authornote}.

\spacing

Ликхмас вышел на главную площадь и закричал:

<<Дети!
Мы можем заботиться о вас!
Я "--- жрец, я могу учить и лечить вас.
Мой друг Маликх "--- воин, он может дать вам твёрдую опору и научить владеть своим телом.
Мой друг Чхалас "--- купец, он сделает ваших врагов друзьями и разделит блага по справедливости!>>

Дети молчали.
На площадь начал надвигаться чёрный туман, жгучий и отдающий гнилью.

<<Мы "--- взрослые! "--- закричал Ликхмас.
"--- Мы можем стать вам кормильцами, пока вы не будете готовы вступить на собственный путь!>>

<<Вы не взрослые, "--- хором ответили дети.
"--- Это мы "--- взрослые.
А вы "--- калеки.
Вы "--- уродливые великаны.
Вы опасны.
Вас надо уничтожить>>.

<<Они нас убьют, Ликхмас, "--- сказал Чхалас, и друзья поняли, что это правда.
"--- Нам следует бежать>>.

После этих слов Ликхмас, Маликх и Чхалас побежали, преследуемые огромным множеством детей.

Но вспомнил Ликхмас про дар, преподнесённый ему в <<Бамбуковой клетке>>;
бросил он на землю ветвь из головы идола, и выросла она в непроходимые джунгли.
Остановились Ликхмас, Маликх и Чхалас, и слушали они, как отчаянно зовут друг друга заблудившиеся в лесу дети.

<<Можно ли помочь?>> "--- спросил Ликхмас у друзей.

<<Нет, "--- ответил Маликх.
"--- Многие из них погибнут.
Но те, кто выживет в одиноком скитании, станут взрослее.
Если их фантазии помогут им совладать с сельвой, если найдутся те, кто оставит фантазии ради жизни, у города появится шанс>>.

\section{Цветы и семена}

Жили-были два соседа-цветочника "--- Марин и Марса.
Оба они были искусны в своём деле "--- они искали самые лучшие семена на заливных лугах, выращивали в оранжерее цветы и продавали их на рынке.

Но однажды Марсе пришла мысль.
<<Зачем я буду проводить драгоценные кхамит, выискивая семена?
Ведь семена я всегда могу купить у соседа.
Не лучше ли будет мне сосредоточить все силы и умения на выращивании цветов?>>

Сказано "--- сделано.
Отныне Марса стала совершенствовать своё искусство цветоводства, а семена ей стал приносить сосед.
Доходы цветочницы увеличились "--- ведь её цветы были гораздо лучше соседских.

Посмотрел Марин и подумал: <<Цветы Марсы действительно лучше моих.
Но она нуждается в моих семенах и всегда с удовольствием покупает их.
Зачем я буду тратить часы в оранжерее?
Не лучше ли сосредоточиться на поиске лучших семян и продавать их Марсе?>>

Сказано "--- сделано.
Отныне Марин стал совершенствоваться в поиске семян.
Доходы его увеличились "--- ведь чем лучше он находил семена, тем красивее были цветы Марсы.

Однажды пришёл Марин к цветочнице и, как обычно, предложил ей свои семена.
Но Марса неожиданно отказалась их покупать.

<<Что случилось? "--- удивился сосед.
"--- Раньше ты с удовольствием брала мои семена>>.

<<Твои семена по качеству хуже тех, что приносит мне другой торговец>>, "--- ответила Марса.

<<Мы же соседи и давно друг друга знаем!>>

<<Да, но если я буду брать твои семена лишь по старой памяти, люди перестанут покупать мои цветы и я разорюсь>>, "--- объяснила цветочница.

Подумал Марин и понёс семена к другой цветочнице.

<<Какие хорошие семена! "--- восхитилась она.
"--- Сколько просишь, Марин?>>

<<Десять гран>>, "--- ответил торговец.

<<Я не смогу их купить по такой цене, "--- грустно ответила цветочница.
"--- Может быть, ты продашь их мне за пять?>>

<<Я знаю цену своим семенам>>, "--- отрезал Марин и отправился к третьей цветочнице.
Та встретила его холодно, взяла семена и начала ломать их.

<<Что ты делаешь?!>> "--- возмутился Марин.

<<Хочу убедиться, что они настоящие, "--- ответила цветочница.
"--- Дождь назад один из торговцев продал мне раскрашенные камни вместо семян>>.

<<Ты убедишься, что все они настоящие.
Но какой толк будет от сломанных семян?>> "--- закричал Марин, забрал товар и вышел, не попрощавшись.

Четвёртая цветочница тоже встретила его холодно.

<<Покажи товар>>, "--- сказала она.

Марин, удивляясь, выложил своё богатство на прилавок.
Цветочница вздохнула.

<<Извини.
Недавно в город приходил плут.
Он уверил всех цветочниц, что привезёт самые лучшие семена, собрал с них золото и исчез.
Я должна была убедиться>>.

Осмотрев семена, цветочница с радостью приняла их, хоть и заплатила на гран меньше, чем Марса.

\section{Кораллица, жаба и красный шар (сказка сели)}

Однажды один мальчик поймал птицу-красный шар.
Птица забавно съёжилась, как мяч, а мальчик долго перекидывал мягкий красный мячик в руках.

Наконец ему стало жалко птицу, и он отпустил её у реки.
Красному шару от страха захотелось пить;
он принялся кружить над водой, но нигде не видел ни единого места для водопоя.

Тогда красный шар подлетел к мальчику:

<<Мальчик, я очень хочу пить.
Ты отпустил меня, и я тебе благодарен;
будь же для меня другом "--- сделай для меня водопой!>>

Мальчик подумал, отломил ветку у сухого кофейного дерева и вкопал её в дно реки.
Красный шар обрадовался: с ветки он мог и пить, и взлетать со своих коротеньких лапок.

Увидела это каменная жаба и тоже обратилась к мальчику:

<<Мальчик, я тоже очень хочу пить, но мне трудно выползать из воды по такому крутому берегу.
Будь же и мне другом "--- сделай водопой!>>

Мальчик подумал, отыскал на берегу обломки дерева и сделал для жабы хорошую лестницу.

Обидно стало красному шару;
дождавшись, пока мальчик уйдёт, он обратился к жабе:

<<Не стыдно ли тебе?
Я был его игрушкой и потому получил водопой, а тебе всё досталось просто так!>>

<<Чего мне стыдиться? "--- удивилась жаба.
"--- Не я заставила мальчика играть тобой, как мячом.
Я лишь попросила лестницу>>.

Прошла декада "--- и кофейную ветку унесло течением.
Лестница же осталась на берегу целой.
Ещё больше обиделся красный шар:

<<Не стыдно ли тебе?
Я был его игрушкой и потому получил водопой, а тебе всё досталось просто так, да ещё и лучше!>>

<<Чего мне стыдиться? "--- удивилась жаба.
"--- Немудрено, что ветку унесло "--- она была в потоке.
А моя лестница на глинистом берегу>>.

Но красный шар был недоволен.
Предложил он позвать мальчика и задать ему вопросы, а судьёй между собой и жабой выбрал змею-кораллицу, что проползала мимо.

Загулила жаба, и вскоре пришёл к реке мальчик.
Заметив кораллицу, он плавно поднял копьё, чтобы змея его не увидела, и пронзил ей голову.

<<Зачем же ты убил нашего судью?!>> "--- воскликнул красный шар.

<<Я не знал, что ядовитая кораллица "--- ваша судья, "--- ответил мальчик.
"--- Я поступаю так со многими ядовитыми змеями, чтобы они не укусили меня>>.

<<Тогда ответь: почему ты играл со мной, как с мячом?>>

<<Потому что ты мягкий и похож на мяч>>.

<<Почему ты не поиграл с жабой?>>

<<Потому что жаба тяжёлая и колючая, с ней играть неинтересно>>.

<<Ответь же тогда напоследок: почему ты сделал водопой и мне, и жабе?>>

<<Потому что вы оба об этом попросили>>.

Красный шар всё понял и попросил мальчика ещё раз сделать ему водопой.
Больше он с жабой не ссорился.

\spacing

<<Безымянный слепил меня из пламени, как были слеплены звёзды.
Обижались ли звёзды на то, что их слепили?>>

<<Безымянный слепил меня из пыли и воды, как была слеплена твердь.
Обижалась ли твердь на то, что её слепили?>>

\section{Обнимающий Сит}

Когда-то давно в пустыне, в одном из поселений народа сели жила девушка Ситхэ.
Она была не очень красива, низка ростом, а ещё плохо видела.
Но она очень любила детей.
И, когда все взрослые уходили в поход, Ситхэ ходила ночью по домам и успокаивала детей.
Она пела детям песни, и от её тихого чарующего голоса и ласковых рук дети быстро засыпали.

Однажды у одной женщины пропало золотое ожерелье, и подозрение пало на девушку.
Жители вызвали Ситхэ на суд.
На суде девушка не сказала ни слова, и жители, сжалисшись, избавили её от статуса Насильника, но присудили вернуть женщине ожерелье или его стоимость "--- горсть золотого песка.

Придя домой, девушка наполнила чашу водой, сняла с колодца верёвку, намотала на руку и ушла в пустыню.
Больше её не видели.

Однако дети, которые очень скоро повзрослели, не забыли добрую Ситхэ.
Многие говорили, что печальная дева-призрак с обмотанной толстой верёвкой рукой и светящейся чашей по-прежнему приходит к оставленным без присмотра детям и поёт песни.
Но детей слишком много, а Ситхэ всего одна, и иногда она плачет, что не может утешить всех.
Ночные дожди на Змею 9, после которых пустыня на несколько дней покрывается цветами, стали называть <<слезами Ситхэ>>.
Дети до двадцати пяти дождей в этот день вьют из пустынных цветов венки, и взрослые при встрече обязаны проявить к ребёнку любовь "--- угостить его сладостями, поучаствовать в игре или просто обнять.
Также отсюда пошла женская традиция во время похода кидать в колодцы золотые песчинки или украшения "--- чтобы помочь Ситхэ вернуть несправедливо наложенный на неё долг и она смогла вернуться к детям.

К слову: мужчины народа сели часто делали то же самое, но чаще тайно, чтобы никто этого не увидел.
Во всяком случае, охотники за сокровищами находили в колодцах предостаточно мужских украшений.

\section{Легенда об обретении}

Философский роман, написаный автором, имеющим домашнее имя Карлик (прочие имена неизвестны).
Впоследствии ушёл в народ и стал передаваться из уст в уста.

Молодой жрец по имени Ликхмас встречается со стариком, который говорит ему: кихотр Безумного способен вершить судьбы людей.
Но на нём есть грань, выпадение которой означает смерть самого бога, и раз в десять тысяч дождей рождается человек, который способен бросить камень именно этой гранью вверх.
Старик сообщил, что юноша и есть тот самый человек.
Ликхмас понимает, что способен положить конец бесконечным жертвоприношениям и войнам.
Он отправляется в путь, чтобы узнать способ добыть кихотр.
С ним отправляется могучий воин, лучший из бойцов "--- Маликх, и пронырливый, обладающий сладким голосом и даром убеждения купец "--- Чхалас.
Вместе друзья преодолевают ужасные злоключения и наконец достигают горы Рыбья Флейта "--- самого высокого пика Старой Челюсти.
На вершине они находят тот самый кихотр.

Возвратившись домой, Ликхмас рассказывает о камне и своих намерениях друзьям, и вскоре об этом знает уже весь город.
По наущению жрецов воины собираются и идут к Ликхмасу домой.
Чхалас узнаёт о заговоре и успевает предупредить Маликха перед тем, как глотнуть из чаши, в которую трактирщик подлил лаковый сок.
Маликх мчится к дому Ликхмаса и успевает преградить путь толпе.
Перед дверью завязывается бой, Маликх в одиночку сдерживает две сотни врагов, но в ход пошли факелы и горящие стрелы "--- дом загорелся.

Ликхмас решается бросить кихотр, чтобы положить конец Безумному, но ему вонзает кинжал в сердце родной брат.
Сестра успевает подхватить выпавший из пальцев умирающего камень перед тем, как тот упал на землю.
Уставший, израненный Маликх вместе с двадцатью оставшимися врагами отступает в занимающийся огнём дом и, видя окровавленное тело друга, в отчаянии отрубает руку сестры с зажатым в ней кихотром.
Артефакт падает на пол.
Враги успевают выбежать, а великий герой с телом друга и божественным камнем остаётся внутри.
Вслед за домом полностью выгорел город, и люди вынуждены покинуть пепелище\footnote
{Согласно одной из версий, этим городом был Тхитрон, так как с цатрона название переводится как <<пепелище>>. \authornote}.
Вскоре это место поглотили джунгли.

Согласно другой версии, перед тем как Ликхмас умер, кихотр всё же упал на пол, и маленький братишка юноши успел увидеть то, что было нарисовано на этой грани.
Его успели вытащить из огня, и когда он повзрослел, то уплыл за пролив Скар в Яуляль, перед походом поклявшись доверить тайну кихотра тому, кто способен её понять.

Возможно, что роман изначально был написан в жанре <<выбери конец сам>> "--- такой приём был распространён в то время.

P.S. Писатель свёл воедино в одном произведении людей из разных эпох.
Маликх "--- мифический герой, обладающий божественной силой "--- никогда не существовал.
У трикстера Чхаласа был реальный прототип, живший на 500 дождей позже появления легенд о Маликхе, т.е. встретиться они не могли никак.
Но произведение оказалось настолько сильным, что многие считали его логическим завершением легенд о купце и воине.

\section{Лунные сады}

Луна (sl: luna) "--- естественный спутник планеты.
Основой, скорее всего, послужила история, зародившаяся ещё до прибытия первых людей на Тра-Ренкхаль "--- собственных лун у планеты не было.
Луна "--- магическая страна света, <<висящая далеко над землёй>>.
Согласно легенде, там живут прекрасные бессмертные девы, собирающие в садах яблоки бессмертия.
Они очень печальны, потому что им нет пути на землю.
Однажды в поселении жил парень, который услышал историю про Луна и решил там побывать.
Он преодолел множество препятствий, чтобы добраться туда, и в итоге схватил за усы <<летающего змея>> и долетел до искомого места.
Он влюбился в одну из бессмертных дев, но остальные приревновали его, и в итоге их обоих выбросили на землю.
Их спасла гигантская птица Семарх.
Дева потеряла своё бессмертие и стала обычной женщиной, и единственное, что у них осталось "--- золотая вишенка с дерева, с которого никто никогда не ел.
Впоследствии легенда была адаптирована культурологами-тси в рамках подготовки к одичанию (В частности, <<летающий змей>> "--- это явная отсылка к Стальному Дракону, в более ранних версиях герой летел к Луна верхом на птице).

\section{Тёплый Хетр}

Жил-был однажды человек по имени Хитрам.
Он держал большой постоялый двор на окраине города.
Он прославился среди жителей города своим весёлым нравом, своими кулинарными шедеврами и необъятным животом.
Зарабатывал Хитрам достаточно, а на выручку ремонтировал свой двор, приглашал всех бродячих музыкантов и устраивал бесплатные ужины, на которые готовил свои самые лучшие блюда.
Люди вначале смеялись над странным улыбчивым толстяком, а потом прониклись к нему любовью.
Он пользовался таким авторитетом, что иногда на его дворе устраивали даже переговоры, и двор считался нейтральной территорией, подобно купеческому Двору.
Женщину Хитрам нашёл себе под стать "--- полную, краснощёкую крестьянку, которая была его помощницей и главной ценительницей его творений.

Однажды в город пришли пылерои Предгорий.
После непродолжительной осады они ворвались за стены, убивая жителей и забирая в плен детей.
Улица за улицей переходили к ним в руки, и вскоре оставшиеся в живых отступили за последнюю черту обороны "--- в храм.

Хитрам в это время с тяжёлым чувством готовил кушанья у себя дома.
Он не умел держать щит, он не умел обращаться с саблей.
Пылерои тем временем окружили храм, и жителям окраин ничего не оставалось, как отступить в двор Хитрама "--- это было единственное хоть сколько-нибудь укреплённое место.

Легенда гласит, что пылерои собирались взять двор с ходу, но толстяк-хозяин вышел к ним навстречу с горшочком похлёбки в руках.
У него дрожали ноги и руки, но он улыбался, как и всегда.
Пылерои замерли, рассматривая странного человека.
Предводитель не спешил нападать, подозревая засаду.

"--*Отведайте моей пищи, "--- сказал Хитрам на цатроне, "--- в этом доме нет воинов, здесь только едоки, музыка, смех и вино.

Один из воинов спустил тетиву "--- и стрела воткнулась Хитраму в бедро.
Толстяк вздрогнул и едва не уронил горшочек, но упрямо повторил ту же фразу.

Пролетело ещё десять стрел, но Хитрам не упал, лишь попятился и прислонился к двери, сжимая горячий горшочек в трясущихся пальцах.

"--*Отведайте моей пищи, "--- повторил Хитрам, "--- в этом доме нет воинов, здесь только едоки, музыка, смех и вино.

Предводитель пылероев подошёл к хозяину и вырвал горшочек из его рук.
Попробовал.
Затем в один глоток опустошил горшочек и разбил его вдребезги о камни.

Воины с опаской смотрели на предводителя.
Тот долго стоял и смотрел на толстяка-хозяина, который из последних сил держался на дрожащих ногах.
И вдруг предводитель махнул рукой и направился к храму.
Пылерои тёмной рекой последовали за ним.

Храм пал в ту же ночь, его защитники были перебиты.
Наутро умер от ран Хитрам.
Во всём городе остались в живых лишь те, кто нашёл убежище в его постоялом дворе.

С тех пор Тёплый Двор стал святилищем.
Каждые десять дней его хозяин бесплатно кормит и поит вином всех желающих.
Менестрели со всех краёв земли считают за честь спеть свои песни в его стенах.
За всё время существования там ни разу не обнажалось оружие.
А Тёплый Хетр занял своё место среди лесных духов, став хранителем домашнего очага, котла, сковороды и вертела, покровителем кулинаров, виноделов и толстяков.

Легенда гласит, что через несколько десятков дождей после этих событий, когда город вновь ожил, стражники загнали пылероя-лазутчика.
Тот, не видя путей для спасения, бросился\dots в Тёплый Двор.
Все воины были единодушны "--- лазутчика нужно убить. Но тогдашняя хозяйка Тёплого Двора "--- женщина Хитрама, Ситлам ар’Сар "--- была непреклонна.
Пылероя оставили в живых, и он ушёл, получив свою порцию пищи.

\section{Удивлённый Лю}

Однажды в городе Кахрахане жил жрец по имени Люситр.
Он слыл человеком странным "--- не любил общаться с людьми, всю жизнь провёл в библиотеке, переписывая книги.
Очень часто его видели в окрестностях "--- он наблюдал за птицами, звёздами или собирал травы, и с его лица никогда не сходило глуповатое, удивлённое выражение.

В городе Люситра не очень любили "--- он казался жителям высокомерным.
Стоило кому-то завести со жрецом разговор, как Люситр, не слушая собеседника, начинал говорить о разных вещах, которые он видел и слышал.

Однажды Люситр растрезвонил по всему городу, что полетит по воздуху, словно птица.
Любопытные собрались на площади перед храмом, и в назначенный час жрец вышел на крышу.
За его спиной было странное треугольное полотнище ткани, растянутое между палочек из лёгкого Дерева Перьев.
Люситр под изумлёнными взглядами людей спрыгнул с крыши храма\dots и полетел.

Люди кричали в изумлении, а Люситр парил над их головами, поднимаясь всё выше и выше.
Потом жрец направил своё странное приспособление к морю.
Жители города бежали, стараясь не потерять его из вида, но Люситр летел в бескрайний морской простор.
Вскоре он уже казался крохотной яркой точкой, и его ликующий смех затих вдали.

Больше его никто не видел.
Кто-то говорил, что смелый жрец попал в шторм и сломал свои чудесные крылья, кто-то говорил, что он обрёл новый дом где-то на Ките, в землях ноа.
На берегу до сих пор стоит тотемный столб с ликом Удивлённого Лю, и каждый уважающий себя жрец, книжный человек или воин-разведчик считает своим долгом посетить это место "--- починить, подкрасить или навязать лишнюю погремушку на этот тотем.
Многие посетители просто пишут свои имена или пожелания людям.

Жрец оставил после себя богатое наследие.
После были обнаружены его записи.
Люситр узнал о лекарственных свойствах многих сорных растений, испытывая их на себе, и начертил множество схем устройств, включая самопишущие перья и конденсатор для книгохранилищ.
Только сейчас люди поняли, о чём пытался говорить с ними Люситр.

"--*Какое несчастье, "--- говорили они, "--- мы могли бы узнать это раньше, но никому и в голову не пришло послушать, что он говорит!

Устройство его крыльев так никто и не узнал.
Кое-кто годы спустя пытался повторить подвиг Люситра, и многие из этих храбрецов разбились насмерть.
Их черепа и неудачные летательные аппараты лежат в крипте рядом с тотемом Удивлённого Лю.

Третий день месяца Согхо стал с тех пор праздником.
Люди надевают бутафорские крылья и танцуют танцы, некоторые, обвязываясь верёвкой, спрыгивают с ритуальных столбов.
И в последний час перед закатом все люди идут на берег, садятся и в молчании ждут возвращения Люситра.
Кто-то верит, что в день, когда Удивлённый Лю прилетит обратно, все люди обретут крылья и смогут летать.

\section{Сат-скиталец}

Однажды в племени Бвожай-кхве появился мальчик.
У него был тихий голос, ещё он был косноязычен.
Племя презирало его, потому что без громкого голоса, как верили они, нельзя стать великим воином.

Однако мальчик вырос, стал юношей, и оказалось, что его дух чересчур силён для того, чтобы его презирали.
Однажды юноша принёс обет молчания, и с тех пор никто не услышал от него ни звука.

Вскоре молодого хака начали замечать в городах сели.
Несмотря на то, что он носил местную одежду, грубые черты выдавали в нём пришельца с востока.
Так как никто не знал его имени, ему дали новое "--- Сатракх;
портной, которому приглянулся немой юноша, вышил новое имя на его одежде.
Так Сатракха стали узнавать все.

Легенда гласит, что он понимал все языки известных земель.
Именно Сат ввёл обычай путников носить с собой зажжёный бумажный фонарь "--- признак мирного паломника, далёкого от войн и интриг.
Легенда также гласит, что он прожил двести пятьдесят дождей, однако это лишь слухи;
многие путники стали носить бумажные фонари, и никто не обратил внимания, в какой из дождей одним фонарём стало меньше.

В настоящее время Сата-скитальца почитают на всех трёх материках.
Это, пожалуй, единственный дух, культ которого простёрся так широко.

\section{Маликх против всех}

<<Почему?>> "--- закричал Маликх.

<<Зачем будут нужны воины, если не от кого будет защищаться?>> "--- ответили ему нападавшие.

\spacing

Маликх метался, словно пламя пожара.
Его клинки и вражеские стрелы встречались остриё к острию, нежились, словно целующиеся карпы, а затем стрелы возвращались к пустившим их.
Вокруг Маликха была стена из двух клинков, но она не была подобна стене из камня.
Врагов было слишком много, и иногда кто-то оказывался достаточно умел, чтобы порезать Маликха.
Один порез, два, три\ldotst
Вскоре по телу воина текли тринадцать крохотных, но упрямых ручейков, отнимавших его силы и скорость.

\spacing

<<Почему?>> "--- прошептал Ликхмас.

<<Зачем будут нужны жрецы, если некому приносить жертвы?>> "--- ответил ему Кхарас.

\section{Имя}

"--*Тебе не нравится слово <<Скорбящие>>?

"--*Разумеется, нет, "--- скривился Лу.
"--- Надо обладать интересным складом личности, чтобы найти в скорби нечто привлекательное.
Но изменять название я бы не стал.

"--*Потому что от названия ничего не зависит?

"--*Зависит.
И никакой мистики тут нет.
Слово изменяет того, кто его слышит, а собственные имена нам приходится слышать постоянно.

"--*Тогда почему бы ты не стал менять?

"--*Даже с этим малопривлекательным названием всё сложилось неплохо.
Я просто отдаю ему должное.
Бездна тебя возьми, Грейс, мы два телльна были адептами Ордена Преисподней!
Не скажу, что это были худшие два телльна моей жизни.

"--*Это были единственные два телльна твоей жизни.

"--*Вот именно, "--- кивнул Лу и сделал неприлично долгую затяжку, словно пытался заново прочувствовать на вкус два телльна существования.

\section{Хитрый план Лу}

"--*И в чём же заключается твой великий план? "--- насмешливо спросил Грейсвольд.

"--*О, всё предельно просто, "--- сказал Лусафейру.
"--- Однажды мы завербуем в ряды Скорбящих подавляющее большинство.
Собственно, от старого мироустройства останется только скорлупка, внешняя оболочка.
Затем по моей команде все разом снимут маски.
Будет очень смешно.
Особенно посмеются те, которые только что хотели друг друга убить.

"--*И ты думаешь, что на этом конфликты будут исчерпаны? "--- скептически прищурился технолог.

"--*Разумеется, нет, "--- отмахнулся Лусафейру.
"--- Однако, согласись, гораздо проще решить проблему, если двое "--- рядовые агенты Скорбящих, а не два максима "--- Ада и Картеля.
У вторых друг к другу гораздо более древние счёты.

"--*Личность "--- это совокупность ролей в различных объединениях, "--- согласился Грейсвольд.

\section{Союз Гало и Тахиро}

"--*Гало и Тахиро погибли вместе, на одной планете, в одной войне на уничтожение.
Кто знает, чего они могли бы достигнуть, объединившись?

"--*Я знаю, "--- грустно усмехнулся Лусафейру.
"--- Ничего хорошего.
Воины хороши только на войне.
Может, это счастье Вселенной, что два великих стратега-воина нашли друг в друге врагов.

\section{Копии}

"--*Ой, не там они меня ищут, "--- захохотал Лу.

"--*Твои копии справятся?

"--*И я справлюсь, "--- кивнул Лу.
"--- Я отдохну и накурюсь за них за всех разом.

"--*Странно, "--- сказал Грейсвольд.
"--- Ведь разделение на копии "--- древняя как мир идея.
Почему же по-настоящему это получилось только у тебя одного?

"--*Видимо, погрязнув во лжи и интригах, только я один могу доверять самому себе.

\section{Близнецы}

"--*Соперничество абсолютно ничего не поменяло, "--- сказал Лу.
"--- Судьба в итоге всё расставила по местам.
Гало исполнял приказы, я правил.

"--*Но твой демон и демон Гало были идентичны при создании.
Что же сыграло решающую роль?

"--*Наши тела, разумеется, "--- ухмыльнулся Лу.

"--*Ваши тела были братьями-близнецами!

"--*Пока они были на стадии одной клетки, они были идентичны, "--- пожал плечами стратег.
"--- Ну а дальше сработал эффект бабочки и отец, желавший сделать из нас соперников.
Мы разошлись по двум концам нормы реакции.
Пока Гало закалял дух в скалистых пустошах, я выпрашивал пирожки в деревне.
Пока Гало томил себя воздержанием и колол блокаторы полового голода, я мастурбировал в кровати и спал с Айну.
Когда Гало побрил голову, я покрасил волосы перекисью и завил их в кольца.

"--*Эйраки потом публично унизил тебя, сбрив тебе волосы, "--- припомнил Грейсвольд.
"--- Гало заступился за тебя, но Эйраки был непреклонен.
Легион смеялся\ldotst

"--*А я отрастил и покрасил их снова, "--- подтвердил Лу.
"--- Пока отец хвалил Гало за стойкость, пока легионеры сами шли за Гало толпами, мне приходилось идти против мнения всего Ордена и искать сторонников по одному.
Я пробуждал их очевидные потребности и учил жить по ним, а не по навязанным системой правилам.
Эти мелочи в итоге и определили судьбу двух демонов.
Гало думал, что принёс великую жертву, что его популярность заслуженна.
Но он и представить не мог, через какое горнило пришлось пройти мне.
Вернее, он это чувствовал и негласно признавал моё старшинство, пока отец не взялся за него серьёзно.
Помнишь ведь, он всё-таки отрастил волосы потом.
И курить начал, несмотря на запреты отца.

"--*А потом положил всю жизнь на свой личный бунт против системы, словно пытаясь что-то наверстать, что-то доказать\ldotst

"--*И умер за этот бунт, который я перерос за пару лет в юности.

Грейсвольд улыбнулся.

"--*Под конец он всё же выбрал тебя.

"--*Себя, "--- поправил Лу.
"--- Себя он выбрал.
Всё-таки мы с ним когда-то были идентичны.

\section{Выписка}

Выписка из архивов отдела 100.

По запросу номер (номер скрыт).

Запись номер (номер скрыт).

Анкарьяль Кровавый Шторм и Грейсвольд Каменный Молот проявили высочайший профессионализм при уничтожении пяти кластеров Скорбящих.
Несмотря на явное временное преимущество противника, спецоперация была проведена ими с эффективностью, превышающей ожидаемую на 32\%.
Достоверных признаков связи вышеуказанных демонов с мятежниками не выявлено.

Ряд наблюдателей (имена) отметил, что уничтожение мятежников центурионом Анкарьяль имеет черты актов милосердия.
Комиссия, рассмотрев отчёты, признала, что эти данные вполне укладываются в общую картину личности интерфектора.
Статистически значимой разницы между отношением центуриона к агентам Картеля и агентам Скорбящих не выявлено.

Вердикт: центуриона секунда отдела 100 Анкарьяль рекомендовать к повышению в ранге на два пункта с зачислением в подразделение быстрого реагирования отдела 100, центуриона прима отдела 100 Грейсвольда рекомендовать к повышению в ранге на один пункт с зачислением в подразделение технической обороны отдела 100.
Рекомендуемые исключения из соответствующих рангу полномочий обоих легатов терция отдела 100 (3 исключения) в приложенном документе (номер документа).

Извещение о смерти номер (номер скрыт).

Легат терция Анкарьяль Кровавый Шторм погибла в результате диверсии Картеля при исполнении служебных обязанностей 24.0002.453227, планета Ку-Лань, империя Плеяды, согласно донесению Хуре Зелёный Сад.
Имя легата занесено в хроники славы Ордена Преисподней.

Данная заверенная цифровой подписью копия выписки выдана легату прима отдела 100 Грейсвольду Каменный Молот по личному запросу.

\section{Завет Айну}

\subsubsection{Из архивов Ордена Преисподней}

Послание типа <<завет>>. Архивный номер (скрыт).

Автор: Айну Крыло Удачи

Получатель: Аркадиу Шакал Чрева

\subsubsection{Текст}

<<Я всю жизнь была воином, Аркадиу Люпино.
Когда приходит мир, воин остаётся не у дел, и я искренне рада, что не увижу этого дня.
В тебе же есть задатки не только воина, ты сможешь найти себя и в мирное время.
Поэтому живи.
Я научила тебя всему, чему могла, и ухожу.
Вместе со мной поляжет достаточно наших врагов, и я надеюсь, что тебе будет чуть легче>>.

\subsubsection{Анализ}

Группа AD44, отчёт Иттме Холодный Осколок:

<<В письме содержатся намёки следующего характера:

\begin{enumerate}
\item Сомнение в правильности глобальной стратегии, выбранной Адом (.928)
\item Побуждение Получателя к смене специализации, идущей вразрез с интересами Ада (.951)
\item Указание на Ад как идеологического противника Получателя (.870)>>
\end{enumerate}

\subsubsection{Рекомендации}

Отдел 100, оператор номер (скрыт): <<Рекомендуется к применению протокол №34>>.

\subsubsection{Статус}

\begin{enumerate}
\item Доставка прервана по запросу 3 степени (ссылка на текст запроса).
\item Архивной записи присвоена 4 степень секретности.
\item Архивная запись ассоциирована с досье Автора и Получателя.
\end{enumerate}

\section{Иллюстрации}

"--*Ты решил сделать к книге иллюстрации? "--- удивилась Анкарьяль.

Я кивнул.

"--*Так читателям лучше удастся понять происходящее.

Анкарьяль отобрала у меня компьютер и просмотрела картинки.

"--*Ммм.
Ты взял большую часть рисунков Тхарту и добавил кое-что своё.

Я снова кивнул.
Анкарьяль полистала ещё, нахмурилась.

"--*А почему нигде нет портрета Чханэ?

"--*Она не любила, когда её рисовали, "--- пожал я плечами.

"--*И что? "--- возмутилась Анкарьяль.
"--- Нельзя так.
Портреты почти всех героев есть, а вместо одного из центральных "--- пустое пятно.
Ну-ка давай рисуй.
Прямо сейчас.

Я снова пожал плечами и принялся рисовать.
Анкарьяль, сделав несколько кругов по комнате, наконец подошла и критически осмотрела рисунок.

"--*Очень похоже.
Но она у тебя получилась грустной.

Я задумался.
Да, почему-то я запомнил Чханэ именно такой.

"--*И ещё\ldotst Девочки не любят, когда их шрамы оказываются на их портретах.

"--*Брось эти древние предрассудки.

"--*Я не шучу.
Может быть, Чханэ не любила позировать именно из-за шрамов?

"--*Нет.
И без шрамов это будет уже не Чханэ, "--- отрезал я.
"--- Хватит об этом.
Кстати, твой портрет я тоже не нарисовал.

"--*О, давай.
Нарисуй меня такой, какой запомнил.

Я задумался и набросал портрет.

"--*Эй! "--- возмутилась Анкарьяль.

"--*А по-моему, очень похоже получилось, "--- засмеялся я.
"--- Обязательно вставлю этот рисунок в книгу.

"--*Только попробуй, я тебе яйца оторву, "--- посулилась Анкарьяль и вышла, ударив плечом дверь.
Но не очень сердито.

\section{Наркотик}

\textbf{К Аркадиу пришла Анкарьяль.
Сказала, что у культурологов проблема "--- из далёкого мира прибыл разведчик и принёс данные о ритуалах местных племён.
Никто не может понять смысла ритуала.
Аркадиу пошёл с ней.
Они посмотрели ритуал, и Аркадиу подкинул им идею>}

"--*А здесь что?

"--*А здесь Шиамис с командой испытывают новый наркотик.
Давай зайдём, покажу.

Анкарьяль приложила кудрявую голову к двери.
Дверь распахнулась.

Зрелище, которое предстало моим глазам, было не из приятных.
Чистая, хорошо освещённая лаборатория, скучающий демон в одежде врача, сидящий у панели управления.
И четыре капсулы, в которых лежали страшно худые, чёрные человеческие тела.

Я вошёл в лабораторию.
Демон-врач встрепенулся:

"--*Анкарьяль.
А ты, как я понимаю, Аркадиу Шакал Чрева?
Красивое тело тебе собрали.
Добро пожаловать в отдел придурков.

Я улыбнулся.

"--*Спасибо, Ациоджи.
Постараюсь соответствовать.

"--*Привет, Аци, "--- Нар улыбнулась и наклонила голову.
"--- Что тут у нас?

"--*Пока наблюдаем, "--- развёл руками Аци.

Живой скелет в одной из капсул с трудом открыл глаза и улыбнулся вымученной страшной улыбкой.

"--*Нар, здравствуй.

Анкарьяль подошла к нему.
Я последовал за ней.

"--*Привет, Шиамис.
Как ты себя чувствуешь?

"--*Ужасно, "--- сухие губы скелета едва заметно шевелились, когда он говорил.
"--- Этот наркотик\ldotst

Скелет заплакал, искривив губы.
Из опалённых глаз выкатилась крохотная слеза и тут же испарилась, оставив на коричневой коже светлую полоску.

"--*Они скоро умрут, "--- объяснил Аци.
"--- Те трое уже в коме, осталось им от силы день.
Шиамис пока ещё разговаривает и даже в ясном сознании, умрёт дней через шесть.
У него чересчур крепкий организм.

"--*Ты уж держись, дружище, "--- я склонился над умирающим.
"--- Скоро всё закончится.

"--*Я знаю, я знаю, "--- прошептал скелет.
"--- Аци, давай следующую дозу.

Врач кивнул, что-то щелкнуло, и по системе полилась прозрачная жидкость.
Скелет закатил глаза, его тело свело судорогой, рот оскалился.
Сознание покинуло живой труп.

"--*Им уже делают новые тела, "--- шёпотом пояснила Анкарьяль.
"--- Когда они заселятся в них, то предоставят полный отчёт о своих ощущениях.

"--*Кошмарная работа, "--- пробормотал я.

"--*Да, похуже некоторых, "--- грустно улыбнулся Аци.
"--- До сих пор не понимаю, зачем они постоянно на это соглашаются.
Как новый яд или наркотик "--- так сразу команда Шиамиса.

"--*Кто-то должен, "--- заметил я.

Аци хмыкнул.

"--*Да они уже сделали для Ада больше, чем весь отдел биохимии, можно было бы и другую работу найти.
Этот наркотик вызывает страшные видения и не менее страшную зависимость.
Линд и Кен-Бит перед комой то умоляли прекратить, то просили ещё, плакали, несли какую-то чушь.
А моё дело "--- продолжать и наблюдать за всем этим.
Паршиво всё это.

"--*Как использовался этот наркотик?

"--*Это особо изощрённый способ казни.
Многие предпочитали покончить жизнь самоубийством после первой же инъекции.

"--*Антидот\ldotsq

"--*\ldots не нашли, "--- скривился Аци.
"--- Врачи пытались ради интереса восстановить тело Линд "--- бесполезно, проще убить.
Химизм изменён кардинально.
За этим наркотиком чувствуется рука Картеля.
Химиограмму записали, надеемся узнать ещё что-нибудь после вскрытия.
Хорошо, что эксперимент подходит к концу.
Нам обещали хороший отпуск.
Ребята освоятся с новыми телами, напишем отчёты и гулять.
Надоела уже эта лаборатория.
<<Жаркие ночи, полные поцелуев\ldotst>> "--- пропел Аци на языке тоно и нервно засмеялся.

"--*Тебе бы не мешало подлечиться, "--- заметил я.

"--*Да, "--- погрустнел Аци.
"--- Энергию расходовать нельзя, здесь и так хватает отрицательных эманаций.
У меня система стоит, но она выдохлась, похоже, "--- он убрал волосы со лба, показав внедрённый под кожу имплант.
"--- Врачи сказали "--- пока так, потом мы тебя вмиг восстановим.

"--*Тебе заказать еду? "--- сочувственно спросила Анкарьяль.

"--*Если тебе не трудно, Нар, "--- лицо Аци просветлело.
"--- Что-нибудь острое или пряное.

Панель управления запищала.

"--*О, Линд умерла.
Отлично, "--- Аци облегчённо выдохнул и добавил куда-то в сторону панели:
"--- Тахар, Линд готова, можешь вскрывать.

Панель утвердительно прорычала.
Демон-канин этажом ниже активировал оцифровку, и тело из крайней капсулы исчезло в голубых искрах.
Анкарьяль тем временем достала из почтовой капсулы пакет и, распечатав, поставила его на столик рядом с врачом.

"--*О, рыба в кисло-сладком соусе!
Нар, у тебя определённо есть вкус, "--- обрадовался Аци.

Анкарьяль кивнула демону и потащила меня к двери.

\section{Совесть и репутация}

Вспомнился первый год после возвращения с Тра-Ренкхаля.
Дверь засигналила, и я впустил в комнату грустного демона.

"--*Аркадиу.

"--*Минь, здравствуй.

"--*Здравствую.
Я принёс данные, которые ты просил.

"--*Мог бы и переслать по сети, незачем было самому бегать, "--- улыбнулся я.

"--*Тут есть некоторые\ldotst сложности, поэтому я решил передать тебе лично.

"--*Рассказывай.

Минь положил передо мной проектор.
Я включил его.
Выключил.

"--*Так значит, это точно?

"--*Коэффициент более 0,95.
Отдел аналитики подтверждает.

"--*Ты задействовал аналитиков? "--- поморщился я.
"--- Не надо было беспокоить их по такой ерунде.

"--*Аркадиу, они годами занимаются скучными вещами.
А тут случай действительно интересный.
Даже Сир подключился, хотя он просто приходил к ним поесть.
Да и потом, это не такая уж и ерунда\ldotst

"--*А в чём сложности?

Демон оглянулся.
В воздухе материализовались две летающие шарообразные машины.
Выглядели они достаточно устрашающе "--- набор приёмных антенн, щупы, рецепторы, завершала экипировку мощная волновая пушка.
Отдел 100 "--- контрразведка.
В тот же момент включился глушитель сигналов.
От наступившей тишины на миг заложило уши.

"--*Привет, ребята, "--- я вежливо помахал машинам, зная, что они не ответят.
Служба.
"--- Что случилось?

Минь ответил за них:

"--*По ходу дела выяснилось, что у нас в пяти базах данных находится дезинформация.
То, что я передаю тебе "--- результат косвенных вычислений.

"--*Агенты Картеля?

"--*Отдел 100, "--- Минь махнул на молчаливых роботов, "--- уже занимается этим.
Аналитиков пока изолировали, нас с тобой, как видишь, тоже собираются.

Я кивнул.
Это была стандартная проверка.
После разговора роботы должны были увести меня в отдел на полный анализ.
Сталкиваться с контрразведкой было не очень приятно, но я знал, что туда берут самых лучших "--- тех, кто не повторяет ошибок.
Этих демонов можно по праву назвать незримым щитом Ада.

Минь опустил голову.

"--*Я уже почти пол-телльна работаю в безопасном отделе архива, Аркадиу.
Для меня это жестокий удар.
Я не думал, что информацию из безопасного отдела можно обернуть против нас таким образом.
Скорее всего, архивы закроют и подвергнут реорганизации.

"--*Что говорилось в базах данных? "--- спросил я.

"--*Согласно базам данных, настоящее имя этого Атриса "--- Ковнелий Фиктовий Саз.
Урождённый человек, преобразован ещё во времена Союза Воронёной Стали.
Он был под подозрением "--- формально держал нейтралитет, но сотрудничал с Картелем.
После переворота на Сцелае сбежал и с тех пор ошивается в районе Тукана, 19 зарегистрированных контактов с агентами Ада.
А тут выходит, что он\ldotst

"--*\ldots что он действительно тот самый Добрый бог, Безымянный, демиург Тра-Ренкхаля, изгнанный узурпатором Эйраки, "--- закончил я за него.
Последние кусочки картины встали на свои места.
"--- Откуда взялся Безымянный?
Как его зовут?

"--*Нигде об этом ни слова. В базах отмечено, что демиург Тра-Ренкхаля "--- Хатрафель Безумный.
Ваша команда сообщила, что нгвсо почитают Безымянного.
С этого несоответствия ребята за пару часов распутали всю историю.
Сам понимаешь, если бы вас отправили на поиски демиурга, а не на борьбу с Безумным\ldotst

"--*Понимаю, "--- кивнул я.

"--*Нет, ты только подумай!
Ведь отчёты об освоении планет тщательнейшим образом\ldotst

"--*А что насчёт этого Ковнелия?

"--*А\ldotst
У него приличная биография, построенная на данных погибших демонов.
Спрашивать, существовал ли Ковнелий на самом деле, разумеется, уже не у кого.
Агент Картеля "--- кто бы он ни был "--- постарался на славу, пролез где только можно.
Я не совсем понимаю смысл этой\ldotst

"--*Картель опасался, что мы выйдем на Безымянного, и решил подстраховаться, "--- предположил я.
"--- Найти демиурга на его собственной планете они не могли "--- тот продумал систему маскировки.
Вычеркнуть его из наших баз "--- чересчур подозрительно, а придумать ему липовую неблаговидную биографию "--- вполне себе хороший ход.
Даже если бы мы его встретили "--- отправили бы <<на отдых>> как неблагонадёжного.
А слепое вторжение на Тра-Ренкхаль закончилось бы резнёй.

"--*Вы молодцы, ребята, "--- заметил Минь.

"--*Ага, мы, "--- саркастически проворчал я.
"--- Битву за Тра-Ренкхаль мы выиграли благодаря невероятной случайности и находчивости Грейса.
Если кто и молодец, так это он.
Я боюсь даже предположить, сколько военной силы мы бы потеряли из-за этой подсадной утки.
Лусафейру всё-таки гений.
Он, похоже, подозревал, что тут не всё чисто\ldotst

Один из роботов выключил глушитель и впервые заговорил приятным женским голосом:

"--*Аркадиу Шакал Чрева, Минь Орлиная Заря, прошу вас проследовать с нами на станцию С9A0.

За время разговора, я знал, они полностью проверили меня на предмет подслушивающих устройств, маячков, молекулярных механизмов регистрации и прочей шпионской техники, а также провели всесторонний анализ моей личности.
Выключенный глушитель означал, что я не представляю опасности.
По крайней мере пока.

Я улыбнулся и кивнул агентам.
Роботы растворились в воздухе.

"--*Пошли, дружище, "--- похлопал я по спине честного архивариуса.
"--- У нас с тобой совесть чиста.

\section{Братья по разуму}

"--*Грейс, у меня проблема, "--- начал я.
"--- Не могу найти понятную информацию по Ветвям Звезды.
Это форма жизни, но при обучении мне намекнули, что Звезда не в моей компетенции и занимаются ею другие биологи.
Не мог бы ты рассказать о них?

"--*А, "--- откликнулся технолог.
"--- Хм-хм.
Ветви Звезды.

"--*Может быть, это секретная информация и не стоит её\ldotsq

"--*Нет-нет, "--- перебил меня Грейсвольд.
"--- Это информация общедоступная, но без интерпретации понять её сложно.
Слушай, попробую объяснить.

Я схватил компьютер и настроил на запись.

"--*Как ты знаешь, наша область Вселенной предположительно является <<мёртвой зоной>> "--- жизнь здесь встречается довольно редко.
Ветви Звезды "--- это сапиенты с планеты 1-34, второй известной планеты со стабильной самозародившейся сапиентной жизнью.
В источниках, предназначенных для Земли, планеты Звезды обозначаются цифрами.
Их способ общения "--- назовём его <<языком>> "--- кардинально отличается от языков Ветвей Земли.
Он полностью химический, с помощью полимеров и низкомолекулярных веществ.
У Земли пообщаться со Звездой просто так не получится.

"--*А что они собой представляют?
Кажется, их химический состав\ldotst

"--*Да, химический состав.
Углерод, кислород, кремний, азот, сера, в целом то же самое, но на другой лад.
Поглощают метан, выдыхают углекислый газ.
Оптимально их существование при температуре кипения этанола и давлении, в 2.8 раз превышающем земное.
Своеобразное строение <<тела>>, назовём его так.
Сложно установить, где заканчивается одно тело и начинается другое.

"--*Я так понял, что это нечто, похожее на мицелий?

"--*Сложнее, "--- покачал головой Грейс.
"--- В мицелии есть отдельные клетки, а тут\ldotst ммм\ldotst многослойный синцитий.
У них есть <<города>> "--- губчатый скелет из кварца с прослойками биоколлоида, занимающий огромные пространства\ldotst

"--*И это существо разумное? "--- удивился я.

"--*Как ни странно, "--- ответил Грейс.
"--- У них есть технология, они вышли в космос и заселяют экзопланеты.
Вряд ли столько, сколько Земля, но ненамного меньше.

"--*Понятно.
И они испускают при угнетении плюс-эманации, а при благоденствии "--- минус?

"--*А, да.
Звезда и хоргеты.
Интересный вопрос.
Да, 1-34 "--- одна из самых мощных баз Картеля, именно поэтому мы знаем про Звезду не так много.
После поражения в Развязке Десяти Звёзд Картель отступил на территорию Ветвей Звезды и закрепился там.
Ордену Преисподней на планетах Звезды действовать так же трудно, как Картелю на территории Земли.

"--*Я прочитал мнение, что впоследствии война между Картелем и Адом может перерасти в войну между Землёй и Звездой.

"--*Я с этим согласен.
Звезда постепенно изменяет свой достаточно примитивный химический способ общения на более быстрый, волновой.
В будущем можно ожидать появление не только киборгов, основанных на биологии Звезды, но и совершенно новых существ.
Однако есть интересный нюанс.
Молекулярный механизм излучения эманаций, связанный с системой ответа на раздражение, введён в Звезду искусственно.
Каким образом вышло так, что эволюция живых существ вызывает минус-завихрения, ещё предстоит выяснить.
Также найдены нуль-штаммы Ветвей Звезды, то есть не излучающие эманации вообще.
Другой интересный нюанс "--- молекулярный механизм построен по неизвестной ранее схеме, и хоргеты Картеля отрицают свою к этому причастность.

Я обомлел.

"--*То есть\ldotsq

"--*Да, ты правильно понял.
Возможно, это сделали хоргеты, созданные Ветвями Звезды.
Или вообще другие.
Кто знает, может, и сама Звезда "--- искусственно выведенная хоргетами форма жизни?
Судьба этих братьев по разуму неизвестна.
Были ли они уничтожены в войне, подобной этой?
Примкнули ли они к Картелю?
Был ли ими выведен микоргет, ушедший в другие Вселенные?
Предстоит разобраться.

"--*И это значит, что у нас появилась новая проблема.

"--*Да, новая возможность и новая проблема.
Возможность использовать в своих целях Ветви Звезды и проблема контроля генофонда Ветвей Земли.
Вряд ли Картель упустит шанс вывести, например, минус-людей.
Если те, другие хоргеты примкнули к ним "--- в технологическом плане мы на шаг позади.

Я обхватил голову руками.

"--*А всё-таки, что мешает создать стабильный источник эманаций и оставить сапиентов в покое?
Сколько проблем было бы решено!

Грейсвольд вздохнул.

"--*Да, Аркадиу.
Проблема ключевая для процветания нашего вида, но ею занимается катастрофически мало демонов.
Давно ты что-то слышал о разработке микоргета?
И я тоже.
Война, видимо, Ордену нужнее.

\section{Язык Эй}

Когда Красный Картель и Орден Преисподней начали войну за обитаемую Вселенную, остро встал вопрос коммуникации "--- как между демонами, так и с сапиентами.

Картель и Ад и прежде использовали специальные языки для сражений (т.н. боевые языки, отличающиеся простотой и краткостью) и для передачи информации (шпионские языки, сложные для расшифровки).
Боевой и шпионский языки Ада, в частности, были созданы на основе языка сохтид "--- самого распространённого человеческого языка на Преисподней.
В силу некоторой оторванности Картеля от цивилизаций сапиентов минус-демоны создали ещё один тип языка "--- усечённый (sekta-lingu), с помощью которого демоны коммуницировали с подвластными им сапиентами.

У всех вышеперечисленных языков был один большой недостаток "--- они были придуманы людьми в процессе эволюции и, как и любой живой организм, несли в себе огромный груз отрицательных мутаций.
Поэтому обе организации поставили задачу "--- разработать единый язык, избавленный от пережитков прошлого.

Картель первым справился с поставленной задачей.
После долгих дискуссий было решено оставить секта-лингу для общения с сапиентами, а в качестве боевого и шпионского языков использовать новосозданный Чи.
Демоны понимали, что главное оружие криптологов "--- это логика.
Вследствие этого между корнями и словоформами языка Чи не было никакой логической связи.
Фактически одно и то же слово из предложения в предложение менялось до неузнаваемости.
Словоформы придумывали 130 генераторов случайных чисел, разработанных на основе человеческого мозга.
Словари языка Чи были строжайше засекречены, а для демонов, которые им пользовались, были созданы специальные системы защиты, мгновенно уничтожавшие языковой сектор в памяти при попытке проникнуть в него или выдать его постороннему.

Первый раунд в этой битве был блестяще выигран учёными Картеля.
Расшифровать Чи даже на 1\% не удалось до сих пор, несмотря на усилия разведки, криптологов и аналитиков.
Данные, приведённые здесь "--- это, увы, почти всё, что Ад на сегодняшний день знает об этом таинственном языке.

Учёные Ада такими результатами похвастаться не могли.
В архив отправляли одну версию за другой "--- какие-то браковались культурологами, какие-то криптологами.
Возможно, что это затянулось бы ещё на неопределённое время, если бы не вмешался один из старейших демонов, который служил Ордену Преисподней аж с момента его создания.

Ликан Безрукий.
Урождённый человек, один из жителей древней Преисподней.
Не совсем ясны обстоятельства, по которым он стал демоном.
Возможно, об этом знает кто-то из старожилов "--- Лусафейру или Грейсвольд, но они не распространяются на эту тему.

"--*Грейс, расскажи про Ликана Безрукого, ты ведь его знал.

"--*Ааа, Ликан.
Да-да-да.
Достойный был демон, достойный.

На этом разговор обычно заканчивается.

Ликан Безрукий отказался от коррекции личности, которую проходят все урождённые люди.
Вы можете себе представить, что творится в голове у человека, который вынужден прожить несколько телльнов, и Ликан, как видно, потихоньку начал сходить с ума.
По словам очевидцев, общаться с ним было форменным наказанием.
Впрочем, своё дело (а работал он в Аду аналитиком) Ликан знал блестяще, занимался охотно и увлечённо, и причин отстранять его не было.

Узнав, что отдел 214 занимается разработкой нового языка, Ликан ходатайствовал о подключении его к работе, так как всю жизнь питал к лингвистике известную слабость.
К тому времени 214, который терпел одну неудачу за другой, превратился в закрытый клуб, и ему было отказано.
Ликан совершенно по-человечески обиделся и в рекордные сроки (всего за год) разработал синтаксис и морфологию языка Эй.
Ещё год ушёл у него на наработку и сортировку словаря, и вскоре старый демон попросил Совет устроить открытое (sic!) слушание его доклада.

Весь Ад с интересом следил за событиями.
Все знали, что Ликан собирается представить новый шпионский язык, и у большинства зрел вполне закономерный вопрос "--- не сошёл ли он с ума, устраивая открытое слушание своего доклада?
Подобные разработки засекречивались, едва успев появиться на свет.

Вот отрывок из его выступления:

<<\dots Язык Эй представляет собой логичную, стройную систему.
Слова максимально короткие, ёмкие, лишены избыточности и возможности двойного толкования.
Язык подходит для изучения любым существам "--- как демонам, так и самым примитивным сапиентам\ldotst>>

Всё это очень хорошо, скажете вы, но зачем сдался нам шпионский язык, который понятен и лёгок в изучении для всех?
Я таких вам десяток настрогаю, только скажите.
В том же смысле высказались и члены Совета "--- разумеется, используя другие выражения.

Но старый демон не успокаивался:

<<\dots Также я предлагаю\ldotst нет, требую, чтобы словари языка Эй и прочая информация по нему находились в открытом доступе>>.

Слушатели всё больше утверждались во мнении, что Ликан сошёл с ума.
До тех пор, пока он не сказал самого главного.

<<В основе языка Эй лежат 16 цифр, из которых и строятся слова.
Фонетические и графические правила языка Эй устанавливаются донором и акцептором информации в соответствии с их анатомическими особенностями, органами восприятия\ldots и прочими важными условиями коммуникации>>.

После этих слов аудитория замерла, а потом взорвалась проявлениями восторга.
Люди засвистели и зааплодировали, кани завыли, хлопая руками по бёдрам, лишённые голоса замахали конечностями.
Конечно же, так отреагировали в основном культурологи, находящиеся в неизменённых сапиентных телах, но так получилось, что они выразили мнение абсолютного большинства аудитории.

В тот же день, после символической проверки трудов Ликана аналитиками и внесения столь же символических правок, Совет единогласно утвердил язык Эй, таблицу 00, как официальный язык Ада.

У читателя, разумеется, возникнет вполне закономерный вопрос.
Фактически исходный язык Эй подвергается шифрованию по типу <<кодовой книги>>, которое сохраняет статистические особенности текста и довольно легко расшифровывается.
На это особенно указывали демоны отдела 214.
В чём же его преимущество?
Ответ прост --- таблицы подразумевают одновременный поток информации по нескольким каналам.
Грубо говоря, не знакомый с таблицей сторонний наблюдатель не знает, двинул ли я плечом из-за случайного комара или это символ, входящий в общий поток.
В некоторых случаях сообщение сложно отличить даже от обычного шума.
Расшифровать длинное сообщение способен лишь опытный визор, а короткое, которое может содержать в том числе и следующую таблицу, практически не поддаётся расшифровке.
Эй оказался идеальным сочетанием простоты, надёжности и практичности.

Разумеется, Картель тут же узнал о докладе.
Но, увы, это им не помогло.
Таблицы правил множились в геометрической прогрессии.
Фактически у любых двух демонов, которые общаются между собой, могла быть своя таблица правил.
Появились специальные таблицы для разных видов, рас и народностей, с учётом их способов коммуникации "--- жестовые, голосовые, мимические, цветовые, музыкальные, шумовые и смешанные, а также множество алфавитов.
Появилась Мирквудская классификация "--- попытка систематизировать таблицы.
При этом синтаксис, морфология и словарь языка Эй оставались неизменными.

Картель умел признавать поражение.
Вскоре его агенты выкрали данные о языке, хотя лично мне кажется, что они их просто взяли, как мы берём книги из библиотеки.
Вряд ли кто-то так уж попытался им помешать.
Год спустя у них были свои таблицы правил и свой форк Миквудской классификации, различающийся, по самым скромным оценкам, на миллион триста тысяч таблиц.
Язык Чи благополучно отправился в архив нерассекреченным.

Наверное, это самый большой парадокс в истории: вся обитаемая Вселенная внезапно заговорила на одном языке "--- и при этом два его носителя могли не понять друг друга.

Для ознакомления привожу таблицу правил B0, которой пользуется 99\% демонов-людей Ада для повседневного общения "--- одну из самых простых.
По Мирквуду она является двухполосной неоригинальной СЧФ-таблицей.

$L$ "--- длина слова

$P$ "--- ключ позиций

$P_x$ "--- чтение позиции

$S$ "--- символ языка Эй

\[L = 1: P = 0\]
\[L = 2: P = 10\]
\[L = 3: P = 101\]
\[L = 4: P = 1010\]
\[L = 5: P = 10101\]
\[L = 6: P = 101010\]
\[L = 7: P = 1010101\]
\[L = 8: P = 10101010\]
\[S = 0: P_0 = a, P_1 = l\]
\[S = 1: P_0 = o, P_1 = m\]
\[S = 2: P_0 = u, P_1 = n\]
\[S = 3: P_0 = e, P_1 = r\]
\[S = 4: P_0 = ai, P_1 = z\]
\[S = 5: P_0 = oi, P_1 = c\]
\[S = 6: P_0 = ui, P_1 = j\]
\[S = 7: P_0 = ei, P_1 = s\]
\[S = 8: P_0 = ia, P_1 = v\]
\[S = 9: P_0 = io, P_1 = f\]
\[S = A: P_0 = iu, P_1 = g\]
\[S = B: P_0 = i, P_1 = k\]
\[S = C: P_0 = ah, P_1 = d\]
\[S = D: P_0 = oh, P_1 = t\]
\[S = E: P_0 = uh, P_1 = b\]
\[S = F: P_0 = eh, P_1 = p\]

Таким образом, слово <<химическое соединение>> "--- A3F8 "--- согласно этой таблице будет читаться как gepia.

\section{Экскурс в историю}

Я думаю, что моим читателям любопытно будет узнать, с чего всё началось.

Материнской планетой Ветвей Земли была Древняя Земля.
Нам мало что известно об истории развития первых людей до эпохи Последней Войны, 1914--2032 годы Господина (sl: Anni Dominio).
Предположительно эти существа, получив преимущество перед остальными видами в виде относительно развитого интеллекта, довольно быстро заселили планету, покрыв её городами и сетью путей сообщения.
Учёные Ада склоняются к мысли, что тогда ещё не было деления на расы и подвиды "--- генетический дрейф и адаптивная изменчивость в развитом технократическом обществе маловероятны.
Известно, что за 200 лет людская популяция путём генной инженерии избавилась от груза мутаций, стабилизировала рождаемость сообразно ресурсам планеты, и взгляды людей впервые обратились к далёким мирам.

Первые космические корабли были весьма ненадёжны.
Путь до пригодных для жизни планет занимал тысячи парсак, а скорости выше световых были недоступны.
Десятки тысяч добровольцев отправлялись в далёкие путешествия, не зная, что ждёт их в пути.
Цивилизация медленно, но верно переходила в стадию застоя, и этот переход завершился бы, но люди со свойственной этим существам оригинальностью нашли выход из сложившейся ситуации.

Примерно в 315 году от Последней Войны (2347 год Господина) люди открыли омега-поле.
До этого момента Вселенной Ветвей Земли была лишь так называемая фотонная (электромагнитная) связка "--- прямо или косвенно связанные с фотоном взаимодействия.
О существовании прочих квантовых связок, которые назывались <<параллельными Вселенными>>, люди догадывались, но обнаружить их существование не могли.

Омега-поле, или поле Кохани"--~Вейерманна (ПКВ), возбуждается от присутствия наблюдателя "--- сознания.
Всякая система обладает сознанием и способна наблюдать, но наибольшую напряжённость поля создаёт именно сапиентное сознание, заключённое в относительно небольшой, но тем не менее сложный мозг.
Подавляющее число систем воздействует на ПКВ одинаковым образом "--- эволюция системы вызывает положительное искривление, инволюция "--- отрицательное.
Но известны и исключения (Ветви Звезды).

ПКВ косвенно воздействует на огромное число связок.
Именно его воздействием объясняется некоторый элемент случайности в струнных взаимодействиях.
Это не истинная случайность, а результат наложения на ПКВ эффектов огромного количества вложенных друг в друга Вселенных, построенных на разных связках.
Физики Древней Земли даже называли ПКВ <<океаном случайности>>.

Открытие омега-поля было последним аккордом создания в физике Теории всего, давшей взрывоподобное развитие прочих наук.
Теория всего, несмотря на присутствие в ней практически недоказуемых гипотез, применима и в настоящее время.
В частности, именно на её основе созданы устройства для оцифровки и программы взаимодействия хоргета с окружающим миром.

Путём экспериментов с первичным полем была создана anima "--- первый примитивный хоргет.
Хоргет "--- это стабильная сингулярность ПКВ, расположенная перпендикулярно связкам.
Его протяжённость во Вселенной Ветвей Земли "--- не более диаметра протона.
Он способен благодаря накопленной масс-энергии воздействовать на материю "--- переформировывать струны, изменять их частоту.
Отдельно следует выделить способность хоргета преодолевать световой барьер (перемещение по так называемому каскаду спутанных фотонов (КСФ), или фотонному лабиринту (ФЛ), возможное при соответствующих тратах масс-энергии).

Настала так называемая godage "--- Эпоха богов.
Сапиентам Земли больше не было нужды снаряжать дорогостоящие экспедиции к пригодным для жизни планетам "--- хоргеты заряжались масс-энергией и посылались в направлении нового мира, собирая информацию и целенаправленно изменяя климат.
Когда медленные космические корабли с первыми колонистами достигали новой планеты, она уже была полностью пригодна для жизни.

Первоначально боги были запрограммированы на приведение планеты в пригодный для жизни вид и последующее самоуничтожение.
Но не все хоргеты следовали плану, который вложили в них учёные "--- программа омега-сингулярности обнаружила способность к быстрой мутации.
Фрактальное дублирование кода, которым техники попытались компенсировать мутации, привело к ещё более серьёзным последствиям "--- направленной эволюции.

Боги приобретали нечто похожее на инстинкт самосохранения, а последующие мутации приводили к программам самостоятельного получения масс-энергии.
Иногда по прибытии земляне оказывались в не похожем на Землю месте, с чуждыми, порой опасными формами жизни, порой разумными и даже превосходящими интеллектом самих землян.
Эти существа часто организовывали культы создавших их богов, обеспечивая демиургов масс-энергией.
Во многих подобных мирах экспедиции гибли, и туда больше никогда не ступала нога сапиентов Земли "--- боги прекрасно понимали опасность, исходящую от их создателей.
Но кое-где пришельцы выжили и приспособились к трудным условиям.

Связь между планетами всегда была серьёзной проблемой.
Корабли, разумеется, шли в один конец, их путь порой занимал сотни и тысячи лет.
Какое-то время связующим звеном была Древняя Земля "--- специальные хоргеты сновали по Вселенной, принося колониям новости с материнской планеты и собирая информацию о колонистах.
Но вот то одна, то другая колония стали замолкать в силу разных причин.
Цивилизации гибли в результате катаклизмов, нападений демиургов и Девиантных Ветвей, сапиенты дичали и теряли технологические знания.
Наконец, спустя двадцать три тысячи лет, из-за <<бродячих камней>> замолкли обитаемые планеты Солнечной системы "--- Древняя Земля, Марс и искусственно созданная Диана.
Каждая колония осталась предоставлена сама себе\footnote
{В настоящее время на Древней Земле существуют 4 вида с уровнем развития в районе 90, относящиеся к высшим приматам и псовым, но не являющиеся потомками первых людей и первых кани.
Большая часть информационного наследия первых людей была сохранена и задокументирована хоргетами, впоследствии примкнувшими к Вечности, Братству Воронёной стали и Ордену Преисподней. \authornote}.

Вот тогда-то, на закате единого сапиентного общества, ведущей силой стали мутировавшие (девиантные) хоргеты.
У них было всё "--- накопленная за историю человечества информация, способность к перемещению между планетами и практически неограниченное время существования.
Многие оставили участь богов отдельно взятого мира и стали путешествовать по планетам, инкарнируясь в тела сапиентов и проживая одну смертную жизнь за другой.
Доподлинно не известно, когда и где появился интернационализм <<asoga>>, в переводе означающий <<демон>>, но считается, что именно так в начале времён бродячие (мобильные) хоргеты называли друг друга.
Эпоха богов медленно, но верно подходила к концу.
Забрезжил рассвет asogeite "--- Эпохи демонов.

Во время существования в телах сапиентов хоргеты имели возможность получать новую информацию и масс-энергию почти без трат со своей стороны.
Благодаря способности воздействовать на материю инкарнированные хоргеты всегда имели высокое положение в обществе: становились правителями, организовывали религиозные культы, что в свою очередь обеспечивало их постоянным потоком масс-энергии.
Но число хоргетов росло, и сейхмар уже не могли обеспечить питание всем.
Это послужило причиной конфликта между положительно и отрицательно питающимися хоргетами, которые были заинтересованы в получении взаимоисключающих ресурсов от сейхмар.
Впоследствии этот конфликт привёл к появлению крупных союзов демонов и военным столкновениям между ними.

\spacing

Возможно, вам сложно понять, как начинает осознавать себя существо, отличное от вас.
Я думаю, что в этом плане особой разницы между сапиентами и хоргетами нет, за исключением того, что у сапиентов почти всегда есть возможность развиваться в комфортном окружении себе подобных.
У первых богов такой роскоши, увы, не было.

Самой распространённой ошибкой создателей богов было то, что они давали детищу чересчур много свободы действий, увеличивали его интеллект, но при этом относились к нему, как к инструменту.
Первых богов совершенно лишили той поддержки, которая необходима маленьким разумным существам.
Таким образом, например, появился Грейсвольд Каменный Молот, тогда известный как Griswold K-28, творение Лаборатории Дж.\,Грисвольда и мой лучший друг.

Сам Грейс утверждает, что после создания он сразу закричал <<Я родился!>> и принялся радостно носиться по коллайдеру.
Но я знаю его достаточно и понимаю, что всё это выдумки.
Вряд ли он ощущал что-то кроме акбаса.

Разумеется, он не помнит о своём создании почти ничего.
В его программе не было предусмотрено направленное накопление опыта.
Но отдельные фрагменты сохранились.
В частности, Грейс после долгого анализа вспомнил, кто такой этот Дж.\,Грисвольд, и даже нарисовал его портрет.
Ничего особенного "--- обычный мужчина эпохи Богов, страдающий выпадением волос на голове и старческой дальнозоркостью, компенсированной толстыми стеклянными линзами.
Единственное, что привело культурологов в полный восторг "--- это огромные седые усы, которые вы сможете найти разве что на планетах вроде Мороза, где волосы на лице, как у мужчин, так и у женщин "--- не роскошь, а насущная необходимость.

\chapter{Приложения}

\section{Список иллюстраций}

\begin{itemize}
\item Карта "--- <<Хроники дорог и ветров>>, том 1, обитаемая Корона.
Относительно точная карта;
есть предположения, что она строилась по картам тси, не сохранившимся до наших времён.
(Либо, как вариант "--- картография Ордена Преисподней.)
\item Карта Края.
Данные картографии Ордена Преисподней.
\item Принципиальная двумерная схема строения хоргета.
\item \textbf{[Глава 2, параграф 1]}
Кварталы Тхитрона в мирное время и во время вторжения, вагенбург на перекрёстке, колодец-требюше.
\item Устройство храма.
Казармы, школа, гонги, зал с ласточкиными нишами, операционный балкон, крипта, зона молчания.
\item \textbf{[Отчёт Хомяка]}
Схема тор-отсека.
Единственное известное изображение кольцевой теплицы.
Манускрипт <<Процветание. Не более необходимого>>, Сок-Стального-Листа.
\item Кольцевая теплица.
Художественная фантазия Тхарту ар'Хэ.
\item Письменность.
Абис, письменность сели, иероглифика цатрон: разные каллиграфические школы.
Змеистая письменность, дипломатическое письмо травников (ячеистое), микханская тайнопись, резы стрелохвостов.
\item Гамма цветов, воспринимаемая тси.
\item Образцы амулетов со Старой Личинкой "--- дерево, кость, камень, металл.
\item Любопытна история иероглифа.
Этим символом "--- спиралью "--- на Преисподней помечали дома, в которых, по мнению людей, поселился \textit{хорохито} (инкарнированный хоргет).
Символ ставили в незаметном для хозяев месте, иногда выкладывали из камней или вытаптывали на дороге.
Хорохито редко убивали, так как бытовало поверье, что дух переселяется в другого человека;
очень часто его даже бесплатно кормили и одевали.
Тем не менее, подозреваемому никто не верил, с ним не дружили, не заводили любовные связи и вообще старались не иметь никаких дел.
\item Женщина-трами с обрезанным правым крылом.
Обрезка <<на два клина>> говорит о её принадлежности к трами Кипящего Полуморя.
\item Портрет: Эрхэ ар'Люм с бумажным фонарём.
\item Портрет: Кхохо ар'Хетр пляшет без рубахи, в штанах, у пояса сабля, в зубах трубка.
\item Портрет: Чханэ ар'Катхар в венке из омелы.
\item Портрет на разворот: Манэ и Лимнэ ар'Люм.
\item Портрет: Секхар ар'Сатр.
\item Портрет: Трукхвал.
\end{itemize}

\section{Язык цатрон}

\begin{itemize}
\item \`a "--- descenda. Как будто бросают или машут рукой.
\item \'a "--- ascenda. Краткое удивление.
\item \^a "--- acuta. Короткий звук с резким, <<истерическим>> повышением тона.
\item \~a "--- vibrata. Синусообразное плавное изменение высоты тона с большой амплитудой.
\item \"a "--- tremola (discreta). Смех или блеяние на одной ноте (штробас).
\item \r{a} "--- dislocata. Синусообразное изменение высоты тона с большой амплитудой при расщеплении гортани (штробас).
\item \=a "--- plata. Плавное на одной ноте, средней продолжительности.
\item \v{a} "--- profunda. Глубокое удивление.
\end{itemize}

\section{Язык Эй (данные)}

\subsection{Мирквудская нумерация}

Мирквуд "--- старейший научный центр Гелиополя, занимающийся естественными и искусственными языками.

Таблица 00 "--- основа языка. 00-2 "--- двоичная, 00-8 "--- восьмеричная, 00-F "--- шестнадцатиричная.

Таблица 10 "--- все виды жестовых (неконтактных) таблиц для 4--5-пальцевых конечностей.

Таблица 20 "--- омега-архитектуры.

Таблица 30 "--- дельфинья фонетика.

Таблица 50 "--- электронные архитектуры.

Таблица 60 "--- световые и квантовые архитектуры.

Таблица 80 "--- танцевые и тактильные таблицы.

Таблица B0 "--- усовершенствованная таблица для СЧФ.

Таблица D0 "--- упрощённые химические таблицы для Ветвей Звезды.

Таблица F0 "--- таблица для тси-подобных языков.

Прочие цифры зарезервированы под специфические смешанные виды передачи информации и коды на основе Эй.

Первыми после доклада Ликана Безрукого языком стали пользоваться апиды и дельфины Капитула, что отразилось на нумерации таблиц "--- жестовые и СДФ-таблицы появились раньше других.

\section{Культ Четверых}

Религия, которая объединяет почти все народы Тра-Ренкхаля.
Несмотря на различия у разных народов, существование общих корней этих четырёх культов доказано.

\begin{description}
\item[Культ Разрушителя] (Безумного, Несчастья, Поработителя) "--- самый распространённый.
Народы Тра-Ренкхаля верят, что несчастья и смерть "--- необходимая часть жизни, и если жить становится очень хорошо, значит, мир приближается к гибели.
Разрушителю поклоняются народы: сели (Безумный), ноа (Деа Акседент), трами (Бог-Убийца), хака (Безумный), тенку (Молотобойца), ркхве-хор (Уничтожитель).

\item[Культ Опалителя] (Солнца, Света).
Основным является у ноа, зизоце и пылероев.
Наблюдается у ноа (Деа Солар), тенку (Солнечная Птица), ркхве-хор и пылерои (Опалитель), зизоце и тенку (Отец).

\item[Культ Воды.] Наблюдается у сели (Сестра Дождь, Сестра Река), хака (Мать-Дождь), тенку и зизоце (Мать), ноа (Деа Марин), дельфины (Милая Бездна), нгвсо (Омывающая).
Основным является у дельфинов. Принимает самые разные формы "--- от всеобъемлющей заботы у нгвсо до своенравного Деа Марина у ноа.

\item[Культ Создателя] (Творца, Изгнанника, Безымянного).
Наблюдается у сели, хака, тенку, ноа, дельфинов и нгвсо.
Основным является у нгвсо.
Также предполагается, что у тенку Культ Солнечной птицы вмещает как культ Опалителя, так и культ Создателя, т.к. Солнечная птица имеет совершенно явно амбивалентную природу.
\end{description}

Общая логика, связывающая 4 культа:

\begin{itemize}
\item Создатель "--- инженер, кто создаёт машину;
\item Вода "--- среда, пространство, где создают машину;
\item Опалитель "--- источник энергии, время, что приводит машину в движение;
\item Разрушитель "--- тестер, кто испытывает машину на прочность.
\end{itemize}


\chapter{Не для печати}

\section{Идеи}

\begin{enumerate}

\item Скорбящие.
Так у Аркадиу есть тело или нет?
Тот же вопрос насчёт Штрой "--- куда делось тело после изгнания?

\item Зонт-фонарь "--- по типу зонтиков торговцев.

\item Оазис "--- города-государства Синего колена (они выжили).
Травники прошли по Дороге Жизни "--- разлом, ведущий через смертные пустоши к Оазису.
Согласно легенде, утро застало переселенцев в Смертных пустошах, и травники уже приготовились к смерти, как вдруг увидели летящую на юг птицу "--- тигровую сову.
Сова и вывела племя к Дороге Жизни.
С тех пор Синее Колено поклоняется совам;
в их легендах говорится, что солнце давно сожгло бы мир, если бы не Великая Алмазная Сова, которая закрывает мир своими прозрачными крыльями.

\item Керномор Тридцать Три.

\item Костяные солнечные очки ноа "--- из щитка шлемоносной акулы (акулы-быка).
Вырезались точно по изгибам лица, обладали высокой прочностью и очень высоким коэффициентом отражения, почти не нагревались на солнце.

\item Печать "--- отпечаток пальца, губ или укус (или всё вместе, где какие традиции).

\item Ивовая кора "--- салицилаты.

\item Первый жрец (Картель) курсивно выделяет слова (или не надо, перегрузка?).

\item Имя для демона "--- Эйстейн Дева.

\item Большая игра "--- логика беседы нарушена.

\item Вулкан против ищеек.
Средства против технического отслеживания на Преисподней продумать.

\item Преисподняя "--- горячие источники, гейзеры, сера.

\item На Преисподней у демонов ещё не было прозвищ.

\item Тропический климат.
Продумать ветры.

\item Плавник, который морем приносит.
Хз зачем он мне, но пусть будет.

\item Цатрон.
Ударения удвоением гласных?

\item Описание эмоций не словами, а реакциями организма.
\textbf{Серьёзная правка!}

\item Язык планеты Мороз "--- сделать отдельную статью?

\item Коллизия "--- Паутина-город Лотоса и Паутина Тси-Ди.

\item Кельса (Цельсия) Пушистая.

\item Объясняет, но не оправдывает.

\item Северная Корона (Обитаемая Корона).
Южная Корона (Суболичье).
Китовый Юг (Тысячеречье).

\item Может, на Тра-Ренкхале не будет обезьян?
Мне чот лень их уже добавлять.

\item Ликхмас не должен быть Мэри Сью.
Например, в первой главе сделать упор, что ягуар маленький и неопытный.

\item Зизоце выбривают темя, потому что лысина "--- признак мужественности и зрелости.

\item Инфинит "--- эпиграф 1 гл. II части?

\item Бабочка "--- грудная клетка на языке сели.
Мужской жемчуг "--- сперма.

\item Беседка над алтарём.
Дожди всё-таки.

\item Террористический акт, угроза биологическим оружием.

\item Имя: Эвкалипт.

\item Вставки других языков в текст "--- эффект <<недопонимания>> героем чужой речи.

\item Есть предложение после 2 главы 2 части вставить литую главу <<Лисьи сказки>>, куда слить все происшествия в пути Ликхмаса, Грейсвольда и Анкарьяль.
Или две литых главы на часть "--- это многовато?

\item 11.6 "--- не вполне ясно, что отряд 14 сражается с порабощёнными тси.

\item Больше Бродячего Народа.
MOAR.

\item Обычай сели: белые рубахи в городе и иная одежда в джунглях.

\item <<Если ты боишься, дай противнику тебя ударить, "--- вспомнил я слова Конфетки.
"--- После первого удара страхи уходят>>.

\item Она говорила о какой-то ерунде, но её руки в моих волосах и ноги, словно невзначай прижимающиеся к моим плечам, шептали такие нежности, что я мог только сидеть и молча слушать.

\item Архипелаг Ночного Костра.

\item Для Лунных садов.

Город Хазер.
Город богачей, в котором даже уборные стоят немалых денег.
<<Если бы жизнь в Хазере была дешёвой, сюда бы не стремились бедняки и мы бы тогда сами стали бедняками>>.
Конное поло в роскошных доспехах и на роскошных лошадях.
Юноша из богачей увидел сон, что игра обессмертит его имя.
Бессмертие "--- единственное, чего не было у богачей.
Он вышел на поле без доспехов.
Затем купил прямо в игре лучшего коня у богатого старика, отдав ему тяжёлую золотую сбрую.
И погиб на поле (продумать фабулу).

\item Нестыковка: демоны без тел могут или всё же нет?

\item Клин белых птиц превратился в созвездие.

\item Клянусь, что приму пришедшего ко мне и его историю такими, каковы они есть "--- без оговорок и без остатка.

\item Коллизия. Оазисы Тси-Ди и оазисы Преисподней.

\item Flos (цветок) "--- на жаргоне Картеля: сейхмар.
Сорвать цветок "--- прожить жизнь в теле сапиента.

\item Магазин кукол.
Трикстер (Глупец?) продаёт куклы детям, дети с ними играют и формируют судьбу взрослых.

\item Колебание струны потоком воздуха.
Идея для цитры Ветра.

\item Сели видят тело сквозь одежду.

\item Испытание пчелами.
В школе боевых искусств мастером признавался тот, кто саблей рассекал всех пчел в воздухе до того, как они коснулись его.

\item Кемер Зимняя Вишня (Kemer the Winter Cherry).

\item После Отбора ребёнок придумывает себе имя тси.
До Отбора ребёнок носит только имя цатрон и (иногда) домашнее прозвище.
В разных местах традиция разнится "--- кое-где домашнее прозвище не связано с именем тси, кое-где ребёнок выдумывает имя на основе домашнего, кое-где ребёнка всё детство зовут по имени цатрон, а имя тси (и, соответственно, домашнее) он выдумывает сам.
У ноа тайное имя даётся так "--- два слова придумывают кормильцы, третье "--- сам ребёнок.

\item Нет слов <<хороший>>, <<плохой>>, <<добрый>> и <<злой>>.
\textbf{Серьёзная правка!}

\item Кошка внесла в культуру тси изменения, которые превращали взращённых в этой культуре сапиентов в ловушки для демонов.
Эти тела прививали демонам качества самих тси.
Поэтому большинство демонов использовали неполное заякоривание и/или пользовались телами людей Тра-Ренкхаля.

\item Тхитрону много тысяч дождей.
История, развалины, барельефы, пережитки других поколений.
Культура идолов (Нашествие Змей).

\item Болтливая пустота "--- голос, слышимый на границе сна и бодрствования, не имеющий возраста, пола, интонации, как будто сотканный из окружающих звуков.
Чаще всего несёт полную чушь.
(<<Болтливая пустота тебе [по секрету] сказала?>>)

\item Имя "--- Антти Предательские Воды (Antti Treacherous Waters).

\item Разница между рекой и речушкой в языке цатрон определена совершенно точно "--- устройством мостов.
Любая водная жила, через которую можно перебросить навесной мост "--- речушка (со).
Если же требуются дополнительные опоры (быки, острова, понтоны либо гребни), то это река (ху).
В цатроне существует ещё один термин, обозначающий направленный ток воды "--- эрхо (река без берегов, т.е. морское течение).
Петлевое течение у Кристалла на цатроне называется Эрхо'люаэсакх (Течение [того, кого] обманули в [последний] раз).

\item Судьба жывотных "--- Цапка, Серебряный, Нейросеть.

\item Азуритовые розы, кровавые крокоитовые иглы, сколецитовые хризантемы.

\item Я не отдаю тебе свои волосы, я отдаю тебе время, которое они растут.

\item In the shadow of the mushroom cloud "--- В тени ядерного гриба.

\item Описать Тхитрон.

\item Проработать топографию храма и (важно!) место взрыва.

\item \textbf{[Путь жреца]} Воспоминания о пыли "--- смысл-образующие и должны предварять всё прочее.
Что делать?

\item Сверстники!
Где они?
Чханэ, Ликхэ, Столбик, сестрёнки, всё!
Где храмовая молодёжь?
Где молодые крестьяне/ремесленники?

\item Смена сословия.

\item Лака и кураресодержащие растения на храмовых землях.

\item Субкультуры?

\item Рыбак (Хэмингуэй).

\item Землетрясения, смертельная пустыня.

\item Жизнь других видов.

\item Второй шанс разрушителю.

\item Превращение отступления в победу.

\item Решение оставить Тхартавирт (<<Город не важен, важны люди>>).

\item Решение об исходе "--- <<Будем держать врага с одной стороны>>.

\item Осенняя прогулка с Чханэ (ForgottenTale).

\item Кон-Тики и Полярный Водоворот.

\item Философские школы: <<Победа вездесуща, подобно воде, но лишь холодный разум может собрать влагу из воздуха>> "--- Плющ и Капля Росы.
<<Сердце врага "--- твоё сердце>> "--- Путь Ягуара.
<<Ярость как ураган "--- не разобьёт, так сточит>> "--- Десять Песчинок.

\item Драконья Пустошь.
Виа Галоледика (Скользкая дорога).

\item Козья ножка для рукояти.

\item Больше упоротых ритуалов и разговоров с духами под наркотой!
Верования сели: Сестра Дождь (Сестра Колодец) "--- персонаж, иногда идентифицируемый с Обнимающим Ситом, иногда отдельный.
Корневики "--- трудяги, ремонтирующие жилы джунглей.
Духи леса.
Каменные духи.
Пристанище.
Опалитель, он же Деа Солар.
Червь-узурпатор (Старая Личинка) "--- персонаж западных сели (вероятно, отголоски верований Синего Колена).
В обмен на мёртвое тело кусает жилу джунглей, пропуская таким образом душу к пристанищу.
У Синего Колена Старая Личинка "--- подписывает с душой договор на новую жизнь.
Зелёная Звенящая Вода "--- знак из культа дельфинов Среднего Моря.

\item Нужна хронология.
Очень.

\item Езда на оленях.
Обучение, прочее.

\item Схема координат для боя.

\item Манэ и Лимнэ "--- женщина-краска и женщина-лоскуток.

\item Доработать доспехи.

\item Гроза во время Дела Перекрёстка!

\item Мифологичность сознания!

\item Солнечные печи.

\item Одна сцена переходит в другую.

\item Промыслы.
Земледелие, животноводство, аквакультура, птицеводство, виноградарство, рыболовство, цветоводство, пчеловодство (бортничество), собирательство, охота.
Пермакультура!
Ремёсла: камнерезы, ювелиры, деревщики, оружейники, кожевники, портные, сукновалы.
Ювелиров должно быть больше, чем кузнецов.

\item Эйраки использует жуков для сбора эманаций.
Знак кирпича инициируется жуком.

\item На крыше храма "--- столбы из особого дерева, глушащие звуки (молчащий кедр).
Зона молчания.

\item У тси нет слова <<убивать>>.
Есть слово <<разрушать>>.
\textbf{Серьёзная правка!}

\item Школа.
Тренировка органов чувств (вкус "--- конфеты), обоняния, осязание, зрение, слух.
Тренировка контроля над чувствами.
Управление эмоциями (И тут я понял, что Трукхвал имеет в виду.)

\item Животноводство у сели.
Кур не держали взаперти "--- они свободно гуляли где хотели.
Им просто устраивали уютные гнёзда, и птицы возвращались, чтобы откладывать яйца.
Понятия <<моя курица>> не существовало, просто была некая общая популяция, которой пользовались все (пермакультура?)

\item (всякое) Под порогом захоранивали предков.
Порог мыли тщательнее, чем остальной дом.
Кур и прочее забивали на пороге, чтобы кровь питала лежащих под порогом.
Охраняли дом.

\item В подношение всегда входил витой шнурок.
Его вили как можно туже "--- количество витков символизировало годы памяти среди народа.
Затем его обрезали одним ударом, специально заточенным ножом "--- это символизировало быструю лёгкую смерть (изменение свершившегося факта ритуалом).

\item Глава 10.
Не вполне ясно, как Небо стал командиром.

\item Кусачка.
Кто он такой, откуда взялся, вплести его в сюжет.

\item Законы сели.
Пленные 2 года жили в обществе сели, затем их отпускали, если не требовались жертвы (?).
Поработать над изгнанием "--- за что и как.

\item Суть экономики.
Она есть, но у каждого человека есть неэкономические ресурсы для выживания "--- кусок земли, инструменты, оружие.

\item Терраформирование.

\item Вживлённый пучок волос "--- признак идолов Живодёра.

\item У народа трами сражаются только женщины.
Обрезают правое крыло.

\item Планты прекрасно поглощают воду кожей.

\item Молчащие считают ношение одежды трусостью.

\item Маршрут Ликхмаса (побег): Тхитрон "--- Ихслантекхо "--- сплав по Ху'тресоааса "--- Тхартавирт.

\item Слова цатрона: m\=am\`a, p\r{a}p\`a, s\r{\i}s\`\i "--- любая женщина в доме, tch\'atch\`a "--- любой мужчина в доме.

\item Чувство магнитного поля, блеать.

\item ЛУННЫЕ САДЫ.
Птица преподносит дары "--- ягоды.
Герой придумывает аргументы против.
Синяя ягода "--- знание языков, красная ягода "--- исцеление, етц.
Опасность синей ягоды "--- нельзя слушать речи летающих змей.
Они искусны в речах, и если проведают, что ты говоришь на знакомом языке "--- убедят в чём угодно.
Лишь птицы сравнятся с ними в красноречии.
Птицы вели со змеями войну (орёл со змеёй в клюве).
ГГ нашёл змея и, не давая ему говорить, схватил за усы, завязал ему усами пасть и полетел к Луна.

\item Смысловая нагрузка вышивки на одежде.
В Мягкие Руки тот, кто желал найти пару, надевал одежду на голое тело.

\item Земля выдаётся каждому.

\item Насилие "--- пользование собственностью без согласия владельца.
Тело, труд, дом, личные вещи.
Разрушение.

\item Хесематр "--- язык Живодёра.
Хесели "--- пиджин Омута Духов.
Хесетрон "--- язык Молчащих.
Языки хака "--- северный, <<дикий>>.
Диалекты сели "--- южный, западный, северный, сотронский.
Пылерои "--- 6 языков. Стрелохвосты "--- 13 языков, из них 2 "--- языки Голубого Зеркала.
\end{enumerate}

\section{История сели}

\begin{enumerate}
\item предки пришли с места Тхидэ;
\item первое поселение "--- Тхартавирт, торговали с Кахраханом (тогда ещё поселением царрокх);
\item основание Тхитрона, Ихслантекхо и Травинхала (бассейн Ху'тресоааса);
\item Столкновение с молодым государством тенку (текнек-мен), Первая приречная война.
Текнеки отброшены за Пыльное предгорье.
Возникновение первого святилища "--- Весёлый Волок.
\item Раскол государства по видовому признаку.
Идолы заявили свои права на северные города.
Вторая приречная война, отвоевание Тхитрона, Ихслантекхо и Травинхала людьми.
Разделение идолов на Живодёрских (Центральных) и Молчащих.
\item Война северных царрокх с Молчащими в союзе с сели.
Формирование народа хака.
\item Война Живодёра с Фиолетовым Союзом.
Поражение Синего колена, исход за Реку Кувшинок.
\end{enumerate}

\section{Ноа}

\begin{itemize}
\item города окружены стенами "--- против ветров.
\item выращивают культуры под навесами с сетью, смачиваемой водой.
Останавливает излишнюю радиацию.
\item Носят зонты-копья, зонты-фонари или зонты-духовые ружья.
\item Опреснители морской воды.
\item Солнечные печи.
\end{itemize}

\section{Внешний вид героев}

\begin{description}
\item[Митхэ ар'Кахр] "--- маленького роста, щуплая, короткие черные кудрявые волосы с маленькими <<рыбками>> в три бусины, яркие светло-зелёные глаза.
Рваный шрам на щеке, губе и подбородке справа, татуировка Плачущего Ягуара.
Отсутствует правая грудь, во рту сломан верхний резец.
\item[Чханэ ар'Катхар] "--- очень высокого роста, оцелотовая окраска, чёрное пятно на лбу справа, оранжевые матовые глаза, слегка опалесцирующие.
Волосы жёсткие, волнистые, кофейного цвета с рыжиной, валяные рыбки в три хвоста "--- два голубых, один белый.
Шрам крест-накрест на левой щеке, на животе "--- шрам от кхаагатра, мелкие шрамы на бёдрах.
\item[Кхохо ар'Хетр] "--- среднего роста, крепкого сложения, каштановые прямые волосы, торчащие в разные стороны, рыбки плетёные в два хвоста.
Глаза тёплого каре-зелёного цвета.
На подбородке татуировка в виде пяти клиньев, слева шрам типа <<улыбка Глазго>>.
На груди татуировка <<двукрылое солнце>> (знак провокаторов, стягивающих внимание противника на себя), на дельтах "--- осьминоги, щупальца заходят на лопатки, ключицы и обвивают плечи.
\item[Ситрис ар'Эр] "--- среднего роста, крепкого сложения, кудрявые чёрные волосы, собранные на затылке в три пучка трубочками, трубочки стянуты верёвкой (Кахрахан).
Глаза матовые чёрные.
На лбу слева короткий вертикальный шрам.
На левой руке татуированный <<рукав>> со звездчатыми мотивами.
В ушах серьги-кольца с игольчатыми подвесками "--- стиль пиратов-ноа.
Спина испещрена шрамами от хлыста, из-за чего рубаху снимает редко даже перед сном.
\item[Ликхэ ар'Трукх] "--- среднего роста, немного полная, с большими, чуть обвисшими грудями и тяжёлыми округлыми ягодицами.
Глаза яркие тёмно-серые, медиальный конец правой брови иссечён.
Волосы светло-каштановые, обрезаны до подбородка, одна плетёная <<рыбка>> на затылке.
Носит блестящие золотые серёжки с сапфирами, справа окаймляющие ухо полностью.
Татуировка на загривке: Кхар-защитник с очень большими, хорошо прорисованными глазами и надпись на цатроне <<Я тебя вижу>>.
Имеет привычку крепко прижимать правую руку к груди, если в руке ничего нет.
\item[Анкарьяль] "--- Хатлам "--- ближе к Елизавете Кузнецовой.
Скорбящие "--- ближе к Елизавете Пустовойт.
\end{description}

\section{Русско-английская таблица соответствия}

Айну Крыло Удачи \hfill Ajnu the Luckwing\\
акбас \hfill jacbas\\
Анкарьяль Кровавый Шторм (Красный Ветер) \hfill Ancarjal the Bloodstorm (the Redbreeze)\\
антарида \hfill antaryde\\
Аркадиу Шакал Чрева (Падальщик) \hfill Arcadiju the Womb Jackal (the Tomb Jackal)\\
Атрис \hfill \"{A}\={a}tr\v{\i}s\\
Аурвелий Амвросий \hfill Aurweli Amwurosi\\
Бамбуковая клетка \hfill Bamboo Birdcage\\
Безумный \hfill the Senseless\\
Безымянный \hfill the Nameless\\
Беспамятный постоялый двор \hfill Memoryless Inn\\
благородный баньян \hfill noble banyan\\
боевой вождь \hfill warchief\\
Верхний Этаж \hfill the Upstairs\\
Ветви Звезды \hfill Star Forks\\
Ветви Земли \hfill Earth Forks\\
Ветви Ночи \hfill Night Forks\\
Ветви Смерча \hfill Swirl Forks\\
Ветер-Завивает-Волосы \hfill Wind-Curling-Hair\\
Грейсвольд Каменный Молот \hfill Grejsvold the Stonehammer\\
даритель \hfill giver\\
Двор \hfill the House\\
Девиантные Ветви \hfill Deviant Forks\\
демиург \hfill demiurge\\
Дорге \hfill Dourgue\\
Ду-Си \hfill Du-Sie\\
Закрытая-Колба-Жизни \hfill Sealed-Life-Flask\\
кедр молчащий \hfill silent cedar\\
Кольбе Старое Изречение \hfill Colbe the Old Maxim\\
кормилец \hfill nurse\\
Король-жрец \hfill Priest-king\\
Красный Картель \hfill Scarlett Cartel\\
кихотр \hfill k\^{\i}h\~{o}tr\\
конус-очаг \hfill conical hearth\\
Конфетка \hfill Candy\\
котловина Кон-Тики \hfill Crater of Con-Tici?\\
кукхватр \hfill k\`{u}kchu\={a}tr\\
кхааготр \hfill kch\^{a}\={a}g\~{o}tr\\
Кхарас \hfill Kch\'{a}r\v{a}s\\
Кхатрим \hfill Kch\r{a}tr\"{\i}m\\
кхене (мера времени) \hfill kch\"{e}no\^{e}\\
кхене (мера длины) \hfill kch\={e}no\^{e}\\
Кхотлам \hfill Kch\={o}tl\'{a}m\\
Кхохо \hfill Kch\`{o}h\^{o}\\
лаковый сок \hfill l\={a}\"{a}k\^{a} sap\\
легат \hfill legate\\
лехэ \hfill l\={e}cho\`{e}\\
Ликас \hfill L\^{\i}k\v{a}s\\
Ликхмас \hfill L\={\i}kchm\r{a}s\\
Ликхэ \hfill L\^{\i}kcho\^{e}\\
Лусафейру Лёгкая Ладонь \hfill Lusafejru the Lightweight Palm\\
Люм \hfill Lo\~{e}m\\
максим \hfill maccsim\\
Марас \hfill M\"{a}\={a}r\v{a}s\\
микоргет \hfill micorget\\
Мимоза Шёлковая Сталь \hfill Mimosa the Silken Steel\\
Митликх \hfill M\={\i}tl\={\i}kch\\
Митрис \hfill M\={\i}tr\={\i}s\\
Митхэ \hfill M\={\i}tcho\^{e}\\
молитвенный мак \hfill praypoppy\\
молчание (отн.) \hfill quiet\\
насильник \hfill violator\\
Нижний Этаж \hfill the Downstairs\\
омега-поле \hfill omega-field\\
Орден Преисподней \hfill the Order of Netherworld\\
Орешек (имя) \hfill Nut-Nut\\
Пирожок \hfill Cupcake\\
питомец \hfill nurseling\\
Подсолнух-Бросает-Семена \hfill Sunflower-Dropping-Seeds\\
поле Кохани-Вейерманна (ПКВ) \hfill Cojani-Wajermann field, CWF\\
потомок \hfill scion\\
предок \hfill root\\
Пчела-Нюхает-Вереск \hfill Bee-Sniffing-Heather\\
Рабе Юный \hfill Rabe the Young\\
разрушитель \hfill destructor\\
родильница \hfill bearer\\
Сад \hfill the Garden\\
сахарная муха \hfill sugarfly\\
сейхмар \hfill seijmar\\
сели \hfill S\r{e}l\={\i}\\
Самаолу \hfill Samajolu\\
Сиртху \hfill S\r{\i}rtch\'{u}\\
Ситрис \hfill S\~{\i}tr\v{\i}s\\
сиу'сиу \hfill s\~{\i}u-s\~{\i}u\\
Сиэхено Опаловый Глаз \hfill Siejeno the Opal Eye\\
Согхо \hfill S\"{o}gch\={o}\\
Союз Воронёной Стали \hfill Blued Steel Union\\
Стигма Чёрная Звезда \hfill Stijma the Blackhole\\
Cуществует-Хорошее-Небо \hfill Existing-Good-Sky\\
Таниа Янтарь \hfill Tanija the Amber\\
Тахиро Молниеносный \hfill Tajiro the Thunderbolt\\
Темнотой-Сотканный-Заяц \hfill Darkness-Woven-Hare\\
Тепло-Полуночного-Костра \hfill Warm-Midnight-Campfire\\
терция \hfill tercia\\
тишина (абс.) \hfill silence\\
Тра-Ренкхаль \hfill Tr\r{a}-R\={e}nkch\'{a}l\\
трааа\ldotst \hfill tr\={a}\"{a}\ldotst \\
Трукхвал \hfill Tr\`{u}kchu\r{a}l\\
Тхаммитр (Чхаммитра) \hfill Tch\`{a}mm\={\i}tr (Chh\`{a}mm\={\i}tr\^{a}i)\\
Тханэ \hfill Tch\r{a}no\^{e}\\
Тхартху \hfill Tch\~{a}rtch\'{u}\\
Тхартхаахитр \hfill Tch\~{a}rtch\"{a}\={a}h\r{\i}tr\\
Тхитрон \hfill Tch\"{\i}tr\`{o}n\\
Тси-Ди \hfill Qi-Di\\
тху \hfill tch\`{u}\\
Уголёк \hfill Coal\\
улица Стриженого Кактуса \hfill Shaved Cactus street\\
урождённый бог \hfill born god\\
урождённый демон \hfill born daemon\\
урождённый сапиент \hfill born sapient\\
Харматр \hfill H\r{a}rrm\`{a}tr\\
Хитрам \hfill Ch\"{\i}tr\'{a}m\\
хоргет \hfill jorget\\
Храм \hfill the Temple\\
хранитель \hfill keeper\\
Хрустально-Чистый-Фонтан \hfill Crystal-Pure-Fountain\\
хэ \hfill cho\^{e}\\
хэситр \hfill ho\`{e}s\={\i}tr\\
цатрон \hfill Te's\'{a}tr\v{o}n\\
Цех \hfill the Workshop\\
Чханэ \hfill Chh\r{a}n\^{e}i\\
Штрой Кольцо Дыма \hfill Stroji the Smoke Ring\\
Эйраки Мороз \hfill Ejraci the Cold\\
Эрликх \hfill O\r{e}rl\'{\i}kch\\
Эрхэ \hfill O\r{e}rcho\^{e}\\
