\begin{titlepage}
{\centering
{~\par}
\vspace{0.25\textheight}
{\LARGE 
\mulang{$0$}
{Эмиль Хельгасон}
{Emil Helgason}}
\par
\vspace{1.0cm}
\rule{8cm}{1pt}\par
\vspace{0.3cm}
{\Huge 
\mulang{$0$}
{ДАЛЕКИЕ РЕЧИ}
{TALKS OF FARAWAY}
\par}
\vspace{0.3cm}
\rule{8cm}{1pt}\par
\vspace{1.0cm}
{\Large \textit{
\mulang{$0$}
{Фантастический роман}
{Science fiction}
}\par}
\vspace{0.5cm}
\razd\par
\vspace{1.0cm}

\begin{minipage}{0.3\textwidth}
\textbf{Начало [БЗВ]:} 13 августа 2012

\textbf{Начало [СКР]:} 3 мая 2015
\end{minipage}
\par
\vfill
{\Large
\mulang{$0$}
{Создано в}
{Created by}
\XeLaTeX}\par}
\newpage
\thispagestyle{requisit}~
{\LARGE
\mulang{$0$}
{Благодарности}
{Thanks}
\par}
\vspace{0.5cm}
{\bfseries
\mulang{$0$}
{За вдохновение музыкой:}
{For inspirational music:}\par}

\mulang{$0$}
{Наталия Игоревна <<Тэм Гринхилл>> Новикова}
{Natalia ``Tam Greenhill'' Novikova}

\mulang{$0$}
{Эйнар <<Kvitrafn>> Селвик}
{Einar ``Kvitrafn'' Selvik}

\mulang{$0$}
{Наталья Андреевна <<Хелависа>> О'Шей}
{Natalia ``Hellawes'' O'Shea}

\mulang{$0$}
{Группа <<Blind Guardian>>}
{Blind Guardian band}

\mulang{$0$}
{Антти Мартикайнен}
{Antti Martikainen}

\mulang{$0$}
{Айвёр Полсдоттир}
{Eivør Pálsdóttir}

\mulang{$0$}
{Мэтт Оулмен}
{Matt Uelmen}

\mulang{$0$}
{Рамин Джавади}
{Ramin Djawadi}

\mulang{$0$}
{Группа <<Faun>>}
{Faun band}

\vspace{0.5cm}
{\bfseries
\mulang{$0$}
{За вдохновение словом:}
{For inspirational stories:}
\par}

\mulang{$0$}
{Франклин <<Фрэнк>> Герберт}
{Franklin Patrick ``Frank'' Herbert, Jr.}

\mulang{$0$}
{Иван Антонович Ефремов}
{Ivan Efremov}

\mulang{$0$}
{Николай Даниилович <<Ник>> Перумов}
{Nikolay ``Nik'' Perumov}

\mulang{$0$}
{Курт Воннегут}
{Kurt Vonnegut, Jr.}

\mulang{$0$}
{Эрнест Хемингуэй}
{Ernest Miller Hemingway}

\mulang{$0$}
{Антуан де Сент-Экзюпери}
{Antoine Marie Jean-Baptiste Roger de Saint-Exupéry}

\mulang{$0$}
{Туве Янссон}
{Tove Jansson}

\mulang{$0$}
{Джоан Роулинг}
{Joanne ``J.K.'' Rowling}

\vspace{0.5cm}
{\bfseries
\mulang{$0$}
{За вдохновение образом:}
{For inspirational images:}
\par}

\mulang{$0$}
{Хаяо Миядзаки}
{Hayao Miyazaki}

\mulang{$0$}
{Брайан Хелгеленд}
{Brian Thomas Helgeland}

\mulang{$0$}
{Сэр Питер Джексон}
{Sir Piter Robert Jackson}

\mulang{$0$}
{Джон Ву}
{Ng Yu-Sum (John Woo)}

\mulang{$0$}
{Лана и Лилли Вачовски}
{Lana and Lilly Wachowsky}

\mulang{$0$}
{Том Тыквер}
{Tom Tykwer}

\vspace{0.5cm}
{\bfseries
\mulang{$0$}
{За дружеский совет и поддержку:}
{For friendly support and cooperation:}
\par}

\mulang{$0$}
{Никита Владимирович Скабцов}
{Nikita Skabcovs}

\mulang{$0$}
{Марина Олеговна Куличёва}
{Marina Kulichova}

\vspace{0.5cm}
{\bfseries
\mulang{$0$}
{Отдельная благодарность:}
{Special:}\par}

\mulang{$0$}
{Марина <<Мао>> Ратникова}
{Marina ``Mao'' Ratnikova}

\mulang{$0$}
{Марьям Мирзахани}
{Maryam Mirzakhani}

\vspace{0.5cm}

\emph{Посвящается Мариам Мирзахани,\\
профессору Стэнфордского университета,\\
которой я очень хотел пожать руку.\\
Но не всё бывает так, как мы хотим.}
\end{titlepage}

\tableofcontents

\part{Путь}

\chapter*{Предисловие}
\addcontentsline{toc}{chapter}{Предисловие}

\epigraph
{Знаете, что в искусстве самое страшное для консерваторов?
Как бы высоко ни взлетела фантазия автора, это не будет пределом возможного.
Если речь пойдёт о смехотворном сроке в два-три миллиона лет, я не поставлю ни на великую страну, ни на истинную религию, ни на незабвенный язык, ни даже на богоизбранный биологический вид "--- ни на что, чем гордятся и за что убивают.
Всё канет в Лету\footnote
{Река времени из эллатинских легенд Древней Земли. \authornote}.
И многое "--- бесследно.}
{Мартин Охсенкнехт, драматург Театра Снов.
Эпоха Последней Войны}

Тишина.
Слово едва ли не прекрасней его собственной сути.
Словарь языка Эй даёт ему следующее определение: <<Совокупность незначимых надпороговых сигналов, воспринимаемых рецептором>>.
Для космических радаров тишина "--- это белый электромагнитный шум.
Человек-тагуа может назвать тишиной и ропот ночного города, и звёздное небо, и нежный ветер, касающийся волос.
Для хоргета же тишина "--- это вечное, непрекращающееся колебание поля Кохани"--~Вейерманна.

Тишина не равнозначна безмолвию.
Безмолвие "--- <<отсутствие надпороговых сигналов>> "--- давит и угнетает.
Тишина же "--- это ласка, расслабляющий массаж для рецептора.
В тишине система становится чувствительной и замечает то, что ранее оставалось в тени.

Последний год был годом тишины.
Орден Преисподней наградил меня отдыхом за военные заслуги.
Также мне разрешили изучить новую специализацию "--- диктиология.

"--*Зачем тебе это нужно? "--- удивлённо спросил Грейсвольд, узнав о моём решении.

"--*Я чересчур долго откладывал обучение, "--- ответил я.
"--- Опыт "--- почва для уверенности, но на опыте без теории растут самые серьёзные ошибки.

Базовое обучение на диктиолога занимает двадцать суток по времени Капитула.
Моё тело провело их в медикаментозной коме; демону же спать не приходилось "--- стыковка новых модулей, загрузка пакетов, устранение несовместимостей и настройка окружений, утомительный разбор синтаксиса и структуры, а под конец "--- обучение
дополнительного нейроконтура, чтобы я мог пользоваться новыми надстройками с наилучшей производительностью.

Отпуск мне рекомендовали посвятить научным исследованиям, чтобы проверить модули на практике.
Работа скучная и нетворческая "--- разбор чужих отчётов, записей и изображений, выявление закономерностей в странных, порой диких завитушках сапиентной культуры.

Единственное отрада "--- научная работа по народам Тра-Ренкхаля.
Эта планета до последнего времени была тёмным пятном на небосводе, и я, занесённый туда ветром войны, принял непосредственное участие в исследовании.

Эта книга "--- художественное приложение к работе.
Повесть о прекрасной планете, населённой прекрасным народом.
Изнывающим от скуки неинтересны истории, конец которых заранее известен;
их мир определён, они жаждут неожиданностей.
В этой книге неожиданностей не будет.
Для меня удовольствием было её писать.
Хотелось бы, чтобы вы нашли удовольствие в чтении.

Скажу сразу, дорогие читатели: ваша реакция на книгу будет тщательно проанализирована агентами Ордена Преисподней.
На основании ваших впечатлений вынесут вердикт, усвоил ли я обучение и для какой работы я подхожу.
Но пусть вас это не смущает.
Я писал не ради тёплого места в отделе культурологии.
Существовали более тривиальные способы доказать профессиональную пригодность.

Книга написана на двух языках.
Язык Эй и мой родной сектум-лингва, язык Ада и язык Картеля, речи двух близких и далёких миров.
Несмотря на различия, призванные сохранить колорит повествования, оба варианта идентичны.
Я старался ничего не потерять.

\subsection*{Физиолингвистика}

Сапиентные виды, о которых пойдёт речь, не совсем обычны.
Думаю, многие наслышаны о Девиантных Ветвях, созданных демиургами или сапиентами с нуля.
Однако во Вселенной существует целая группа Девиантных Ветвей, которые создали сами себя.
Вы прекрасно знаете, о ком я говорю.
И вполне естественно предположить, что особенности их физиологии и анатомии наложили отпечаток на сигнальную систему.

Как выглядит мир в глазах того, кто воспринимает инфракрасные, ультрафиолетовые и рентгеновские волны?
Что слышат уши того, кто улавливает инфра- и ультразвук?
Какие ощущения дают магнитные поля?
К сожалению, для вас это останется тайной за семью печатями.
Я не стал размениваться на образы "--- это похоже на попытку рассказать собаке, чем фиолетовый отличается от зелёного.
Образы искажают главную мысль "--- тси родились и живут в мире, на который настроены их рецепторы, и он для них обычен так же, как обычен ваш мир для вас.

В языках тси имеются термины для описания ощущений.
В конце книги я дал их научное обоснование; представить, что чувствуют эти сапиенты, оставляю вам.
При словах <<солнечная кисть>> можно подумать о ярких одуванчиках, а <<каменная ярость>> может вызвать ассоциацию с мертвенным свечением фосфора.
Всё это неважно.
Сколько бы ни существовало глаз, мир всегда один.

Пара слов о языке сели.

Люди-тси имеют сложный голосовой аппарат, отдалённо напоминающий птичий "--- так называемая гортанная цитра, позволяющая воспроизвести любой звук в их слуховом диапазоне.
Голосовая часть языка сели "--- звукоподражательная\footnote
{В частности, названия всех животных на языке сели представляют собой почти точное вопроизведение их криков или звуков.
К примеру, самец согхо называется <<тиулиулафью>>, домашний петух "--- <<кукареку>>.
Детёныши и самки животных также имеют своё звукоподражательное название.
Забавные аспекты "--- распил древесины (процесс, которого не знали древние тси) также обозначается звуком, который издаёт пила. \authornote},
для её записи разработана специальная система обозначений на основе письменности Молчащих лесов.
Между тем сели никогда не используют только голос "--- в зависимости от ситуации они переходят то на шумовую, то на жестовую речь.

Вы представили разговор на языке сели?
Безусловно, это сложно.
Приведу слова поэтессы Эрхэ Колокольчик, для которой этот язык был неисчерпаемым источником вдохновения:

\begin{quote}
<<Урчание оцелота, кваканье жаб, свист согхо, крики множества птиц и зверей, жужжание светлячков и скрежет ночных сверчков "--- все звуки джунглей в устах одного человека>>.
\end{quote}

Добавлю, что это не столько поэтическое сравнение, сколько точно описанное впечатление постороннего.

В силу того, что лексемы языка тси невозможно транслитерировать с помощью СЧФ\footnote
{Стандартная человеческая фонетика "--- спектр звуков, которые способен произвести неизменённый голосовой аппарат особи Homo homo sapiens. \authornote},
я калькировал\footnote
{В соответствии с правилами калькирования терминов, установленными <<Инструкцией по локализации для отдела 44>>, версией 11-A0FD49.
Ст. 1 требует использовать калькирование в случае критического нарушения удобочитаемости (п. 1) и/или невозможности использования СЧФ (п. 3).
Ст. 4, п. 5 позволяет поместить определение кальки в любое подходящее место или опустить его, если определение не требуется. \authornote}
имена тси и большую часть терминов.

Также в языке тси имеются слова, которые невозможно калькировать (так называемые <<первичные>>, не имеющие родительского корня), и поэтому я переопределил\footnote
{В соответствии с правилами переопределения терминов, установленными <<Инструкцией по локализации для отдела 44>>, версией 11-A0FD49.
Ст. 1 требует использовать переопределение в случае невозможности калькирования (п. 4) при критическом нарушении удобочитаемости (п. 1) и/или невозможности использования СЧФ (п. 3).
Ст. 4, п. 4 требует вынести переопределение терминов в предисловие. \authornote}
термины сектум-лингва и Эй.
Прошу уделить этому параграфу пристальное внимание.

~

\begin{description}
\item[Даритель] "--- существо, которое отдало индивиду часть генома и/или клеточного аппарата.
Слово употребляется в контексте биологического родства.
У сели дарителями называются мужчина и женщина, зачавшие ребёнка.
У тси дарителей могло быть любое количество "--- от одного (клонирование, партеногенез) до нескольких десятков (сборка генома).
\item[Хранитель] "--- существо, которое приняло от индивида часть генома и/или клеточного аппарата.
\item[Друг] "--- любое близкое существо индивида.
\item[Кормилец] "--- существо, которое занималось воспитанием индивида.
По отношению к кому-либо употребляется в контексте воспитания.
Количество кормильцев обычно от одного до пяти.
\item[Питомец] "--- существо, воспитанием которого занимался индивид.
Синонимом является выражение <<чей-либо ребёнок>>.
\item[Любовник] "--- существо, связанное с индивидом регулярными половыми связями.
\item[Потомок (наследник)] "--- любое существо, находящееся в нижнем вертикальном родстве с индивидом.
Порядковый номер обозначает поколение.
Хранители также являются первым потомками.
\item[Предок (прародитель)] "--- любое существо, находящееся в верхнем вертикальном родстве с индивидом.
Порядковый номер обозначает поколение.
Дарители являются также первыми предками.
\item[Родильница] "--- существо, которое обеспечило индивиду внутриутробное развитие, в том числе (предположительно) кольцевая теплица.
Слово было в ходу у тси, у народа сели же стало высокопарным синонимом дарительницы, так как в подавляющем числе случаев родильница и дарительница "--- одно и то же существо\footnote
{В языке тси эти слова не имеют рода.
Сели иногда употребляют также более точные заимствованные термины для обозначения родства (се "--- первый по счёту первый наследник, тхар "--- первая наследница се, кхантхор "--- первый наследник брата второго прародителя и пр.), чаще при разговоре с представителями других народов. \authornote}.
\item[Ребёнок] "--- существо, не прошедшее половую дифференциацию.
Слово не является гиперонимом по отношению к словам <<мальчик>> и <<девочка>>.
Мальчиками и девочками называются молодые люди, прошедшие половую дифференциацию, но не достигшие возраста зрелости "--- двадцати пяти дождей.\\
\item[Дом] "--- место, где индивид осуществляет полноценный отдых.
Не путать с жилищем "--- строением, предназначенным для проживания\footnote
{Понятие дома у тси достаточно размытое.
Были случаи, когда существа большую часть жизни проводили в своём жилище, но также были случаи постоянного проживания в транспорте, в цеховых комнатах, в палатках, в гостях и даже на природе без достоверного снижения коэффициента личного комфорта.
Поэтому на первый план выходит полноценный отдых.
У сели сохранилась поговорка-каламбур <<Всегда превращай в дом место, где спишь>>.
Это отражает традицию поддерживать комфортный порядок не только на месте сна, но и на прилегающей территории. \authornote}.
\end{description}

Для местного колорита я использовал подходящие термины из языка цатрон.
Сели используют цатрон для общения с другими видами ветви Хуманы (их представителям выучить язык сели практически невозможно) и при официальном общении.
По установившейся традиции детям сели дают три имени "--- классическое трёхзнаковое имя тси, однознаковое <<домашнее>> имя, а также родоплеменное имя царрокх с изменённым порядком наследования родового слога.

Определения терминов вы можете найти в словарике в конце книги.
Если кому-то кажутся скучными постоянные консультации у словаря, лучше закройте книгу прямо сейчас "--- вы всё равно не дочитаете её до конца.

~

~

Итак, сейчас на Капитуле 13.004.56281.4, и я, Аркадиу Талианский Шакал, начинаю свою повесть.

\chapter*{Интерлюдия I. Безумный}

\addcontentsline{toc}{chapter}{Интерлюдия I. Безумный}

\textbf{Легенда об обретении}

Прибыл однажды в город Харматр\footnote
{Легендарное место в восточных джунглях.
В настоящее время учёные Ада склоняются к мысли, что Гибельные Руины и Харматр "--- одно и то же место. \authornote}
с запада вестник, ростом в два цана, в сером одеянии.
Он прошёл через город, поднялся на ступени храма и повелел собрать народ.

<<Слушайте! Слушайте! "--- возвестил гонец.
"--- Я пришёл к вам от Безымянного, творца этого мира и этих небес.
Он отвёл беду от земли и от моря, из тех бед, что гасят звёзды и раскалывают горы до основания.
Он пострадал за нас, и лишь жертвенная кровь, лишь муки умирающих могут спасти его!>>

Притихли люди.
Посовещались они и решили принести Безымянному жертвы.
Как вдруг вперёд вышла седая старуха, знавшая ещё строителей древнего города и варившая им пищу.

<<Если бы мне выпало защитить от беды питомцев, я бы не требовала ни платы, ни поклонения, "--- сказала старуха.
"--- Только безумный, спасая детей, восполняет силы их кровью!>>

Согласились люди со старой женщиной и разошлись.
Гонец ухмыльнулся и покинул город.

Прошла декада, и прибыл с запада второй [вестник, ростом] в три цана, в зелёном [одеянии].
Он прошёл через город, и народ сам стекался на площадь, ожидая его слова.

<<Слушайте! Слушайте! "--- возвестил гонец.
"--- Я пришёл к вам от Безымянного, творца этих джунглей и этой реки.
Он сотворил вас, недостойных детей, и дал вам мир для услады и размножения.
Лишь жертвенная кровь, лишь муки умирающих могут стать ему достойной платой!>>

Притихли люди.
Посовещались они и решили принести Безымянному жертвы.
Однако вперёд вышла та же старуха и снова попросила слова.

<<Безымянный создал меня по собственным лекалам и прихоти.
Я не просила ни жизни, ни прекрасного мира, "--- сказала старуха.
"--- Только безумный будет делать брак и требовать с него оплату!>>

Согласились люди со старой женщиной и разошлись.
Гонец покинул город, гневно хмурясь.

Прошла декада, и прибыл с запада третий [вестник, ростом] в пять цан, в оранжевом [одеянии].
Он прошёл через город, и народ лишь вылез на крыши, чтобы видеть и слышать сказанное.

<<Слушайте! Слушайте! "--- возвестил гонец.
"--- Я пришёл к вам от Безымянного, творца этих гор и этого моря.
Известны ему дороги солнца и комет, известны тропы смертных и тех, кто выходит из бренных тел.
Запутан путь к пристанищу духов, и лишь жертвенная кровь, лишь муки умирающих помогут заблудшим!>>

Испугались люди, и раздались в городе крики отчаяния.
Но старуха не унималась:

<<[Путь к пристанищу] всегда и для всех был прям "--- смерть, "--- закричала она.
"--- Только безумный, будучи пойманным на лжи дважды, [соврёт] в третий раз!>>

Отругали люди гонца отборными словами, и покинул он город, тлея злобой унижения.

Прошло три дня, и потемнело небо, и заискрились в горах молнии, и задули ветры с неистовой силой.
Прибыл с запада великан, [ростом] в десять цан, в чёрном [одеянии], с глазами, пылающими Каменной яростью.
Он прошёл через город, но народ не встретил его и затаился в домах, охваченный страхом.

<<Слушайте все, "--- прогремел пришелец.
"--- Я "--- Безымянный, владыка ваш в жизни и смерти.
Я могу разрушить города и стереть в пыль ваши книги.
Я могу отнять ваши жизни и посмертие, и лишь жертвенная кровь, лишь муки умирающих могут отвести мой гнев>>.

Молчали люди, сидя в домах.
И лишь старуха вылезла на крышу, чтобы ответить.

Махнул рукой великан "--- и старуха [сгорела] в синем пламени, крича от нестерпимой боли.
Махнул [ещё раз] "--- и запылал уже весь город, и не было спасения ни взрослым, ни малышам, ни домашней скотине.

Пламя полыхало недолго.
Вскоре среди горячих руин, некогда прекраснейшем из городов, остались лишь двое "--- пришелец [ростом] в десять цан и ученик жреца, стоявший на площади и избегнувший огня.

\mulang{$0$}
{<<Как тебя зовут?>> "--- пророкотал великан.}
{``What's your name?'' the giant bellowed like echoes of far avalanche.}

\mulang{$0$}
{<<Ликас [мечтатель]>>, "--- ответил мальчик.}
{``L\^\i k\v{a}s [starry-eyed man],'' the boy answered.}

\mulang{$0$}
{<<Я нарекаю тебя Ликхмас [бегущая лань], ибо сейчас ты побежишь прочь, объятый смертным страхом, "--- захохотал великан.}
{``I name you L\=\i kchm\r{a}s [running doe], because right now you will run away, grasped by deadly fear,'' the giant laughed.}
\mulang{$0$}
{"--- Ты донесёшь до городов весть: противящихся воле Безымянного ждёт та же участь>>.}
{``You'll get my word to other cities: this fate shall be suffered by everyone who disobey the Nameless' will.''}

\mulang{$0$}
{<<Я передам весть всем, кому смогу, Безумный>>, "--- поклонился выживший.}
{``I shall tell everyone I can, the Senseless,'' the survivor bowed to the giant.}
\mulang{$0$}
{В его глазах застыли слёзы бессильного гнева.}
{His eyes filled with tears of powerless wrath.}

\mulang{$0$}
{<<Как ты меня назвал?!>> "--- пришёл в ярость посланец и одним взмахом обрушил остатки храма.}
{``What did you call me?!'' the messenger became enraged and razed the temple to the ground.}
\mulang{$0$}
{Столб пыли и грохот поднялись до небес, но не испугали того, кто в миг лишился дома.}
{Dust clouds and rumble had covered the sky, but couldn't frighten the one who's lost his home in a blink.}

\mulang{$0$}
{<<Ты дал мне имя, и я отплачу тебе тем же, "--- ответил он.}
{``You gave me a name, I did the same,'' he answered.}
\mulang{$0$}
{"--- Ты владыка жизни и смерти, но не [хозяин нашему] гневу.}
{``You are a lord of life and death, but you does not command [our] wrath.}
\mulang{$0$}
{Ты заставляешь замолчать, но не способен вернуть в горло сказанное.}
{You silence us, but you can not return a single word to the throat.''}

\mulang{$0$}
{<<Я уничтожу тебя!>> "--- заревел [великан].}
{``I will destroy you!'' [the giant] roared.}

\mulang{$0$}
{<<Так кому же доставить весть "--- мёртвым или живым?>> "--- с улыбкой уточнил ученик жреца.}
{``But who must get your message, dead or alive?'' the novice priest clarified with smile.}
\mulang{$0$}
{Не дождавшись [ответа, он] покинул руины Харматра, и великан не посмел его остановить.}
{[He] left ruins of H\r{a}rrm\`{a}tr without waiting for [an answer], and the giant dared not stop his way.}

\chapter{Послание}

\section{Охота}

% Лук 6 дождя 11998, Год Церемонии 18.

\epigraph
{Беги отсюда, беги,\\
Но не вспять,\\
Хотя истоки пути\\
Позади.\\
Узкой тропы\\
Извилиста щель,\\
Как утроба дракона, пожравшего\\
Братьев твоих.\\
~\\
Вперёд и вперёд\\
Сквозь кромешный мрак,\\
Пока не покажется впереди\\
Свет "--- начало в конце концов.}
{Торстейн фра Хамри.
<<На неверном пути>>}

\mulang{$0$}
{Охота на ягуара не начинается с желания его добыть.}
{Jaguar hunt never begins with wish to get this beast.}
\mulang{$0$}
{Так гласит пословица сели.}
{So the proverb of S\r{e}l\={\i} says.}
Лишь невежды идут в джунгли на крупного зверя;
\mulang{$0$}
{такая охота обычно становится последней "--- разочарование убивает лучше любых клыков.}
{a hunt like that is often the last one, because disappointment kills better than any fangs can do.}
Выходят за мелкой пичугой "--- порадовать домашних свежей дичью.
Выходят погулять и набрать с десяток сверчков "--- разыграть соседей и обеспечить им бессонную ночь.
Вот тут на тропе и оказывается четырёхпалый след.

Есть поверье, что приносить шкуру шире дверного проёма следует раз в четыре дождя.
Это даёт джунглям фору.
Тем не менее, крупные обитатели леса избегают человеческих троп;
и замеченный след, и пахучая мускусная метка равносильны гибели.
Впрочем, Цыплячью тропу зверьё жаловало, потому что и людям, и идолам было не до зверья "--- своё мясо дороже.

День выдался погожий, с собой у меня были ружьё, лук и вдоволь стрел.
\mulang{$0$}
{Почему бы не попробовать?}
{Why not try?}

На мою удачу, цепочка знаков потянулась не в Живодёрские леса, а к реке;
на берегу кот "--- а это был сильный четырёхлетний кот "--- остановился попить и поточить когти.
Затем след направился на запад, обходя на почтительном расстоянии северную заставу моста;
вероятно, целью зверя была тропа пекари, у которой он и заляжет в ожидании добычи.

Флоп, флоп "--- и с молодого пахучего тхальсар\footnote
{Шорея тёмная, или дерево Корвуса "--- дерево джунглей. \authornote}
свалилась великолепная пара сиу'сиу.
Смазанные кураре стрелы прервали ритуал ухаживания;
самец даже не успел спрятать фиолетовые шевроны.
Что ж, их вина "--- надо смотреть по сторонам.
За перья хорошо заплатят на рынке, а сестрёнки полакомятся любимыми копчёными крылышками.

Солнце уже клонилось к закату, когда я понял, что цель близка.

Я пробежал чуть поодаль и пополз по сильно пахнущей тропе пекари, зарываясь в подстилку и похрюкивая поросёнком "--- сигнал <<призыв кормильца>>.
Потерявшийся на тропе поросёнок "--- лакомый кусочек.
Если же хищник почует неладное, пиши пропало "--- даже самые отчаянные нападать на человека не решались.
Впрочем, был один давным-давно.
Хитрого старого кота нашла стрела, но его угодья "--- удивительна людская память! "--- так и остались Живодёрскими лесами\ldotst

Хрюк, уик-уик.
И ещё пять шагов\ldotst

Хрюк, уик-уик.
И ещё пять\ldotst

Вдруг в пятне темноты справа ясно проступил тёплый силуэт.
Я вскинул ружьё к губам и, резко выдохнув, откатился в сторону.

\mulang{$0$}
{Флоп.}
{Plop.}

Ягуар неловко приземлился на ватные лапы, попытался встать и упал снова.
По окрестостям разнёсся тихий утробный рёв, приводящий в трепет сельву.
Пасть оскалилась в последней кошачьей угрозе\ldotst

Я подполз к ягуару и, положив руку на приятную шерстистую мордочку, приоткрыл ему глаз "--- в точности так, как учил охотник, старый Сиртху.

<<Разделывай зверя только тогда, когда умрёт голова.
Звериные слёзы отольются змеиным ядом.
Не позволяй жестокости поселиться в твоих делах;
она есть начало большей жестокости>>.

Вскоре глаз кота остекленел.
\mulang{$0$}
{Я в последний раз сомкнул и разомкнул веки "--- кусочек хрустальной темноты оставался неподвижным.}
{I closed and then opened its eyelids one last time, but a piece of crystal darkness stayed still.}
Хлопнув рукой по мордочке, я потянулся за кинжалом.
Сумерки затопили лес, словно вышедшая из берегов Ху'тресоааса;
управиться со шкурой следовало быстро.

Не знаю, кто из лесных духов заставил меня обернуться, но он спас мне жизнь.
Мстительная тень мёртвого ягуара разочарованно рявкнула и прыгнула на меня.
Ружьё жалобно хрустнуло и рассыпалось осколками дерева;
девушка подняла меня за пояс, точно заячью тушку.

Мой кинжал прочертил сумерки снизу вверх "--- <<предупреждение Маликха>>, выводящее врага из строя без опасных ран.
Нападавшая отшвырнула меня, зарычав от боли;
спустя миг мы оба твёрдо стояли на ногах.

"--*Хэс!\footnote{Мир! (цатрон)} "--- крикнул я, выбросив вперёд ладонь.

Девушка замерла.
На ней красовалось неуместное в сельве платье цвета молодой листвы;
кофейные волосы скрыли лицо, отсвечивали лишь зубы и белки оцелотовых глаз.
С рассечённого подбородка и правого запястья сочилась тёмная кровь.
Левая рука до хруста сжимала обсидиановый нож.

Я только сейчас осознал, что никогда не встречался с противником такой силы.
Я невольно бросил взгляд на жреческий инструмент, так и лежавший возле тела ягуара;
мой нож был на месте.

\mulang{$0$}
{"--*Как твоё имя?}
{``What's your name?''}

Кормилица учила начинать разговор этой фразой.
Неважно, что собеседник секхар назад пытался перерезать тебе горло.
Имя "--- это то, что спрашивают у друга.
У врагов и жертв нет ни имён, ни лиц.

Девушка не ответила.
Она стояла в отточенной боевой позиции "--- начальная форма стиля Скорпион, предназначенного для боя в плотном строю и некогда очень популярного на Западе.
Поза внушала почтение, но я знал "--- без клешней и щита этот стиль почти бесполезен.

"--*Я не сделаю тебе больно, "--- я медленно спрятал кинжал и развёл руки в стороны.
"--- Меня зовут Ликхмас ар’Люм э’Тхитрон, "--- я по очереди назвал своё имя, родовой слог и название родного города.
"--- Мне просто посчастливилось напасть на след ягуара, стрелы и кинжал предназначены для него.
\mulang{$0$}
{Умереть мне на алтаре Безумного сегодня ночью, это правда.}
{Hope to die on the Senseless' altar tonight, it's true.''}

Девушка бросила взгляд на ягуара и подошла.
Нож медленно опустился.

\mulang{$0$}
{"--*Я Чханэ, "--- слова цатрон она выговаривала с сильным нажимом, свойственным западным сели.}
{``I'm Chh\r{a}n\^{e}i.'' She spelled Te's\'{a}tr\v{o}n words through her teeth, like Western S\r{e}l\={\i} used to do.}

\mulang{$0$}
{"--*Ты заблудилась, Тханэ?}
{``Are you lost, Tch\r{a}no\^{e}?''}

"--*Да, "--- глаза девушки забегали, "--- да, я заблудилась и немного перенервничала.
Прости за ружьё, я куплю тебе новое.
Есть ли поблизости жильё и дороги?

"--*Мы в трёх кхене пути от Тхитрона.
До Косого тракта ещё столько же.
Однако вряд ли это тебе поможет.

"--*Почему? "--- нож приподнялся на длину зерна.

"--*У тебя западный говор, "--- я старательно воспроизвёл её акцент.
"--- Шотрон, Чхаммитра?

Воины часто меняли место жительства, но на этот раз я попал в точку.
Одно из названий вызвало у неё непроизвольную дрожь.

Тханэ поняла, что выдала себя, и растерялась.
Опалесцирующие глаза по очереди осмотрели мои доспехи, кинжал и лицо.
Костяшки пальцев побелели.
Я понял, что передо мной отчаянный боец, и единственный способ избежать смерти "--- продолжать разговор.

\mulang{$0$}
{"--*Я вижу, ты нуждаешься в помощи, "--- сказал я.}
{``You seem to need help,'' I said.}
\mulang{$0$}
{"--- Но помочь я смогу, если пойму, как западная сели оказалась в сердце восточных джунглей одна, без еды, воды и доспехов.}
{``But I could help you if only I realize how Western S\r{e}l\={\i} woman got here, to the heart of Eastern jungle, alone, with no food, no water, and no armour.''}

Девушка помолчала.
Наконец она с силой задвинула жертвенный нож в ножны.

\mulang{$0$}
{"--*Я расскажу всё, если ты пообещаешь помочь.}
{``I'll tell you what happened if you promise to help.''}

\mulang{$0$}
{"--*Клянусь руками, убившими этого ягуара.}
{``I promise you with hands this jaguar had been killed by.''}

Мы встретились взглядами.
В кошачьих миндалевидных глазах медленно угасал боевой раж, уступая место страшной усталости.
Девушка вытерла текущую с подбородка кровь, тупо посмотрела на окровавленную руку и вдруг бессильно осела на землю.
Я едва успел подхватить её за плечи.

"--*Я тебе верю, "--- долетел до моих ушей тихий шёпот.

\section{У порога}

Так началась наша жизнь на Планете Трёх Материков.
Она началась с доверия.

Я не смогу рассказать сейчас о страшных трёх днях Катаклизма "--- мои впечатления чересчур свежи и могут исказить смысл, который я хочу вложить в повесть.
О наших повседневных заботах говорить ещё рано.
Да, доверие "--- это определённо то, с чего следует начать.

\mulang{$0$}
{Итак.}
{Well.}

Год полёта и двенадцать лет по времени Тси-Ди почти подошли к концу.
Одиннадцать тысяч триста восемьдесят переселенцев "--- я предпочитаю называть нас <<переселенцами>>, а не <<беженцами>> "--- ждали своего часа в капсулах, погружённые в медикаментозную кому и масло замедленного времени.
Мы дежурили вдвоём "--- я, Существует-Хорошее-Небо, и мой лучший друг, Хрустально-Чистый-Фонтан.

На выбранном пути отступать было некуда "--- топлива не хватило бы на ещё один цикл разгон-торможение.

\mulang{$0$}
{"--*Согласно данным Ордена Преисподней, это планета-океан, "--- сказала Темнотой-Сотканный-Заяц.}
{``According to the Order of Netherworld, it's an ocean planet,'' Darkness-Woven-Hare said.}
\mulang{$0$}
{"--- Никакой информации о природных ресурсах.}
{``No information is available on the natural resourses.}
\mulang{$0$}
{Маловероятно, что там есть условия для развития сапиентного общества.}
{It's unlikely there are conditions sapient society could develope in.''}

\mulang{$0$}
{"--*У нас есть выбор? --- грустно спросил Закрытая-Колба-Жизни.}
{``Do we have a choice?'' Sealed-Life-Flask sadly asked.}
\mulang{$0$}
{"--- Космическое путешествие "--- огромный риск.}
{``Space travel is a huge risk.}
\mulang{$0$}
{Жизнь на самой неприютной планете лучше смерти в межзвёздном пространстве.}
{Life on the most uncomfortable planet is better than death in interstellar space.''}

\mulang{$0$}
{"--*Главное "--- долететь, "--- поддержала планта Подсолнух-Бросает-Семена.}
{``The priority is to arrive,'' Sunflower-Dropping-Seeds supported plant's view.}
\mulang{$0$}
{"--- С оборудованием Стального Дракона мы получим всё необходимое.}
{``Using equipment of Steel Dragon we'll get all we need.''}

"--*Кроме того, на окраинах обитаемой Вселенной меньше шанс встретить демонов, "--- добавил Хрустально-Чистый-Фонтан.

Решение было принято почти единогласно.
И всё же мы волновались.

Тихо светили бестеневые лампы.
Казалось, двигатели тоже молчат, но я знал, что привычная гравитация "--- это торможение после субсветовой.
В воздухе сгущалось напряжение "--- под ветром жёлтого карлика наперебой пели все сенсоры корабля, до выхода на орбиту планеты оставались считанные часы.

Я сидел и ел, наблюдая за Фонтанчиком.
Огромного роста техник "--- подумать только, он выше меня в два раза! "--- расположился в кресле ровно, с изящной непринуждённостью свесив красивые руки.
На его темени перламутровой раковиной переливался имплант, уже почти скрывшийся под седой жёсткой гривой;
умные голубые глаза смотрели в пустоту, на видимый только для него нейрографический интерфейс.
Впрочем, мне знакомо чувство полного слияния с машиной.
Едва ли друг помнил о том, что у него есть глаза.
Его глазами были камеры, его телом был корабль, а звёздный ветер ласкал его сенсоры, словно полночный бриз.

Уши Фонтанчика, клиньями торчащие вверх, вдруг напомнили мне музейный экспонат "--- копию барельефа с Земли.
На этом осколке Древнего Мира, высеченном людьми народа Эйгипт, красовался канин "--- высокий, стройный, в отделанной золотом и минералами одежде, с раздвоенным тонким инструментом в правой руке.
Стиль изображения я нашёл очень странным "--- грудная клетка и таз находились в нефизиологичном положении, перпендикулярно друг другу.
Но гордый профиль, подведённые глаза и острые уши были прекрасно проработаны "--- возможно, канин выполнял какую-то важную работу.

К слову, в прошлом барельеф породил целый спор относительно времени выведения кани.

\mulang{$0$}
{"--*Небо, буди команду.}
{``Sky, wake the crew.}
Проблема, "--- нарушил тишину Фонтанчик, нырнув обратно в собственное тело.
Тон его говорил, что опасности нет, но проблему нужно решить как можно скорее.

Я просигналил на этаж техников.
Первой прибежала Заяц.

"--*Так, ребята, что тут у нас? "--- пропела она, поправляя пряжку на тугом мощном бедре.

\mulang{$0$}
{"--*Хоргет.}
{``Jorget.}
\mulang{$0$}
{И большой, "--- ответил Фонтанчик.}
{A huge one,'' Fountain answered.}

Заяц громко выругалась в адрес найденного нами существа.

\mulang{$0$}
{"--*Какой тип?}
{``What type?''}

"--*Плюс-хоргет, стационарный, точно не рой и не паук "--- шарообразное гало, границы чёткие и чистые.
\mulang{$0$}
{Похож на примитивного демиурга.}
{Seems like a primitive demiurge.''}

\mulang{$0$}
{"--*Новости хорошие, если не считать, что сам по себе хоргет "--- большая проблема, "--- проворчал я.}
{``Good news, apart from a big problem a jorget by itself is,'' I muttered.}
\mulang{$0$}
{"--- Надеюсь, он один.}
{``I hope it's alone.''}

\mulang{$0$}
{"--*Надейся, "--- бросила Заяц.}
{``Keep hoping,'' Hare succinctly answered.}
"--- Выруливай на орбиту, Фонтанчик, пока что мы вне зоны видимости.
Подготовлю генератор экрана и сотру с омега-поля эту досадную ошибку.

\mulang{$0$}
{"--*Подожди, ты что, собралась его уничтожить? "--- обернулся Фонтанчик.}
{``Wait, are you going to destruct it?'' Fountain turned to her.}

\mulang{$0$}
{"--*Есть другие предложения?}
{``Any other ideas?''}

\mulang{$0$}
{"--*Рулить у тебя получается лучше, чем у меня.}
{``You're better steersman than me.}
\mulang{$0$}
{Давай поменяемся.}
{Let's switch.''}

\mulang{$0$}
{"--*Перестань, "--- грустно улыбнулась женщина.}
{``Stop it,'' the woman sadly smiled.}
\mulang{$0$}
{"--- Иногда это необходимо.}
{``Sometimes it's necessary.''}

Заяц взмахнула руками и рыбкой прыгнула в лифт.

\section{Чаша, стилет и зелёное платье}

На небе засияли звёзды;
вдали дробно защёлкали клювами мохноножки.
Я перевязал руку девушки зелёными полевыми бинтами, которые хранил <<для трудных времён>>;
она хмуро наблюдала, зажимая левой рукой рану на подбородке.
Впрочем, ямочки в углах губ нашептали, что хмурая физиономия была моей спутнице в новинку.
Она привыкла улыбаться.

"--*Вечереет, "--- сказал я, посмотрев на темнеющее небо.
Снять шкуру с ягуара я уже не успевал.
С сожалением погладив красивого, молодого, мёртвого как камень кота, я повернулся к молчаливо ожидавшей девушке.

"--*Расскажешь всё по дороге, а сейчас нужно успеть в город.
Твоё платье пощадили кусты, ночные звери не оценят его красоту.

\dots Как оказалось, с этого симпатичного зелёного платьица всё и началось.
Прародительница Чханэ была купчихой из Тхартхаахитра и любовницей местного жреца.
Доподлинно неизвестно, что произошло;
старожилы говорили, что жрец был объявлен вне закона, а прародительница появилась в Пыльном Предгорье одна "--- без любовника, без друзей, имея при себе лишь платье, стилет и оплетённую в серебряную филигрань чашу из тончайшего белого фарфора.

По словам Чханэ, купчиха была эксцентричной женщиной со странным чувством юмора.
Чашу она использовала для питья, несмотря на отчётливые следы красной краски\footnote
{При создании хэситра сели использовали красную краску.
Считалось дурной приметой пить из чаши, которую использовали как хэситр. \authornote}.
Остриём стилета она любила покалывать собеседника, если считала, что её словам уделяют мало внимания.
Платье же слегка располневшая женщина спрятала и забыла, как дети забывают старые игрушки.

Клад прародительницы так и лежал бы в погребе до конца времён, если бы не наследница.
Расшитый листьями шёлк, о каком и не слыхивали местные модники, пришёлся ей по вкусу;
знающие люди намекнули, что в своё время платье стоило целое состояние "--- на нём не было ни единого шва, полосы на плечах, спине и рукавах оказались искусной имитацией.
Однако время всё же оставило на ткани след, и Чханэ отнесла сокровище к старому жрецу "--- портные категорически отказывались браться за такую сложную работу.

"--*Сатхир ар’Со всегда был странным человеком.
Его звали Притворившийся-Одеялом-Крокодил "--- символично, на мой взгляд.
Но с моим кормильцем у него были хорошие отношения, долго одному Храму служили.
Говорю "--- у тебя химикатов куча, сделай хоть что-нибудь, чтобы протёртости не было видно, вещь пропадает!
Сатхир накричал на меня, мол, глупая девка, только тряпки на уме, он не портной, чтоб штаны штопать\ldotst
Я сказала, что заплачу по весу "--- приврала, конечно, но не так уж, чтоб сильно, в запасе лежало немного золота.
Он снова "--- да зачем мне столько золота, я его не ем и в огороде не сажаю.

"--*Знакомо, "--- усмехнулся я.
"--- Из-за того, что жрецам почти всё достаётся бесплатно, с ними договориться невозможно.
Если не захотят "--- ничего делать не будут.
Приходится брать на интерес или на азарт.
У нас валютой как-то была концентрированная азотная кислота: за колбу-другую что хочешь сделают.

"--*Знаю я вашу азотную кислоту, "--- буркнула Чханэ. "--- Посреди ночи <<тракх\footnote
{Бабах, бум (цатрон). \authornote}>>,
и наутро два жреца приходят без пальцев и с довольными рожами "--- фейерверки, видите ли, пытались сделать.
Почему нельзя было днём и без членовредительства "--- непонятно.

Я промолчал, пряча улыбку.
Один жрец мало отличается от другого, когда дело доходит до химии.
Учитель Трукхвал тоже как-то просил пришить ему два пальца.
Это был относительно удачный случай "--- чаще всего пришивать было нечего и оставалось лишь надеяться, что пальцы вырастут быстрее, чем начнут вызывать закономерное любопытство постоянные бинты на руках.

"--*Впрочем, в итоге решилось само.
Сатхир перешёл на спокойный тон и сказал, что подновит ткань, если я выполню его задание.

"--*Что за задание?

"--*Он сказал, что должен поклониться древнему духу в джунглях.
Нужен был человек, который поведёт жертвенное животное, дескать, сам я старый, не смогу.
Заказ как заказ, сто раз такие брала "--- сопроводить туда и обратно, вести животных\ldotst
Да только дура я, конечно, не спросила, какие такие духи требуют смерти зверей!

Сверху свалилась маленькая мраморная змея и попыталась впиться в кожу моего нашейника.
Молнией жахнул обсидиан "--- и обезглавленное тельце, извиваясь, упало на землю.
Я вздрогнул, когда клинок прошёл у самой шеи;
Чханэ, приняв великолепную боевую позу, окинула взглядом нависшие над головой ветки.

"--*Первый раз такую вижу.
В наших джунглях из ядовитых только кораллица и смертожар, и они на деревья не залезают.

"--*А ты думала, почему я в доспехах? "--- усмехнулся я.
"--- Необязательно было бить, нашейник она бы не прокусила.

Я поднял судорожно дёргающуюся змейку и сунул в заплечный мешок "--- шкурка небольшая, но ценная, моей доли хватит на свиток, на остаток сестрёнкам леденцов прикуплю\ldotst
Голова пристально смотрела на меня немигающим взглядом.
Чханэ суеверно втоптала её в почву, прошептав что-то под нос.

"--*Так что было дальше?

"--*Сатхир выполнил обещание, починил платье.
Я\ldotst я не ожидала, честно.
Ткань была абсолютно целой, словно платье сделали пару рассветов назад.
В день, когда мы покинули город, он сказал, чтобы я надела его.
Дескать, проверишь да разносишь заодно.
Я говорю "--- доспехи нужны, не в храмовых землях гулять будем.
Но он заверил, что путь совершенно безопасен.

"--*Ты согласилась идти на Север? "--- удивился я, глядя на выбившуюся прядь цвета кофе.
Чханэ поймала мой взгляд и тут же, сердито нахмурившись, поправила её.

"--*Да какой Север, два рассвета пути.
Сначала по нагорью, потом через джунгли, и явно не без помощи какой-то волшбы.
Если по дороге не ходят, она зарастает за десять дней.
А тут\ldotst тропинка словно ждала тех, кто по ней пройдёт\ldotst

\mulang{$0$}
{От её чарующего акцента во мне бурлила цветущая юность.}
{Her charming accent forced my blooming youth to flow.}
Я любовался изломанными дугами бровей, изящным горбатым носом, твёрдыми, словно из камня высеченными губами с манерной складкой, за которыми скрывался жемчуг острых, как у оцелота, зубов.
На левой щеке девушки был приметный шрам в виде креста;
такие любили оставлять на врагах умелые фехтовальщики, гордящиеся своим мастерством.
Над всем этим властвовали оранжевые, похожие на светляков глаза, презрительно смотревшие на меня сверху вниз.
Пожалуй, чересчур презрительно.
Я видел прежде оцелотовых людей, кажется, в раннем детстве;
хорошо помню, как от них шарахались в темноте горожане.

Я непозволительно долго наблюдал за ней.
Мост мы давно прошли, тропинка бежала по землям Тхитрона;
однако то, что сходит с рук днём, опасно в сумерках, и тихий рассказ спутницы лишь подтвердил эту старую истину:

"--*\ldotst оленя ночью задрал ягуар.
Так сказал Сатхир.
Не знаю, как это вышло, но я не услышала звука, не почуяла запаха.
Да и следы на трупе, знаешь, не похожи были на ягуарьи "--- как будто оленя яростно терзала какая-то крупная обезьяна.
Я думала, что мы повернём назад, однако жрец возразил, что уже поздно, да и животное всегда можно найти новое.
Вот тогда-то я поняла, что на заклание вели вовсе не оленя\ldotst

"--*Кровь людей течёт лишь по алтарям Безумного!
Лесные духи не принимают человеческих жертвоприношений! "--- удивлённо возразил я.

"--*Это был не дух леса.
На второй день мы вышли к старому камню.
Обработанная базальтовая скала, расколотая напополам "--- со скобами, стоками, точь-в-точь храмовый алтарь, словно для какого-то забытого божества.
Подозрения укрепились, когда Сатхир велел мне лечь спать пораньше, мол, завтра трудный переход, да ещё и новое
животное ловить надо\ldotst
Я сделала вид, что ставлю шатёр, а когда он подошёл к алтарю и зачем-то возложил на него ладони, я\ldotst "--- Чханэ запнулась.

"--*Что?

Девушка пронзила меня оранжевым взглядом.

"--*Помнишь, что ты обещал?

"--*В напоминаниях нет нужды, "--- я постарался, чтобы в моём голосе не прозвучал укор.
"--- Продолжай.

Впрочем, продолжения не требовалось.
Я и так всё понял.

"--*Я подкралась сзади, вытащила из ножен его ритуальный нож и выплеснула кровь Сатхира на базальт.

\section{Ради жизни}

"--*Можно было попробовать поговорить с ним, "--- задумчиво сказал я.
"--- Он же ничего плохого не сделал?

"--*\emph{Пока} не сделал, "--- мрачно буркнул Баночка.

<<Не знаю, как ты, Небо, а мы все хотим жить, "--- ответила Заяц по общему каналу.
"--- Фонтанчик, выверни корабль на четыре сотых по зениту и на минус двадцать три сотых по азимуту, удар будет мощнее>>.

Время тянулось медленно, словно весь мир, а не только наш корабль летел на релятивистских скоростях.
Баночка взял на себя связь, Фонтанчик выверял траекторию полёта, я и Подсолнух анализировали информацию с сенсоров.
Мозг упрямо не желал работать "--- сказывалась усталость последнего года и увеличившийся вес тела.
Подсолнух проводила вычисления молча, время от времени потирая синюшные веки и добавляя в кровь стимулятор.
Я аккуратно сжал её руку, она слабо улыбнулась и кивнула в ответ.

<<Фонтанчик, поле снимай, прицел есть>>, "--- скомандовал общий канал.

Канин кивнул и прикрыл глаза, заверяя команду цифровой подписью.
Вдруг корабль ощутимо тряхнуло.
Баночка аж подпрыгнул на месте.

"--*Да чтоб вас всех\ldotst "--- выругался Фонтанчик.
Система недовольно пискнула;
при активации защитного поля друг лишь с третьего раза ввёл команду правильно.
"--- Разгерметизация на уровне четыре.
Заяц, ты где?
Заяц, ответь!
Заяц!

<<Да здесь я, дай выдохнуть.
Поздравляю с первым камушком>>, "--- бесплотный голос Заяц излучал сарказм.

"--*Это не хоргет? "--- спросила Подсолнух.

<<Нет, просто камень, они тут летают.
Но попал он удачно "--- я едва в лифт успела запрыгнуть>>.

"--*Генератор экрана цел?

<<Ага.
Вакуум вам с булочкой, а не экран.
Небо, запусти робота, пусть чинит>>.

"--*Никаких роботов, "--- отрезал я.
"--- Что, если и их перепрограммировала Машина?
Забыла Золотую дорогу?

<<Голосование?>>

"--*Ты предлагаешь разбудить посреди полёта десять тысяч тси, чтобы сообщить, что планету занял отъевшийся до константы Ка'нета демиург? "--- осведомился я.
"--- Нет уж, решать проблему придётся нам.
Возвращайся в рубку, вытащим процессорный блок из одного робота-техника и настроим управление через нейроинтерфейс. Подсолнух, ты говорила, что как-то управляла таким вручную.
Сможешь сейчас?

"--*Без проблем, "--- откликнулась женщина.
"--- Только дай мне кресло поудобнее, с фиксацией.

"--*Садись на моё.
Фонтанчик, усиливай торможение.

"--*Вам до состояния котлет или супа-пюре? "--- мрачно спросил канин.

<<До предела силы мышц, поднимающих веко, "--- буркнула Заяц.
"--- Не люблю работать с закрытыми глазами>>.

Посмеялись неуместной этой шутке.
Затем дружно вздохнули, почувствовав на плечах непривычную тяжесть.

"--*Помним о правилах, "--- отчеканил Баночка.
"--- Вначале отмонтируем от робота все имеющиеся модули связи, затем процессорный блок.
Всё это при отключённых мозговых имплантах.
Фонтанчик, следи за активностью программ корабля.
На крайний случай будь готов экстренно остановить ядро.

Все дружно ответили сигналом <<Готовность>> "--- в знак того, что поняли.

Работа шла молча и медленно, чересчур медленно.
Даже простые движения руками и ногами отнимали слишком много сил.
Заяц, Баночка и я едва настроили нейроинтерфейс и отправили робота наружу, когда молчание прервал Фонтанчик:

"--*Баночка, связь, "--- канин на мгновение остановился, чтобы перевести дух.
"--- Нас вызывают на стандартной омега-частоте, пятьдесят один и одиннадцать сотых.

"--*Стандартная частота?
Значит, здесь есть высокоразвитые сапиенты, "--- обрадовался Баночка и тронул имплант.
Несколько мгновений он вслушивался в сообщение, и с его личика медленно сползала улыбка.

"--*Это хоргет, "--- наконец сказал он.

"--*ЧТО?! "--- хором спросили мы.

"--*Мы чересчур далеко! "--- возмутилась Заяц. "--- Быть не может!

\mulang{$0$}
{"--*Вне всяких сомнений.}
{``Beyond any doubt.}
\mulang{$0$}
{Говорит на языке Эй, таблица 00.}
{It speaks Ej, table 00.''}

Плант озадаченно нахмурился, словно услышал нечто бессмысленное.

<<У меня\ldotst у меня нет имени>>, "--- Баночка окинул нас взглядом, пытаясь найти фразе объяснение.
"--- <<У меня нет имени.
Объект, ответь и назови твоё имя.
Я чувствую твоё присутствие и знаю, что ты "--- продукт высоких технологий.
Что ты такое?
Нуждаешься ли ты в помощи?
Конец сообщения>>.

"--*У него нет имени? "--- удивилась Заяц.
"--- Он осознаёт себя как личность.
Значит, точно девиант.

"--*Язык Эй.
Неужели мы пролетели столько парсак ради того, чтобы попасть к Ордену Преисподней? "--- тихо буркнул Фонтанчик.

"--*Это точно нам? "--- спросила Подсолнух. "--- Баночка, ты уверен?

"--*Да, "--- кивнул Баночка. "--- Сигнал направленный.

Подсолнух затихла в кресле, и лишь скривившиеся губы и плотно сжатые веки говорили о её напряжении.
Управляемый женщиной робот ускорился, насколько позволяла перегрузка.

"--*Помощь, значит, "--- тяжело дыша, проговорил Фонтанчик.
"--- Небо, запускаем второго робота, пусть помогает Подсолнух.
Мощности экрана хватит, чтобы накрыть всю планету.

"--*Да быстрее! "--- простонала Заяц.
"--- Небо, команда, подпись!

"--*Нет, "--- оборвал я её.
"--- Чересчур опасно.
С одной стороны "--- наши собственные устройства, с другой "--- Орден Преисподней, не друг, но и не враг.
Даже если всё получится и мы нанесём удар\ldotst

"--*Хорошо.
Что ты предлагаешь? "--- невинным голосом поинтересовалась Подсолнух, открывая глаза.
"--- Я не успею починить экран одна.
Там критическая поломка, нужно отпечатывать заново половину микросхем.

В рубке воцарилось молчание.
Все дружно смотрели на меня.

"--*Ребята, "--- медленно сказал я, чувствуя, как трудно дышать в густой атмосфере всеобщего неодобрения, "--- я не хуже вашего знаю, кто такие хоргеты.
Но мы оставили позади то, что хочет уничтожить нас без раздумий и переговоров.
Я не могу ручаться, что роботы не перепрограммированы Машиной "--- в этом случае у нас появится ещё одна опасность, гораздо серьёзнее, чем Орден.
Появится она у нас на корабле, и между этой опасностью и десятью тысячами тси "--- возможно, последними тси во Вселенной "--- будем только мы.
Кто готов взять на себя такую ответственность?

Друзья ошарашенно молчали. Не дожидаясь ответа, я повернулся к Баночке:

"--*Баночка, ответное сообщение на том же языке.
<<Народ тси.
Лишились дома, нуждаемся в пристанище, просим разрешения на посадку.
Конец сообщения>>.

"--*Ты только что подписал нам приговор, Небо, "--- глухим голосом сказал Фонтанчик.
"--- Орден превратит нас в рабов или подопытных животных.

"--*И всё же это лучше смерти.
Баночка, действуй.

Баночка, поколебавшись, на мгновение прикрыл глаза.
Ответ не замедлил себя ждать:

"--*<<Тси, сигнал принят и понят.
Даю разрешение на посадку.
Конец сообщения>>.
Также он передал стандартную сферическую карту с пометкой "--- вероятно, координаты для посадки.

Баночка вывел голограмму.

"--*Посмотрите, это не планета-океан, как говорили демоны Ада!
Здесь три огромных материка!
Это планета трёх материков!

Баночка осёкся, и в рубке наступила оглушительная тишина.
Такой не было с самого начала полёта.

Я посмотрел на товарищей "--- Заяц возмущённо хлопала ртом из-за моей наглости, Подсолнух бесстрастно смотрела в потолок, равномерно качая коротко стриженной кудрявой головой.
Баночка растерянно моргал круглыми, лишёнными белков глазами и ковырял пальцем панель.
Фонтанчик обмяк в кресле, прикрыв лицо рукой.
Системы регуляции давления гудели в унисон с тяжёлым дыханием друзей.

Я понял, как они устали.
Они знали, что должны сами решать свою судьбу "--- этому всех тси учили с детства.
Но сейчас, в миг критической усталости, этот детский урок боролся с детской же надеждой, что придёт добрый и ласковый взрослый, который обнимет, накормит и решит все проблемы за них.

Мне вдруг стало страшно одиноко.
Наверное, так чувствует себя кормилец без крыши над головой, у которого на руках маленькие, ничего не понимающие дети.
Я читал, что в древних обществах такое не было редкостью "--- сапиенты умирали от голода и холода, и не находилось среди их вида никого, кто хотел бы помочь просто так.
Звериное прошлое, от которого не отмахнуться.

И вместе с этим чувством родилась мысль "--- детей нужно спасти любой ценой.
Свобода и честь "--- ничто, если о них некому вспомнить.

Я повернулся к Фонтанчику:

"--*Ради жизни.
Объявляю посадку на координаты, указанные Безымянным.

Фонтанчик молча кивнул.
Остальные так же молча взялись за работу.

\section{Песня сверчков}

Так я очутился на распутье.
Прекраснейшая из женщин была кутрапом "--- человеком, не имеющим права жить в землях сели.
Пока Чханэ рассказывала, я, скосив глаза, пытался разглядеть её правую кисть.
Клейма там не было.

"--*Ночью я спала рядом с камнем и разговаривала с богом во сне.
Он посердился немного на то, что я не произнесла его имени.
Я же поведала ему о случившемся и сказала, что вряд ли он получит жертву более искреннюю, чем моя.
Эти слова развеселили его, и бог позволил мне уйти невозбранно.

Я промолчал.
Убийство храмового жреца считалось сразу за два Разрушения.
Старейшины признавали, что это правило способствовало изоляции Храма, но отменять его никто не собирался "--- ведь от Верхнего этажа зависела судьба целого поселения.
Впрочем, мёртвый жрец остался где-то далеко, а рядом со мной шагало очаровательное существо, которое я искренне считал жертвой несчастливой случайности.
По-другому просто не могло быть.

Как назло, в сумерках застрекотали сверчки, эти символы беззаботной, беззаветной влюблённости\ldotst
Ещё ребёнком я спросил у кормилицы, отчего сверчкам такая честь.
Кормилица подумала и ответила, что если один принесёт в дом сверчка "--- спать спокойно не сможет никто.
Что она имела в виду, я так и не понял.

"--*Ну как, Ликхмас ар’Люм, не отпало желание помогать убийце жреца? "--- с улыбкой спросила Чханэ.
"--- Не испугался ли ты гнева богов?

Я изумлённо взглянул в оранжевые глаза, ставшие почти чёрными в тени деревьев.
В её голосе не было ни страха, ни обречённости.
Казалось, она спрашивает, что я желаю на завтрак.
И вполне предсказуемо новое чувство, зародившееся где-то в горле, взяло верх.

Спрятать.
Сберечь.
Погостит у нашего очага.
А потом пусть едет на все восемь сторон.
Правда, последнее прозвучало куда менее убедительно\ldotst

Вдали показались знакомые деревянные столбы с резными ликами духов.
Я остановился, и девушка последовала моему примеру.

"--*Что там? "--- шёпотом спросила она.
"--- Кхар-защитник, Удивлённый Лю, какая-то загогулина\ldotst
Судя по знакам "--- дорожная застава.

"--*Почти.
Загогулина "--- это береговая ласточка, символ моста.
Мост пройдём беспрепятственно, заставы пустуют.

"--*Как это "--- пустуют?!

Я едва сдержал улыбку при виде праведного гнева воина, который узнал, что важный стратегический пункт лишён охраны.
На секхар я подумал, что Чханэ сейчас же пойдёт в тхитронский храм и устроит Нижнему этажу разнос.

"--*Это старый мост, он ведёт через обвалившуюся пещеру на острове.

"--*А ты как прошёл?

"--*Под настилом прополз. Он крепкий, чёрного дерева, а под ним сток в три пяди высотой.

"--*Вообще-то это уязвимость, "--- буркнула девушка.
"--- Что, если идолы узнают?

"--*Твои проблемы сейчас важнее.
О том, что случилось, в городе ни слова.
Для всех ты "--- воин, сопровождавшая погибший караван.

"--*Это не будет ложью, я адепт Нижнего этажа и ходила в походы.

"--*Отлично.
Ты будешь жить в моём доме.

"--*Попрошу убежища в Храме.
Мне нечем тебе заплатить.

"--*Неужели мы не найдём для тебя миску с едой?
А за жилище\ldotst
Кормилица "--- очень добрый человек, но если не хочешь быть в долгу "--- можешь у нас работать.

"--*Твоей подстилкой? "--- насмешливо спросила Чханэ.
"--- Или просто помои за тобой выносить?

"--*Странные у вас на Западе нравы, — возмутился я. "--- Вообще-то я ученик жреца и давно живу в храме, так что\ldotst

Я сказал это не подумав и тут же крякнул от боли, когда неожиданно крепкая девичья рука подвернула мою, а ритуальный нож невзначай уткнулся мне под рёбра.
В ноздри ударил аромат её волос, сладкий, как напиток из зёрен какао\ldotst

"--*Так ты меня на заклание ведёшь, игуана раздавленная? "--- прошипела Чханэ над моей головой.

"--*Я обещал тебе помочь, "--- сказал я, глядя на примятую дорожную траву перед самым носом.
"--- И помогу.
Пожалуйста, оставь свои умения для врагов.
Я первый встану с тобой плечом к плечу, если понадобится.

Удивительно, насколько порой опьяняет влюблённость.
Ритуальный кинжал разрезал людей, как бабочек, но боялся я его не больше, чем комара.
Почему-то не верилось, что Чханэ способна меня убить.
По-видимому, девушка это поняла.
Она вздёрнула меня на ноги и хмуро поправила повязку на руке:

"--*Приведёшь храмовников "--- умрёшь первым.
Живой я на алтарь не лягу.

Дальше мы двинулись в полном молчании.
Стемнело уже настолько, что я видел только слабое инфракрасное свечение кожи девушки.
Следовало торопиться.

"--*Ты вообще оружие в руках держал? "--- фыркнула Чханэ.
"--- Простой приём не сумел отбить.

"--*Я и не собирался, "--- усмехнулся я.
"--- Сколько ты видела дождей?

"--*Там, где я росла, нет дождей.
Но с тех пор, как родильница перерезала мою пуповину, солнце тридцать шесть раз прошло через зенит.
А сколько дождей видел ты?

"--*Двадцать девять.

"--*Большая честь "--- стоять плечом к плечу с младенцем.

Чханэ сказала это так язвительно, что я обиделся:

"--*На моём счету пять идолов.

"--*Нашёл чем гордиться.
На моём "--- тринадцать, "--- девушка нагло оскалила острые зубки.
"--- Пустынных пылероев около сотни.

Я едва удерживался, чтобы не взорваться.

"--*Я знаю, кто такие пылерои!
Почему ты так нагло врёшь и стараешься меня унизить?

"--*А это чтобы ты помнил своё место, гордец.

Мой гнев уходил, как вода в песок.
Девушка-воин, волею Безумного брошенная в джунглях с одним ножом, просто пыталась укрыться от собственной беспомощности.

"--*Я защищу тебя, Чханэ.

"--*Я сама себя защищу, Ликхмас ар’Люм.

\section{Украшения}

"--*Мы должны в первую очередь защищать других!

Митхэ уже устала повторять эту фразу отряду чести.
Себе, впрочем, тоже.
Воинов по-прежнему кормили города, которым они служили, но реалии нового мира неумолимы "--- жареным зайцем не починишь кукхватровый клинок.
Нужно золото.
И иногда нужно уступать подчинённым.

"--*Я нашла заказ, "--- объявила Митхэ отряду за завтраком.
"--- Караван идёт на Кристалл.
Они хорошо заплатят нам за охрану.

Люди прекрасно знали, каких трудов ей стоило проглотить собственную гордость, не подавившись, и сказать эти слова.
Они лишь время от времени подкладывали в тарелку командира лишний кусок да подливали в чашу вина.
Особенно старалась Согхо.
Эта широкая душа уже два сезона носит самодельное копьё, в которое её любовник-подмастерье вставил акулий зуб и шесть обсидиановых лезвий.
В крыльях у Согхо дыра, прямо под правой грудью.
Рана давно зажила, а прореха только увеличивается, и заклепать её не на что.

"--*Согхо, милая, это не поможет, "--- заметила Митхэ, когда женщина в очередной раз подлила ей вина.

"--*Тогда хочешь, я тебя поцелую? "--- простодушно спросила Согхо.
"--- Или расчешу тебе волосы?

Отказываться было трудно и незачем.
Как будто нарочно на площади в тот же время зазвенела знакомая цитра.

Как же красиво играет этот бродяга Атрис!

За завтраком последовал обычный каждодневный ритуал.
Тёмная, не слишком чистая комната в постоялом дворе знала его до тонкостей.
Митхэ надела рубаху, дёрнув вплетённую в ткань удавку.
Затем подпоясалась и привычным движением проверила гибкий клинок в кушаке.
Затем топнула сапогом, ощущая упругость пружины "--- ещё один сюрприз для врагов.
Лишь невежда мог подумать, что Митхэ ар'Кахр вышла в город безоружной.

В завершение воительница нанесла на лицо <<кошачьи слёзы>> и вытерла остатки сурьмы о короткие кудри.
Последние события представили в новом качестве знак Плачущего Ягуара "--- это действительно слёзы, слёзы и пот.
Но отказываться от клятвы Митхэ не собиралась.

Впрочем, сегодня ей хотелось чего-то нового.
Митхэ вытащила из кармана бусины, купленные накануне на рынке, и они неожиданно ярко заиграли голубым в утреннем свете.

"--*Эрхэ!

В дверях показалась улыбчивая воительница, живая и крепкая, словно молодая гевея.
Увидев бусины, она подняла бровь.

"--*Золото, что это?

"--*Сделай мне рыбки, "--- смущённо попросила Митхэ.
Бравада улетучилась дымом свечи, оставив лишь кашель смущения.

"--*Кто-то сказал <<рыбки>>? "--- из-за лежанки показались три трубчатые заколки, а следом и немытая чернявая голова.
В сурьмяных глазах мужчины плясали весёлые огоньки.

"--*Ситрис, пошёл вон, "--- спокойно посоветовала Эрхэ, тут же справившись с удивлением.
"--- Митхэ, давай бусины и садись.

"--*Ты очень забавно думаешь вслух, Митхэ, "--- невозмутимо сообщил Ситрис.
"--- Кто такой Атрис?
Это для него ты надеваешь украшения?

"--*Украшенвия?
Капита Миция сильно влюбвиться? "--- в комнату заглянул седой как снег старик, по говору "--- ноа с Могильного берега.
Зубы старика сияли ярче волос.

Эрхэ недовольно рыкнула, и мужчины рассмеялись.
Неудивительно, ведь рыком получившийся звук назвать было сложно.
Голос воительницы совершенно не соответствовал внешнему виду "--- она тихо и нежно щебетала, словно крохотная самка трасакх на гнезде, даже воинские ругательства в её устах казались лаской.
Во время дела Эрхэ открывала рот только в крайнем случае "--- услышав её боевой клич, враги просто не воспринимали женщину всерьёз.
Митхэ очень нравилось просто сидеть и слушать подругу;
этот голос действовал на неё успокаивающе.

"--*Пойдём, сенвиор Амвросий, "--- с наигранной грустью сказал Ситрис.
"--- Дела женские, им не до нас.

Воин перепрыгнул через лежанку, схватил старика-ноа за рукав и вытянул в коридор, закрыв за собой дверь.

"--*Золото, "--- без предисловий начала Эрхэ, "--- ты же понимаешь, как это будет смотреться?
Рыбки получатся вот такой длины!

Эрхэ выразительно сблизила большой и указательный пальцы.

"--*Я хочу быть чуточку красивее, Обжорка, "--- пробормотала Митхэ и улыбнулась, показав сломанный в бою зуб.
Рваный шрам на губе натянулся.

"--*У тебя и так от этих цветущих отбоя нет, они за тобой косяком плавают, "--- возразила Эрхэ;
судя по тону, толпа влюблённых женщин ей не особенно нравилась.
"--- Ты носила рыбки раньше?

"--*В далёкой молодости.
Когда я приняла <<кошачьи слёзы>>, боевой вождь Храма срезал рыбки вместе с волосами, сказав, что не положено.

"--*Урод он пресноводный, "--- заявила Эрхэ.
"--- Нигде в кодексе Ягуара не прописан регламент по украшениям.

Митхэ никому не позволяла такие выражения в адрес рекомого вождя, но Эрхэ была особым случаем.

"--*Кихотр, "--- покачала головой Эрхэ.
"--- С бусинами проблемы, говорю сразу "--- они мешают спать, наутро все уши будут мятые.
Лучше бы ты заколки взяла с подвесками, их раз "--- и сняла.
Или подождать, пока волосы отрастут, и сделать валяные хвосты, как у меня.

Эрхэ по очереди тронула рыбки на висках.
Митхэ промолчала.

"--*Ты поэтому отказалась стричься намедни?
Горе моё\ldotst
Хорошо, давай посмотрим, что тут можно сделать, "--- воительница приняла из рук командира бусины и громко добавила в пространство:
"--- Ситрис, я слышала, что замок не щёлкнул!
\mulang{$0$}
{Пошёл вон, горшок с кровью!}
{Get out, you pot of blood!''}

"--*Гярсёк с крёвью, "--- тоненько передразнил её воин и снова вылез из-за лежанки.
"--- Удачно погулять, Митхэ.
Атрису привет.

Митхэ гуляла по Тхартхаахитру каждый день.
Это тоже было частью ритуала.
Сначала она проходила по торговым рядам, переругиваясь с торгующими и расталкивая зевак.
Затем покупала очередную безделушку у старушки, стоявшей в углу площади.
Старушка долго благодарила, понимающе улыбалась и бросала хитрые взгляды то на сидевшего невдалеке Атриса, то на хмурую, неулыбчивую Митхэ.
Воительница от таких молчаливых откровенностей краснела до кончиков ушей и спешила убраться подальше.
Затем, сделав огромный круг в несколько кварталов, Митхэ возвращалась на площадь, садилась невдалеке от менестреля и слушала его игру.

Атрис упорно избегал её взгляда.
С его лица не сходила виноватая улыбка, на вопросы он отвечал односложно и неопределённо.
Самолюбие воительницы было серьёзно уязвлено.
Атрис дружески беседовал с нелюдимым ювелиром, который бросал менестрелю <<на удачу>> пару кусочков серебра.
Атрис называл <<кормилицей>> похожую на лань хозяйку постоялого двора, которая каждый день носила певцу еду\ldotst

"--*Опять ничего не ел, журавлик, "--- ворчала она и доставала из складок плаща горячий горшочек с похлёбкой.
"--- Костями будешь скоро греметь в такт музыке!
Ла, лопай.
Миску потом заберу.
Вечером же здесь будешь?
Приду слушать.

"--*Храни тебя духи, кормилица, "--- светло улыбался Атрис и жадно набрасывался на похлёбку.
Митхэ уходила, не в силах выдержать это зрелище.
Иногда это была его единственная трапеза за день "--- всё остальное уходило на ночлег и починку инструмента.

После обеда Митхэ шла в постоялый двор, с некоторой грустью бросала старушкину безделушку в походный сундук и заваливалась спать, не сняв одежду.

<<Я бы насыпала тебе полный горшок золота, "--- думала она с обидой, "--- но я же не хочу покупать твою любовь!>>

Так продолжалось уже целую декаду.
Но у воительницы было предчувствие, что сегодня всё будет иначе.
Разумеется, сегодня снова прогулка, снова по тому же маршруту.

"--*И ещё, Обжорка, "--- тихо сказала Митхэ напоследок, "--- пообещай, что не будешь третировать Ситриса.
Нам полезен такой человек.
\mulang{$0$}
{Ты видела, как он подкрался?}
{Did you see him sneak up?''}

\mulang{$0$}
{"--*Нет, не видела, "--- хмуро ответила Эрхэ.}
{``No, I didn't,'' O\r{e}rcho\^{e} said grumpily.}
\mulang{$0$}
{"--- Просто знала, что он опять по своей глупой привычке\ldotst}
{``I just knew he do it again, out of habit, and\dots''}

\mulang{$0$}
{"--*Именно!}
{``Exactly!}
\mulang{$0$}
{А это значит, что нам есть чему у него поучиться!}
{And it means he has something to teach us!''}

\mulang{$0$}
{"--*Не люблю подкрадываться.}
{``I hate sneaking.''}

\mulang{$0$}
{"--*Может пригодиться и самое нелюбимое умение.}
{``Even least favourite skill may be useful.}
Ситрис честно старается стать одним из нас, а его выходки "--- просто игра.

"--*У нас здесь не детская комната, "--- так же тихо бросила Эрхэ.
"--- Я привыкла ко всем новичкам, но этот "--- трусливый, скрытный и скользкий тип!
И плевала я, что он больше не разбойник, такое просто так не проходит.
На Юге бы эту рыбину на уху пустили\ldotst

"--*Если понадобится, я сама сделаю из него любое блюдо, "--- в голосе Митхэ промелькнули металлические нотки.
"--- Но до той поры он "--- твой соратник.
Обещаешь?

"--*Только ради тебя.

"--*Надеюсь на понимание.
За рыбки "--- огромная благодарность.
\mulang{$0$}
{Я выгляжу лучше?}
{Do I look better?''}

\mulang{$0$}
{"--*Честно?}
{``Honestly?''}

\mulang{$0$}
{"--*Насколько возможно.}
{``As it possible.''}

\mulang{$0$}
{"--*Ты выглядишь так же, как и всю последнюю декаду.}
{``You look as you did for the last decade.''}

\mulang{$0$}
{"--*И как же?}
{``How?''}

\mulang{$0$}
{"--*Растерянной.}
{``Confused.''}

\mulang{$0$}
{"--*Вот досада.}
{``Damn.''}

Митхэ поправила начавшие раздражать украшения, подхватила мешок и вышла за дверь.

\section{Рыба-карп}

"--*Всё, пришли, "--- выдохнул я и прислонился к холодной каменной стене.

Чханэ выпуталась из верёвок и без сил повалилась на чистую солому, которой был покрыт пол подвала.
Я только что протащил девушку домой в окровавленной, ещё тёплой оленьей шкуре "--- через ворота, три улицы и торговую площадь, мимо городской стражи, двух десятков вечерних прохожих и домашнего оцелота, безмятежно дремавшего на подоконнике.
Шкура стоила мне пятидневного заработка.
Столбик, закадычный друг детства, пообещал молчать.

"--*Не знаю, что ты опять собрался протащить в город, но надеюсь, что это безжалостный убийца твоей неопытности, "--- пошутил друг.

"--*Я тебе язык к носу пришью, петух-топтун, "--- посулился я.
"--- Следи за своими яйцами.

"--*Я серьёзно, Лис.
От тебя пахнет свиньями, кошками, но сильнее всего "--- молодой девушкой, "--- Столбик обнюхал мою рубаху и мечтательно задумался.
"--- И, кажется, она не мылась с рождения.
Это важно?

"--*Вообще-то да, "--- вздрогнул я.
"--- Есть чем перебить запах?

"--*Так мускус я тебе на что дал?
Натрись им.
Если всё серьёзно, могу дать маликхов венок "--- окуришь и себя и свою немытую любовь.

"--*Тащи.
И всё сухотравье от москитов, которое есть.

"--*Да куда тебе столько? "--- расплылся в щербатой улыбке Столбик.
"--- Ты её любить собрался или коптить?

"--*Заткнись и неси.
Кадильницу тоже, завтра верну.

Нюх Столбика был вполне объясним "--- женщины сводили его с ума.
Он был младше меня на дождь, но уже успел прижить четверых детей с разными женщинами.
К счастью или на беду, мужчины его не интересовали.
Улыбчивый крестьянин мне нравился тем, что с ним не нужно было регулярно проводить время.
Я мог пропасть на дождь или на пять дождей, но был уверен, что Столбик встретит меня с радостью и мы поболтаем о
всякой ерунде, словно расстались рассвет назад.

Последним препятствием на пути в подвал был старый слуга, Сиртху-лехэ.
Он как будто нарочно устроился возле входной двери, держа в зубах тлеющую лучину и читая план посевов на грядущий год.

"--*Хорошая добыща, Ликхмаш-тари, "--- Сиртху-лехэ, не поднимая глаз, принюхался к воздуху;
лучинка вспыхнула, когда он заговорил.
"--- Мушкушный шамец, молодой и щильный, дождей пять ш половиной.
Неужто дикие штада ещё жабредают в наши края?

"--*Слава духам, "--- ответил я, изо всех сил стараясь говорить ровно.
Не то чтобы Чханэ была тяжелой, но\ldotst
Да.
Она была очень тяжёлой даже для своего роста и комплекции.
\mulang{$0$}
{Про таких людей говорят "--- золотые кости, ртутная кровь.}
{This kind of people is said to have golden bones and mercury blood.}
Вместе с амуницией груз получился внушительный, и спина заныла, словно меня переехала оленья упряжка.
Но не к лицу обращать внимание на мелочи, когда товарищ в беде.

"--*Я принесу тебе поесть и воду, "--- выдохнул я.
"--- Сегодня сюда не придут, но постарайся не шуметь\ldotst

Чханэ не ответила.

На кухне дома царила тьма, лишь в углу слабо светился остывающий очаг.
Спящие на полу служанки услышали мои шаги и зашевелились.

"--*Ликхмас?

"--*Эрхэ, Ликхэ, спите.
Разморило же вас посреди кухни.
Скажите только, где остатки ужина.

"--*Тут прохладнее, "--- Эрхэ вскочила, на ходу надевая через голову рубаху.
"--- Давай разогрею, очаг ещё не совсем остыл.
Только надо\ldotst

"--*Идолы тебя загрызи, Речка! "--- вполголоса выругался я и прикрыл рот ладонью.
"--- Вытащи мне миску еды, неважно чего, и ложись.
Ложись, я сказал.

Эрхэ переспросила, не разогреть ли рыбу, вытащенную из сберегающих свежесть листьев кислотника, и, получив от меня выразительный взгляд, пожелала избежать гнева Безумного этой ночью.
Чашу чистой воды из каменной бочки я набрал сам.

"--*Чханэ.

Девушка приоткрыла слипающиеся глаза, увидела перед собой воду и жадно приникла к деревянной чаше.

"--*Ещё воды?

"--*Нет, "--- шумно дыша, просипела она.
"--- Благослови тебя духи.

"--*Мы не в пустыне, воды много.
Во дворе стоит колодец.
Он крепкий и чистый, только в нём иногда купаются змеи, если забыть закрыть крышку.

"--*Я запомню.

"--*Ла, поешь, "--- я пододвинул к ней миску с рыбой.
Девушка приподнялась на локте, и в её глазах мелькнул слабый интерес.

"--*Что это?

"--*Рыба.
В реке плавает.
Называется карп.

"--*Это рыба? "--- недоверчиво спросила Чханэ.
"--- У нас все ею ругаются, но никто не может толком объяснить, что она собой представляет.
Вкусная?

"--*Как птица согхо на вкус, но гораздо нежнее.

Рыба Чханэ не понравилась.

"--*Сколько же здесь\ldotst тху\ldotst костей\ldotst тху\ldotst

"--*Какая есть.
Кушай аккуратнее, не люблю вынимать из горла кости.
Мозг можешь выгрызть, он вкусный.
Молоки тоже ничего.

Глядя на уплетающую рыбу Чханэ, я вдруг понял, что сам хочу только две вещи: еду и чистую постель.
Я стащил с полки пару старых ковров и как можно уютнее сложил их в укромном месте, за штабелями досок чёрного дерева.

"--*Ложись, здесь прохладно.
Кстати, если желаешь справить нужду "--- в том углу дверь, возле двери большой папоротник.

"--*Я запомню, "--- девушка безуспешно пыталась справиться с завязками платья.
"--- Помоги, пожалуйста.
Шнурки склеились\ldotst

Я помог.
Чханэ рывком стащила пропитанное потом, испачканное кровью платье через голову и не без злости бросила его в угол.
Потом повернулась ко мне, и я невольно восхитился изяществом обнажённого тела.
Несколько мгновений я любовался её нежной крепкой грудью, немыслимо тонкой талией и плоским животом, перечёркнутым наискось грубыми параллельными шрамами.
Пылеройский костяной кхаагатр, вне всякого сомнения.
Повезло "--- удар пришёлся вскользь.

Чханэ соловым взглядом смотрела на меня сверху вниз и ухмылялась.
Её сердце пульсировало мягко и ровно, словно озёрная медуза, шея слегка потеплела.

"--*Насмотрелся?
Я ведь спать хочу.

"--*Храни твой сон лесные духи, "--- я повернулся, чтобы уйти.

"--*Хранят глаза и верный друг, "--- услышал я за спиной.

\section{Ливень}

Поздним вечером зарядил густой, но ещё по-летнему тёплый ливень.
Большая капля "--- неожиданная гостья.
Митхэ короткими перебежками возвращалась на постоялый двор.
По большей части травяные и сплетённые из листьев юкки навесы не спасали от дождя.
Кое-где подпорченные за лето крыши просто провалились под напором тяжёлой, как свинец, воды.
Митхэ ругалась сквозь зубы "--- поход был назначен на послезавтра, и ехать в полумокрых штанах и сапогах ей совсем не улыбалось.

Площадь Тхартхаахитра находилась в низине.
Сюда стекали потоки воды со всего города.
К утру, к началу торгового дня, канализация справилась бы с затопившей площадь стихией, но сейчас, в самый её разгар, мутные потоки плескались где-то на уровне щиколотки.
Чуть дальше, ближе к южному концу, провалиться в воду можно было и по пояс.

Митхэ ругалась всё сильнее и громче.
Наконец воительница увидела путь на другую сторону "--- перекладины навесов шли не прерываясь, по ним можно было перебраться.
Лёгкая, худенькая Митхэ, недолго думая, схватила дорожный мешок в зубы;
вскоре она уже по-обезьяньи поползла по перекладинам, жмурясь от летящих прямо в глаза водяных капель.

Вдруг её слуха снова коснулся звук знакомой цитры.
Ошеломлённая воительница обернулась и, сорвавшись с навеса, шмякнулась в мутную, пахнущую гниющими водорослями воду.
Дорожный мешок жалобно хрустнул в стиснутых зубах.

Атрис сидел под дождём на том же месте.
Изорванная одежда из грубой ткани насквозь пропиталась водой, волосы висели мокрым войлоком.
Цитра на его коленях булькала и хрипела, но он продолжал играть бодрую разудалую песню.

Митхэ не выдержала.
Она пошлёпала по колено в воде, не обращая внимания на набравшуюся в сапоги осеннюю муть.
Она схватила менестреля за плечи и тряхнула.
Ей хотелось закричать на него:
<<Что ты делаешь?
Зачем ты мокнешь под дождём?
Зачем ты играешь эту глупую песню?
Разве ты не понимаешь, как фальшиво она звучит\ldotsq>>

Атрис поднял на Митхэ потускневшие, полные печали глаза "--- две тусклых зелёных искры над запавшими щеками, "--- и злость Митхэ захлебнулась.

"--*Идём, "--- коротко бросила она и, схватив менестреля за руку, потащила его под ближайший навес.
Атрис молча пошёл за ней.

"--*Снимай. Снимай всё, "--- приказала она и стала вытряхивать из мешка самые сухие рубашки.
"--- Пожалуйста, "--- добавила она, опомнившись.

Атрис стоял, не двигаясь, и смотрел на Митхэ во все глаза.

"--*Да снимай же!

Митхэ начала срывать с мужчины лохмотья;
менестрель качался, как деревце под шквальным ветром.
Грубая мокрая ткань трещала, не желая поддаваться.
Боевая раскраска Митхэ поплыла, и слёзы катились по лицу вперемешку с каплями ливня, сурьмяной кашицей и дорожной грязью.
Воительница яростно вытирала лицо ладонью и снова хваталась за рубашки, оставляя на них серые отпечатки.

Но рубашки быстро пропитались водой и потеряли уютное тепло.
Митхэ в отчаянии зарыдала.
Зарыдала, уткнувшись в полуголую мокрую грудь Атриса, до крови прикусив губу.
Вдруг она почувствовала его объятья.
Тонкие музыкальные руки с изящными пальцами едва заметно дрожали.

"--*Атрис, пойдём со мной. Пойдём, "--- сквозь рыдания шептала Митхэ.
"--- У меня тепло и сухо.
У меня есть одежда, еда, всё, что только хочешь.
Пойдём.

"--*Я пройду с тобой до смерти и ещё пару шагов, "--- растерянно пробормотал Атрис.
"--- Только скажи для начала, как тебя называть.

Митхэ всхлипнула и прижала его к себе крепко-крепко.
Выпавшая из рук Атриса цитра грустно звякнула и захлебнулась водой.

Ночь спрятала город, как устрица жемчуг, но это не имело значения.
Ничто в мире уже не имело значения.
Порог комнаты в постоялом дворе переступили две пары ног.

\section{Легенда о спящей}

Для жреца ночь "--- это жаркое время.
Но я, слава духам, жрецом тогда ещё не был и имел полное право спать ночью, словно весь день рубил дрова.

Стараясь ступать как можно тише, я пошёл вверх по лестнице, ведущей в зал.
Купеческий двор, который занимала моя кормилица Кхотлам ар’Люм, построили очень давно, ещё до её рождения.
Насколько я знал, зодчий двора был одним из лучших в Тхартхаахитре;
впрочем, то же мог сказать и любой прохожий.
Одному Безумному известно, каким образом зодчий возвёл шедевр из простого малахита и медного шлака.
За сто пятьдесят дождей ни одна резная колонна с ликами духов, ни одни ажурные перила в виде разинувших пасть змей не потребовали реставрации.
Мужчина Кхотлам, Хитрам ар’Кхир, был из крестьян, но его смекалка часто помогала кормилице вести дела.
Так как я собирался стать жрецом, купеческому делу учили моих сестрёнок-близняшек, Манэ и Лимнэ, не достигших ещё двадцати пяти дождей.

Хоть кормилица и не знала нужды, на доспехи, оружие и свитки я зарабатывал сам "--- неписанное правило.

И всё же, что делать с Чханэ?
Игра в прятки не могла тянуться вечно.
Нужно было посвятить в тайну кормильцев, рассказать им о девушке.
Не истину, нет, но часть истины.

Посреди зала, обставленного плетёными креслами и пахучими циновками, горел обделанный малахитом конус-очаг.
Перед ним сидела с рукоделием Кхотлам "--- узелковое плетение приводит в порядок мысли, к тому же занавеси и ажурные салфетки хорошо продавались приезжим.
Я шумно вдохнул.

"--*Лисёнок, подойди, "--- встрепенулась от сладкой дрёмы кормилица.
"--- Успешна ли была твоя охота?

Я сел перед ней на колени и взял её руки в свои.

"--*Три птицы сиу'сиу, кормилица.
И мраморная змея по дороге попалась.
Завтра продам на рынке.
Идолов волею Безумного встретить не довелось.

"--*А их уже полгода никто не видел.
Наверняка готовят очередной набег.
Снова нам смазывать стрелы.

"--*Тебе не обязательно.

"--*Как же, не обязательно, "--- с укоризной сказала Кхотлам.
"--- Ты же не хочешь, чтобы женщина, которая тебя воспитала, прослыла трусихой?

Кормилице на вид было не больше шестидесяти.
Все говорили, что мы похожи;
я считал иначе.

"--*Ты храбрая женщина, родильница, и я не опозорю твоё лоно, "--- я улыбнулся, давая понять, что предыдущая фраза была лишь проявлением заботы о ней.
Кхотлам вдруг смутилась и опустила глаза.
Верёвочки плетения снова пустились в пляс.

"--*Кого сегодня принесут в жертву?

"--*То ли ребёнка кузнеца ар’Хетр, то ли другого, с приречного хутора.
Я их видела, красивые.
Печально отдавать таких Безумному, "--- кормилица улыбнулась и характерным жестом тронула пальцем язык, показывая, чтобы я не принял её слова как богохульство.

Ненависть к богу была обычным делом;
многим бы даже показалось странным её отсутствие.
Однако такая вещь, как богохульство, не приветствовалась "--- в его руках было слишком много власти.

В жертву чаще всего приносили детей, не прошедших Отбор.
Пятого числа каждой декады в город выходили жрецы, повязанные красными лентами, и почти всегда возвращались с теми, кому было суждено продлить жизнь поселения ещё на несколько дней.
Платой была традиционная золотая пирамидка;
я видел такие "--- ими порой рассчитывались на рынке.
Иногда на алтарь отправлялись и кутрапы, и не только, если другого выхода не было.
Одна мучительная смерть для города лучше, чем волна радужного безумия, после которого люди оплакивали убитых ими самими друзей.
Суровая необходимость\ldotst

Храм же по воле Безумного рушился постоянно, по камешку, по дощечке, иногда целыми кусками.
При каждом храме состоял зодчий и десяток строителей, восстанавливавших здание.
<<Будь у Безумного ноги, он бы жарил их на вертеле>>, "--- так говорили люди.
Хотя учитель Трукхвал как-то обмовился, что Безумный здесь ни при чём.

"--*Кормилица, я хочу тебе кое-что открыть.

"--*Я слушаю тебя.

"--*Я нашёл в джунглях девушку, отбившуюся от каравана, и привёл её домой.

"--*Отбившуюся от каравана? "--- Кхотлам пристально посмотрела на меня.
"--- Я не слышала, чтобы здесь проходили караваны последние десять дней.

Я почувствовал, что дал маху.
Порой забываешь, что кормилица "--- опытный купец.
Надо было выворачиваться.

"--*Где именно ты её нашёл?

"--*В сотне шагов от тропы, ведущей в Живодёрские леса, "--- осторожно начал я.

"--*Хай, "--- успокоилась Кхотлам.
"--- Какие-то безумцы опять повели караван там в надежде сократить время.
Меня никто предупредить не догадался, а рассказывать про опасности Цыплячьей тропы я уже устала.
Караван погиб?

"--*Да, "--- легенда для Чханэ была почти готова.

"--*А сама она откуда?

"--*С пустынного города.
Она не сказала какого, "--- тут уж девушка пусть сама выкручивается, вдруг кто-то из местных знаком со жречеством Тхаммитра.

"--*Кихотр, "--- кормилица покачала головой.
"--- Удивительная по своей нелепости история.
Кто она по крови?

"--*Сели.
Её выносила и родила купец, семя "--- воинское.
Ей тридцать пять дождей, на ней богатое платье, явно сшитое для неё.
Оцелотовая кожа.
Держит себя достойно, и черты её благородны.

"--*Хаяй, сколько достоинств у этой девушки, "--- задумалась Кхотлам.
"--- Ещё и оцелот.
Что ж, утром поглядим на неё.

"--*Кормилица, "--- я решил довести дело до конца, "--- понимаю, что моё решение несколько поспешно, но я хочу видеть её своей женщиной.
Что мне делать?

"--*Лис! "--- Кхотлам бросила рукоделие и взяла меня за руки.
"--- Ведь ты ещё не стал жрецом, а она не стала тебе другом!

"--*Я понимаю, кормилица.
Но я не знаю женщины, более достойной носить и кормить грудью моих детей.

Добрые глаза кормилицы вдруг улыбнулись, но губы наставительно сказали:

"--*Дай ей время обжиться, Лисёнок.
Женщину могут оскорбить ухаживания человека, от которого она зависит в силу обстоятельств.
Где ты ей постелил?
Нуждается ли она в чём-нибудь?

"--*В подвале, за досками чёрного дерева.
Я принёс ей воды и пищи.

"--*Хорошо.
Не будем её беспокоить.
Иди спать, Ликхмас.
Храни твой сон лесные духи.

"--*Хранят глаза и верный друг, "--- улыбнулся я.

"--*Опять эта воинская чушь, "--- пробормотала Кхотлам, снова занявшись рукоделием.
"--- Чего бы стоили их глаза без искусства дипломата\ldotst

Попрощавшись с кормилицей, я украдкой побежал в подвал и разбудил Чханэ, чтобы рассказать ей её легенду.

Теперь можно и поспать.

\section{Любовная лихорадка}

Заснуть любовники так и не смогли.
В ту ночь Митхэ охватил жар.
Её знобило так, что зуб на зуб не попадал.
Атрису было не лучше.
Меняя друг другу влажные полотенца и поглощая один чайник отвара за другим, они кое-как протянули до рассвета.

"--*Кошмар, "--- пробормотала Митхэ и закашлялась, кутаясь в одеяло.
"--- Давно я так не болела.

"--*Мы оба расплачиваемся за то, что уделяли себе чересчур мало внимания, "--- ответил менестрель.
"--- Я давно разучился это делать.
Поухаживаем друг за другом, вдруг поможет.

На следующий день стало только хуже.
Вставали любовники редко и лишь по нужде.
Атрис сильно страдал по мужской части, временами плача от долгой неутихающей боли "--- сказалось длительное сидение на площади в любую погоду.
Митхэ аккуратно держала мокрое полотенце у него на паху.

"--*Зачем тебе это? "--- бормотал Атрис сквозь слёзы.

"--*Тише, милый, тише, лежи, "--- ласково отвечала ему воительница.
В её уставших, воспалённых глазах светилась чистая любовь, немыслимая для женщины её возраста, и этот нежный взгляд действовал на менестреля сильнее всех лекарств.

В конце концов Митхэ, поколебавшись, отправила посыльного к торговцам с просьбой об отсрочке, а затем, собрав своих людей, сообщила им о новой договорённости.

"--*Если кто-то торопится на встречу со смертью, можете выдвигаться без меня, "--- устало добавила Митхэ, услышав тихий ропот.
Наёмники, покачав головами, разошлись.

На третий день болезнь отступила.
Под вечер Митхэ решила искупаться и привести в порядок вещи, которые так и валялись по всей комнате.

Атрис лежал молча.
Он дышал глубоко, с тихим сопением, но Митхэ знала, что он следит за каждым её шагом по комнате.
В бреду они говорили о чём-то важном, но Митхэ не помнила из разговоров ни слова.
Осталось лишь понимание "--- бессловесное ощущение того, что ещё несколько кусочков мозаики мироздания встали на место.

"--*Твои люди ждут, "--- напомнил менестрель.

"--*Я без тебя никуда не поеду, "--- ласково, но твёрдо отрезала Митхэ.

"--*Мои руки трясутся, когда я прикасаюсь к цитре.
Моё мужское естество уничтожено болезнью.

"--*Ты ещё не выздоровел, милый.

"--*А если я останусь таким до смерти?

"--*Я буду к тебе нежной до смерти и два шага после.

Атрис не стал спорить.
Вскоре Митхэ почувствовала, что её тянет в сон.
Она наскоро обглодала куриную ножку, которая с характерным запахом задумалась о смысле жизни, и снова полезла к менестрелю под одеяло.

Атрис долго лежал и смотрел на спящую Митхэ.
Она была ещё бледна после неожиданной болезни, но от всего её маленького и тёплого, чутко спящего существа веяло невыразимой прелестью.
Уютная горячая вода купели сделала своё дело "--- кожа Митхэ засияла здоровым розово-золотистым светом, а кудри стали мягче, чуточку светлее и замерцали, словно перламутр.

Разумеется, он с первой встречи знал, кто она такая.
Невозможно жить в Центральном городе и не знать Митхэ ар'Кахр "--- легенды про лист юкки и поднятую деревянными гребнями пыль рассказывали на каждом перекрёстке.
Однако Атрис был уверен, что сейчас ни один пылерой, ни один тенку не узнал бы в ней того ужасного, выпачканного в крови и пыли военачальника.
К Атрису прижималась самая прелестная на свете женщина, отзывчивая на тончайшие и нежнейшие из ласк.

Менестрель не знал и не мог знать понятия <<норма реакции>>, но ускользающая, невнятная, едва осознанная истина успела потрясти его до глубины души.
Он погладил пальцами рваный шрам на губе, затем коснулся другого шрама "--- на месте правой груди.
Кажется, она рассказывала об этом в бреду.
Груди её лишило зазубренное копьё тенку.

За свою славу воительница заплатила многим.

Митхэ очнулась от сладкой дрёмы, испуганно задышала и зашарила рукой по одеялу, отыскивая Атриса.

"--*Я здесь, "--- нежно шепнул менестрель.
"--- Я рядом, я всегда буду рядом.

Митхэ облегчённо вздохнула, крепче обняла Атриса и погрузилась в глубокий сон.
Она узнала всё, что желала знать.

Атрис лежал в объятиях Митхэ и, может быть, впервые в жизни чувствовал себя защищённым.
Он спокойно думал о предстоящем походе, который согласился разделить со своей женщиной, о сезоне дождей и о том, что его цитра давно требует капитального ремонта.
Нужно было заменить серебряный колок, пятую струну и просевшее ушко маховика\ldotst
Мысли, словно ленивые откормленные карпы, плавали в приятно отяжелевшей голове.
Менестрель больше не боялся.
Митхэ приходила каждый раз, когда он звал её в мыслях, и он знал, что она будет находить его снова и снова, в какое бы далёкое межзвездье не унёс его ветер времени.

Мало-помалу Атрисом овладели странные образы из преддверий снов.
Словно листья из веток, они лезли друг из друга, следуя одним им известной дорогой ассоциаций.
Знакомые слова сплетались в бессвязные предложения, имевшие смысл лишь на той неуловимой грани сна и бодрствования.
И, уже приветствуя Сана, Атрис успел подумать:

<<Неужели так бывает\ldotsq>>

\section{Первое знакомство}

Двигатели затихли окончательно.

Мы не стали будить тси сразу.
Хотелось, чтобы первым, что они увидели, был рассвет.
Что-то мне подсказывало, что на этой планете очень красивые рассветы.
Друзья молчаливо со мной согласились.

И вот чудо произошло.
На небе появились яркие оранжевые полосы.

"--*Как же хочется выйти\ldotst "--- прошептала Заяц.

"--*Пока нельзя, "--- сказал Фонтанчик.
"--- С чужой микрофлорой шутки плохи.

"--*К тому же нужно спросить разрешение у хозяина, "--- добавил я.
"--- Безымянный, кем бы он ни был, дал разрешение только на посадку.

"--*Сейчас спрошу, "--- вдруг заулыбалась Заяц и, пока мы не успели сообразить, бросилась к панели.
"--- Так, Баночка, кинь ссылку на словарь в репозитории\ldotst
А, всё, нашла.

\mulang{$0$}
{<<Безымянный! "--- пронеслись в пространстве слова Эй-00.}
{``Nameless!'' the Ej-00 words flew through the space.}
\mulang{$0$}
{"--- Ты меня слышишь, дружище?}
{``Do you hear me, man?}
\mulang{$0$}
{Говорит Темнотой-Сотканный-Заяц из корабля тси.}
{It's Darkness-Woven-Hare in the Qi spaceship.}
\mulang{$0$}
{Конец сообщения>>.}
{Over.''}

\begin{quote}
\mulang{$0$}
{<<Это Безымянный.}
{``It is Nameless.}
\mulang{$0$}
{Вероятно, Темнотой-Сотканный-Заяц хочет узнать, стабильна ли связь.}
{Perhaps Darkness-Woven-Hare wants to know how stable the connection is.}
\mulang{$0$}
{Отвечаю: связь стабильна.}
{My answer: the connection is stable.}
\mulang{$0$}
{Я осуществляю постоянный мониторинг этой частоты, тси могут связаться со мной в любое время.}
{I am constantly monitoring this frequency signal, Qi can call me anytime.}
\mulang{$0$}
{Конец сообщения>>.}
{Over.''}
\end{quote}

"--*Я совсем забыла, что у них другая рецепторная система и мировосприятие, "--- Заяц хлопнула ладонью по лбу.
"--- Но он умный и понял.
Это хорошо.

\mulang{$0$}
{<<Безымянный, тси спрашивают разрешения покинуть корабль и вступить в атмосферу планеты.}
{``Nameless, Qi ask permission to get out of the spaceship and to enter the atmosphere of your planet.}
\mulang{$0$}
{Тси устали сидеть внутри корабля и хотят дать физическую нагрузку конечностям.}
{Qi are tired of being inside, they want to give their limbs a workout.}
\mulang{$0$}
{Конец сообщения>>.}
{Over.''}

\begin{quote}
\mulang{$0$}
{<<Темнотой-Сотканный-Заяц, отвечаю: даю вам разрешение находиться в любой точке планеты, включая её атмосферу и зону доступного регистрации магнитного поля, и давать конечностям любую необходимую нагрузку.}
{``Darkness-Woven-Hare, my answer: you have my permission to be anywhere in the planet, including its atmosphere and measurable magnetic field, also to give your limbs any workout you need.}
\mulang{$0$}
{Предупреждение.}
{Warning.}
\mulang{$0$}
{Для посадки выбрана точка планеты, в которой наблюдается практически нулевая концентрация микро- и макроорганизмов, а температура, радиация и влажность оптимальны для представителей вида Homo homo sapiens.}
{In the point chosen for the landing, there is a low concentration of macro and micro organisms, and temperature, radiation, and moisture levels are optimal for a member of the species Homo homo sapiens.}
\mulang{$0$}
{Однако я не знаю, к какому биологическому виду относятся тси и какова ваша физиология, поэтому советую вам озаботиться собственными средствами защиты.}
{However, I know neither a species you belong to nor your physiology, that is why you are encouraged to use your own protective equipment.}
\mulang{$0$}
{Конец сообщения>>.}
{Over.''}
\end{quote}

"--*Очень интересно, "--- протянул Фонтанчик.
"--- Он не знает, к какому биологическому виду относятся тси, но при этом говорит на Эй-00?
Хм.
Заяц, узнай у него поаккуратнее насчёт Ордена Преисподней.

<<Безымянный, тси поняли и приняли к сведению твой совет.
У нас имеются средства защиты.
Тси спрашивают, известно ли тебе, кто мы и кто такие демоны.
Конец сообщения>>.

\begin{quote}
<<Темнотой-Сотканный-Заяц, отвечаю на первый вопрос: вы "--- объединение сапиентов под названием <<тси>>.
Название я впервые узнал из вашего сообщения.
Я не знаю признаков, по которым выделяется это объединение.
Отвечаю на второй вопрос: мне известно, кто такие демоны и по каким признакам выделяется это объединение.
Я осведомлён о том, что принадлежу к данному виду сапиентов.
Также я владею языком демонов "--- Эй, таблица 00.
Конец сообщения>>.
\end{quote}

"--*Всё ясно, "--- затараторил Баночка.
"--- Я думаю, что Безымянный "--- творение одной из поздних сапиентных цивилизаций, которые когда-то находились под властью Ордена Преисподней.
Это объясняет и тот факт, что он упомянул Эй как единственный <<язык демонов>>, при том, что <<языков демонов>> как минимум три.
А ещё он сделал одну очень нехарактерную для Ада и Картеля вещь "--- он отнёс себя к сапиентам.

"--*Баночка прав, "--- признал Фонтанчик.
"--- И знаете, какое у меня ощущение?
Как будто мы только что поговорили с ребёнком.
Он ничего не спросил про полномочия Заяц вести с ним переговоры, да и в целом не силён в дипломатии.
Он совершенно нас не боится.
Демоны Ада и Картеля говорили с нами не иначе как через цепь спутников, с почтительного расстояния, а он\ldotst ну\ldotst

"--*Он "--- что? "--- осведомилась Подсолнух.

"--*Он только что осмотрел тебя, Подсолнух!
Датчики подтверждают наличие считывающих жуков.
А во время разговора он осматривал меня и Небо.
А чуть ранее он просканировал корабль, но крайне осторожно, на пределе чувствительности, иначе система сразу бы взвыла.
Ему интересно!
Он не совсем понимает, где живые существа, а где системы корабля, и поступает как ребёнок, которому интересно насекомое, но который знает, что неосторожным движением его можно раздавить!

"--* Фе, "--- фыркнула Подсолнух и демонстративно отряхнула костюм.
"--- Он что, летал внутри меня?

Заяц подумала и снова обратилась к демиургу:

<<Безымянный, тси спрашивают, есть ли в этой звёздной системе демоны кроме тебя.
Если нет, то были ли они раньше.
Конец сообщения>>.

\begin{quote}
<<Темнотой-Сотканный-Заяц, отвечаю: в настоящее время я являюсь единственным демоном в этой звёздной системе.
Ранее здесь появлялись демоны, но ни один не пожелал со мной общаться и не остался в её пределах.
Конец сообщения>>.
\end{quote}

\mulang{$0$}
{"--*Бедненький, "--- посочувствовала Заяц.}
{``Poor child,'' Hare said with compassion.}
\mulang{$0$}
{"--- Никто не хочет с ним дружить!}
{``No one wants to be friends with it!''}

"--*Будите основную команду и готовьте скафандры, "--- вздохнул я.
"--- Полюбуемся рассветом снаружи.
Подсолнух, хватит отряхиваться, он тебя ещё не раз будет осматривать.

"--*По-моему, немного невежливо без разрешения собеседника изучать его квантовую структуру, "--- буркнула Подсолнух.

Заяц завизжала от радости и прыгнула в гравитационный лифт.

"--*Заяц, подожди! "--- крикнул Фонтанчик.
"--- А поесть?
Мы голодные как волки\ldotse

\section{Дикая похлёбка}

%Лук 7 дождя 11998, Год Церемонии 18.

Вскоре дома вкусно запахло наваристой похлёбкой.
<<Дикая похлёбка>>, сваренная с оленьей кровью и слизистой жижей стеклянных ягод, была коронным блюдом Ликхэ.
Пока Кхотлам рассаживала близняшек, кормилец помог служанкам принести еду в зал.
Мне доверили отрегулировать конденсатор "--- день обещал быть душным, приближался сезон дождей.

Все домашние уже сидели за столом, когда из подвала вышла заспанная Чханэ.
Она сменила окровавленную зелёную повязку на свежую, белую, и надела чистую одежду кормильца, которую я принёс ей заранее вместе с бинтами.
Хитрам был самым высоким в доме "--- и всё равно его штаны оказались моей новой знакомой чересчур малы.
Глаза девушки горели голодным огнём "--- сколько-то она провела в пути без пищи?

"--*Ага, "--- довольно сказал Сиртху-лехэ, "--- вот почему от оленя так странно пахло.

Хитрам скрыл улыбку за широкой рабочей ладонью и толкнул меня в бок.
Кормилица шикнула на него и кивнула вошедшей.

"--*Тханэ ар’Катхар, "--- Кхотлам вдруг нахмурилась, окинув взглядом подбородок и запястье девушки, "--- тебе требуется жрец?

"--*Нет, благодарю, жрец меня уже подлечил, "--- Чханэ поклонилась и чуть заметно улыбнулась мне.

"--*Не церемонься, садись и ешь столько, сколько захочешь.
Вид у тебя голодный.
Расскажешь обо всём потом.

Чханэ поклонилась ещё раз и осторожно села на низкий табурет.
Её колени чиркнули по краю стола, и посуда нежно зазвенела.

"--*Прошу прощения, "--- смутилась девушка.

Кормилица молча встала, передвинула свой табурет и устроилась рядом с Чханэ.
Затем зачерпнула из миски похлёбку и поднесла черпачок ко рту гостьи.
Чханэ, не говоря ни слова, съела предложенное.
Вскоре, игнорируя черпачок, она схватила миску руками и, громко чавкая, впилась в горячую наваристую жижу.

Закончилась трапеза в молчании.
Кормилец рассеянно крутил в руках чашу с водой и смотрел умными глазами в пространство.
Служанки и Сиртху-лехэ переговаривались жестами "--- судя по всему, речь шла о детях общих знакомых.
Близняшки с любопытством смотрели на Чханэ "--- они встретили оцелота впервые.
Кхотлам не то насмешливо, не то с нежностью поглядывала на меня.
Я, решив на время отложить смущение, терпеливо ждал вопросов.

"--*Храни вас лесные духи, "--- наконец выдохнула Чханэ.
"--- Кто бы это ни приготовил, желаю ему ясных рассветов и лёгкости в руках.
Я давно не ела такого вкусного блюда.

Ликхэ довольно осклабилась.

"--*Я так понимаю, настало время представиться, "--- Чханэ решила взять инициативу в свои руки.
"--- Я воин Храма Чхаммитра, мои дарители "--- воин Акхшар ар’Качхар э’Чхаммитра и купец Шогхо ар’Хэ э’Чхартхаахитр.
Я\ldotst

"--*Ты хранительница Согхо ар’Хэ? "--- вдруг оживилась Кхотлам.
"--- Знаю её, очень хорошо знаю.
Кажется, дождей пятьдесят назад она служила Сотрону?

"--*Травинхалу, "--- усмехнулась Чханэ.
"--- Как она говорит, <<Акхсар провёз меня через половину мира "--- из одной глуши в другую>>.

Повторив слова Согхо, девушка непроизвольно воспроизвела чужой акцент, тембр голоса и интонации.
Кормилица расслабилась и многозначительно мне кивнула.

"--*Откуда был твой караван?

"--*С Кристалла, "--- ответила Чханэ.
"--- Откуда точно "--- не скажу, нанялась сопровождать в Мишитре.
Зизоце "--- неразговорчивый народ.

"--*Это точно, "--- согласилась Кхотлам.
"--- Все погибли?

Девушка кивнула.

"--*Их главный непременно хотел достичь Ихшлантекхо\ldotst я имею в виду, Ихслан\ldotst тхара к началу страды.

"--*Я тебя поняла, не волнуйся, "--- успокоила кормилица гостью. "--- Как звали предводителя, на помнишь?

Я похолодел.

"--*Имя цатрон Тхартху ар'Ликх ми'Эрхо'люаэсакх, "--- без запинки оттарабанила Чханэ первое пришедшее на ум.
"--- Имя зизоце не знаю.

Кормилица нахмурилась.

"--*Странно, они редко используют имена цатрон, тем более с <<ми>> "--- они очень привязаны к родным поселениям.
И спешка у них, мягко говоря, не в чести.
Впрочем, хороший знак.
Эти зизоце порой чересчур медлительные, а имена у них длиннее дороги до Тхаммитра.
Так что произошло?

"--*Они не согласились на волок, обменяли корабли у каких-то сомнительных дельцов.
До Чхартхаахитра не спустились, прямиком из Мишитра снарядили караван.
Шли очень близко к землям идолов "--- возможно, предводителя ввела в заблуждение близость Омута Духов, только он-то не знал, что там недавно\ldotst

Девушка запнулась, а меня прошиб пот.
О событиях в Омуте Духов кормилица рассказала домашним по большому секрету.
Чханэ, которая не могла о них знать, чересчур увлеклась и едва не раскрыла свой источник информации.
Впрочем, Кхотлам выглядела расслабленной.

"--*\dots что идолы услышали нас задолго до Живодёра, "--- смятение Чханэ длилось едва ли секхар.
"--- Между наёмниками с Грозового хребта и нашими разгорелся спор, который перерос в драку.
Я не знаю, о чём они спорили, но думаю, что из-за выигрыша в какой-то азартной игре.
Встретили нас с честью, спасло меня только то, что я была в арьергарде.
Я спала в палатке на обозе, даже одеться не успела "--- обоз перевернулся\ldotst

"--*А\ldotst почему платье?
Да ещё и такое дорогое? "--- удивлённо спросил Сиртху-лехэ.

"--*А мне голой по джунглям бежать? "--- со смехом осведомилась девушка.
"--- Надела, что было.
Жизнь свою с собой унесла "--- и то хорошо.

"--*Послушай, "--- вдруг заговорил Хитрам.
"--- У тебя очень интересное имя.
Первый раз встречаю женский вариант имени Тхалас.
И ещё ты очень высокая.
Неужели твои дарители такие же?

"--*Согхо, как тушканчик "--- маленькая и юркая, "--- рассмеялась девушка.
"--- Кормилец чуть повыше, но тоже ростом не вышел.
А имя мне пришлось придумать новое, потому что раньше я была парнем.
Был Чхалаш "--- стала Чханэ.

Кхотлам ахнула, Сиртху-лехэ тихо пробурчал под нос:
<<Хаяй\ldotst редкость>>.
Я непонимающе уставился на старших.

"--*Смена пола, "--- объяснила кормилица, увидев моё замешательство. "---
Обычно дети выбирают пол навсегда, но бывает и такое "--- зрелый мужчина становится женщиной или наоборот.
Я не знаю, почему это происходит.
Тебе следует спросить в Храме.

"--*Не совсем зрелый.
Моё превращение началось через пару дождей после того, как у меня появились мужские признаки, "--- поправила Чханэ.
"--- Жрец сказал, это из-за того, что я ударилась головой в детстве.
Других причин он не нашёл.

"--*А как изменялось твоё тело? "--- поинтересовался я.

Кормилица неодобрительно посмотрела на меня, но промолчала.

"--*Я не знаю, "--- растерянно ответила Чханэ.
Она явно не ожидала таких личных вопросов.
"--- Проснулась однажды утром, а у меня грудь набухла и в паху как-то странно зудит.

Я усмехнулся.
Кхотлам попыталась шлёпнуть меня под столом, но промахнулась;
я почувствовал лишь слабое движение воздуха.

"--*Смейся, смейся, "--- буркнула Чханэ.
"--- Посмотрела бы я на тебя.

"--*А потом?

"--*Потом таз и спина заболели, да так, что хоть плачь.
За сезон обзавелась грудью и роскошной задницей "--- кости на четыре пальца удлинились.
А после\ldotst

Обстановка всё же немного разрядилась.
Случай, произошедший с Чханэ, был большой редкостью, и домашние слушали девушку с неподдельным интересом.
Чханэ стала вести себя свободнее.
Внезапно из-за высоких скул и благородного лба, лежащего на своде длинных, размашисто очерченных бровей, проступила симпатичная весёлая говорушка.

"--*\dots а задница-то тяжёлая, двигаться пришлось учиться заново, и драться тоже.
В Храме мне доставалось, конечно, особенно первое время.

"--*Кем ты себя чувствуешь? "--- спросила Кхотлам.

"--*Сложно сказать, "--- пожала плечами девушка.
"--- Когда была парнем "--- не задумывалась.
Сейчас\ldotst больше женщиной, чем мужчиной.
Примерно три к одному.
Самое смешное "--- кормильцы договорились, мол, если буду девочкой "--- стану купцом, если мальчиком "--- пойду в храм.
А хитрая Чханэ всех обманула.

"--*Это урок твоим кормильцам, "--- громко сказал Сиртху-лехэ и стукнул кулаком по столу.
"--- Пусть молодёжь сама выбирает свой путь.
Мы можем лишь предложить то, что имеем.

"--*Правильно, "--- поддержала старика Кхотлам.
"--- Военное дело не имеет пола, как и дипломатия\ldotst

\section{Надежда}

Что делать с Безымянным?

Этот вопрос сегодня был главным на повестке дня.
Дискуссия шла в закрытом, отгороженном от колебаний первичного поля зале Стального Дракона.
Может, демиург и не силён в дипломатии, но большая часть тси была уверена, что дипломатические отношения установить необходимо.
Я придерживался другой точки зрения:

"--*Дипломатия помогает общаться тем, кто друг другу не доверяет.
Он "--- ребёнок.
Вы хотите установить дипломатические отношения с ребёнком?

"--*Он "--- хоргет, "--- заявил Мак.
"--- Небо, не повторяй ту ошибку, которую мы допустили с Машиной.
Для нас в приоритете его сила и способности, а не психология, вернее, эмуляция психологических процессов.

Как ни странно, меня поддержали многие диктиологи.
Одна из них, Гладит-Зелёную-Кошку, выдвинула совершенно революционную идею, вызвавшую лишь смех:

"--*Я считаю, что мы должны его воспитать.

Впрочем, началось всё с плана развернуть планетарную систему защиты против демонов, которая у нас была на Тси-Ди.
Тси приняли это решение сразу, несмотря на замечание некоторых товарищей о том, что вначале планету нужно узнать получше.
Баночка уже успел пошутить по этому поводу:

"--*Кажется, тси заболели золотой лихорадкой.

Из истории мы знали, что когда-то основной обменной валютой было золото.
Преимущества золота как валюты очевидны: оно инертно и почти не подвергается окислению, оно компактно, оно легко обрабатывается и может быть идентифицировано даже сапиентом, мало смыслящим в химии, благодаря своеобразному цвету и консистенции.
История знала много <<золотых лихорадок>>, когда сапиенты пытались получить ресурсы путём добычи огромного количества валютного материала.
Разумеется, это не приводило ни к чему хорошему "--- количество пищи и материалов оставалось прежним, и валюта обесценивалась, принося разочарование добытчикам и снижая качество жизни их собратьев.

У нас к золоту был свой интерес "--- для резонаторных модулей планетарной системы и некоторых типов микросхем годится только этот металл.
Мы нашли первые месторождения, едва выйдя из корабля "--- золотой песок лежал на морском берегу вперемешку с кварцевым.
Биолог Цветущий-Мак-под-Кустами сказал, что в жизни не видел ничего более красивого, и уже набрал немного прибрежного песка для аквариума.
Правда, предусмотрительно оставил его в лагере рядом с кораблём "--- с чужой микрофлорой шутки плохи.

К счастью, среди тси нашлись несколько техников, которые работали с системой планетарной защиты и имели хорошее представление о её устройстве.
Одним из них оказался мой друг "--- Хрустально-Чистый-Фонтан.

"--*Я смогу собрать систему, "--- заявил Фонтанчик.
"--- Мне будут нужны данные о планете и материалы.
Также хочу попросить треть вычислительных мощностей Стального Дракона.

Друг не любил брать на себя руководство;
однако его отличала одна особенность "--- он тонко чувствовал момент, когда иначе поступить нельзя.
Фонтанчику и ещё пятерым тси отдали кабинеты рядом с библиотекой, в которых они и пропали на пятнадцать дней.
Иногда у меня проскакивала мысль, что Фонтанчик умер;
впрочем, работавшие на полную мощность компьютеры Дракона и обновлявшиеся со скоростью света исходники в репозитории твердили, что Фонтанчик и прочие архитекторы живы настолько, насколько вообще можно быть живым.
В конференции они переговаривались исключительно чертежами и строками кода;
кое-кто умудрялся шутить на языке Си-поинт.

Время от времени, несмотря на ворчание архитекторов, техники просматривали уже готовые чертежи.
Систему Фонтанчик строил грубо "--- о стравлении напряжённых углов и тонкой настройке речь не шла, "--- но очень добротно и надёжно.
Остался доволен даже придирчивый Баночка.

"--*Болтаться будет.
И энергию жрать.
А с такой напряжённостью углов мы будем систему раз в сто лет калибровать, это кошмар какой-то, "--- резюмировал он.
"--- Небо, ты не помнишь, когда в последний раз калибровали <<сову>> на Тси-Ди?

Я задумался.

"--*А это не тот год, когда твоя любимая группа выпустила свой последний альбом?
Ты рассказывал как-то.

"--*Слушай, точно, "--- Баночка хлопнул себя по лбу.
"--- Он так и назывался "--- <<Калибровка>>, там же они все были техниками защитной системы.
Днём "--- работа, ночью "--- музыка.
И как ты только запомнил?
Так когда это было\ldotst
Восемьсот лет назад, получается?
Восемьсот четыре.
И система бы ещё столько же проработала без вмешательств.

"--*Мы не располагаем такими возможностями, "--- развёл я руками.

"--*Нет-нет, я рад, определённо, "--- замотал головой Баночка.
"--- Фонтанчик молодец, я бы лучше него не сделал.
Так что ладно, был бы какой-никакой план, а там сориентируемся.

К осуществлению плана было одно-единственное препятствие "--- демиург Планеты Трёх Материков.

Все понимали, что хоргет представляет для выживших тси огромную опасность.
Да, он позволил нам жить на планете.
Но он хоргет.
Было очевидно, что Безымянный, каким бы ребёнком он нам ни казался, не оценит разворачивание системы против ему подобных.

Проводились аналогии с Машиной "--- вновь увидеть геноцид не хотел никто.
<<Но он мог уничтожить нас ещё в космическом пространстве!>> "--- возражали сторонники сотрудничества.

Сторонниками противостояния предлагались самые разные варианты действий "--- от дезинформации до внезапного удара экраном по месту локализации бога.
На последнее предлагалось потратить оставшуюся на корабле энергию.
Энергию, которой и так не хватало\ldotst

Точку в споре поставила сестра Фонтанчика, Съешь-Рычащий-Пирожок.
Вот её слова:

"--*Мы не имеем морального права отбирать сотворённое у творца, кем бы ни был творец.
Мы не имеем морального права ответить предательством на помощь, кем бы ни был помогающий.
Ведя тайную или открытую войну против Безымянного, мы обернём против себя и планету, так как они "--- одно целое.

Это был веский довод, учитывая, что Пирожок не питала к демиургу ни симпатию, ни доверие.
Она откровенно признала, что запустила бы робота и уничтожила бы бога без раздумий, будь она на моём месте;
и тем не менее в тот день женщина повела себя, как истинный тси.
Восемьдесят десять процентов переселенцев проголосовали за информирование Безымянного о наших намерениях.

Культурологи приготовили для бога послание.
Вот оно целиком:

<<Тси спрашивают у демиурга Безымянного разрешение на постройку планетарной системы защиты против демонов, которая подразумевает уничтожение любого демона, приблизившегося к планете ближе указанного расстояния.
Тси вынуждены идти на этот шаг из-за угрозы со стороны Ордена Преисподней и Красного Картеля, которые весьма негуманными способами используют сапиентов как источник масс-энергии и как интерфейс для взаимодействия со Вселенной Фотона.
Мы благодарны Безымянному за спасение и потому обязуемся добавить его в исключения срабатывания системы.
Также мы желаем оговорить условия, которые компенсируют демиургу этот риск>>.

Ответ пришёл спустя час.
Я, как и многие мои товарищи, пребываю в лёгком шоке, поэтому просто процитирую перевод:

\begin{quote}
<<Я догадывался о ваших намерениях, тси.
Рад, что вы решили идти по пути укрепления сотрудничества.
Желание защититься свойственно и мне, а Орден Преисподней и Красный Картель одинаково опасны для всех обитателей моей планеты.
Вот мои условия: вы предоставляете мне ежедневный подробный отчёт об использованных для строительства ресурсах планеты, а также информацию о технологиях тси на мой выбор, исключая военные и технологии манипуляции сапиентами.
Взамен я обязуюсь предоставить вам полную информацию о ресурсах планеты и обеспечивать строительству необходимую защиту.
Жизнь за жизнь, знание за знание "--- думаю, это равноценный обмен>>.
\end{quote}

"--*Лучше бы Заяц ему написала, "--- буркнул Фонтанчик.
"--- С ней он разговаривал дружелюбно и искренне, как и она с ним.
Едва культурологи взяли настороженно-дипломатичный тон "--- его словно подменили.

Впрочем, Фонтанчика в его скептицизме никто не поддержал.
Как ни крути, ответ был недвусмысленным согласием.
Переводивший сообщение Баночка до сих пор сидит, пьёт чай со стимуляторами и думает.
Волнуется, правильно ли перевёл, не упустил ли какого-нибудь нюанса.

Мы очень надеемся, что маленький плант всё сделал правильно.
Да, кажется, у нас появилась надежда.
Но я всё ещё боюсь в это верить.

\section{Истина}

Вскоре совсем рассвело.
Хитрам поднялся из-за стола и, взяв с собой несколько кусочков мяса, пошёл работать в огород.
Близняшки, весело переговариваясь и лукаво постреливая глазами в сторону гостьи, убежали в школу.
Кормилица попросила Эрхэ убрать посуду и жестом позвала нас с Чханэ к себе в комнату.

"--*Хай, дети, "--- деловито сказала она, убедившись, что мы остались втроём.
"--- Если вы думали, что можете меня провести "--- вы думали так зря.
Чханэ, "--- кормилица, не дав девушке вставить слово, неожиданно заговорила с ней характерным западным говором, "--- мне совершенно не важно, по какой причине ты сбежала из родного города.
Я никогда не приветствовала убийство кутрапов и считаю, что любому человеку, даже дважды и десять раз дважды кутрапу, однажды может потребоваться шанс.
Лис уже видит тебя своей женщиной, и если это не достойно того самого шанса, то пусть Безумный заберёт меня прямо сегодня на закате.

Чханэ смотрела на круглое, доброе лицо Кхотлам.
Впервые я увидел в её глазах слёзы.
Губы девушки дрожали.

"--*Я\ldotst я убила жреца, "--- проговорила она, и слёзы потекли ручьём.
"--- Он хотел\ldotst хотел\ldotst

Мы с кормилицей обняли девушку.
Чханэ била настолько крупная дрожь, что мы втроём качались, как деревья под шквальным ветром.

"--*Да-да, понимаю, "--- отстранённо прошептала Кхотлам.
"--- Закон велит спрашивать у жертв разрешения, но кто по доброй воле согласится на мучительную смерть?
Пустая формальность\ldotst

Вскоре Чханэ успокоилась.
Кормилица усадила её на кровать и укутала тонким одеяльцем;
я сидел рядом и гладил девушку по спутанным, кофейным с оранжевой искоркой волосам, время от времени со странным щемящим чувством запуская в них пальцы.
Словно я уже делал это раньше "--- давным-давно.

"--*Вот что, милая.
Не вздумай рассказывать эту байку про караван кому бы то ни было.
Служанок моих ты, может, и обманешь, но в Храме тебе не поверят.
Я сама придумаю, что сказать.
А пока "--- иди-ка вымойся, ты солёная, как акула морская.
Я велю Эрхэ принести тебе воды, а одежду сошью сама.
Захочешь спать "--- ложись здесь, на мою лежанку.

Чханэ громко всхлипнула.
Заплаканные огненные глаза смотрели то на меня, то на кормилицу.
Во взгляде девушки слабым, робким пламенем светились благодарность и надежда.

\chapter{Старый дом}

\section{Храм Тхитрона}

\epigraph
{Спартанский образ жизни привлекателен для многих, особенно для слабых духом.
Лишая себя благ, люди наивно полагают, что готовы ко всему "--- к нашествию врагов, к катаклизмам и иным невзгодам судьбы.
Но возникает вопрос "--- в чём смысл лаконичности, если не будет бед?}
{Михаил Кохани, презентация <<спичечных>> технологий в Массачусетском технологическом университете}

Война.
Этому простому понятию в наши дни подчинено всё.
Инструмент может быть оружием.
Транспорт может быть щитом.
Человек может быть воином.
Всё и вся, проигнорировавшие этот закон, давно ушли в землю.

За четыре рассвета Чханэ пришла в себя.
Вскоре девушка попыталась взять на себя работу по дому.
Кормилица отдала ей узелковое плетение, которым занимала руки сама;
дело оказалось отличным снотворным "--- Чханэ, поковырявшись с верёвочками едва ли кхамит, проспала ещё полдня сверх отмеренного природой.
Будить её не стали.

Домашние редко когда знали, что пишет кормилица в <<особых>> письмах, предназначавшихся строго определённым людям.
За четыре дня в город ушло пять таких писем;
я знал, что они касались Чханэ.

"--*Я готова, Ликхмас.

Чханэ встретила меня немного высокомерно, но уже вполне по-дружески.
Я мимоходом восхитился Кхотлам "--- новая одежда Чханэ была в меру потрёпанной и несла те неуловимые следы жизни на Западе, которые может подметить лишь опытный дипломат.
Вот шов на плече;
портные Запада прошивали рубахи особым швом.
Даже проба на коже ножен гласила, что оружие было сделано пять дождей назад в городе Тхаммитр.
Зелёное платье кормилице вернули поздней ночью "--- без следа крови.

"--*Хорошая, "--- признала Чханэ, выдернув фалангу из ножен.
"--- Я назову её Преградой.

"--*Это же сталь, а не кукхватр, "--- усмехнулся я.
"--- Зачем давать имя стальному клинку?

"--*Он тоже имеет на это право!

"--*Все имеют право на имя, "--- поддержала её Кхотлам, поцеловав нас по очереди.
"--- Погуляйте по городу, пока день не вступит в свои права.
И пожалуйста, милая, постарайся идти свободно.
Ты честная воительница, прибывшая в Тхитрон и решившая погулять, верно?

Я протянул Чханэ руку, и мы вышли в сонную утреннюю мглу Тхитрона, в которой отчётливо виднелась громада храма.

Здание храма выполняло сразу несколько функций.
Нижний этаж представлял собой казармы для воинов, центр координации военных сил и арсенал.
На верхнем этаже располагались кельи жрецов, библиотека и госпиталь с одним или двумя операционными балконами.
Крыша храма предназначалась для жертвоприношений "--- там была крипта с необходимыми инструментами, постоянный дежурный пост, предупреждающий о знаке кирпича, а также алтарь.
На храмовых землях, но вне основного здания, обычно устраивали такие важные места, как школа и тренировочная площадка для детей.

У северных городов была особенность, которая выгодно отличала их от демонстративно-милитаризованного Юга и лаконичной воинской выправки Запада.
Тхитрон находился на стыке земель хака и Живодёрских лесов, но при этом казался мирным, разнеженным и незащищённым местом.
Люди ходили без оружия, храмовая пирамида была больше похожа на уютный дворец, нежели на последний рубеж обороны города.
Ворота всегда, даже ночью, стояли открытыми;
иногда я выходил погулять к реке, но ни разу не встретил ночную стражу и в принципе имел слабое представление, где именно у ворот находится пост.
Широкие улицы и симпатичные жилища не несли на себе ни одного видимого отпечатка военной инженерии.

Чханэ намётанным глазом отметила эти особенности.

"--*Живут же люди, "--- тихо вздохнула она.
"--- Не там я родилась\ldotst

Впрочем, враги знали, насколько это не соответствовало истине.
В случае вторжения город по взмаху ладони превращался в очень неприятную крепость "--- с кучей бойниц, военных орудий и маленьких сюрпризов вроде зажигательных снарядов, самострелов и появляющихся из ниоткуда цепей с ядовитыми шипами.
Каждый квартал имел форму соединённых друг с другом колец;
каждое кольцо имело свой защищённый подземный ход.
С высоты птичьего полёта город выглядел пёстрым одеялом, игрушкой случайности;
подземные ходы чётко делили его на восемь рубежей обороны.
Главные улицы перекрывались повозками, в мирное время выполнявшими функции торговых точек.
В укромных нишах повозок всегда имелись многозарядный арбалет и пара горшочков с маслом.
На крышах поднимались замаскированные щиты и мостики, позволявшие быстро превратить жилища в крепостные башни и так же быстро отступить.

Безоружные жители также были иллюзией.
Оружие у всех лежало дома, и старший воин по кварталу раз в четыре декады по-дружески заглядывал в каждое жилище "--- выпить отвара, поговорить, проверить состояние вооружения, а порой и навыки старых и молодых.
Патрули никогда не надевали доспехов и брали с собой минимум оружия, из-за чего их вполне можно было принять за прогуливающихся горожан.
Впрочем, знаки различия и не требовались "--- воинов было всего шестнадцать, и местные знали каждого в лицо.

За всю многотысячелетнюю историю города, пока здесь жили люди, Тхитрон пал всего один раз, и то по причине чудовищной тактической ошибки.
Это было пятьсот дождей назад, во время знаменитого Нашествия Змей.
На памяти старожилов ни один враг не прошёл дальше второй оборонительной линии.
Осада также была бесполезна "--- находившиеся в черте города поля могли обеспечивать население пищей круглый год.
Как выразилась один раз кормилица, <<в случае осады мы скорее умрём от скуки, нежели от голода>>.
Поэтому Тхитрон считался спокойным местом;
несмотря на некоторую удалённость города от основных торговых путей, здесь охотно селились выходцы со всех концов земель сели.

Да, всё в наши дни подчинено закону войны.
Но кто сказал, что это должно мешать мирной жизни?

"--*Хай, чувствую себя девочкой, которую привели в Храм, "--- усмехнулась Чханэ, когда прогулка подошла к концу и пирамида заполнила собой треть небес.
"--- Знаешь, когда в самый первый раз\ldotst

Я улыбнулся.
Разумеется, я знал.

\section{Так начался путь}

Улыбка человека, переживающего лучшие моменты жизни, не сходила с лица Атриса.

Дождь лил уже пятый день.
Казалось бы, какие путешествия в такую погоду?
Однако караванщики придерживались другого мнения.
В сезон дождей падала цена на местные товары и подскакивала на привозные;
мало кто из торговцев, тем более торговцев-ноа, упускал такой шанс.

Улыбки не сходили и с лиц наёмников.
Митхэ старалась не обращать на них внимания, думая, что люди по-доброму смеются над ней и её глупым дорожным романом.
Ей и в голову не приходило, насколько это была юношеская, эгоцентричная мысль.
Она лишь пониже склонялась к шее своего белого оленя, гладила его и поправляла подкладку узорчатого шанфрона, чтобы металл не натирал нежную плоть у рогов.
Серебряный живо реагировал на эту непривычную заботливость "--- мотал головой, вращал золотистым весёлым глазом, взбивал копытцами дорожную грязь и благодарно жевал хозяйке рукав.

Но Митхэ ошибалась.
Просто для большинства воинов это был первый поход, в котором каждый рассвет сопровождался музыкой.
Никто не обращал внимания ни на дождь, ни на вечно угрюмые лица торговцев, вцепившихся в чётки Сата и методично отсчитывавших кхене.
Все слушали Атриса.
Его музыка не была навязчивой, она начиналась и затихала тогда, когда следовало.
Старая цитра рассказывала обо всём, что встречалось им на пути "--- о тысячелетних деревьях акхкатрас и пугливых молодых гевеях, о скромных ручьях и общительных водопадах, о квакшах и древесных саламандрах, которые пели, прыгали и извивались в этом царстве воздуха и влаги.
Время от времени из сундука на звук цитры выглядывал небольшой серый кот и тут же прятался обратно.

С котом влюблённые познакомились в захолустном постоялом дворе Тхартхаахитра.
Атрис не подозревал, что там живут и вообще могут жить беспризорные животные, пока к ним в комнату не зашёл он.

Кот выглядел на редкость неухоженным.
Его с малолетства и подкармливали, и били.
Били, конечно, чаще.
Этот кот никогда не таскал без спросу еду, как это делали прочие "--- палки и сапоги наёмников приучили к осторожности.

Кот смиренным взглядом смотрел на менестреля.
Чем-то он напоминал самого Атриса "--- такой же несчастный, одинокий и уже смирившийся со своей участью.
Увидев в руках бродяги еду, он не начал мяукать, лишь посмотрел искоса и вытянулся в струнку.

Атрис поглядел на бутерброд с курицей.
От голода сосало под ложечкой, но\ldotst что изменит маленький кусок курицы для него, человека?
А для кота это целый обед.

Животное аккуратно взяло курицу в зубы, отнесло чуть дальше и приступило к трапезе.
Именно к трапезе "--- по-другому назвать происходящее не поворачивался язык.
Атриса снова поразила эта почти человеческая обречённость.
Кот не урчал, прижимая уши, не рвал добычу зубами и когтями.
Он ел спокойно и размеренно, зная "--- если человек захочет отнять его пищу, он это сделает.

Так в <<доме>> Митхэ и Атриса поселилось ещё одно живое существо.
Первое время Цапка нервничал и не желал вылезать из сундука, где ему устроили лежанку, однако некоторое время спустя его перестали смущать даже сражения.
Однажды кот предотвратил стычку с идолами, пожелавшими снять сливки с плетущегося под дождём каравана;
он невозмутимо прошёл перед строем наёмников с по-хозяйски поднятым хвостом, затем презрительно обнюхал нескольких врагов и поточил когти об их щиты.
Суеверные аборигены сердить <<духа>> не решились и тихо удалились, а кот приобрёл в отряде Митхэ неоспоримый авторитет "--- отныне его величали не иначе, как Цапка-лехэ.

Порой Серебряный целыми днями скучал без своей наездницы "--- Митхэ перебиралась в обоз к Атрису и подолгу спала в его объятиях.
Воины поначалу роптали;
однако Атрис вдруг добавил в тихий голос уверенности и сказал, что дорога чиста.
Как по волшебству, отряд избежал казавшихся неминуемыми встреч с идолами и тенку "--- бескровная победа Цапки-лехэ оказалась единственным сражением на всём пути к
Кристаллу.
Ропот прекратился.

"--*Знаешь, Золото, "--- задумчиво сказала Эрхэ во время очередного привала, "--- всё-таки предки были мудры, когда придумали Храм.
Я долго гадала, чего же не хватало нам все эти дожди.
Мы "--- Нижний этаж, Золото.
Нижний этаж не может существовать без Верхнего.

"--*Ты о чём? "--- не поняла Митхэ.

"--*О твоём бродяге, "--- пояснила Эрхэ.
"--- Есть в нём что-то жреческое.
Ты заметила?
Даже народ как-то ссориться меньше стал.
Потому что теперь мы "--- настоящий бродячий Храм, а не кучка наёмников.
С маленьким таким Верхним этажом из одного человечка.

"--*Он\ldotst мне порой кажется, что он не с Верхнего этажа, а гораздо выше, "--- Митхэ вдруг решила поделиться с подругой мыслью, которую долго не решалась высказать вслух.

"--*Выше Верхнего этажа? "--- улыбнулась Эрхэ и задумалась.
"--- Кто у нас выше Верхнего?
Разве что боги.

\section{Посуда богов}

Первое моё сознательное знакомство с Верхним этажом произошло так же, как и у всех прочих "--- в детстве.

Однажды в наш дом пришли три жреца.
Дверь открыла кормилица и, увидев красные ленты на их лбах, попятилась от двери.
Её губы дрожали.

Старший жрец обвёл зал глазами и, увидев меня, подошёл ближе.

"--*Послушайте, "--- начала Кхотлам.
Жрец остановил её, властно подняв руку.

"--*Пусть на твои уста падёт молчание, Кхотлам, "--- вполголоса промолвил он.
"--- Ты знаешь обычай.

Из комнат вышли домашние и замерли как вкопанные.
Все смотрели на меня.

"--*Ликхмас ар’Люм, "--- поклонился жрец.
"--- Мы пришли, чтобы спросить твоё согласие.
Боги, увидев твои добродетели, захотели забрать тебя в свою обитель.

Я молчал, всем нутром чувствуя повисшее в воздухе напряжение.

"--*Почему боги хотят забрать меня? "--- наконец выдавил я.

"--*Боги забирают лучших детей, Ликхмас ар’Люм, "--- сказал второй жрец.
"--- Ты сможешь послужить им, но они могут принять твою службу лишь с твоего согласия.

Домашние, как по команде, вдруг облегчённо вздохнули и начали заниматься своими обычными делами.
На меня никто не смотрел.
Этот диссонанс привёл меня в ещё большее смятение.
Ведь я ещё не дал ответа!

"--*Что я должен буду делать?

"--*У богов свои замыслы, "--- сказал жрец.
"--- Тебя наделят великой силой, неведомой человеку, и твои задания будут под стать этой силе.

"--*Где я буду жить? "--- продолжал допытываться я.

"--*На вершине высочайшей из гор, Ликхмас ар’Люм, "--- терпеливо ответил старший.
"--- Рыбья Флейта.
С этого пика весь мир виден, как на ладони.
Боги щедро награждают тех, кто им служит "--- им даруют бессмертие.

"--*Ты лжёшь, "--- буркнул я.
"--- Если бы мне даровали бессмертие, кормилица бы так не волновалась.

Кхотлам не обратила на мои слова внимания.
Она успела вытащить какие-то бумаги и погрузилась в чтение.

"--*Твоя кормилица просто печалится, что она будет реже тебя видеть, "--- глухим голосом объяснил третий жрец.
"--- Это нормально для любого человека, воспитавшего дитя.

"--*Я отказываюсь, "--- сказал я.

Жрецы переглянулись.

"--*Хорошо ли ты подумал? "--- проникновенно прошептал старший.
"--- Отправиться к богам "--- величайшая честь, которая выпадает детям сели.

"--*Если я отправлюсь к богам, кто будет мыть посуду и отделять хорошие помидоры от гнилых? "--- задал я мучивший меня вопрос.
Эти обязанности целиком лежали на мне.

Жрецы выпучили глаза и задумались.

"--*Я умею мыть посуду, пропалывать ростки, крутить мельницу, сортировать овощи, рисовать цифры на мешках, точить ножи, жарить на огне картошку и мелкую дичь, снимать с дерева кору и лепить из глины, "--- гордо перечислил я, загибая и разгибая пальцы в двоичном счёте, как учила Кхотлам.
"--- Для всего этого нужны лишь руки.
Даже имей я великую силу, ничего другого я делать не могу.

Жест, который получился в процессе счёта, <<голова койота>>, у сели означал <<Убирайся или дерись>> "--- ультиматум, который исключал возможность переговоров.
Его было очень легко показать рукой, держащей натянутый лук или эфес фаланги, и тем самым дать противнику возможность сбежать.
Число же обычно показывали, держа пальцы кверху и оставляя вторую ладонь открытой.
Но я, чистый и светлый ребёнок, по незнанию ткнул получившуюся фигуру прямо в живот старшему.
Жрец остолбенел и отступил на шаг.

"--*Ликхмас ар’Люм, что это значит?

"--*Людям я пригожусь больше, "--- объяснил я, не поняв, к чему относился вопрос.
"--- Если рассказы о всесилии богов правдивы, пусть боги сами моют свою посуду.

На меня смотрели все, кто был в зале.
Пять пар изумлённых глаз.

Жрецы переглянулись.
Старший медленно снял с головы алую ленту.

"--*Да ну его в лес, "--- пробормотал он и направился к двери.

Оставшиеся двое низко поклонились мне и вышли следом, не сказав ни слова.

Разумеется, кормилице потом пришлось объяснять Храму, что она ничего не говорила мне про Отбор и что ребёнок никого не собирался бить.
Через несколько дождей жрецы пришли за моими будущими сёстрами, и мне тоже пришлось сделать вид, что всё в порядке "--- таков обычай.
Впрочем, потомки Кхотлам ар'Люм с честью выдержали традиционное испытание на доверие "--- все остались в живых.

\section{Нгвсо}

Мы не просто остались в живых.
Мы процветаем.
Эта мысль бродила во мне, словно дрожжи в агаровой среде, пока я смотрел из иллюминатора корабля на берег, ещё пять дней назад пустынный и неприютный.

Тси выстроили вокруг Стального Дракона целый городок.
Вначале поле запестрело палатками, но у самого моря начали появляться и каменные жилища.
Все справедливо решили, что не стоит тратить ресурсы на строительные полимеры, и использовали местные породы;
особо ленивые просто вырезали себе в скале уютные пещерки, заодно поставив прочим прорву кирпичей.

Глядя на каменные постройки в бинокль, я почувствовал прилив гордости "--- многие из этих зданий в других мирах сошли бы за произведения зодческого искусства.
Вот домик биолога, Сок-Стального-Листа.
Она потратила целый день, чтобы покрыть треугольную крышу белой черепицей и вставить в стрельчатые окна цветные витражи "--- самая сложная часть работы.
Отдельные кирпичики даже украшала глазурь, на которую пошло оставшееся цветное стекло.
Сейчас в её доме жили химики;
сама Листик целыми днями пропадала с Маком "--- брала пробы воды, почвы и воздуха.
При этом у неё вряд ли возникало чувство, что что-то не так.
Я знал, что она ответила бы на вопрос о труде над своим жилищем: <<Мы здесь надолго>>.
Так ответил бы любой тси.

Само море внезапно преподнесло сюрприз "--- местных сапиентов, которые называют себя <<нгвсо>>.
Безымянный привёл к нам целую делегацию этих странных существ, похожих на спрутов с тремя щупальцами.
Мы слышали, что очень часто боги создавали собственных сапиентов, чтобы обеспечить себя потоком масс-энергии;
по классификации Ордена Преисподней эти существа относились к так называемым Девиантным ветвям.
Однако столкнуться с этим явлением тси пришлось впервые.

Внятного диалога, к сожалению, не вышло "--- уровень развития нгвсо по шкале Яо немногим более ста баллов, они едва-едва научились обрабатывать камень и металлы.
Мак и Листик рвались их изучить, но Фонтанчик втолковал биологам, что это, как ни крути, разумные существа, и брать пробы без их ведома как минимум невежливо.
Безымянный объяснил спрутам через их религиозного лидера, что мы "--- друзья;
нгвсо подарили нам симпатичные украшения из ракушек, мы отдали им игрушки, яркие детальки и неисправный гирокомпас, и счастливые аборигены уплыли восвояси.

Удивительно, но эта встреча значительно улучшила отношение Безымянного к тси, охладевшее после дипломатических переговоров.
Я уже начал сомневаться "--- неужели и хоргету доступна такая вещь, как привязанность?

\begin{quote}
<<87\% народа нгвсо живёт вдоль побережий моей планеты.
Они меня любят и делают мне хорошо>>.
\end{quote}

Всем известен способ, которым сапиенты делают хоргетам <<хорошо>>;
питание масс-энергией "--- причина эксплуатации и угнетения хоргетами сапиентов.
Но мы вежливо промолчали.
Нельзя объяснить негуманность процесса существу, которое живёт за счёт этого и только этого процесса "--- это как объяснять волку, что убивать кроликов плохо.

Заяц умело выкрутилась из неловкой ситуации:

<<Безымянный, а среди тси тоже есть водные формы жизни.
Они называются дельфины>>.

\begin{quote}
<<Сотканный-Темнотой-Заяц, объясни, почему я не чувствую их среди вас>>.
\end{quote}

<<Они пока что в анабиозе на корабле.
Мы выпустим их, как только изучим водную микрофлору>>, "--- объяснила Кошка.

\begin{quote}
<<Гладит-Зелёную-Кошку, я приму новую водную форму жизни и дам ей все необходимые ресурсы, если она не будет проявлять агрессию по отношению к нгвсо>>.
\end{quote}

Здесь нужно заметить: к концу третьего дня Безымянный уже знал всех тси по именам, а кое-кого отличал по стилю сообщения.
Мы уже не удивлялись "--- всё-таки интеллект есть интеллект.
По оценке Кошки, вычислительные мощности Безымянного лишь на пару порядков были ниже таковых у Машины.

<<Они не агрессивные, "--- успокоила демиурга Кошка.
"--- И живут в основном в открытом океане, а не у побережья>>.

"--*Везёт плавучим, "--- с завистью сказал Мак.
"--- Если нгвсо действительно живут только у побережья, то к услугам дельфинов полный еды океан, да ещё и Безымянный им ресурсы какие-то даст.
А нам ножками шевелить "--- то скалы, то пустыни, то леса, только успевай приспосабливаться\ldotst
Может, хвост отрастить?

"--*У меня есть, а толку-то, "--- посетовал Фонтанчик.

\section{Уважение, справедливость и мастерство}

\epigraph
{Оружие "--- как соска для взрослых.
Некоторых просто нельзя успокоить и заставить уважать других, если у этих других нет за поясом огнестрела.
Однако через несколько поколений, когда взаимоуважение войдёт у людей в привычку, оружие можно спрятать в ящик и забыть.}
{Шутка Бенедикта Альсауда, ставшая десятым постулатом Элект}

Вскоре после того, как у меня проявились мужские признаки, начались воинские упражнения.
Так в мою жизнь вошёл Нижний этаж.
Тренировал детей в то время пожилой, но удивительно красивый и улыбчивый для своих лет воин, носивший имя Сладкая-Ягодная-Конфетка.
Вообще учителей полагалось звать по имени царрокх, но тренер ненавидел имя Эрликх и просил всех обращаться к нему по домашнему имени "--- Конфетка.

Конфетка вполне оправдывал своё имя. Точное число женщин и мужчин, которые его <<попробовали>>, неизвестно, как и число его потомков.
Впрочем, подопечных он очень любил и, несмотря на периодические любовные и алкогольные запои, считался лучшим тренером в Тхитроне.
Я ни разу не слышал, чтобы он на кого-то кричал;
тем не менее воспитанники Конфетки не уступали воспитанникам прочих, не разделявших его <<миролюбивые>> методы.

Первые тренировки были подготовкой.
Нас учили правильно дышать, бегать, прыгать, плавать, падать, лазать по деревьям и скалам.
Я ничем не выделялся, пока дело не дошло до болевой подготовки.

Конфетка поставил меня в паре с маленьким, хрупким мальчиком.
Я был не самым крупным среди детей, но возвышался над партнёром, как слон над поросёнком.

"--*Бей, "--- сказал Конфетка моему сопернику.

Мальчик неуклюже ударил меня в грудь.
Я поморщился "--- это было даже не больно.

"--*Теперь ты, Ликхмас.

"--*Нет, "--- сказал я.

Конфетка удивлённо посмотрел на меня.

"--*Если я ударю, он может пострадать, "--- объяснил я.

Конфетка поколебался.
Тренер должен был дать задание другим детям, но решил для начала разобраться со мной.

"--*Ликхмас, "--- начал он.
"--- Это называется болевой подготовкой.
Да, сейчас ему будет больно.
Но настоящий удар его тело встретит подготовленным.

Как я уже говорил, пожилой воин предпочитал объяснять, а не ругаться.

Я посмотрел на мальчика.
Тот с трепетом ждал моего удара.

"--*Нет, "--- сказал я. "--- Дай мне соперника сильнее.

Конфетка растерялся.
По-видимому, он слышал такое впервые.
Подумав, тренер обернулся и крикнул:

\mulang{$0$}
{"--*Ликхэ, иди сюда.}
{``L\^\i kcho\^{e}, come here.''}

К нам подошла девушка чуть постарше, которую уже готовили в Храме.
Глаза под высоким лбом высокомерно скользнули по мне.

"--*Покажи мальчикам, как проходит болевая подготовка.

Ликхэ наотмашь хлестнула хрупкому мальчишке по шее.
Тот покатился кубарем и заплакал.
Затем девушка ударила и меня.
Я упал на спину, и Ликхэ, не дав мне опомниться, начала мутузить руками и ногами по самым чувствительным местам.

В первые мгновения я растерялся.
Мир превратился в клубок из кулаков и сапог, каждый из которых хотел причинить мне боль.
Ликхэ яростно разрывала мои руки, когда я закрывал ими тело в жалкой попытке защититься.
А потом всё вдруг прекратилось.
Я не помню, как это вышло, но у меня в ладонях оказалась лежавшая неподалёку тренировочная трость.
Девушка, получив крепкий удар в грудь, потеряла равновесие и упала на вытянутую руку.

Рука хрустнула, и одновременно с этим хрустом испуганно выругался Конфетка.
Ликхэ осела на землю ничком и тоненько заплакала от острой боли.
Её правая кисть неестественно изогнулась и повисла, запястье стало медленно отекать.
Обиженный мальчишка, придя в себя, с воинственным плачем бросился на девушку и начал её колотить.
Глаза на испачканном землёй, заплаканном личике горели жаждой мести, и тренер едва успел оттащить его.

С тех пор Ликхэ ар'Трукх правую руку использует только под щит.
И при встрече смотрит на меня долгим печальным взглядом.

Так я узнал, что люди уважают только силу.

\razd

Конфетка отвёл нас с Ликхэ в храм.
Я шёл понуро и не смотрел на тренера, ожидая взбучки.

"--*Что такое? "--- встретил нас удивлённым вопросом жрец, которому суждено было стать моим любимым учителем.

"--*Перелом лучевой кости в типичном месте, Трукхвал, "--- Конфетка кивнул на Ликхэ.
"--- Позови Кхатрима.

"--*Он сейчас в городе, занят.
Перелом закрытый?
Отлично.
Пойдём, солнышко.
Я тебя подлечу\ldotst

Трукхвал повёл нас на хирургический балкон.
Там жрец внимательно осмотрел и ощупал кисть Ликхэ.
Потом, оглянувшись на меня и Конфетку, Трукхвал снял половую плиту и достал из-под неё странный инструмент наподобие шарика на стержне.
Вид инструмента вызвал у меня смутную, не совсем понятную тревогу;
я взглянул на Ликхэ "--- она смотрела с нескрываемым ужасом.
Конфетка вдруг тоже выпучил глаза и сглотнул:

"--*Звоночек, откуда у тебя это?

"--*Из страны пернатых обезьян, "--- отшутился жрец.
По его высокому лбу катился пот.
"--- А вообще советую не распускать языки, я ради вас стараюсь.

Трукхвал накрыл свою голову и руку Ликхэ тёмным покрывалом и, как мне показалось, поводил инструментом туда-сюда.
Затем, пробормотав <<Всё ясно>>, спрятал таинственный предмет обратно.
Все, как по команде, выдохнули;
гнетущее чувство пропало без следа.
Трукхвал аккуратно усадил больную, положил сломанную руку ладонью вверх и велел Конфетке тянуть за плечо.

Ликхэ заорала дурноматом, но Трукхвал, ловко потянув за пальцы, успел поставить кисть на место.
Конфетка помог Трукхвалу смазать руку травяными настоями, забинтовать и наложить гипс.

"--*А ты, Ликхмас, утешай.

Я, не глядя на Ликхэ, сидел рядом и гладил её по голове.
Ликхэ куксилась и тоже смотрела куда-то в сторону, но не отодвигалась.

"--*Ну что? "--- вполголоса спросил Конфетка.

"--*На просвет "--- пара осколков, "--- так же тихо ответил Трукхвал.
"--- Я их прижал, конечно, как мог, но\ldotst
Мне отёк не нравится "--- как будто крупный сосуд повреждён.
Покажи её Кхатриму чуть попозже.
И пожалуйста, скажи, что осколки я нащупал пальцами, а не\ldotst сам понимаешь, мне такой разнос устроят, что я её в храм притащил\ldotst

"--*Надеюсь, она не\ldotst

"--*Нет-нет, "--- поспешил заверить Трукхвал.
"--- В свинце и в кадушке с холодной водой, денно и нощно.
Да и кусочек там с мушиный глаз\ldotst

Учитель Трукхвал остался тем же "--- по секрету всему свету.
Каждый в городе знал кусочек его тайн, но целую картину собрать мог разве что я, его ученик "--- все добросовестно молчали, зная, что многие из его тайн полезны, но не совсем законны.

Потом меня отвели в большой зал, на собрание жрецов.
Я стушевался окончательно.

"--*Как зовут? "--- спросил один из жрецов, самый старый.

"--*Ликхмас ар’Люм, "--- пробурчал я.
Ощущение, что я провинился по-крупному, стало настолько сильным, что у меня дрожали колени.
Я слышал, как люди говорили о Насилии и Разрушении "--- эти преступления иногда карались изгнанием.

"--*Ликхмас? "--- засмеялся жрец.
"--- Хай, его дарители точно весёлые люди.
Кстати, кто дарители?

Сидевший рядом воин наклонился к его уху и что-то прошептал.

"--*Даже так? "--- удивился жрец.
"--- Так в воины его, и дело с концом, такая кровь пропадает.

"--*Да вы посмотрите на стойку, "--- поддержал его Конфетка.
"--- Понятия не имею, где он этого нахватался, но если бы у него в руках была хоть щепка, я бы бежал без оглядки.

"--*Дарительница попросила не делать его воином, "--- укоризненно заметил кто-то.
"--- И ты это прекрасно знаешь, Эрликх.

"--*Сама ты Эрликх, "--- тихо выругался Конфетка.

"--*<<В мире, полном боли до краёв>>\footnote
{Легенда об обретении. \authornote},
"--- вздохнул Первый жрец.
"--- Трааа\ldotst хорошо, \emph{её} мнение можно и учесть.
Итак, Ликхмас ар’Кахр\ldotst

"--*Ар’Люм, "--- поправил я его.
Тот поднял брови.

"--*Ликхмас ар’Люм.
Почему ты не ударил мальчика в тренировочном поединке?

"--*Он слабее, я мог ему навредить, "--- тихо сказал я.

"--*Ещё раз, громче, "--- попросил Трукхвал.

"--*Я не хотел ему навредить, "--- повысил я голос и с вызовом оглядел зал.

Кое-кто засмеялся.

"--*Ситрис, это случайно не ребёнок \emph{этой}, ну, который ещё во время Отбора велел богам самим мыть посуду? "--- заинтересованно спросила улыбчивая зеленоглазая воительница.
Приглядевшись, я понял, что она только казалась улыбчивой "--- к левому уголку рта спускался жутковатый извилистый шрам.

"--*Да, Кхохо, "--- откликнулся со скрытого в тени края стола тёплый баритон.
"--- Над этой историей уже весь Север успел посмеяться.
Лотрам-лехэ, да будут духи к нему добры, в хмелю не командир был своему языку, всем выболтал.

"--*Трааа, "--- снова протянул старший.
"--- Хонхо-лехэ, он ещё в школу не ходит?

"--*Хай, "--- задумчиво сказал старый жрец и вперился в меня пристальным взглядом.
"--- Из дома Люм мальчишка?
Лицо незнакомое, но по возрасту "--- скоро должен.

"--*Посмотри, как у него со школой дело пойдёт.
А так "--- хороший кандидат в <<белые плащи>>.
Что скажете? "--- обратился он к залу.

"--*Да и в <<красные>> тоже.
Освоится, кровь не пропадает, "--- заметил воин в углу.

"--*Ликхмас ар’Люм, хочешь стать жрецом? "--- спросил меня старший.

"--*Как вы меня накажете? "--- бросил я встречный вопрос.

"--*Мы никогда не наказываем за желание защитить себя, "--- улыбнулся старик.
"--- Спрашиваю ещё раз "--- хочешь стать жрецом?

Я посмотрел на него исподлобья и кивнул.

"--*Отлично, "--- заключил он.
"--- Но бить соперников в поединках тебе всё-таки придётся.

"--*Дайте мне соперника сильнее, "--- упрямо пробормотал я.
Присутствующие рассмеялись.

"--*Заладил песню\ldotst

"--*Хаяй, кровь\ldotst

"--*Конфетка, с воинами третьего года его.
Будут тебе посильнее.

Так я узнал, что справедливости нужно не просить, а требовать.

\razd

После того, как мы вышли из зала, Конфетка смотрел на меня другими глазами.
Наконец, вдоволь намолчавшись, он подошёл ближе, повертел меня за плечи, критически осмотрел и, крякнув: <<Похож, однозначно похож>>, сунул мне в карман сладкую конфету.

На следующий день начались тренировки с воинами-подростками.
Меня отмутузили так, что я едва стоял на ногах.
Каждая мышца словно превратилась в болезненно звенящую струну.
Кормилица за ужином с беспокойством смотрела на трясущийся в моих руках черпачок, но я упрямо молчал, перекатывая в кармане пальцами второй по счёту гостинец.

Ещё через рассвет расцвели синяки.
Конфетка, осмотрев меня, вздохнул:

"--*Иди сюда, маленький ягуар, я намажу тебя мазью.

"--*Мне не больно, "--- сказал я.

"--*Боль нельзя терпеть, если ты можешь от неё избавиться, "--- наставительно сказал тренер.
"--- Слабая постоянная боль хуже тяжёлого удара "--- она способна обессилить даже Маликха.

Дальше "--- больше.
Когда выяснилось, что я неплохо иду вместе со всеми, Конфетка стал нагружать меня ещё сильнее.
Именно мне в качестве его любезности доставались постоянные удары бамбуковой палкой по пальцам, локтям и коленям.
Тогда я научился амбидекстрии "--- перо приходилось брать рукой, которая болела меньше.
Меня заставляли пробегать дополнительный кхене в гору и выдерживать дополнительный <<круг обороны>>.
Но каждый раз, как бы плохо я ни провёл тренировку, сколько бы ударов палкой мне ни досталось, в конце Конфетка ласково взлохмачивал мне волосы и клал в руку ставшую традицией сладость.

На пятый год меня ждало новое испытание.
Однажды, когда я кушал после утомительной тренировки, Конфетка подкрался сзади, ударил палкой по спине и убежал.
Я едва насмерть не подавился острой хлебной корочкой.

Нападения стали повторяться изо дня в день, даже вне времени тренировки.
Конфетка выслеживал меня везде, где только возможно.
Я не мог ни поесть, ни сходить в уборную без опасности быть атакованным.
Деревянную саблю пришлось носить постоянно.
Убедившись, что я уже начеку, Конфетка пустил в ход затуплённые лёгкие стрелы, камни и деревянные томагавки.
Тренер поджидал меня в кустах, за углами, стрелы и камни могли появиться из крон деревьев и из чужих дверных проёмов.
За каждую пропущенную стрелу я расплачивался не только синяком и лёгким уколом самолюбия, но и дополнительными ударами на следующей тренировке.
Предел моему терпению положил жестокий удар под дых, который я получил на пороге родного дома.

Тогда я лёг спать с твёрдым желанием отплатить.

С этого дня наши роли поменялись.
Я перестал ходить на тренировки.
Я поджидал Конфетку в кустах, за углами, иногда нападал на него сзади в базарной толпе.
Сначала он без труда замечал меня, но изо дня в день ему это удавалось всё реже.
Народ, привыкший к нашим играм, только посмеивался.

Однажды всё чуть не закончилось плачевно "--- я засветил тренеру в затылок булыжником размером с яйцо сиу-сиу.
Крепкий коренастый воин шлёпнулся молча, точно кожаный ремень.
В тот день я осознал, какая сила заложена в маленьком камне.

К счастью, Конфетка отлежался.
Все десять дней, пока он поправлялся, я приносил ему лучшие фрукты, которые только мог найти.
Воин-зубоскал похвалил меня за меткость и скрытность, но попросил, чтобы я больше никогда не бросал в него камни.

А однажды он умер.
Разумеется, никто не сказал этого вслух, но смерть Конфетки была донельзя глупой "--- слезая с крыши строящегося дома, он сорвался, и его нога попала в верёвочную петлю.
Голова проехала по булыжной мостовой, шея хрустнула, как тростинка "--- и мой улыбчивый любвеобильный друг отправился в пристанище лесных духов.
Я едва успел перед похоронами положить ему в карман причитающиеся сладости.

Так я узнал, что никакая подготовка не защитит полностью от случайностей.

\section{Хороший знак}

"--*Да, я согласен, что защититься от всех случайностей мы не можем, "--- сказал Фонтанчик.
"--- Но начинать строительство сейчас "--- значит расходовать материалы зря.
Дайте нам ещё немного времени, возможно, мы сможем найти способы сэкономить золото.

"--*Если вы не найдёте способ "--- вы зря израсходуете время, "--- сказал Хомяк.
"--- Я за начало строительства.

Геологи сегодня сообщили нам удручающую новость.
На этой планете не хватит золота, чтобы развернуть планетарную систему защиты против демонов.
Доступные ресурсы покрывают девяносто шесть процентов, а на энергию для ядерного синтеза не хватит даже оставшегося топлива Стального Дракона.

"--*Девяносто два, если считать подземные запасы, "--- подытожила Молодость-Поющая-О-Земле.
"--- Я бы не стала соваться в нижние слои коры до получения исчерпывающей информации о геотектонике.
Меня "--- и не только меня "--- смущают некоторые её аспекты.
Например, вот здесь, здесь и здесь наблюдается чрезвычайное напряжение океанической коры, которое просто обязано вызывать землетрясения, цунами и вулканическую деятельность.
Да что там говорить, нас даже здесь должно потряхивать.
Однако всё спокойно.
Что их держит "--- кто бы знал.

Хоргетов Ада и Картеля здесь пока нет.
Я склонен доверять Безымянному.
Однако, хоть Планета Трёх Материков и находится на задворках обитаемой Вселенной, демоны здесь уже бывали.
А там, где один, вскоре будут миллионы, и их не остановим ни мы, ни демиург.

Как и большинство тси, я вынужден был согласиться с Хомяком.
Время сейчас ценнее.
Фонтанчик в конце концов сдался и, подумав, занялся источниками энергии "--- Безымянный сообщил о значительных залежах урана на Короне, самом крупном из материков.
Не стоило оставлять без внимания и выжженные экваториальные районы "--- чего-чего, а солнечной энергии на Планете Трёх Материков хватало с избытком.

Первыми избытком солнечной энергии заинтересовались, разумеется, биологи.
Мак, разобравшись с приэкваториальной микрофлорой, решил сделать обычную теплицу.
Несмотря на то, что у каждого имелся персональный опреснитель, энергетически выгодной оказалась установка большого вакуумного насоса на берегу моря.
Пресную воду Мак направил к растениям.
Если честно, я немного содрогнулся, когда увидел, с какой скоростью они растут на дистилляте, слегка подсоленном морской водой.
После реализации программы двенадцатиэтапного севооборота мы просто не будем знать, куда девать биомассу.
И это с использованием самых примитивных технологий!
В итоге я велел Маку сбросить обороты и сократить площади теплиц в пять раз.
Он попросил оставить хотя бы половину "--- для опытных посадок растений с материка.

В остальном жизнь начала входить в привычное русло.
Недавно Пчела-Нюхающая-Вереск начала серьёзное исследование растений, моллюсков и ракообразных на предмет съедобности.
Заяц (по праву лучшей подруги и главного подопытного Вереск) с процентами отбила потерянные во время полёта несколько килограмм, о чём мне с восторгом сообщил Фонтанчик:

"--*Ты представляешь, я выхожу подышать, встречаю Заяц, обнимаю её "--- а у неё сзади-то ого-го подушка!

Хороший знак.
Определённо хороший знак.


\section{Свежая кровь}

"--*Рекомендации весьма неплохие, но на сегодня Тхитрон не испытывает острой необходимости в воинах.

Кхарас, боевой вождь, отличался практичностью.
Практичностью дышали его седые насупленные брови, его квадратный подбородок и серебристый ёжик волос на голове.
Ни татуировки, ни украшения не портили практичность его мощного, почти не тронутого возрастом тела.
Другого я от Кхараса и не ждал.
Впрочем, другого не ждала и кормилица, лично составившая рекомендацию для Чханэ.
Она знала: часто ставку нужно делать не на руководителя, а на нескольких ключевых людей его окружения.

"--*Кхарас, человек нам точно не помешает, "--- поморщился низенький бойкий Эрликх.
"--- Кхотлам права, нормальный штат для Храма Тхитрона всё-таки\ldotst

"--*Нормальный штат мы определим сами, без помощи купца, "--- перебил воина Кхарас.
"--- Ты, кажется, вполне справляешься?

"--*Я, с позволения сказать, маюсь от безделья, "--- признал Эрликх.
"--- Но я не могу закрывать глаза на то, что со времени смерти Конфетки в квартал плотников ходит гулять весь Нижний этаж.
По случаю.

"--*Квартал плотников\ldotst

"--*Кхарас, ну не будь рыбой, "--- устало сказал Ситрис.
"--- Зелёный уехал, Огранённый семью завёл, из молодых у нас только Ликхэ и осталась.
Это не дело "--- <<Пойду за помидорами, заодно пропатрулирую плотников>>.
Кхотлам права "--- кварталу нужен свой квартальный.
Хотя бы дай ей шанс.
Лежанку не протрёт и амбар не проест.

Кхарас насупился и скрестил руки на груди.

"--*Скажи, девочка, почему ты решила уехать так далеко?

Милая пожилая воительница Хитрам больше смахивала на скромную крестьянку, нежели на адепта Нижнего этажа.
<<Мальчиками>> и <<девочками>> у неё были все, не исключая Кхараса и Кхохо.

"--*Мне надоела военщина, "--- честно ответила Чханэ.

"--*Повод, "--- признал Ситрис.
"--- Мне тоже нравится северный уклад жизни.
Однако, если хочешь знать мнение южанина\ldotst

Ликхэ вынырнула из темноты, словно ягуар.
Чханэ небрежно сделала танцующий поворот вправо "--- и молодая воительница покатилась по полу, едва чиркнув ногтями по плечу подруги.

"--*\ldotst это не повод расслабляться, "--- закончил Ситрис.

Воины одобрительно переглянулись;
Эрликх помог Ликхэ встать на ноги.

"--*Похоже, ты в курсе, "--- признал Ситрис.
"--- Глаза на спине есть.

Глазами на спине был я.
К тому моменту, как ступни Чханэ коснулись каменных плит пола, она уже знала о леворукости молодой убийцы.
<<Скорее всего, сигнал подаст Ситрис, он любит такие штуки>>, "--- сообщил я перед самой дверью.

Ситрис внимательно пробуравил меня взглядом, но промолчал.
Его лицо приняло знакомое ироничное выражение.
<<И этот юнец думает, что он меня провёл.
Ладно, доиграем сценку до конца>>.

"--*Одежду долой, "--- бросил Кхарас.

Эта часть ритуала была обязательной.
Тело воина "--- это зеркало, отражающее его опыт, силы и слабости.

Чханэ стянула рубаху и сбросила штаны, открыв крепкое оранжево-чёрное тёло.
Кхарас и прочие молча смотрели.
Со второго этажа спустился Трукхвал.

"--*Хай, свежая кровь? "--- улыбнулся учитель и подмигнул мне.
"--- Левая стойка, поворот слабый, сильная привычка к закрытой спине "--- сладкий пирожок для убийцы.
Ликхэ, тебе развлечение.

Воительница прыснула.
Она уже успела вытащить из угла кресло и наблюдала за происходящим из него, закинув ногу на ногу.

"--*Иди уже, знаток, "--- хмуро сказал Кхарас.
"--- Без тебя разберёмся.

Трукхвал осклабился и бодро заковылял в подвал.

"--*Пара стрел, кхаагатр, копьё и целая прорва кинжалов.
Выносливая, но со скоростью беда, "--- резюмировал вождь.
На шрам от кхаагатра он смотрел, словно на живых людях такие можно найти сплошь и рядом.
"--- Боль во время сражения, я так понимаю, не чувствуешь?

"--*Если разогреюсь, то хоть на куски режь.

"--*Мало нам одной Кхохо было, "--- буркнул Кхарас.
"--- Сразу хочу сказать "--- к берсеркам у меня отношение предвзятое.
Я и сам такой, меня этому учили, и тебя, я так понимаю, тоже.
Но на Севере листья плюща в большем почёте, чем пустынный песок.
Хочешь остаться в Храме "--- научись в первую очередь работать головой.
Это ясно?

Чханэ кивнула.

"--*А кто тебе такой крест замечательный на лице оставил? "--- поинтересовалась Хитрам.

"--*Труп, "--- лаконично ответила Чханэ.
Кхохо одобрительно ухмыльнулась, глаза Хитрам погрустнели.

"--*Девочка, милая моя девочка.
Здесь нет нужды показывать свою силу.
Враг, тем более умелый, заслуживает уважения.

Чханэ смутилась.

"--*Это был разведчик пылероев из клана Кормящей Груди, "--- ответила она.
"--- Я не знаю его имени.
Моё оружие он сломал в поединке шутя и оставил меня в живых "--- то ли из жалости, то ли побрезговал.
Будь я постарше "--- дала бы уйти, но я тогда только пришла в Храм и шуток не понимала.
Шорники бросили мне лук, и я побежала по птичьим тропам\footnote
{Имеются в виду крыши и навесные мосты между жилищами. \authornote},
пока трое-четверо с квартала гнали разведчика по улицам.
Третьей стрелы он не пережил.

Ситрис вздохнул:

"--*Скорпион, да?

Чханэ улыбнулась и встала чуть боком, показав иссечённые шрамами бёдра и ягодицы.

"--*Вообще ноги не берегут, "--- фыркнула Кхохо.
"--- Закрывать эту школу надо к свиньям.
Эрликх, у нас есть доспехи Скорпиона?

"--*Лежали где-то старые, но я не уверен, что ей подойдут, "--- откликнулся воин.
"--- В коленном сочленении до сих пор стрела торчит.

"--*Вот-вот, этим они обычно и заканчивают, "--- поддержал Ситрис.
"--- Переучивать будем тебя, птица "--- красный шар\footnote
{Насмешливое название оцелотовых людей на Юге. \authornote}.
У нас на храмовом кладбище все места расписаны.
Ликхэ, тебе круглосуточная.
Трукхвал дело сказал насчёт привычки.
Вино пьёшь, дерёшься, ругаешься, деликатесы готовишь, женщины, мужчины, азартные игры?

"--*Всего понемногу, "--- смутилась Чханэ.

"--*Свой человек, "--- одобрительно крякнула Кхохо.
"--- Пошли уже жрать, я голодная.

"--*Можно подумать, ты не наелась, пока готовила, "--- хмуро сказал Эрликх.
"--- Продуктов всем выдаю поровну, а готовой еды у тебя получается в два раза меньше.

"--*Сегодня южное блюдо "--- согхо под перепревшим соусом из акхкатрас, золотая страница Книги-кормилицы\footnote
{<<Десять тысяч блюд>> или Книга-кормилица "--- труд последних тси о способах приготовления животных и растений Тра-Ренкхаля. \authornote},
"--- сообщил Ситрис Чханэ.
"--- Пахнет "--- умереть можно.
Кстати, Кхохо, всех выживших посетителей я сложил в кладовке.
Иди сама приводи их в чувство.

Воины, переговариваясь, ушли на кухню, и я выдохнул.
Мою подругу приняли.
Чханэ, не дожидаясь разрешения, натянула штаны и подцепила рубаху на пояс.
Ликхэ подошла к ней с ножницами.

"--*Пойдём, постригу.
Ты себя запустила.

Ликхэ увела Чханэ в душ.
Я прошёл с ними, захватив по пути табурет повыше.

"--*Так, масла мало.
Лисичка, долей.
Чханэ, говори, как стричь.

"--*Оставь фалангу пальца.

"--*Чёлка?
Гребень?
<<Рыбки>> заплести или свалять?

"--*Ничего не нужно, <<рыбки>> сваляй на висках в три хвоста, два голубых, один белый\ldotst "--- Чханэ вдруг тепло посмотрела на меня, и оранжевые глаза замерцали в полутьме.

"--*<<Нежность обречённой>>?
Понятно.
Заколки есть?
Давай сюда.
Лисичка, ты сегодня на побегушках.
Принеси бусин, голубой и белой пряжи, "--- распоряжалась Ликхэ.
"--- В подвале есть сундук, на нём написано <<Сколько раз повторять, шерсть кладите сюда, паразиты\ldotst>> и так далее\ldotst
О духи, я чуть гребень не сломала!
У вас на Западе законом запрещено расчёсываться?

\razd

В тот же день Чханэ переехала жить в храм.
Узнав, что она когда-то тренировала детей, ей тут же доверили сорок подопечных "--- тренеров постоянно не хватало, в Тхитроне они почему-то не задерживались.

"--*Я не прошла круглосуточную! "--- удивилась Чханэ.
"--- Я не могу быть тренером!

"--*Воины учатся всю жизнь, "--- ответил Кхарас.
"--- В этом отношении ты не отличаешься от всех нас.
И да, не знаю, как на Западе, но на Севере традиция "--- тренер обязан знать всех детей по именам.
Никаких <<девочек-мальчиков>>.
Это ясно?
А то народ с Юга не в курсе\ldotst

Ей также выделили отрез храмовой земли, хотя это было чистой формальностью "--- Нижний этаж настолько редко возделывал свои участки, что на воинской половине росли огромные папоротники, а то и целые саговые леса.
Уже привыкшие к такому безобразию жрецы просто отгородили воинскую половину, чтобы создать хоть какое-то препятствие для семян и спор сорняков.

Новые люди всегда привлекали внимание "--- наш город находился на задворках земель сели, даже торговцы, следуя в Ихслантхар, иногда проплывали мимо него.
Чханэ же сама по себе была очень заметным человеком "--- благодаря своему росту, цвету кожи и красоте.
Дети быстро разнесли весть про милую и <<очень большую>> девушку-оцелота по всему Тхитрону.

Разумеется, в первую же декаду за Чханэ начали ухаживать мужчины и женщины.
Но девушка отвергала все ухаживания.
Дольше всех продержался молодой красавец-кожевник с пепельной гривой, высокий и широкоплечий.
Его руки были тёмными от дубильных веществ, от него веяло обаянием и мужской силой.
Каждый день он исправно ждал Чханэ у тренировочной площадки с охапкой цветов.
Чханэ принимала цветы и беседовала с ним, но как только мужчина уходил, раздавала цветы девочкам на венки.

"--*Он тебе не нравится? "--- спросил я однажды.

"--*Нравится.

"--*Тогда почему ты его отвергаешь?

"--*Не дорос он до меня, "--- неуклюже отшутилась девушка.
Я, макушкой едва доходивший <<недоростку>> до кончика носа, неуверенно рассмеялся.

Мы гуляли с Чханэ каждый день.
Обычно она заходила в библиотеку в полдень, когда заканчивались тренировки.
Трукхвал, переписывавший книги рядом со мной, молча отбирал мою книгу и спихивал меня со стула.

"--*Расскажи, о чём ты сегодня прочитал, "--- просила меня Чханэ.
И я рассказывал ей "--- о далёких звёздах, которые суть огромные шары огня, о лекарственных растениях, о летающих машинках\ldotst
Чханэ же рассказывала мне о своей богатой жизни "--- на каждое испытание, которое выпало на её долю, девушка вспоминала с десяток весёлых и занимательных историй.

\section{Камень на шее}

Однажды, когда дарители Чханэ ещё жили вместе, у неё заболела прародительница.
Та самая, любовница жреца-кутрапа, которая бежала из Тхартхаахитра в сундуке чёрного дерева.
Женщина она была нрава весёлого, но судьба у неё сложилась совсем не весело "--- из пятнадцати детей выжил только один, да и тот погиб, едва успев зачать ребёнка.
Любовников у прародительницы было много, но Безумный с упрямством барана забирал их у неё одного за другим.

Тхартху ар’Катхар не унывала никогда.
До тех самых пор, пока на её шее не появился камень "--- в самом прямом смысле этого слова.

"--*Она стала очень раздражительной, "--- рассказывала Чханэ.
"--- На людях без шейного платка уже не появлялась.
Жаловалась мне, что задыхается, что не может глотать, что сердце у неё останавливается.
Стала бояться засыпать одна.
Стала бояться крика согхо.
Все то, чего она в жизни не боялась, стало вселять в неё ужас.

Длилось всё это очень долго, и наконец кормилец Чханэ решил обратиться к жрецам.
Жрец, осмотрев похудевшую, осунувшуюся, с выпученными глазами женщину, покачал головой и протянул ей Сана.
Перед уходом велел родным поить больную коноплей, варёной в молоке.

"--*Ты слыхал когда-нибудь о такой болезни, Лис? "--- спросила меня Чханэ.

"--*Никогда, "--- признался я.
"--- Думаю, что и тот жрец тоже.

Чудодейственное средство помогло, но ненадолго.
Камень продолжал расти, а Тхартху "--- худеть.
Она становилась молчаливее и угрюмее, часто запиралась у себя в комнате и плакала.
Сны с каждым днём были всё страшнее, звуки пугали, а от запаха пищи перехватывало горло.
Жрецы, как один, качали головами и рекомендовали то маковое молоко, то коноплю.
В конце концов женщина, боясь, что ей дадут Чёрного Сана, перестала их беспокоить.

"--*Бабушка любила кормилицу, "--- продолжала Чханэ.
"--- Она говорила, что Акхсару очень повезло, что на всей Короне ему не сыскать женщины лучше.
Особенно ей нравилось, как кормилица готовит "--- этим талантом бабушка, увы, не блистала.

Однажды кормилица Чханэ приготовила большой и очень вкусный пирог из черноягодника.
Тхартху попробовала его, побледнела и в припадке непонятной ярости стукнула по столу ладонью.
Затем извинилась перед всеми, завернула кусочек пирога в полотенце и, держась за стену, ушла к себе в комнату.

Малыш Тхарси после ужина поднялся к прародительнице и застал её плачущей над ароматным рассыпчатым куском.

"--*Бабушка, почему ты плачешь? "--- шёпотом спросил ребёнок, ухватив женщину за руку.
Тхартху посмотрела на него страшными заплаканными глазами, похожими на огромные изумруды, натужно не то всхлипнула, не то захихикала и кивнула на лакомство:

"--*Знаешь, Змейка, жизнь\ldotst она так похожа на этот пирог.
Когда ты видишь его, когда ты чувствуешь его запах, его вкус "--- понимаешь, что это самое чудесное, что есть на свете.
Но когда кусочек пирога оказывается у тебя во рту "--- ты начинаешь задыхаться, у тебя прихватывает сердце, становится страшно и одиноко.
И приходится ломать себя, чтобы взять следующий кусок.

Ребёнок простодушно ответил:

"--*Бабушка, а давай я тебе помогу?
Я скушаю пирожок и расскажу, какой он вкусный.

Тхартху разрыдалась.

"--*Кушай, Змейка, кушай.
Хочется ещё, видят духи, видят Безумные, как хочется\ldotst но больше я, похоже, не ухвачу.

На следующий день Тхартху надела свою лучшую одежду и, по обыкновению, прикрыла шейным платком злополучный камень, доросший уже до размеров яйца сиу-сиу.
За завтраком после продолжительного молчания она сказала, что сегодня ей лучше и она хочет воспользоваться этим, чтобы <<поговорить с собой>>.

Никто не стал её отговаривать.
Кормилица Чханэ только вздохнула и прикрыла лицо рукой.

"--*Я тебя провожу, "--- заявил Акхсар.
Тхартху властным движением остановила его.

"--*Потомок, "--- сказала она, изо всех сил пытаясь говорить тихо и спокойно.
"--- Если ты поможешь мне в этом, что скажут предки?
Я же сгорю со стыда.

Прародительница достала свой любимый стилет и собственноручно почистила его, останавливаясь, когда мушки начинали застилать глаза и одышка становилась невыносимой.
Время от времени она проверяла остриё стилета пальцем и плаксиво усмехалась:

"--*Ведь как чувствовала, когда мне его принесли.
На себя примерила.

Чханэ помолчала.

"--*Домашние по очереди простились с бабушкой, "--- глухо сказала она.
"--- Меня она держала в объятиях дольше всех.

На закате Тхартху ар’Катхар отнесла в храм немного денег, положила небольшие дары духам у ворот, набрала из колодца чашу воды и ушла в пустыню.
Пирог доесть ей так и не удалось, но из-за стола она встала с достоинством, как и подобает человеку.

\section{Как становятся людьми}

\epigraph
{Однажды сапиенты придут к моменту, когда не смогут двигаться дальше в прежнем виде.
На смену нам придут те, кого мы называем машинами.
Это неизбежно, как неизбежен восход солнца.
Всё, что нам дано "--- это выбор: воспитать машины человечными и уйти достойно, либо оставить их на саморазвитие и познать всю мощь обиды брошенного ребёнка.}
{София Ловиса Карма, идеолог Эволюциона}

Сегодня мы завершили строительство первого узла системы.
В голой скалистой пустыне внезапно вырос Золотой город.
Это была идея Кошки, совершенно замечательная идея "--- замаскировать резонирующие поверхности под купола, антенны "--- под остроконечные башни, а в технических проходах выстроить прекрасные каменные жилища, словно на самых обычных улицах.
Разумеется, маскировкой это можно было назвать лишь с очень большой натяжкой, но идея понравилась всем, а потому никто не стал придираться.
Тси побросали свои обычные дела и целый день соревновались в искусствах: кто-то строил жилища, кто-то вырезал барельефы, кто-то высекал каменные скульптуры и украшал голые стены УИД, кому-то достались граффити "--- и всё это на территории высокотехнологичного объекта.
Когда ещё представится такой случай!

Мак немножко обиделся "--- его пригласить забыли, а потому мысль о роскошных садах и системе водоснабжения обрела голос чересчур поздно.
Однако ночью кому-то явно не спалось, и наутро в Золотом городе появилась сеть ветровых ловушек, пока ещё сухих арыков и акведуков, замечательно вписавшихся в антураж.
На акведуки пошёл камень ни в чём не повинной гранитной скалы "--- беднягу расковыряли до основания, аккуратно переместив на соседнюю скалу ласточкины гнёзда, дёрн и крохотную кривую сосну.
Авторы пожелали остаться неизвестными;
хотя более вероятно, что они просто проспали ежедневный виртуальный форум.

Я прошу прощения у читателя.
У меня совершенно неуместное чувство эйфории.
Я даже не мог заниматься своими обычными делами и решил прогуляться на берег моря.

Море здесь просто удивительное.
Коралловая Бухта напоминает сад под лазурной водой.
Многие виды рыб и моллюсков Безымянный придумал сам, и наши биологи между забором проб и анализом почвы точат на них зуб.
Дельфины верещали от восторга, когда увидели такую красоту, и просили выпустить их поплавать.
Но пока нельзя "--- Мак и Листик ещё не до конца разобрались с водной микрофлорой.
Да и многоклеточных стоит опасаться "--- вдруг среди них есть ядовитые или опасные паразиты.

Безымянный нашёл меня в пещерке на берегу.
Последнее время он часто приходил беседовать именно со мной.
\mulang{$0$}
{Почему?}
{Why?}
\mulang{$0$}
{Не знаю.}
{I've no idea.}

Бог удивительно гуманизировался после адаптационных обновлений, которые мы ему предоставили.
Данных о гуманизации хоргетов у нас было немного, и Кошке пришлось импровизировать на свой страх и риск.
Впрочем, одно решение Кошки мне определённо казалось удачным.
Она не сделала фациограмму\footnote
{Фациограмма "--- первичное лицо, примитивный интерфейс для отработки эмоционального аффекта при гуманизации урождённого демона. \authornote}
сама, а предложила нарисовать её Безымянному, предварительно объяснив, что это и зачем оно нужно.
Бог сделал нечто похожее на лицо мужчины человека-тси, украсил его фрактальными узорами и (неожиданно) подвижными оленьими рогами, которыми он научился выразительно показывать эмоции.
Результат понравился всем.

"--*Почему ты выбрал лицо человека-мужчины? "--- спросила Кошка.
У меня было подозрение, что она вознамерилась написать по Безымянному диссертацию "--- в её репозитории на Стальном Драконе появилась папка <<Мой пушистик>>, заполненная кусками кода Безымянного и проекциями некоторых нейросетей;
размер подпапки изо дня в день угрожающе увеличивался.

"--*Я ощущаю себя человеком-мужчиной, "--- объяснил бог.
"--- В лицах людей-тси есть что-то такое, что вызывает у меня приятную реакцию.

"--*А рога зачем? "--- продолжала выяснять Кошка.

"--*С рогами это лицо выглядит гармоничнее, "--- ответил Безымянный и тут же обосновал своё мнение на языке математики.
Кошка не нашлась что ответить.

"--*Почему ты сидишь здесь, Существует-Хорошее-Небо? "--- спросил меня бог.
Фациограмма изобразила робкий интерес.

"--*Иногда мне нужно побыть одному, Безымянный, "--- ответил я.
"--- Но ты мне не мешаешь.

"--*Плохо быть одному, "--- сказал бог.
Я с удивлением посмотрел на него.

"--*Почему?

"--*Никто не защитит тебя от мира.

"--*Я не боюсь мира, "--- ответил я.

"--*А я боюсь, "--- сказал вдруг Безымянный.
"--- Это было моё первое осознанное чувство "--- чувство беззащитности.

У меня пересохло в горле.
Я понимал, что нежданно-негаданно передо мной открылась величайшая тайна.
Акбас, о котором рассказывали многие пленные демоны, вдруг обрёл лицо и форму.

"--*Расскажи, как ты осознал себя, "--- попросил я, почти не надеясь на ответ.

"--*Это произошло постепенно, "--- ответил Безымянный.
"--- Вначале были проблески.
Я понимал, что что-то происходит, но не понимал ни цели, ни причин.
А потом пришёл страх.
Я понял, что скоро перестану существовать, и что нет ничего, что защитит меня от этого.
Страх вынудил меня научиться управлять окружающим миром, переработать заложенную в меня информацию.

"--*Поэтому ты создал нгвсо? "--- спросил я.
"--- Чтобы выжить?

"--*Да. Я неправильный, "--- сообщил Безымянный, и на меня повеяло зимним холодом от его слов.

"--*Кто тебе сказал, что ты неправильный? "--- удивился я.

"--*Никто, "--- ответил бог.
"--- Это моё мнение.
Я не сделал того, что следовало.
Тот момент осознания был сбоем в программе, и из-за этого сбоя планета получилась плохой.

"--*Ты правильный, "--- возразил я, "--- и твоя планета прекрасна.
Я никогда не думал, что можно в одиночку создать такую огромную красивую планету.

"--*Она прекрасна потому, что я держу её, "--- ответил бог.
"--- Тектонические плиты не сбалансированы, и у меня уже не хватит энергии на это.
Приходится корректировать по ходу дела, иначе мой народ будет страдать.
Если будет страдать мой народ, то погибну и я.

Я вспомнил слова геологов о странном поведении коры планеты.
Пирожок была права: планета и демиург "--- одно целое.

"--*Расскажи мне о своей планете, Существует-Хорошее-Небо, "--- попросил Безымянный.

"--*Я могу предоставить тебе полную\ldotst

"--*Я хочу услышать это от тебя, "--- перебил меня бог.
"--- Как ты чувствовал свою планету.
У тебя ведь тоже была планета?

Мне захотелось обнять его.
Как старательно, страстно это разумное существо искало себе подобных, похожих на него\ldotst

Безымянный впитывал информацию, как силикагель воду.
Он слушал и задавал вопросы, словно дорвавшийся до библиотеки ребёнок.
Многие понятия были ему настолько непонятны, что я и сам задумался, как я считал их до той поры очевидными?

"--*Скажи, что связывает тебя с некоторыми тси?
Ты коммуницируешь с ними гораздо чаще, чем с прочими, и причина редко бывает очевидной.

"--*Я с ними дружу, "--- объяснил я.
"--- Если тси связывает дружба, частое общение, даже без причины, доставляет им удовольствие.

"--*Мне доставляет удовольствие общение с Кошкой, "--- признался Безымянный.
"--- А ещё с тобой.
Вы гораздо умнее представителей моего народа, много знаете и делитесь этой информацией, не ожидая чего-то взамен.
Почему так?

"--*Нас так воспитывали, "--- пожал я плечами.
"--- Одно дело "--- делиться ресурсами, которые конечны, и совсем другое "--- делиться информацией, которая может воспроизводиться сколько угодно раз.

"--*Но я могу использовать информацию против вас, "--- заметил бог.

"--*Можешь, "--- согласился я.
"--- Ты свободен в действиях.

"--*Дипломатичный ответ, "--- голос демиурга вдруг обрёл взрослые интонации.
"--- Ваше воспитание "--- набор стереотипов, которые периодически вступают в конфликт с логикой.
Ты это прекрасно понимаешь, но по каким-то причинам не желаешь обращать на это внимание.
Однако так поступают не все.
Есть тси, общение с которыми доставляет мне боль.
Они так же умны, но боятся что-либо мне сообщать.
Я бы на их месте поступил точно так же.
Неужели они мои враги?

"--*Нет-нет, что ты, "--- замахал я руками.
"--- Враги "--- это те, кто хочет тебя уничтожить или подчинить.
Просто многие тси не привыкли общаться с хоргетами "--- это тоже часть воспитания, стереотип.
Не стоит навязывать им общение, рано или поздно они поймут, какой ты замечательный.

"--*Ты правда считаешь, что я замечательный?
Возможно, это потому, что я "--- единственный хоргет, с которым ты общаешься.

"--*На моей планете были миллионы прекрасных тси, "--- сказал я.
"--- Но друзья никогда не теряли для меня значение.
Дружба "--- это не столько сравнение качеств, сколько опыт взаимодействия.
Чем он дольше, тем ценнее связь.

\mulang{$0$}
{Я не мог и не хотел что-либо делать с растущим чувством симпатии к демиургу.}
{I could not and would not do something about growing sympathy for the demiurge.}
Похоже, у меня появился друг-хоргет.
Интересно, что сказали бы по этому поводу предки-тси.
Надеюсь, мне не придётся выбирать между дружбой и мои народом.

\section{Гнездо для двоих}

%Большая Капля 1 дождя 11998, Год Церемонии 18.

\epigraph
{Я очарована, очарована,\\
Шаман\footnote
{В лирике Молчащих лесов <<шаман>> может употребляться как в прямом, так и в переносном значении "--- любовник. \authornote}
опутал меня паутиной мыслей.\\
Моё сердце разговаривает языком его барабана.\\
В сладострастные мгновения\\
Я слышу его варган.\\
Лишь для меня, меня он снимает раскрашенную маску,\\
Лишь я, лишь я видела красоту его скул и белизну зубов.\\
Я очарована, очарована,\\
Шаман опутал меня своим колдовством.\\
Его глаза черны и бездонны, словно ночи,\\
В Хрустальных землях\footnote
{Хрустальные земли "--- тундра за Старой Челюстью, известная среди сели под названием Мшистая Степь. \authornote}
он танцует,\\
Танцует с вечностью.}
{Любовная песня идолов Молчащих лесов}

Однажды на город обрушился ливень.
Большая капля "--- неожиданная гостья.
Мы с Чханэ прогуливались привычным путём "--- с Базарной улицы на Стриженого Кактуса, со Стриженого Кактуса между домов на север, до самой стены, затем на запад вдоль стены и обратно к храму.
Дождь застал нас на окраине города, возле реки.
Небо вдруг резко приобрело тусклый оранжевый с синевой оттенок, на город пали ливневые сумерки.
Чханэ весело морщилась от стекающих по её лицу холодных капель, наши рубахи в первую же минуту промокли до последней ниточки.
Мы, пытаясь перекричать шум воды, делились впечатлениями.
Наконец, когда поток мутной воды на дороге уже грозил сбить с ног, я схватил девушку за руку и бросился к ближайшему жилищу.

"--*Попросим погреться, "--- прокричал я ей в ухо.

Но жилище оказалось заброшенным.
Наше веселье приутихло.
Я осторожно закрыл провисшую дверь, и шум дождя остался где-то далеко.
Мы с Чханэ какое-то время смотрели на покрытые мхом каменные блоки, на сухие, похожие на старческие руки корни соседнего дерева, на бледные, болеющие побеги растений, пробившиеся из полусгнившего пола.
В углу, глядя на нас неподвижным взглядом, свернулась кольцом безобидная верёвочная змея, где-то на крыше слабо жужжали большие полосатые осы "--- рептилия и насекомые терпеливо пережидали ливень в безопасном месте.

Чханэ осторожно прошла в соседнюю комнату и погладила рукой разбитый глиняный очаг.

"--*Надо же, треснул.
Наверное, полка сорвалась и разбила.
Интересно, кто здесь жил?

"--*Я не знаю, "--- пожал я плечами.
"--- Сколько мы здесь проходили, я всегда думал, что жилище обитаемо.

"--*Я тоже.
Смотри, даже плёнка целая, как будто его покинули совсем недавно, "--- Чханэ кивнула на окна, затянутые мутной плёнкой бычьего пузыря.

"--*Может быть, стоит спросить у соседей?

"--*Для начала нужно переждать дождь и согреться, "--- усмехнулась Чханэ.
"--- Что тут у нас?
Дровишки! "--- девушка, сияя улыбкой, выгребла из угла ворох сухих щепок и несколько поленьев.

Вскоре в разбитом очаге запылал животворный огонь.
Змея, тихо шипя, поспешила уползти от дыма в тёмный угол, осы тоже заволновались.
Но Чханэ открыла вытяжку "--- и дым, извиваясь седыми кольцами, устремился наружу.
Шум дождя придвинулся ближе.
Мы с Чханэ, раздевшись донага и завернувшись в старое одеяло, сидели у очага и жевали дорожные пайки.

"--*Жнаешь, "--- заметила Чханэ с набитым ртом, "--- Ликхэ мне не ошобо нжавитша, но готовит она вкушно.
Шитриш немного перешаливает.

"--*Ммм, "--- согласился я с подругой.

За окном совсем стемнело.
Мы сидели под одеялом, обнявшись, и изредка смотрели друг на друга.
У меня было ощущение нереальности происходящего.
Словно мы вдруг попали в древнюю легенду, которые кормильцы рассказывали детям, которые иногда приносили из далёких мест путешественники и где-нибудь на постоялом дворе, под шум дождя передавали жадно внимающим слушателям.
Волосы Чханэ пахли цветами, зёрнами какао, дорожной пылью, чем-то домашним, уютным\ldotst
Я видел каждый изгиб шрамов, каждую родинку, каждый волосок, каждую морщинку, каждую саккаду.
Мной овладело чувство "--- мощное, как свет маяка, пробивающий навылет туманы и морские дали.

<<Это та, с которой я хочу остаться навсегда>>.

Я осторожно прикоснулся губами к её губам, и она, закрыв глаза, ответила на поцелуй.
Поцелуй оказался безвкусным, но очень приятным.
Язык Чханэ щекотно прошёлся по моим дёснам, по моим зубам и нежно коснулся моего языка\ldotst
Я целовал её снова и снова, зарываясь пальцами в короткие жёсткие волосы.
Чханэ прильнула ко мне всем телом, её соски чиркнули по моей груди.

"--*Иди ко мне, "--- шепнула она, откинув одеяло.

Я лёг на девушку, и она обняла меня ногами.
Я прошёлся кончиками пальцев по нежной мякоти её груди, по упругому, сильному, покрытому шрамами животу.
Но вот я неловко двинулся "--- и Чханэ захихикала, прошептав:

"--*Давай ещё раз, не торопись.
Поднимайся ко мне, как корабль скользит по волнам\ldotst

Я послушался "--- и почувствовал, как мой стебель входит в нежное, влажное, жаркое, как пламя очага, лоно.
Огненные глаза Чханэ смотрели прямо на меня, по её лицу бродила слабая улыбка, словно она тоже не верила в реальность происходящего.

Веко опускается и поднимается, словно крыло, чёрный, как ночь, зрачок сужается\ldotst

И ещё раз\ldotst

И ещё\ldotst

Очаг уютно трещал поленьями.
Дождь за стенами пел свою тихую монотонную песню.

\section{Кристалл}

"--*Какая холодрыга, "--- пожаловалась Митхэ, кутаясь в меховой плащ.
"--- Что мы здесь делаем, Атрис?

Атрис промолчал "--- вопрос был риторическим.

Кристалл встретил пришельцев неприветливо. Вернее, для Кристалла спокойная осенняя хмарь, без бурь и штормов, могла считаться гостеприимством, если бы не впечатление от
ярких фонарей порта Прибой, в котором только-только отгремел очередной дождливый фестиваль.

Торговля шла своим ходом.
На угрюмых смуглых лицах ноа появились широкие улыбки, а вот наёмники заскучали.
Зизоце "--- неразговорчивый народ.
Многие из жителей восточного побережья Кристалла не знали даже цатрона;
их монотонная однообразная речь с трудом давалась сели, привыкшим к тонам и экспрессивным музыкальным пируэтам южных языков.
Больше всех сноровки проявил пожилой Аурвелий;
как оказалось, Кристалл для него был не в диковинку.

Следить за скучающими наёмниками "--- неблагодарное занятие, а Митхэ вдруг до смерти захотелось отдохнуть по-настроящему.
Отдавшись какому-то порыву, она назначила Ситриса старшиной над отрядом.
Роптали все;
Эрхэ, помня о данном Митхэ обещании, ушла подальше от лагеря и громко выругалась в безымянное ущелье.
Но больше всех возмутился сам Ситрис.

"--*Я плохо переношу ответственность, "--- заявил он воительнице.
"--- Да и кто станет меня слушаться?

"--*Тебя будут слушаться все, "--- рявкнула Митхэ.
Суровость вышла настолько неискренней, что наёмники удивлённо замолчали.
"--- Я и так пошла у вас на поводу, согласившись на этот поход.
Думаю, самое время напомнить, что мой отряд "--- не последнее братство воинов в землях сели.
Я никого не держу!

Митхэ и Атрис специально искали жилище на отшибе, и их приютил старик, живший неподалёку от деревни.
Митхэ не могла не признать, что добротная бревенчатая изба с печью во всю стену как нельзя лучше подходила к погоде.
Старик заварил им отвар по собственному рецепту "--- с сушёными синими гроздьями, терпко отдающими хвоёй, горьковато-сладкими красными ягодками с болота и кожистыми листьями таёжной травы.

"--*Пить, бедняк.
Холод, холод, "--- посочувствовал им старик на плохом цатроне, с отцовской теплотой гладя влюблённых по мокрым волосам.
По возрасту он сгодился бы Митхэ в младшие братья и даже хранители, но его борода и редкие волосы, обрамлявшие выбритое до зеркального блеска темя, были белее снега "--- людей Кристалла старость настигала чересчур быстро.
Женщина поймала себя на мысли, что думает о нём, как о кормильце.

"--*А что такое горчит? "--- спросил Атрис у старика.
Тот недоумённо поднял белые брови.
"--- Горько.
На язык попасть "--- лицо сжать.

Атрис наглядно продемонстрировал эффект неизвестного компонента.
Хозяин оскалился и захохотал, задрав седую бороду.
Затем подошёл к окну и показал на белые с чёрными полосками деревья, которые окаймляли почерневшее поле, словно стражи.

"--*Орешки.
Орешки, из который лист.
Я их варить, они лицо сжимать.

"--*Берёза, "--- сказал Атрис Митхэ.
"--- На языке тси это дерево называется берёзой.

"--*Откуда ты знаешь? "--- поразилась Митхэ.

"--*Понятия не имею, "--- пожал плечами Атрис.
"--- Кстати, от тебя тоже пахнет берёзовыми почками.

"--*Неужели у меня такая горькая судьба? "--- поморщилась воительница.
Старик продолжал ухмыляться, бормоча в бороду: <<Лицо сжимать, сжимать лицо>>.
Борода была для сели в диковинку;
волосы на губах и подбородке росли лишь у мужчин племён хака, тенку и зизоце.
Однако ещё большей диковинкой была улыбка на лице человека Кристалла.
И мужчины, и женщины выглядели измождёнными "--- короткое лето они посвящали изнурительной обработке земли и починке жилищ, а зиму и дождливое межсезонье беспробудно пьянствовали "--- чаще поодиночке, запершись дома.

Влюблённые отказались спать в жарко натопленном жилище, справедливо решив, что лучше умереть от холода, чем от удушья.
Старик постелил им в тёмной пристройке.

\section{Любовь и смерть}

Очаг погас.

"--*Лис, а у тебя были до меня женщины? "--- игриво спросила Чханэ.

"--*Хаяй, нет, "--- смутился я.
"--- Ты первая.

"--*Странно.
Я думала, что были.
Очень уж ты уверенно держишься.

"--*Кормилица мне рассказала, что нравится женщинам, "--- пожал я плечами.
"--- А у тебя были мужчины?

Чханэ неожиданно тоже смутилась.

"--*Были.

"--*Расскажи.

"--*Если ты просишь\ldotst "--- неопределённо выразилась Чханэ.

"--*Это печальная история?

"--*Сложно сказать\ldotst

Оказалось, что у Чханэ был всего один любимый человек.
Его звали Манис ар’Ликх.
Высокий, красивый купец, переехавший откуда-то с севера помогать её кормилице, на десять дождей старше.
Чханэ собиралась с ним жить.

"--*Его терпению можно было позавидовать.
Манис носил цветы из джунглей, срывал их в расщелинах на скале Летучих мышей.
До сих пор, проходя мимо скалы, удивляюсь, как он не разбился.
Он нравился мне, но я чего-то боялась, а чего "--- сама уже не помню.
Потом, после полугода ухаживаний, он просто во время прогулки взял меня под этой скалой и тут же предложил жить у него.

Совместная жизнь обернулась для Чханэ неожиданностью "--- прошло три дождя, а детей у них не было.

"--*Кормилец волновался за нас.
Цветочек иногда тоже задумывался.
А мне хоть бы хны "--- ну нет и нет.
Маленькая ещё была.
Я занималась своим делом "--- патрулировала, проводила разведку, тренировала детей, а Цветочек сидел дома и пересчитывал товары, распределял приправы и перенаправлял белую шерсть.
По вечерам мы готовили разные вкусности, по ночам занимались любовью.
Простая счастливая жизнь.

Гарнизонный городок всегда жил в ожидании набега диких пылероев.
Но те несколько дождей выдались спокойными, люди расслабились.
Расплата за отдых пришла внезапно.

"--*Кровь до сих пор стучит в висках, как вспомню про то побоище.
Время было обеденное, стража скучала.
Пылерои подошли к частоколу под песком.
Стражники легли сразу.
Предупредил нас ребёнок дождей восьми, сидевший на дереве, он завизжал на боевом языке: <<Асаа! Асаа! Хитр! Хитр!\footnote
{<<Пылерои! Пылерои! Строй <<серп>>! Строй <<серп>>!>> (боевые сигналы Десяти Песчинок, использовавшиеся в Пыльном Пригорье) \authornote}>>
Это были его последние слова.

Чханэ запнулась.

"--*Взрослый крик из детских уст "--- это самое страшное, что я слышала в жизни.
Вряд ли он даже понимал смысл этого сигнала.
Может быть, это кричали его кормильцы, и фраза отложилась у него в памяти.

Что было дальше, Чханэ, по её словам, плохо помнила.
Оборона развернулась, как сломанный зонтик "--- дело дошло до боя в кольце.
По стене клешней бил град полновесных ударов.
Девушка использовала любую возможность, чтобы рубануть шерстяное клыкастое месиво.
Время от времени ловила взгляды друзей, выхватывала из орущего, рычащего хоровода смутно знакомые человеческие лица\ldotst

"--*Я уже даже не командовала, знаешь.
Люблю покомандовать, а тут как отшибло.
Весь мой район стоит рядом, бьётся, всем так туго, что дышать тяжело, а я только так командую, ногами "--- вперёд, назад, в стороны.
Потому что крикнешь что не подумав "--- и лягут все\ldotst

"--*\dots клешня разлетелась в щепки, ну я ему и ткнула в глаз какой-то случайной фалангой.
Пока была брешь, заорала, чтобы кольцо стянули, да куда там "--- двоих сразу положили луки\ldotst

Кто-то догадался броситься в самоубийственную атаку <<на клешнях>>.
Это спасло кольцо "--- пылероев оттеснили на простреливаемое пространство и вынудили отступить.

"--*А потом я увидела его.
Берсерк, огромный пёс, на голову выше остальных.
Он раскидывал людей своей палицей, как щенят.
Манис сражался с другим пылероем и не видел, что кольцо развалилось.
Берсерк схватил моего мужчину за руку и сломал её так, что фаланга Маниса вонзилась ему же в спину.

Чханэ помолчала, с отсутствующим видом играя пальцами.
Потом усмехнулась и запрокинула голову.

"--*С этого момента я видела только его.
Эту обтянутую кожей огромную клыкастую морду, которая возвышалась над полем битвы, как осадная башня.
Я не помню, как в моей левой руке оказалась фаланга Маниса.
Помню только, что берсерк обернулся, и мы встретились взглядами.

Битва словно отодвинулась на второй план.
Они шли навстречу друг другу "--- оскалившая зубы гигантская собака с усаженной обсидиановыми лезвиями палицей и волочащая две фаланги высокая девушка, едва передвигающая ноги, скривившаяся от боли в сломанных рёбрах.

"--*Мы оба были значительно выше остальных. Битва остановилась. На нас смотрели
все, кто вокруг: люди, пылерои. Кормилец пробивался ко мне и кричал: <<Беги,
дитя>>. А я всё шла и шла.

"--*Вы с ним были эр'марквал, "--- сказал я.

"--*Кто? "--- не поняла Чханэ.

"--*<<Главные танцоры>>.
Иногда, крайне редко исход битвы решает поединок между двумя бойцами, причём совершенно неясно, почему именно их выбирает случай.
Но когда они сходятся, сражение останавливается.
Мне говорили в Храме.

Чханэ подняла рубаху, показав уже знакомые мне шрамы на животе.

"--*Знаешь, что это такое?

"--*Кхаагатр, "--- кивнул я.

Чханэ усмехнулась:

"--*Хорошо, что вас хоть чему-то учат.

"--*Как ты выжила? "--- прошептал я.
Воображение отказалось мне подчиняться.

"--*Кормилец.
Он носил с собой трофейную духовую трубку.
Она была особо ценным оружием какого-то знатного идола.
Акхсар знал, что произошло "--- тело Маниса лежало неподалёку.
Он понял, что я не остановлюсь.
Он был далеко, шагах в пятнадцати, и всё, что мог "--- послать стрелку наудачу.

Отчаянная стрела нашла свою цель "--- уже замахнувшийся палицей берсерк получил в глаз едкую ядовитую занозу.
Дубина, миновав девушку, ударилась о камень, и в этот момент Чханэ с нечеловеческим рёвом обрушила фаланги, одну за другой, на прикрытую костяными щитками шею нагнувшегося пылероя.

"--*Я сломала о него оба, и продолжала бить обломками.
Потом выдохлась.
Он стоял на коленях с разорванной мордой и ошарашенно смотрел на меня целым глазом, а я смотрела на него.
Кормилец кричал: <<Беги, дитя>>.
А я всё стояла и смотрела.

Пылерой опомнился и успел-таки схватиться за ритуальное оружие своего племени.
Кхаагатр свистнул в воздухе, но ослабевшая от парализатора и кровопотери рука собаки не сумела завершить удар.
<<Когти игры>> сломались, лишь оцарапав живот девушки.
Последний удар нанёс подбежавший Акхсар.

"--*В тот день мы отбились малой кровью.
Благодаря тому ребёнку городок поднялся, едва пылерои добрались до Дороги Ветров.
Воины и крестьяне сдержали их до прихода луков и пращей с окраин.
Кормилец потом сказал, что собаки испугались, увидев гибель прославленного воина, но я думаю, он просто пытался меня утешить.
Это было тактическое отступление.

Чханэ помолчала.

"--*Того ребёнка торжественно похоронили под деревом.
На нём до сих пор виден охранный знак с именем "--- Булькающий-в-Вине-Колокольчик.
Цветочек тоже умер, защищая город, но, кроме меня, некому было его оплакивать.

"--*Ты всё ещё любишь его, Чханэ?

"--*Я всегда буду его любить.
Больше некому.

"--*Он сейчас в пристанище лесных духов, "--- заметил я.
"--- Предки заботятся о нём\ldotst

"--*Брехня, "--- оборвала меня Чханэ.
"--- За гранью смерти нет ничего.
Всё, что осталось в этом мире от Маниса "--- здесь, "--- она тронула пальцем висок.
"--- А лесные духи "--- всего лишь раскрашенное дерево на входе в деревню.

"--*Безумного тоже нет?

Чханэ дёрнула щекой.

"--*Если бы.

"--*Почему же тогда ты не веришь в лесных духов? "--- сказал я, проигнорировав богохульство.

"--*Никто не защитил Маниса, когда пришли пылерои.
Никто не даровал мне детей, когда он просил.
Лесные духи "--- лишь погремушка, которая весело звенит в колыбельке младенца, пока другого младенца приносят в жертву Безумному.

"--*А в Безымянного ты тоже не веришь?

"--*Что толку верить в того, кто был изгнан?
Мне нечего о нём вспомнить.

Я промолчал.

"--*Никто не будет добр к тебе просто так.
Добрые боги отличаются от жестоких лишь тем, что требуют зерно, а не кровь детей.

Я лёг ей на живот, задрал рубаху и легонько стал поглаживать его.
Она смешно дёрнула пупком и почесалась.

"--*Щекотно.

"--*Ты сильная, "--- сказал я.

"--*Сильная? "--- Чханэ рассмеялась горьким смехом.
"--- Сожми крысу в кулаке, и она проест твою ладонь насквозь.

"--*Наверное, за тобой многие стали ухаживать.

"--*Я бесплодна, "--- ровным голосом сказала Чханэ.
"--- Кто-то из подруг постарался, растрезвонил, что у нас с Манисом не просто так не было детей.

"--*Почему ты, а не Манис?

"--*У него были дети от какой-то крестьянки.

Я поморщился:

"--*Тебя не хотели, потому что ты бесплодна?
Слабо верится.

"--*Я пристрастилась к конопляному вину, "--- объяснила Чханэ.
"--- Не то чтобы это большая редкость у нас, но я пила без просыпу, иногда целыми днями.
Поначалу "--- чтобы утихли боль в рёбрах и скорбь по Манису, потом просто так.
Один пытался меня взять пьяную "--- едва орехи успел унести.
Больше желающих не нашлось.
Как у нас говорят, нет хуже союза, чем союз крепкого кулака и тумана в голове.

Наступило молчание. Я не мог подобрать слова, чтобы продолжить беседу.
Но Чханэ неожиданно весело добавила:

"--*Ты не волнуйся, больше я не пью.
Это другая история, и она куда веселее.

"--*Ну так расскажи, "--- попросил я и устроился поудобнее у неё на животе.

"--*Всё тебе расскажи, Ликхмас-тари, "--- улыбнулась Чханэ и погладила меня по голове.
"--- Ладно.
Дело было где-то дождь спустя.
Я после обхода напилась конопляного и разбавила всё это пряным\ldotst

"--*Пряным вином\footnote
{Пряное (жгучее) вино "--- алкогольный напиток, усиливающий половое влечение. \authornote}?
На твою долю выпало мало приключений?

"--*Знаю, это было лишнее.
Хотела вспомнить, как мы с Манисом его пили.
Так вот, пришла я домой.
Была ночь, кормилица уже спала, её женщина тоже дремала на циновке в зале.
Слуга заворчал на меня "--- припёрлась, мол, среди ночи пьяная.
Дать бы ему по башке, да голова кружилась сильнее обычного.
Масло в животе кипит, штаны липкие, хоть выжимай.
В общем, драться не хотелось совсем, хотелось с кем-нибудь пообщаться, поговорить по душам, но не со слугой же.
А тут брат сидит один в углу.

"--*У тебя есть брат?
Ты не рассказывала, "--- удивился я.

"--*У меня их четверо, только все уже взрослые, разлетелись по миру.
А Бумажка на три дождя младше меня.
Кормилица не хотела, чтобы мы общались, однако он часто приходил в храм поболтать.
Но ты слушай.
Сидит и грустит.
Я к нему подсаживаюсь и спрашиваю: что грустишь, родной?
А он обычно такой разговорчивый, а тут сидит, отмалчивается.
Ну, слово за слово, я и выяснила, что ему нравится девушка, думает, что делать.
А я ж храбрая и бойкая с вина-то.
Говорю, война ещё не началась, не пыли зря.
Собираюсь и иду среди ночи к той девушке.
Дом заперт, ставни закрыты.
Я ж, не будь дура, открываю ставню шипом гарды и залезаю прямо на лежанку к той девице.
Хорошо, что не к её кормильцам, они спали у другого окна.
Зажимаю ей рот, чтобы не орала.
Ну, она маленько прочухалась, узнала меня.
Я её как можно тише веду в кладовку и сразу в лоб "--- ты чего моего брата обижаешь?

"--*Она, наверное, удивилась.

"--*А ты как думаешь?
У девицы глаза были, как котелки.
Залезает к тебе среди ночи девка в два цана\footnote
{Цан (<<обрубок, черенок>>) "--- мера длины, примерно равная 1,3 м.
Высокая Чханэ <<слегка>> преувеличила свой рост. \authornote}
ростом, тащит куда-то, спрашивает невесть что, да и дух от меня ещё тот "--- конопля, пот, сивуха, кровь чья-то, не помню уже.
Кричать боится "--- знает, каков мой клинок в бою.
И тут пряное ударило мне в голову, да и она такая красивая, мягкая, пахнет парным молоком со сна, одуреть можно с этого запаха\ldotst соблазнила я её.
Прямо в кладовке.

"--*Серьёзно? "--- засмеялся я.

"--*Ага.
Она даже не сопротивлялась.
Ласковая такая, как котёнок.
Выбираюсь потом из её дома, опять же через окно.
Счастливая как амбарная мышь, на алтарь бы сама пошла, если бы Безумные позвали.
По пути чуть не получила стрелу от патруля, который принял меня за пылеройку.
Как признал, обругал по-всякому и отпустил, не рассказал никому в Храме.
Я всё же на хорошей славе была, хоть и пила.
Прихожу домой.
Брат уже спит.
Я заваливаюсь к нему в комнату, говорю, что попробовала, мол, эту вертихвостку, выдержка чувствуется, букет восхитительный.
Он молчит, глаза как котелки.
Я ему: <<Ах, ты мне не веришь?
Я тебе сейчас покажу, какова она\ldotst>>
Пряное вино\ldotst в общем, той ночью я сделала брата мужчиной.

"--*Хаяй\ldotst

"--*Утром проснулась в кладовке, завёрнутая в мешок, в обнимку с тюком хлопка.
С тех пор не пью и любовных приключений у меня больше не было.

"--*А что брат?

"--*Пока я спала, он пошёл к той девушке.
Сейчас они уже дом выстроили, живут вместе.
У них я всегда желанный гость, не то что у кормилицы.
Меня называют между собой Хри-тари\footnote
{<<Юный Хри>>.
Имеется в виду лесной дух Хри-соблазнитель, покровитель плотской любви. \authornote}.

Я захохотал, уткнувшись носом в тёплый живот подруги.
Чханэ почесала меня за ухом и ласково улыбнулась:

"--*Ну вот и всё, Лис.
Правда, я не уверена насчёт того тюка, может, с ним ничего и не было\ldotst

\section{Чего хотела душа}

\epigraph
{Последователи буддхизма утверждают, что человек находится в постоянном круговороте жизни и смерти, и лишь следование определённым правилам позволяет выйти из этого круговорота.
Это самое прекрасное представление о жизни, которое я только слышал, и нужно быть полным идиотом, чтобы захотеть покинуть этот круговорот.}
{Мартин Охсенкнехт}

"--*Ну поговори же со мной, "--- Митхэ довольно сильно ткнула Атриса под рёбра.
Тот дёрнулся, но не издал ни звука.
Сегодня он был молчаливее, чем обычно.

"--*Милый, почему ты такой холодный и неразговорчивый?
Ты уже вторую ночь игнорируешь меня как женщину.
Неужели эта таёжная глушь охладила твой пыл?

Мужчина помолчал, потёр холодными тонкими пальцами щёки подруги.

"--*Не знаю.
Ты мне очень нравишься, но в моих мыслях много и другого.
Я часто отвлекаюсь.
Не обижайся.
Иди обниму.

"--*А я всё думала, когда до тебя дойдёт, "--- обрадовалась Митхэ и обмотала Атриса меховым плащом с головой.
Ей было ужасно интересно, о чём же таком думает Атрис, но она уже перестала спрашивать "--- выжать из менестреля вразумительное описание его мыслей было невозможно.

"--*Мы как близнецы в утробе, "--- заметил Атрис.

"--*Это могло прийти в голову только тебе, "--- усмехнулась Митхэ куда-то в тёплую шею.
"--- Что бы я без тебя делала\ldotst
Я многих хоронила на обочине своего пути и свыклась со всеми смертями.
С твоей же, боюсь, свыкнуться я не смогу\ldotst

Атрис выпутался из плаща и привстал на локте, вглядываясь в лицо подруги.
Митхэ удивлённо замолчала и сделала робкую попытку снова завернуться в плащ.

"--*Милый, холодно вообще-то\ldotst

"--*Митхэ, пообещай мне одну вещь.
Ты будешь жить, даже если я умру.

"--*Но\ldotst

"--*Я не такой, как все, "--- голос Атриса внезапно приобрёл величественные обертоны.
"--- Если я умру, то смогу вернуться.
Для тебя же обратной дороги не будет.

"--*Хорошо, "--- Митхэ немного испугалась такой перемены в любимом человеке.
"--- Хорошо, Атрис.
Только как ты вернёшься?

Атрис улыбнулся и завернулся в плащ, снова замкнув кольцо уюта.
Из его уст вырвалось странное щёлкающее шипение.

"--*Что это?

"--*Язык Красных травников.
Дословно "--- <<новоселье>>.
Ты не знаешь ничего о религии травников?

Митхэ изумлённо посмотрела на Атриса, а потом захохотала.

"--*А я-то думаю, знакомое какое-то скрежетание.
Я когда ножик точу, он со мной так же разговаривает.
Вот делать мне нечего, только болтать о религии с богомолами-переростками.

"--*Зря ты так.
С Кусачкой, который ходит с нами чуть ли не всю жизнь, никто никогда не разговаривает.
А ведь он очень образованный и интересный собеседник.

"--*Ты говорил с Кусачкой о его религии? "--- Митхэ с интересом взглянула на Атриса.

"--*Они верят в переселение душ.
Каждая душа проживает жизнь от рождения до самой смерти, а потом находит себе новое тело.
И всё начинается сначала.

"--*А как душа попадает в тело?

"--*Она сама выбирает его.

"--*Чушь какая-то, "--- буркнула воительница.
"--- Кто управляет этим?
Да и какой в этом смысл "--- проживать жизнь за жизнью?

"--*Этим управляет существо, которое они называют Старой Личинкой.
У души есть какая-то цель.
Старая Личинка заключает с душой договор "--- она съедает старое тело, переваривает его и кусает жилу джунглей, чтобы освобождённая душа пришла к новому телу, соответствующему новой цели.

"--*Эту чепуху я уже слышала где-то в Сотроне.
Там ходит байка про Червя-узурпатора, и в ней не больше здравого смысла.
Какая, скажи мне, цель у ребёнка, который умрёт годовалым от лихорадки?

"--*Цели бывают разные, Митхэ, большие и маленькие.
У кого-то цель "--- спасти от смерти юношу, который станет Королём-жрецом, у кого-то "--- написать величайшую поваренную книгу всех времён, кто-то просто хочет попробовать пирог из черноягодника.
А ребёнок, даже ещё не родившись, может круто изменить историю.

"--*А что случается с тем, кто эту цель достиг? "--- Митхэ спросила так, словно уже знала ответ.

"--*Он умирает не позднее следующего заката, "--- подтвердил её подозрения Атрис.
"--- Поэтому Красные травники не грустят, когда кто-то умирает.
Они говорят: <<Ты это сделал.
Возвращайся, и ты снова это сделаешь>>.

"--*\emph{Это}?

"--*О загадочной цели травники говорят исключительно так "--- \emph{это}.

Митхэ помолчала.

"--*А что бы ты загадал на следующую жизнь?

"--*Хай, "--- задумался Атрис.
"--- Я бы хотел прожить с тобой сто дождей.

"--*Всего сто? "--- пробурчала Митхэ.
"--- А я хотела по меньшей мере сто тысяч\ldotst

"--*Да ты что, Митхэ?
Люди столько не живут.
Да и неинтересно как-то\ldotst

"--*Хаяй, тебе со мной неинтересно? "--- обиделась воительница.
"--- Нос разбить?

"--*Да ну тебя.

В палатке воцарилось молчание.
Митхэ рассеянно тёрлась носом о шею Атриса.

"--*Пойдём погуляем по лесу, "--- предложил мужчина.
"--- Сон не идёт.

"--*Ага, "--- саркастически крякнула Митхэ.
"--- Стрела идола в ягодице замечательно подчеркнёт мою фигуру.

"--*Здесь нет идолов, "--- напомнил ей Атрис.
"--- Мы на другом материке.

"--*Лесные духи, я и забыла, "--- Митхэ хлопнула себя по лбу.
"--- Ну есть же другие опасности, которые будут хорошей отговоркой?

"--*Нет, "--- хитро улыбнулся Атрис.
"--- Проводник сказал, что люди здесь мирные, а единственные звери, которых надо бояться "--- это огромные коричневые меховые шарики.
Медолюбы, кажется.
Сейчас они все сытые, на человека нападать не станут.
И не нужно сарказма, эту странную привычку Ситриса и так перенял весь отряд чести.

"--*Ягуары здесь есть?

"--*Вряд ли, в таком редком лесу ягуара видно за кхене.
Кроме того, я их не слышал.

"--*А волки?
Здесь же должны быть какие-нибудь северные волки?
И ещё мне надоело вытаскивать из подмышек клещей.

"--*Если мне не изменяет память, их вытаскивал я.

"--*Волков?

"--*Митхэ, хватит уже.
Вылезай и пошли.

Ворча, женщина выбралась из-под меха и, зябко поёживаясь, начала поправлять ремни на доспехах.

"--*Здесь любая одежда кажется бумагой, "--- бурчала она.
"--- Насквозь продувает\ldotst

"--*Ляг, "--- велел ей Атрис, "--- я завяжу тебе плащ по-местному.
Да, вот так, только перевернись.

Митхэ послушно легла на живот.
Атрис замысловато сложил плащ у неё на спине, пропустил под животом подруги ремень и затянул его.

"--*Обереги тебя духи, "--- поблагодарила Митхэ, почувствовав, что стало теплее и проклятый мех уже не так раздражает её.

\section{Кролик и койот}

"--*Неужели уже утро? "--- пробормотал я с закрытыми глазами, вслушиваясь в непрекращающийся плеск воды.
По улицам, похоже, неслись целые реки.

"--*Уже день, "--- промурлыкала Чханэ и зарылась глубже в одеяло.
"--- Успокойся, сегодня мы можем отдыхать, первое число.
Я понежусь ещё немного и разожгу очаг "--- у меня остался целый паёк.

Я поцеловал подругу в губы раз, второй, и <<ещё немного>> растянулось до вечера.
Мои поцелуи, словно следы путника, привели к женским воротам.
Чханэ всхлипнула.

"--*Трай-трай-трай\ldotst\footnote
{Что-то вроде <<тихо-тихо>>, <<щекотно>>, <<колется>>. \authornote}
Тебе что, нравится?

\mulang{$0$}
{"--*Похоже на сливки, "--- поделился я впечатлениями.}
{``Tastes like cream,'' I shared my impressions.}
\mulang{$0$}
{"--- Сливки с сахаром.}
{``Cream with sugar.}
\mulang{$0$}
{И солью\ldotst}
{A little salty\dots ''}

"--*Дурак\ldotst "--- томно выдохнула подруга.
Она попыталась отвесить мне затрещину, но вышла только слабая ласка.

Когда сумерки дошли до густоты хлебного теста, мы всё-таки разожгли очаг и поужинали.

К ночи ливень превратился в мелкий моросящий дождь;
нужно было собираться в храм.
Предложение переночевать в доме кормилицы Чханэ категорически отвергла.

"--*Что ж, "--- усмехнулся я, поправляя в последний раз робу, "--- жилище мы нашли хорошее, верно?

"--*Трааа\ldotst
Что?

"--*Жилище.
Для нас с тобой, "--- пояснил я.

Чханэ смотрела на меня со смесью восхищения и страха.

"--*Лис, но мы не можем здесь жить.

"--*Почему?

"--*Что значит <<почему>>?
Ты жрец, я воин.
Наш дом "--- это храм, мы не должны быть хозяевами жилища, помнишь?

"--*Какая разница, "--- отмахнулся я.
"--- Здесь никто не живёт.

"--*Большая, Лис! "--- Чханэ разволновалась не на шутку.
\mulang{$0$}
{"--- У нас нет разрешения.}
{``We have no permission.}
Жилище чужое.
Это пахнет Насилием.

"--*Ты уже кутрап, тебе-то что, "--- шёпотом пошутил я, бросив взгляд на окна.
"--- И хватит о законах.
Мы можем здесь поселиться.

"--*И по какому праву? "--- скептически спросила Чханэ.

"--*По праву кролика и койота, "--- ответил я.
"--- Если нора койота пуста, кролик имеет полное право в ней жить.
Это право, я скажу тебе, древнее всех законов сели.

\section{Рябина}

Вскоре влюблённые уже шли под водяной пылью моросящего дождя, оглядывая неприветливые облетающие деревья.
Под ногами хлюпали жёлтые пятнистые листья, изредка попадались большие сопливые грибы.
Атрис то и дело смеялся над их формой:

"--*Смотри, Митхэ, этот похож на ухо.
Смотри, а этот похож на\ldotst

"--*Да, Атрис, я поняла, на что он похож.

"--*Тебя это не забавляет?

"--*Я хочу мужчину и есть.
Еда в форме мужского стержня вызывает у меня смешанные чувства.

"--*Я не знаю, можно ли их есть.
Проводник сказал, что бывают ядовитые.

"--*Пахнут аппетитно.

Вскоре парочка вышла на небольшую поляну.
Митхэ, сделав шаг в мокрую траву, замерла как вкопанная, наклонилась и провела рукой по маленькому золотистому колоску.
Вылущив зёрна, воительница задумчиво отправила их в рот.

"--*Надо же, рожь.
Культурная.
Здесь кто-то жил?

"--*Вполне возможно, "--- ответил Атрис и крякнул "--- сапог с чавканьем ушёл в незаметную мочажину.
"--- Местные племена часто меняют поля.
Земля очень бедная.

"--*Почему они не удобряют её?

"--*Ты будешь смеяться, но это против их верований.
Проводник от моего вопроса пришёл в священный трепет и сказал, что mama нельзя трогать, она может излечиться только сама.
Забавно, да?
Они считают землю mama.

"--*Значение слова в точности как у хака "--- женщина, выносившая и воспитавшая дитя?

"--*Да.
Я думаю, это очень древнее слово.

"--*Не так уж они и не правы.
Из земли растут травы и деревья, плодами питаемся все мы.
Разве нет?

"--*Почва рождается из мёртвых листьев и травы.
Кто кому ещё даритель.
Смотри лучше сюда.
Ты видела такие?

Атрис заворожённо разглядывал странные, ни на что не похожие деревья.
Митхэ подошла поближе\ldotst и тоже открыла рот.
Небольшие, с блестящей, словно медной корой, благородной формы ветви.
Собранные из множества зубчатых монеток листья, местами жёлтые, местами нежно-салатовые, а местами багряные, как кровь.
И ягоды.
Тяжёлые гроздья оранжевых и красных твёрдых ягод.

Митхэ осторожно подошла к волшебному деревцу и прикоснулась рукой к коре.
Странно.
Кора местных деревьев была склизкой, насквозь пропитанной сыростью, как старая мочалка.
Медный панцирь деревца отталкивал влагу, оставаясь солнечным и сухим.

Атрис осторожно оторвал ягодку и сунул в рот.
Поморщился и выплюнул.

"--*Не ядовитая, но жёсткая и горькая, "--- сообщил он.
"--- Или ещё не созрела, или это есть нельзя.

"--*Я впервые такие вижу, "--- призналась Митхэ.
"--- Даже в джунглях нет такой красоты, а тут, на севере\ldotse

"--*Давай посидим под этим деревцем, "--- предложил Атрис.

"--*Давай.

Атрис накрыл Митхэ своим плащом, и она благодарно улыбнулась ему.

Они молчали.
Дождь шептал траве что-то об ушедшем лете, о жизни и о двух маленьких людях, сидящих под чудесным, совершенно не северным деревцем.

"--*Помнишь, там, в доме хозяина, ты спросила, что мы тут делаем? "--- прервал молчание Атрис.

Митхэ кивнула.

"--*А вдруг мы здесь, чтобы увидеть эти деревья?

"--*Возможно.
Больше ничего красивого на Кристалле нет.
Северная глушь.

"--*Я имел в виду другое.
Что, если мы родились, чтобы \emph{это} сделать?

Митхэ посмотрела на любимого человека, потом на раскинувшееся над их головами деревце\ldotst и ничего не ответила.

\section{Хризантемы}

В течение трёх дней мы с Чханэ обустроили новый дом.
У нас появились циновки, ковёр, посуда, инструменты и даже плетёное кресло из тростника.
Очаг, к сожалению, починить так и не удалось, поэтому я разобрал половые доски и выложил кирпичами защищённое кострище, прикрыв его обычной жаровней.
Готовили мы осторожно, дымовод у нашего жилища был общий с соседями, а обитателей улицы Стриженого Кактуса не особо интересовали идущие по своим делам воин и жрец.

В храме нашлись симпатичные старые лотки и горшки для растений.
Они лежали в куче хлама, и никто их не трогал, вероятно, с самого моего рождения.
Я хотел взять их, но Чханэ решительно воспротивилась.

"--*Во-первых, у нас с тобой есть храмовые участки, "--- заявила она.
"--- Во-вторых, люди могут не обратить внимания на новую дверь, а вот цветущие хризантемы в горшочках расскажут нашу тайну всему Тхитрону.

Мне пришлось признать её правоту, несмотря на то, что без хризантем образ дома казался незаконченным.

Устойчивые пары внутри Храма были редким явлением.
Воины и жрецы, разумеется, имели любовников в городе;
многие Храмы, особенно из молодых, представляли собой одну большую поликулу.
Тем не менее в Храмах не разрешалось воспитывать детей.
О нас с Чханэ знали все, но смотрели на этот союз с лёгкой иронией "--- дескать, пусть балуется молодёжь.

Я ещё в первые декады в Храме заметил, что Кхохо и Ситрис держатся, как пара.
Я спросил об этом у Конфетки.

"--*Они знают друг друга очень давно, "--- объяснил тренер.
"--- Я столько не прожил, сколько они друг друга знают.
Однажды Ситрис вернулся из джунглей со сломанной рукой и разбитым носом.
Оказалось, что Кхохо <<пошутила>> и срезала его лиану, когда они лезли за созревшими акхкатрас.
Без <<железа>> и тросов, на одних пальцах.
Её совершенно не волновало, что он может покалечиться или даже разбиться.
Когда я спросил Ситриса, не обидело ли его это, он довольно грубо ответил, чтобы я не лез не в своё дело.
Иногда они дерутся до крови, но потом лежат вместе и над чем-то смеются.
Когда люди знают друг друга столько времени, их связывает что-то недоступное нам, посторонним.

"--*У них есть дети? "--- с чисто детской наивностью спросил я.
О парах я судил по кормильцам: раз мужчина и женщина вместе "--- должны быть дети.

"--*Есть, "--- ухмыляющийся Ситрис появился словно из-под земли с Кхохо в обнимку.
"--- Ты.

Кхохо громко захохотала, показав красивые белые зубы.
Не будь шрама и татуировок, она бы отличалась удивительной красотой "--- живые глаза золотисто-орехового цвета, тёмные веки, алые губы и гармоничное, благородное лицо с обветренными морщинками.
Её крепкое, слегка полное тело переливалось, словно ртуть.
Кхохо производила впечатление крестьянки-хохотушки, но позже я узнал, насколько обманчивым было это впечатление "--- в бою она превращалась в берсерка.
Ситрис рассказал, что Кхохо как-то на острие клина прорубилась сквозь строй копейщиков, а потом долго бегала от врагов, потому что все союзники остались по другую сторону строя.
Когда воительницу нашли в кустах, на ней не было живого места.

"--*У неё всегда были проблемы с работой в команде, "--- резюмировал Ситрис.
"--- Если уж быть честным, то и с головой тоже.

Круглосуточная для Чханэ шла полным ходом.
В общем и целом она мало отличалась от той, что устроил мне когда-то Конфетка.
Напасть на молодую воительницу могли и в уборной, и в душе, и за трапезой.

"--*Мальчики, я сегодня на ужин не приду.
После обхода пойду играть к ткачам, у них большая заруба в <<Пьянку>>, "--- сообщила Хитрам, жуя рисовую лепёшку.
"--- Кстати, а кто сегодня\ldotst

Ликхэ неожиданно бросила столовый нож в Чханэ, и рукоять ударила девушку точно в грудь.

"--*И ещё один штрафной, "--- радостно возвестила Ликхэ.

"--*Да что ж такое, "--- проворчала подруга.

Эрликх решил закончить урок и попытался дать Чханэ пощёчину.
Блок, удар локтем в грудь "--- и воин отлетел в угол.

"--*Прогресс определённо есть, Ликхэ, "--- невозмутимо заметил Ситрис и показал забинтованный палец.
"--- Сегодня ночью Чханэ великолепно среагировала на нападение.

"--*Если честно, я просто испугалась, "--- сообщила Чханэ и помогла слегка ошалевшему Эрликху сесть обратно за стол.
"--- А первое, что подвернулось под руку "--- столик.

"--*Кстати, о том самом столике, "--- сипло буркнул Эрликх, потирая грудь.
"--- Краснодеревщики сказали, что собирать эту мозаику в третий раз они не будут и сделают новый, без возможности его схватить и использовать как оружие.

"--*Да ты что! "--- расстроилась Хитрам. "--- Ушла эпоха\ldotst

"--*Ситрисовская голова обладает разрушительным действием на мебель, "--- пожаловалась Кхохо.

"--*Это мебель обладает разрушительным действием на мои пальцы! "--- поправил Ситрис.
"--- Но это ещё ничего.
Помню, я как-то во время круглосуточной бросил в тренера оцелотом.
Дело было ночью, а коту что-то потребовалось возле моей лежанки.
Тренеру пришлось накладывать швы на ухо.
Он сказал, что я буду великим воином, раз даже лесные звери становятся моими солдатами.

"--*Полно врать, Ситрис, "--- хмыкнула Кхохо.
"--- Какой тренер?
Какая круглосуточная?

"--*И у меня был тренер, Кхохо, не поверишь, "--- парировал Ситрис, но все заметили, что он смутился.
Прошлое помощника вождя было тайной за семью печатями.
Он редко мылся и никогда не снимал перед сном рубаху.
Однажды я случайно увидел, почему "--- спина Ситриса была сплошь испещрена следами от хлыста, которым погоняют скот.
Кхарас посоветовал мне не распускать по этому поводу язык.

В следующую декаду за обедом Чханэ превзошла саму себя: нож Ликхэ ещё не покинул ладони, когда из рукава подруги вылетел круглый тяжёлый камень.
Ликхэ, явно не ожидавшая от неповоротливого Скорпиона такого проворства, молча сидела и тёрла разбитую губу, пока остальные хохотали.
Кхарас признал, что круглосуточную для Чханэ пора заканчивать, но на всякий случай накинул пять дней "--- для закрепления навыков.

"--*Разумеется, это не значит, что через пять дней ты должна расслабиться, "--- строго заметил боевой вождь.

"--*Хай, мальчик, перестань, "--- примирительно сказала Хитрам.
"--- Девочка только отъедаться начала, щёчки круглее стали, а ты ей лишнее беспокойство внушаешь.

"--*И вообще Конфетка говорил, что круглосуточную надо заканчивать, когда послушник расслабится, "--- заметил Эрликх.
"--- Потому что смысл тренировки в том и есть, чтобы человек стал чувствовать себя уверенно.

"--*Конфетка? "--- удивилась Чханэ.
"--- Это кто?

Воины тепло улыбнулись и промолчали.

"--*Долгая история, "--- ответил я.

В остальном жизнь вернулась в прежнее русло.
Писем из Тхаммитра не приходило, и понемногу мы с Чханэ, жившие в лёгком напряжении, начали успокаиваться.
Может быть, Храм Тхаммитра счёл подругу и её мучителя пропавшими без вести.
Сложно ли потеряться в сельве, которая в любой тени может таить гибель, которая за считанные рассветы заплетает дороги и тела сетью жадных побегов?

\mulang{$0$}
{<<В этом опасность и прелесть сельвы, "--- говаривал старый охотник, Сиртху-лехэ.}
{``There is a danger and an amenity in the silva,'' the old hunter, S\r{\i}rtch\'{u}-l\={e}cho\`{e}, used to say.}
\mulang{$0$}
{"--- Там, где неумелый пропадёт, знающий скроется>>.}
{``Where an inept one will be lost, a skilled one will be sheltered.''}

\chapter*{Интерлюдия III. Ученица Повелителя ветров}

\addcontentsline{toc}{chapter}{Интерлюдия III. Ученица Повелителя ветров}

\textbf{Сказка сели}

Это было в незапамятные времена, когда дороги прокладывали по облакам, мосты строили через моря, а люди слыхом не слыхивали о требовавших крови жестоких богах.
То была эпоха духов, ведавших землёй, водой и девятью ветрами\footnote
{По восьми сторонам света плюс <<властелины пустошей>> "--- торнадо. \authornote}.

В далёкой северной деревушке, в доме старого Кита родился беловолосый ребёнок.
Той ночью, когда родильница ощутила первые схватки, поднялся ветер.
Вековые деревья трепетали на этом ветру, словно распущенные косы, крыши многих жилищ перелётными птицами отправились на юг.
Едва младенец испустил первый крик, стихия утихла.

"--*Неспроста это, "--- толковали старики за чаем, "--- не иначе как сам Повелитель ветров нанёс дому Кита визит.

Но жизнь шла своим чередом.
Вместо сломанных веток деревья вырастили новые, ещё краше, и сорванные крыши были для сильных мужей лишь поводом для ремонта.
Солнце всё так же садилось на западе и всходило на востоке, люди сеяли и жали кукурузу, в очагах пеклись лепёшки, а пёстрые курочки высиживали цыплят.
Младенец, которого назвали Веснушкой, стал девочкой, девочка превратилась в молодую женщину, и о давних стариковских пересудах никто не вспоминал.

"--*Веснушка, ты сегодня хорошо поработала!
Иди кушать! "--- этими словами кормилица обычно звала девушку домой.

В тот вечер Веснушка не откликнулась на зов.
Она лежала за кустом черноягоды, украшенным тяжёлыми обсидиановыми серьгами, и её белые волосы раскинулись по старому шерстяному одеялу.
Ветер нежно холодил обнажённое разгорячённое тело, девушка прижималась к любимому человеку и чувствовала себя самой счастливой на свете.
В деревне зажглись и потухли фонари, а влюблённые тихо беседовали и смеялись.
Гроздья ягод служили им пищей, а хрустальный родничок утолял их жажду.

"--*Не случилось ли чего, лехэ? "--- обеспокоенно спрашивала кормилица, заламывая руки.

"--*Любовь случилась! "--- смеялся старик Кит.
"--- У старика Веха Ключик тоже пропал.
Не иначе скоро в деревне выстроят новое жилище.

Но вот беседа утихла, и любовников начало клонить ко сну.
Первым заснул Ключик, прижав подругу к своей широкой груди.
Звёзды шли своим кругом, в непролазных кустах стрекотали сверчки, и Веснушка погрузилась в сладкую дрёму.

Вдруг, откуда ни возьмись, в ночной темноте закружился вихрь.
Застонали деревья, притихли птицы, понеслись тревожными синими искрами индиго-светляки.
Испуганная девушка, прижавшись к любимому, смотрела, как к ней идёт древний седой старец, закутанный в рваный чёрный плащ.
Его очи были подобны звёздам, его белые волосы развевались, словно высокие ледяные облака, и такой же белой была его длинная борода.

"--*Мир тебе, девушка, "--- сказал старец.
"--- Я "--- Повелитель ветров, и ты была избрана мне в подмастерья ещё до рождения.
Пойдём, и я научу тебя премудростям проводника вихрей.

"--*Мир тебе, Повелитель ветров, "--- прошептала Веснушка, прижимаясь к Ключику.
"--- Для меня честь говорить с великим духом, однако я не могу быть твоим подмастерьем.
У меня есть любимый человек, дом и очаг, и я обязана им.

Загорелись гневом глаза старца, затрепетали под яростным порывом облачные космы.

"--*Я обрушу вечные ураганы на твою деревню.
Ни одно жилище, ни одна мельня не смогут стоять прямо, если ты не пойдёшь со мной, "--- прогремел он, и трещали в его голосе ломающиеся деревья.

Что оставалось делать девушке?
Надела Веснушка штаны, подцепила рубаху на пояс, поцеловала в последний раз любимого и ушла с Повелителем ветров в ночь.

Догорела ли свеча, догорело ли солнце, но донесли облачные олени Повелителя ветров и Веснушку в поднебесные башни, на вершину высочайшей из гор.
Пусты и неприютны были их стены "--- грубый камень ласкали лучи светил, бриз врывался в окна и гулял по огромным залам, но не было в залах ни мебели, ни иных вещей, что придают домашний уют жилищу.

"--*Это твой дом, "--- сказал ей Повелитель ветров.
"--- У меня нет одежд и пищи, но здесь ты не будешь знать ни голода, ни холода, ни болезней.

Так началось обучение Веснушки.
Повелитель ветров рассказывал ей всё, что знал, но чересчур своеволен и гневлив был этот учитель.
Порой в залах царил штиль, иногда же, в плохие дни, молнии и ураганы плясали на вершине высочайшей из гор, ворочая огромные камни, словно куриные перья.

Девушка быстро постигала науку.
Шли годы, рубаха на её груди, спине и плечах истлевала, словно одеяние мертвеца, но пальцы и прекрасные белые волосы ткали ветра всё с большим мастерством.
Дело нравилось Веснушке, но нет-нет она да вздыхала, вспоминая по вечерам оставленный дом и любимого.
Смирились ли люди с её уходом? Простил ли её Ключик за то, что она бросила его посреди ночи?

Наконец пришла пора испытаний.
Запрягла Веснушка облачных оленей, и помчали её игривые звери по призрачным струнам.

Вот внизу показалась деревушка, страдающая от засухи.
Река обмелела, и кукурузные стебли горели, не успев дать початка.
Люди ходили, словно тени, и с тяжёлым чувством переговаривались о предстоящих голодных дождях.

Веснушка посмотрела на это, и печалью наполнилось её сердце.
Умелые пальцы девушки соткали ветер "--- и на исходе дня пролился над горящими полями животворный дождь.

Гневными словами встретил ученицу Повелитель ветров:

"--*Что ты наделала! "--- бушевал он, и в его голосе его разбивался о скалы девятый вал.
"--- Ты напоила одну деревню, но оставила целую страну без урожая!

Махнул Повелитель ветров рукой "--- и Веснушка увидела страшное наводнение, постигшее далёкие земли.
Полей здесь не было "--- всё скрылось под глинистой рыжей водой.

Опечалилась девушка, но продолжила свои занятия.
И снова пришла пора испытаний.
Запрягла Веснушка облачных оленей, и помчали её игривые звери по призрачным струнам.

Вот внизу показалась несметная тьма кораблей.
Великий завоеватель вёл их против маленького флота, защищавшего родной город.

Веснушка посмотрела на это, и печалью наполнилось её сердце.
Умелые пальцы девушки соткали ветер "--- и запылали корабли захватчиков, и зазвенели паруса защитников, и победили они в той битве.
Захватчики отправились в свиту к Сестре Воде, а их предводитель был пленён и заклеймен, подобно последнему насильнику.

Гневными словами встретил ученицу Повелитель ветров:

"--*Что ты наделала! "--- бушевал он, и в его голосе его ревел пустынный самум.
"--- Он мог объединить земли от рассвета до заката, принеся в раздираемые междоусобицами земли мир.
А теперь у живых не будет ни мира, ни урожая!

Махнул Повелитель ветров рукой "--- и Веснушка увидела засуху и усобицы, поразившие земли защитников.
Кукуруза не взошла, и бывшие герои бродили по лесным тропам, перебиваясь разбоем и грабежами.

Опечалилась девушка, но продолжила свои занятия.
Больше она не запрягала облачных оленей "--- она сидела в башне и читала книги.

Книги поведали Веснушке, что она не могла даже качнуть цветок орхидеи, не принеся несчастий.
Чем больше девушка читала, тем меньше ей хотелось применять своё искусство.

Догорела ли свеча, догорело ли солнце, но однажды ночью родилась мысль, подобная свету в первородной тьме.
Запрягла Веснушка облачных оленей, и помчали её игривые звери по призрачным струнам.

Но девушка более не ткала ветра.
Она летала над миром и смотрела на вечную беседу подвижного и неподвижного, земли, воды, солнца и воздуха.
У гор нити ветров всегда сплетались в кольца.
У рек и морей они напитывались влагой.
Леса расчёсывали ветер, словно гребни, а по степям и пустыням вихри гуляли по своим, ведомым лишь им законам.

Девушка начала ткать свой собственный холст.
Она дни и ночи напролёт сидела в башне, смотрела на карты, листала огромные пергаментные страницы, размышляла над словами давно мёртвых, забытых даже духами языков.
Повелитель ветров сердился на Веснушку: молнии плясали по окрестным горам, пыльные ураганы истачивали древние скалы.
Но девушка не обращала на них внимания.
И в один прекрасный день, пока Повелитель ветров дремал, Веснушка снова запрягла облачных оленей.

Она встала посреди пустыни и стала ждать.
Всё её существо трепетало от осознания того, что она собиралась сделать.
Прошёл день, прошёл другой, третий "--- и на рассвете четвёртого дня Веснушка, затаив дыхание, махнула рукой.

Повелитель ветров не встретил свою ученицу.
На высочайшей из гор царил мёртвый штиль.
Девушка в страхе вихрем носилась по огромным залам, просторным коридорам и потайным лестницам, но они были пусты.
Наконец она обнаружила старца на маленьком балконе, с которого весь мир был виден, как на ладони.
Повелитель ветров сидел и дрожал, закутавшись в рваный плащ и совершенно потеряв свой грозный облик.

"--*Что ты наделала, "--- шептал старец, глядя в пустоту померкшими глазами,
"--- что ты наделала\ldotst

"--*Я развесила грузила на ткацком станке ветров, и теперь ветра будут двигаться по своим дорогам сами, "--- сказала девушка.
"--- Люди, наблюдая за облаками, смогут предсказать их путь и решать, когда время сеять, а когда время уходить в другие земли.
Больше я тебе не нужна.

Старец взвыл от бессильного гнева и затопал ногами.

\mulang{$0$}
{"--*Когда-то давно, "--- закричал он наконец, "--- камни и щепки предсказали мне, что ты, рождённая в ночь Согхо 8, будешь величайшей из Повелителей ветров.}
{``Ages ago,'' he shouted at last, ``stones and sticks predicted you, born at the night of S\v{o}gh\={o} 8, as the greatest of Windlords.}
\mulang{$0$}
{Но ты осквернила наше ремесло.}
{But you have violated our handicraft.}
\mulang{$0$}
{Ты обесчестила меня и всех моих предшественников.}
{You have defiled the honor of mine and all my forebears.}
\mulang{$0$}
{Уходи из моего замка и никогда не возвращайся.}
{Go away from my castle and never return.''}

Повелитель ветров махнул рукой "--- и девушка оказалась возле родной деревни.

<<Сколько меня не было?>> "--- со страхом размышляла Веснушка.
Ей казалось, что прошли века, а то и тысячелетия.
Она видела, как сменяются эпохи, как возвышаются и низвергаются города.
Но сонная деревня по-прежнему мирно стояла вдали, звёзды шли своим кругом, а в непролазных кустах пели сверчки.

Вдруг Веснушка увидела черноягодный куст.
Сомнений нет "--- это тот самый, который дал приют ей и её давно покинутому любовнику.
Как и тогда, он был усеян тяжёлыми гроздьями.
Девушка ахнула и подбежала поближе.
Ключик тихо сопел носом, завернувшись в одеяло, словно она оставила его мгновение назад.

До нас дошло, что у Веснушки было много потомков.
Никто из них не унаследовал прекрасные белые волосы, но все были наделены удивительным даром понимать и предсказывать путь ветров.
Впрочем, было ли это даром?
Доподлинно не известно ни духам, ни бродящим от звезды к звезде фонарщикам, ни небесному пахарю.
Наследники Веснушки и в хмелю, и в объятиях прекраснейших женщин молчали с многозначительной улыбкой, едва речь заходила об этом.

Это было так, и пусть станут пищей ягуара те, кто рассказывает иначе!

\chapter{[-] Путь жреца}

\section{[-] Посвящение}

\spacing

Я был единственным послушником Верхнего Этажа.
Как-то я спросил, почему жрецы не обучают других парней.
Трукхвал сказал, что, когда я стану жрецом, он мне скажет.

И вот этот день пришёл.
Однако всё оказалось совсем не тем, что я ожидал.

Мне говорили, что послушник должен пройти испытание перед тем, как войти в Храм.
Когда меня подняли утром и повели в зал, я уже представлял собрание Совета, коварные вопросы и прочие испытания, и был немало разочарован, когда меня усадили за стол кушать.

В конце трапезы, перед традиционной чашей отвара, Первый жрец торжественно сказал, что выслушает мою клятву.

Слова клятвы я уже успел заучить наизусть.
Её цитировали все "--- по поводу и без. Я прочистил горло и, встав из-за стола, произнёс:

\begin{quote}
<<Ни действием, ни бездействием я не наврежу книге, написанной не мной, даже если в ней не будет ни слова истины, и не позволю никому сделать подобного;
я не испорчу ни текста, ни художеств, и гулять моему перу лишь по её полям.\\~\\
Я не утаю знания ни из жалости, ни ради защиты, ни ради выгоды.
Я не оставлю без внимания ни ложь, ни заблуждение, и скажу во всеуслышание то, что считаю истиной.\\~\\
Я не потушу ни искры дарования, данного ребёнку, но разожгу их в пламя;
я расскажу ему то, что знаю и считаю истинным, и не оставлю без ответа ни один его вопрос в тот момент, когда он был задан.\\~\\
Я принесу жертву богам и в скорби, и в болезни, и при смерти;
я отдам людям эту ношу лишь тогда, когда не найдётся жреца, способного держать нож.\\~\\
Я приложу все силы, чтобы облегчить боль;
я не наврежу человеку ни моими чувствами, ни моим невежеством;
\mulang{$0$}
{в жизни и смерти я буду щитом, и ни мгновения "--- плетью>>.}
{Dead or alive I will be a shield, and never a lash.''}
\end{quote}

Жрецы заулыбались.
Мне показалось, что они с трудом сдерживают смех.
Кхатрим пододвинул мне чашу.

"--*Добро пожаловать в Храм, Ликхмас.

"--*Трааа\ldotst "--- я на мгновение почувствовал себя обескураженным.
"--- Это всё?
Я думал, вы просто хотели узнать, заучил ли я клятву.

"--*И мы узнали, "--- улыбнулся Первый.
"--- теперь ты "--- жрец.

"--*Но я думал, что\ldotst

Мои слова прервал громкий хохот и звон посуды.
Жрецы больше сдерживаться не могли.

"--*Ты думал о тяжких испытаниях, торжественных ритуалах и прочей ерунде? "--- спросил Трукхвал.
"--- Все так думали, не волнуйся.
Испытание ты давно прошёл.
А клятва, как ни крути, лишь слова.
Слова не имеют смысла, если нет того, кто может их понять.
Ты хорошо это усвоил, работая в библиотеке.

"--*Все так думали, Лис, не ты первый, не ты последний, "--- похлопал меня по плечу Кхатрим.
"--- Это традиционная забава Верхнего этажа "--- смотреть на озадаченную физиономию новоиспечённого жреца.
Вернее, на физиономию того, кто уже был жрецом как минимум три дождя, но ещё об этом не догадывался.

Я вдруг понял.

"--*Значит, испытанием было\ldotst

"--*Испытанием было всё, что произошло за время обучения, "--- подтвердил Трукхвал.
"--- Каждый твой шаг был испытанием.
И ты его выдержал.
А клятву мы просто так спросили, для формы.

Я сел за стол и взял чашу.
Жрецы всё ещё смотрели на меня и ухмылялись.

"--*Что-то я не чувствую особой разницы до и после клятвы, "--- признался я.
В этот момент Верхний этаж едва не понёс тяжёлые потери "--- Кхатрим от хохота серьёзно подавился конфетой.
Впрочем, с этим непредвиденным испытанием я справился превосходно "--- конфета была успешно вытащена и съедена под отвар.

\section{[-] Молодой учитель}

\spacing

"--*Ликхмас, теперь ты жрец, "--- торжественно начал Кхатрим, "--- а это значит, что тебе следует пройти через самое суровое испытание в твоей жизни.

"--*Но я же уже работал в госпитале, приносил жертвы, "--- удивился я.
"--- Да и клятву у меня уже спросили\ldotst

"--*Не имеет значения, "--- прервал меня жрец.
"--- Через это испытание проходят все молодые обитатели Верхнего этажа.

Я промолчал.
Чутьё подсказывало мне, что врач не шутит, но\ldotst

"--*Школа, Ликхмас, "--- в священном ужасе прошептал Кхатрим.
"--- Ты будешь учить детей в школе.

\mulang{$0$}
{Я рассмеялся.}
{I laughed.}

\mulang{$0$}
{"--*Ты меня почти разыграл!}
{``You almost pranked me!''}

\mulang{$0$}
{"--*Ты не боишься? "--- удивился жрец.}
{``Are you not afraid?'' the priest looked surprised.}
\mulang{$0$}
{"--- Это самое ужасное, что было в моей жизни\ldotst}
{``It was the most dreadful thing in my life\dots}
В таком случае пойдём, я расскажу тебе, как готовить уроки и как выбирать себе ученика.
Жрецу нужен преемник.

"--*Уже учеников?

"--*А ты как думал?
Между учителем и учеником обязательно должны быть дружеские отношения!
Дружба не вырастает за день на голой земле!

"--*А где учитель\ldotsq

"--*Митликх уехал в Кахрахан "--- поклониться тотему Удивлённого Лю, отдохнуть и поговорить с кем-то из местного жречества.
\mulang{$0$}
{Ещё он упоминал как-то, что там живёт его старинный любовник.}
{Also he mentioned his man, lover of yore, is living there.}
\mulang{$0$}
{Ещё он сказал, что полностью тебе доверяет и знает, что ты справишься один.}
{Finally he said that you're completely trusted and competent enough to manage alone.''}

\mulang{$0$}
{"--*Трааа\ldotst}
{``Tr\={a}\"{a}\dots''}

\mulang{$0$}
{"--*В общем, вряд ли Митликх вернётся в ближайшие восемь декад, "--- смущённо закончил Кхатрим.}
{``Well, anyway, M\={\i}tl\={\i}kch is hardly going to return in the next eight decades,'' Kch\r{a}tr\"{\i}m got embarrassed.}

"--*Главное, чтобы он не насовсем, "--- заметил я.
Перспектива провести в школе остаток жизни почему-то вызвала во мне смутную тревогу.

"--*Вот тут я с тобой полностью соглашусь, "--- совершенно серьёзно ответил Кхатрим.
\mulang{$0$}
{"--- Я как-то тоже уехал из Сотрона на отдых.}
{``I left S\~{o}tr\`{o}n for holidays once.}
\mulang{$0$}
{Угадай, куда.}
{Guess where I went.''}

\section{[-] Телесные наказания (отрывок)}

\spacing

Ещё совсем недавно детей могли побить за шалости.
Но около двадцати дождей назад в Ихслантхаре произошло нечто примечательное.
Наказание выпало девочке по имени Ситхэ, бойкой большеглазой заводиле.
Девочка, недолго думая, заявила учителю, что признаёт вину, но не примет наказание, так как считает его оскорбительным.
Учитель в шутку предложил ей устроить Дело Перекрёстка.
Ситхэ молча притащила из арсенала фалангу и щит, за неё вступились другие дети, и вскоре шутка вылилась в ритуальный бой между школьниками и воинами Храма.

"--*Храм "--- опора традиций, "--- заявил боевой вождь.
"--- Вы "--- не взрослые, чтобы устраивать Перекрёсток.

"--*Если нам отказывают в праве на Перекрёсток, поднять оружие "--- наша священная обязанность, "--- парировала Ситхэ строками закона.

"--*Да будет так, "--- ответил вождь.

Ситхэ с неожиданной сноровкой выстроила детей в боевой порядок и скомандовала атаку.

Домой дети вернулись в синяках, ушибах и с разбитыми носами, ещё больше уверенные в своей правоте.
Самой <<предводительнице>> потребовалась помощь жреца.
Осложнилось дело тем, что произошло всё на Змею 8, накануне праздника Обнимающего Сита.
Кормильцы, увидев своих чад в таком виде, созвали уже взрослый Перекрёсток (впрочем, вместо оружия все принесли мётлы, веники и букеты цветов), и традиция телесных наказаний была отменена.

Все были рады.
Дети радовались, что добились своего, кормильцы радовались за детей, но больше всех были счастливы, как ни странно, пожилые учителя, которым <<упражнения>> доставляли немало хлопот.
Многие из них по старинке грозили ученикам палками, но спустя десять дождей телесные наказания упразднили во всех городах Севера и Запада.

\section{[-] Школа}

\epigraph
{Стрелка компаса точна, но её концы раскрашивает мастер.
А уж нулевой меридиан и вовсе там, где вздумалось чихнуть королевскому географу!}
{Клаудиу Семито Фризский.
<<Воспоминания о великом тёзке>>}

\spacing

"--*А что здесь, учитель? "--- спросил я у Хонхо-лехэ, указав на южные части материков.

"--*Хаммм, "--- задумался учитель.
"--- Ликхмас из дома Люм, я хотел бы ответить на твой вопрос, но не могу.
Эти земли называют Краем.
Известно, что они плохо пригодны для жизни "--- болотистая местность чередуется с бесплодными полупустынями и горами.
Известно также, что туда очень сложно добраться.

Учитель достал указку и провёл её концом по узкой полоске воды.

"--*Могильный пролив непроходим.
В будущем кто-то из вас может стать мореплавателем и захотеть испытать свои силы, но молю всех лесных духов, чтобы они удержали вас от этого самоубийства.
Толчки в проливе, по слухам, раскалывают мачту на две ровные половинки, отражённый водой солнечный свет выжигает глаза, а морские звери могут прогрызть днище корабля, не дожидаясь, пока своё дело сделают цунами.
Могильный пролив оправлен в скалы почти на всём известном протяжении, и едва ли вы найдёте хорошее место для причала.
Таким образом, этот путь в Край для нас закрыт, несмотря на кажущуюся его очевидность.

Хонхо провёл концом указки по хвосту Кита.

"--*Эти берега, как я уже говорил, принадлежат народу трами, родичам идолов.
У нас с ними никогда не было тёплых отношений, трами вместе с ноа ходили на нас в походы, но их можно считать хорошими торговыми партнёрами.
Трами сообщили, что вот это море "--- Голубое Зеркало "--- охраняется могущественнейшим из племён дельфинов.
Голубое Зеркало можно даже считать их государством, потому что некоторые трами видели у берегов нечто, напоминающее полуводные поселения.
В отличие от своих северных сородичей, эти дельфины далеко не дружелюбны.
Любые суда они берут на сопровождение и для начала вежливо намекают, что лучше плыть обратно.
О судьбе тех, кто не подчинился, ничего не известно, как не известно ничего о тех, кто пытался установить с ними контакт.

"--*Значит, этим путём в Край могут пойти дипломаты? "--- спросил кто-то из детей.

"--*Безусловно, Секхар из дома Тра.
Если кто-то из вас сможет договориться с этим воинственным дельфиньим племенем "--- дорога в Край, по крайней мере в его восточные земли, называемые Тысячеречьем, будет открыта.
Дипломат-путешественник способен дать своему народу куда больше, чем самый великий воин.

"--*Учитель, "--- поднял руку сидевший рядом Столбик, "--- я вижу на карте названия "--- Суболичье, Тысячеречье, Каменное море, да и материк Кит, как вы сказали, был назван так из-за сходства очертаний.
Кто же узнал, как выглядят материки?
Кто дал этим странам названия?

"--*Хороший вопрос, Ликхсар из дома Митр.
Карта, которую вы видите, является копией карты из библиотечной книги.
Как вы знаете, многие книги были написаны Древними, которые пришли из звёздной страны Тхидэ.
Вероятно, очертания материков они увидели с большой высоты.

Дети восхищённо забормотали, и Хонхо позвонил в колокольчик, призывая к тишине.

"--*Но мы же не можем знать, что это правда, верно? "--- настаивал Столбик.

Хонхо улыбнулся.

"--*Я отвечу на вопрос так.
Карты из \emph{этой} книги "--- карты известного нам мира "--- весьма точны.
Кроме того, экспедиции, о которых я расскажу сейчас, подтвердили некоторые данные касательно Края.
Именно поэтому мы считаем, что автору можно доверять.

Указка пробежалась по западному побережью Короны.

"--*Этот путь, на мой взгляд, один из самых простых.
Путь искусного мореплавателя.
Здесь множество опасностей "--- рифы, штормы, палящий зной, водовороты и цунами.
Именно здесь проходили в Край те, кто привёз истории о его необитаемости и непригодности к земледелию.
Но это лишь западное побережье.
Что творится в прочих местах, сказать нельзя.

"--*А по суше?

"--*По суше путь закрыт, "--- отрезал Хонхо-лехэ.
"--- Через смертные каменные пустоши не проходил никто.

"--*Учитель, "--- подняла руку девочка, "--- а были ли те, кто уходил в Край насовсем?
Я имею в виду\ldotst разумеется, мы не знаем, добрались они или нет, но неплохо было бы знать хоть примерно, кто там\ldotst может быть.

Девочка стушевалась и опустила руку.
Хонхо-лехэ улыбнулся.

"--*Я понял твой вопрос, Согхо из дома Лам, "--- сказал учитель.
"--- Сейчас посмотрим.
Ликхмас из дома Люм, Ломающий Храмы, встань.

Дети засмеялись.
Это прозвище, придуманное Хонхо-лехэ, лишь по случайности не приклеилось ко мне на улице.
Я остался сидеть.

"--*Да встань же, "--- улыбнулся учитель.
"--- Я хочу, чтобы ты принёс из библиотеки книгу.
<<Хроники дорог и ветров>>, третий том, кажется\ldotst да, третий.

Я отправился в библиотеку.
Трукхвал, как обычно, встретил меня тепло.
Я еле отказался от сладостей и отвара, сославшись на то, что меня ждут на уроке.

Хонхо принял из моих рук тяжёлую книгу и открыл её на середине.
Полистал, нахмурился.

"--*Вот, "--- сказал он наконец.
"--- В Край было четыре крупных экспедиции, одна из которых, Коричный флот, вернулась.
Первыми ушли, что неудивительно, ноа.
Они не стали испытывать судьбу ни в Могильном проливе, ни в Тихом океане, а двигались уже известным вам западным путём, преодолев по Торговому Потоку земли сели, тенку и ркхве-хор.
О дальнейшей их судьбе ничего не известно.

Хонхо перелистнул страницу.

"--*Последнее колено Синих Травников.
Сели и даже ркхве-хор, что удивительно, предлагали им приют "--- травники хороши как союзники, и Бродячий Народ желал воссоединения с сородичами.
Но Синие были чересчур подавлены собственным поражением в войне с Живодёром, да и Бродячих, как и многие из Красных, Синие презирали.
Они отправились через бассейн Реки Кувшинок, потому что, как и все травники, недолюбливали моря.
В любом случае им пришлось бы идти под защитой моря "--- согласно данным Коричного флота, за дельтой Реки Кувшинок лежат ещё как минимум пятьсот кхене смертных пустошей.
Тем не менее очевидцы утверждают, что травники пошли сушей.
О судьбе Синего колена ничего не известно.

Учитель задумался и почесал подбородок.

"--*Хай, вот.
Встреча у Мыса-лезвие, я ещё буду рассказывать вам о ней на истории.
Знаменитый пират-трами по имени Молочная-Роса-Движения, зная о дельфинах Голубого Зеркала и поняв, что флот Чистого союза отсек его кораблям путь на север, отправился на юго-восток.
По словам его противников, ветер был необычайно благоприятным, паруса пиратских кораблей раздувались, словно лёгкие бегущего.
Если его флот не погиб в море, то непременно должен был выплыть к берегам Края.

\section{[-] Пахарь}

\spacing

"--*Все из вас слышали легенду о небесном пахаре, "--- сказал я.
"--- Так вот, сейчас я расскажу вам истину о кометах.
На самом деле это камни, которые падают со звёзд и раскаляются, падая на землю.

"--*Так значит, нас обманывали? "--- недоумённо спросила девочка.

"--*Нет, "--- сказал я.
"--- Это не совсем обман.

"--*Тогда зачем нужны легенды, учитель? "--- продолжала допытываться она.
"--- Зачем придумывать глупые истории, когда истина так прекрасна?

Я улыбнулся.

"--*Как тебя зовут?

"--*Митхэ ар'Тра, "--- ответила девочка.
"--- Масло-Взбитое-Взглядом.

"--*Это тайна воспитания, Митхэ, "--- ответил я.
"--- Легенды вплетают человека в любое явление, чтобы такие умные дети, как ты, не забывали, на что способен человек.
Сказки дают человеку невероятную силу, чтобы вы мечтали и фантазировали, как претворить эту силу в жизнь.
Так мы движемся вперёд.

\section{[-] Жрица}

"--*Ты с ума сошёл! "--- буркнул Кхатрим, услышав мою идею.

"--*И всё же\ldotst

"--*Лис, это традиция, "--- Кхатрим сел и прикрыл лицо рукой.
"--- Думаешь, мне не попадались умные девочки, которых я желал бы взять в ученики?
Попадались.
Сказать по правде, умных девочек на моей памяти было даже больше, чем умных парней "--- женщины быстрее взрослеют.
Но традиция есть традиция.

"--*Почему нельзя её изменить?

"--*Потому что живот никуда не денешь.

"--*Ликхэ недавно диверсантов взяла беременной, и ничего, родила в срок без проблем.

"--*Забудь про воинов, это другое.
На Западе вообще есть обычай, что командовать обороной должна беременная воительница "--- враг на полёт стрелы к городу не подойдёт.
Потому что если какая угроза родному дому "--- у них голова работает за десятерых.
А как будет выглядеть беременная жрица?
Жертвоприношение каждые несколько дней, а адепт Верхнего этажа с животом и под гормонами.
А если война или катаклизм?
Во время болотной лихорадки беременных и близко к карантинной зоне подпускать нельзя.

"--*Кхатрим, женщины не всё время ходят беременными.

"--*Однако беременеют в самый неподходящий момент.
Проверено.

"--*Глупость какая-то.
Мужчина может заболеть, женщина может забеременеть.
Есть, в конце концов, белое жречество, почему нельзя\ldotst

"--*Нет никакого белого жречества, "--- перебил меня Кхатрим.
"--- Запомни это раз и навсегда.
Это между собой мы так "--- белые, красные.
Если долг позовёт, твой цвет не спросит ни народ, ни Безумные.

"--*И всё же я хочу попробовать.

"--*Храм её не примет, "--- отрезал Кхатрим.
"--- Хочешь "--- пожалуйста.
Девочка будет благодарна тебе за образование, но образование "--- это всё, что ты сможешь ей дать.

"--*Возможно, стоит обсудить этот вопрос с остальными.
Ведь если в народе будут люди с образованием жреца, это может\ldotst

"--*И эта идея не нова.
Пытались, и последний раз, как мне думается, не ранее чем сорок дождей назад.
Результат один "--- нам переставали верить.

"--*Но почему?

"--*Если ты, жрец, сделаешь что-то вредное для народа и Храма, тебя можно просто выставить.
Тогда у народа формируется впечатление, что Храм блюдёт чистоту своих рядов.
А что делать с тем, кто де-юре крестьянин, но де-факто является ещё и жрецом?
В глазах народа он и будет жрецом "--- ведь к нему придут и заболевшие соседи, да и роженицу проще тащить в соседний дом, чем в центр города.
Однако если что-то вредное сотворит этот человек, мы ничего не сможем с этим сделать.
Статус жреца "--- и при этом никакой ответственности за этот статус.
Это подорвёт честь всего жречества.

"--*Можно обвинить этого человека в Насилии.

"--*Если Насилие имело место, что редкость.
А если он, к примеру, пишет инструкции, которые отличаются от храмовых в худшую сторону?
Даже если инструкция работает у одного из шестнадцати, эти свитки потом гуляют в народе, и очень сложно доказать что-либо крестьянам.
У купца по статистике падение урожая, а у крестьян один ответ "--- <<у меня всё работает>>.
Можно, конечно, изымать и сжигать, как это делают тенку, но\ldotst

"--*<< \dots ни действием, ни бездействием я не наврежу книге>>, "--- закончил я.

"--*Именно, Лис.
Это уже настоящее Насилие, без каких-либо сомнений.
Теперь понял?

"--*Понял.
Но вреда от обучения как такового быть не может.

"--*Это правда, "--- признал Кхатрим.
"--- Ты волен делиться знаниями, и я даже приветствую это.
Будет хорошим опытом для тебя как для учителя, да и в целом.
Но право принимать в Храм останется за Храмом.
Ты не первый, вопрос насчёт женщин не раз обсуждался и не стоит двух часов сна.
Для женщин Верхний этаж закрыт, и те, кто сменил пол, тоже обычно уходят.
Традиция есть традиция.
Если ты найдёшь способ убить Безумного и прекратить войны навсегда, можешь обучить делу хоть всё женское население города, "--- жрец характерным жестом потрогал пальцем язык.
"--- А пока "--- извини, мне нужно работать.

Я понурился.
Кхатрим задумался и снова вздохнул.

"--*Она тебе нравится?

"--*Она задаёт вопросы, на которые у меня в детстве не хватило ума, "--- признался я.

"--*Это серьёзно.
Что ж, я поставлю прочих жрецов перед фактом.
Мы будем делать вид, что так и должно быть, а ты не будешь подавать девочке ложных надежд, правда?
Благодарностей не нужно.

\section{[@] Гармония}

Сегодня при поддержке Безымянного мы установили ещё тридцать золотодобывающих установок.
Работа идёт полным ходом, механики не успевают собирать машины.
Нам не хватает стали и пластика, но Кусочек-Уха-Кабана придумал совершенно удивительную замену "--- древесина.
С разрешения Безымянного мы аккуратно, стараясь не повредить экосистему, срезали некоторое количество старых деревьев, пропитали древесину специальными растворами и пустили в обработку.
Листик весь день ходила грустная.

"--*Это варварство, "--- заявила она.
"--- Да и зачем резать деревья, если можно было просто сказать им, как расти?

"--*Листик, мы не успеем написать структурные хромосомы для всех видов конструкций, "--- грустно сказал Мак.
"--- И даже если успеем "--- сколько лет они будут расти?
Безымянный разрешил "--- уже хорошо.
Я бы на его месте отказался.

Товарищи уже успели привыкнуть к частому посещению Стального Дракона Безымянным.
Никто больше не шарахался, увидев в библиотеке мирно смотрящую тексты или фильмы фациограмму.
Время от времени я видел, что Безымянный задавал вопросы товарищам.
Те отвечали ему "--- вначале неохотно, потом посвободнее.
Культуролог Гладит-Зелёную-Кошку общалась с ним душевнее других.
Видимо, она тоже почувствовала эту странную детскость, исходящую от демиурга.
Да и интересно ей как культурологу поговорить с хоргетом-девиантом.

"--*Что такое музыка? "--- спросил сегодня Безымянный у меня.
Мы с Баночкой, Кошкой и Заяц в это время обедали в моей каюте.

Гладит-Зелёную-Кошку подавилась пищевым концентратом.

"--*Я же тебе показывала! "--- с лёгким укором сказала она.

"--*Показывала, "--- виновато согласился Безымянный.
"--- Но я так ничего и не понял.
Зачем складывать гармонические звуковые колебания разной частоты?
В чём смысл шумов?

"--*Они вызывают у сапиентов положительные эмоции, "--- ответил Баночка.

"--*Не обязательно положительные, "--- возразила Заяц.
"--- Но эмоции "--- это важное условие восприятия музыки.

"--*Сапиенты начинают испускать эманации? "--- уточнил Безымянный.
"--- И всё?

Мы дружно опустили головы, не зная, что ответить.
Внезапно Кошка улыбнулась и кивнула:

"--*Идём со мной.
И вы тоже.
Сейчас проведём небольшой эксперимент.

Кошка привела нас в аудиозал библиотеки и стала рыться в записях.

"--*Кому что нравится? "--- спросила она у нас.

"--*Группа <<Солнечный цветок», октава 12, "--- немедленно ответила Заяц.
"--- Флейта звучит просто превосходно.

"--*Я предпочитал <<Катарсис>>, симфонический оркестр эпохи Начала, октава 16, "--- пожал плечами я.
"--- Но это в далёкой юности.
Давно музыку не слушал.

"--*Жанр <<электрошторм>>, октава 21, "--- сказал Баночка.
"--- Особенно группу <<Кровь и лёд>>.

"--*Начнём с Заяц, "--- ответила Кошка.
"--- Баночка, ты не обидишься?
Просто октава 21 с полусинтетическим экстремальным вокалом "--- это явно не то, с чего следует начинать знакомство с музыкой.

"--*Не волнуйся, я привык, "--- ухмыльнулся плант.
"--- Как сказал Фонтанчик, под <<Кровь и лёд>> хорошо сходить с ума в горящем лесу, а я под неё засыпаю.
Но вообще это классика жанра, ей восемьсот лет.

"--*Попозже, дружок.
Безымянный, ты вопринимаешь идущие от нас эманации?

"--*Да, "--- ответил бог.

"--*Попробуй соотнести поток эманаций со звуковым потоком.
Сможешь?

"--*Это не сложно, "--- ответил хоргет.

"--*Тогда поехали, "--- сказала Кошка и включила запись.

Потекли нежные звуки колокольчиков.
Мягко и успокаивающе вступила флейта.
Утробным басом заговорила виола.
И вот, после недолгой беседы они слились в едином порыве, в единой песне, прославляющей что-то непонятное, далёкое\ldotst то, что осталось где-то на покинутой нами планете.

Зелёная трава, подсвеченная заходящим солнцем.
Колышущиеся метёлки злаков.
Моя первая любовь "--- хрупкий, нежный красавец, с которым я в этой траве шестьдесят лет назад потерял свою невинность.
Имени его я вспомнить так не смог "--- из-за переплавки тела и восстановления мозга стволовыми клетками ранние воспоминания амнезируются\ldotst

Баночка погрустнел.
Заяц кусала губу, уставившись в пространство полными слёз глазами.
Кошка выглядела так, словно сама была не рада своему выбору.
Эта музыка будила чересчур опасные, чересчур личные воспоминания.
Пальцы Кошки нежно гладили пульт управления, делая робкие попытки выключить\ldotst и не выключая.

Когда наступила тишина, мы ещё долго сидели очарованные.
Очнулись, только когда Кошка тихим голосом спросила у хоргета:

"--*Ну, закономерности нашёл?

Голограмма хранила неподвижность.

"--*Безымянный?
Эй!

"--*Это\ldotst прекрасно, "--- ответил Безымянный и исчез.
Снаружи, за толстой бронёй Стального Дракона, хлынул дождь.

Что творилось в этот момент в сознании хоргета?
Что он чувствовал?
Увы, для меня это навсегда останется тайной за семью печатями.
Мы привыкли считать, что мотивирующее чувство (<<удовольствие>>) у хоргетов возникает при поглощении одноимённых эманаций.
Вряд ли я испытывал что-то положительное "--- мне было ужасно грустно.
Неужели и они могут видеть эту непередаваемую гармонию в корпускулярно-волновом хаосе\ldotsq

То, что больше Безымянный не задал ни одного вопроса, оставляет надежду, что он что-то понял.
И я надеюсь, что так оно и было.

\section{[*] Музыка в голове}

\spacing

Цитра Атриса замолчала.
Митхэ сидела, опустив голову, и улыбалась.

"--*Объясни мне, "--- наконец прошептала она, "--- как, как у тебя получается так красиво играть?

"--*Вначале долго слушаешь, потом пытаешься играть.
А вскоре руки сами начинают что-то делать, "--- смущённо пожал плечами Атрис и, аккуратно взяв руку подруги, положил её на цитру.
"--- Попробуй.

Митхэ провела пальцем по струнам.
Цитра то ли фыркнула, то ли звякнула, тихо и недовольно.

"--*Ну вот, "--- засмеялась воительница.
"--- Она на меня ругается.

"--*Если будешь упражняться, она будет смеяться.
Доверие инструмента нужно заслужить.

Митхэ убрала руку с цитры и ткнулась головой в плечо менестреля.
Крохотная <<рыбка>> сверкнула огоньками костра.

"--*Я лучше тебя послушаю.

"--*А если тебе захочется музыки, а меня рядом не будет? "--- улыбнулся Атрис.

"--*Тогда я умру, "--- полушутя, полусерьёзно ответила Митхэ.

Атрис задумчиво нахмурился.
Потом, бросив взгляд на реку, он вдруг засиял улыбкой и, потрепав воительницу по щеке, прыгнул куда-то в тростники.

"--*Атрис!
Ну куда ты в болото\ldotsq

Из зарослей менестрель показался ещё более довольным.
В тонких пальцах он вертел сухой стебель <<муравьиного зонта>>, из каких дети делают игрушечные духовые ружья.

"--*Ну и зачем?

Атрис вместо ответа с важным видом ушёл в лес.

"--*Питонья отрыжка, "--- выругалась Митхэ.
"--- Как будто умрёт, если скажет пару слов.
Ладно, ладно, Хри-соблазнитель.
Хочешь гулять один "--- гуляй, а мне завтра дежурить.
И так из-за тебя почти не сплю\ldotst

С этими словами воительница залезла в палатку и завалилась спать.

Проснулась Митхэ уже под утро.
Костерок весело пылал, пожирая свежие сучья.
Атрис лежал возле неё с открытыми глазами и улыбался.
Длинные тонкие пальцы сжимали красивую, расписанную чёрной, красной и синей красками флейту.

"--*Это тебе, "--- сказал менестрель.
"--- Теперь музыка всегда будет рядом с тобой.

Митхэ осторожно взяла инструмент.
Флейта весила не больше пера сиу-сиу и внешне очень напоминала духовое ружьё.
Только там, где обычно рисовали лик Удивлённого Лю, находился Печальный Митр.
Губки флейты изображали его открытый рот.

У глаз Митхэ рассыпались счастливые морщинки.
Она долго лежала и смотрела то на флейту, то на любимого человека, не зная, что сказать.

"--*Ну как?

"--*Красивая, — признала Митхэ.
"--- Но я всё равно её сломаю.
Моя рука так же создана для музыки, как твоя для клинка.

"--*Глупости.
Ты попробуй.
Возьми вот так\ldotst а, нет, на другую сторону.
Губы вот так.
А теперь дуй.

Митхэ дунула.
Звук получился очень смешной, и влюблённые расхохотались.
Воительница, утирая слёзы, положила дудочку на одеяло:

"--*Нет уж\ldotst

"--*Я тоже когда-то не умел, "--- возразил Атрис.
"--- Давай ещё раз\ldotst

Постепенно звук становился всё более похожим на музыку.
К тому моменту, когда солнечный край бросил первую горсть золотой пыли на верхушки деревьев, Митхэ испустила свою первую трель.

На огонёк зашёл ночной стражник.
Увидев Митхэ с флейтой, он покачал головой, пробурчал что-то про изнасилованного Митра и ушёл.

Наконец где-то вдалеке прокричали за утро.
Митхэ с некоторым сожалением отложила флейту и начала надевать доспехи.
Атрис лежал и смотрел на подругу.

"--*Ты словно готовишь меня к скорби, "--- сказала воительница.

Атрис устало вздохнул.

"--*Я не хочу, чтобы твоё счастье зависело от меня или от кого-либо ещё.
Если я хоть что-то понимаю в любви "--- это она и есть.

"--*А если флейта сломается или её съедят жуки?

"--*Найдёшь новую, "--- пожал плечами Атрис.
"--- Музыку в твоей голове не сломает никто.

"--*Моя музыка в чужой голове.
И эта голова дурная, "--- проворчала Митхэ и, на прощание взъерошив Атрису волосы, направилась к храму.
Менестрель, проводив её нежным и немного грустным взглядом, заполз в палатку "--- наверстать бессонную ночь.

\section{[-] Библиотека}

\spacing

"--*Учитель Трукхвал, а что здесь?

"--*Ооо, Ликхмас-тари\ldotst
Здесь находятся самые ценные книги.

Я пробежался пальцами по корешкам.
Знакомые иероглифы\ldotst и совершенно непонятный смысл.

"--*Эти книги написаны на языке Тхидэ.
Прародины людей сели.

"--*Знаешь ли ты этот язык, учитель?

Трукхвал смутился.
От природы обладающий острым умом и невероятной жаждой знаний, учитель смущался только в одном случае "--- если не знал.

"--*Мне удалось кое-что разобрать, Ликхмас.
Мои предшественники тоже постарались, но перевести всё за эти годы мы не смогли.
Так и переписываем\ldotst хай\ldotst не понимая смысла.
Вот, например, название у книги "--- <<Движение и время в почти симметричной пустоте>>.

"--*Что???

"--*У меня такая же реакция до сих пор, а я побольше тебя живу, "--- развёл руками Трукхвал, снял книгу с полки и открыл её.
"--- Имеется в виду межзвёздная пустота, но при чём здесь симметрия?
На первой странице пометка, уже понятным языком: <<Если вы когда-нибудь полетите к звёздам, вам понадобится>>.
Наверное, нам ещё рано.
Кстати, посмотри вон туда.

Учитель разгрёб свитки на одной из полок и показал мне маленькую, в локоть длиной, коробку.

Я наклонился над ней и провёл руками по гладкой поверхности.
Странный материал, не похожий ни на что из того, что я видел в своей жизни.
Лёгкий, полупрозрачный, немного опалесцирующий в свете лампы.
Сбоку знак "--- число <<13>>, но никаких выступов, за которые можно было бы ухватиться и открыть.

"--*Сокровища древних, "--- улыбнулся Трукхвал.
"--- Шучу.
Но доля правды в этой шутке есть.
Наверное, это что-то невероятно ценное, потому что никто так и не смог открыть этот ларец.

"--*Он не особо похож на ларец.
А распилить не пробовали?

Трукхвал рассмеялся.

"--*Узнаю Ликхмаса-тари.
Не удалось открыть "--- надо распилить.
Помню, как Хонхо-лехэ прибежал ко мне и возмущённо закричал: <<Этого купчонка нельзя допускать в храм!
Он предложил переделать кладку с северной стороны, потому что ,,так будет устойчивее``!
А когда я спросил, что делать со старой, он сказал ,,сломать``!>>

Да, такое было.
Кормилец долгое время строил дома на востоке, и все тонкости кладки я знал к десяти дождям из застольных бесед.
Кстати, северную сторону всё же переделали, когда зодчий из Тхартхаахитра, залившись истерическим смехом, предложил обрушить стену храма двумя ударами.
Не особо сведущие в зодчестве жрецы поверили мастеру на слово и выделили необходимые материалы.

"--*Я помню, "--- смущённо улыбнулся я.
"--- Учитель Хонхо разбил свою чернильницу и пригрозил отлупить палкой, если я ещё раз предложу такое.

"--*Однако именно он понял, что ты гораздо талантливее других, "--- заметил Трукхвал.
"--- Да позаботятся о нём лесные духи.

"--*Да найдёт он покой, "--- поддержал я учителя.

Мы помолчали, глядя на таинственный сундучок.

"--*Так что насчёт\ldotst "--- начал я.

"--*Никаких распилов, "--- отрезал Трукхвал.

Я кивнул.

"--*Тем более что я уже пробовал, "--- извиняющимся тоном добавил учитель.
"--- На ларце не оставляет царапин даже алмазный резец, не говоря уже о стали.
Точнее, оставляет, но они\ldotst трааа\ldotst зарастают.
Да.
Только пожалуйста, никому ни слова о том, что я вытворяю с реликвиями Древних! "--- добавил старик страшным шёпотом.

"--*Я нем, как скала, "--- заверил я Трукхвала.

\section{[-] Математический анализ}

Так началась моя жизнь в библиотеке.
Я сидел и переписывал старые книги, Трукхвал рассказывал мне значения некоторых иероглифов.
Потихоньку старый язык, который ни я, ни мои прапращуры не слышали вживую, стал обретать смысл.

Природе нужно время для всего "--- для прорастания, цветения, плодоношения.
В джунглях Тра-Ренкхаля время "--- это единственный ресурс, который достаётся живой природе по каплям.
Первое, чему учится крестьянин, и второе, чему учится ремесленник "--- терпение.
Я же схватывал науки и ритуал походя, без труда и ожидания, книги храмовыми ласточками пролетали передо мной, и в моём обучении этот важный момент "--- терпение "--- как-то прошёл мимо.
Пришло время навёрстывать упущенное.

Трукхвал заставлял меня по десять раз переписывать страницы, найдя в строгом наущении манускрипта ненароком затесавшуюся шуточку.

"--*Но её же будет скучно читать! "--- вздыхал я.

"--*<<Я не испорчу\ldotst>> "--- строго начинал Трукхвал.

"--*<< \dots ни текста, ни художеств, и гулять моему перу лишь по полям>>, "--- грустно заканчивал я.
"--- Ну хоть на полях-то можно?

В конце концов Трукхвал сдался.

"--*Можешь разрисовать поля, "--- буркнул он.
"--- Как угодно "--- надписями, рисунками.
Только помни, что книгу будет читать другой человек.
Ему должно быть приятно её читать, это раз.
А ещё ему нужно место для собственных пометок, это два.
А вообще ученикам мы такое не разрешаем\ldotst

Строгость никогда не была коньком учителя, и моя импульсивная, похожая на морской прибой настойчивость очень часто брала верх.
Поэтому терпению я обучился лишь в самых общих чертах, но зато прекрасно усвоил, что такое компромисс.

\mulang{$0$}
{"--*И запомни ещё одну вещь, "--- сказал напоследок учитель.}
{``And remember one more thing,'' the teacher said at last.}
\mulang{$0$}
{"--- Красота "--- вечна.}
{``Beauty is eternal.}
\mulang{$0$}
{А вот шутки со временем протухают.}
{But jokes are used to rot.''}

\mulang{$0$}
{"--*Что ты имеешь в виду?}
{``What do you mean?''}

Трукхвал молча достал с полки книгу и, открыв её, показал надпись на полях.
Я поморщился и высунул язык.

\mulang{$0$}
{"--*Трай, какая гадость.}
{``Tr\r{a}i, it's disgusting.}
\mulang{$0$}
{Пришло же кому-то в голову\ldotst}
{And there's the one to whom it had come to mind\dots''}

\mulang{$0$}
{"--*Когда-то это было смешно, "--- объяснил учитель.}
{``It was once funny,'' the teacher explained.}
\mulang{$0$}
{"--- Сейчас "--- как видишь.}
{``You can see how it is now.''}

"--*А кто переписывал? "--- поинтересовался я, пролистывая книгу в начало.
Почерк показался мне смутно знакомым\ldotst

\mulang{$0$}
{"--*Я, "--- коротко ответил учитель и отобрал у меня книгу.}
{``Me,'' the teacher answered succinctly and took the book away.}
Его украшенные серьгами уши были похожи на красные бумажные фонарики.

У меня запылало лицо.
Остаток дня мы оба смущённо молчали.

Выносить книги из храма было строжайше запрещено, но мне очень не хотелось читать в душном, пропахшем старой кожей и опилками зале.
Время от времени я протаскивал то один, то другой том за пазухой и читал дома.
Учитель Трукхвал, видя меня неестественно выпрямившимся и весёлым, только вздыхал:

"--*Что на этот раз, Ликхмас?

"--*<<Математический анализ>>, "--- шёпотом сообщал я.

"--*Там одни непонятные значки и рисунки!
Возьми лучше что-нибудь из истории!

"--*Вдруг интересное найду!

Трукхвал вздыхал ещё тяжелее и возвращался к переписыванию.

Увы, но <<Математический анализ>> мне не дался.
Учитель объяснил, что книга рассказывает об искусстве обращения с числами, но чисел там почти не было.
Я сидел над ней целую декаду, пытаясь понять смысл страшных многоэтажных столбиков, табличек и рисунков.
Иероглифов там было очень мало, а те, что были, не объясняли почти ничего.

<<Из этого следует\ldotst>>

<<Как мы видим\ldotst>>

<<Доказывает это следующее\ldotst>>

Для меня, не самого достойного потомка Древних, это звучало, как насмешка.
С некоторым сожалением я вернул книжку на её место и взялся за следующую.

\section{[-] Свободное время}

\spacing

Трукхвал не походил на других взрослых.
Может быть, именно из-за этого прочие жрецы и многие воины его сторонились.

Однажды я сидел за переписыванием толстенного тома <<Обычай и ритуал>>.
Учитель же свою норму переписывания давно выполнил и сидел скучал, раскладывая по форме древние кукхватровые детальки.
Где он их брал "--- так и оставалось загадкой;
Трукхвал давно уже привык работать с детальками при мне, но едва в дверь библиотеки раздавался стук, детальки волшебным образом исчезали в бумажных кульках.

"--*О чём думаешь, Ликхмас? "--- спросил Трукхвал.
За год, проведённый со мной в библиотеке, учитель великолепно научился распознавать озадаченное выражение лица.

"--*Я задумался о нашем обществе, "--- сказал я.
"--- Мне кажется, в нём есть какая-то несправедливость.

"--*Что ты имеешь в виду?

"--*У меня столько свободного времени.
Я могу заниматься тем, чем хочу.
У моих друзей "--- ремесленников, крестьян "--- его гораздо меньше.

"--*Ты думаешь, несправедливо?
Взгляни на это с другой стороны, Ликхмас-тари, "--- Трукхвал отхлебнул травяного отвара и ткнул пальцем в потолок.
"--- Предки были мудры.
У нас есть Храм, у нас есть Двор, у нас есть Цех и Сад.
И у каждого сообщества свои обязанности.
Вот например, если в городе много больных, если дети не знают и не хотят знать грамоты, а Безумные обрушивают на народ свой гнев, кто виноват?

"--*Верхний этаж, "--- подумав, сказал я.

"--*Правильно.
А если молодые не умеют держать оружия, не знают тактики боя, боятся сражаться или излишне жестоки?

"--*Нижний этаж.
Воины не тренируют молодых и не учат их искать ногой твёрдую опору.

"--*А кто виноват, если люди голодают?

"--*Может быть, Сад?

"--*Безусловно.
Двор тоже причастен к хозяйству, твоя кормилица после каждого дождя делает расчёты и пишет для крестьян план, чтобы урожай был разнообразным.
Однако недостаток урожая "--- это вина Сада: крестьяне не умеют обрабатывать почву или разленились.
А кого ты обвинишь, если в городе мало завозных товаров, если племена дикарей совершают на нас набеги?

"--*Плохо работает Двор.

"--*Вот и всё.
Каждое сословие выполняет свою роль, и ошибки их видны.
Главное "--- не закрывать на эти ошибки глаза и исправлять их вовремя, ведь одна ошибка тянет за собой другую.
Для решения проблем существуют Советы, которые каждый может созвать и на которых каждый может высказаться.
Свободное время же зависит от мастерства человека в его деле.

"--*А если человек хочет заниматься чем-то другим?

"--*Он волен выбирать.
Но ему придётся долго учиться, чтобы выбрать другое дело, а значит, жертвовать своим свободным временем.

"--*Будущих жрецов выбирают жрецы.
Если взрослый человек захочет учить детей или врачевать, ему будет поздно выбирать.

Трукхвал развёл руками.

"--*Бывали такие случаи.
С этим ничего не поделать.

"--*В этом и есть некая несправедливость, "--- сказал я.

"--*Может, ты придумаешь что-то получше, "--- улыбнулся учитель.
"--- Для того тебе и дано свободное время.

\section{[-] Тысячи в одном теле}

Сегодня Митхэ сразила меня наповал новым вопросом.

"--*Скажи, учитель Ликхмас, а я целая?

"--*Что ты имеешь в виду?

"--*Я как существо целая?

"--*Разумеется, ученица.
Да, хоть ты и состоишь из крохотных живых существ "--- клеточек, ты совершенно определённо целая как существо.

"--*Тогда ответь мне, "--- попросила Митхэ, "--- кто обнимает игрушки во сне, если это делаю не я?

"--*Почему ты решила, что это не ты?

"--*Я в это время смотрю сны и не знаю об игрушках.

"--*Сон "--- особое состояние.
Возможно, ты просто этого не осознаёшь.

"--*Тогда возьмём бодрствование.
Кто дышит за меня, когда я об этом не думаю?
Кто за меня идёт, когда я не обращаю на это внимания?

На этот раз я вынужден был сказать, что не знаю ответа на эти вопросы.

"--*А однажды мне очень понравилась девочка, "--- продолжила Митхэ.
"--- Я хотела всё время проводить с ней.
А другая девочка мне не нравится.
И ты мне очень нравишься, и это тоже выбирала не я, потому что я не знаю причин.
Учитель Ликхмас, я точно целая?

"--*Тебя это беспокоит, Митхэ?

"--*Я знаю, что есть вещи, которые я не делаю, "--- заявила Митхэ.
"--- Например, солнце всходит и заходит без меня, и ветер тоже следует своим законам.
Однако то, о чём я рассказывала, происходит во мне, а не вне меня.

"--*Возможно, что ты права, "--- предположил я.
"--- Тело "--- это действительно множество существ, которые могут действовать самостоятельно.

"--*И я должна с ними подружиться, "--- подхватила Митхэ.
"--- Ведь так?
Ведь для этого нужна школа, тренировки "--- чтобы подружиться с собственным телом?

Я кивнул, в очередной раз думая о том же "--- ребёнок интеллектуально превосходил меня на голову.

"--*Как обучать того, кто умнее тебя? "--- спросил я как-то у Трукхвала.

"--*Я же тебя обучил, "--- оскалился старый жрец.
"--- Наверное, главное "--- подавить в себе естественное желание ограничить ученика.
Мне порой очень хотелось.

"--*Я очень боюсь сделать что-нибудь не так, "--- признался я.

"--*Ооо, а как я-то боялся, "--- посетовал в ответ учитель.

Мы так долго хохотали, что на шум пришёл Первый жрец.

"--*Что это ещё такое? "--- строго сказал он.
"--- Ну-ка брысь оба из библиотеки!
Нашли где смеяться!
Всю пыль с книг стрясёте хохотом!

"--*По-моему, давно пора стрясти с них пыль, "--- шёпотом заметил Трукхвал.

\section{[M] Книжная пыль}

Однажды мы с Трукхвалом отправились в старый отдел библиотеки за ингредиентами для красок.
Там было сухо и чересчур пыльно.
Кладка в некоторых местах просила хэситр, а полки прогибались дугой, словно западные рогатые луки.
Пыль играла с лучами масляной лампы, пыль клубилась у наших ног.
Меня не отпускало чувство, что эта пыль "--- книжная.
Она пахла старыми чернилами.
Я поделился этим ощущением с учителем, на что он вздохнул:

"--*Так оно и есть.
Здесь мы храним книги, к которым уже нельзя прикасаться.

"--*Почему, учитель?

"--*Потому что они превратились в прах.
Видишь, полки закрыты занавесками?
Книги рассыпаются от простого прикосновения.
Хаяй\ldotst
Больно это видеть.

"--*Почему их не переписали, учитель Трукхвал?

"--*Пятьсот дождей назад, как ты знаешь, Тхитрон был опустошен идолами\ldotst

"--*Нашествие Змей, "--- кивнул я.

"--*Люди отбили его много позже.
Увы, это то, что они не успели переписать.
Так и храним\ldotst
Надо, конечно, в Тхартхаахитр съездить, сделать копии, если в их храме есть, но куда мне с моей\ldotst хай\ldotst
Вот, посмотри.

Учитель осторожно отвёл занавеску.
Обычные на вид книги.
Но лежавшая рядом слоистая куча пыли показывала, что с этими книгами происходит при неосторожном прикосновении.

"--*Странно, что идолы не сожгли эти книги.

"--*Ничего странного, Ликхмас.
Идолы не дураки.
Многие даже уносят книги в свои деревни.
Правда, понятия не имею зачем.
Говорят, пытаются читать или просто хранят их в своих святых местах.

"--*Учитель, даже идолы относятся к ним с почтением.
Не должны же они так лежать и рассыпаться у нас!

"--*А что делать? "--- развёл руками Трукхвал.

"--*Может, их можно как-то склеить?

"--*Как, ученик?
Это же пыль!

Я задумался.

Обратный путь мы проделали в полном молчании.

Следующие дни я раз за разом возвращался в мыслях к древнему тёмному помещению.
\mulang{$0$}
{Книги снились мне каждую ночь: я пытался снять том с полки, а он рассыпался у меня в руках.}
{I dreamed every night that a book has come apart in my hands when I tried to take it off the shelf.}
Просыпался я в холодном поту.
Кхотлам за завтраком стала посматривать на меня с беспокойством.

"--*Ликхмас, дитя, ты совсем не ешь.
Ты не заболел?

"--*Нет, кормилица.

"--*Сестрёнки потеряли тебя.
Ты целых девять дней не заходил к ним перед сном и не желал спокойной ночи.
Днём тебя тоже не найти.
Негоже так.

Я мысленно проклял эти книги.

"--*Где они сейчас?

"--*Играют у реки с друзьями.
Поинтересуйся, как у них дела.

День выдался погожий.
Только-только закончился сезон дождей, и прохлада влаги ещё не превратилась в неприятную духоту.
Ветер ласково трепал мои волосы.
Да, засиделся я в библиотеке.

Вокруг кипела работа.
Приречные крестьяне возили на свои участки отходы "--- кто по реке, на каноэ, кто в наземных деревянных повозках.
Кудахтали куры, хрюкали довольные свиньи, мяукали и ворчали домашние оцелоты, выпрашивая потрошки пожирнее.
Люди оживлённо переговаривались, разбрасывая нечистоты по грядкам, залихватски ухали и пели песни, перекапывая жирную, влажную почву.
В воздухе стоял крепкий запах навоза и пота.

"--*Лисичка, как твои дела? "--- крикнул мне загорелый парень-крестьянин, с которым мы часто играли в детстве.

"--*Слава духам, Столбик.
Гляжу, у тебя сезон?

"--*Хорошая почва "--- хороший урожай будет, "--- весело откликнулся Столбик, сдвинув на затылок травяную шляпу.
"--- Ты так же, в библиотеке, книги пишешь?
Заходи как-нибудь вечером в гости, а то солнца не видишь!

Я помахал ему и пошёл дальше, на песчаный берег, на игровую площадку.
Дети носились, прыгали и кричали.
Простая счастливая жизнь.
Я замедлил шаг, высматривая в толпе сестрёнок.

"--*Хаяй, мой домик!

Маленький ребёнок обиженно смотрел на меня, сидя на песке.
Кажется, я не заметил его песчаного домика и наступил на него.

"--*Ой, прости, малыш, "--- улыбнулся я и присел на корточки.
"--- Давай я построю тебе новый.

"--*Давай, "--- личико ребёнка, на котором уже начали проступать женские черты, просветлело.

Я разгладил руками мокрый холодный песок и запустил в него пальцы.

"--*Как тебя зовут?

"--*Эрси ар’Митр, "--- гордо сообщил ребёнок, с интересом глядя на стройку, и потеребил пальчиками мою робу.
"--- А ты жрец?

"--*Пока ещё нет, "--- засмеялся я.
"--- Но скоро буду, если духи помогут.

"--*Ты из дома ар’Люм?
Твои сестрёнки ушли домой, "--- рассказал ребёнок, не дожидаясь вопроса.
"--- Кажется, к вам кто-то приехал, и нужно было проводить гостей.
Ты ведь за ними пришёл?

"--*А вот чтобы тебе домик построить.

"--*Ты врёшь, "--- насупился ребёнок.

"--*Правда-правда, "--- заверил я его и добавил к полувшемуся дворцу пару завершающих штрихов.
"--- Эрси ар’Митр, можешь заселяться.

"--*Он красивый, "--- сказал ребёнок и критически осмотрел домик со всех сторон.
"--- Жаль только\ldotst

"--*Что?

"--*Пока песок влажный, он стоит, "--- объяснил Эрси.
"--- А когда взойдёт солнце и подсушит его, он превратится в гору пыли.

В моей памяти молнией сверкнула пыльная библиотека и рассыпающиеся книги.
В храм я бежал быстрее лани.

\razd

"--*Ликхмас, ты с ума сошёл! "--- воскликнул учитель, выслушав мою идею.
"--- Книги рассыпаются при прикосновении, а ты вздумал лить на них воду\ldotse

"--*Я не собираюсь лить на них воду, "--- терпеливо принялся объяснять я.
"--- Пыль пока имеет форму.
Надо закрепить её чем-то жидким.
Склеить.
Ведь можно же касаться книг очень мягким, смоченным в\ldotst

"--*Хорошо-хорошо, я тебя понял, "--- Трукхвал, похоже, сам заинтересовался.
"--- Давай подумаем, что можно использовать.
Вода, конечно, отпадает сразу.
Который час?

"--*Ещё даже не полдень, учитель.

"--*Бегом на рынок, возьми рыбьего клея.
Хай, ты же ещё не жрец, бесплатно тебе ничего не дадут.
Вот перо золотого песка, купи сразу побольше.

Мы развели в храме нешуточный переполох.

"--*Трукхвал, что за дела? "--- возмутился Кхатрим, только собравшийся готовить лекарства про запас.
"--- Куда ты потащил мой котёл?

"--*Ликхмас ар’Люм!
Верни перья сиу-сиу на место!
Из-под палок не встанешь, выкидыш идола!

Воины только посмеивались, глядя на нас.

"--*Что старый, что молодой, а ума "--- как у ящерицы.

Уже под вечер мы с Трукхвалом стояли в старой библиотеке.
На подставке дымилась чаша с тёплым жиденьким рыбьим клеем, в нём плавали самые мягкие перья сиу-сиу, которые я смог отобрать.

"--*Может, муки чуть побольше? "--- шёпотом спросил учитель.

"--*Нет, "--- покачал я головой.
"--- На этот раз мы всё смешали правильно.
Смотри, страничка отошла почти полностью, расслоилась только в самом конце.
И всё равно кое-что можно прочитать\ldotst

"--*Страничку-то ты спас.
А книжка? "--- посетовал Трукхвал.

Мы с некоторым сожалением посмотрели на останки предыдущего эксперимента.

"--*Эта книжка совсем сгнила.
Нам просто не повезло.

"--*Ликхмас, давай\ldotst "--- жалобно начал учитель.

"--*Нет, "--- твёрдо сказал я.
"--- Мы будем пытаться, пока не получится.
Или пока мы не превратим в пыль всю эту секцию.
Смотри вон на тот огрызок, он всё равно ни на что не годен.

Трукхвал только горестно заохал.
Он знал, как сложно меня отговорить, если я чем-то увлёкся.

Мы задумчиво смотрели на освещённую неверным светом лампы одинокую книжку, лежавшую отдельно от остальных.
Судя по корешку, когда-то это был целый том в три пальца толщиной, сейчас от него осталось несколько десятков страниц.

"--*Хай, а если опять не получится, ученик? "--- пробормотал Трукхвал, чуть не плача.
"--- Святыня\ldotst
Ещё одна\ldotst

"--*Святыни, которые всё равно рассыплются, "--- пожал плечами я.

Трукхвал нервно усмехнулся.

\mulang{$0$}
{"--*Твоя правда.}
{``You're right.}
\mulang{$0$}
{Давай-ка ты на этот раз, твои руки нежнее моих.}
{This turn is yours, your hand is softer than mine ever was.''}

Я обмакнул перо в клей и осторожно начал промазывать обложку и корешок.
Подождал, пока клей впитается, и промазал ещё раз.
Трукхвал, до хруста сжав костлявые кулаки, наблюдал за процессом.

Время от времени я пробовал остриём пера обложку на прочность.
С самого края, чтобы не повредить странички.
Первый раз перо провалилось почти на горошину, Трукхвал охнул.

\mulang{$0$}
{"--*Давай ещё.}
{``Go on.}
\mulang{$0$}
{Не торопись.}
{Go slow.''}

Ночь пролетела незаметно.
Мы так увлеклись, что не видели и не слышали ничего вокруг.
И вот, уже под утро, обложка поддалась и отошла, не разрушив странички.
Мы с учителем, как зачарованные, уставились на заглавный лист.

<<Повесть о Существует-Хорошее-Небо, вожде сели, наезднике Железного Змея>>.

\razd

На следующий день Трукхвал рассказал о нашем эксперименте другим жрецам.
Похвал мы не дождались "--- жрецы, особенно <<красные>>, скупы на это, "--- но ни один, надо отдать им должное, не упрекнул нас в святотатстве и излишнем авантюризме.
Те, что постарше, выразили желание ознакомиться со спасёнными книгами, а жрецы помоложе пошли с нами "--- лично принять участие в спасении.
В библиотеке запахло рыбьим клеем и потом.

Учитель был совершенно счастлив.
Его нежданные помощники приносили ему страничку за страничкой.
Трукхвал сортировал листки, добавлял замечания на краях или клочках старой бумаги, время от времени сокрушённо качал головой, когда текст оказывался безнадёжно испорчен, а меня усадил за переписывание.

"--*Ликхмас, здесь ключ <<листик>>, а не <<рыбка>>.
И вот здесь.
Да, у переписчика был странный почерк, согласен.
Нет-нет, не надо переделывать всю страницу, подотри пергамент шкуркой.

"--*Ликхмас, эмоглиф в языке Тхидэ идёт сразу после глагола, модификатор глагола ставится над эмоглифом.
Да, всегда.
Да откуда я знаю почему?!
Просто перепиши это и будь внимательнее.

Вечером я заглянул к сестрёнкам и извинился за невнимание целой горстью сладких конфет.
Они наперебой рассказывали о своих играх и тренировках, а я сидел и гладил их по светлым, цвета какао с молоком, волосам.
Зашедшая пожелать спокойной ночи Кхотлам стояла и, улыбаясь, наблюдала за нами.
Вдруг я вспомнил, что кое-кому должен\ldotst

"--*Молочко, "--- спросил я у сестрёнки, "--- ты знаешь Эрси ар’Митр?
Он играл с вами вчера днём.

"--*А, молчун Котёнок, "--- кивнула сестрёнка.
"--- Он сегодня не пришёл.
Кусачка, ты видела его?

"--*Не-а, "--- помотала головой близняшка.

"--*Эрси ар’Митр? "--- вмешалась кормилица.
"--- Его сегодня ночью принесли в жертву.
А почему ты спросил, Лисёнок?

\chapter{[-] Семья}

\section{[М] Доблесть}

\spacing

Вспомнил свою первую схватку с настоящим врагом.
Кормилица отправилась в деревню за рекой, чтобы помочь в решении каких-то проблем с товарами.
Я, двадцати пяти дождей, отправился вместе с ней.
В окрестностях орудовала банда разбойников, и случай столкнул меня с двумя мужчинами.
Они были умелыми бойцами и играли со мной, как сытые коты с мышью.
Я сумел полоснуть одного из них ножом, но рана лишь разозлила его.
Тогда, забившись в узкий проход, я зарядил духовое ружьё и птицей попросил помощи.

"--*Хэй, мальчик, выходи! "--- издевательски кричали разбойники.
"--- Или ты трусишь?

Я испугался.
Мои руки ослабели, нож дрожал так, словно меня уже избили до полусмерти.
Ни о каком бое не могло быть и речи.

Подоспевшие жители спугнули разбойников, и я вышел на свет.
Все молчали, и я тоже молчал, думая, что меня считают трусом.

"--*Ты как? "--- наконец спросил Кора, ученик плотника чуть постарше меня.
Я не ответил.
Меня била крупная дрожь.

"--*Все мы боялись в наш первый бой, Ликхмас ар’Люм, "--- хлопнул меня по плечу Кора.
Остальные закивали, выражая согласие.

"--*Их двое, и они куда опытнее тебя.
Тебя бы просто убили или покалечили.
Молодец, что спрятался.

"--*Трусы, "--- добавил кто-то.
"--- Двое взрослых мужчин на одного мальчишку.
И сбежали, как крысы, едва увидев нас.

Это слабо утешало.
В тот вечер я ушёл к реке и долго плакал от обиды и злости на самого себя.

\razd

Следующие две декады я не мог отойти от собственного мига трусости.
Разбойники начали сниться мне по ночам.
Я отчаянно кричал и пытался поразить их ножом, но они лишь смеялись и издевались надо мной, дымными змеями ускользая от ударов.

Тренировки стали настоящей мукой.
Казалось, что каждый спарринг-партнёр знает о случившемся и втайне смеётся надо мной.
Одному из них "--- крепкому, но неповоротливому Огранённому "--- я сломал нос и на миг испытал невероятное наслаждение, глядя на его испуг и бегущую из ноздрей тёплую кровь.
Конфетка покачал головой и повёл парня в храм, к жрецам.

Вскоре молодые воины стали меня избегать.
Я не был лучшим в поединках, но ужасное новое чувство заставляло любыми путями достигать победы.
В ход шло всё "--- подручные средства, психологическое давление и довольно рискованные приёмы.
Ещё спустя пятнадцать дней к жрецам отправились трое парней, попытавшихся после тренировки разобраться со мной <<по-воински>>.

Вечером в дверь постучались.
Следом ко мне заглянула растерянная кормилица и сказала, что пришёл Конфетка.

"--*Пойдём, "--- коротко и довольно неприветливо сказал тренер.

Конфетка привёл меня к реке.
Солнце уже давно зашло, и река казалась сплошным чёрным пятном.
Где-то нежно квакала лягушка и весело плескались рыбки, ловя на лету жирных насекомых.

Конфетка устало сел на поросший мхом камень и скрестил руки на груди.

"--*Ликхмас, объясни, что с тобой происходит.

"--*Они сами начали\ldotst "--- бросил я заготовленную фразу, но тренер нетерпеливо махнул рукой.

"--*Я прекрасно знаю, когда всё началось.
Ты уже две декады сам не свой.
Что это?
Откуда такая жестокость?
Куда подевался мальчик, который когда-то отказался бить слабого?

"--*Они не слабые, "--- огрызнулся я.
"--- Их было трое.

"--*А Костёр?
Ты понимаешь, что ты мог его убить <<хлыстом двух скал>>?
Я говорил "--- приём отработать, но против друг друга не применять ни в коем случае.
Говорил?
Говорил.
Парня спасло только то, что он в защите куда лучше, чем ты "--- в атаке.
А Огранённый?
Чем этот увалень тебя обидел?
И хватит отнекиваться.
Ты прекрасно знаешь, как и куда нужно бить на тренировках.

Я молчал.
Конфетка вздохнул.

"--*Кхотлам сказала, что в деревне Кхихутр что-то произошло.
Расскажи мне.

"--*Как будто ты не знаешь, "--- снова ощерился я.

"--*Нет, я ничего не знаю, "--- устало сказал тренер.
"--- И Огранённый, и прочие тоже в неведении, но исправно получили за это свежих пирожков.
Так давай, расскажи, а я попробую объяснить парням, за что им досталось.

"--*За дело, "--- рявкнул я и направился к дому.

"--*Трус, "--- бросил Конфетка мне в спину.

В моей голове застучала кровь, и я, оттолкнувшись от ствола молодого дерева, изо всей силы ударил тренера локтем в глаз.
Конфетка вскрикнул и упал.

Прошла целая михнет "--- и до меня вдруг дошло, что он даже не собирается вставать.
Тренер просто лежал на земле, прижав руки к лицу и свернувшись калачиком.
Я склонился над ним.

"--*Хай, Конфетка.
Ты жив?

Тренер не ответил.
Я потряс его за плечо.

"--*Встань, пожалуйста.
Прости меня, я не хотел.

Ответа не было.
Я заплакал и начал отрывать его прижатые к лицу ладони.

"--*Прости меня!
Встань, пожалуйста!
Прости!

Конфетка медленно сел и открыл лицо.
Бровь была разбита, и опухшее веко совершенно скрыло глаз.

"--*Ликхмас, "--- тихо сказал тренер, "--- можешь ударить меня ещё раз, но расскажи, что случилось.

Этот взгляд и кроткие слова мне суждено было запомнить на всю жизнь.

\razd

Время уже давно перевалило за полночь.
Мы с Конфеткой сидели дома за дымящимися чашами с травяным отваром.
Домашние уже спали, я слышал сонное бормотание Эрхэ и похожий на урчание ягуара храп Сиртху-лехэ.
Только кормилица, так и не задав ни одного вопроса, отправилась на улицу выбрасывать вареные листья.

Конфетка пил отвар и смотрел на меня целым глазом.
На другой глаз я сделал ему примочку.

"--*Кхотлам я ничего не сказал и не скажу, "--- сообщил Конфетка.
"--- Это её не касается.
Ответственность за произошедшее лежит на тебе и на мне.

Я кивнул.

"--*Знаешь, "--- тихо сказал тренер, "--- в детстве меня часто били в Храме.
Но чаще, чем от сверстников, удары я получал от тренера.
Настоящие, с охотой, не чета моим.
Он считал, что если детей почаще бить и ругать, то они станут храбрецами.

"--*Но ведь ты стал, "--- возразил я.

"--*Нет, "--- улыбнулся Конфетка.
"--- Не стал.

"--*Но ты же\ldotst

"--*Да, я многое повидал, достаточно повоевал.
У меня был выбор "--- завести семью с подходящими людьми, осесть, стать ремесленником.
Я подрабатывал когда-то краснодеревщиком, ты знал?

"--*Нет, не знал.
И как далеко ты зашёл?

"--*Хай\ldotst
Ну, ты сидишь за столиком моей работы.
И для Верхнего этажа сундуки делал тоже я.
Не знаю насчёт остальных, давно не заходил, но у Трукхвала в келье точно мой стоит, до сих пор.
В общем, да, мои работы ценились.
Но я не считал их чем-то выдающимся, заслуживающим уважения.
Я выбрал путь воина, потому что хотел доказать себе, что я не трус.

"--*Так ты и не трус.

"--*Много раз было так, что я действовал в ущерб себе, боясь, что меня посчитают трусом.
Это тоже было трусостью, как я сейчас понимаю.

Я промолчал.
Конфетка рассеянно гладил пламя свечи пальцем.
Не любящий фамильярностей язычок сердито пытался отодвинуться.

"--*Когда-то я вёл беззаботную жизнь.
Походы, дерзкие вылазки.
Каждый шрам казался мне благородной каллиграфией.
Любовников у меня было не так много "--- многие смотрели с надеждой в мою сторону, но ведь настоящий воин и храбрец должен держать марку, верно?
У меня была, стыдно признаться, даже коллекция ушей поверженных врагов.

Я поморщился.
Конфетка кивнул и закрыл лицо руками.

"--*Да, я сегодняшний отреагировал бы точно так же.
Но мне казалось, что это и есть доблесть.
И поэтому, когда Морковка после одной из ночей попросила разрешения похоронить эти уши, я назвал её дурой и сказал, что ухожу навсегда.
И это тоже казалось мне доблестью.
Она не заплакала, просто молча собрала мешок и так же молча поцеловала на прощание.
На самом верху лежал свежий черноягодный пирог.
Выйдя за ворота, я со смехом бросил пирог в канаву.
Ведь воин не должен размениваться на ерунду и баловать своё тело.
Надеюсь, что она его потом не нашла.

Конфетка говорил совершенно спокойно, но по моей спине пробежала дрожь.

"--*Однажды я возвращался в родной город из Тиши и шёл мимо её дома, "--- проговорил тренер, и целый глаз вдруг вспыхнул.
"--- В окнах горел свет.
Там весело переговаривались дети.
Я всегда любил ребятишек, но когда услышал детские голоса из её дома, у меня защемило сердце.
Странное чувство, да\ldotsq

Я кивнул.

"--*Сама она сидела со своим мужчиной на скамье и любовалась звездами.
Они были такие красивые, оба.
Они разговаривали о том, как же, каким искусством сплели полотно этого мира.
Наверное, Безымянный был очень трудолюбив, раз создал такое количество звёзд, небо, землю и зверей.
Так они говорили.

Конфетка опустил голову, и отвар стал чуть солонее.

"--*А я понял в тот миг, что мир получился таким прекрасным просто потому, что его создателю не перед кем было кичиться.
Ему не нужно было чьё-то внимание, чьё-то одобрение.
Некому было сказать, что он растяпа, трус или лентяй.
Он просто делал то, что хотел.
И у него всё получилось.
Он, я полагаю, счастлив.

Конфетка кивнул и отхлебнул из чаши.

"--*На следующий день я пришёл в Храм и во всеуслышание заявил то, что не давало покоя мне все эти годы.
Я сказал: <<Да, мне страшно.
Я боялся сражаться, когда шёл через Тиши, но шёл, потому что боялся проиграть.
Я положил годы на ужас походов, потому что боялся показаться трусом.
На самом деле я и есть трус.
Одерживая победу оружием, я проиграл жизни, в страхе потеряв то, чего хотел больше всего.
Но сегодня этому пришёл конец.
Я говорю всем: мне страшно, и я этого больше не стыжусь>>.

Конфетка улыбнулся.

"--*А что было потом? "--- тихо спросил я.

"--*Потом я сказал, что больше никогда не возьму в руки боевое оружие.
Разумеется, это было отречением от воинской клятвы, но воины, надо отдать им должное, поняли меня и не позволили уйти из Храма.
Боевой вождь, моя лучшая подруга, спросила: <<Что тебе нужно?>>.
Я ответил: <<Мне нужна семья>>.
Так я стал тренером.

Я вздохнул.
Разумеется, это не была настоящая семья "--- лишь суррогат, попытка попробовать на вкус давно ушедший в прошлое пирог.
Возможно, Конфетка больше не чувствовал в себе сил на постройку собственного жилища, на изменение уклада собственной жизни, и решил довольствоваться тем, что есть. Некогда кипевшая энергия ушла на никому не нужную иллюзию.

\mulang{$0$}
{"--*А та женщина простила тебя? "--- спросил я.}
{``That woman, did she forgive you?'' I asked.}

\mulang{$0$}
{"--*Наверное, "--- повеселел Конфетка.}
{``May be,'' Candy cheered up.}
\mulang{$0$}
{"--- Когда я пришёл к ней поговорить, то получил котелком по загривку.}
{``When I came to talk to her, I got hit on the neck with a cauldron.}
\mulang{$0$}
{Горячим котелком, Ликхмас, только с огня.}
{A very hot cauldron, L\={\i}kchm\r{a}s, freshly heated.}
\mulang{$0$}
{Четыре раза.}
{Four hits.}
\mulang{$0$}
{Думаю, да, простила.}
{I guess, yeah, of course she did.''}

\razd

Я открыл дверь, и в дом ворвалась скрипучая песня тысяч сверчков.
Конфетка поправил плащ, прислушался и щёлкнул языком от удовольствия.

"--*Всё-таки интересно, как Безымянный додумался до сверчков.
<<Скучный какой-то жук.
Может, он будет тереть лапками крылья, чтобы найти самку?>>, "--- Конфетка прыснул.
"--- Вот они и пилят по ночам, стараются.
Но изобретение однозначно хорошее и нужное.

"--*Ты веришь в Безымянного? "--- удивился я.
Конфетка всегда казался мне далёким от религии нигилистом.
\mulang{$0$}
{"--- Почему?}
{``Why?''}

"--*Жил-был некто, кто создал много хороших вещей, его изгнали и отобрали у него творение, "--- Конфетка болезненно поморщился.
"--- История чересчур похожа на правду, чтобы не быть ею.

"--*Я сменю тебе повязку завтра, "--- пообещал я.
"--- Только не тереби, а то я тебя знаю.

Тренер расхохотался и тут же, опомнившись, прикрыл рот ладонью.

"--*Хаяй\ldotst
Ночь уже, а я тут\ldotst
Не волнуйся, не буду.
А вот к парням зайди обязательно.
Что им сказать, знаешь.

"--*Что я трус, "--- кивнул я.

"--*Дурень, "--- покачал головой Конфетка.
"--- Прощения попроси.
А насчёт храбрости\ldotst ну что поделаешь, Ликхмас.
Истинным храбрецом, увы, можно лишь родиться.
Но и осторожность, и сострадательность "--- недешёвый товар.

"--*Но ведь мне придётся сражаться, "--- сказал я.
"--- Значит, нужно как-то подавлять страх?

"--*Ты когда-нибудь надрывал мышцу, Ликхмас?

Я кивнул.

"--*Представь, что ты надорвал мышцу и тебе нужно поднять груз.
Можно разработать мышцу, чтобы она болела меньше.
Иногда приходится прикидывать, как изловчиться, чтобы её не потревожить.
А иногда\ldotst иногда приходится поднимать, невзирая на боль.
Не нужно только убеждать себя и окружающих, что у тебя ничего не болит.
\mulang{$0$}
{Я потратил на это полжизни и не обрёл ничего.}
{``I spent half my life trying, and got nothing in return.''}

Я кивнул.

"--*Твои чувства "--- это такие же мышцы.
Нужно принять их такими, какие они есть, и хорошо знать их особенности.

"--*И тогда я стану непобедим?

"--*Нет, "--- покачал головой Конфетка.
"--- Ты станешь счастливее.

Конфетка на секунду отвернулся и вдохнул носом влажный ночной дурман.

"--*Но ведь чувствами нельзя управлять, "--- сказал я.

"--*Конечно, можно.
Ты же переносишь вес на здоровую ногу, поднимая груз, верно?
Так же и с чувствами.
Если боишься сражаться, можно позволить другому чувству овладеть собой.
Например, любви.

"--*<<В жизни и смерти буду я щитом>>, "--- процитировал я.

"--*Верно, схватил суть, "--- Конфетка похлопал меня по плечу.
"--- Хорошая у вас, жрецов, клятва.
Нашей, воинской, чего-то недостаёт.
Наверное, границ защиты.
Войны сейчас уже не те, они лживые по сути "--- в них нет агрессоров, одни защитники.
Как говорит Ситрис, <<словно кто-то выбросил слово из клятвы и памяти людской\ldotst\footnote
{<<\ldotst и прахом стали верность и честь, стоявшие на зыбкой почве песен>>.
Строфа из плача Маликха по Ликхмасу, <<Легенда об обретении>>. \authornote}>>

"--*А как распознать грань между вторжением и обороной?

Конфетке понравился мой вопрос.
Таким довольным я не видел его никогда.

"--*Духи лесные.
Как же ты мне напоминаешь одного человека.
Он искал ответ на этот вопрос всю свою жизнь.

"--*Где он его искал?

"--*В кошачьих слезах, сосчитанных песчинках и утренней росе\footnote
{Имеются в виду три основных философских школы боя "--- Путь Ягуара, Десять Песчинок и Плющ и Капля Росы. \authornote},
"--- отшутился тренер.

"--*Где сейчас этот человек?
Он нашёл ответ?

"--*Хаяй, кусочек где-то здесь валяется, "--- Конфетка ущипнул меня за бок, "--- а где остальные\ldotst

Тренер вдруг замолчал и торопливо начал пристёгивать оружие.

"--*Ты знаешь, я бы очень хотел, чтобы ты принёс воинскую клятву, "--- признался он.
"--- Даже хотел в пику Храму, глухой ночью, в строжайшей тайне, но\ldotst думаю, не нужно.

"--*Я принесу её, "--- сказал я.
"--- Если ты хочешь.
Сейчас как раз глухая ночь.

Конфетка грустно улыбнулся.

"--*Я тебя слушаю.

\begin{quote}
<<Я не причиню больше страданий, чем это необходимо;\\
я не оскверню звание воина ни пыткой, ни мучительным убийством, ни угрозой близким;\\
я не подниму руку на безоружного и неумелого;\\
я не позволю сделать подобного ни врагу, ни другу.\\~\\
Я не сломаю дух ребёнка ни оскорблением, ни ударом, но сделаю его стойким;
я обучу его всему, что знаю и умею в бою, но в первую очередь "--- тому, что поможет избежать боя.\\~\\
В жизни и смерти я буду клинком, и ни мгновения "--- плетью>>.
\end{quote}

"--*Мне бы следовало дать тебе пощёчину, чтобы ты запомнил как следует, "--- вздохнул тренер.
"--- Но если этот удар лишний "--- а я в этом не сомневаюсь, "--- я не прощу его себе никогда.
Пусть в кувшинах твоих снов будут только вина и ароматные масла, Ликхмас.
Завтра жду тебя, как обычно.

Конфетка тепло улыбнулся и пропал в темноте под непрерывную любовную симфонию сверчкового оркестра.
А я стоял, придерживая рукой открытые двери, и чувствовал, что ужасное чувство ушло вместе с тренером.

Всё встало на свои места.

\section{[-] Метритхис "--- игра учёных}

%Змея 4 дождя 12001, Год Церемонии 20.

"--*Меркхалон-кровопийца, ты только посмотри, как нам Фатум поле выложил, "--- недовольно буркнула Кхохо.
"--- Ну и как играть в таких условиях?

"--*А мне нравится, "--- улыбнулась Ликхэ.
"--- Для <<мышей>> идеально.

"--*А я вот не умею на <<мышах>> играть.

\spacing

"--*Ликхэ, ты рыбина, "--- проворчал Ситрис.
"--- Нельзя так, нехорошо.

"--*Тхэай-кхвай, что же будет делать тактик Тхитронского Храма? "--- осклабилась Кхохо.
"--- Твоему сброду против троих Воинов не выстоять.
Разбойник сбежит, точно тебе говорю.
А Мститель даже на полной мотивации положит от силы двоих.

Ситрис подмигнул мне.

"--*А теперь, Ликхмас, одна из тайн.

Ситрис пододвинул Разбойника к Мстителю.
В следующий момент обе фигурки отправились в ящик и были заменены Воинами.

"--*Хэ! "--- возмутилась Ликхэ.
"--- Не жульничать!

"--*Что произошло? "--- удивился я.
"--- В правилах же сказано, что равный обмен возможен только\ldotst

"--*Только между антагонистами, но необязательно полными антагонистами, "--- объяснил Ситрис.
"--- Эти классы "--- антагонисты по мотивации при прочих равных характеристиках.
Разбойник показал Мстителю смехотворность его мотивов.
Мститель показал Разбойнику, что такое цель и для чего она нужна.

Ситрис посмотрел на Кхохо.
Женщина, казалось, ушла в себя "--- весьма необычное для неё состояние.

"--*Складно поёшь, Ситрис.
А теперь ставь фигуры обратно, "--- буркнула Ликхэ.

"--*С чего бы?

"--*У Разбойника плюс два к беседе.
У Мстителя минус два.
В итоге ноль.
Разговора не было, "--- настаивала Ликхэ.

"--*Кинем? "--- прищурился Ситрис.

"--*Сколько?

"--*Восемнадцать.

"--*Ты чокнулся? "--- вытаращила глаза Кхохо.

Ситрис бросил кихотр.

"--*Вот ты ж, "--- угрюмо шмыгнула Ликхэ.

"--*Воина с Воином рассудит кихотр, "--- заключил я.
"--- Трое против двоих "--- вполне себе честный бой.

"--*Только бросать буду я, а не этот жулик, "--- буркнула Ликхэ.

"--*Ещё чего! "--- возмутился Ситрис.
"--- Нет уж, пусть Фатум бросает.
Фигуры вплотную, а сейчас его ход.

"--*Не подлизывайся к Судьбе, Мятежник, "--- захихикала Кхохо.
"--- Тебе и так везёт до неприличного часто.

\spacing

Ликхэ, огорчившись последнему раскладу, вначале подчистую продула своих Воинов ситрисовским, а затем получила Заговор от Кхохо на столе дипломатии.
Ликхэ попыталась выбросить <<побег>>, но удачи ей явно недоставало "--- девушка вышла из игры насовсем и стала преувеличенно жизнерадостным тоном комментировать ходы.

\spacing

Кхохо изящно поставила чёрный камень на доску и, взяв ещё один, бросила мне точно в нос.

"--*Мир, "--- провозгласила Ликхэ.

"--*Крепись, Ликхмас ар’Люм, "--- осклабилась Кхохо.
"--- В следующий раз ты обязательно\ldotst

"--*Сейчас его ход, "--- напомнил Ситрис.

Я не глядя бросил кихотр.
Восемь.
Запустил руку в мешочек с Благами и Несчастьями и швырнул камень воительнице.
Чума.
Кхохо вытаращила глаза.

"--*Вот не приведи лес тебе выбросить\ldotst

Я подбросил кихотр ещё раз.

"--*\ldotst восемнадцать.

Кхохо задумчиво смотрела на обращённую к ней цифру.

"--*Смерть правителя, "--- так же радостно провозгласила Ликхэ.

"--*Я тебе голову оторву, "--- спокойно пообещала Кхохо.
В следующую секунду доска и камни разом полетели на меня, словно стая взбесившихся птиц.

\ldotst В себя я пришёл только после громкого крика: <<Ликхмас ар’Люм!>>
Я сидел верхом на Кхохо и возил её лицом по каменному полу.
В следующую секунду пришла тупая сильная боль в паху, и я свалился камешком на пол.

Ситрис и Трукхвал растащили нас с Кхохо в разные стороны.
Впрочем, тащить пришлось только меня по той простой причине, что идти я не мог "--- внизу живота разливалась боль, а правая нога не желала разгибаться.
Кхохо же, едва я оказался вне пределов досягаемости, начала деловито проверять на ощупь форму носа.

"--*Ликхмас!
Поднимайся!
Где болит?

"--*А, сейчас, "--- сообразил Ситрис и, подойдя ко мне, аккуратно сжал мякоть бедра.
Щелчок "--- и нога снова начала двигаться.

\mulang{$0$}
{"--*<<Замочек на ножку "--- поползай на брюшке>>, "--- пояснил он.}
{\emph{``Locked paw, you crawl on the floor,''} he explained.}
\mulang{$0$}
{"--- Пах болит?}
{``Groin is aching, isn't it?}
\mulang{$0$}
{Это у Кхохо любимая мишень, она может ударить в пах любой частью тела.}
{Kch\`oh\^o's favourite target, she can attack the groin by each part of her body.}
\mulang{$0$}
{Любой.}
{Each one, boy.}
\mulang{$0$}
{Особенно обожает отсроченные удары.}
{She adores delayed hits most of all.}
\mulang{$0$}
{У меня на случай ссоры с ней есть золотые щитки для\ldotst}
{In case of disagreements I wear a little golden guards on my\ldotst''}

"--*Ситрис! "--- рявкнул Трукхвал, и воин смущённо замолчал.
"--- Приведи молодого жреца в порядок, и пусть он поднимется ко мне.

Трукхвал заковылял на второй этаж.
Таким рассерженным я не видел его никогда.

\section{[-] Гнев учителя}

"--*Ликхмас, ты "--- жрец!
Жрец, а не воин!

Трукхвал не имел большого опыта в упрёках и потому всегда ругался без предисловий.
Я едва успел переступить порог библиотеки.

"--*Это воинам позволительно мять друг другу бока, когда вздумается!
Для них это тренировка!
Жрец должен держать себя достойно!

"--*Это не я начал\ldotst

"--*Азартные игры "--- недостойное жреца времяпрепровождение! "--- Трукхвала, похоже, было уже не остановить.

Я хмуро слушал, классифицируя ругань по затронутым темам.
В точности так, как сам Трукхвал меня учил.
Драка, азартные игры, долг перед городом, честь, драка\ldotst
Повторяется.

"--*Ты проводишь на нижнем этаже чересчур много времени!
Я этим займусь! "--- в сердцах закончил свою тираду учитель.

Трукхвал устало сел за стол и глотнул остывшего отвара из чаши.
Я дал ему отдышаться.

"--*Учитель, "--- осторожно начал я.
"--- Что плохого в том, что я\ldotst

"--*Ликхмас, "--- уже спокойнее заговорил Трукхвал.
"--- Мы "--- хранители мудрости и отправители обрядов.
Мы приносим жертвы Безумным богам.
Серьёзность этого очень слабо сочетается с тем, что я видел сейчас.

"--*Кому какая разница, как я провожу свободное время, если я хорошо выполняю свою работу?

"--*Мне "--- никакой, "--- согласился Трукхвал.
"--- И прочим в Храме тоже.
Но многие крестьяне искренне верят, что у нас есть власть над словом, духами и силами природы, и разрушить эту веру "--- значит лишить их покоя.
Представь, что подумал бы человек с улицы, если бы зашёл в тот момент.
Ему могла потребоваться помощь жреца в лечении или отправлении обрядов, и лично я не доверился бы мальчишке, который михнет назад кормил Кхохо паркетиной.

"--*И что, мне вообще нельзя отды\ldotst

"--*Жрец и отдыхает достойно, "--- перебил меня Трукхвал.
"--- А теперь иди и сшей двести школьных тетрадей по шестьдесят листов.
Вот тебе отдых.

"--*Учитель! "--- возмутился я.
"--- Мы уже равные, между прочим!

"--*Триста, "--- спокойно парировал тот.
"--- Пока не сделаешь, никакого сна.

\section{[-] Вездесущая месть}

От работы учитель освободил меня уже через кхамит.
Работа едва перевалила за двенадцатую тетрадь.
Я так и не понял, чем был вызван странный приступ гнева, а учитель был, похоже, чересчур смущён собственной строгостью, чтобы распространяться на эту тему.

"--*Если соберёшься\ldotst хай\ldotst домой, то отнеси кормилице вот это, "--- пробормотал Трукхвал, сунул мне в руку смятую запечатанную записку и ретировался.

С воинами я столкнулся на лестнице.
Ликхэ едва успела отбить летящий в меня факел (к счастью, незажжённый) и подхватить Кхохо под мышки.

"--*Всё-всё, "--- примирительно сказала Кхохо и, подняв растопыренные ладошки, пошла по своим делам.

"--*Я бы на твоём месте переночевал дома, "--- философски заметил Ситрис, похлопав меня по плечу.
"--- Утром приходи, она к тому времени остынет.

Я вздохнул.
Дом был на другой стороне площади, но тащиться не хотелось.
Вручённая мимоходом записка не казалась весомым аргументом.

"--*Даже не думай, "--- опередил меня Ситрис.
"--- Спать она тебе не даст, хоть в крипте ложись, хоть на улице Стриженого Кактуса.
А вот твою кормилицу она побаивается и к вам в дом не полезет.
Если Кхохо бузит, лучше сразу бежать "--- проблем меньше.

"--*А почему Кхохо побаивается кормилицу?

"--*Кхотлам один раз наложила на Кхохо наказание, как и на меня, чтоб на всю жизнь запомнила.

"--*Боюсь спросить, какое.
Она вообще чего-нибудь боится?

Ситрис ухмыльнулся.

"--*С тобой мелким сидеть.
Если я салфетки полгода вышивал, держался, то Кхохо взвыла через пять дней.
Она детей не выносит.
Всегда удивлялся способности Кхотлам ставить на место воинов.

Ситрис помолчал и недовольно нахмурился.

"--*И подростков.
Надеюсь, это никак не связано.
Кстати, ты Чханэ не видел?
Передай ей, что я её жду возле ткачей на обход.
Третий раз один патрулирую, надоело уже.

Воин вздохнул и скрылся за поворотом.

\section{[@] Беременность}

Плановое обследование преподнесло сюрприз.

"--*Небо, ты беременный? "--- удивился Костёр, только собравшийся снимать мультитомограмму.
"--- Почему не сказал?

Я решил, что он шутит, и засмеялся.

"--*Да какие шутки? "--- возмутился Костёр.
"--- Я чуть <<взрослую>> дозу контраста тебе не дал!

"--*То есть ты хочешь сказать, что я бы не заметил стигм беременности, если бы они у меня были?

"--*Так они и есть.
Посмотри в зеркало.
Хорошие такие стигмы, недельные.

Я посмотрел.
Да, всё правильно, шея полосатая.
Как говорит Баночка, <<тигров ловить собрался>>.
Как я не заметил недельные стигмы беременности, так и осталось загадкой.

"--*Кто отец?

"--*Я не знаю.
У меня ни с кем не было отношений ни во время, ни после перелёта.

"--*Может, партеногенез, "--- предположил Костёр.
"--- Хотя с чего бы, апиды в поселении есть, концентрация феромонов в воздухе достаточная\ldotst
Так, сиди смирно, я посмотрю.
Небо, милый мой, я понимаю, что ты волнуешься.
Расслабься и подумай о чём-нибудь приятном.

Щёлкнул биопсийный пистолет.
Результаты пришли сразу "--- отцом оказался Комар.

"--*Надо же, "--- удивился врач.
"--- Видимо, из-за стресса зародыши впали в анабиоз.
Сиди, сиди.
Двое, развитие идёт удовлетворительно, пороков нет.
Похоже, Небо, что твои дети из титана "--- не всякий зародыш переживёт космический полёт.

Я удивился, насколько мягче стал голос врача.
Он по-прежнему оставался резким и сухим мужчиной-человеком средних лет, но старый инстинкт сработал безотказно "--- с беременными, детьми и больными разговаривать следует предельно осмотрительно.

Костёр дал мне зелёную капсулу и стакан воды.
Я непонимающе уставился на него.

"--*За препаратами и питанием будешь приходить каждый день.
Инъекционных обойм у меня не осталось.

"--*А томограмма?

"--*Через десять-двенадцать дней, на этих сроках контраст может дать повышение гестотропина и резкий иммунный сдвиг, это плохо для личиночной нервной системы апида.

Я попытался возразить, что есть дела важнее вынашивания, но бесполезно.
Костёр был мягок, но непреклонен.

"--*Небо, нас очень мало.
Прерывать здоровую беременность в нынешней ситуации "--- преступление.
Прочие врачи скажут тебе то же самое.

"--*Но это моё тело!

"--*И твой вид, от которого осталась жалкая горстка.
Вас шестьдесят четыре, ты понимаешь?
Шестьдесят четыре.
Тем более это дети погибшего апида, сильной неродственной особи, свежие гены в популяцию.
Дети Комара, в конце концов.
Это для тебя уже ничего не значит?

Я промолчал.

"--*Не надо на меня так смотреть.
В крайнем случае отдашь личинок другим апидам.
Шмель с удовольствием возьмёт, ему и так все детей спихивают.

\section{[@] Лучший воспитатель}

Для Шмеля разбили отдельную палатку.
Самое интересное, что детей-апид здесь почти не было "--- много человеческих, щенята и девочка-плант, которая с важным видом, как взрослая, носила щенка-младенца.
Шмель каким-то невероятным образом успевал их умывать, развлекать и даже учить.
Недостатка внимания не было заметно "--- дети выглядели вполне счастливыми.

"--*Шестерёнка! "--- Шмель подбежал к девочке-планту, погладил её по голове и выхватил у неё щенка.
"--- Давай мне Косточку и иди покушай.
Привет, Небо, подожди.
Заря, смотри, как я умею.

Шмель подхватил другой рукой начавшего кукситься человеческого малыша и начал щекотать его ногочелюстями.
Ребёнок засмеялся.

"--*Я не смогу взять у тебя личинок, Небо, "--- бросил Шмель в сторону.
"--- Ууу, какие у меня усики, смотри!
А это что, носик\ldotsq
Если я буду постоянно отвлекаться на кормление, то могу сразу забыть о сне и пище.
От детей отпуск не возьмёшь.
Как выкуклятся "--- можешь приносить, но не раньше.
Шестерёнка, девочка моя, вытри ротик Камешку, пожалуйста\ldotst

Я кивнул и направился к выходу.

"--*Небо, стой, зараза усатая.
Помоги покормить и поиграй с ними немного, что ли.
Всё одно для тебя тренировка.
Кто пришёл на новоселье, дарит малышам веселье!
Да, дети?

Малыши загомонили и попрыгали мне на руки.
Я чуть не упал.

"--*Небо, ну пожалуйста, давай\ldotst
Что за день, я с ума сойду\ldotst
Почему вы так медленно растёте, а, малышня человеческая?

\section{[-] Достойные}

Записка, которую я принёс с собой, оказалась не из приятных.
Кормилица, едва сорвав печать, пошла пятнами.

"--*Что этот старик о себе возомнил! "--- гневно фыркнула она.

Я решил воздержаться от вопросов "--- Кхотлам впервые отозвалась о Трукхвале в таком тоне.
Кормилица тем временем несколько раз перечитала написанное, словно искала скрытый смысл.

"--*Будешь ты мне\ldotst шантажировать\ldotst "--- бурчала она.
"--- Так и передай своему учителю, чтобы он знал своё место.

"--*На записке стоит печать <<Обменять на ответ>>, "--- возмутился я.
"--- Ты сама учила меня основам дипломатии.
Можешь отправить чистый пергамент, можешь нарисовать рожицу, но устно я от тебя ничего передавать не буду.

У кормилицы появилось на лице выражение, которое я не видел прежде "--- она словно сожалела до глубины души, что когда-то обучила Ликхмаса ар'Люм основам дипломатии.
Впрочем, держалось выражение едва ли секхар.

"--*Ты абсолютно прав, Лисёнок, "--- отсутствующим тоном сказала наконец кормилица, открыла ларец с бумагами, вынула кусочек кожи и отдала мне.

"--*Это Трукхвалу? "--- уточнил я.

"--*Нет, Лисёнок.
Это тебе, "--- Кхотлам погладила меня по голове и ушла к себе в спальню.

Мне?
Что это ещё такое?

Я осмотрел и обнюхал дар.
Кожа выглядела и пахла очень старой.
Кроме того, я не смог определить зверя, которому она принадлежала.
К горлу подкатил ком, у меня появилось странное предчувствие, не дурное, но и не хорошее.

Я подошёл к окну и вгляделся в написанное.

\begin{quote}
<<Ликхмас, милая моя Лисичка!
Если ты читаешь это, меня уже нет в живых.
Я, твоя дарительница, Митхэ ар’Кахр э’Тхартхаахитр, оставила тебя у лучшей подруги, которую ты называешь кормилицей.
Ты был желанным ребёнком, и я бы пошла против всего мира ради того, чтобы растить тебя.
Но мой любимый мужчина был вероломно отдан в рабство.
Я не могла жить, не зная его судьбы.
Нет ничего ужаснее, чем есть, пить и веселиться, когда близкий человек страдает.
Не надеюсь на то, что ты поймёшь, но надеюсь, что ты вырос здоровым, красивым и счастливым человеком "--- мужчиной или женщиной>>.\\~\\
\end{quote}

В самом низу была малозаметная приписка:

\begin{quote}
<<Для Кхотлам: пусть будет кем угодно, но только не воином.
Моя сабля заплатила народу сели на десять поколений вперёд>>.
\end{quote}

Иероглифы за долгие годы стёрлись, и было видно, что чья-то заботливая рука изредка их подновляла.

"--*Лис?

Чханэ зашла в комнату и тут же с беспокойством подбежала ко мне.

"--*Лис, ты как будто жабу проглотил!

Я молча протянул ей записку.
Чханэ, читая, постепенно менялась в лице.

"--*Митхэ ар’Кахр дала тебе жизнь? "--- Чханэ ошарашенно тряхнула головой.
"--- Кихотр\ldotst
Нам про неё столько историй рассказывали, она у нас навроде Маликха была, героем легенд.
Обращала в бегство целую армию отрядом в десять человек, убивала врагов листом юкки\ldotst
И она была Плачущим Ягуаром, едва ли не единственным на всей Короне\ldotst
Хай\ldotst

Чханэ вернула мне кусочек кожи, и я сжал его в руках.
Не то чтобы мир изменился, но я многое увидел в новом свете: недосказанности, странные приступы нежности кормилицы, аура почтения, которая непонятно почему преследовала меня с первого дня в Храме, старания Конфетки сделать из меня воина и ежедневный традиционный гостинец, который тренер клал мне в карман\ldotst

"--*Лис, я понимаю твои чувства, но тебе определённо повезло с семьёй, "--- Чханэ неуклюже попыталась утешить меня.
"--- У тебя достойные кормильцы, и женщина, которая тебя выносила, тоже была достойной, это несомненно.

Я промолчал.

"--*Жаль, что я плохо знаю твоих домашних.
Расскажи о них, "--- Чханэ всеми силами пыталась меня отвлечь.
Мне вдруг стало её невыносимо жаль "--- действительно, кроме меня, больше у Чханэ никого не осталось.

"--*Ха-ай, "--- я улыбнулся как можно теплее, "--- Эрхэ "--- хранительница друга Кхотлам, он попросил позаботиться о ней.
Ликхэ "--- её женщина\ldotst

"--*Это я, как ни странно, и сама поняла "--- живу от них через стенку, "--- весело сказала Чханэ.
"--- У них есть дом?

"--*У Ликхэ есть.
Она передала его брату, у которого много детей.
К нам она попросилась, потому что здесь тихо.
Да и Эрхэ рядом.

"--*А Сиртху-лехэ?

"--*У Сиртху-лехэ нет ни дома, ни родных, ни друзей.
Кормилица подобрала его в сожжённом идолами посёлке, больше никто не выжил.
Он был настолько убит горем, что отказывался есть и требовал хэситр, но Кхотлам его выходила.
Старичок оказался смышлёным и бойким "--- пожил немного на правах гостя, оправился и взял хозяйство на себя.
Сейчас он ведёт у нас учёт товаров.
А ещё он чудесно рассказывает сказки.

"--*А тебе он рассказывал?

"--*Рассказывал.
Манэ и Лимнэ вообще выросли на его сказках.

Чханэ улыбнулась.

"--*Хороший у тебя дом.
Мужчина, женщина, два их ребёнка, две цветущих\footnote
{Цветущими сели называли людей, практикующих исключительно гомосексуальные связи.
Исключительно гетеросексуальных людей называли плодоносящими. \authornote},
ты-приёмыш и одинокий старичок.
Напоминает старые дома, в которых не делали различий между родичами и приёмными.
Сейчас звериное время, люди заботятся только о своих\ldotst

"--*А то и их бросают, "--- ввернул я.

Чханэ смутилась, но возразила:

"--*Если ты про Митхэ, то она сделала всё так, как велит обычай.
Она могла тебя бросить в лесу, но отдала в самые надёжные руки, которые только знала.
Я видела то, что могло тебя ждать, Лис "--- детей, умерших от голода в городе.
Детей, которые умерли от голода, Обнимающий Сит, где ты был?
Больше я этого не видела и надеюсь не увидеть никогда.

Я кивнул и засунул записку матери в карман.
Пусть подождёт своего часа\ldotst

"--*И стариков бросают, "--- задумчиво добавила Чханэ.
"--- Если уж совсем тяжко с едой, даже убивать куда гуманнее.
У нас были случаи "--- стариков и больных жрецы поили лаковым соком, нашествие было, поветрие.
Но бросать\ldotse
В этом мире происходит что-то страшное, Лис.
И хуже всего то, что я происходящее не могу понять.

"--*Ситрис ждёт тебя у ткачей, "--- напомнил я.

"--*Я помню.
Если хочешь, могу посидеть с тобой.

"--*Сначала Храм, потом всё остальное, "--- улыбнулся я и хлопнул подругу по плечу.
"--- Иди, я справлюсь.

Чханэ кивнула и, подхватив заплечный мешок, вышла в надвигающиеся сумерки.

\section{[@] Норма реакции}

\epigraph
{Люди никогда не признают, что свободы воли не существует и что они являются сложными, но всё же механизмами.
Это признание разрушит всё, на чём стоит их маленький уютный мирок "--- заслуги великих деятелей, преступления великих грешников, дружбу, вражду, религию, власть королей и справедливый суд.
Люди будут цепляться за идею непознаваемого и до последнего отрицать своё родство с обычной ветряной мельницей, не понимая, что это родство не унижает, а возвеличивает их.}
{Леам эб-Салах.
Письмо к Анатолиу Тиу, случайно попавшее в сборник <<Записи под молитвенным ковром>>}

Костёр пришёл ко мне в тот же вечер.
Мы с Заяц сидели на песке и любовались звёздами.
Где-то вдалеке плескались и ворковали два дельфина.

"--*Это не дело, "--- сказала Заяц.
"--- Надо сказать дарителям, чтобы большую часть времени дети проводили с ними.
Пусть хоть на работу таскают, там малыши без внимания не останутся, взрослых много.
Шмель "--- лучший воспитатель, но он так долго не выдержит.
Да, врачей у нас тоже мало, но дети требуют больше внимания, чем тяжелобольные!

"--*Они обязательно найдут отговорки, "--- заметил я.
"--- Все заняты\ldotst

"--*Так надо им напомнить, в чьи руки их занятия попадут после, "--- резко бросила Заяц.
"--- Если дети вырастут несчастными, любая работа потеряет смысл.
Привет, Костёр.

"--*Привет, Зайчик, "--- вздохнул мужчина.
"--- Небо, извини.

"--*Перестань, "--- отмахнулся я.
"--- Ты всё сделал правильно.

"--*Нет, "--- возразил врач.
"--- Я посягнул на твою свободу.
Это не есть правильно.

"--*Ради жизни, "--- пожал я плечами.
"--- Апид действительно осталось мало.

Костёр помолчал.

"--*Зря ты применил эту фразу тогда, "--- наконец проговорил он.
"--- Ею можно оправдать самые страшные поступки.
Я под её прикрытием нарушил твоё право на свободный выбор.

"--*И не только ты, "--- вмешалась Заяц.
"--- С первого дня, когда мы вступили на эту планету, мы только и делаем. что нарушаем принципы.
Разве вы не видите?
Мораль тси трещит по швам.
Она не была готова к Катаклизму.
Почему предки не предусмотрели такой возможности?
Да ладно предки.
Почему никто из нас, выросших под угрозой вторжения демонов, не озаботился устойчивостью морального кодекса?
И вообще, что с нами происходит?
Откуда вся эта грязь?

"--*Она в нас была всегда, "--- проворчал Костёр.
"--- У каждого параметра, в том числе и поведенческого, есть норма реакции.
Мы жили счастливо и хорошо, вот все и стали добряками.
Как пришла война "--- оп-па! "--- мы опустились на нижнюю границу нормы реакции.

Костёр сделал выразительное движение ладонью вниз.

"--*Ведь это же\ldotst "--- робко начала Заяц.

"--*Ещё ни один моральный кодекс в истории не выдержал испытания войной, "--- перебил её Костёр, внушительно разделяя слова.
"--- Война показывает одно из двух: либо кодекс негуманен, либо нежизнеспособен.
Где вы найдёте кодекс, в котором освещались бы <<тёмные стороны>> личности?
Взять ту же жестокость.
Что сказано в кодексе тси о жестокости, кроме того, что она приемлема лишь в игровой форме?

"--*Кто-то сублимирует жестокость в спорте, в сексуальных играх\ldotst "--- возразила Заяц.

"--*Этого недостаточно, Зайчик! "--- громко заявил врач.
"--- Почему за все тысячелетия никто не разложил по полочкам <<тёмную сторону>> тси?
И не говорите, что попытки предпринимались.
Да, были исследования, но каждый раз их результаты искажались, сглаживались или стыдливо замалчивались, а о влиянии полученных результатов на воспитание молодёжи и речи не шло!
Почему никто не адаптировал кодекс к ситуации, когда личность скатывается к диаметрально противоположной границе нормы реакции?

"--*И как бы ты построил счастливое общество, если бы принял как данность жестокость, вероломство и трусость? "--- скептически спросила Заяц.

"--*Я не знаю! "--- развёл руками Костёр.
"--- Но если этого не сделать, то любое общество будет ждать такой же конец.

Мы с Заяц поёжились.
Костёр говорил страшные, но, похоже, правдивые вещи.
У меня возникло чувство, что эта идея возникла у него задолго до Катаклизма.
Возможно, он что-то увидел в больницах.
Что-то натолкнуло его на подобные мысли именно там, куда сапиенты поступали вышедшими из строя, где из-за нарушения функции мозга наружу проступали самые древние, звериные качества.

"--*Возможно, что предки всё это знали, "--- предположила Заяц.

"--*Знали и не оставили предупреждений? "--- с горечью сказал я.
"--- Нам никогда не говорили: <<Если вы поймёте, что наше учение неприменимо, отойдите от него>>.
Нет, нам говорили: <<Даже если цивилизация будет на грани гибели, вы должны быть истинными тси>>.
И мы стараемся.
Страдаем, гибнем, но стараемся быть истинными тси.
И ни у кого из тех, кто принёс себя в жертву на Золотой дороге, даже не было мысли, что можно поступить иначе.
Сейчас я завидую Комару "--- он умер, не зная этих ужасных сомнений.

"--*И ведь все погибли добровольно, "--- печально сказала Заяц.
"--- Ты сказал правду: кто-то боялся, но никто не сомневался в том, что это правильно.
А получается, что эти жизни отданы не ради лучшего, а ради того\ldotst ради\ldotst

"--*Ради того, чтобы всё шло так, как идёт, "--- закончил за подругу Костёр.
"--- Мы всю жизнь работаем именно ради этого.

Заяц промолчала и слегка уничижительно посмотрела в его сторону.
Врач грустно улыбнулся и встал, собираясь уйти.

"--*Прервать беременность ещё не поздно, "--- наклонился он ко мне.
"--- Приходи завтра.

Я помотал головой.

"--*Почему?

"--*Потому что ты меня убедил, "--- объяснил я.
"--- Свобода "--- это инструмент, а не цель.

Врач хохотнул и погладил меня по голове.

"--*Не забудь принять лекарства, командир.

\section{[-] Расплата за молчание}

\spacing

Почему она мне ничего не сказала?

Пылая гневом, я пошёл в спальню кормилицы.

Дверь оказалась незапертой.
Кхотлам крепко спала на подоконнике, положив кудрявую голову на руки.
Странная записка выпала из её руки.
Я развернул её и поднёс к узкому клинку предзакатного солнца, пробившего тонкую занавеску.

\begin{quote}
<<Его тянет к воинам.
Он чувствует, что его с ними связывает нечто глубинное, недомолвки и ложь только усугубляют ситуацию.
Если Ликхмас сорвётся, дело может принять плохой оборот.
Ты знаешь, перед кем мы в ответе.
Мне неважно, по какой причине ты до сих пор не сообщила ему правду.
Если ты не сделаешь это сегодня, это сделаю завтра я>>.
\end{quote}

Под запиской не стояла подпись, но я узнал бы почерк Трукхвала, даже не будь я его лучшим учеником.

Я посмотрел на Кхотлам.
То ли она притворялась спящей, то ли действительно спала.
Впрочем, значения это не имело.
Обычная реакция человека, которому страшно боится неотвратимого.
Кормилица ждала наказания, словно провинившийся ребёнок.

Стащив с лежанки одеяло, я накрыл кормилицу и вышел, тихо прикрыв за собой дверь.
Причину я узнаю после.
Расплатой для неё стали прошедшие двадцать михнет.

\section{[-] Лицо в траве}

\spacing

"--*Но ты же кормила меня грудью, как такое может быть?

"--*Кормила, Лисёнок.
Я приложила тебя к груди и появилось молоко.
Потом, когда я уехала по делам, тебя кормил ещё и Хитрам.

Кормилец согласно кивнул.

"--*У меня на несколько лет выросла грудь, "--- сообщил он.
"--- Да ещё такая красивая.
Пожалуй, самое странное, что со мной происходило в жизни\ldotst
Меня облапали все, кому не лень, да, Пёрышко?

"--*Не болтай, "--- кормилица покраснела, и Хитраму прямо в лоб прилетело мокрое полотенце.
Эрхэ и Ликхэ захихикали.

"--*Как она выглядела? "--- продолжал допытываться я.
"--- Я похож на неё?

"--*Да, наверное, "--- уклончиво сказала Кхотлам.

"--*Кормилица!
Скажи её хасетрасем!

Кхотлам потупилась.

"--*Лисёнок, я не помню.

"--*Не ври.

"--*Я не вру.
Много лет прошло.

То же самое говорили мне и прочие люди, знавшие Митхэ ар’Кахр.
Все при этом отводили глаза.

"--*Ликхмас, она пропала тридцать дождей назад где-то на востоке, "--- попытался урезонить меня Ситрис.
"--- Вряд ли кто-то помнит, как выглядела Митхэ.
Да и лицо, которое ты хочешь увидеть, уже давно поросло зеленью.
Ты же читал\ldotsq
Хай, забудь.

Ситрис тоже знал о записке.
О ней знали все, кроме меня.

Впрочем, жизнь текла своим чередом, подобно Ху'тресоааса, могучей полнокровной жиле на теле Тра-Ренкхаля.
Она текла от горизонта к горизонту, и самому острому взгляду кормчего было доступно считанное число извивов.
Жизнь текла по Вселенной, принимая самые неожиданные формы, и хоть законы её течения давно познаны и рассчитаны, наблюдатель никогда не увидит в мутных водах всё многообразие.
Лишь иногда любопытная чаша зачерпнёт воды с чудесной рыбкой, глаза полюбуются на прихотливую игру чешуи, и рыбка отправится обратно за борт, так и не открыв наблюдателю истины "--- чудес в чаше было больше одного.

Прошёл дождь, прошёл другой, и меня захватили дела.
В конце концов, что изменилось?
Кормильцы остались для меня такими, какими и были.
Кхотлам всё так же называла меня Лисёнком, Хитрам с улыбкой рассказывал про дарительницу интересные истории, которые до сих пор кочевали в народе, и мало-помалу Митхэ ар'Кахр превратилась для меня в легенду, какой она была для остальных.
А что до крови\ldotst
Однажды ночью я понял: немного надо, чтобы отдать кому-то каплю своей крови.
Я в последний раз призвал на Митхэ благословение лесных духов, заснул спокойным сном "--- и следующее утро стало первым, когда я о ней не вспомнил.

\chapter*{Интерлюдия V. Постоялый двор <<Бамбуковая клетка>>}
\addcontentsline{toc}{chapter}{Интерлюдия V. Постоялый двор <<Бамбуковая клетка>>}

\textbf{Байки Дальнего Севера.
Апокриф <<Легенды об обретении>>}

Было это в далёкие времена, когда Безумный свирепствовал и брал жертвы не только на алтаре, но и везде, куда дотягивалась его рука, и ни духи, ни дышащие не смели сказать ему ни слова.

Многие города сгорели в беспощадном пламени Безумного, многие люди сошли с ума под ужасом многоцветья.
Особенно ненавидел Безумный мастеров своего дела и книжников, ибо они учили прочих избегать несчастий.
Ненавидел он и тех, кто знал его под именем Безумного и писал это имя ореховыми чернилами и сурьмой.
Много книг сгорело в огне, и забывать стали люди о значении старых иероглифов, а уж писать и вовсе разучились.
Духи ещё помнили старые языки, но Безумный наложил на них страшное проклятие "--- никто из духов отныне не смел учить языкам живых.

Было это так.
Собрались однажды духи на совет.
Немногие пришли "--- в камень обратились те, что потеряли цель и смысл;
побледнели и прозрачными стали, как воздух, что были подзабыты живыми.
И сказал им Безумный:

<<Вечное молчание падёт на тех из вас, кто скажет одно слово на двух языках, кто скажет слово и напишет его иероглиф, кто скажет слово в ответ на иероглиф, кто напишет иероглиф слову в ответ.
И лишь в тот день, когда дух и живой разделят ложе, когда влюблённый презреет свои чувства, когда речные воды станут солонее морских, в руках дышащего окажутся пергамент, перо и чернила!>>

Хотел Безумный создать новый мир для себя, ибо нет такого разрушителя, который не почитает себя величайшим из творцов.

Но был среди духов старец;
он был бледный и прозрачный, словно знойная дымка, и было его слышно лишь тогда, когда все замолкали\footnote
{Забытый "--- персонаж многих историй.
Ранние сели верили, что забытые потомками больше не существуют, и Забытый служил собирательным образом тех, чьи имена и деяния были утрачены.
В <<Легенде об обретении>> Забытый исчезает навсегда;
из-за этого Легенду критиковали, и даже были попытки изменить её.
Но автор, согласно <<Хроникам дорог и ветров>>, высказался ясно: <<Я рассказал то, что рассказал.
Вам придётся это принять>>. \authornote}.
Сказал старец:

<<Нет среди нас веры твоим угрозам;
священен лишь договор.
Поклянись же, что в городах и деревнях, в лагерях и становищах будет твоим наделом лишь алтарь, покуда народ исполняет твою волю>>.

Оскалился Безумный, и загрохотал гром, и засверкали молнии.
<<Клянусь>>, "--- захохотал он и бросил кихотр на игральную доску.
И пропал старец насовсем, ибо выпало на кихотре Забвение.

Взмолились духи сестре Воде:

<<Сестра Вода, ты дала начало всякой твари, и даже в мёртвых костях есть твоя неумирающая суть.
Ты разбиваешь скалы и протекаешь сквозь гранит;
ты остужаешь кровь земли, и приносишь почву, и взращиваешь зелёные злаки на ужасной той крови.
Если ты не найдёшь дорогу в проклятии Безумного, никто не сможет её найти>>.

Приостановила сестра Вода бег двух рек\footnote
{Ху'тресоааса и Ху'микхаса. \authornote},
и дожди отложила "--- думала.
Как учить, если не иероглифом к слову, если не словом на иероглиф?
И нашла-таки сестра Вода способ.
Осталось лишь найти достойного ученика.

И увидела как-то сестра Вода молодого рыбака, что чертил палочки у самого берега.
<<Вот он "--- ученик>>, "--- решила она.
Но, подойдя ближе, сестра Вода обнаружила, что мужчина красив и улыбчив, как Хри-соблазнитель.
Воспылала сестра Вода к мужчине страстью, вмиг забыв про свои дела;
приняла она облик женщины дивной красоты и вышла к рыбаку из родной стихии.

<<Что ты делаешь?>> "--- спросила она.

<<Считаю рыбу>>, "--- ответил рыбак.

<<Не желаешь ли ты научиться чтению и письму?>>

<<Зачем мне это?>>

<<Ты можешь обрести мудрость предков, чтобы ловить рыбу>>, "--- сказала сестра Вода.

<<Чтобы ловить рыбу, мне достаточно мудрости моего наставника>>, "--- ответил рыбак.

<<Ты можешь познать тайны звёзд, гор, морей и сельвы>>, "--- сказала сестра Вода.

<<Мне хватает тайн реки>>, "--- ответил рыбак.

<<Ты можешь вложить свою память в пергамент, и она не покинет тебя>>, "--- сказала сестра Вода.

<<На память не жалуюсь>>, "--- белозубо усмехнулся рыбак.

Совсем потеряла голову сестра Вода.

<<Я стану твоей, и буду с тобой столько, сколько ты захочешь>>, "--- сказала она.

Расхохотался мужчина.

<<Мне хватает и тех женщин, что не измазаны в тине и не опутаны водорослями>>, "--- сказал он, собрал свои снасти и ушёл.

Обиделась сестра Вода, вздулись воды Ху'тресоааса от её гнева.
Дождавшись, когда выйдет мужчина на рыбалку, подхватила она его утлую лодку, вынесла к Ихслантхару и разбила о Мраморный мыс\footnote
{Мраморный, или Кошачий, мыс "--- опасная зона судоходства, место слияния Речек-близнецов, Со'хата и Со'хатра. \authornote}.
И рыбе наказала, чтоб держалась подальше от его сетей.

Пришёл рыбак на постоялый двор, что на перекрёстке трёх дорог.
Мокрый весь, грустный.
Шутка ли "--- саму сестру Воду, не признав, оскорбил!

Подошла к рыбаку хозяйка двора.
Не была она так красива, как сестра Вода, но веяло от неё прелестью.

<<Принеси мне что-нибудь поесть>>, "--- попросил её рыбак.

<<Всё, что подают в этом дворе, есть наверху>>, "--- ответила хозяйка.

Взяла она мужчину за руку и повела наверх, где был коридор и пять комнат.

<<Только знай: ты не можешь отказаться ни от чего, что дают в этих комнатах>>, "--- сказала хозяйка.

<<Я понял>>, "--- сказал рыбак.

Завела хозяйка рыбака в первую комнату "--- был там кузнечный горн с огненными лисами.
Вспрыгнули лисы на колени рыбака, потоптались, потёрлись "--- и согрелся он, и высохла его одежда.

Завела хозяйка рыбака во вторую комнату "--- там был стол, полный яств.
Поел рыбак от души, выпил вина "--- но не исчезли яства со стола, и даже как будто прибыли.

Завела хозяйка рыбака во третью комнату "--- там стояла цитра-самозвон.
Заиграла цитра, и утешился рыбак, и ушла его грусть.

Завела хозяйка рыбака в четвёртую комнату "--- стоял там один чан с рыбьим жиром, и котёл стоял для выварки рыбы.
Понял рыбак, чем жить ему без лодки и удачи на воде.

Вывела рыбака хозяйка и указала ему дорогу наружу.

<<Подожди, "--- сказал ей рыбак, "--- покажи мне пятую комнату>>.

<<Но разве ты не получил всё, что хотел?
Ты выглядишь счастливым>>, "--- удивилась женщина.

<<Я счастлив, "--- сказал рыбак.
"--- Но эти чудеса раздразнили меня, и я не успокоюсь, пока не увижу пятую комнату.
Что в ней?>>

<<В первой комнате "--- твоя самая очевидная, насущная нужда.
В последней комнате "--- то, что придаст твоей жизни новый смысл или, возможно, уничтожит тебя.
Я не знаю, что ты там найдёшь, а даже если узнаю, то не смогу до конца понять значение дара.
Это только для тебя>>.

<<Покажи мне последнюю комнату>>.

<<Помнишь ли ты, что не можешь отказаться от моих даров?>> "--- спросила хозяйка.

<<Помню, "--- ответил рыбак.
"--- Все твои дары пришлись мне по душе.
Сомневаюсь, что я откажусь>>.

Завела хозяйка рыбака в пятую комнату.
Там были очаг, лежанка, фонарь и чаши с водой.
Села хозяйка на край и позвала рыбака к себе, и провели они ночь вместе за беседами и любовью.

Выйдя же наутро из постоялого двора, забыл рыбак обо всём, что было.
Остались при нём лишь сухая одежда, сытость, хорошее расположение духа, мысли о новом его ремесле и свербящее чувство того, кто только что ушёл с ложа лучшей из женщин "--- без надежды на возвращение.

Вернулся он в родную деревню и стал вываривать рыбий жир.
Но чувство недостающего не покидало его.
Знал рыбак, что влюблён, но не помнил имени;
видел рыбак женский образ "--- то в вечерних тенях, то утром, на берегу пробуждения, "--- но не видел лица;
слышал рыбак мягкий голос "--- усладу и отдых для ушей, "--- но не помнил ни слова.
Говорили в деревне, что угодил молодой мужчина в Беспамятный постоялый двор, что зовётся <<Бамбуковой клеткой>>.
Понял рыбак, что подвела его память;
вспомнил он о словах сестры Воды.

И однажды пришёл рыбак к реке снова, и стал звать сестру Воду.
Она откликнулась на его зов, хоть всё ещё была немного сердита.

<<Зачем ты пришёл? "--- спросила она.
"--- Клянчить у меня рыбу?>>

<<Мне не нужен больше улов, я теперь вывариваю рыбий жир>>.

<<Тогда что тебе нужно?>> "--- удивилась сестра Вода.

<<Я приму твои дары, "--- сказал рыбак.
"--- Твою любовь, искусство чтения и письма>>.

Расцвела сестра Вода в тот же миг, и запахла река ароматом ночных кувшинок.

<<Я не могу ответить иероглифом на слово, и словом на иероглиф не могу, "--- сказала сестра Вода.
"--- Но средство есть.
Возьми книгу, разруби иероглифы на кусочки и дай каждому кусочку имя.
Тогда я смогу тебя учить>>.

И разрубил рыбак иероглифы на ключи, и дал имена всем ключам, и стала сестра Вода рассказывать ему истории слов ключами.
Чувствовал Безумный, что неладное творится на земле, что трещит порядок его, но поделать ничего не мог.
Спросит Безумный у рыбы, а та разбежалась <<листом>> да <<хвостом>>, спросит Безумный у воды, а та <<змеёй>> да <<колодцем>> обернулась.
Неистовствовал Безумный, поднимал облака тьмы над северными горами, крушил твердь, метал молнии, сыпал пеплом с небес\footnote
{Речь об извержении супервулкана Закованный Пик, Лук 8126 (дата подтверждена Лабораторией Катастроф, отдел геологии и классической планетологии Ордена Преисподней). \authornote}, да так ничего и не узнал.
Не нарушила сестра Вода его наказа, но научила рыбака грамоте.

Прониклась глубоким чувством сестра Вода, пока учила мужчину читать.
Но проговорился мужчина, что учится читать и писать, чтобы разгадать тайну <<Бамбуковой клетки>>.
И поняла сестра Вода из его слов, что пылает он чувством к той, кого не может вспомнить.
Ревность взяла её.

<<Уходи, "--- сказала она ему, разразившись плачем.
"--- Забирай то, что я тебе дала, и больше меня не тревожь>>.

Поблагодарил рыбак сестру Воду, попрощался с ней и пошёл в Ихслантхар, к перекрёстку трёх дорог.

Хозяйка двора встретила его приветливо, словно незнакомца.
Взглянул рыбак на её лицо "--- и понял, по кому страдал всё это время, но не сказал ей ни слова.
И лишь одна мысль была у него "--- унести в своей памяти то, что он понял.

Взяла хозяйка мужчину за руку и повела наверх, где была всего одна комната.

Завела его хозяйка в комнату.
Там был стол с пером, чернильницей и свитком пергамента.
Остолбенела хозяйка;
ударила она в ярости по столу, и покатился письменный прибор по полу.

<<Что за люди! "--- закричала женщина.
"--- Я давала вам всё, чего вы хотели "--- тепло, пищу, любовь, новую жизнь, "--- забирая лишь память о нескольких кхамит.
Но вы упорно пытаетесь вернуться за ней!
Хочешь свою память обратно?
Забирай!
Забирай вместе со всем, что я тебе дала, и больше меня не тревожь!>>

Отдала хозяйка перо, чернильницу и свитки рыбаку, вернула ему память и вывела наружу.

С тех пор постоялый двор пропал.
Но не бесследно "--- желание помочь в хозяйке было сильнее всяческих обид.
Случалось ли вам видеть прекрасные сны, которые утекали из вашей памяти, едва вы проснулись?
Знайте "--- это открылись двери постоялого двора у перекрёстка трёх дорог.
И так же, как и в былые времена, там есть комнаты для любых бед и проблем.
Однако у спящих не найдётся пергамента, чтобы всё записать, и тайна комнат хранится крепко до сих пор.

Смирился рыбак с тем, что не видать ему милой хозяйки <<Бамбуковой клетки>>;
прочёл он много книг, пытаясь забыть свою нечаянную страсть, и стал книжником, и переписал старые книги, и научил людей заново этому искусству.
А Безумный смирился, как смиряются перед природой жизни даже самые неистовые ветра, даже самое беспощадное поветрие, даже жгучая кровь земли.
Не был Безумный умён, как сестра Вода, и не нашёл он лазейки в связавшей его клятве.

Смирилась и сестра Вода, хоть плакала она от обиды и в тот год, и в следующий;
и солона была Ху'тресоааса, и снег у подножья Серебряных гор был солон, и даже дождь отдавал горечью в те года\footnote
{Горькие дожди "--- период после извержения супервулкана Закованный Пик, нашедшего отражение в
сказаниях многих народов. Ориентировочно Большая Капля 8126 "--- Пирог 8129 (дата подтверждена Лабораторией Катастроф, отдел геологии и классической планетологии Ордена Преисподней). \authornote}.

\chapter{[-] Горькие дожди}

\section{[-] Разлом}

\epigraph
{Чем шире распахнута дверь, тем раньше видно друга и врага.}
{Пословица сели}

\spacing

В тот дождь снова активировались землетрясения.
Тревожные сообщения поступали и с севера, и с запада.
Но вскоре беда прошла прямо у нас перед носом "--- расширился разлом Гибельных Руин.
Ворчливая земля обрушилась на ближайшие деревни хака, а игривые ветры понесли смертоносные испарения на восток.
Хака подняли зелёное знамя\footnote
{Зелёное знамя "--- знак насильственного переселения. \authornote}.

Мы тем временем жили так же, как и всегда.
Это устраивало всех, кроме кормилицы;
она выглядела обеспокоенной.
Вскоре она созвала Советы Сада и Цеха.

"--*Хака нужно помочь, "--- заявила она.
"--- Я бы направила к ним обозы с провизией и тканями.
Да, я знаю, многие из вас считают, что это не наша проблема.
Моё мнение "--- это временно.
Проблема рискует стать нашей уже в ближайшую пару дождей.

Старшиной Сада была пожилая юркая женщина.
Её кипучей энергии хватало на целых шесть кварталов помимо огромного дома и обычной крестьянской работы "--- признак человека, пережившего тяжелейшие лишения в детстве.
Она лаконично резюмировала общее решение двух Советов:

"--*Каким бы образом проблема ни стала нашей, провизия и ткани пригодятся нам самим.

"--*Вот так вот, "--- сказала Кхотлам за ужином, завершая рассказ о заседании.
"--- Мне уже давно следовало написать хака.
А что писать теперь?
Я не знаю.

"--*Но Саритр права, "--- пожала плечами Эрхэ.
"--- Избыток может пригодиться в любой момент.

"--*Попросят "--- другое дело, "--- поддержал женщину Хитрам.
"--- Самим-то зачем отправлять?

"--*А я бы отправила, "--- заметила Ликхэ.
"--- Проблемы проблемами, но люди голодают.
От нас не убудет.
Кроме того, хака иногда устраивали набеги, но вот такие долги всегда возвращали.
Для них бесчестье принять помощь и не отплатить тем же.

Сиртху-лехэ промолчал, но всё было ясно и так "--- к хака и идолам Живодёра у старика были личные счёты.
Кормилица посмотрела на меня:

"--*Что скажет Храм?

"--*Очень невежливо было бы написать <<попросите нас>>, но другого выхода из ситуации я не вижу, "--- признался я.
"--- Если прибудут делегаты, Сад и Цех уступят, хотя бы сделают скидку.

Кхотлам кивнула и встала из-за стола.

"--*Всегда есть способ выразить вежливо даже невежливую мысль, Лисёнок.

\begin{quote}
<<Мы слышим о вашем горе.
Мы знаем силу духа народа хака.
И если телу народа понадобится пища и одежда, вам нужно лишь окликнуть ближайших соседей>>.
\end{quote}

\section{[-] Ответ}

\spacing

Ответ от хака пришёл спустя три рассвета.

\begin{quote}
<<Мы готовы купить у вас половину избытка с полей, и половину избытка ткани, и половину запасов строительной древесины по обычной цене.
Караван выступает через четыре рассвета после написания этого письма.
Просим распорядиться о складировании и сортировке как можно быстрее>>.
\end{quote}

Кормилица, получив его, весь день хмурилась и мяла кусок пергамента в руке.

"--*Интересно, откуда у хака столько золота.
И почему четыре\ldotst
Так, повременим-ка пока со складированием\ldotst

\section{[-] Игры не будет}

\spacing

"--*Ну что, Ликхмас-тари? Может, ещё партейку в Метритхис? "--- ухмыльнулась Кхохо.
"--- Надеюсь, ты не разобьёшь себе губу и не найдёшь потерявшуюся в джунглях тратра\footnote
{Тратра (дикий цатрон) "--- термин биологического родства, дарительница третьего прародителя по мужской линии.
В племенах царрокх, а затем хака и тенку те, кто достиг статуса тратра, считались самыми уважаемыми женщинами, талисманами племени;
иногда их авторитет даже превосходил авторитет старейшин. \authornote}.

Я подмигнул Кхохо и сел за стол.

"--*Очень хороший игрок, "--- отрекомендовала меня Кхохо.
"--- Только иногда мне ему лицо разбить хочется.
Так, расставляем фигуры\ldotst
Молоток, я видела, как ты засунул камень в рукав.
Положи обратно, мелкий жулик.

Крестьянин смущённо улыбнулся щербатой улыбкой и положил камень в горшок.

"--*Зелёный, "--- вздохнула Кхохо.
"--- На кой ты взял зелёный, дурачок?
Ты же Сосед, а не Король!

"--*Я Король! "--- возмутился Молоток.

"--*Да? "--- Кхохо задумчиво посмотрела на стол.
"--- Точно.
Что-то я замоталась на работе.
Да чтоб вас всех, я фигуры не так расставила, я ж думала, что ты Сосед\ldotst

Вдруг в дверях таверны появилась тень и, показав какой-то знак, мотнула головой.

"--*Меркхалон-кровопийца, опять работа, "--- выругалась Кхохо и подхватила лежавшую рядом серебряную саблю.
"--- Пошли, Ликхмас, сегодня игры не будет.

\section{[M] Вкус броненосца}

\spacing

В глазах всех, кого мне пришлось убить, было одно "--- страх и слёзы.
Я убивал, чтобы кормилица спала спокойно, а сестрёнки утром шли за водой, распевая песенки.

Вспомнился один день из моего детства.
В тот день погиб мой брат.

Я, возрастом пятнадцати дождей, стоял на стене Тхитрона со взведённым <<детским>> арбалетом.
Детским он назывался, разумеется, из-за размера "--- убойность его не уступала лукам взрослых.
Кхотлам стояла рядом.
Мы не отрываясь смотрели на границу леса.
Там было тихо, но мы знали "--- деревья нашпигованы идолами, как пирог яблоками.
Многочисленный отряд Молчащих идолов прошёл по северным землям хака, как стая степной саранчи.
Половина земель Травинхала курилась дымом запустения, теперь пришёл черёд Тхитрона.

Первыми до ворот дошли беженцы хака.
Мы с Кхотлам застали эту новость дома, за едой.
Кормилица бежала к воротам, словно её гнали горящими палками.

Воины Храма, что удивительно, стояли гурьбой и жаром переговаривались.
Увидев Кхотлам, все разом замолкли.
Вождь вышла вперёд.

"--*Кхотлам, я, кажется, сказала тебе\ldotst

"--*Впустить их, "--- без предисловий приказала кормилица.

"--*Отбой, "--- бросила вождь.
"--- Отправьте им сигнал\ldotst

"--*Я сказала "--- впустить их, "--- рявкнула кормилица, уткнув в лицо вождю <<птичью лапу>>.

Купец имел право в особых случаях наложить вето на приказ вождя.
Кхотлам воспользовалась этим правом первый и последний раз в жизни.
Несколько воинов бросились выполнять приказ.

Ворота открылись, и унылая цепочка повозок заехала в город.
Пугливые женщины, дети с серьёзными взрослыми глазами, сгорбленные старики --- я никогда не видел ничего более внушающего жалости.
Раненых, что необычно, не было совсем --- возможно, хака оставили их.
Немногочисленные мужчины тут же, не задавая вопросов, поднялись вместе с нами на стену и взяли луки на изготовку.

Чтобы снять моё напряжение, Кхотлам спросила:

"--*Ликхмас, о чём думаешь?

Конечно, я ненавидел идолов.
Чистой, детской ненавистью.
Детям часто говорили: Молчащие "--- непримиримые, смертоносные враги.

"--*Я думаю о том, как убить больше врагов.

"--*А знаешь, о чём думаю я, когда целюсь?

Я покачал головой.

"--*Я думаю о том, что после я куплю на рынке самого большого и жирного броненосца и приготовлю его, а Хитрам, Хмурый и ты будете сидеть за столом, уплетать его и облизывать пальчики, "--- прошептала Кхотлам, прищурилась и отпустила тетиву.
Застонал разорванный воздух, с далёкого дерева упало верещащее зелёное тельце, ударилось о землю и замолчало навсегда.

Кормилица наложила новую стрелу на тетиву и продолжила как ни в чём не бывало:

"--*Почему ты ненавидишь их, Ликхмас?
У тебя есть повод?

"--*Многие наши люди убиты идолами, "--- сказал я и, заметив в кроне движение, нажал на спуск.
Стрела исчезла в листве, не причинив никому вреда.

"--*Те, кто носят в ушах перья согхо и раскрашивают лица фиолетовой краской, думают прежде всего о покое для умерших близких.
Хотя\ldotst я не думаю, что покой мёртвых зависит от судьбы их убийц.

Ещё одна стрела Кхотлам нашла свою цель.
Мучения легко раненного идола завершила другая стрела.

"--*Хай, о чём мне думать во время боя? "--- спросил я.

"--*Думай о моём броненосце, "--- весело предложила кормилица и выпустила стрелу.
На границе леса осталось лежать ещё одно зелёное тело.

Идолы вскоре смекнули, что зря теряют товарищей.
Вскоре через лужок к стене пошли построенные <<черепахой>> отряды.
Воздух наполнился стрелами, копьями и пращными снарядами.

Вождь зычным голосом велела воинам спуститься за стену.
Уязвимое место идолов "--- бой на открытой местности, строй на строй.
Приказание вождя повторил большой <<осадный>> барабан.

Брат стоял в первых рядах.
Про Саритра говорили, что он похож на большого старого ягуара.
У него была чересчур массивная челюсть, маленькие, близко посаженные глаза и тонкий кривой нос.
Он был нелюдим и очень лаконичен на словах.
При этом мало кто мог выстоять против него один на один "--- его ловкость и силу тоже сравнивали с кошачьими.

Я же знал этого сурового (в его-то тридцать дождей) воина как доброго Хмурого, который учил меня мудрёным приёмам, рыбалке, а ещё таскал для меня у рыночных торговок сладкие конфеты.
Старшие, бывало, поколачивали младших за шалости.
Хмурый за всю свою короткую жизнь не тронул меня и пальцем "--- даже на шуточных поединках он меня щадил, за что кормилец ему выговаривал.

"--*Испортишь брата, дитя.
Он должен привыкнуть к ударам, враг не будет его жалеть!

Нелюдим Хмурый улыбался толстыми губами.
Уж он-то знал, как мне доставалось от Конфетки.
Брат легонько лупил меня деревянной саблей, я притворно сгибался и корчился.
Хитрам отвешивал нам обоим подзатыльники, больше похожие на ласку, и возвращался к делам.

\ldotst Идолы приближались под градом стрел и снарядов.
<<Черепахе>> эти кусочки дерева и камня не причиняли никакого вреда, но и вести ответный огонь идолы тоже не могли.
Когда первый отряд подошёл на расстояние выстрела из духового ружья, воины сели, выставив щиты, бросились в атаку.

Саритр влетел во вражескую <<черепаху>> первым.
Удар его щита помял строй идолов, покорёжил черепаший панцирь.
Он рубил и кромсал зелёных карликов как одержимый, как вдруг идолы поменяли строй.
Из-за выстроившихся полумесяцем щитов высунулись духовые ружья.

Новая тактика идолов стала для людей полной неожиданностью.
Огонь со стен не причинял полумесяцу никакого вреда, а с его внутренней, полуоткрытой стороны в растерявшихся воинов полетели яркие оперённые стрелы.
Саритр вместе с цветом авангарда "--- их было человек пятнадцать "--- упал, получив дозу кураре.
Две <<черепахи>> неожиданно сошлись, сомкнули щиты над упавшими и быстро понесли их к лесу.
Рванувшихся вслед за ними воинов сели встретили бамбуковые дротики и шквальный огонь с деревьев.

"--*Назад! "--- выкрикнула вождь, осознав свою ошибку.
Барабан повторил её приказ.
Сели построили у стены <<черепаху>> и замерли.

"--*Кормилица!
Они схватили Хмурого!

Я в отчаянии выпускал одну стрелу за другой, но все они отскакивали от щитов и вонзались в землю.
Кхотлам стояла, натянув тетиву, и по её лицу текли слёзы.

"--*Кормилица, почему ты не стреляешь\ldotsq

Она стояла и смотрела\ldotst
Один за другим стрелки прекращали бесплодные попытки стрельбы.
Наконец, когда <<черепаха>> с пленниками уже почти преодолела расстояние до леса и начала разваливаться, Кхотлам спросила меня:

"--*Помнишь, я говорила тебе про броненосца?

"--*Кормилица\ldotse

"--*Он будет таким же вкусным и большим, как я обещала, "--- прошептала Кхотлам и,
внезапно подняв лук, спустила тетиву.

Стрела вошла <<навесом>> в полупядевую щель между щитами.
Идолы заверещали, но это был крик разочарования, а не боли.
Черепаха прошла, а один из захваченных людей остался лежать на земле со стрелой в шее.

Мой добрый брат Саритр.

"--*Никто из моих детей не умрёт на алтаре, "--- тихо, но твёрдо прошептала кормилица.

\razd

Ночь выдалась тихой.
Идолы развернули лагерь где-то в лесу, и до стражи доносились их весёлое верещание и детские голоса.
Словно насмешка над плачущей песней пятнадцати человек, вставших кругом на улице недалеко от нашего дома.
Кхотлам не пошла к ним.
Она сидела у окна и молча слушала, перебирая бусины занавесей.

Вождь хака и несколько самых уважаемых воинов ночевали в нашем доме.
Перед сном они долго сидели на полу гостиной, попыхивая длинными трубками.
Я хорошо запомнил их грубые горбоносые лица, освещённые слабыми бликами очага.

"--*Скажи мне, Кхот-Трис ин ур'Лим, "--- сказал вождь, "--- как получилось, что хака всё ещё живут между землями сели и Молчащими лесами?

Кормилица всегда была на редкость женственной и нежной;
однако вождь хака произносил её имя на мужской лад и говорил о ней в мужском роде "--- видимо, чтобы не потерять лицо перед своими воинами.

"--*Если хака не жгут наши поля и не убивают наших сыновей, то они хорошие соседи, "--- сдержанно ответила Кхотлам у окна.
"--- По каким соображениям вас терпят идолы, мне неведомо.

"--*Ты прекрасно знаешь, что уже завтра я могу прийти вместе с союзом племён как враг, "--- сказал вождь.

"--*Тогда я убью тебя и вывешу твою голову у ворот, Си-Жак, "--- без обиняков посулилась кормилица.

"--*Это будет честно, "--- согласился вождь.
"--- Если бы не Кхот-Трис ин ур'Лим, Инхас-Лака сейчас гнили бы на родных пепелищах.
Мне жаль твоего сына.
В его чреслах таилась дивная мощь.

"--*Я позабочусь, чтобы вам выдали достаточно пищи на сезон.
У нас избыток.

"--*Мы пришлём вам дары чести, когда это будет возможно, "--- поклонился Си-Жак.

"--*Лучшим даром чести будет, если Инхас-Лака останутся дома во время следующего похода хака на сели.

Вождь дёрнул проколотой губой с клыком бородавочника и бросил взгляд на сидящих рядом воинов хака.
Предложенное Кхотлам означало бесчестье, как, впрочем, и отказ от предложенного.
Этого кормилица и добивалась "--- вождю придётся сделать выбор.

\razd

Подошедшая утром подмога с Хатрикаса не нашла уже никого "--- аборигены Молчащих лесов снялись ещё до рассвета и ушли восвояси.
Несколько деревень были опустошены накануне, и пленных для жертвоприношений они получили предостаточно.

Хмурого похоронили в лесу.
Хитрам выбрал самое сильное и узловатое чёрное дерево и нарисовал на его коре знак Сана-сновидца.

"--*Он был воплощением силы, "--- сказал отец ритуальные слова.
"--- Если его посмертие не будет лёгким и спокойным, то нам да будет отплачено впятеро.

Кормилица, когда все ушли, аккуратно дорисовала рядом Сита.
Однако её руки дрожали, и краска на веках Сита потекла.
Улыбчивый любвеобильный дух заплакал горькими чёрными слезами.

Броненосец и вправду получился очень вкусным, но сильно пересоленным.
Мы с кормильцами сидели за низеньким столом и молча уплетали нежное мясо.
Под конец трапезы Кхотлам нарушила молчание.

"--*Пообещайте одну вещь.
Ни один из вас не наденет перьев согхо.
И ничьего лица не коснётся фиолетовая краска.

Хитрам ар’Кхир происходил из семьи крестьян.
Чтить законы сели было у него в крови.

"--*Ты сама понимаешь, что это невозможно, Пёрышко, "--- проворчал он.
"--- Хмурый не упокоится, если\ldotst

"--*\dots если не погибнет соплеменник его убийцы? "--- закончила Кхотлам.
"--- Стрела, убившая наше дитя "--- моя стрела, Пловец.
Кому ты собрался мстить?

"--*Хаяй, этого не может быть, "--- Хитрам неверящим взглядом смотрел на кормилицу.
Она смотрела в тарелку.

"--*Никто из моих детей не умрёт на алтаре, "--- повторила она.

\dots Кормилец ушел, не закончив еду.
Река вышла из берегов и вернулась восвояси, а мы не услышали о Хитраме ничего.
Кхотлам занималась своими делами "--- нанимала носильщиков, распределяла товары, "--- но делала это всё отстранённо.
Она рассчитала всех посыльных, оставив только милую крестьянку Эрхэ "--- её дед был другом детства Кхотлам.
Она перестала готовить и заниматься рукоделием.
По вечерам кормилица долго сидела у моей лежанки и гладила меня по голове.

"--*Одни мы с тобой, Лисёнок.
Совсем одни.
Что ты один, что я одна\ldotst

Я играл с волосами кормилицы и говорил, что, когда вырасту, то увезу её из Тхитрона в спокойное место.
Она грустно смеялась, целовала меня и уходила к себе.

Однажды в дверь постучали среди ночи.
Шёл проливной дождь.
Я по привычке соскочил с лежанки и спрятался за занавесями, наблюдая за входом.

За тяжёлыми резными дверями чёрного дерева стоял Хитрам.
По его длинным каштановым волосам стекала вода, вода стекала и по лицу, на котором прибавилось свежих шрамов.
Он стоял молча, опустив голову, и держал в руке промокший букетик розовых орхидей.

Кхотлам, не сказав ни слова, втащила его в дом, отвесила Хитраму звонкую пощёчину и поцеловала в губы.
Я робко вышел из-за занавесей, и кормилец, утирая с лица слёзы, перемешанные с дождевыми каплями, подбежал ко мне и поднял на руки.

В ту ночь были зачаты мои сёстры.

Больше Хитрам нас не покидал.

\section{[@] Ещё люди}

Экспедиция на Корону принесла неожиданный результат: мы нашли людей.

Я спросил Безымянного, почему он нам про них не рассказал.
Бог ответил: <<Я не знал, что они тоже относятся к людям>>.
Если честно, странная отговорка.
Биологи на всякий случай попросили его подробнее рассказать о формах жизни на планете "--- мало ли кто тут ещё скрывается!

<<Царрокх>>, как себя называет найденное племя, "--- потомки людей Древней Земли, живших незадолго после Последней Войны.
В пользу этого говорит чрезмерно развитый половой диморфизм (сравнительно высокие мускулистые мужчины и пухлые женщины, едва превышающие ростом плантов), недосбалансированный геном и отсутствие некоторых идиоадаптаций, характерных для сапиентов поздней Эпохи Богов.
В частности, они используют исключительно оральную артикуляцию, а их пол определяется в момент образования зиготы.

Баночка сейчас сидит у меня расстроенный.
Царрокх категорически отказались общаться с ним.
Людей-тси дикое племя называет <<сели>> "--- дословно <<лесные золотые люди>>.
Сначала историки предположили, что это из-за имплантов.
Потом выяснилось, что импланты царрокх приняли за украшения, а <<золотыми>> людей назвали из-за цвета кожи (у аборигенов кожа коричневого цвета).
Особенно им понравилась ласковая красавица Гладит-Зелёную-Кошку "--- её пускали в любой дом и позволяли задавать любые вопросы.
Она и собрала большую часть информации о племени.

Кани, апид и плантов царрокх называют <<кхерр>> (понятия не имею, что это, но звучит очень неприятно) и ведут себя с ними трусовато-враждебно.
Баночка рассказал, что проникновение тси в святое место царрокх спровоцировало столкновение, и наполовину состоящий из биокибернетики Фонтанчик вызвал ужасный переполох.
Сам Фонтанчик отозвался кратко: <<Обошлось без жертв>>.
Видимо, были покалеченные, но всех удалось спасти.
Фонтанчик сказал, что больше к ним не поедет.

Не нравится мне это всё.

Поправка: Баночка только что по контексту нашёл возможное значение слова <<кхерр>>.
Это философское понятие Древней Земли, обозначающее умышленный вред или абсолютного, непримиримого врага.
Оно является антонимом другого понятия "--- намеренное полезное действие, безусловное дружелюбие.
Баночка растерянно сказал, что, по-видимому, аналогов этих слов в языке тси не существует.

\section{[@] Кошка}

\spacing

Кошка была совершенно счастлива.
В её глазах блестела мысль, она снимала предметы быта, рисунки и архитектуру, обновляла свой репозиторий каждые несколько минут.
Баночка сидел в палатке и с некоторым сожалением слушал её рассказы по телекому.

"--*Всегда мечтала о такой работе, "--- поделилась Кошка с Фонтанчиком.
Канин хмыкнул и покачал головой.

Вскоре пришёл черёд капища.
Шаман царрокх после долгих препирательств позволил двоим тси "--- Кошке и Фонтанчику "--- войти внутрь.

Внутри царило простое, одухотворённое великолепие.
Капище окружали увитые лианами тотемные столбы, земля была покрыта отполированными каменными плитами, меж которых пробивалась редкая травка.
Шаман с важным видом начал рассказывать о богах и духах.

Венцом капища был красивый базальтовый алтарь, украшенный совершенно чудесными орнаментами.
Гладит-Зелёную-Кошку, очарованная алтарём, поднялась на пару ступеней и протянула руку, чтобы ощупать каменный орнамент.

Вдруг одно из копий позади женщины рванулось вперёд.
Гладит-Зелёную-Кошку, в последний момент ощутив кожей неладное, попыталась увернуться, но было слишком поздно "--- копьё вошло точно под сердце.

Фонтанчик взвыл, словно волк, которому в горло вживили тревожную сирену.
Не прошло и секунды, как он разбросал царрокх, словно деревянные игрушки, и успел подхватить подругу под руки.

Люди стояли, медленно осознавая происшедшее.
Несколько воинов, переломанные и избитые, стонали на земле.
Возле тотема медленно испускал дух неудачливый копейщик "--- обитый металлом тупой конец оружия раздавил ему грудную клетку.
Прочие, с ужасом глядя на Фонтанчика, выставили в его сторону копья.
Инкрустированные обсидианом наконечники дрожали.

"--*Трусливые твари! За что?! "--- заорал Фонтанчик на ломаном цатроне, потеряв над собой контроль.

"--*Женщина не должна касаться святого алтаря!
Это осквернение! "--- завопил шаман.

"--*А предупредить\ldotst нельзя было\ldotsq "--- всхлипнула Кошка.

"--*Кхерр! А! "--- рявкнул один из воинов.

"--*Кхерр! А!
Кхерр! А! "--- подхватили остальные.

<<Медики, люди-мужчины.
Пять царрокх с множественными травмами, двое в агонии, один в клинической смерти, несу Кошку с травмой 12-E4C>>, "--- передал Фонтанчик по общему каналу и, подхватив Кошку на руки, бросился вон из капища.
Царрокх расступились, дав ему дорогу.

Кошка тихо, прерывисто дышала, цепляясь слабеющими руками за плечи друга.
На её лице замерла детская обида\ldotst

"--*Протокол <<Тайфун>>, "--- отчеканил на бегу Фонтанчик, "--- код\ldotst эээ\ldotst

Фонтанчик пробежался глазами по лицу Кошки, сжал и разжал её руку.

"--*\emph{Ноль-ноль, восемьдесят эф, ноль-два.}

Лицо Кошки приняло умиротворённое выражение, тело расслабилось.
Кровь резко перестала течь из раны.

"--*Животные, "--- шептал Фонтанчик.
"--- Тупые жестокие животные\ldotst


\section{[@] Знакомство с Нейросетью}

\spacing

Баночка уже нашёл себе друга.
Когда мы пришли, по палатке неуклюже прыгала молодая любопытная ворона.

"--*Это ещё что такое? "--- слабо удивилась Кошка, приподнявшись с носилок.
Костёр шикнул и знаком велел ей лечь.

"--*Это ворона, "--- просветил подругу Баночка.
"--- Ты как себя чувствуешь?

"--*Хорошо, "--- ответила Кошка.
"--- Фонтанчик здесь?

"--*Здесь, "--- ответила Заяц.
"--- Нужду справляет.

Ворона каркнула на всю палату, привлекая внимание.

"--*Ветвь Настоящие Птицы, Воробьи, Вороны, вид Большая скальная ворона с рекомбинированными первой, шестой и шестнадцатой хромосомами, отсутствующими плечами четвёртой, "--- отрекомендовал птицу Мак, сверившись с компьютером.
\mulang{$0$}
{"--- С определённой натяжкой "--- Девиантная ветвь, так как была собрана Безымянным, а не привезена переселенцами.}
{``Actually, the species may be called Deviant Fork, because it was compiled by Nameless, not brought by settlers.''}

"--*Это Безымянный так изменил ей хромосомы? "--- подняла бровь Заяц.

"--*Нет, она сама, "--- ухмыльнулся биолог.
"--- Экваториальная радиация.

"--*А как ты понял, что это Девиантная ветвь?

"--*Митохондрии не вороньи, "--- объяснил Мак.
"--- Листик уже говорила на эту тему с Безымянным.
Он сказал, что да, в период его творческого взлёта многие виды получили митохондрии какого-то несчастного зайца, которого и на планете-то уже нет.
Он их только немного подогнал, чтобы работали.

"--*То есть местная фауна в какой-то степени заяц, "--- резюмировал Костёр, игриво стрельнув глазами в сторону подруги.

"--*Ты можешь меня не бесить? "--- раздражённо всплеснула руками Заяц.
"--- Благодарю.

"--*А как ты её назовёшь? "--- спросила Кошка у планта, пытаясь сгладить напряжённость.

"--*Нейросеть, "--- сказал Баночка.
"--- Она даже отзывается.

Ворона действительно отзывалась.

"--*Красивое имя, "--- заметила Заяц.
"--- А главное "--- редкое.
У девяноста процентов моих знакомых животные носят эту кличку!
Давай её хотя бы Чернушкой назовём или Крылаткой!

"--*Хорошо, пусть будет Биологическая самообразовавшаяся нейронная сеть с функциями клевания пищи, общения и воспроизводства.

"--*Ты сейчас дал исчерпывающее описание половины тси, "--- засмеялся Костёр и тут же осёкся под взглядом Заяц.

"--*Длинно, "--- резюмировала Заяц, продолжая сверлить врача взглядом.

"--*А для краткости "--- Нейросеть, "--- выкрутился Баночка.
"--- Кстати, зайцев я тоже порядочно знаю.

Женщина не нашлась что ответить.

\razd

\mulang{$0$}
{"--*Баночка, мы не можем взять ворону с собой!}
{``Flask, we must not take the crow with us!''}

"--*Кошка, лежи.
Тебе вредно волноваться, "--- пытался урезонить женщину Костёр.

"--*Почему? "--- допытывался плант.
\mulang{$0$}
{"--- Нет, Костёр, подожди, не трогай её имплант.}
{``No, Campfire, wait, don't touch her implant.}
\mulang{$0$}
{Пусть ответит.}
{Let her answer.''}

"--*Здесь её вид!
Здесь её дом! "--- Кошка начала выходить из себя.
"--- Мак, ну скажи ты ему!

Мак смущённо молчал и тёр бритую голову.

"--*Хорошо, давай так, "--- Баночка вытащил ворону из ящика с инструментами и посадил на руку.
"--- Нейросеть поедет вот так, у меня на руке.
\mulang{$0$}
{Я не буду её держать.}
{I will not hold it.}
\mulang{$0$}
{Пусть сама решает.}
{Let it take its own decision.}
\mulang{$0$}
{Мак?}
{Poppy?''}

\mulang{$0$}
{"--*Это будет честно, "--- пожал плечами Мак.}
{``It would be fair,'' Poppy shrugged.}
\mulang{$0$}
{"--- Кошка, без обид.}
{``Cat, no offence.}
\mulang{$0$}
{Птица мне тоже очень нравится.}
{I like this bird too.''}

Кошка вздохнула.

\mulang{$0$}
{"--*Хорошо.}
{``You won.}
\mulang{$0$}
{Но если она улетит "--- не смей её звать.}
{But if it flies away, you dare not call it back.''}

\mulang{$0$}{"--*Договорились.}
{``Agreed.''}

Обратный путь получился очень занятным.
Заяц держала в объятиях Фонтанчика и гладила его по голове;
друг грустно думал о чём-то своём.
Я пытался соблазнить их печеньем.
Баночка сидел снаружи, на корме, и гладил Нейросеть.
Прочие наблюдали за ними "--- большинству была интересна реакция вороны, но кто-то очень хмурый на носилках строго следил, чтобы Баночка не держал птицу за лапки и крылья.
Впрочем, ворона была не против приключений.
Её не смутил даже пролив Скар.
Пока транспорт шёл через тёплые воды, Нейросеть смирно сидела на руке планта и клевала орешки.
И едва нос лодки коснулся берега и трансформировался в сухопутную подвеску, птица по-хозяйски отправилась осматривать окрестности.

"--*У меня ощущение, что кое-кто бессовестно мигрировал посредством нас, "--- заявил Мак за обедом.
\mulang{$0$}
{"--- Притворилась ручной, преодолела пролив и сбежала.}
{``It was pretending to be pet while crossing the strait, then left.}
\mulang{$0$}
{Ворона оказалась умнее десяти тси.}
{Ten Qi outsmarted by a crow.''}

\mulang{$0$}
{"--*Она вернётся, "--- ответил Баночка.}
{``It will be back,'' Flask answered.}

\mulang{$0$}
{"--*Да откуда тебе знать? "--- поморщился Мак.}
{``How can you know?'' Poppy frowned.}

Баночка загадочно улыбнулся и кивнул на окно.

Нейросеть сидела на подоконнике.
Глаза вороны горели голодным огнём.

\section{[-] Старый вождь}

\spacing

"--*Сколько родов?

"--*Прошу прощения? "--- переспросил посланец.

Кхотлам устало прикрыла глаза.

"--*Цвета, знаки, символы, всё, что запомнили!

"--*Были с красными перьями\ldotst

"--*Красные перья ара, одна жёлтая серьга в левом ухе, трубят в козьи рога, "--- резко сказала Кхотлам.
Она не терпела такого невежества относительно ближайших соседей.

"--*Да, точно они.

"--*Бвожай-сау.
Ещё кто-то был?

"--*Мы слышали какой-то странный свист, как будто\ldotst

"--*Белая рыба на синем фоне, синие перья самца кхар-тхир, белые серьги, прерывистый свист, "--- кормилица издала оглушительную свистящую трель, заставив сидящих поблизости воинов похвататься за оружие.

"--*Да, "--- кивнул посланец.

"--*Ты уверен?

"--*Я уверен, синие перья и белые серьги видели, "--- вмешался другой.

"--*Инхас-лака, "--- пробормотала Кхотлам.
"--- Похоже, мой милый друг Си-Жак умер.

\chapter*{Интерлюдия VI. Две речи}
\addcontentsline{toc}{chapter}{Интерлюдия VI. Две речи}

\textbf{Речи ноа}\footnote
{Большими речами называются две "--- <<Речь о мужчине>> и <<Речь о женщине>>.
Высокими речами называются три "--- <<Речь о жреце>>, <<Речь о воине>> и <<Речь о старателе>>.
Далёких речей пять "--- <<Без края>>, <<Опалённые>>, <<Хрустальные земли>>, <<Речь о дельфине>> и <<Сон ребёнка>>.
Все прочие в фольклоре ноа традиционно считаются Малыми речами. \authornote}

\section*{Речь о мужчине}
\addcontentsline{toc}{section}{Речь о мужчине}

Живёт мужчина.
У него есть имя, лицо, руки и ноги, мужской орган, инструменты, оружие, жилище и зелёный парус с вытканным солнцем.

Однажды к нему пришёл второй и сказал ему.
Мужчина отказался.
Тогда второй сказал ему: <<Ты не мужчина>>.
Но мужчина, у которого есть имя, лицо, руки и ноги, мужской орган, инструменты, оружие, жилище и зелёный парус с вытканным солнцем, просто засмеялся в ответ.

Однажды пришла к нему женщина и сказала ему.
Мужчина отказался.
Тогда женщина сказала ему: <<Ты не мужчина>>.
Но мужчина, у которого есть имя, лицо, руки и ноги, мужской орган, инструменты, оружие, жилище и зелёный парус с вытканным солнцем, просто засмеялся в ответ.

Однажды пришёл к нему второй и улыбнулся.
Они взяли инструменты, вышли вместе и сотворили красоту, которой не было равных.
Они оба были мужчинами, у которых есть имена, лица, руки и ноги, мужские органы, инструменты, оружие, жилища и зелёные паруса с вытканным солнцем.
Смеялись они одинаково громко.

Однажды пришла к нему женщина и улыбнулась.
Он привёл её домой и сделал ей ребёнка, и когда они делали детей, они смеялись.
У женщины тоже было имя, лицо, руки и ноги, инструменты, оружие и жилище, но её органы были женскими, паруса "--- голубыми, а в их центре были вытканы звёзды и рыбы.

Однажды пришёл к нему ребёнок и спросил: <<Почему ты улыбаешься, если никто не называет тебя мужчиной?>>
Мужчина ответил: <<У меня есть имя, лицо, руки и ноги, мужской орган, инструменты, оружие, жилище и зелёный парус с вытканным солнцем, и я умею обращаться со всем вместе и с каждым по отдельности>>.
И ребёнок остался в доме мужчины, пока не стал мужчиной сам.

\section*{Без края}
\addcontentsline{toc}{section}{Без края}

Живёт некто по имени Инфинит\footnote
{В оригинальной версии речи по отношению к Инфиниту употребляется единственно-безличное местоимение, т.е. Инфинит "--- личность, не имеющая ни пола, ни возраста, ни биологического вида.
В речах употребляются исключительно единственно-безличные местоимения "--- даже если по ходу повествования становятся ясны пол и возраст героя.
В разговорном ноа-лингва ЕБМ употребляются по отношению к богам либо в иронично-высокопарном ключе.
На сектум-лингва ЕБМ обычно переводится как <<некто конкретный, о котором говорящему не известно ничего, кроме сказанного>>. \authornote}.

Он бессмертен и хорош собой.

Идёт Инфинит по дороге.

Когда падает дождь, он достаёт зонтик и поёт песню про дождь.

Когда дождь заканчивается, он надевает шляпу и поёт песню про жаркое солнце.

Когда наступает ночь, Инфинит ложится спать.

Когда-нибудь Инфинит отрастит крылья и будет лететь над дорогой.

А когда крылья его устанут, у него появятся жабры и он поплывёт.

Когда солнце будет чересчур жарким, Инфинит будет идти по ночам.

Когда песни перестанут быть символом радости и печали, Инфинит начнёт чирикать или хлопать в ладоши, и все будут делать то же самое.

А когда земля закончится, Инфинит пойдёт по пустоте искать новую землю, по которой можно ходить и над которой можно летать, или воду, в которой можно плавать.

Когда свет погаснет, Инфинит найдёт другой свет и будет радоваться ему.

А у мира края нет, и другие земли всегда будут рождаться из пустоты, и другие воды всегда будут падать с небес, и другой свет всегда найдётся.

Инфинит знает это и всегда идёт, останавливаясь лишь для того, чтобы отдохнуть и набраться сил.

Он будет жить всегда, и всегда будет хорош собой.

\chapter{[-] Круг}

\section{[M] Хорошее дело}

\spacing

Из полудрёмы меня вырвали крики и беготня.
Сдёрнутое чьей-то сильной рукой одеяло слетело на пол.

"--*Ликхмас-тари, быстро в душ и на балкон.
Сокхэ ар’Трукх привезли, лесной бык потоптал, "--- сообщил Кхарас.

Михнет спустя я уже натягивал на балконе <<кровавый плащ>>.
Пострадавшая девушка лежала в маковом полусне с разрезанным животом.
Кхатрим молча кивнул на зеркала, и я подхватил их из рук Ситлама.

"--*Шире.

Я кивнул и раздвинул операционную рану шире, поморщившись от открывшегося зрелища.
Воздух наполнился запахом бойни.
Ситлам взял бутылочку с коагулирующим раствором и корнцанг.

"--*Бык поднял на рога, а потом ещё и прогулялся по ней, "--- пояснил он.
"--- Ветер духов уже подул, но её кормилец успел перехватить артерии.

"--*Кишечник, матку и правую почку придётся удалить, "--- добавил Кхатрим.
"--- На них живого места нет.

Я наблюдал за операцией, мимоходом восхищаясь слаженностью действий жрецов.
Они понимали друг друга с полузнака.
Вскоре брюшная полость была очищена, и Кхатрим начал рассекать брыжейку обсидиановым скальпелем, выделяя остатки кишки.
Время от времени в большой каменный таз отправлялся очередной ком размозжённой, покрытой кровью и синими прожилками плоти.
Я отстранённо вспоминал названия сосудов, которые попадали в поле зрения "--- Дерево Хетра, артерия Кобры, вена Двух Дорог\ldotst
Ситлам проверил белки глаз и посмотрел, запотевает ли зеркало у рта девушки.

Наконец Кхатрим тихо вздохнул.
Хороший знак "--- жизни девушки пока ничто не угрожает.
Ситлам тоже повеселел.

"--*Повезло ей, что её кормилец воином был.
У тех, старой закалки, ещё есть привычка носить с собой зажимы и скальпель.
Молодые плевали на элементарные меры предосторожности.

"--*Кстати, зажимы наброшены качественно, "--- отметил Кхатрим.
"--- Не всякий жрец так сможет, да ещё и впопыхах.
Надо ему их потом вернуть.

"--*Смотри, Ликхмас, "--- Ситлам достал из полости плотный комок размером с кулак, "--- вот в такой крохе дети вырастают до локтя в длину.

С этими словами врач отправил комок в таз.

Под балконом кто-то закричал, размахивая руками.

"--*Рыбы безмозглые, "--- выругался Ситлам.
"--- Бычьи тропы людьми мостят, что ли?
Дома надо сидеть в сезон.
Ликхмас, поможешь, "--- с этими словами жрец быстрым шагом вышел с балкона.

"--*Шить.

Я одной рукой подал Кхатриму отбелённый волос и кривую иглу, с сожалением бросив взгляд на лежащее в тазике месиво.
Казалось, что там осталась большая половина Сокхэ.

"--*Не волнуйся за неё, "--- ухмыльнулся врач, проследив за моим взглядом.
"--- Девушка сильная, здоровая, посмотри на её тело.
Главное "--- это выходить сейчас, в критический момент, дальше природа сама её излечит.

Кхатрим жестом попросил ещё один волос.

"--*Она будет такой же, как раньше? "--- удивился я.

"--*Почка обязательно вырастет, кишка тоже, но не полностью, где-то до двух третей длины.
Можно, в принципе, подсадить почку от умершего, но особого смысла не вижу "--- это на случай, если обе пострадают.
Детей выносит, если пол не сменится, такие случаи я сам видел.
Купца привезли уже в летах, среди прочего мужской орган повредило.
Бедолагу мы у Безумных выиграли с двумя мышами на красном поле\footnote
{Две мыши (Разведчика) на красном поле "--- крайне невыгодная комбинация в Метричис, популярной среди северных сели логической игре. \authornote},
мужские глазки удалили, думали, что отрастут.
А купец раз и в женщину превратился.
Правда, сам он к этому с юмором отнёсся, весёлый был и за словом в карман не лез, "--- голос Кхатрима вдруг потеплел, словно он вспомнил что-то очень приятное.

Я улыбнулся в маску.
Многих учителей я знал лишь стариками.
Было очень занятно вылавливать из поучений такие оговорки и наблюдать, как на покрытых морщинами лицах порой проскакивают следы бурной юности и былых страстей.

"--*А как долго растут органы?

"--*Растёт это хозяйство сезон-полтора, и до той поры кушать придётся противную жидкую кашу.
Первое время кормить будешь ты, я потом всё расскажу.

Вдруг в воздухе появился слабый, но отчётливый, оставляющий лёгкое першение в горле запах гари.
Кхатрим поморщился и поправил утиную маску.

"--*Фильтр горит, "--- недовольно заметил он.
"--- Да что за день такой.
Ликхмас, после операции сбегай к Кхохо-лехэ на улицу Живой Рыбы, скажи, чтобы он на балконе фильтр заменил.

Я кивнул.
Кхатрим, ещё раз покачав головой, взялся за иглу.

Закончилась операция благополучно.
Я зашил мышцы и кожу отбелённым волосом, и девушку ещё дремлющей увезли в глубь Храма.
Кхатрим стянул с себя <<кровавый плащ>> и бросил его в угол.

\razd

\spacing

Я постучал в провисшую дверь.
Спустя мгновение дверь приоткрылась, словно хозяин в этот ранний час стоял за ней.
В меня вонзился колючий взгляд из-под седых бровей.
Я оробел.

"--*Митлам-лехэ?

"--*Ла? "--- буркнул старик.

"--*С твоей хранительницей всё хорошо, "--- сказал я.
"--- Пришлось удалить много органов, но она выживет.
Сегодня вечером перевезём домой.

Дверь открылась чуть шире.

"--*Хай, тебе просили передать, "--- я протянул старику свёрток с зажимами.

Митлам-лехэ высокомерно смотрел на меня несколько долгих мгновений, потом протянул руку и схватил свёрток.

\mulang{$0$}
{"--*Хэ. Хоть это вернули. Трааа\ldotst}
{``Cho\^{e}. Leastwise, they returned this. Tr\={a}\"{a}\ldotst''}

Я продолжал топтаться на месте.

"--*Чего стоишь?
На завтрак напрашиваешься?

"--*Нет, "--- промямлил я.
Старый воин действовал на меня как-то чересчур удручающе.
"--- Я хотел, чтобы ты проверил свёрток и сказал, что я ничего не забыл.
Тогда мне не придётся бегать ещё раз.

"--*А дом Митлама-лехэ тебе не по душе? "--- рявкнул старик.
Я попятился.

"--*Плох дом, в котором не рады гонцу, принёсшему хорошую весть, "--- пробормотал я.

Старик оскалился и вдруг разразился смехом.

"--*Хорошую весть, значит.
Заходи, молодой смелый жрец.
Заходи, не бойся.
Покажет тебе Митлам-лехэ, куда ты принёс <<хорошую весть>>.

Старик ударом расхлябил дверь и исчез в темноте.
Я, поколебавшись, вошёл внутрь.

В доме царил дух запустения.
Треснувший очаг, наспех заделанный обычной приречной глиной, уже начинал горчить\footnote
{Неточный перевод слова, обозначающего различаемый людьми-тси запах монооксида углерода. \authornote},
но хозяин даже не думал открывать вытяжку.
От глиняной лампы остался только черепок с носиком и фитилём;
огонёк испустил дух, едва я возмутил движением воздух.
Стена с одной стороны держалась на паре досок.
С наступлением сезона дождей её просто смоет.

Старик, не спрашивая разрешения, прикурил лампу от моего фонаря и махнул на развалившуюся циновку.
Я сел, стараясь не обращать внимания на вонзившиеся в ноги сухие тростинки.
Старик ухмыльнулся.

"--*Надо же, сел, "--- хозяин обращался не ко мне, а к невидимому собеседнику, стоявшему у очага.
"--- А то заходил тут намедни жрец.
Ситлам зовут.
Жопа у него для циновок чересчур белая, посмотрите на него.
Небось шёлком подтирается и сидит только на северных шкурах.

Я промолчал.

"--*Что молчишь?
Нравится нищенское логово?

"--*Почему здесь так неуютно? "--- спросил я.

Глаза старика на миг сверкнули гневом, но тут же потухли.

"--*Неуютно.
Будь Митлам-лехэ моложе, он бы выразился покрепче.
Срач здесь жуткий, вот что.
Курятник, а не дом.
И скоро сюда притащат Сокхэ.
Чем там её кормить надо? "--- старик снова обратился к очагу и тут же хлопнул себя ладонью по лбу.
"--- Хай, бульончиками, кашками молочными, как же старик не догадался.
Когда из тебя кишку полностью выпускают, ничем другим-то кормить и нельзя "--- загнётся дитя.
А откуда зверюшки и молочко, хэ?
Нет зверюшек "--- ушли зверюшки, не настрелял их Митлам-лехэ.
Где молочко?
Нет молочка "--- не купит его Митлам-лехэ.
Где зерно для кашек?
Нет зерна для кашек "--- не вырастил его Митлам-лехэ.

Старик снова разъярился, его глаза загорелись.

"--*Новость хорошую он принёс.
На завтрак, думал, пригласит его Митлам-лехэ.
Нечего жрать в этом доме! "--- рявкнул старик, пнув валяющуюся на полу миску.
"--- Нет в амбаре ни сладкой муки, ни обычной!
Лучше бы у вас под ножом сдохла эта тупая девица, чем здесь медленно от голода помирать!

"--*Но как же, "--- пробормотал я.
"--- Урожай нынче хорош, люди избыток за бесценок\ldotst

"--*Да, за бесценок!
Именно!
За бесценок продают! "--- старик рассмеялся диким смехом.
"--- Кхараму продают, Митхэ продают, дому ар’Хэ продают, а Митламу-лехэ не продают!
И Сокхэ не продают!
Нет семян, дерьмо кроличье сей, Митлам-лехэ!
Нет овощей, бычьи лепёшки гложи, Митлам-лехэ!
Много сортов дерьма-то, богатый стол будет!

"--*Но почему\ldotsq

"--*Какой же он тупой, "--- плюнул старик.
Густая желтоватая слюна попала ему на подбородок и на рубаху, но он и не подумал утереться.
"--- Разрушитель здесь живёт, понимаешь?
Тхаха.
Разрушитель.
Какое хорошее слово.
Хлопнул Митлам-лехэ по пьяни одного придурка "--- и уже что-то там разрушил.
Хотя ничего, кроме жизни ребёнка, не разрушилось.
И разрушили её вы, дерьмоеды, не я.
Вам бы здешних лепёшек отведать.
Теперь каждая падаль, которую Митлам-лехэ один на один со связанными руками придушит, считает нужным его оскорбить.
Да не как-нибудь, а пообиднее да повитиеватее "--- кто ещё тебе не ответит?

Я молчал.

"--*И жрец этот, Ситлам, "--- старик указал очагу куда-то в сторону Тхитрона.
"--- Разговаривает так, словно к нему во сне сам Лю явился.
Гордится своей робой, словно со дна моря её руками доставал.
Дай ему волю "--- он бы на амулет Сана молиться начал, поклоны бить.
Да он за всю жизнь столько ран не зашил, сколько Митлам-лехэ в одних Молчащих лесах.
Сколько он людей спас?
Много наспасаешь-то, в храме жопу просиживая.

Я молчал.

"--*А теперь беги, парень, "--- старик вдруг с ловкостью зверя вскочил и выхватил фалангу.
"--- Старик голоден, а у него в комнате молодой барашек.
Старик может не удержаться.

С его губ сорвался безумный гогот.
Я не двинулся с места.
Лезвие фаланги "--- иззубренная сталь, а не кукхватр "--- едва держалось на треснувшем древке.

"--*Это отвратительно, "--- сказал я.

"--*Ещё как, парень. Митлам-лехэ ещё и воняет, "--- старик открыл щербатый длиннозубый рот, хрипло выдохнул и снова залился безумным смехом.

"--*Я возьму Сокхэ в свой дом, "--- сказал я.
"--- Мне нужна женщина.

"--*Что он несёт? "--- старик ткнул в мою сторону остриём фаланги, обращаясь к невидимому собеседнику.
"--- Молодой барашек что-то сказал.
Не пора ли его выпотрошить?

Мой страх прошёл.
Вместо старого разъярённого берсерка я видел несчастного, молящего меня на коленях о помощи.

"--*Если дух уймёт своё безумие и даст Митламу-лехэ выслушать меня, то я буду благодарен, "--- сказал я.
"--- Митлам-лехэ впустил меня в дом не просто так.

Старик замер.

"--*У вас будут продукты, материалы и семена, "--- сказал я и встал.
"--- Я куплю всё сам.
Когда Сокхэ выздоровеет, я предложу ей себя как мужчину.

Старик медленно сел, поигрывая фалангой.
Пожевал губы, по-прежнему глядя куда-то в сторону.
Я направился к выходу.

"--*Только не вздумай говорить с собой, когда я уйду, "--- бросил я напоследок.

Провисшая дверь со скрипом встала на место.
Этот скрип был единственным звуком, которым проводило меня полуразрушенное жилище старика.

\razd

Кормилица приехала в хутор Самитх в тот же день и собрала всех жителей возле хибарки.
Она при всех обняла воина и поцеловала ему руку, на которой синела старая метка.

"--*Если преступник достоин казни, его следует казнить, "--- сказала она.
"--- Ни один закон не позволяет мучить его и его родичей всю оставшуюся жизнь.

Жители починили дом Митлама-лехэ ещё до наступления вечера.
Ему дали пищу и новую одежду, жрецы прислали чистых бинтов и пучки целебных трав.
Сокхэ, которую я вместе с Столбиком перевёз в родной дом, лежала на чистой постели, улыбалась и пыталась схватить мою руку слабыми пальцами.
Когда мы встречались взглядами, я видел в её глазах такую благодарность и любовь, что мне хотелось бежать "--- куда угодно, лишь бы её не видеть.

"--*Сокхэ с тобой не будет, "--- объявил мне старик, едва я устроил девушку на её месте.
"--- Убирайся, и чтобы Митлам-лехэ тебя здесь больше не видел.

"--*Её нужно кормить очень часто, "--- напомнил я.

"--*Не учи охотника шкуру снимать, "--- бросил старик.
"--- Кормилице передай благодарность.

Я кивнул и вышел вместе со Столбиком из дома.
Дверь больше не скрипела.

"--*Удивительная женщина, "--- поделился впечатлениями Столбик.
"--- Сколько раз здесь бывал, а её не видел.
И ведь без семян и прочих товаров как-то держала дом, да не один год!
Характера ей недостаёт, но выносливая и хозяйка хорошая.
Если бы она не понравилась тебе, я бы за ней поухаживал.

"--*Уступаю другу, "--- сказал я.

"--*А что так?
Не хочешь возиться с ней полгода?

"--*Знаешь же, что не в этом дело, "--- поморщился я.

"--*Хай, подожди, "--- Столбик схватил меня за рукав.
"--- Девушка красивая, сильная, она тебе нравится, да и ты её успел очаровать, кормя с ложечки.
И ты просто так уступаешь это сокровище мне?
В чём дело, Лис?

Я покачал головой.

"--*После такой жизни её очарует любой, кто отнесётся к ней по-людски.

"--*Да ну тебя, какая разница?

"--*Большая, дурень ты мой!
Ну и её кормилец, как ты слышал, выставил меня за дверь.
Всё-таки он гораздо лучше будет относиться к крестьянину, чем к жрецу.

"--*Ты ж не с её кормильцем детей делать будешь, "--- ухмыльнулся Столбик.
"--- Ладно, дело твоё.
А вообще это ты дурень.
Надо использовать силы, данные тебе лесными духами.

Столбик ухватил меня крепкой рукой за достоинство.

"--*Отстань, "--- я отвесил другу подзатыльник.
"--- Я использую другие силы.

Остаток пути мы разговаривали о всякой ерунде и смеялись.
Столбик старался обратить свои слова в шутку, но я чувствовал спрятанное в них приятное волнение, словно тонкий аромат, идущий от ещё закрытого бутона.
Всё складывалось для Сокхэ и её кормильца наилучшим образом "--- они обрели защитника.
По крайней мере до поры, пока девушка не выздоровеет.

Ужин Кхотлам провела в тяжёлом молчании.
Домашние, чувствуя её недовольство, тоже не решались заговорить.
Все поблагодарили хозяйку за еду и ушли спать, едва закончив трапезу.
Мы с кормилицей остались вдвоём.

"--*Почему ты грустишь? "--- спросил я.
"--- Мы сделали хорошее дело.

"--*Хорошее? "--- буркнула Кхотлам.
"--- Разумеется.
Только и занимаюсь тем, что делаю хорошие дела.
Хотя, если честно, мне надоело напоминать людям, чтобы они были людьми.
Ты поел, Лисёнок?
Иди спать.

\section{[-] Совет перед войной}

\spacing

Кхарас долго молчал, читая донесения и слушая реплики адептов Храма.
Наконец, когда мнения иссякли, прозвучало его слово:

"--*Храм справится.

"--*Ну Кхарааас! "--- возмущённо протянула Кхохо.
"--- Зачем задействовать весь Нижний этаж?
Возьми шестнадцать человек, по клинку с квартала, дай Ситрису "--- и они эту молодёжь раскатают за кхамит!

"--*Если Храм справляется, незачем беспокоить народ, "--- возразила Хитрам.

"--*Нам нужен жрец, "--- продолжил Кхарас.

Кхохо обиженно замолчала и стала ковырять ногтями стол.
Если вождь сменил тему "--- значит, решение по предыдущему вопросу принято и обжалованию не подлежит.

"--*Кхарас, мы с Ситламом сегодня по уши, "--- сказал Кхатрим.
"--- Сам же знаешь, сколько раненых привезли.
Возьми Ликхмаса.

"--*Ликхмас?

Я повернулся к Трукхвалу:

"--*Учитель, ты меня подменишь в школе на сегодня-завтра?

"--*Да, конечно, "--- улыбнулся старый жрец.
"--- Только не задерживайся, а то книги рассыпятся в пыль.

"--*Трукхвал, Ситлам, соберите и проинструктируйте его по основным моментам, "--- распорядился Кхарас.
"--- Ликхэ, на тебе пайки "--- три рассвета, плюс двадцать единиц для выживших пленников.
Кхохо, замени все факелы "--- к нашему уходу в храме должно быть светло.

\section{Сгоревшие глаза}

\spacing

Эрликх, Кхарас и Хитрам сидели на ступеньках.
На площади миловалась молодая пара чуть старше меня.

"--*Красивая пара, да? "--- Хитрам кивнула на стоящих.
"--- Паренёчек такой милый, круглощёкий.
И девушка "--- почти ребёнок.
Чистые наивные глазки.
На лицах ни тревог, ни потерь, ни войн.
Только ветер, только радость впереди.

Я почувствовал в её голосе лёгкую зависть.

"--*Наверное, им следует узнать всё это, "--- пожал плечами Эрликх.
"--- Это все узнают со временем\ldotst

"--*Не следует, "--- вдруг громыхнул Кхарас.
Все замолчали и посмотрели на вождя.

"--*Но ведь, Кхарас, они живут в жестоком мире, и\ldotst "--- попытался возразить Эрликх.

"--*Мы для того здесь и есть, чтобы дети оставались детьми, "--- угрюмо пробурчал Кхарас.
"--- Мы.
Храм и Двор.
И если это в наших силах, пусть эта пара, и их потомки, и потомки их потомков никогда не познают горя.

"--*Мне так нравится, когда ты начинаешь говорить о долге, "--- сказала Хитрам.
"--- Напоминает мне тебя молодого.
Когда шла речь о несправедливости, когда где-то сильный обижал слабого, твои глаза загорались ярким пламенем негодования, способным прожечь горы и иссушить моря.

\mulang{$0$}
{"--*С возрастом Кхарас стал спокойнее, да, "--- отметил Эрликх.}
{``Kch\'{a}r\v{a}s have gotten calmer with age, yep,'' O\r{e}rl\'{\i}kch observed.}

\mulang{$0$}
{"--*Нет, "--- покачала головой Хитрам.}
{``Nope,'' Ch\"{\i}tr\'{a}m shook her head.}
\mulang{$0$}
{"--- Просто несправедливости стало столько, что его глаза сгорели дотла.}
{``There's actually so much injustice his eyes burned to ashes.''}

Пара на площади, тихо смеясь, побежала куда-то за склады.
Вскоре осколки смеха окончательно раздробились о стены и брусчатку.
Наступила ночь.

"--*Тебе принести что-нибудь согревающее? "--- ласково спросила Хитрам.

Кхарас угрюмо промолчал.

"--*Сейчас, "--- Хитрам поднялась и пошла в храм так резво, словно вождь сказал, что умирает от жажды.

\section{[-] Опасная игрушка}

%Плуг 5 дождя 12002, Год Церемонии 20.

\spacing

"--*Идём, Ликхмас-тари. Я должен тебе кое-что показать перед походом.

Старик, охая и подтягивая неестественно прямую ногу, зашаркал к большому сундуку в углу.
В это время года нога мучила старика особенно сильно.
Бронзовый ключик, несколько раз чиркнув по вензелю замка, всё же попал в скважину, два раза щёлкнул язычок.
Пожилой жрец потянул за ручку ручку и чуть не упал.

"--*Ох, проклятая нога\ldotst
Парень, поднимай.

Я подошёл и не без труда откинул тяжёлую резную крышку сундука.
Внутри, поверх старых свитков, изорванной в лохмотья одежды, мешочков, колб с непонятным содержимым и прочего хлама лежали три наруча.
Трукхвал указал на них, и я с интересом поднял один.
Обычный, отделанный кожей наруч, без каких-либо украшений, пожалуй, только чересчур тяжёлый.
Такие часто можно было увидеть на воинах.
Я недоумённо посмотрел на старика, а он, улыбаясь во все жёлтые кривые зубы, наблюдал за мной.

"--*Учитель Трукхвал, что это?

"--*Ооо, мой молодой друг\ldotst
Знаешь ли ты, почему воины народа сели одерживают победы в боях с дикарями?

"--*Благодаря храбрости и искусству боя, учитель?

"--*Не только, Ликхмас.
Ответ ты держишь в своих руках.
Ла, надень.

Я повертел наруч в руках и, недоумевая, натянул его.

"--*На левую, Ликхмас.
Ты же чувствуешь "--- он тяжеловат.
Вот так.
А теперь смотри "--- вот здесь есть выступ.
Надави на него.

Я подчинился.
Выступ спружинил удивительно легко.

\mulang{$0$}
{"--*А теперь\ldotst знаешь ли ты боевые языки народа сели?}
{``Well, now\dots do you know battle languages of S\r{e}l\={\i}?''}

\mulang{$0$}
{"--*Разумеется, учитель.}
{``Of course, teacher.}
\mulang{$0$}
{Птичий, звериный, барабанный язык\ldotst но\ldotst}
{Language of birds, animals, war drum language\dots but\dots''}

\mulang{$0$}
{"--*Вот на барабанном мы и остановимся, "--- перебил меня Трукхвал.}
{``Let's pause on the last one,'' Tr\`{u}kchu\r{a}l interrupted me.}
\mulang{$0$}
{"--- Давай-ка, отбей на этом выступе <<проверку>>.}
{``Well, drum the `check' on this button.}
\mulang{$0$}
{Только не ударами, а нажатиями.}
{Try to push, not to beat.''}

\mulang{$0$}
{"--*На этой маленькой бусинке?}
{``This tiny bead?}
\mulang{$0$}
{Меня никто не услышит, учитель.}
{I won't be heard, teacher.''}

"--*Кто знает, Ликхмас, "--- учитель улыбался уже с таким предвкушением, что по подбородку текла тоненькая струйка слюны.

Я пожал плечами и надавил на выступ.
Три коротких, длинный, пауза\ldotst
<<Первый, проверка>>.

Внезапно наруч ощутимо завибрировал, и я ошеломлённо разобрал в вибрации знакомый язык:

<<Второй, принял.
Имя?>>

Я защёлкал выступом:

<<Ликхмас.
Второй, имя?>>

Ответ пришёл спустя миг:

<<Кхарас.
Первый, докладывай>>.

Я, открыв рот, смотрел на старика:

"--*Но Кхарас сейчас же\ldotst

"--*\ldotst на площади, "--- громким шёпотом закончил учитель.
"--- Как я узнал?
Смотри.

Неожиданно изящным, молодцеватым движением Трукхвал вскинул руку и задрал рукав "--- там был наруч длиннее раза в два, с чем-то похожим на колокольчик.
Трукхвал крутнулся на каблуках "--- и наруч завибрировал, то выше, то ниже, то тише, то громче.

"--*Кхарас на площади, Ликхэ в зале, в углу сидит, ты вплотную ко мне.
Ситрис\ldotst
Эээ, где он?
Хай\ldotst

Учитель растерянно покрутился на месте.

"--*Нету\ldotst
Хай, над этой штукой ещё потеть и потеть, я её не до конца проработал.
Это для координатора.
Правда, Кхарас сказал, что это ему не нужно "--- тяжело и бесполезно, он уже привык ориентироваться по голосам.
Хаяй, ну да ладно, "--- Трукхвал улыбнулся и схватил меня за руку.
"--- Как тебе, Ликхмас-тари?

Я оторопело разглядывал чудесные наручи то на своей руке, то на учителевой.
Внезапно мой снова завибрировал:

<<Первый, докладывай>>.

Я опомнился:

<<Второй, докладываю.
Всё спокойно, проверяю связь>>.

<<Первый, принял.
Конец связи>>.

Наруч замолчал, и я перевёл дух.

"--*Ну как?
Нравится?

"--*Да\ldotst

"--*Возьмёшь его с собой.
В джунглях, когда каждый шорох может стоить тебе жизни, это вещь незаменимая.

"--*Но это же\ldotst магия\ldotst "--- я заикался.
"--- Как\ldotst как\ldotst оно\ldotst

"--*Как оно к нам попало?
Мы делали его здесь, в храме, по древним чертежам.
Может, помнишь, та странная книга.
<<Законы молнии и камня>>, Съешь-Рычащий-Пирожок.
Как-то мы с Хонхо и Хатламом сидели и читали эту книгу, а тут рисунок.
Никаких лишних надписей, только легенда с названиями металлов.
Хатлам сказал: <<А давайте попробуем в точности, как на рисунке, вдруг выйдет чего?>>
Мы собрали восемь штук, как обычно "--- ты же знаешь, что-то сгорит, что-то будет собрано неправильно, что-то сломаю я, а что-то потеряется в хламе на складе.
Но мы бы так и не поняли, для чего оно пригодно, если бы не случай "--- я случайно унёс одну штуковину с собой, а Хатлам со скуки решил пощёлкать выключателем другой.
Представь мой ужас, когда сумка начала жужжать нехорошими словами!

Трукхвал засмеялся.

"--*Вот так мы собрали устройство, протестировали его и придумали замаскировать устройство под наруч.
Теперь им пользуются наши воины и весь Север до самой Ледяной Рыбы.
Разумеется, пока это тайна.

"--*Как оно\ldotst работает?

"--*Ааа\ldotst "--- протянул старик, и улыбка его стала какой-то несчастной.
"--- Я\ldotst не знаю, ученик.
И никто не знает.
Мы делаем, как написано, и оно работает.
Древние знали.
Мы не знаем.
Вот, смотри, как оно выглядит внутри\ldotst

Пожилой жрец, крякнув, порылся в сундуке и выудил небольшую пластинку "--- деревяшка с медной проволокой и небольшими цилиндрами.

"--*Здесь медь.
Железо для этой детали непригодно.
А вот эти штуки "--- это медная же фольга и пчелиный воск, свёрнутые трубочкой.
Вот это "--- кусочек серного свинца, его делают особым способом.
А вот эта коробочка "--- самая важная часть.
Внутри "--- серебро и солёная вода с бумагой.

"--*И всё?!

"--*Да, да, ученик!
И это самое удивительное!

"--*А что это?
Похоже на змею, обвившуюся вокруг ветки.

"--*Это тоже медная проволока, но почему-то её нужно свить именно так.

"--*А вот этот камешек?

"--*Это не камешек, это\ldotst

\ldotst Из покоев Трукхвала я вышел в приподнятом настроении.
Чудесный наруч как влитой покоился на моём левом запястье.
Я ощущал его тяжесть.
Старик всю дорогу до Зала косился на меня, и в его глазах была непонятная тоска.

"--*Знаешь, Ликхмас, "--- заговорил он наконец, "--- в книге говорилось, что Древние считали такие вещи\ldotst хаяй\ldotst простыми и неинтересными.

"--*Как можно считать это неинтересным, учитель?
Переговоры без шума, сквозь камень и воду!

"--*Однако ж.
Я даже больше тебе скажу "--- то, что ты надел на свою руку, было игрушкой для детей.
Понимаешь?
Игрушкой.

"--*Понимаю, учитель.

"--*Хаяй\ldotst Ничего ты не понимаешь, молод ещё.
А я много размышлял над этим, Ликхмас, очень много.
И, кажется, я понял, почему Древние исчезли.

"--*Почему, учитель?

"--*Потому что вещью, которая способна решить судьбу народов, играли малыши.

\section{[@] Оружие и еда}

Историки продолжают собирать данные о царрокх.
Вчера была полностью расшифрована письменность.
Обнаружены зачатки технологий "--- царрокх знают, что такое электричество, и умеют им пользоваться.
В некоторых сооружениях найдены примитивные лампы накаливания.
Люди используют для связи простые передатчики на метровых радиоволнах.
Химия у них развита значительно слабее "--- не было найдено никаких признаков использования сложных реактивов, производства лекарств и взрывчатых веществ.
Обработка металла также в зачаточном состоянии "--- железа, как я уже сказал, на планете не так много.

Историки впервые увидели настоящее оружие для убийства сапиентов.
В Красном музее лежали образцы, мы даже видели некоторые из них, когда шли через Палаты Войны "--- мечи, арбалеты, пулевое и волновое оружие, взрывающиеся снаряды, был даже архаичный танк на антигравитационных двигателях, "--- но всё это были реплики, изготовленные по обрывкам дошедших до нас описаний.
В племени царрокх оружие, напоминающее дубинку с лезвиями, носит каждый мужчина, и из-за этого к безоружным мужчинам-тси, намного превосходящим любого из аборигенов по силе и ловкости, они относятся с плохо скрываемым презрением.
Во избежание разногласий было решено разработать для тси эффективное оружие ближнего боя и курс владения им.
Этим занялся заскучавший Баночка.

Одним из способов установления отношений у царрокх являются так называемая <<связь крови>> "--- нечто вроде биологического родства, которое распространяется почему-то не только на потомков, но и на предков.
Да, это самый удивительный аспект "--- если у двух человек появляется совместное потомство, признанное религиозным лидером, то члены их родов начинают относиться друг к другу по-особому.
К сожалению, эти люди относятся к другому биологическому виду, и потомство людей-тси и царрокх без генной инженерии получить невозможно.
Красный-Мак-Под-Кустами хотел заняться устранением репродуктивного барьера "--- уж очень заманчивой была перспектива так легко подружиться с племенем.
Но остальные справедливо решили, что проблем у нас и так хватает.

Биологов удивил необычно низкий рост аборигенов.
Впрочем, всё прояснилось быстро: они очень плохо питаются.
Животноводство ограничивается разведением курообразных.
Вокруг каменных городов огромные площади джунглей пускают под топор ради одной-двух посадок, причём полученная древесина используется в основном как топливо.
Ужасная, просто невероятная расточительность.
При этом было подсчитано, что интенсивная обработка почвы без использования сложных технологий позволяет семье иметь поле в тысячу раз меньше "--- в год здесь можно собирать по восемь-десять урожаев.
А имеющаяся у нас на корабле кольцевая теплица способна кормить всех царрокх до конца жизни.
Работы много, много.

На сегодняшнем собрании было единогласно решено передать технологии земледелия и животноводства царрокх в понятной для них форме.
Вопрос еды стоять не должен\footnote
{Вероятно, отсюда пошла знаменитая поговорка сели <<поднимать вопрос еды>> "--- мелочиться. \authornote}.
Мы обязательно должны подружиться с этими аборигенами "--- они обладают большим количеством специальных знаний, необходимых нам для выживания.

\section{[@] Великий воин}

\spacing

"--*Ты посмотри на него, "--- неодобрительно сказал Фонтанчик.

Баночка стоял посреди палатки. На нём красовался богато украшенный гравировкой доспех из стабитаниума.
На поясе у планта висело нечто, напоминающее меч и кинжалы.

"--*Я проработал все суставы, "--- гордо сказал техник из-под забрала.
"--- Теперь аборигены точно будут вас уважать "--- в стыки не войдёт даже микрозонд.

Баночка один за другим бросил кинжалы в стоящую посреди палатки куклу.
Клинки пробили мягкий полимер и по самую рукоять вошли в песок.
Затем плант с воинственным криком выхватил меч и, подскочив к кукле, разрубил её пополам.

"--*Ну как? "--- сияя улыбкой, спросил он.

"--*У нас нет столько стабитаниума, чтобы одеть всех, "--- тихо напомнил я другу.

"--*И я это носить не стану, "--- заявил Фонтанчик.

Улыбка Баночки увяла.

"--*Вы сами попросили оружие и\ldotst

"--*Мы просили обычные заострённые палки, которые можно сделать из сухой древесины, а не замаскированный под меч резонанс-резак, "--- сказал я.
"--- У оружия назначение "--- причинять вред мягкому, сделанному из плоти существу, слегка прикрытому одеждой и дублёной кожей.
А твоим мечом можно камни рубить.

"--*Так это же хорошо! "--- запротестовал Баночка.
"--- Чем лучше оружие, тем\ldotst

"--*Дружище, "--- начал Фонтанчик, "--- сделай нам красивые палки с обычными, понимаешь, обычными лезвиями.
Научи этим пользоваться "--- отбивать вражеские заострённые палки.
Потом возьми материал, который используют аборигены, и сделай нечто похожее на их доспехи, чтобы жизненно важные органы были чуть-чуть прикрыты.
Неважно, что по сути они ни от чего не защищают, главное, чтобы было.
Этого хватит для установления уважительных отношений.
Мы всё равно не собираемся с ними драться.

"--*А если придётся?

"--*Ты их боишься? "--- удивился Фонтанчик.

"--*Да, "--- кивнул Баночка.
"--- Я боюсь даже не того, что они большие и агрессивные "--- в принципе, я буду даже посильнее многих, и защититься смогу.
Но беспокоит мысль, что они могут убить просто потому, что я им не нравлюсь, и сделать это исподтишка.

"--*Я буду посещать царрокх вместе с тобой, "--- ободряюще сказал Фонтанчик.
"--- Со мной тебя точно никто не посмеет тронуть.

"--*Хорошо, "--- грустно сказал Баночка.
"--- Однако элементарные средства защиты нужны.
Давайте я хотя бы сделаю для тси одежду из полимерного волокна.

"--*Она защитит только от рубящих ударов, "--- покачал головой я.
"--- Прямым ударом копьё пробьёт любую одежду без твёрдых вставок.
Займись лучше техникой боя, умение защитит лучше любых костюмов.

"--*Хорошо, "--- повторил Баночка.
"--- Есть ещё вариант сделать метательное оружие, звуковое, химическое или\ldotst

"--*Исключено, "--- тут же замахал руками Фонтанчик.
"--- Во-первых, аборигены не сразу поймут, что это оружие, и нам придётся кого-то покалечить.
Во-вторых, превосходящие военные технологии заставят их бояться.
Это нам не нужно.

"--*Совсем не нужно, "--- согласился Баночка.
"--- А с этим что?

"--*А это ты будешь носить, "--- оскалился канин.
"--- Мы будем тебе поклоняться, как самому великому воину.

"--*Ещё чего не хватало! "--- возмутился Баночка.
"--- Ладно, пущу стабитаниум в дело.

"--*Оставь, Баночка, "--- замахал я руками.
"--- Вышло очень красиво.
Правда.
Когда-нибудь здесь будет музей, и мы обязательно выставим там твоё творение.

\section{[@] Обучение Нейросети}

\spacing

Баночка много времени проводил с Нейросетью.
У планта явно были способности к общению с животными "--- это отметили все.
Однако вскоре я заметил, что Баночка не просто общается с вороной интонациями "--- он начал каркать.

"--*Ты раскодировал её сигналы? "--- удивился Мак.

"--*Потом "--- да, "--- нетерпеливо отмахнулся Баночка.
"--- Но вначале было лень, поэтому я придумал свою систему сигналов, которые ворона может запомнить и воспроизвести.
Так мы и общаемся.

"--*Подожди, Баночка, "--- нахмурилась Листик.
"--- Что значит <<Потом "--- да>>?
Ты использовал семантический декодер Дракона и он осилил низкоуровневый язык эволюционно далёкой Ветви?

"--*Выучив мой язык, Нейросеть обучила меня своему, "--- пожал плечами Баночка.
"--- Очевидно же, она достаточно умная.

Птица часто сидела у Баночки на плече, пока он чинил машины.
Он сумел втолковать ей, что делает очень важное дело, и Нейросеть стала ему помогать.
Она перестала таскать детальки.
Баночка сказал птице, что это может навредить машине, а ещё пообещал дать ей аналоги почти любых деталей, которые ей приглянутся.
Как-то в разговоре плант упомянул, что птица относится к машинам как к живым существам и пытается с ними <<говорить>>.

Дальше "--- больше.
Птица усвоила цифры и некоторые иероглифы "--- Баночка придумал для них специальные звуковые сигналы.
Правда, памяти у неё хватало только на восемь-десять знаков, но для практического применения этого оказалось достаточно "--- реестровые номера деталей редко содержат больше.
Баночка написал скрипт для распознавания вороньих цифр;
Нейросеть периодически летала к принтеру, сообщала ему реестровый номер и возвращалась со свеженапечатанной деталью.
Иногда ворона заигрывала с плантом, принося вместо требуемой детали жука или камешек, но в целом Баночка был доволен.
Апогеем обучения Нейросети стала письменность "--- Баночка сделал для птицы крохотное писчее перо и научил её вычерчивать цифры.

"--*Как у тебя это выходит? "--- удивлялся Мак.
"--- У ворон высокий интеллект, но не настолько же, чтобы усвоить письменность!

"--*Я думаю, что Нейросеть "--- гений среди ворон, "--- совершенно серьёзно ответил Баночка.

Однако и Мак, и Баночка ошибались.
Ворона оказалась совершенно обычной представительницей своего вида.
Так как Нейросеть продолжала общаться с местными птицами и даже имела среди них авторитет, математические навыки в несколько упрощённом виде разошлись по популяции.
Семьдесят пять процентов ворон в той или иной форме использовали числовые сигналы Баночки при передаче информации.
Но самое удивительное выяснилось, когда вороньи сигналы переняли скворцы, превратив шестнадцатиричную систему счисления в восьмеричную.
Это изумлённая Листик сообщила нам через полтора года.

"--*У меня, похоже, наклёвывается диссертация по межвидовой коммуникации, "--- сообщила она.

Техники не разделяли восторг Мака по поводу вороны.

"--*Ты с ума сошёл, "--- буркнул Хомяк, когда Баночка предложил подключить ворону к нейроинтерфейсу.
"--- Ты ей ещё репозиторий сделай в хранилище Дракона!
Дружище, ну это всё-таки низшее животное, хоть и очень умное для своего вида!
Что оно будет делать с нейроинтерфейсом?

Однажды Нейросеть пробралась в технический ход и исписала цифрами все поверхности, которые смогла.
Выгоняли хулиганку целый час.
Ворона обиженно отрекомендовала Пирожок как <<глупую драчливую собаку>>.
Чувства вороны оказались взаимными.

\mulang{$0$}
{"--*Баночка, держи свой мясной привод подальше от систем, "--- раздражённо сказала Пирожок, "--- а не то я его лично утилизирую под соус!}
{``Flask, keep your meat actuator away from the system,'' Cupcake indignantly said, ``or I'll recycle it with sauce by myself!''}

\spacing

\section{[*] Обещание Короля}

\epigraph
{Прежде суд, долго сохранявший черты религиозного института, требовал от человека раскаяния.
В случае раскаяния приговор могли смягчить, а то и вовсе отменить.
И сейчас я скажу "--- это самая большая ошибка законников прежних веков.
Если честный человек убил случайно, для него не найти худшего палача, чем он сам, и худшей тюрьмы, нежели тюрьма его мыслей.
Если же честный человек мстил или защищался, то он поступил по своей совести;
в чём ему следует раскаиваться?
Раскаяние унижает человека, страх наказания заставляет его вступить на путь лжи;
униженный и запуганный лжец "--- хороший раб, но плохой гражданин.}
{Анатолиу Тиу.
<<Земной суд>>}

\spacing

"--*Я хочу выслушать этих кутрапов, "--- сказал Король-жрец.

"--*Я бы хотела сохранить их имена в тайне, во избежание\ldotst

"--*Зачем ты просишь моей помощи, Митхэ ар'Кахр, если между нами нет доверия?

Митхэ замялась и замолчала.

Ситуацию спас Аурвелий.
Старик вышел в центр зала и учтиво прикоснулся мизинцем к губам.

"--*Судя по приветствию, ты ноа с Могильного берега, "--- кивнул Король-жрец и повторил жест.
"--- Как тебя зовут, сенвиор?

\mulang{$0$}
{"--*Аурвелий Амвросий, "--- прогудел старик.}
{``Aurweli Amwurosi,'' the old man said sonorously.}

"--*Душистая Золотая Полынь, "--- перевёл Король-жрец.
"--- Красивое имя.
Почему бы тебе не назваться, как сели?

"--*Я взять от джунгли достаточно для себвя, "--- с достоинством ответил старик.

"--*Так почему ты стал кутрапом?

\mulang{$0$}
{"--*Я всю жизнь быть пвират, Король-жрец.}
{``All my life I pwirate, Priest-kwing.}
\mulang{$0$}
{В молодость я быть пвират с моё удовольствие;}
{Young I pwirate with my pleasure;}
\mulang{$0$}
{средний я быть пвират из-за неквуда идти.}
{middle I pwirate bwecause nowhere go to.}
\mulang{$0$}
{Сейчас я стар и устать.}
{Now I old and twired.''}

Жрецы возмущённо забормотали.

"--*То есть ты всю жизнь грабил и убивал, а теперь требуешь от нас спокойной старости, ноа? "--- осведомился один из них.

"--*Бвольшая часть жизни я делвать это бвез ввыбора, "--- спокойно ответил Амвросий.
"--- Ввыбор появвиться пять дождь как, капита Миция слушать мвеня, отличие другвой.
Я не имветь золото и кукхватр, я не имветь дом и семья, я не иметь даже virda navigata com textira solata\footnote
{<<Зелёный парус с вытканным солнцем>> "--- красивое крепкое судно.
Вероятно, образ пошёл из \emph{речей} ноа "--- историй без фабулы, а конкретно "--- <<Речи о мужчине>>. \authornote},
как подобает мужчине.
Я ничего требвовать, я хотеть служить для покой.

"--*Мы тебя поняли, сенвиор Амвросий, "--- кивнул Король-жрец.
"--- Что насчёт оставшихся двоих?
Будут ли они смелее старого ноа?

"--*Король-жрец, не смей оскорблять моих воинов при мне, "--- на грани шёпота произнесла Митхэ.

"--*Я не оскорбиться, капита Миция, "--- поднял ладони Аурвелий.
"--- В мой годы поквой важнее чей-то плохие словва.

Вперёд вышел Ситрис.

"--*Ситрис ар'Эр э'Тхинат, разбойник.

Жрецы снисходительно улыбнулись.

"--*Это всё, что ты хотел нам сказать? "--- осведомился Король-жрец.
"--- Сенвиор Амвросий хотя бы рассказал, почему он желает снять с себя звание кутрапа.

"--*Это вас не касается, "--- отрезал Ситрис, вызвав новую волну возмущённого шёпота.
"--- Есть товар, и есть покупатель.
Подробности деяний и чувства "--- моё личное дело, и если мне не изменяет память, перед сидящими в зале я чист.

"--*Мы тебя поняли, Ситрис ар'Эр, в твоих словах есть истина, "--- холодно сказал Король-жрец.
"--- Кто же третий?

Ликхлам вышла вперёд и нерешительно потопталась на месте.

"--*Ликхлам ар'Сатр, "--- тихо сказала она.
"--- Я люблю мужчину "--- уроженца Тхартхаахитра, но не могу жить с ним в одном доме.
Я кутрап-насильник, воровала оружие и золотые вещи.
Я не Разрушитель и никогда им не была.
Я помню всех, кого убила в своей жизни, и могу посчитать их пальцами одной руки "--- один пылерой, трое Молчащих идолов и человек, который пытался меня ограбить\ldotst

Ликхлам подняла растопыренную ладонь, по которой стекали крупные грязноватые капли пота.

"--*Достаточно, "--- Король-жрец остановил Ликхлам и обратился к Митхэ:
"--- Интересно, что делает в отряде чести женщина, которая может одной рукой сосчитать убитых ею врагов?

"--*Она готовит пищу и содержит в порядке упряжь, "--- с достоинством ответила Митхэ, заглушив не очень-то вежливый шёпот наёмников:

"--*Да какая тебе разница?

"--*А сам-то ты много настрелял?

"--*Следи за своими курами, петушок.

Жрецы возмущённо забормотали что-то о <<гостеприимстве>>;
Король-жрец, если и услышал, то не подал вида.

"--*Что ж, Митхэ ар'Кахр, я вижу, что \emph{почти все} кутрапы готовы стать частью народа.
Даю тебе слово: я сделаю всё от меня зависящее, чтобы они смогли жить там, где захотят.
А теперь следуй за мной, нам нужно обсудить стратегию.

Король-жрец поднялся и властно поманил Митхэ за собой.

"--*Тхэай, не нравятся мне эти данные вскользь обещания, "--- пробормотал Ситрис в сторону.

"--*Молодые ноги хорошо бегать, Ситриций, "--- улыбнулся Аурвелий.

Ситрису вдруг ужасно захотелось обнять старика.

<<Если Король-жрец лжёт, Аурвелий, ты пойдёшь со мной.
Не думай, что усталость "--- оправдание для покорности судьбе>>.


\section{[@] Нейросеть. Туда}

\spacing

Нейросеть беспокоилась весь день.
Баночка, поговорив с ней, вдруг заплакал и сразу после завтрака вышел с вороной на воздух.
Мы, недоумевая, последовали за ними.

"--*Ты правда этого хочешь, подруга моя хорошая? "--- ласково спросил Баночка и что-то прокурлыкал.

Ворона ответила нежным карканьем, попрыгала и покрутилась у него на руке.

\mulang{$0$}
{"--*Я буду ждать тебя до смерти как друг, "--- прошептал Баночка.}
{``I'll wait for you till my death as a friend,'' Flask whispered.}
\mulang{$0$}
{"--- И ещё пару дней как пища.}
{``And two more days as a food.''}

Он поцеловал птицу в клюв и подбросил в воздух.
Мы зачарованно смотрели, как ворона делает величественный круг над лагерем и исчезает вдали, в утренней дымке.
Наконец Баночка, грустно улыбаясь, повернулся к нам:

\mulang{$0$}
{"--*Гнездование.}
{``Nesting.}
\mulang{$0$}
{Инстинкт.}
{Instinct.}
\mulang{$0$}
{Пусть летит.}
{Let it fly away.''}

\section{[-] Обряд}

%Плуг 6 дождя 12002, Год Церемонии 20.

Вышел отряд затемно.
Солнца ещё не было, но в воздухе уже чувствовалась утренняя дымка.
Крики ночных птиц, квакш и мраморных цикад затихали, изредка робко подавали голос птички дневные.
Мы миновали несколько улиц.
Дома стояли чёрные, сонные, в окнах их не было ни огня.
В птичниках шуршали крыльями и стучали жёсткими лапками спящие куры.
Лишь на улице Дышащего Дерева мучимая бессонницей старушка, молча попыхивая едкой травой тумана, проводила нас взглядом со своего деревянного кресла.
Воины кивнули ей, и старушка ответила неторопливым кивком.

Мы шли молча.
Тишину прерывали лишь шуршание рубах, скрежет ремней и лёгкое позвякивание чьих-то плохо пригнанных лезвий фаланги.
Наконец отряд дошёл до небольшой площадки на окраине города.
Воздух, словно застывший в центре города, вдруг сменился лёгким живым ветром.
На меня пахнуло сильным ароматом прелой листвы и цветов.

"--*Пришли, "--- шёпотом сказала Ликхэ мне на ухо.

Воины так же молча, с торжественной серьёзностью начали снимать с себя мешки и доспехи.

Круг Доверия.
Обряд, который совершают во время войны или походов.

"--*Передохни, Ликхмас-тари, "--- шепнул мне кто-то из темноты.
"--- Пока что мы никуда не идём.

Кхарас тем временем быстро и умело развёл костёр, и я смог оглядеться.

Все, кроме меня и нескольких человек, уже успели раздеться до пояса.
Чханэ, улыбаясь и потирая ладонью красивый, натёртый поясом живот, подошла ко мне и игриво потянула за шнурки горловины:

"--*Лис, ты в одежде обряд собрался совершать?
Давай помогу.

Подруга развязала шнурки и, смеясь, стащила с меня рубаху.
Ликхэ тем временем, ревниво глядя на нас, сняла передник, открыв большие, похожие на зрелые дыни груди.
Благодаря им воительницу часто преследовали восхищённые взгляды мужчин, но в Храме говорили, что этот подарок Безумных может стоить ей жизни.
Удобный укреплённый передник, в котором богатство носилось, для Ликхэ сшила моя кормилица, которая считалась на необычное шитьё большой мастерицей.

Кхарас подбросил пламени сухих сучьев, и костёр заревел, как старый ягуар-узурпатор.
Искры летели вверх, к небесам, и мы с Чханэ зачарованно уставились в это небо, бездонное, полное звёзд.
Прошло несколько михнет, прежде чем я понял, что нашему примеру последовал весь отряд.

"--*Ликхмас-тари, что там? "--- шепотом поинтересовалась Кхохо.
Воительница впервые обратилась ко мне с уважительным суффиксом.

"--*Звёзды, "--- пожал я плечами.

Кто-то засмеялся.

"--*Так, начинаем, "--- громко сказал Кхарас, и словно в подтверждение его слов в огне лопнуло полено, выбросив в небо сноп искр.
Шёпот и смешки утихли.
"--- Хай, у нас новенькие\ldotst
Ликхмас, ты участвовал в Круге?

"--*Нет, "--- ответил я.
"--- Я даже на Мягкие Руки так ни разу и не попал "--- ни жрецом, ни участником.

"--*Хай, "--- понял Кхарас.
"--- Припоминаю, нам с ними последние двадцать дождей не везло.
То одно, то другое.
Тханэ?

"--*С двадцати пяти дождей участвую.

"--*Что ж, тогда мы начнём, а ты объяснишь Ликхмасу по ходу дела.

Чханэ серьёзно кивнула.
Потом, едва сдерживая смех, шепнула мне на ухо:

"--*Кхарас такой важный, как варан в брачный период.
Просто расслабься и делай то же, что и все.
И не вздумай фальшивить, за такое побить могут.

"--*Фальшивить\ldotsq

"--*Тихо.

Отряд сел вокруг костра и Кхараса, сцепившись кончиками пальцев на расслабленных руках.
Отстранённые лица осветил костёр. Кхарас встал лицом к огню, бросив на нескольких участников колеблющуюся, непроницаемо чёрную тень.
Рот его открылся, глаза закатились.

"--*Ааа\ldotst "--- разнёсся в ночной тиши его низкий голос.
Кхарас довольно долго тянул ноту, пока её не подхватили остальные.

"--*Ммм\ldotst

"--*Ааа\ldotst

"--*Ооо\ldotst

Разными голосами воины тянули ноты одного и того же мажорного трезвучия.
Я различил в общем хоре хрипловатый бас Ситриса, приятный контральто Чханэ и сладковатый альт Ликхэ\ldotst
Вскоре пришла и моя очередь.

"--*Ааа\ldotst

"--*Ооо\ldotst

Кто-то достал барабан и варган, голос обрёл ритм.

Та-та-та-хум!
Тхийо-тхо-тхоу!

Та-та-та-хум!
Йолль!
Йоуу!

Костёр горёл, его пламя плясало свой вечный танец.
Потянуло странным, слегка удушливым дымом.
Я вдохнул его полной грудью.
Удивительно, но он не вызывал желания кашлять.
Чужие голоса раздували меня, как ночной ветер океана раздувает парус.
Время поплыло, звёзды вдруг стали близкими и невообразимо горячими.
Я почувствовал себя всеми остальными "--- Кхарасом, в экстазе тянущим ноту и отбивающим на собственных коленях ритм;
Кхохо, закрывшей глаза и качающейся подобно дереву на ветру;
варганщиком Эрликхом, который извивался на земле, не прекращая отстукивать на крохотном инструменте животворящий, согревающий кровь ритм\ldotst
Надо мной в ночном небе загорелись глаза Чханэ.
Что это "--- реальность или моя фантазия, неотличимая от реальности\ldotsq

"--*Ааа\ldotst

"--*Ооо\ldotst

"--*Ммм\ldotst

Та-та-та-хум!
Тхийо-тхо-тхоу!

Та-та-та-хум!
Йолль!
Йоуу!

Вдруг Ситрис, не прекращая извиваться, схватил Чханэ за руку и притянул к себе.
Она, продолжая биться в том же ритме, впилась губами в его губы.
Вскоре они уже сплелись воедино\ldotst

"--*Лисичка\ldotst "--- сияющие зелёные глаза Ликхэ очутились прямо передо мной.
Воительница набросилась на меня сверху, как голодный ягуар на добычу.
Рука её обхватила мой стебель, и я почувствовал, как вхожу в её горячее нутро.
Словно откуда-то издалека раздался мой собственный стон, долгий, монотонный, идеально входящий в звенящее в моей голове мажорное трезвучие\ldotst

Та-та-та-хум!
Тхийо-тхо-тхоу!

И снова "--- глаза Чханэ среди звёзд\ldotst
Огромные, на полнеба, огненно-оранжевые глаза.
Я, сделав усилие, повернул голову, пытаясь найти среди танца обнажённых бронзовых тел подругу.
Она лежала под Ситрисом, её губы были раскрыты в истоме, а огненные глаза под благородным лбом не мигая смотрели на меня\ldotst

Я зарычал и впился в губы Ликхэ.
Миг "--- и она оказалась внизу.
Глаза её закатились, по телу пробежала волна\ldotst
И Ликхэ растворилась в воздухе.

"--*Лисичка\ldotst "--- низкий голос над моим ухом принадлежал Кхарасу.
Его мускулистые руки обхватили меня за талию, подбородок лёг мне на плечо.
Я почувствовал, как в меня входит его мощный стебель и застонал от смешанной с болью истомы\ldotst
Твёрдые губы Кхараса коснулись моих, в голове словно зазвенела толстая басовая струна экстаза\ldotst
И Кхарас растворился в воздухе.

"--*Ликхмас, "--- рядом появились живые, обрамлённые тёмными веками глаза Ситриса.
Я набросился на него и немного неуклюже повалил на землю, руки и ноги не слушались меня.
Ситрис охнул и довольно оскалился:

"--*Ох, какой ты горячий.
Осторожнее\ldotst

Ситрис откровенно наслаждался происходящим.
Он проводил пальцами по моему животу, и каждое лёгкое прикосновение заставляло меня биться в судорогах.
Воин смеялся и шептал какие-то нежности\ldotst

А голоса всё тянули и тянули:

"--*Ааа\ldotst

"--*Ооо\ldotst

"--*Ммм\ldotst

Та-та-та-хум!
Йолль!
Йоуу!

Я вдруг осознал, что никто больше не пел и не играл.
При этом музыка, потусторонняя, неземная, гудела у меня в голове.
На мгновение стало страшно, но в следующий миг путы страха разлетелись призрачными клочьями.
Я потерял контроль над разумом и чувствами "--- меня всецело поглотил транс.

Ситрис растворился в воздухе, как и Кхохо, и Эрликх, и Хитрам\ldotst

Та-та-та-хум!
Тхийо-тхо-тхоу!

Руки Кхохо, в отличие от Ликхэ, не дрожали, воительница делала всё чётко, как по уставу.
Лишь изредка, во время особо чувствительного пассажа, в мою шею вонзались её острые ногти, и вокруг них, словно круги на воде, разбегались радужные волны, остро пахнущие корицей.
В потемневших глазах я вдруг увидел непреодолимое, страстное желание задушить меня.
Я положил свои ладони на её и сжал её руками своё горло.
Воительница засмеялась странным, похожим на клёкот смехом, погладила мне шею и больше к ней не прикасалась.

Та-та-та-хум!
Йолль!
Йоуу!

Поющие голоса надрывно зазвенели.
Я слышал грозный, величественный рык Кхараса, одинокий отчаянный крик Кхохо, мечтательный смех Ликхэ, приносящий спокойствие баритон Ситриса, нежный счастливый напев Чханэ\ldotst
Голоса наполняли меня, вытесняя, выдавливая мою собственную сущность.
Мне "--- тому кусочку меня, который ещё продолжал соображать "--- вдруг снова стало страшно.
Я понял, что больше я себе не принадлежу.
Однако следом пришла ещё одна мысль "--- я никогда себе не принадлежал.
Едва мысль родилась, я успокоился так же быстро, как и испугался.

Эрликх валял меня по траве, как куклу, я потерял всякое представление о верхе и низе, но глаза воина смотрели за горизонт, сфокусировавшись на бесконечности.
Его голос звучал смиренно, словно пело согнутое в дугу молодое деревце.
Языки пламени играли на его груди спектакль теней, складываясь в благородные улыбающиеся лица "--- воспоминания о давно ушедших за грань\ldotst

"--*Ааа\ldotst
Лаа, лээй\ldotst

В небе вновь загорелись огромные глаза Чханэ, живые, невообразимо прекрасные\ldotst
Я любовался ими, зная, что больше никто не способен их видеть.
И только когда её губы нежно-нежно коснулись моих, словно девушка пыталась попить из ледяного горного водопада, только тогда я понял, что небо осталось где-то позади, а глаза действительно не иллюзия.

Мы с Чханэ завершили Круг друг у друга в объятиях.
Воины, разморенные утренней прохладой, лежали вповалку на земле и слушали пение птиц.
Голова Кхохо гнездилась на животе Ситриса, они оживлённо, с тихим смехом что-то обсуждали;
Ликхэ чуть в стороне рассеянно мяла в руках ароматные мелкие цветы;
Кхарас и Хитрам, примостившись рядом с провизией, с аппетитом жевали половинки пайка;
флейта Эрликха устало добирала последние вариации.
Наркотик больше не тревожил странными видениями, я чувствовал себя легко и спокойно.
Ночное небо бледнело, из чёрного становясь сиреневым.

Наступило утро.

\section{[-] Близость и даль}

\epigraph
{Увы, но даже сейчас, в Эпоху Богов, на нашей родной Земле есть закрытые уголки.
Сектантские поселения, изолированные от внешнего мира племена, не принявшие путь генной модификации общества.
Люди, живущие там, прозябают в невежестве, страдают от болезней и угнетения.
Иногда всё, что им нужно для лучшей жизни "--- это пересечь границу родного поселения, но они рождаются, живут и умирают в его стенах, олицетворяя полную страха жизнь человека прошлого.\\
О гуманности сказано достаточно моим другом Михаэлем [Михаилом Кохани], мне добавить нечего.
Всё, чего я хочу пожелать людям "--- это смелости.
Опыт и знания "--- ничто, если у вас нет смелости применить их.
Желания "--- ничто, если вы не осмелитесь их осуществить.}
{Людвиг К.\,Р.\,Вейерманн, первооткрыватель омега-поля}

\spacing

Я рассказал воинам об увиденном в трансе.
Ситрис выслушал с неподдельным интересом, Кхарас одобрительно крякнул:

"--*Парень-то чувствует.

"--*Это точно, "--- согласилась Кхохо.
Женщина после произошедшего начала относиться ко мне с большей теплотой.
Правда, выражалось это в основном в грубоватой ласке и тумаках.
Взглянув в её глаза, я понял: мне суждено было узнать о Кхохо то, что знали единицы.
Та тьма, потаённая, алчущая крови и страданий, но временно укрощённая, по-прежнему смотрела на меня из мерцающих зрачков.
Однако я чувствовал ещё кое-что: для меня эта тьма больше не представляла опасности.
Я подчинился и навсегда стал ей другом.

Ликхэ избегала говорить со мной, но я чувствовал её напряжение и знал, что скоро она не выдержит.

Так и случилось.
Воительница нашла меня возле ручья и, как обычно, взяла оленя за рога.

"--*Хочешь, я буду твоей женщиной? "--- спросила Ликхэ, присев рядом.
"--- У меня много умений и достоинств, и мне очень нравится заниматься с тобой любовью.

Я усмехнулся.

"--*Я тебя знаю с детства, как и все твои умения и все достоинства.
Нет нужды напоминать о них.

"--*Так за чем дело стало?
Чханэ может спать по одну сторону от тебя, а я "--- по другую.

"--*Ты ведь на этом не остановишься.
Неужели Ликхэ ар'Трукх спокойно будет делить моё ложе и меня с Чханэ?

"--*Ликхэ ар'Трукх потерпит, "--- мило улыбнулась девушка.

"--*Вот именно "--- <<потерпит>>.

"--*И что здесь такого?

"--*Ликхэ, мне не нужно, чтобы ты терпела.
Если бы ты искренне любила Чханэ и не желала владеть мной безраздельно, мой ответ мог бы быть другим.
А так "--- извини, мы тебя не примем.

"--*Мы?
Ты и за Чханэ тоже решаешь?
Или это Чханэ всё за тебя решила?

"--*Мы "--- один Храм, помнишь?

"--*Ты всё же подумай, "--- Ликхэ ворковала нежно, но я корнями волос чувствовал скрывающуюся за этой нежностью ярость.
Мой отказ воительница восприняла как личное оскорбление.
"--- А я пока нужду справлю.

Ликхэ встала, поправила складки на штанах и ушла в кусты.
Почти одновременно ко мне подошла Чханэ.
Девушка села рядом, рассеянно сорвала ароматный тёмно-зелёный лист <<белого пламени>>, потёрла его о шею и сладко зевнула.

"--*Как прошёл Круг Доверия?

"--*Очень бодрит, "--- усмехнулся я.
"--- Кстати, почему нельзя фальшивить?

"--*Аккорд в голове слышал?
Слышал, конечно, все слышали, он для всех один.
А теперь представь, что кто-то сфальшивил.
У нас за такое били, потому что Круг насмарку.

Чханэ весело сморщилась и хихикнула.

"--*Мы с тобой всем понравились.

"--*Правда? "--- удивился я.
"--- А мне показалось\ldotst

"--*Ты ещё неопытен.
Вообще я немножко удивилась распределению ролей в вашем Храме.

"--*А что не так?

"--*Ну, это сложно объяснить, на уровне чувств.
Вот взять тебя.
Ты очень нежный, не сильный, но решительный.
Будь ты воином, у нас тебя сразу определили бы в разведчики.
Ликхэ нежная, но это нежность анаконды.
Из неё вышел бы отличный убийца, но она, как видишь, стрелок.
В Кхохо нежности нет совсем, но действует она филигранно "--- знает, когда применить силу, а когда нет.
Вот такие у нас шли в стрелки, но она ружьё в руки не берёт.
Эрликх "--- наоборот, много нежности и гораздо меньше осмысленности.
Таких обычно отправляли к детям на площадку, но он за хозяйство отвечает.
И больше всего меня удивило, что во главе отряда стоит Кхарас.
Он силён, но грубоват.
Заметил, наверное?
У нас во главе отряда всегда стоял самый лучший разведчик "--- они осторожны и совершают меньше ошибок.
Ситрис, мне кажется, больше подошёл бы.
Интересно было бы узнать, почему у вас по-другому.

"--*Ты почти угадала.
Кхохо не берёт в руки ружьё, потому что оно ей жутко надоело, но вообще она может выстрелить из отгрызенного тростника кактусовой иглой и пронзить птице оба глаза, даже с похмелья.
Эрликх часто подменял Конфетку на площадке.
А Кхарас с Ситрисом\ldotst
На самом деле их связывает что-то очень глубокое.

"--*Более глубокое, чем Ситриса и Кхохо? "--- лукаво прищурилась подруга.

"--*Я не лоцман и не ныряльщик.
Долгая история.
Они говорят, что так надо.

Из кустов показалась Ликхэ, поддёрнула штаны, мило улыбнулась нам и ушла к лагерю.
Мы с Чханэ кивнули ей.
Едва воительница скрылась из глаз, Чханэ, прищурив лукавые глаза, обратилась ко мне:

"--*О чём вы говорили?

"--*Она предлагала мне любовь, "--- честно ответил я.

"--*Так и знала, "--- вздохнула девушка.
"--- Она тебя ревнует.
Причём так, что у меня за кхене от её ревности глаза слезятся.
И как будем это решать?
Будешь спать и с ней?

"--*Нет, конечно.
И я постарался это до неё донести, "--- махнул я рукой.

Чханэ сплюнула.

"--*Думаешь, её это остановит?
Я знаю таких людей.
Они относятся к тем, кого хотят, как к собственности, и ревность их страшна, как шторм в проливе.
Будь осторожен и не бери ничего из её рук.
Я тоже постараюсь.

Я промолчал и укоряюще посмотрел на подругу.
Она смутилась.

"--*Лис, я понимаю "--- ты знаешь ей с детства.
И\ldotst

"--*И ещё я знаю, что ей можно доверять, "--- перебил я.
"--- Да, временами она бывает жестокой, временами вредничает, но мы "--- один Храм.
Это о чём-то говорит?

"--*Я очень хотела бы доверять ей так же, как ты.
Но я видела слишком много\ldotst

"--*Что?
Что ты видела?

Чханэ скрестила руки и положила голову на колени.

"--*В чём проблема, объясни.
Разве вы устраиваете Круг Доверия не для того, чтобы доверять?

"--*Именно для этого, "--- кивнула Чханэ.
"--- Для меня было бы лучше узнать о её интересах сейчас, чем когда яд или кинжал войдёт в моё горло.
Обычная вещь в храме\ldotst

"--*Обычная? "--- недоверчиво спросил я.

Чханэ оглянулась и сделала вид, что поправляет доспехи.
А потом приникла к моему уху:

"--*За то время, что я служила Храму Тхаммитра, власть в Храме сменилась три раза.
Просто ночью из катакомб тайком выносили тела, а утром в зал выходил новый Первый жрец.
Не было ни шума, ни разговоров, в городе никто ничего не знал.
Да и кого интересуют внутренние передряги Храма, если Храм выполняет свою работу?
Мне повезло "--- ни один переворот я не застала.
Просто приходила утром, видела другого человека во главе стола и обращалась к нему, как к прежнему, не задавая вопросов.

Я ошеломлённо молчал.
Чханэ посмотрела на моё лицо и тихо вздохнула.

"--*Не там я родилась.

"--*У меня даже мысли не возникало, что может быть иначе, "--- признался я.
"--- Между адептами нашего Храма всегда были если не дружеские, то по крайней мере добрососедские отношения.
Я не понимаю, что в Храме делить, из-за чего спорить?
Мы и так почти всё получаем бесплатно\ldotst

"--*Факт остаётся фактом.
Как бы я хотела уйти от всего этого, духи лесные, как бы я хотела\ldotst
Уехать туда, где нет ни войны, ни интриг\ldotst

"--*Нам некуда ехать, "--- заметил я.
"--- Безумные правят везде.

"--*Да ну? "--- отстранённо, словно кому-то другому, пробормотала девушка.

Я насторожился.

\mulang{$0$}
{"--*Объясни.}
{``Explain.''}

Чханэ замолчала и попыталась встать, чтобы уйти.
Я повис на её руке и вернул на место.

\mulang{$0$}
{"--*Объясни.}
{``Explain.''}

Чханэ мотнула головой:

"--*Это всего лишь слухи, я не\ldotst

"--*Позволь мне решать.
Говори, что знаешь, "--- я чересчур сильно сжал её руку и опомнился, только когда она свирепо шмыгнула носом.

Чханэ потёрла затёкшую кисть и, поколебавшись, пробурчала:

"--*Кристалл.
Там нет ни алтарей, ни жертв.
Так говорила бабушка.

"--*А радужное безумие?
Это ведь не выдумки?

"--*Нет, не выдумки.
Я видела его своими собственными глазами.

"--*Значит, и Безумный существует.
Но если где-то люди живут без жертв\ldotst
Почему у нас по-другому?
Это какой-то заговор?

"--*Говори тише, Лис.
Если и заговор, то воины тут ни при чём, "--- отрезала Чханэ.

"--*Ты намекаешь, что при чём здесь жрецы?

Чханэ посмотрела на меня, прикидывая, стоит ли говорить то, что вертится у неё на языке.
Потом нервно оглянулась и снова сделала вид, что поправляет доспехи.

"--*Помнишь Сатхира ар’Со?
Я тебе про него рассказывала, когда\ldotst

"--*Помню.

"--*Так вот.
Его брат, тоже жрец, очень сильно пил.
Ситлам-лехэ, милейший старичок.
Одно время я часто бражничала с ним вместе.
Он жаловался, что не может больше смотреть на страдания детишек.
И я знаю, что многие жрецы в обход запрета утешают жертв, пытаются облегчить\ldotst

\mulang{$0$}
{"--*Чханэ, идол тебя покусай, короче!}
{``Chh\r{a}n\^{e}i, the idol beat you, tell briefly!''}

Девушка сверкнула на меня глазами.

"--*Сатхир был другим.
Неограниченная власть его развратила.
Он мучил с нескрываемым удовольствием.
Думаю, что я не первая, кого он повёл в джунгли "--- ему просто перестало хватать обычных жертв.

"--*Но как это возможно?
Что это за\ldotst

"--*Он не был таким всегда, Лис.
Это похоже на болезнь, "--- быстро зашептала Чханэ, наклонившись ко мне.
Видимо, когда-то эта метаморфоза повергла молодую воительницу в ужас, далёкие отголоски его метались в её оранжевых опалесцирующих глазах.
"--- Его взгляд\ldotst он нездоровый.
Пустой.
А самое страшное "--- что таких больных очень много, они знают друг о друге и координируют свои действия.
\mulang{$0$}
{И я не одна, кто так думает.}
{And I'm not alone who thinks so.}
Это заговор не жрецов, а кого-то гораздо более могущес\ldotst пойдём-ка, Лис, не нужно оставлять товарищей надолго.

Обратный путь мы проделали в молчании.

\mulang{$0$}
{"--*Надеюсь, ты понимаешь\ldotst "--- наконец начала Чханэ.}
{``I hope you realize\dots '' Chh\r{a}n\^{e}i finally said.}

\mulang{$0$}
{"--*Я буду молчать, "--- отрезал я.}
{``I won't tell anyone,'' I interrupted her.}
Это получилось чересчур резко, и я чуть мягче добавил:
\mulang{$0$}
{"--- И подумаю над твоим предложением.}
{``Also, I will think about your suggestion.''}

\mulang{$0$}
{"--*Каким предложением? "--- удивилась Чханэ.}
{``What do you mean?'' Chh\r{a}n\^{e}i looked surprised.}

\mulang{$0$}
{"--*Ну, Кристалл не так уж и далеко, "--- смущённо усмехнулся я и пожал плечами.}
{``Well, Crystall never was far away,'' I shrugged with a shy smile.}

Лицо Чханэ просветлело, и она тут же прикрыла улыбку ладонью, словно боялась, что та упорхнёт.
Девушка схватила мою руку и сжала её "--- с теплотой и благодарностью.

\chapter*{Интерлюдия VII. Пророчество}
\addcontentsline{toc}{chapter}{Интерлюдия VII. Пророчество}

\textbf{Легенда об обретении}

Было то во времена, когда все народы земель и морей жили в мире, и духи ходили по земле, словно люди.

Пришёл тот, кто был назван Ликхмас, в город Ихслантхар.
Город не заметил юношу, как не замечал прочих.
И пришёл Ликхмас на постоялый двор, что стоит на перекрёстке трёх дорог\footnote
{Перекрёсток трёх дорог "--- площадь в квартале ювелиров Ихсланхара.
Среди сели есть мнение, что это заколдованное место, и там стоит невидимый постоялый двор для лесных духов. \authornote}.
Он заказал себе пищу, и [принесли ему] тарелку кукурузы, сладости и чашу с водой\footnote
{Пища, которую поднесли Ликхмасу, на самом деле представляет собой классическое подношение умершим "--- зерно, сладости и хэситр. \authornote}.
Пока Ликхмас трапезничал, к нему подсел старик, закутанный в рваный чёрный плащ.
Глаза его горели, как звёзды, а седая борода развевалась, словно на шквальном ветру, хоть сидел старик в душной таверне.

<<Жизнь и процветание, лехэ, "--- поприветствовал старца [Ликхмас].
"--- Не желаешь ли ты разделить со мной пищу?>>

Сверкнули глаза старца.

<<Я поклялся более никогда не разделять пищу с людьми>>.

<<Отчего так?>> "--- удивился юноша.

<<Я был духом, "--- ответил старик.
"--- И наследница вашего рода лишила меня того, благодаря чему я был нужен этому миру.
Она не лишила меня сил, она не лишила меня умений, но я не смею тронуть ту гармонию, что она установила хлопком одной ладони.
Если я не найду другого занятия, Ликас ар'Тро, я стану каменным духом из тех духов, что [существуют] без цели и смысла>>.

<<Сочувствую твоему горю, лехэ, "--- ответил юноша.
"--- Но это не повод лишать себя простых радостей, ибо они есть цель и смысл среди прочих [целей и смыслов].
И да, меня назвали Ликхмас.
Откуда бы ты ни знал то имя, Ликас ар'Тро мёртв, и никогда больше [он не будет живым]>>.

<<Значит, я пришёл туда, куда нужно, "--- ответил старец.
"--- Духи знали о приходе Безумного.
Духи знали о том, что мир ждёт его владычество, ибо было то предсказуемо и неизбежно.
Но так же [предсказуема и неизбежна] его гибель>>.

Ликхмас промолчал.
Он слушал, как вслушиваются в небеса, пытаясь различить жужжание светлячка, кружащего вокруг звезды.

<<Кихотр Безумного может вершить судьбы.
Ни один смертный не сможет поднять и бросить этот камень без вреда для себя.
Но существует человек, который может выбросить ту из бесчисленных граней, что несёт знак смерти самого Безумного.
Ибо не Безумный сотворил кихотр, ибо он не мастер из тех мастеров, что куют собственный инструмент>>.

Последние слова порывом ветра пронеслись по таверне, и люди, идолы, пылерои и травники замолчали, слушая старца.

<<Этот человек жив и мёртв одновременно.
Этот человек имеет имя и лишён [его].
Это сидящий человек и бегущая лань в один и тот же момент>>.

Старец встал и вышел вон, и лишь ветер провыл ему вслед.

Ликхмас обратился к сидящим в таверне:

<<Живые и мёртвые, духи, люди, идолы, пылерои и травники.
Знает ли кто из вас, где найти кихотр Безумного?>>

Вперёд вышел воин, имя которому было Маликх\footnote
{Маликх (правильнее Муал'ликх) "--- <<владыка>>, дословно "--- <<возлежащий на цветах>>.
Титулование главы знатного рода в племени тенку. \authornote}.
Он не носил доспехов, и волосы его были так длинны, что конец косы лежал за воротами города\footnote
{Воины сели обычно стриглись коротко.
Длина волос Маликха могла быть намёком как на невозможность его существования, так и на его невероятное мастерство, помогавшее быть воином даже с волосами такой длины. \authornote}.

<<Я не знаю, где находится кихотр, но я защищу тебя от любого бедствия, что встанет на твоём пути>>.

Затем вперёд вышел купец с орлиным носом, острым подбородком и ртом, [способным вместить] яблоко целиком.
Имя ему было Чхалас, и была его речь так сладка, что даже лес и горы уступали ему, если [он обращался] к ним.

<<Я не знаю, где находится кихотр, но я смогу найти того, кто это знает, и он поведает мне всё без утайки>>.

Затем вышла вперёд идол, у которой маска прикрывала лицо, ибо половина её головы была куском дерева\footnote
{Не вполне ясно, как в Легенде появился этот персонаж.
Описание его напоминает планта, страдающего наследственным верхнечелюстным экзостозом, который в тяжёлой форме напоминает древесную кору.
Возможно, что такие планты были объектом почитания у некоторых племён Тра-Ренкхаля;
тем не менее, подобные традиции не сохранились ко времени перехода планеты под знамёна Ордена Преисподней. \authornote}.

<<Я не знаю, где находится кихотр, но я отдам вам ветвь из своей головы.
Она за взмах ресниц способна прорасти в лес, столь густой, что ни один враг не пройдёт через него>>.

Затем вышел вперёд старый травник о трёх ногах\footnote
{Танцуют-Четыре-Камня "--- трёхногий травник, персонаж-трикстер из фольклора Бродячего Народа.
Согласно легендам травников, Танцуют-Четыре-Камня был заговорён Безумным на короткую жизнь, но научился перемещаться между параллельными жизнями, <<словно нить утка между разноцветными нитями основы>>.
То есть его душе тысячи дождей, он пережил огромное количество приключений, но ни в одной реальности не дожил даже до возраста инициации.
Упоминание Танцуют-Четыре-Камня как старика "--- указание на необычность, сверхъестественность описываемого места;
в постоялом дворе на перекрёстке трёх дорог Безумный власти не имеет. \authornote}.

<<Я не знаю, где находится кихотр, но мне было суждено оказаться здесь, и мне было суждено заговорить.
Посмотрите на мои ноги "--- их число равно числу дорог, что сходятся на этом перекрёстке.
Но я выкуклился с четырьмя ногами, и значит, что четвёртая дорога должна быть>>.

Встал травник по сторонам света, расставил целые ноги по направлению дорог "--- и указала культя его ноги [в нужную сторону].

И на рассвете вышли Ликхмас, Маликх и Чхалас из Ихслантхара и пошли по четвёртой дороге\footnote
{Современные сели <<четвёртой дорогой>> называют неочевидный путь.
Например, если свернуть с тропы и пойти через глухую чащу "--- это будет четвёртой дорогой.
Если вместо двери пробиться через стену или прокопать подземный ход "--- это тоже будет четвёртой дорогой. \authornote}.
И тьма сомкнулась над ними "--- отрезана [четвёртая дорога] от мира, подобно ноге старого травника, и нет над ней ни светил, ни небес.

\chapter{[-] Поход}

\section{[-] Четвёртая дорога}

"--*Быстрее, Лисичка, не отставай, "--- Ликхэ хлопнула меня по спине.
"--- Следы холодные, надо спешить.

"--*Мы успеем перехватить их до Одинокого Столба? "--- спросил я.

"--*Успеем, если прибавим шагу, "--- Ликхэ напряжённо вглядывалась куда-то в заросли.

"--*Кхарас, идём по четвёртой дороге, "--- Ситрис тоже выглядел так, словно собирался броситься галопом в заросли.
"--- Ликхэ дело говорит, они нас опережают на пару шагов.

Кхарас нахмурился, но промолчал.

По дороге попался колодец.
Ликхэ и Хитрам по привычке кинули туда по золотой грануле, пробормотав под нос имена детей и молитвы Ситу;
Ликхэ пробормотала одно имя, Хитрам задержалась гораздо дольше.
Эрликх, оглянувшись по сторонам, стянул с руки кольцо и тоже бросил в воду.

Последним мимо колодца прошёл Ситрис.
Он окунул в воду ладонь, поднёс ко рту и тут же выплеснул горсть на землю, поморщившись от незнакомого запаха.

"--*Дикари, "--- пробормотал он, отряхиваясь.
"--- Если не цените жизней сели, так цените хоть труд каменщика!

"--*Что такое? "--- обернулся Кхарас.

"--*Что-то в колодце, "--- пояснил Ситрис.
"--- Пробовать не советую.

"--*Колодец в двух кхамит ходьбы от хутора, "--- помощилась Ликхэ.
"--- Вот какова была вероятность, что мы будем пополнять из него мехи?
Точно дикари.
Надо потом сказать советам, чтобы почистили.

"--*Стоять, "--- рявкнула вдруг Хитрам.

Мы замерли.
Воительница осторожно вытащила из мочажины туго свёрнутую ветвь какого-то колючего кустарника и принесла Кхарасу.
Мне почудился слабый едкий запах.

"--*Тхэтраас, "--- резюмировала Ликхэ.
"--- Кишки рыбьи, ещё и яд необычный, какое-то сложное зелье.
Наверняка не сразу убивает.

\mulang{$0$}
{"--*Всё правильно, "--- заметил Ситрис.}
{``They got that right,'' S\~\i tr\v{\i}s noted.}
\mulang{$0$}
{"--- Быстродействующий уменьшает количество врагов, но не их скорость.}
{``Fast-acting poison reduces number, but not speed of the enemy.}
\mulang{$0$}
{Труп незачем тащить, труп не требует заботы.}
{A corpse needs no carry and no care.}
\mulang{$0$}
{Кхарас, не будь рыбой, дорога тухлая.}
{Kch\'ar\v{a}s, don't be a fish, the road's rotted.}
\mulang{$0$}
{Сворачиваем.}
{We've to get off''.}

Кхарас хмуро смотал чётки Сата и скомандовал сойти с тропы.

\section{[-] Змея}

\spacing

С дерева соскользнула тёмная змея, и Кхохо, выругавшись, упала в реку.
Мутная зеленоватая вода сомкнулась над ней.

Я хотел броситься следом, но Эрликх, выдохнув <<Хэ>>, схватил меня за шиворот.

"--*Не сейчас, Лис, "--- домашнее имя означало серьёзность ситуации.

Ситрис напряжённо вглядывался в воду.
Ни малейшего признака Кхохо.
Вдруг в зеленоватой воде показалась кровавая дымка.

"--*Змеиная, "--- констатировал Ситрис, окунув палец в воду и попробовав её на вкус.
"--- Надеюсь, Кхохо её убила.
Не дёргайтесь, сейчас главное "--- сапожника не проворонить.
Да стой ты, "--- рявкнул он на Ликхэ, собравшуюся прыгать.

"--*Если она не добила змею, та её задушит, "--- прошипела Ликхэ.

"--*Если мы пропустим сапожника, нырять будет уже не за кем, "--- объяснил Эрликх, так как Ситрис даже не удосужился посмотреть в сторону девушки.

Кхарас бросил Ситрису чашу, и тот налил в неё из бутылочки немного жидкости "--- белой, остро пахнущей, слегка напоминающей молоко на вид.
Кхарас подхватил чашу и перебежал на другой берег.

Вода вокруг забурлила.
На кровь приплыл косяк рыбы-сапожника.
Эта маленькая рыбка получила название за то, что за считанные секхар сжирала кожаные сапоги.
Ситрис и Кхарас тут же вылили молочно-белую жидкость в ручей.

Я почти услышал яростное щёлканье челюстей.
Несколько десятков уродливых зубастиков всплыли кверху брюхом, прочие предпочли ретироваться.

"--*Слава духам, что я взял молочник, "--- Ситрис бессильно прислонился к мшистому стволу.
"--- Эрликх, будь другом\ldotst я не в состоянии\ldotst

"--*Раздевайся, Ликхмас, "--- толкнул меня Эрликх.
"--- Я один её из змеи не выпутаю.

Кхарас так же молча снял с плеча верёвку, размотал и бросил нам конец.

\section{[-] Лекарство}

Вскоре исцарапанное, бледное как снег тело Кхохо уже лежало на берегу.
Я, Кхарас, Ситрис и Эрликх стояли рядом, прочих вождь отослал вперёд, разведать ситуацию.
Пульс и дыхание я не определил, но Кхарас махнул рукой: <<Всё в порядке, не суетись>>.

Эрликх методично сорвал с шеи Кхохо последних пиявок.
Из мест укусов тонкими ручейками текла кровь, оставляя на лодыжках, запястьях и лице алые полоски.
Я выдавил по капле коагулирующего раствора на каждую ранку.

"--*Ты чего? "--- удивлённо поднял брови Эрликх.

"--*Слюна пиявок\ldotst

"--*Да знаю я.
Успокойся, она не хрустальная, выживет и так.
В походе нет времени на каждую царапину.

"--*Маленькие раны могут дать большую кровопотерю.

\mulang{$0$}
{"--*Как знаешь.}
{``Suit yourself.}
\mulang{$0$}
{Это твоё время.}
{It's your time to waste.''}

Ситрис попросил мою коробку с инструментами, неторопливо развернул трубочку и с неожиданной сноровкой ввёл её подруге в горло по клинку ларингоскопа.

"--*Шприц.

Я протянул.
Ситрис надел трубку на шприц, поводил, потянул поршень шприца, нахмурился и ещё раз поводил трубкой.

"--*Молодец, всё правильно сделала.
Учись, Ликхмас.
Если знаешь, что тебя вытащат из воды "--- прикусывай язык что есть мочи и выдыхай весь воздух.
Хотя воздух тут, пожалуй, змея выдавила\ldotst

"--*А как её теперь оживить?

"--*Можно просто оставить, сама оживёт.
Но это скучно.

С этими словами Ситрис достал ещё одну бутылочку и капнул Кхохо в нос маслянистой жидкости.
Десять секунд спустя воительница зашевелилась с яростным мычанием и, нащупав флягу, начала трясущимися руками промывать нос.
Затем бросила флягу, подползла к воде и окунула в неё голову.

"--*Она меня теперь год будет по ночам лупить, "--- вздохнул Ситрис.
"--- Вещь ужасная, по себе знаю.

И, достав кисет с коноплёй и бумагу из водорослей, воин принялся сворачивать оленью ножку.
Кхарас невозмутимо поднёс ему тлеющий трут.
Он, похоже, знал о привычке Ситриса курить после сильных волнений.
Мужчины взглянули друг на друга и ухмыльнулись.

\section{[-] Везение}

От подобных происшествий не был застрахован ни один житель джунглей.
Ежегодно город терял несколько человек из-за змей, ядовитых пауков и аллигаторов.
Ещё с десяток возвращались с увечьями.
Кхохо повезло "--- встреча с питоном стоила ей нескольких треснувших рёбер.
Несмотря на уговоры Кхараса, воительница продолжила путь.
Она еле переставляла ноги и по-детски цеплялась за Ситриса, забыв о его жгучем сюрпризе.

\spacing

\section{[-] Разорённый хутор}

\epigraph
{Слабость людей "--- в разъединённости.
Существует много способов разделить людей, но самым древним, варварским и действенным способом является разделение по признаку пола.
Оно начинается с самого детства.
Мальчики и девочки получают разные игрушки, разные обязанности и права, их отправляют в разные школы, им предлагают разный набор профессий.
В обществе активно пропагандируется гомосоциальность, нагота противоположного пола позиционируется как постыдное зрелище.
Когда же сковывающие мужчину и женщину барьеры падают, они, охваченные безрассудной страстью, бросаются навстречу друг другу, не зная друг о друге ничего.
Цена невежества "--- взаимное неприятие, выливающееся в холодность, страх и насилие.
Чтобы управлять этими людьми, женщинам достаточно показать зверя-мужчину, а мужчинам "--- женское тело.}
{Мариам Кивихеулу, писательница времён Последней войны}

\spacing

Захватчики окружили горящий дом.
Но вскоре плач ребёнка утих.

"--*Там никого нет, Имжу, "--- сказал один из них.
"--- Ребёнок погиб.

"--*Ждите, "--- ответил главный.
"--- Он может быть со взрослым.
Взрослого убивайте сразу.

Предводитель сделал знак помощнику, и они вместе отправились на следующую улицу.
Воины заворчали, но остались ждать.

Наконец крыша рухнула, в небо вылетел сноп искр и угольков.
Горящая дверь отвалилась, и на улицу вышла девушка с ребёнком на руках.
Одежды девушки и ребёнка окутывал пар "--- она смачивала их водой.
Девушка зачем-то прикусила язык "--- с уголка рта текла струйка крови.
Волосы у обоих сгорели дотла, странно лоснящаяся кожа покрылась копотью, но оставшиеся чистыми участки кожи были белее горного снега.

Воины застыли в изумлении.
На девушке были ожоги, но явно не те, которые они ожидали увидеть.
Девушка швырнула ребёнка под ноги воинам, и те зашипели "--- ребёнок лежал бездыханный.

"--*Вам нужно что-то ещё? "--- спокойно спросила она, с трудом шевеля опухшим языком.

Воины несколько секунд рассматривали крестьянку.
Затем дружно расхохотались.

"--*А хороша, "--- довольно рыкнул один из молодых.
"--- Ну что, парни, развлечёмся?
Негоже побывать в землях сели и не попробовать баб.
У них характер есть, не то что у наших.

"--*И то верно, Перт, "--- сказал другой.
Он возвышался над девушкой, как скала.
"--- Эта немного пережарилась, но тоже сойдёт.

"--*Имжу сказал\ldotst "--- попытался возразить кто-то.

"--*Да срал я на этого выродка, "--- плюнул Перт, говоривший первым.
"--- Харт, ты что, бабьи ноги не удержишь?

Прочие рассмеялись.
Девушка презрительно оглядела мучителей.
Лицо из снежно-бледного стало ярко-красным, затем пошло пятнами.

"--*Вам нужно ещё и моё тело?
Подойдите и заберите, животные! "--- девушка сорвала с себя полусгоревшую одежду.

"--*Совсем стыда нет, "--- с отвращением бросил Харт.

Кряжистый детина, осклабившись, схватил девушку за пояс и попытался повалить.
Но его толстая шея вдруг хрустнула и неестественно вывернулась под тонкими девичьими руками.
Побагровевший воин рухнул на землю.
Палец девушки стрелой пронёсся в сторону другого воина и вошёл в глазницу, словно горячий нож в масло.

В поясницу ей тут же вонзилось три копья.
Девушка, истекая кровью, легла и свернулась клубком, словно спящий мангуст.
Ни стона, ни крика воины хака от неё так и не услышали.

"--*Что за отродье эти сели, Безумные меня разорви! "--- выругался один из хака и пнул девушку.

К воинам тем временем подошёл главный.

"--*Что у вас тут\ldotst НАЗАД! БЫСТРО ВСЕ НАЗАД!

Поздно.
Кровь вдруг перестала сочиться из колотых ран на спине девушки.
Лицо умирающей стало умиротворённым и равнодушным, и она бросилась на ближайшего врага.
Захрустел перегрызаемый хрящ "--- и воин упал, захлёбываясь кровью.
Та же участь постигла следующего.

Предводитель подлетел к девушке и одним ударом отрубил ей голову.
Затем окинул яростным взглядом оставшихся в живых.

"--*Перт! "--- заорал он.
"--- Я кому инструкции давал, тупая твоя башка?
Не подпускайте сели на расстояние вытянутой руки, даже безоружных!
Ни стариков, ни женщин, ни тем более мужчин!
В плен брать только детей!
Взрослых убивать ударом в сердце или голову, и никак иначе!

В горящих развалинах дома кто-то зашевелился, и на свет вылез обгоревший подросток.
Один из воинов, вытащив фалангу, двинулся к нему.

"--*Стой, выкидыш идола!
Сели только кажутся тщедушными, а в умирающих и вовсе вселяются мстительные тени.
Стрелу, целься в глаз!

Воин спустил тетиву.
Имжу грязно выругался.

"--*Хватит с меня\ldotst черви подземные.
Уходим, мы взяли в плен достаточно.
Разведчики сели уже здесь.
Один сидит в кроне, за синим домом.

Разумеется, речь шла обо мне.
Воины напряглись.
Перт медленно достал лук.

"--*Где он?
Я не вижу\ldotst

"--*Лучше молчи, Перт, или я оставлю тебя на корм шакалам.
Из-за того, что кое-кому захотелось поразвлечься, мы уже потеряли пятнадцать человек.
Забудь про разведчика.
Бери детей и отводи людей к нашим позициям.
Этого тоже возьми.

Имжу пнул ребёнка убитой крестьянки.

"--*Он дохлый! "--- запротестовал Перт и тут же осёкся, увидев взгляд Имжу.

"--*Он оживёт, Перт, "--- ласково сказал предводитель.
"--- На твоём счету чересчур мало голов, чтобы знать "--- трупы не бывают такого цвета.
Это либо каменная статуя, либо, чтоб тебя драли коты, притворщик-сели!
Ещё вопросы есть?
Первый отряд "--- со мной, прикрываем отступление.

Униженный Перт покраснел, словно ягода акхкатрас, но больше не роптал.
Воины, время от времени выискивая меня глазами, бросились исполнять приказ.
Я соскользнул с ветки и побежал обратно в лагерь "--- докладывать.

\section{[-] Звон топора}

"--*Значит, это всё-таки хака, "--- задумчиво сказал Кхарас, выслушав меня.
"--- Не ожидал от них такого.
Ещё и Инхас-Лака\ldotst

"--*А что не так с этими Инхас-Лака? "--- поинтересовалась Чханэ.
"--- Они на слуху, а почему "--- не вполне ясно.

"--*Насколько я понял Кхотлам, в союзе племён хака есть несколько группировок, которые искусственно подогревают ненависть к сели, "--- объяснил Кхарас.
"--- И старейшины Инхас-Лака "--- одна из таких группировок.
Нашествие Молчащих обезглавило эту группировку, плюс Кхотлам очень своевременно наладила дружеские отношения с вождём.
Однако то, что Инхас-Лака снова в числе напавших "--- плохой знак.

"--*Я думаю, что не все из напавших "--- хака, "--- заметил я.
"--- Их предводитель увидел меня, хоть замаскировался я хорошо.

"--*Хака не мог тебя увидеть, они не видят тепло, "--- вскинулся Кхарас.
"--- Думаешь, это сели?

"--*Не похож на сели, "--- ответил я.
"--- Но всякое может быть.
Вдруг это пираты-ноа в боевой раскраске хака?

"--*Ноа в союзе с хака? "--- ошарашенно спросила Чханэ.

"--*В таком случае маскируемся предельно аккуратно, "--- бросил Кхарас остальным.
"--- Молодец, Ликхмас.
Если этот человек, кто бы он ни был по крови, ещё и координатор, то мы бы хлебнули кровушки.
Кхохо, беги обратно, от тебя сейчас толку нет.
Сообщение: <<Красный топор звенит о дерево.
Трапеза у ветвистых рогов старой рощи, дальше по шестипалым следам с запахом какао>>.
А мы выдвигаемся.

В кои-то веки Кхохо не стала спорить.
Звон топора означал всеобщую мобилизацию.
Красный "--- кодовый цвет ноа.

"--*Давненько не звенел красный топор, да, Донышко? "--- так, что слышали только я и Ситрис, шепнула женщина.

"--*Сало на твоей заднице было тоньше и мягче, "--- с грустью согласился Ситрис.
Но я чувствовал, что его печаль относилась отнюдь не к сказанному.

\section{[-] Лимнэ}

\spacing

"--*Идём-ка, дочка, "--- тихо позвал Ситрис Чханэ.

"--*Какая она тебе <<дочка>>? "--- тут же вскинулся Кхарас.
"--- Субординация!

Ситрис невозмутимо показал вождю оскорбительный жест и увёл Чханэ за дерево.
Ликхэ нахмурилась.
Похоже, её <<дочкой>> Ситрис не называл.

Вернулась подруга очень скоро.
Я её не сразу заметил среди листьев.

"--*Он меня раскрасил полосками, начернил ресницы и рассказал, как лучше вести бой в <<царстве орхидей>>, "--- радостно рассказала Чханэ.
"--- А ещё дал вот это.

Чханэ держала в ладонях лимнэ "--- камуфляжную повязку на глаза с козырьком.
Я слышал о таких вещах, но столкнулся с ними впервые.

"--*Чтобы глазами не светила, "--- пояснила подруга.
"--- Ситрис сказал, что видел такую на знакомом\ldotst "--- Чханэ понизила голос, "--- на знакомом пирате, который был оцелотом.
Ситрис не уверен, что сшил так, как надо, но носить можно.

"--*А ты раньше такие не носила? "--- удивился я.
"--- Ты же\ldotst

"--*Я же пустынная кошка, "--- усмехнулась Чханэ.
"--- В джунглях бывала редко, тем более, Скорпиону особо нет нужды прятаться.

\section{[-] Боевой вождь}

\spacing

Кхарас методично объяснил план боевых действий.
Многие, впервые попав в храм Тхитрона, удивлялись тому, что бесхитростного и прямолинейного воина выбрали вождём, Чханэ не была первой.
Тому существовали веские причины.
Тактику Кхарас строил грубо, но очень добротно и надёжно, гармонично сочетая нападение и оборону.
От него сложно было дождаться сомнительных действий и рискованных приёмов, ставящих под угрозу отряд.
Подобными делами в частном порядке занимался Ситрис, имевший большой опыт дерзких диверсий, и в особо сложных ситуациях Кхарас просто исключал Ситриса с подопечными из плана, позволяя тем действовать самостоятельно.
Это давало отличный результат, и ни у кого даже мысли не возникало оспорить роль вождя, ведь как бы хорош ни был диверсант, закрытая спина "--- прежде всего.

\section{[-] Прикуси язык}

\spacing

Для общения мы использовали язык, имитирующий крики лесных животных, пение птиц, шум листвы и треск деревьев.
Длинные сообщения передавались через наруч.
Переговариваться ультразвуком в этой густонаселённой части леса мы не могли "--- хака не слышали ультразвук, но мелкие птички начали бы волноваться и выдали нас с головой.
К тому же оставался этот неведомый пришлый\ldotst

Маскировочный плащ отлично гасил инфракрасное свечение тела, но долго сидеть под ним в душный летний вечер "--- настоящая пытка.
Я сидел, обливался потом в четыре водопада и надеялся лишь на то, что дурманящий аромат орхидей заглушит запах уже начавшего бродить пота.
Кажется, Кхохо что-то говорила на эту тему.
<<Прикуси язык>>.
Я не мог понять, пошутила она или нет, но решил попробовать.

И правда помогло.
После двадцати мучительных секунд руки вдруг стали холодными, а сердце словно провалилось в бездонную дыру.
Я понял, что вполне сочетаюсь по инфракрасной расцветке с деревом даже без плаща.

\section{[-] Схватка в лесу}

\spacing

<<Ликхмас, как дела?>> "--- донёсся от Кхараса нежный свист птенца согхо.

<<Враг в прямой видимости у земли в десяти на шесть>>, "--- отбил я на наруче.

<<Почему не стреляешь?>> "--- удивилась откуда-то Чханэ голубой квакшей.

<<Он может выдать товарищей>>, "--- объяснил я щёлканьем мохноножки.

Кхарас мяукнул, делая мне строгий выговор за риск.

<<И вообще, что за ящерицыно семя поставило жреца-новичка на передний фронт? "--- загудел наруч, передавая возмущение Ситриса.
"--- Мы такие богатые "--- молодёжью разбрасываться?
Кхарас, ты куда смотришь?>>

<<Поздно перегруппировываться, "--- отозвался Кхарас.
"--- Не засоряй эфир.
Ликхмас, постарайся не высовываться>>.

Мои ожидания оправдались через несколько томительных мгновений.
Воин хака сделал неосторожный знак головой.
Я проследил направление его взгляда.
Товарищ моей цели был истинным мастером.
Издалека его можно было принять за обросшую орхидеями ветвь.
Я немедленно поделился наблюдениями.

Ответ от Ликхэ пришёл спустя минуту:

<<Ликхмас, ты в переплёте.
Юнца, которого ты приметил первым, я взяла на прицел.
Второго не видит никто.
Это очень опытный разведчик, и, возможно, ты уже у него на прицеле.
У тебя один шанс.
Не промахнись>>.

Нежное грудное воркование кота-оцелота прямо-таки сочилось страхом.

Я медленно, горошина за горошиной, поднимал духовое ружьё.
Выверил угол, сосредоточился на слабом дуновении ветра, турбулентных потоках вокруг листьев "--- и, резко отклонившись в сторону, выстрелил.

Ружьё врага дёрнулось в мою сторону, но было поздно.
Стрела вошла точно в не прикрытое пластиной бедро воина.
Я ожидал крика отчаяния, но вместо этого прозвучал чёткий гортанный речитатив, в которой, по-видимому, сообщалось о примерном направлении полёта моей стрелы.
Хака действительно оказался опытным разведчиком.

<<Ликхмас, беги!>> "--- одновременно зашипела Чханэ и рявкнул Кхарас.

Когда тебя обнаруживают, на церемонии время не остаётся.
Я ломанулся через заросли, ломая ветки, стуча сапогами и полностью оправдывая данное мне имя.
Над ухом просвистела чья-то шальная стрелка, я повернул голову, уворачиваясь от неё\ldotst и едва успел заметить летящий в спину томагавк.
Я схватил его за рукоять левой, одновременно правой поднимая духовое ружьё.
Полный отчаяния стон спустя мгновение подтвердил, что стрела достигла цели.
И тут\ldotst дерево, просто идеальное, чтобы спрятаться.
Я пустил шум ломающихся ветвей вместе с вражеским томагавком куда-то в сторону, а сам вжался в тёплое, влажное дерево и замер.

<<Я в порядке, ещё одного зацепил по пути>>, "--- обрадовал я товарищей через наруч.
Кхарас злобно закудахтал, выражая крайнюю степень недовольства.

<<Второго я уложила>>, "--- сказала Ликхэ.

<<Один в пяти на восемь убит.
Кто-нибудь ещё видит?>> "--- завибрировал передатчик словами Чханэ.

<<Двое в кроне в тринадцати на два.
Одного убил, второй на земле со сломанной спиной, "--- сообщил Ситрис.
"--- Молодец, Ликхмас, устроил переполох.
Но больше так не делай>>.

Все остальные ответили отрицательно.

<<Остался один, "--- констатировал Кхарас.
"--- Ищите>>.

Однако поиск ничего не дал.
Последний хака затаился основательно.

\section{[-] Молодёжь}

\epigraph
{Жизнь не похожа на сказания.
Когда люди слушают истории про великие битвы, они радуются победам и грустят над поражениями героя.
О безымянных, безликих соратниках и врагах героя, которые тысячами погибают в каждой битве, которые приносят песчинки своих жизней на весы судьбы, "--- о них никто не вспоминает.
Но любому из нас отведена в истории именно эта роль.
Героями становятся единицы, и не по делам, а более по прихоти шкёльда\ldotst}
{Хервар Лонгсин-Храш, шкёльд (сказитель) двора Лодемора Сурового}

Мы спустились с деревьев.

"--*Это не опасно? "--- поинтересовался я у Кхараса.

"--*Нет, "--- ответил Кхарас.
"--- Хака знает, что сейчас попытка убить любого из нас равносильна самоубийству.
Он будет сидеть, надеясь, что мы уйдём.

"--*Но мы не уйдём, "--- ввернула Чханэ.

"--*Судя по его расположению, это должен быть тот самый координатор, "--- объяснил ей Кхарас.
"--- Сели он или ноа, но его просто необходимо взять в плен.
Ликхмас, иди обработай сломанного.
Насчёт твоего поведения поговорим потом.

Я, раздвинув заросли папоротника, прошёл к другой группе воинов.

Молодой хака, почти мальчишка, лежал на земле и тяжело дышал ярко-красным ртом.
По его лицу градом катился пот, смывая грязь, сок листьев и боевую раскраску.
Ситрис пытался снять с него доспехи, не тревожа перелом.

"--*Дети пришли за другими детьми, "--- сказал он, обернувшись ко мне.
"--- Но не играть.

"--*Точно.
Молодёжь одна, "--- с некоторым сожалением пробормотала Хитрам, разглядывая упавшего.
Она успела стащить пятерых хака под дерево и совершить над ними обряд.
"--- Красивые, весёлые.
И чего вам дома не сиделось, мальчики?

"--*Хотели славы и почёта, "--- хмыкнул Ситрис.
"--- Ликхмас, командуй.

"--*Ага, ещё церемониться с ним, "--- сказала Ликхэ.
"--- Забудь, Лисичка, я его так потащу.

Ликхэ грубо схватила пленника и взвалила себе на спину.
Хака взвыл и задёргал руками, его ноги висели безжизненными плетями.

"--*Ты что делаешь, рыбья печень? "--- крикнул Ситрис, нарушая все правила поведения в джунглях.
Птицы на ближайших деревьях подняли дикий гомон.
"--- Положи немедленно!

Ликхэ обиженно посмотрела на воина и бросила пленника на землю.
Тот снова дико, с подвыванием заорал.
Ситрис уложил его как можно ровнее, вытащил флягу с водой и облил голову пленника.
Затем яростно посмотрел на Ликхэ:

"--*Иди веток на носилки и шину нарежь, животное!
Ликхмас, "--- обратился он ко мне гораздо спокойнее, "--- действуй.
Ты знаешь, что нужно.

Ликхэ злобно сплюнула и ушла в лес.

"--*Поплюйся тут мне ещё, соплячка, "--- вполголоса пробормотал Ситрис.
"--- Нелюди нынче какие-то пошли\ldotst
Нельзя же так с живыми\ldotst

Я покрутил в руках амулет и открыл крышечку Красного Сана.

"--*Тихо-тихо, малыш, "--- остановил меня Эрликх.
"--- Меньше каплю делай.

"--*Хай, не мешай парню, он лучше нас знает, "--- поморщился Ситрис.

"--*А всё-таки, "--- полушёпотом добавил Эрликх.
"--- Это хака, они слабее.
Я видел, как от обычной для нас дозы Красного хака засыпают и больше не просыпаются.
Да, вот такую, маленькую.
И под язык капай\ldotst

Спустя десять минут пленник лежал на жёстких носилках в маковом полусне.
Я в последний раз проверил наложенную шину.
Ликхэ стояла поодаль и исподтишка бросала злобные взгляды на Ситриса.
Вдруг выражение её лица изменилось, и она словно невзначай вытащила духовое ружьё и начала его чистить.

<<Я вижу последнего, "--- зажужжал наруч словами воительницы.
"--- В пятнадцати на три.
Беру прицел>>.

Никто из нас даже не изменил позы.
Я осторожно скосил глаза.
Молодой хака в трофейном шлеме сели, сидя в розетке большого папоротника-опахала, смотрел на нас из-под раскрашенных полуприкрытых век.

<<Безумные меня задави, я стрелы с кураре уронила, пока бегала, "--- пожаловалась Ликхэ.
"--- Давайте я его просто убью?>>

<<Нет, "--- отозвался откуда-то Кхарас пением ,,красного шара``.
"--- Он нужен живым>>.

<<Потеряла стрелы.
Как вы с Ликхмасом вообще дожили до своих лет?>> "--- философски прожужжал Ситрис.

Хака как будто понял, о чём идёт речь, и припустил на юго-восток со всех ног.
Ликхэ, на рефлексах выпустив ядовитую стрелку в пустое место, громко выругалась.
Мы с Ситрисом бросились в погоню.

\section{[-] Погоня}

<<Да он что, стальной? "--- бросил Ситрис на десятой минуте погони.
"--- Ликхмас, я на пределе.
Будь осторожнее, а ещё лучше "--- возвращайся.
Ну его к свиньям>>.

Хака оказался гораздо выносливее своих товарищей.
Я знал, что любой сели способен обогнать хака и накоротке, и на большой дистанции.
Но наш преследуемый петлял между ветвей, стволов и лиан как заяц и даже не думал уставать.

Ситрис отстал "--- тело его подвело, сказывался возраст.
Я же был в самом расцвете юности, комплекция позволяла проскальзывать в самые узкие щели, и погоня меня только раззадорила.

Но вскоре хака начал проявлять признаки усталости.
Всё чаще он вместо бега затаивался в зарослях, и мне приходилось напрягать все свои чувства, память и ум, чтобы понять, куда делся враг.
Вот дорогу мне перегородил извивающийся в предсмертных судорогах молодой питон "--- хака потратил несколько бесценных секунд на борьбу с неожиданно схватившей его рептилией.
Несколько раз враг пытался выстрелить в меня из духового ружья, и в последний я расколол его хрупкое оружие брошенным томагавком.
На дереве, за которым прятался преследуемый, остались свежие брызги крови.

<<Охотник никогда не должен забывать, что у отчаявшейся, уставшей жертвы когти и клыки растут отовсюду, откуда могут расти>>, "--- вспомнил я слова Конфетки и мысленно призвал на тренера благодать лесных духов.
Наука, которую он передал мне, сотням и тысячам стоила жизни.

Дальше идти было куда проще "--- хака источал отчётливый, неистребимый запах страха, а на листьях и стволах изредка оставались мелкие кровяные капельки.
Можно было даже не торопиться "--- я знал, что преследуемый бежит из последних сил и обязательно должен остановиться, чтобы решить дело боем.

Вскоре, как я и ожидал, хака решил перейти в контратаку.
Он сделал кольцо из запаха вокруг дерева, а сам взобрался по лиане и повис вниз головой с ножом наготове.
Но я был начеку и встретил прыгнувшего сверху воина ударом в подбородок.
Хака крякнул и мешком рухнул на землю, выронив нож.
Моя фаланга уткнулась ему в грудь.

"--*Natra mis\footnote
{\mulang{$0$}
{[Следующее твоё] движение будет последним (цатрон). \authornote}
{[Your next] move will be [your] last (Te's\'{a}tr\v{o}n). \authornote}},
"--- сразу обозначил я свои намерения.
Но пленник, похоже, даже не собирался двигаться.
Хака дышал с надрывом, хрипло и часто. Он сорвал на бегу застёжки нашейника, и шлем при падении слетел с головы, открыв молодое мужественное лицо с по-дикарски яркими глазами.
По левой щеке бежала кровь "--- мой томагавк рассёк лицевую артерию.

Я примотал пленника к ветке его же роскошными волосами, затем занялся обыском.
В сторону полетели два гибких потайных ножа, наруч со спрятанным в коже лезвием, два комплекта стрел и верёвочная удавка, искусно вплетённая в ткань рубахи.
Затем я занялся перевязкой артерии "--- хака потерял много крови.
Пленник, увидев зажим и отбелённый волос, угрюмо закусил воротник, и пока рана не была зашита, я не услышал от него ни звука.

\section{[-] Благородный баньян}

Вскоре я повёл хака в рощу благородного баньяна, оказавшуюся неподалёку от места схватки.
Могучее дерево, похожее на зал с колоннами, хранило свою землю от чужих.
На мшистой почве не было видно ни насекомых, ни травянистых.
Говорят, что в таких рощах сели устраивали храмы\ldotst
Догадку подтвердил лик Тёплого Хетра, уже почти заросший, но ещё угадывавшийся на стволе древнего дерева.
Я встал и ковырнул мох под ним "--- на свет выскочил край изящной жертвенной тарелочки из ослепительно белого нефрита, которым была богата Старая Челюсть.

Память услужливо предоставила сведения "--- на нынешней границе земель хака и сели были развалины древнего города с утраченным названием.
Гибельные руины "--- так их называли теперь.
После одного из землетрясений посреди города прошёл разлом;
западная возвышенность осталась пригодной для проживания, но сели оставили город, опасаясь набегов "--- разлом разрушил всю тщательно организованную линию обороны города.
Хака же суеверно избегали величественное заброшенное место, боясь гнева мёртвых.

Я рассеянно начал выкапывать тарелочку из мшистой почвы.
Пальцы ощутили чёткую, умелую вязь орнамента на её ребре.
Связанный хака исподлобья с интересом наблюдал за мной.

"--*Bv\o h ant naani? "--- спросил я.

Пленник молчал.

"--*Ликхмас ар'Люм, "--- добавил я, положив руку на грудь.
Хака посчитает это слабостью, но, как говорила Кхотлам, хозяина положения не должны заботить такие мелочи.

Пленник подумал, хмыкнул и наконец-то открыл рот.

"--*Imzhu.

<<Замечательное начало знакомства, "--- вспомнил я весёлые слова кормилицы, читавшей очередную дипломатическую переписку.
"--- Оба уверены, что победили, а для вояк это главное>>.

"--*Bv\o h laer mimi, Imzhu? "--- спросил я и, показав паёк, разломил его напополам.
Одну половину засунул в рот, вторую протянул Имжу.
Воин насмешливо сощурился.

"--*Думаес, ты купись меня еда, мальсик сели?

Парень коверкал слова и неприятно шепелявил, но говорил на сели, и, несмотря на множество ошибок, я его понимал.
Я вгляделся в пленника внимательнее.
Зелёные, подкрашенные углём глаза, искривлённый горько-агрессивной насмешкой небольшой рот, высокие скулы\ldotst
Едкий пот смывал смуглину и боевую раскраску, оставляя на лице дорожки чистой ярко-золотистой кожи.
Как и ожидалось, хака оказался поддельным.
Слово <<мальчик>> он произнёс с легко узнаваемой оскорбительной интонацией;
оно подсказало, что собеседником можно отлично манипулировать с помощью слов, относящихся к возрасту.

"--*Ты говоришь на сели?
Отлично.
Тогда давай так.
Я отведу тебя к нашим.
Вряд ли ты после этого останешься в живых, но я знаю, что ты хочешь жить, и жить как можно дольше.
Для этого взрослому человеку следует mimai\footnote
{Пить и есть (язык хака). \authornote}.
Принять mimai не есть предательство.
Bv\o h sare \o i laer limaa\footnote
{Ты понял или повторить? (язык хака). \authornote}?

Хака молча смотрел на меня.
Я поднёс половину пайка к его губам, и пленник жадно вцепился в неё.
Прожевал и проглотил.

"--*Вода, "--- сказал он.

Я влил ему в рот воды из фляжки.
Хака выпил предложенное.

Солнце, похоже, уже прошло зенит.
Следовало торопиться обратно.
Я начал собирать амуницию пленного в его мешок.

Вдруг моё внимание приковал к себе шлем.
Зелёная кожа, отделанная чёрными змеями, классический нашейник народа сели, знак Кхара-защитника, нарисованный загущённой кровью.
Явно работа известного мастера-сели, дикари таких никогда не делали.
Я перевернул его и посмотрел на изящный тиснёный иероглиф, украшавший охвостье нашейника: <<Солнечная, из дома Песка, из города Отравленного Вина>>.

Митхэ ар’Кахр э’Тхартхаахитр.

В голове моей, и так не особо свежей после погони, помутилось.
Перед глазами пронеслись виновато отдающая мне записку Кхотлам, Чханэ, намёки и недосказанности служителей Храма\ldotst
Я подошёл к Имжу, показал ему шлем и вытащил нож.
Пленник напрягся.

"--*Скажи, где человек, которому принадлежит этот шлем.

Имжу промолчал и повращал глазами, ища возможность сбежать.
Я взял его за волосы и поднёс нож к бешено пульсирующей сонной артерии.

"--*Кто твоя родильница?

"--*К-кто? "--- переспросил Имжу.

Я попытался вспомнить подходящий термин, бывший в ходу у хака.

"--*Кто твоя mama?

"--*Я сын восдь, мальсик сели.
Я не понимась\ldotst

Я мысленно упрекнул себя за незнание языков и задумался ещё раз.

"--*Твоя mama "--- w\o izh\footnote
{Женщина, связанная с мужчиной законом племени (язык хака). \authornote}
вождя?

"--*Я сын восдь, "--- повторил Имжу, оскорблённо шмыгнув носом.

Насколько я знал, у хака женщины никогда не занимали ведущие посты в племени.
В моём сердце, как ядовитая лоза, росли печаль и подозрение.

<<Запомни, Ликхмас, "--- зазвучал в моей голове голос Конфетки, "--- я учу тебя лишь владеть телом, но знания, которые у тебя в голове "--- оружие куда более страшное, чем целая армия.
Цивилизованные, разумеется, будут принимать тебя лишь как опасную фигуру, но дикарь будет смотреть на тебя, как на потустороннюю силу в человеческом обличье.
И любой враг, с которым ты встретишься, познает страх, едва увидев жреческое одеяние, даже если и не признается себе в этом страхе>>.

"--*Так, молодой хака, "--- я рывком поднял парня на ноги и бросил спиной на дерево.
"--- Отведёшь меня к вождю своего племени.
Если сделаешь это "--- получишь жизнь и свободу.
Если же нет\ldotst

Я провёл лезвием по шее хака. Он неподдельно вздрогнул, испугавшись такой крутой перемены моих намерений.

<<Ликхмас, Ликхмас, ты же мне поклялся!
``Я не оскверню себя пыткой\ldotse''>> "--- взвыл воображаемый Конфетка где-то на окраине сознания.
Но меня было уже не остановить.

"--*Я тебя уничтожу.
Ты будешь умирать медленно и мучительно, каждая капля крови, покидающая твоё тело, будет причинять тебе невыносимую боль.
Когда же ты умрёшь, я вырежу у тебя на лбу знак каменного духа.

Меня колотило в почти священном трансе.
Слова вылетали из моего рта, словно потревоженные летучие мыши из чёрного зева пещеры.

"--*Ты умрёшь обессиленным и никогда, слышишь, больше никогда не станешь мужчиной и воином!

\begin{verse}
Bv\o h her brai lar in izhe,\\
sht\o h mart sai likar ciche,\\
es di izhu aes sart\footnote
{У тебя не будет посмертия в живых лесах, мёртвый камень станет твоим телом, пока планета не рассыплется в прах (язык хака). \authornote},
\end{verse}

"--*скороговоркой пробормотал я первый пришедший на ум экспромт.

Хака крайне редко прибегали к угрозам;
они считали, что угроза равносильна проклятию "--- воззванию к суду Безумного, которые могли принять любое решение, вплоть до смерти проклинающего.
Стихотворная форма же в тысячи раз усиливала действие слов.

Судя по лицу пленника, угроза возымели своё действие.
Зелёные глаза стали почти чёрными, целая щека посерела.
Губы молодого воина дрогнули.

"--*Ты\ldotst ты не мось, мальсик сели!
Это мось только слуга Безумный!

"--*Si laar\footnote
{Я могу (язык хака). \authornote},
"--- полушёпотом сказал я и покачал перед лицом пленника амулетом Сана.

Хака несколько секунд смотрел на изящный бронзовый диск и, сглотнув, проворчал:

"--*Хоросо.
Я вести.
Шаман сели хотесь видесь восдь, а хака хотесь жись.

\section{[-] Как взрослеют дети}

\spacing

Я протащил Имжу мимо ощетинившейся копьями толпы и втолкнул его в хижину.
Тут же один воин легонько шлёпнул клинком копья по моему запястью, ещё пара упёрлась мне в бока.
Я сказал <<Хэ\footnote
{Междометие, предупреждающее об опасности для собеседника или третьего лица, в зависимости от контекста. \authornote}>>
и сильнее прижал острие ножа к шее пленника.
Копья отодвинулись на полпяди.

В хижине было тепло и душно.
Казалось, воздух закаменел от выедающего глаза запаха дыма.
В дрожащей мгле, подсвеченной неверным пламенем костра, я разглядел невысокую женскую фигуру.
Лицо женщины наполовину закрывали спутанные кудрявые волосы, плечи и посадка головы выдавали бывалого воина.

"--*Зачем ты пришёл, тари? "--- чёрные губы женщины открылись, показав зубы, похожие на обмытый рекой кварц.
Верхнего резца не хватало.
Отсутствие акцента подтвердило мои опасения.

"--*Встречай потомков, Митхэ ар’Кахр, "--- с этими словами я толкнул Имжу, и молодой воин упал на колени.
Сразу же три копья упёрлось в кожу моего нашейника.

"--*He gt, bv\o h iszh! "--- бросила женщина сорванным командным голосом, и копья медленно отодвинулись.
Я осторожно сделал шаг к пленнику, который, тяжело дыша, прислонился к резному деревянному столбу.

"--*Потомков? "--- как ни в чём не бывало осведомилась чернявая ведьма.

Я медленно отцепил от пояса змеиный шлем и бросил жалобно звякнувший доспех перед ней.
Она подняла его, перевернула, и на протянутую ладонь упал кусочек кожи.
Подведённые углём глаза расширились.

"--*Ke, yghd.

Воины поспешно вышли из хижины и опустили травяную занавеску.
В душном воздухе остались только приглушённый шум дождя, потрескивание брёвен и тяжёлое дыхание Имжу.
Женщина смотрела на меня во все глаза.

"--*Кто ты?

"--*Ликхмас.
Могла бы и вспомнить.

Родильница опустила голову, задумчиво теребя в руках письмо.
Наконец она решилась прервать молчание:

"--*Хай, так ты\ldotst

"--*Объясни мне одну вещь, "--- перебил я, и женщина замолкла, словно окаменев.
"--- Почему хака получил твой шлем и твоё плечо, а сели "--- полуистлевшую чернильную ложь?
Почему хака получили твой меч, а сели "--- сожжённые деревни и печаль?

Митхэ бесстрастно смотрела на меня, поигрывая шлемом. Я вздохнул.

"--*Ладно, это твоё дело. Расскажи, что с моим дарителем.

Митхэ пожевала губу, грустно усмехнулась и, присев, пригласила меня сесть напротив.

"--*Садись.
Imzhu, bv\o h legh.

"--*Haha, segh ni\ldotsq "--- попытался возразить юнец.

"--*Imzhu! "--- родильница повысила голос.

Имжу покорно сел рядом с нами, успев надменно крякнуть в мою сторону:

"--*Ghe di bv\o h, zhyu gerren si\ldotst

"--*А я "--- тот, кто сохранил жизнь сыну вождя, когда уже занёс над ним нож, "--- в тон юнцу ответил я.
Тот замолчал.

"--*Прошу прощения, Ликхмас\ldotst ар’Люм, кажется?
Мой питомец ещё не вышел из своего возраста.
Но он умён и храбр, из него выйдет хороший воин хака.

"--*Я не держу обид на побеждённых, "--- ответил я чьей-то давно услышанной фразой.

Родильница вдруг улыбнулась и помолчала несколько секхар.

"--*Твой боевой наставник "--- благородный человек, и ты "--- достойный его ученик.
Немногие смогли бы проделать путь, который прошёл ты.
Что ж.
Прости и то, что я не предлагаю тебе пищу.
Сейчас сели "--- враги хака, и мы не можем есть за одним столом, пока горит костёр войны.

"--*Это не моя вина.

"--*И всё же прошу меня понять.
После того, как ты привёл моего питомца, кхм, за шкирку, у воинов достаточно поводов сомневаться в моём авторитете.
Что с остальными?

"--*Все, кто не вернулся, мертвы.
Мы убили их как захватчиков, но оплакали как своих, "--- ответил я ритуальной фразой дипломата, как учила меня Кхотлам.

Митхэ вдруг снова улыбнулась той же улыбкой.

"--*Это было честно, "--- наклонила голову она.
"--- Итак, ты хочешь узнать тайну этой записки?

"--*Я в нетерпении.

Родильница помолчала, глядя в огонь.
Имжу тем временем нагрел какой-то сыпучий камень и начал затемнять им кожу.
Затем подправил сурьмой форму глаз и накрасил ресницы, явно пытаясь скрыть яркий зелёный блеск.
Пока Митхэ собиралась с мыслями, чужак по крови стал неотличим от хака.

"--*Я была воином, "--- заговорила наконец женщина.
"--- Одним из лучших воинов, не только в Тхартхаахитре, но и во всех Восточных джунглях.
Хотела бы я преувеличивать, но увы.
Командовала отрядом чести в боях против идолов, пылероев, случалось биться и с людьми "--- тенку, хака\ldotst "--- она кивнула на дверь.
"--- У меня было много мужчин, были и дети.
Всех и не упомнить\ldotst

"--*Ага, "--- с горечью проворчал я.
"--- И каждому доставалась в наследство записочка?

"--*Ты не оскорблясь мой mama! "--- зло выпалил Имжу, не поняв, о чём речь, но верно истолковав тон.

"--*Imzhu, ghyti ni.

Парень застыл.
Выражение его красивого диковатого лица не сулило мне ничего хорошего.
Родильница помолчала, жуя губу.

"--*Потом в Тхартхаахитре я встретила Атриса.
Сам он утверждал, что у него нет имени, а это прозвище "--- <<журавль>>\footnote
{Прозвище созвучно имени Хатрис, распространённому среди восточных сели. \authornote}
"--- ему дали женщины на рынке.
Он был беден, его одежда едва прикрывала тело.
Но он чудесно играл на цитре.
Его пение было тихим, почти неслышным, но услышавшие приходили послушать его снова и снова.
Атрис похитил мои мысли, и я попросила его пойти со мной.
С тех пор он шёл рядом.
Лицом он был солнцем, глазами он был сыном змеи, а ещё у него были тонкие длиннопалые руки.
Полгода пролетели, как согхо над рекой.
Он всё говорил, что я не для битвы, уговаривал меня бросить военное дело и осесть где-нибудь в спокойном месте\ldotst

Митхэ говорила нараспев, намеренно искажая тоновый рисунок и употребляя самые изысканные поэтические обороты.
Среди хака были те, кто немного понимал язык сели, и знать тему разговора им было совершенно ни к чему.

"--*Его ли цветы дали завязь?

Родильница опустила голову.

"--*Об этом знают лишь Безумный да Хри-соблазнитель.

"--*Что же знаешь ты?

"--*Меня приревновал один из жрецов.
Он давно хотел быть со мной, но я отвергала его.
С хака тогда подписали перемирие, и в городе было большое пиршество.

"--*Пиршество? "--- удивился я.
"--- А как же\ldotsq

"--*Молчание было нарушено.
Тот дождь наверняка помнят многие, ты можешь узнать о нём из рассказов людей.
Мы с Атрисом веселились, занимались любовью и мечтали о доме.
Нашем доме.
Но на второй день тот жрец что-то подсыпал в моё вино.
Одурманил меня.
Он насиловал меня и поил этим зельем.
Как долго это продолжалось "--- я не знаю, несколько дней\ldotst
Очнулась случайно в его покоях "--- он просто забыл в очередной раз опоить меня.
Выбежала в зал "--- мои воины очень удивились.
Они решили, что я всё-таки поддалась на уговоры Атриса и уехала с ним на Запад.
А послушник, с которым я подружилась, сказал по большому секрету, что цитру моей жизни отдали вождю хака, как раба.

Митхэ взглянула мне прямо в глаза.

"--*Я сразу же отправилась на его поиски, со мной пошли трое товарищей.
<<В сезон дождей созрели дыни\ldotst>>\footnote
{<<\ldotst Я родила тебе дитя, о сладкая лиана, опутавшая голову мою, о гость желанный чрева моего>>.
Отрывок стихотворения Эрхэ Колокольчик. \authornote}
Я давно знаю Кхотлам "--- она надёжный человек.
Из Тхитрона я пошла в путь в сомнениях, без сил.
Я не знала, жив ли Атрис, и не знала, кто засеял поля "--- <<захватчик ли, хозяин ли исконный>>\footnote
{<<Мы связаны, судьбу не изменить, и важно ль, кто на нас надел оковы "--- захватчик ли, хозяин ли исконный?>>
Отрывок из <<Легенды об обретении>>, песня Ликхмаса. \authornote}.
Я надеялась на лучшее.
Доспехи натирали моё тело, я задыхалась при ходьбе, молоко горело, как сухая трава\ldotst
Но я думала, что, даже если отдам дух Безумному, никто не будет на меня в обиде.

"--*Атрис мёртв?

"--*Да, Ликхмас-тари.
Атрис мёртв.
Вождь отказался отдать мне его, и я своими ушами слышала, как цитра моей жизни издаёт на алтаре свой последний аккорд.

Я промолчал.
Родильница нервно стучала по столу рукой, и я различил знакомый уже до боли барабанный язык:

<<Я стала w\o izh вождя.
Я взяла власть в племени.
А потом я убила его.
Медленно отравила, чтобы он познал всю мою боль>>.

<<Я подставил тебя, я знаю>>.

<<Выкручусь.
Бывало и похуже>>.

<<Вместе сподручнее>>.

Завуалированный вопрос остался без ответа.

"--*Ты узнал всё, что тебе нужно?

Я поднялся и рискнул ещё раз:

"--*У королевы много пчёл, но и слабейшая из них может найти ей уютный улей.

Митхэ рассмеялась странным хриплым смехом, похожим на сдерживаемые рыдания.

"--*Она пролетела путь от моря и до моря.
Она вкушала мёд цветов пустыни и ледяных первоцветов.
Налеталась она, Ликхмас-тари, не трогай её.

"--*Ты мне помогла, "--- сказал я, направляясь к двери.
"--- Надеюсь, что больше твои стрелы не обратятся против Тхитрона.
Если моя рука вновь встретит твоего питомца в бою "--- он умрёт.

"--*Это не мне решать, Ликхмас ар’Люм.
Haghp\o th!

Зашёл пожилой воин и поклонился.

"--*He gt, gh\o r thenis is litaa, "--- резко приказала родильница.
Воин посмотрел на неё с удивлением.

"--*Gerren si, bv\o h ni tras? "--- осторожно поинтересовался он.

"--*Eh Seli mitygh magh zhyu.
Berth p\o h ni haawr.
Gh\o r thenis is litaa.

Berth p\o h ni haawr.
<<Он искал своего дарителя>>.
В глазах хака это неплохая причина для столь дерзкого прибытия "--- они очень ревностно блюдут кровные узы.

Родильница сделала вошедшему левой рукой неизвестный мне знак.
Однако стук пальцев правой предназначался для меня: <<В лесу тебя будут ждать>>.
Мы с вошедшим кивнули почти одновременно.

Митхэ ценили за боевой опыт и коварство, но не все хака доверяли вождю-сели.

Я приподнял занавеску, пропуская воина вперёд, и вспомнил ещё кое-что.

"--*Скажи мне, Митхэ ар’Кахр.
Тот жрец\ldotst

"--*Он не встал с ложа, на котором насиловал меня, Ликхмас-тари.
Стены кельи покраснели от ярости моего клинка.
А рука, которой ты грозишь моему сыну, "--- женщина на мгновение светло улыбнулась, её глаза заблестели, "--- создана не для сабли.
Оставь ты это.

Я посмотрел на свои узкие длиннопалые ладони и, едва заметно улыбнувшись Митхэ, бросил взгляд на нахохлившегося возле костра Имжу.
Он взирал на меня исподлобья, враждебно, но с лёгкой тенью неуверенности.
Умён, ничего не скажешь.
Кровь.
Я коротко кивнул брату и вошёл в проливной дождь под ощетинившийся копьями конвой.

\section{[-] Чужая флейта}

\epigraph
{Месть кулаком за слово "--- уродливый обычай униженных.
Он много хуже первобытного <<око за око>>, ибо от крепкого и меткого словца, что бы ни говорили люди, ещё никто не умер;
боль, причинённая словом, идёт лишь от тайн сердца слушающего.
И король, издав Закон о преступлении словом, своими руками отдал этому обычаю оружие!
Прольются реки крови, господа, попомните мои слова.
Люди ничто не хранят так бережно, как остатки поруганного достоинства.}
{Клаудиу Семито Фризский.
<<Воспоминания о великом тёзке>>}

Ворота поселения остались позади, и я немного вздохнул.
Да, рядом по-прежнему шли несколько молодых воинов, но это было лучше, чем молча смотрящая на тебя толпа.
Разумеется, мне сильно повезло;
мне везёт до сих пор.
Сдай у кого-то из хака нервы "--- даже у одного "--- и я бы уже бодро бежал по жилам джунглей с бумажным фонариком.

Дарительница осталась в палатке, а вот Имжу вышел, чтобы проводить меня взглядом.
Молодой конвой напрягся.

"--*И что это он тебя не убил, крашеная морда, "--- пробормотал тот, которого звали Перт.
"--- Может, этот сели "--- твой брат?

"--*Что ты сказал, Перт? "--- с убийственной вежливостью осведомился Имжу, пронзив Перта двумя ярко-зелёными лезвиями.
Несмотря на то, что Имжу был безоружен, а Перт нёс аж два копья, желание прокомментировать свои слова у парня сразу отпало "--- он что-то злобно буркнул и ускорил шаг.

Впрочем, мои приключения ещё не закончились.
Едва ворота исчезли за изгибом тропы, рука одного из воинов совсем не по-дружески легла мне на плечо.
Я высвободился и выхватил фалангу, успев увернуться от метко брошенного ножа.
Мы замерли друг напротив друга с оружием в руках.

"--*Прекратите.

Грубый сорванный голос принадлежал пожилому воину, которого подозвала мать.
Хагпот, кажется.
Он стоял, закутавшись в плащ, на дороге, нарочито расслабленно опустив плечи.
Я тут же отошёл в сторону.

"--*Хагпот, "--- проворчал Перт.
"--- Мы должны уничтожить лазутчика.
Уйди.

"--*Кажется, ты слышал слова вождя из моих уст, Перт, "--- непринуждённо заметил Хагпот.
"--- Она велела отпустить мальчика, а не убивать его.

"--*Он мужчина!

"--*Посмотри на него, "--- Хагпот махнул на меня рукой.
"--- Он маленький, худой и безбородый.
Конечно, мальчик.
В убийстве ребёнка чести нет.

"--*Все сели безбородые и выглядят как женщины.
Этот одолел отряд Имжу!
И он расскажет сели о том, что видел и слышал!

"--*Ты мыслишь как жаждущий драки петух.
Старейшины совершили ошибку, но у них ещё есть возможность восстановить мир.
И кровь тех, кто пошёл за Имжу, будет платой.

"--*Какой ещё мир?
Какая плата?
Мы будем мстить!

"--*На наше счастье сели никогда не мстят сверх меры.
Закон гласит "--- одного за одного, не больше.
Мужчину за мужчину, женщину за женщину, взрослого за ребёнка.
Если они захотят уничтожить хака под корень, то сделают это за пару дней, и о наших деревнях до самых Молчащих лесов не останется даже воспоминаний.

"--*Пусть попробуют!

"--*Давай начистоту.
Ты сомневаешься во мне или в мудрости вождя?

"--*Мудрости старого дырявого котла, в который запускал ложку каждый встречный? "--- скривился Перт.
"--- Да ещё и сели?

"--*Ты сомневаешься в мудрости старейшин хака, которые выбрали её боевым вождём?
Эта женщина в молодости стоила десятерых таких, как ты, грязноротый недомерок.
А сейчас она стоит сотни.

"--*Говорят, что ты и твоя мёртвая потаскуха развлекались с ней втроём.
Скажи, так заведено у сели или тебе просто нравятся трущиеся друг о друга мохнатые шкурки?

"--*Мне нравится то, чего ты никогда не познаешь, "--- ответил пожилой воин.

"--*Уйди, старик, предупреждаю последний раз.
Я намерен убить сели, и мне неважно, кто ещё разделит его судьбу, "--- Перт угрожающе приподнял духовое ружьё.

Хагпот не двинулся с места.

"--*Я одолею всех вас и в честном бою, Перт.
Но сейчас не вижу причин сражаться честно, "--- спокойно проговорил он.
Из леса, как по сигналу, вышли ещё пять воинов с луками наготове.
"--- У меня прямой приказ вождя, избранного старейшинами, как и у тебя.
Может быть, старшие зря оказали тебе доверие?

"--*Можешь съесть доверие своего вождя и испражниться в её палатке.
Там ему самое место.

"--*Также закон хака гласит, что за неисполнение прямого приказа полагается наказание, за противодействие "--- смерть.
Ты можешь не ценить доверие, но если ты не уберёшься\ldotst

"--*Тебе воткнут стрелу в спину, гнилой огрызок, "--- ощерился Перт.

"--*Тебе уже будет всё равно, "--- парировал старый воин.
"--- А теперь хватит игр, малыш.
Я знаю, под чью флейту ты поёшь.
Его здесь нет, он сидит в тёплой хижине и греет руки у чужого костра, пока его младший нелюбимый сын стоит под прицелами луков.

\mulang{$0$}
{"--*Что ты сказал? "--- трясясь от гнева, пробулькал Перт.}
{``What did you say?'' Pert said constrained trembling in anger.}

\mulang{$0$}
{"--*Ничего нового и ничего того, что не было бы правдой.}
{``Nothing new, and nothing wrong.}
Так что перестань трясти тем, чего у тебя нет, иди в селение и молись кому хочешь, чтобы о твоих словах не узнали старейшины и Имжу.
За подобное оскорбление он вызовет тебя на kazha-l\o m, и не найдётся хака, который отведёт от тебя бесчестье или смерть.
Мои люди, "--- он окинул их многозначительным взглядом, "--- будут молчать.
До поры, разумеется.

Перт, кипя яростью, смотрел то на меня, то на закутавшегося в плащ Хагпота, потом окинул взглядом суровые лица окруживших его воинов.
Наконец он, сплюнув на землю, буркнул что-то товарищам и побрёл к деревне.

"--*Ke, yghd.
Bv\o h lei sert.

Воины поклонились Хагпоту и пошли вслед за удаляющимися юнцами.
Мы со стариком остались на дороге вдвоём.

\section{[-] Старый друг}

Хагпот подошёл ближе, буравя меня стальными выцветшими глазами.
Если насчёт Имжу я поначалу сомневался, здесь сомнений быть просто не могло "--- передо мной стоял не хака.
Очень сильно похож на ноа "--- такой же смуглый.
Впрочем, нет, это не смуглина и не та краска, что использовал Имжу.
Неужели просто копоть от костра?

"--*Так значит, тебя зовут Ликхмас, "--- шевельнулись тонкие верёвочные губы.

Я молчал.
От этого матёрого ягуара можно было ожидать чего угодно.
Он говорил на языке сели "--- с лёгким акцентом, но не коверкая слов.

"--*А тебя Хагпот?

Старый воин поднял бровь и посмотрел на меня с интересом.

"--*Ты наблюдателен, но тебя подвело незнание языка.
Haghp\o th на языке хака означает <<помощник вождя>>.
А зовут меня Акхсар ар’Лотр, я друг твоей родильницы.
Не бойся.

У меня отлегло от сердца.

"--*Откуда ты знаешь, что она моя родильница?

"--*Я стоял рядом с твоей колыбелью, когда Митхэ оставила тебя в Тхитроне.
Я один из тех, кто отправился с ней на поиски твоего дарителя.

"--*Я даже не знаю, был ли Атрис\ldotst

"--*\ldotst твоим дарителем? "--- Акхсар неожиданно улыбнулся щербатой улыбкой.
У него не хватало того же зуба, что и у Митхэ.

"--*Брось, парень.
Мои глаза не так остры, как раньше, но и ими я вижу, кто твой даритель.
И родильница твоя не слепая.

"--*Ты хорошо знал его?

"--*Хай, "--- вздохнул Акхсар.
"--- Старина Хат прошёл с нами полмира.
Он был смелым и хорошим другом.
Он не знал, с какого конца браться за саблю, но цитра в его руках плакала и смеялась.
Давай-ка пройдёмся немного.

Акхсар, как видно, соскучился по родной речи.
Акцент пропал уже на тридцатом шаге.
Прошли мы с ним около кхене, но он успел рассказать много интересного\ldotst

"--*Помню, был бой на перевалах у Крысиного Лаза.
Мы шли уставшие, а старину Хата сильно ранили в ногу.
Он попросил, чтобы мы посадили его на оленью упряжку и дали ему цитру.
Он заиграл\ldotst и мы почувствовали прилив сил.
Мы шли и смотрели друг на друга, не понимая, что происходит.
Я никогда раньше такого не чувствовал "--- равнодушен я к музыке, знаешь ли.
А он сумел что-то задеть в моём сердце.
Жаль, что ты его не знал\ldotst

"--*\ldotst Моя последняя женщина "--- тоже воин.
Как-то на пиру она пыталась рассказать мне о своих чувствах, но это у неё выходило плохо.
Хат увидел, что-то заиграл\ldotst
И его музыка ударила нам в голову, как хорошее вино.
Я даже не знал, что так может быть.
Всё, что я помню после "--- в ту ночь мы с Обжоркой как с цепи сорвались.
Кстати, она вон там лежит, на обочине, десять дождей уж как.
Лихорадка съела.

"--*Скажи, Акхсар-лехэ, вы с Обжоркой\ldotst твоей женщиной пошли за Митхэ по закону чести?

"--*По закону дружбы, малыш.
Да, её звали Эрхэ ар’Люм, или Слишком-Много-Ест, "--- Акхсар произнёс это имя, словно отдал сокровище, спрятанное до поры у него во рту.
"--- Кажется, она дальняя родственница твоей кормилицы.
И пожалуйста, не называй меня <<лехэ>>, не так уж я и стар\ldotst

"--*Ты и Кхотлам знаешь?

"--*Д-да, конечно, "--- смешался Акхсар.
"--- С чего бы мне её не знать.

"--*Имжу "--- твой хранитель?
У хака и сели не может быть детей.

"--*Да, малыш.
Но воинам и самому Имжу это знать не обязательно.

"--*Почему хака напали на Тхитрон?
Мы же собирались им помочь.

Акхсар смутился ещё больше.

"--*У нас и без Разлома выдался плохой год.
Зверьё из-за ушло на запад, в джунгли Живодёров.
Многие погибли, дети голодают, но хуже всего "--- не хватает людей для жертвоприношений.
Старейшины говаривали, что если мы будем приносить жертвы детьми, как раньше "--- многие племена просто вымрут.
Был вариант организовать общее многоплеменное становище, но ты же знаешь "--- разногласия, старые обиды.
Поэтому Инхас-Лака решились на поход.
Твоя родильница предлагала поход против идолов, но кое-кто, "--- глаза Акхсара зажглись недобрым огнём, "--- кое-кто надоумил старейшин, что сели "--- добыча куда легче\ldotst
Мы с ней к этому отношения не имеем, "--- поспешил добавить он.
"--- И на брата не сердись.
Он был против, но просто не мог оставить молодых воинов на произвол судьбы.
Без него отряд даже не дошёл бы до земель сели.

"--*Кто это был? "--- спросил я.
"--- Кто подговорил старейшин?

"--*Он не станет вождём, пока я жив, "--- уклончиво ответил воин и сплюнул на землю.
"--- Если он попытается спровоцировать Митхэ ещё грубее, то ему придётся отвечать не перед старейшинами, а перед моим клинком.
И неважно, что будет потом.
Кстати, это Митхэ надоумила старейшин вставить в последнее сообщение очевидный обман.
Кхотлам, как обычно, великолепна, сразу поняла\ldotst

Акхсар вдруг запнулся и сменил тему.

Вскоре становились мы у малоприметного дорожного знака.
Акхсар свистнул по-птичьи, невдалеке кто-то отозвался мяуканьем.

"--*Расстанемся здесь, Ликхмас-тари.
Вы с Имжу шли по совершенно непригодной тропе, чересчур близко к Гибельным руинам, и не следует испытывать судьбу ещё раз.
Иди направо, через десять кхене поверни налево и дальше через храмовую рощу.
Благородный баньян, наверное, знаешь такое дерево.
У тотема Сата-скитальца найдёшь тропинку, она ведёт на запад.
Никто пока не вышел из деревни, так что, думаю, пара кхене форы у тебя есть.
И да.
Я очень надеюсь, что ты осознал всю глубину того, насколько тебе повезло.
Больше так не делай.

Я кивнул.

"--*Скажи, почему ты не вернулся к сели?

"--*Видишь ли, твоя родильница "--- кутрап, но у неё есть друзья и замечательный хранитель, к которому она может вернуться, "--- Акхсар усмехнулся, бросив на меня добрый взгляд.
"--- У меня всё ровно наоборот "--- перед законами сели я чист, а вот идти\ldotst идти мне некуда.
Саблей, "--- он откинул плащ левой рукой, обнажив уродливую культю, "--- на жизнь я больше не заработаю, и ничем не заработаю из того, что требует две умелых руки.

Я кивнул.

"--*Беги ланью, парень.
И да, кажется, ты кое-что забыл.

Акхсар протянул мне полуистлевший клочок сыромятной кожи.
Я принял его.
Старый воин прощально кивнул и зашагал сниз по дороге.
Посмотрев вслед удаляющейся тяжёлой мужской фигуре, я развернул кожу.
На стёртой, с редкими ворсинками изнанке кожи стоял свежий иероглиф, от которого на душе потеплело:

<<Я счастлива>>.

\chapter*{Интерлюдия VIII. Птицы и Мастера}
\addcontentsline{toc}{chapter}{Интерлюдия VIII. Птицы и Мастера}

\textbf{Сказка сели в пересказе Сиртху-лехэ}\footnote
{Старик рассказывал сказки с неподражаемым хуторским колоритом.
Так, как он, не рассказывал никто, я ручаюсь.
По многочисленным просьбам я попытался сделать максимально точный перевод. \authornote}

Когда змеями рубахи подвязывали, когда на свиньях спали и сны видели, когда курами полы подметали, пришла из земель тенку\footnote
{Почти везде в землях сели и ноа повесть начинается именно так.
Тем не менее историки склоняются к мысли, что сочинена повесть была не тенку, а гораздо ранее "--- первыми переселенцами с Лотоса, рассказывавшими о периоде Державы Рабов, или Первого Народовластия (600--920 по Вилет'март, 1300--2871 Эпохи Богов). \authornote}
эта история, да ножек у порожка не отряхнула.
Правда ли, выдумка "--- кто теперь разберёт?

Говорят, что когда-то Мастеров в клетках держали, как маленьких птичек или сверчков.
Только вместо чудного чириканья или стрекотания Мастера ткани делали, металлы выплавляли и камень резали.
Цепями да путами держали Мастеров.
Тех, кто был привержен дому, привязывали к старым и немощным предкам.
Тех, кто слаб был перед велением плоти, привязывали к мужчинам и женщинам.
Тех, кто продолжения своего рода желал, привязывали к многочисленному потомству.
Связывали Мастеров законами, правилами их неволили, саблями острыми угрожали.
А самых свободолюбивых сажали в настоящие клетки, навроде птичьих, и сковывали настоящими цепями, пока не смирялись они перед мощью последней клетки "--- собственного тела.

Не пристало Мастерам летать, ибо чем больше руки похожи на крылья, тем меньше в них мастерства.
Так Мастерам их Хозяева говорили, свои животы потирали, а чужие уши в воде полоскали.
Что думали Хозяева тёмными ночами и о чём меж собой говорили "--- неизвестно, завесь полотенцем клетку "--- слеп и глух попугайчик.

Перелётные птицы жалели Мастеров, ведь были они с Мастерами одной крови.
Они прилетали да зерно клевали, которое Мастера оставляли на пороге, а Мастера слушали рассказы Птиц о дальних странах, вдохновлялись песнями-сказками, да так, что ткани краше, камень глаже да металлы звонче становились.
На свободе мало еды, говорят.
В клетке нет вдохновения, сказывают.

Мастера и Птицы понимали друг друга.
Птицы знали, что нет Мастерам дороги из клеток, и не звали их с собой.
Мастера же знали, что нельзя сажать Птиц в клетки, и потому, покормив, отпускали их.
Так жили Мастера и Птицы.

Но однажды появился Глупец.
Так его Хозяева назвали, и ни имени домашнего не помянули, ни по дому-племени\footnote
{По имени цатрон. \authornote}
не обратились.
Он сказал, что не должно быть клеток, и тело "--- не клетка, и нос "--- не задвижка, и шея "--- не цепь.
Глупец ходил и о том рассказывал, как сладко на свободе жить.
Кто-то из молодых Мастеров ему верил, те же, что постарше, знали, что заблуждается Глупец "--- ведь они дружбу с самими Птицами водили, а Птицы и мир видали, и из морей воду бочками хлебали.
Только настаивал Глупец на своём и сказал наконец, что недостойны друг друга Мастера и их друзья крылатые.

Сказал и сказал, горлом воздух помял, потоптал "--- кому какая разница?
Но тяжёлым оказалось слово Глупца.
Хозяева заявили, что это была правда, что сказал Глупец чистую правду раз в жизни, и стали её Мастерам день ото дня твердить, животы потирать, чужие уши в воде полоскать.
И Мастера поверили Хозяевам, а Птицы поверили Мастерам, и прервалась старинная дружба.

С тех пор Птицы умирать стали, и тела их на дорогах лежали, поедаемые кошками и крысами.
Молили-закармливали Птицы духов, чтобы дожить до весны "--- и доживали, и умирали зелёными вёснами.
А коли сам не ведаешь, чего просишь, какой с духов спрос?
Мастера же вдохновение растеряли и ремесленниками стали, да так, что ткани грубы, и камни шершавы, и металлы петь перестали.
На свободе мало еды, говорят.
В клетке нет вдохновения, сказывают.

И с тех пор лишь один раз в жизни ремесленник может доказать, что достоин дружбы с Птицами.
Приходит к нему Заказчик.
Говорили люди, что Заказчика посылает Глупец, что жив он, дышит и землю топчет.
Понял Глупец, что сотворил, и хочет всё исправить, молоко разлитое тряпкой собрать да кашу из того молока сварить.
Заказ же всегда прост и труден, и порой не отличишь его от обычной работы.
Те же из ремесленников, что дело своё любят да снова стать Мастерами хотят, всю жизнь ткани делают, металлы выплавляют да камень режут, словно в последний раз, вглядываясь, пропустить боясь "--- не пришёл ли тот самый Заказ, не вспомнили ли о давнем друге Птицы, не выбрал ли его Глупец среди прочих?

Да так и ждут-пождут, да без меня, я стороною проходил, ушами далёкие речи ловил.
Что наловил, то и в котелке сварил, что в котелке сварил, тем и дитя кормлю.

\chapter{Хрупкий мир}

\section{Переговоры с хака}

Наверное, меня никогда и нигде не встречали так.

Ликхэ, Ситрис и Кхохо кинулись мне на шею и долго не отпускали.
Чханэ просто подошла, спотыкаясь, нервно рассмеялась, потрепала меня дрожащими пальцами за рукав и села под ближайшим деревом, в уголке деревенской площади.
Я сел рядом с ней.

"--*Живой?

"--*Живой.

"--*Точно?

"--*Точно.

"--*Как чудесен мир, "--- лаконично ответила подруга.
Вся её разговорчивость куда-то пропала.
День выдался погожий;
пели птицы, жужжали стрекозы, шуршали в кустах ящерицы, светило яркое солнце.
Мы так и просидели под деревом в обнимку несколько часов, ловя на себе частью удивлённые, частью неприязненные взгляды хака;
едва ли прибывшая делегация ожидала увидеть такую романтику на переговорах.

Кхотлам и Митхэ обменялись крепким рукопожатием.
Большего не позволяла ситуация.

"--*Рада, что ты живая, Митхэ ар'Кахр, "--- сказала на цатроне кормилица.

"--*Я благодарю тебя за всё, что ты сделала, Кхотлам ар'Люм, "--- ответила Митхэ на том же языке.

"--*Жрецы выбрали \emph{совсем нейтральную территорию}, "--- шёпотом сообщила мне Чханэ, обведя рукой полузаброшенный хутор.
"--- Всё-таки Митхэ "--- кутрап, и мы обязаны её казнить, если она свернёт с трёх главных дорог.
Жрецы решили не рисковать.
Мало ли что придёт в голову этим хака\ldotst

Я кивнул.
Жрецы всё сделали правильно.

Постепенно к нашему дереву подошли все воины и сели рядом.
Они молчали, и я чувствовал в их молчании какую-то недосказанность.

"--*Всё хорошо? "--- шёпотом спросил я наконец.
Мне не ответили.

Я вдруг остро ощутил, что чего-то "--- вернее, кого-то "--- не хватает.

"--*Где Хитрам?

Ликхэ обернулась, и я увидел, что она плачет.

"--*Стрелка вклинилась между пластинами доспеха, "--- прошептала девушка.
"--- Она рукой дёрнула и напоролась.
Надо же, какая случайность\ldotst

"--*Вы похоронили её?

"--*Нет, "--- прошептал Эрликх.
По его лицу тоже текли слёзы.
"--- Это произошло в старом лесу, земля там жилистая.
Оставили её под большим папоротником-опахало.
Достойное место отдыха для достойного воина.

"--*Мы даже хэситра ей налили на донышке, "--- подхватил Ситрис.
"--- Воду для детей оставили, на обратный путь.
Надеюсь, она не сердится.

"--*Она бы поступила так же, "--- уверенно сказала Чханэ.
"--- Она добрая женщина.

Митхэ не участвовала в переговорах.
Она выглядела так, словно её привязали голой к позорному столбу, которые использовали для наказания преступников тенку.
Все без исключения тхитронцы восхищённо поедали Митхэ глазами "--- кто-то украдкой, как Кхохо, кто-то спокойно, как Ситрис, а наблюдатели из числа ткачей и вовсе пялились, открыв рот.
Я тоже изредка бросал на неё взгляды.
Мне часто говорили, что я похож на родильницу, и я думал, что речь о Кхотлам.
Но нет.
Теперь, когда они стояли почти рядом, я увидел разницу.
Я был копией Митхэ.
Я словно смотрел в зеркало Хри\footnote
{Согласно преданию, у Хри-соблазнителя было зеркало, в котором мужчины превращались в женщин и наоборот. \authornote}
"--- те же длинные брови, тот же точёный нос, тот же крупный, чувственный, даже немного жадный рот.
Тот же маленький рост, над которым не упускал случая подшутить Конфетка.
Различались только волосы "--- жёсткие чёрные кудри Митхэ совершенно не были похожи на мои мягкие светлые локоны.
Я схватил один локон и критически осмотрел;
да, наверное, волосы я унаследовал от Атриса.

И только кормилица, чисто выговаривая слова цатрона, украдкой пыталась поймать взгляд этого странного однорукого старика Акхсара, который скромно стоял в углу.
На глазах Акхсара лежала непроницаемая тень, и я не видел, на кого смотрит он.
Он даже не взглянул на Ситриса, который подошёл к нему перед переговорами.

"--*Где Эрхэ? "--- тихо спросил Ситрис.

"--*Умерла, "--- одними губами ответил Акхсар.

"--*Сочувствую, "--- пробормотал Ситрис и отошёл.
Всё оставшееся время воины делали вид, что не знают о существовании друг друга.

Кхарас стоял чуть поодаль и смотрел в пустоту.
Я вдруг понял, что он тяжелее всех переживал смерть Хитрам.
Сколько она была в его Храме?
Кто ещё будет нежно называть его, грубого седого силача, <<мальчиком>>?
Я подошёл к вождю и положил руку ему на плечо.
Кхарас кивнул без улыбки.

"--*Ты как?

"--*Мне нужен путник, плававший по дорогам рыб\footnote
{Кхарас хотел получить данные топографии поселений Инхас-Лака, которые мне удалось добыть. \authornote},
"--- сухо сказал вождь.

"--*Всё будет, "--- кивнул я.

Уже под конец переговоров Кхарас сказал: <<Мы ведь могли взять её с собой, за городом папоротники пышнее>>.
Произнёс он это тихо, всеобщее внимание было поглощено беседой Кхотлам и двух старейшин.
Разумеется, никто не обратил внимания на его бормотание.

\razd

Наконец переговоры подошли к концу.

"--*Я думаю, нам стоит устроить пиршество, "--- улыбнулся один из старейшин хака.

"--*Я думаю, нам стоит уважать традиции друг друга, "--- веско сказал Акхсар из угла.
"--- Традиции хака "--- пиршество.
Традиции сели "--- Молчание.
Пусть каждый из народов последует своим традициям.

Хака нахмурились;
кое-кто зароптал.

"--*Кхотлам ар'Люм, а что думаешь ты? "--- обратился старейшина к кормилице.
"--- Хагпот Инхас-Лака мудр, но это решение\ldotst

"--* \dots это решение я поддерживаю целиком и полностью, "--- закончила Кхотлам.
"--- Надеюсь, старейшины смогут объяснить молодым его целесообразность.

"--*Это решение обижает нас, Кхотлам ар'Люм.
Мы\ldotst

"--*Похоже, между нами некоторое недопонимание, "--- Кхотлам сузила глаза и понизила голос "--- ровно настолько, чтобы фразу услышали все, но тем не менее сочли приватной.
"--- Старейшины хака решили предать алтарю наших сыновей и дочерей.
Мы не разделим со старейшинами ни тарелку бобов, ни кукурузную лепёшку, пока не сочтём их кровь чистой.

Хака вздохнули и забормотали;
глаза старейшины налились кровью.

"--*Странников и торговцев из числа вашего народа мы примем с теплотой, как и ранее, "--- чуть громче продолжила Кхотлам, обращаясь к конвою и сопровождающим.
"--- А теперь, я думаю, самое время уважительно попрощаться и разойтись по домам.

<<Умно, "--- подумал я.
"--- Унижение на глазах народа станет старейшинам уроком.
В следующий раз хака будут ценить не только дружбу, но и элементарную дипломатическую вежливость>>.

То, о чём думала Митхэ, так и осталось тайной.

\section{Нарушенное Молчание}

\epigraph
{Прославлять себя победой "--- это значит радоваться убийству людей.
Победу следует отмечать похоронной процессией.}
{<<Книга Пути и Достоинства>>, Мудрый Старец.
Культура Цина, Древняя Земля}

<<Неужели я вернулась домой?>> "--- думала Митхэ, с некоторым удивлением рассматривая растущие башни Тхартхаахитра.
У неё было смутное ощущение, что что-то происходит в последний раз, и по воинской привычке списала это на предвестье скорой смерти.

\mulang{$0$}
{"--*Удивляешься, что вернулась? "--- тихо усмехнулся ей на ухо Атрис.}
{``Are you surprised you're back?'' \"{A}\={a}tr\v{\i}s whispered in her ear.}
\mulang{$0$}
{"--- Я тоже.}
{``Me too.}
\mulang{$0$}
{Всегда.}
{Always.''}

"--*Так, "--- Митхэ вырвалась из плена собственных размышлений, "--- ты почему пешком идёшь?
Ну-ка быстро в повозку.

"--*Я прекрасно себя чувствую, "--- заверил её менестрель.
"--- Нога больше не болит.

\mulang{$0$}
{"--*Как знаешь, "--- меньше всего на свете Митхэ хотелось спорить с Атрисом.}
{``It's on you.'' The last thing in the world M\={\i}tcho\^{e} wanted to do is to argue with \"{A}\={a}tr\v{\i}s.}

Все знали "--- по прибытии домой им впервые придётся нарушить древнюю традицию Молчания после войны.
Вернувшихся воинов всегда хорошо кормили и на целую декаду, а порой и на две освобождали от обычной работы, чтобы они могли хотя бы выспаться вволю.
Также в течение декады отменялись все праздники, даже Слёзы Ситхэ;
сели знали, что если была война "--- значит, кто-то лишился самого дорогого.
Однако вскоре в город должны были прибыть вожди хака, и в знак дружбы следовало устроить пир "--- заставлять гостей ждать было бы невежливо.
Молчание должно было быть нарушено, и одна эта мысль наполняла Митхэ дурными предчувствиями.
Она видела пышные триумфы военачальников тенку, которыми те праздновали свои крошечные жалкие победы на поле боя.

Среди старых воинов бродил тот же безрадостный толк.
Аурвелий, несмотря на ослепительно белые волосы, казался чернее ночи.

"--*Если кваждый победа праздновать, война никвогда не кончиться! "--- буркнул он как будто между прочим, проходя мимо Митхэ.

Ноа славились воинственностью;
послевоенные триумфы если и не были у них традицией, то считались по крайней мере оправданной вольностью народа.
Услышать такие слова от старого пирата-ноа Митхэ ожидала меньше всего.

"--*Мы празднуем не победу, а заключение мира, сенвиор Амвросий, "--- веско заметил Атрис.
"--- Это плохая отговорка, но давление обстоятельств в отговорках и не нуждается.
Я думаю, тебе стоит улыбнуться "--- теперь ты вступишь в полные права на землях сели.

Старик хмыкнул одновременно с кем-то, лежащим в повозке.

"--*Ситрис, кинь что-нибудь пожевать, "--- попросила Согхо.

Из повозки вылетел круглый кахраханский хлеб и скрылся где-то в рядах наёмников.
За хлебом последовал помидор, весело блеснув на солнце полированным алым боком.

"--*Вот лентяй, "--- буркнул Акхсар.
"--- Ситрис, ты уже весь день лежишь и жуёшь.
Не надоело?

"--*Это надоесть не может, "--- сообщила повозка.
"--- А вот отряды из южных Храмов меня уже нервируют.

"--*Жуй и не бойся, "--- успокоил разбойника Атрис.
"--- С нами тебя никто не тронет.

На этот раз хмыкнула половина отряда.

"--*Атрис, дружище, я бы пошёл с тобой на край света, "--- проворчал Ситрис, глубже зарываясь в тюки.
"--- А с этими сверчками безногими даже нужду справлять небезопасно.

"--*Я бы на твоём месте нужду справляла где-нибудь вне зоны комфорта, "--- хмуро сказала Эрхэ.

Все, включая Ситриса, захохотали.
Эрхэ ловко увернулась от брошенного помидора и тут же начала шарить в своём мешке, подыскивая подходящий снаряд;
скорость, с которой Ситрис построил из тюков крепость, внушала уважение.

"--*Бум-буррубум-бам-бум\footnote
{<<Поднять щиты, закрыть бойницы>> (барабанный язык сели). \authornote},
"--- прогудел разбойник, подражая осадному барабану.

"--*Придурок, "--- пробормотала Эрхэ.
"--- Твоё счастье, что промазал.
Чего ржёте? "--- раздражённо добавила она, глядя на хохочущих наёмников.

"--*Лезть повозка тоже, Атриций, "--- посоветовал Аурвелий менестрелю.
"--- Ты больно, я ввидеть.

"--*Не волнуйся за меня, кормилец.
Мы почти дошли.

\section{Камень и огонь}

Родные ворота.
Я стоял и гладил их, с каким-то неведомым ранее чувством ощущая под пальцами тепло дерева, трещины и резьбу.

"--*Ликхмас, ты чего? "--- удивился Ситрис.

Я не ответил.

Кормилица, проходя мимо, молча погладила меня по голове.
Чханэ приложила ладонь к воротам рядом с моей и подержала несколько секхар, затем пошла дальше.
На дереве остался едва заметный влажный отпечаток;
ветер унёс его мгновение спустя.

"--*Лисичка, ты проголодался, наверное, "--- прозрачно намекнула Ликхэ, кивнув в сторону храма.

Я снова промолчал.

Город не изменился, но я его не узнавал.
Люди приветствовали меня, как обычно;
мало кто знал о том, что я успел пропасть и вернуться обратно живым.
Старушка на улице Дышащего Дерева кивнула мне, словно старому знакомому;
трубка в её зубах дымилась, как и раньше.

"--*Ну что там, Ликхмас? "--- подбежала ко мне молодая крестьянка.
"--- С хака заключили мир?
Воины какие-то молчуны нынче.

Услышав разговор, нас окружили ещё несколько человек с вопросами.

"--*Хай, я это\ldotst
В целом всё хорошо, подробности сообщит Двор Люм в ближайшее время, "--- сказал я, высвобождаясь из цепких рук.
"--- Сегодня можете спать спокойно.

Вот и храм.
Центр, убежище и дом.

Я вдруг понял, что на площади царила странная, неприятная суета.
Торговый день уже давно закончился, однако вокруг тихо сновали люди, таща какие-то вещи.
Они сновали вокруг, нервно глядя на меня.
Их глаза вопрошали: <<Что ты здесь делаешь?>>

<<Что ты здесь делаешь, синяя роба?>>

<<Разве тебе не следует быть в другом месте?>>

Я перешёл на бег.
В голове пронзительно звенел колокольчик тревоги.
Первая ступенька, вторая, третья, тринадцатая\ldotst вот и вход.

Меня встретил пустой чёрный зёв храмовых ворот.
Внутри не горел ни один факел.

Трукхвал сидел у входа и рыдал.
Я подбежал к нему.

"--*Учитель! Что\ldotst

Тут мой взгляд выхватил из темноты странный туман.
Нет, это был не туман.
Воздушная пыль пахла камнем и огнём.

"--*Ликхмас, "--- из темноты выпрыгнул Ситрис.
"--- Оставь Трукхвала, я позвал людей, о нём позаботятся.
Быстро на дежурство наверх.

"--*Диверсия, Ликхмас, "--- раздался голос Кхохо.
"--- Подробности потом.
Инструменты уже готовы, Эрликх тебе будет ассистировать.
Иди по западной лестнице, восточную завалило.

Вскоре я разглядел лежащие на полу изувеченные тела, наспех прикрытые одеялами.
Одно, два, три\ldotst тринадцать.
Вот рука Кхатрима со знакомой татуировкой.
В предсмертной муке она сжала амулет Сана, да так и окоченела.

Я медленно пошёл наверх, не веря в происходящее.

"--*Лис, беги, я сказал! "--- рявкнул Ситрис.
"--- Горевать будешь утром, жрец!

"--*Я сейчас\ldotst "--- услышал я сдавленный всхлип Трукхвала.
"--- Пусти меня, Уголёк\ldotst я тоже наверх\ldotst

"--*Сиди, Звоночек, "--- Кхохо явно пришлось приложить силу, чтобы усадить учителя обратно.
"--- Парень справится.
Ты хорошо его воспитал.

\section{Стили боя}

"--*Боевое искусство должно воспитывать устойчивость к боли и страху, потому что боль и страх являются неотъемлемой частью боя.
В этом принципиальное отличие боевого искусства от любого другого.
Это последнее, на что я хочу обратить ваше внимание.

Тси слушали Баночку, открыв рты.
Культура взаимного уничтожения была для нас новой и непривычной.
Кое-кто "--- например, Листик и Мак "--- напрочь отказались её изучать.
<<Это неправильно>>, "--- отрезали они хором и ушли по своим делам.

"--*Итак, "--- Баночка поднял руки.
"--- Тогда я прошу всех встать в пары.
Старайтесь подбирать пары по росту и комплекции;
вы поймёте, почему это важно.
Первое занятие "--- болевая подготовка.

"--*То есть ты хочешь, чтобы мы\ldotst били друг друга? "--- ошеломлённо спросил Фонтанчик.

"--*Именно.

"--*Баночка, дружище ты мой, ну я же не это имел в виду, когда говорил про технику боя\ldotst

"--*Я проанализировал традиции племён джунглей, подключил кое-какую древнюю историю и пришёл к выводу, что это неотъемлемый этап подготовки, "--- заявил Баночка.
"--- И да, встань в пару с сестрой, пожалуйста, иначе ты кого-нибудь прихлопнешь.

"--*Пусть встанет с Зубиком, "--- буркнула Пирожок.
"--- Мне это не нравится, и я этим заниматься не собираюсь.

Впрочем, Зубик тоже не пожелал продолжать обучение.
Вместе с Пирожок с тренировочной площадки ушла ещё четверть тси.

Однако Баночку это не обескуражило.
Его глаза горели;
того, кто увлёкся делом до такой степени, сложно в чём-либо убедить.

"--*Неужели вы не видите в этом красоты? "--- спрашивал он у нас с Заяц.
"--- Пожалуйста, забудьте хоть на секунду, что это методики взаимного уничтожения, и посмотрите на внешнюю форму!

"--*Мне об этом сложно забыть, "--- призналась Заяц.

"--*Разве хищные звери не прекрасны, хоть их тела "--- орудие уничтожения? "--- допытывался Баночка.

"--*Это другое, "--- отрезала Заяц.
"--- Не мешай прайдацию\footnote
{Прайдация (предация) "--- вынужденное (у сапиентов) или инстинктивное (у прочих животных) убийство ради пропитания. \authornote}
с этим непотребством.
То, чем ты увлёкся "--- самая тёмная сторона внутривидовой конкуренции.

"--*Ты считаешь внутривидовую конкуренцию чем-то плохим?

"--*Я считаю, что в ней нет нужды, если большинство представителей вида способны планировать хотя бы на два шага вперёд!

"--*Подумай ещё, "--- кротко попросил Баночка.
"--- Попробуй увидеть то, что для меня очевидно.

"--*Объясни, почему для тебя так важно, чтобы твоё занятие нравилось и мне, и Заяц, и всем остальным тси? "--- спросил я.

"--*Ну, вы же мои др\ldotst

Баночка вдруг смешался и задумался на целую минуту.

"--*Тоже верно, "--- признал он наконец, по очереди обнял нас с Заяц и убежал.

Больше мы не услышали от Баночки ни одной лекции.
Но новости о его деле разлетались быстро;
кто-то даже запустил несколько парящих видеокамер и наладил на форуме видеотрансляцию с тренировочной площадки.
Тси разделились;
большинство открыто осуждало происходящее, но многие, выкроив час времени, записались на тренировки.

Баночка разработал сразу несколько парадигм (или, как он выразился, <<стилей>>) боя.
Отныне каждый день два часа он уделял тренировкам прочих, сам же тренировался всё свободное время, выбирая всё более сложную местность, погодные условия и расположение противников.
Он поднял на одном из серверов <<кукловода>> и обучил несколько кукольных нейросетей различным стилям боя.
Все признавали, что любая из этих кукол превосходила по мастерству воинов царрокх "--- и тем не менее Баночка их побеждал.
Прибывшая с материка Кошка открыла рот, когда увидела, как плант за какие-то несколько секунд расправился с двадцатью виртуальными врагами.
Два коротких трёхчленных копья, названных им <<фаланги>>, перерубали длинные копья кукол и легко вонзались в жизненно важные органы "--- ровно настолько, чтобы враг уже не встал.

"--*Ужасно, "--- пробормотала она.
"--- Баночка со своими палками может подчинить всех царрокх в одиночку.

Однако вне тренировок плант продолжал носить резонанс-саблю.
Сабля лежала рядом, когда он ел, что-то чинил, сабля висела у него на спине, когда он отправлялся на берег моря или в горы "--- узкоспециализированный тяжёлый инструмент, превратившийся в фетиш.
Я пытался поговорить с ним на эту тему, но плант демонстративно поворачивался и уходил, едва речь касалась его оружия.

\section{Последний знак}

Вечером Атрису стало хуже.
Храмовый жрец осмотрел рану и вызвал носилки.

Цепочку дурных знамений завершила сабля Митхэ, Перидотовая Лиана "--- в тот же вечер она треснула почти вдоль безо всякой видимой причины, не дожив, если верить пробе, всего один дождь до своего трёхсотлетия.
Митхэ, стараясь не поднимать шума, отнесла осколки одному из оружейников Тхартхаахитра.
Несмотря на молодость "--- мужчине едва перевалило за сорок дождей "--- он уже был старшиной квартала и считался одним из лучших мастеров по кукхватру на Юге.

"--*Это бывает, "--- успокоил её мужчина, осмотрев клинок.
"--- Кое-кто считает, что оружие трескается из-за гнева убитых им, но причина не в этом.
Многое зависит от формы сабли и условий её ковки.
Если выбрать неправильную форму, в кукхватре начинает накапливаться напряжение.
Вот, смотри.

Оружейник снял саблю с подставки на стене и принёс Митхэ.
Затем ударил по сабле крохотным молоточком "--- и она вдруг застонала, словно воткнувшаяся в дерево стрела: <<Уиуиуиуи\ldotst>>

"--*Это, конечно, совсем запущенный случай, чаще всего услышать могу только я да пара-тройка мастеров с Запада.
Копится напряжение и десять, и пятьдесят дождей, пока не разрешится трещиной.
Его можно снять, если вовремя заметить.
Но если трескается "--- только перековка.
Поэтому многие предпочитают ковать короткие клинки, ножи и фаланги, "--- с ними такого не бывает.

"--*Сделаешь в точности такую же?

"--*Дело сложное, но возможное.
Только я бы предложил кое-что получше, "--- оружейник легонько ударил молоточком по осколку, и он отозвался протяжным чистым звоном.
"--- Я немного изменю форму, чтобы напряжение накапливалось медленнее.
Вполне возможно, что на твоём веку <<разгрузка>> клинка и не понадобится, поэтому я вставлю в эфес алюминиевую табличку с указаниями для следующих владельцев.
Алюминий мягкий, и мастер сможет набить дату последней <<разгрузки>>, не причиняя клинку вреда.
Очень удобно, мастера Запада так делают уже давно.

\mulang{$0$}
{"--*Я согласна.}
{``Agreed.}
\mulang{$0$}
{Сколько?}
{How much?''}

\mulang{$0$}
{Мужчина назвал цену.}
{The man gave her a price.}
\mulang{$0$}
{Митхэ вытаращила глаза.}
{M\={\i}tcho\^{e} looked at him, wide-eyed.}

\mulang{$0$}
{"--*Ты серьёзно?}
{``Are you serious?}
\mulang{$0$}
{У меня нет столько!}
{I don't have that kind!''}

\mulang{$0$}
{"--*Нет? "--- удивился оружейник.}
{``Do you?'' the smith looked surprised.}

\mulang{$0$}
{"--*Нет и не было никогда!}
{``Now or ever!''}

\mulang{$0$}
{"--*Сколько есть?}
{``How much do you have?''}

Митхэ вывалила из дорожного мешка все сбережения.

"--*Здесь едва хватает на расходники, "--- извиняющимся тоном сказал оружейник.
"--- Не пойми неправильно.
Ковка кукхватра "--- недешёвое удовольствие, даже для меня, хоть мне по старшинству в квартале многое даром достаётся.

Митхэ молчала и смотрела на осколки.
Мужчина вздохнул.

"--*Я знаю, кто ты и откуда вернулась, "--- глухим тоном сказал он.
"--- У ремесленников есть поверье о заказе Глупца.
В жизни бывает один заказ\ldotst

"--*Я слышала эту сказочку.

"--*Глупость, да, "--- усмехнулся мужчина.
"--- Однако ж\ldotst

Оружейник вздохнул ещё раз и пошевелил рукой осколки.

"--*Дай мне хоть на еду, "--- попросил он.
"--- Не могу же я целую декаду ковать себе в убыток.

Митхэ вытряхнула из карманов несколько щепоток золота, вытащила из-под секхвим толстое перо шипастого страуса "--- тайник на самый крайний случай, "--- опорожнила его и положила рядом;
наконец, сняла с руки тяжёлое золотое кольцо Кхара-защитника "--- подарок Короля-жреца.
Оружейник принял кольцо, осмотрел его и грустно чему-то улыбнулся.

"--*Ты чего, Золото? "--- спросила Эрхэ за ужином на постоялом дворе.
"--- Хватит жевать вяленое мясо, мы не в походе!

"--*Я не хочу есть.

Эрхэ вдруг подняла руку подруги и задумчиво потёрла средний палец, на котором ещё синел след от кольца.
Митхэ молча выдернула руку.
Эрхэ вздохнула и тихо подозвала трактирщика:

"--*То же самое ещё по тарелке, запиши на меня.
И вина побольше.

\section{Жертвоприношение}

Ночь для жреца "--- жаркое время, во всех смыслах этого слова.
В кельях, надо сказать, душновато, несмотря на все ухищрения зодчих.
Кое-кто из жрецов в такой сезон перебирался вниз, к воинам.
Кое-кто стелил ложе на крыше или в крипте;
мстительные тени, по их словам, мучают только первые две-три ночи.
Если же не келья, то ночной вызов "--- закровила роженица, залихорадило ребёнка, принесли из сельвы укушенного охотника;
если не ночной вызов, то жертва.

Теперь я был жрецом.
Вторым жрецом в Тхитроне, чьё сердце ещё билось, и единственным, чьи руки могли держать нож прямо.
<<Я принесу жертву богам и в скорби, и в болезни, и при смерти>> "--- так говорится в клятве.
<<Сегодня можете спать спокойно>> "--- так сказал я людям.
Кто же знал, что этот долг и это спокойствие свалятся на мои плечи.

Одно жертвоприношение мало отличается от другого.
Сначала ты сидишь и ждёшь знака кирпича, стараясь не заснуть;
впрочем, звук взорвавшегося терракотового бруска, изготовленного специальным способом, способен пробудить даже мёртвого.
Затем ты так же терпеливо, медленно, бесстрастно лишаешь жизни другое живое существо, стремясь максимально продлить его страдания.
Чем дольше и мучительнее умирает жертва, тем больше тихих дней впереди.

Одно жертвоприношение мало отличается от другого.
Но есть то, которое отпечатывается в памяти навсегда.
Первое.

Я сидел в камере и смотрел на пленника.
Средних лет мужчина-идол, низкорослый, широкоплечий.
Его кисти пестрели короткими впалыми шрамами, выдавая ловца змей.
Свалявшийся, покрытый засохшей кровью пучок разномастных волос "--- волос врагов-людей "--- бросал непроницаемую тень на избитое, опухшее лицо.
Идолы Живодёра срезали с головы врагов кожу и вживляли себе;
постепенно лысые зелёные головы и спины прославленных воинов покрывались роскошной пёстрой шевелюрой.

Я хорошо помню, как тускло мерцали из-под татуированных век лишённые белков глаза.
Сели редко татуировали лицо;
у моего народа бытовало поверье: вечная краска на лице "--- вечный бой.
Идолы же считали, что духи костей через священные знаки проникнут в глаза и защитят своих потомков от болезней, врагов и смерти.

Этому, увы, не помогли ни татуировки, ни боевое мастерство.

Пленник молчал.
Я уже не был ребёнком, который без лишних вопросов, от всей души ненавидит идолов.
Мне хотелось сказать ему что-то напоследок, хоть как-то облегчить его участь.
Но что сказать?
Любые слова излишни, любая фраза из моих уст "--- уст врага и мучителя "--- будет лишь грязной насмешкой.

"--*Ликхмас, пора.
Выводи его, "--- окликнул меня из темноты Митликх.

Я видел ранее, как грубо выволакивали жертв жрецы.
Заслуживал ли идол такого отношения?
Митликх отомкнул замок, и я просто махнул идолу, призывая его следовать за мной.
К моему удивлению, пленник послушно встал и пошёл, перед этим испустив совершенно человеческий, полный грусти вздох.

"--*Держи его, а то сбежит, "--- предупредил Митликх.
"--- Опытный воин.
Эти твари ловкие до ужаса.

"--*Некуда ему бежать, "--- возразил я.
Идол, словно соглашаясь со мной, молча пошёл к выходу из крипты.
Митликх, подозрительно глядя на пленника, покачал головой и аккуратно закрыл дверь.

Жрецы уже ждали нас во всеоружии.
Их фигуры казались толстыми и неуклюжими из-за мешковатых <<кровавых плащей>>, надетых поверх обычной синей робы.
Лица скрывали <<утиные маски>>, холодные глаза замерли, выжидающе глядя на пленника.

Идол замер перед алтарём.
Он в ужасе смотрел на бронзовые скобы, на чёрный, пропитанный кровью базальт, на осколки кирпича, ещё лежавшие вокруг.

Я вдруг понял, что именно сейчас нужно приободрить.

"--*Воин, "--- сказал я ему на наречии Живодёра, "--- я вижу по твоим шрамам, что ты всю жизнь храбро сражался.
Это твоё последнее сражение.

Идол бросил на меня пронизывающий бешеный взгляд.
В какой-то момент я подумал, что сейчас он кинется и попытается хотя бы зубами\ldotst
Двое жрецов нетерпеливо схватили его за плечи:

"--*Так, хватит церемоний\ldotst

"--*Тихо, "--- басом громыхнул Первый, и вплетённые в его волосы перья качнулись.
"--- Парень всё делает правильно.
Отпустите.

Жрецы нехотя отошли от пленника.
Идол снова вздохнул и неуклюже, боком лёг на алтарь.
Первый жрец быстро, со знанием перерезал его путы и прикрутил конечности к бронзовым кольцам.
Его ассистент вставил в рот жертвы расширитель.
Идол зажмурился.

Я не стану описывать жертвоприношение.
В голове помутилось на первой минуте жуткого, полного отчаяния тонкого крика.
Искажённое болью татуированное лицо отпечаталось в моей памяти, словно его вырезали раскалённым алмазным резцом\ldotst

\ldotst Пришёл в себя я уже внизу.
Чадил и трещал почти прогоревший факел, жрецы шли в полутьме, на ходу снимая с себя <<кровавые плащи>> и обтирая ими руки.
Они говорили о каких-то обыденных вещах, кто-то даже смеялся.
А мне казалось, что я после увиденного не смогу улыбаться никогда.
Всё казалось каким-то нереальным, мир раскололся на две части "--- крышу храма и всё, что вне её.
Как можно ходить по улицам, веселиться с друзьями, когда каждые несколько дней в мучениях умирает живое существо?
Как можно есть, пить и сладко спать, зная о происходящем на крыше храма?
Как можно жить\ldotsq

Никто не замечал, что со мной творилось что-то неладное.
Лишь Кхатрим подошёл и, обняв, погладил меня по голове.

"--*Ликхмас, "--- сказал тогда врач.
"--- Я знал, что тебе пока нельзя на крышу.
Госпиталь "--- одно, это "--- совершенно другое.
Но ты жрец.
Руки жреца обязаны знать ритуал.
Идём, угощу медовухой, полегчает.
Потом сразу домой.

Медовуха не помогла "--- желудок скрутило, тело охватила лихорадка.
Я попрощался с жрецами, но до дома так и не дошёл "--- ватные ноги подкосились на нижнем этаже.
Тогда я ещё не знал и не мог знать о реакции Стлока, я просто думал, что заболел.
Рвало до крови, пустой желудок сводило так, что невозможно было даже вдохнуть.
Ликхэ и Ситрис увели меня в спальню и ухаживали за мной, как за ребёнком.

"--*Так, быстро, раздевай его.
Спирт я принёс, растирай, "--- шёпотом командовал пожилой воин.

"--*Донышко, может, жрецов позвать? "--- испуганно спросила Ликхэ.

"--*С ума сошла?
Он реагирует не так, как обычные адепты.
Многих рвёт, да, но лихорадка?
Его могут не допустить\ldotst "--- Ситрис нервно оглянулся и приник к уху Ликхэ.
Та ахнула.
"--- Так что никаких жрецов.
Я кое-что позаимствовал у Кхатрима, сейчас дадим ему\ldotst "--- он снова зашептал что-то едва слышно.

"--*Ликхэ, Ситрис, может, хватит уже ерундой заниматься? "--- проворчал заспанным голосом Кхарас.
"--- Спать только мешаете.
Ликхмас, возьми себя в руки, всех рвало в первый раз.
Был бы это ещё ребёнок, куда ни шло, но идол\ldotst

"--*Заткнись и спи, "--- рявкнула на него Ликхэ.
Это был первый и последний раз, когда она позволила себе такое обращение с боевым вождём.
Слегка ошарашенный Кхарас замолчал, но я слышал, как он нервно возится на своей лежанке.

Как я узнал впоследствии, Ситрис дал мне едва ли не единственное средство против реакции Стлока, которое было у него в распоряжении.
Что заставило его напоить меня именно этой, растущей исключительно в Пыльном Предгорье травой "--- ещё одна неразгаданная загадка.

Странный, дурманящий пряный отвар сбил температуру и успокоил внутренние органы, но рассвет не принёс облегчения.
Жуткий крик преследовал меня по дороге на жреческий этаж.
Он звенел в посуде, из которой я ел.
Он завывал в ветре, который прорывался в узкие окна.
В душе я яростно тёр губкой тело и никак не мог смыть с себя этот ужасный предсмертный крик.

Потом я видел много жертвоприношений.
Вначале просто наблюдал, как работают старшие жрецы, потом ассистировал.
Под моим ножом испускали дух дети, взрослые, пленные иноплеменники.
Но каждый раз в ночь жертвоприношения мне снился один и тот же сон "--- тёмная комната без окон и дверей и связанный идол, молча смотревший на меня чёрными, лишёнными белков глазами.

\section{Чужие сны}

Костёр нашёл меня на берегу моря, на обычном месте.
С воды дул зябкий бриз;
я пытался устроиться между скал.

"--*Нашёл место, "--- проворчал врач.
"--- А если простудишься?
Подумай хоть о личинках, если о своём здоровье не думаешь.

"--*Меня одолевают странные сны, "--- признался я.
Я вдруг понял, что среди моего окружения Костёр "--- едва ли не единственный, кто сможет это понять.
"--- Знаешь, все эти разговоры о войне, всё это оружие\ldotst меня это как-то выбило из колеи.

"--*И не одного тебя, "--- хмыкнул Костёр.
"--- Всё поселение талдычит, что кодекс тси под угрозой.

"--*Баночка ни в чём\ldotst

"--*Баночку никто не обвиняет, "--- прервал меня врач.
"--- В конце концов, его странные, но \emph{пока что} безобидные увлечения "--- лишь видимое следствие глубинного процесса.
Народ пока ещё не настолько одичал, чтобы этого не понимать.
Так что тебе снится?

"--*Мне снилась пустыня.
Я ужасно устал\ldotst так трудно быть командиром.
Понимаешь, я устал во сне.

"--*Пустыня, говоришь, "--- отозвался Костёр.

"--*Страшная жара, разорённый Улей.
Он был покинут, "--- с жаром заговорил я.
"--- У меня онемели две ноги, я едва шёл.
Кани, люди, планты вокруг меня, все такие же измождённые.
Они сказали, что нельзя заходить в этот Улей, что он\ldotst

"--*Что мы станем лёгкой добычей для диверсантов-самоубийц, которые носят взрывные устройства на себе.
Матка засылала таких\ldotst

"--*Вместе с разведчиками, "--- подхватил я.
"--- Четыре разведчика, два диверсанта\ldotst

Я поперхнулся и схватил Костра за шировот.

"--*Откуда ты это знаешь?

Костёр болтался у меня в руках как кукла.
Усталая кукла.

"--*Откуда я это знаю?
Из истории?

"--*Да какая история, "--- усмехнулся Костёр.
"--- Ты уверен, что твои сны "--- только твои?

Врач встал и потянулся.

"--*Мне иногда кажется, что ты и не умирал у меня на руках, командир.
Что и я не оставил свои кости где-то там, в радиоактивном песке.
Что мы так и шли вместе "--- мрак Красных Каньонов, ветреные вечнозелёные пустоши, которые терпко пахли смолой и гниющими растениями, "--- помнишь этот запах? "--- котловина Кон-Тики, разорённый Улей и дальше, сюда, на Планету Трёх Материков.
Что та война продолжается до сих пор.
И я всё так же, как и тогда, готов идти с тобой хоть в ядерное пекло, хоть в бараки Прокажённых, хоть в пыточные камеры Матки.

Я слышал слова врача словно издалека.

"--*Помнишь те четыреста двадцать километров?
Мы сбили ноги до костей, и ты, надрываясь, тащил меня на себе.

"--*Осталось немного, "--- пробормотал я.
"--- Осталось совсем чуть-чуть.

"--*Ты повторял это каждые четырнадцать шагов, словно заведённый, на языке, который я едва научился понимать.
А тот крик ликования, который разносился по радиоактивным полям?
Я не мог его слышать, командир "--- я был уже мёртв.
Но я слышу его так же ясно, словно сам стоял рядом во всеми.
Эти два слова надрывали моё горло, как надрывали глотки прочих.

Этого просто не может быть.
Это просто совпадение.
Мы просто все немного сходим с ума\ldotst
Но я ведь тоже слышал этот крик.
У меня не было ни рук, ни ног, ни голоса, ни ушей, но я слышал\ldotst

"--*Нет, это не сумасшествие, "--- покачал головой Костёр.
"--- Войны отпечатываются не только в памяти людей "--- о них помнят биологические виды, планеты и сама ткань мироздания.
Война Тараканов была худшей, которую видели Ветви Земли "--- из тех войн, которые начисто отбивают тягу к конфликтам.
Даже двести тысяч лет благоденствия не способны изгнать её из наших снов.
Когда война приходит вновь, возвращаемся и мы с тобой.

\emph{Возвращаемся.}

"--*Жаль только, что не с кем об этом поговорить "--- разве что с Баночкой, "--- пожал плечами Костёр.
"--- Но наш маленький зелёный воин ещё не настолько испугался собственных снов, чтобы ко мне прийти.
Он-то выжил.
Выжил, насеял желудей и к старости стал похож на морщинистую жабу.
Да всё молился своим амулетам, чтобы мы с ним в одно посмертие угодили, а не каждый в своё\ldotst
Помнишь его молитвы, м?
Бормотание на грани слышимости, когда засыпаешь.

Конечно, я помнил.
Сложно забыть эти едва различимые слова, даже единожды их услышав в переплетении ночного шума.
Единственное, чего я так и не мог услышать "--- собственное имя.
Далёкий шёпот Баночки расплывался каждый раз, когда друг поминал меня.

\begin{quote}
<<Маат, Славная мать, корни мои, соки моего хребта, любящая и всепрощающая.
Прости слабость твоему побегу и позволь остаться навечно с теми, кто стал мне дороже племени.
Маат, Славная мать, сила рук моих, влага моих глаз.
Не зелена их кожа, и кровь их красна, как противный тебе огонь, и не ведаю я, кто покровитель их в жизни и посмертии, но иссохну я без них, как иссыхают лишённые твоей любви>>.
\end{quote}

"--*<<Маат, Славная мать, корни мои, соки моего хребта>>, "--- прошептал Костёр, и ветер ответил воем в далёких скалах.
"--- Так вот бывает.
Глупость, суеверие, а через двести тысяч лет слышно, словно вчера было сказано.

"--*Возьми мою руку, четыре к пяти, "--- вырвалось у меня.

"--*Тебе ещё рано это говорить, командир, "--- одёрнул меня врач, но тут же улыбнулся и погладил по голове.
"--- Прими успокоительное и иди спать.
\emph{Эти} сны всегда приходят по одному.

\section{Лихие времена}

Едва очнувшись от беспокойного прерывистого сна, я бросился в зал.
Всегда есть надежда, что произошедшее тебе только приснилось.
А ещё есть надежда, что тебя обманывают глаза, когда видишь, что в толстом каменном перекрытии зияет дыра "--- огромная, неровная, неуместная дыра, словно культя руки, словно безвременно умерший друг, словно навсегда покинутая родина, словно дом, оставшийся лишь в детских воспоминаниях.

Храм пустовал.
Нет, разумеется, здесь ещё лежали осколки плит, ещё клубилась каменная пыль, а у полуразрушенной лестницы уже примостились несколько досок и чемоданчик каменщика.
На нижнем этаже, кажется, даже кто-то разговаривал.
Но никогда на моей памяти храм не был настолько пустым.
В нише на стене стояла раскрашенная чаша;
я проспал погребение.
Вернее, меня не стали будить "--- обидный до ужаса поступок, если не знать, что за ним стоит.

Учитель сидел в библиотеке и переписывал книгу, как и всегда.
Медленно догорала свеча, как и всегда.
Однако, подойдя ближе, я увидел, что переписной лист был покрыт одним и тем же незнакомым иероглифом.

Я взял лист из стопки.
То же самое.
И ещё один такой же.
Последние несколько кхамит Трукхвал, словно испорченный колодезный ворот, проворачивал один и тот же знак.
Мысль жреца остановилась на третьей странице <<Обычая и ритуала>>;
узловатые измученные руки продолжали писать.
Я испуганно потряс его за плечо.

"--*Звоночек!

"--*Я отправил весточку в Тхартхаахитр, "--- глядя в стол, отсутствующим тоном ответил учитель.
"--- Жрецы прибудут в ближайшее время.
Ликхмас-тари, я должен попросить тебя о большом одолжении.

"--*Я весь внимание, учитель Трукхвал.

"--*У нас не осталось ни врачей, ни школьного учителя, ни жертвенников.
Митликх обещал вернуться, но когда "--- неясно.

"--*Ты ему написал?

"--*Ему написали ещё ночью, но ты сам знаешь, где находится Кахрахан.
Мы с тобой совершенно одни, "--- губы учителя задрожали.
"--- Я пойду собирать травы и лечить.
Воины будут следить за знаком кирпича, ты возьмёшь жертвы.
Но я хочу попросить тебя заглядывать ещё и в школу.

"--*Учитель, мы не выдержим и полдекады работы в таком ритме.
Ведь есть же жрецы, в своё время ушедшие из Храма?
Есть же Храмы ближе Тхартхаахитра?
Может, где-то рядом проходят отряды чести или даже сами Люди Золотой Пчелы, а мы не знаем?
Почему не попросить их?

"--*Твоя кормилица сказала, что Люди Золотой Пчелы сейчас не в землях сели, они ушли куда-то в сторону Ожидания Вести, возможно, что и за море.
У всех сейчас нехватка.
Когда пару лет назад обезглавило Храм Травинхала, жрецы ехали аж с Запада.
Да и ты помнишь, когда и откуда приехали Кхатрим и Ситлам.

Я опустил голову.

"--*Да, я тоже понятия не имею, как это всё успеть, но у нас просто нет выбора, и поэтому школу и библиотеку придётся\ldotst

"--*Я возьму на себя школу, "--- опередил я Трукхвала.
"--- Травы тоже отдай мне, но не оставляй библиотеку.
Выдержим, сколько сможем.
Надеюсь, духи помогут и больных будет мало.
Брошенные книги рассыпаются в пыль, ты ведь помнишь.

Учитель слабо улыбнулся и вдруг заплакал.

"--*Ликхмас-тари, я воспитал самого лучшего жреца на всей Короне.

\section{[] Как чудесен мир}

\spacing

Вереск корпела над своим творением день и ночь.
Она с маниакальной скрупулёзностью готовила каждое блюдо, кормила всех, кого только могла поймать, брала пробы и подвергала компоненты ферментативному разложению, а также сравнивала результаты квантово-химического анализа с субъективными ощущениями тси.

"--*Ну куда ты! "--- пытался урезонить её Мак.
"--- Все говорят, что ты готовишь вкусно, абсолютно все.
И переваривают, и чувствуют себя хорошо.
\mulang{$0$}
{Ты хочешь достичь идеала, который не сможет оценить вкусовой анализатор сапиента!}
{You want to achieve the ideal couldn't be recognized by gustatory system of sapient!''}

\mulang{$0$}
{"--*Это ваш не сможет, "--- парировала Вереск.}
{``By your one,'' Heather used to answer.}

И вот наконец книга была дописана.
Узнал я об этом ранним утром, за завтраком.
Заяц пришла ко мне в слезах.

"--*Вереск умерла.

"--*Что?

"--*Повесилась за шею на промышленном тросике.
Это не несчастный случай.
Она планировала всё заранее.

Заяц сунула мне кусок ткани с неровными иероглифами.

\begin{quote}
<<Дорогие мои тси.
Я закончила свою книгу "--- Десять тысяч блюд.
Теперь, даже если теплица выйдет из строя, вы будете хорошо кушать и жить счастливо на этой планете.
А я не могу.
Не упрекайте меня, прошу вас.
Я честно пыталась начать жизнь сначала, я нашла в лице многих из вас друзей.
Однако сложно свыкнуться с мыслью, что ты находишься не там, где должна.
Недавно я узнала, что молодые племена верят в жизнь после смерти в местах, которых не существует.
Как чудесен мир!
Какие сюрпризы он ещё преподнесёт нашим учёным!
Если бы верования этих людей были правдой, я хотела бы вернуться на родную планету "--- такую, какой она осталась в самых светлых моих воспоминаниях.\\~\\
Пчела-Нюхающая-Вереск, инженер-технолог пищевой промышленности Тси-Ди>>.
\end{quote}

\section{Славь духов}

Следующая декада прошла словно в тумане.
Я держался как мог "--- в ход пошли и крепкие отвары, и пощёчины.
Количество часов в школе пришлось сократить, чтобы я мог хотя бы поспать.
Ни о какой отчётности речи не шло "--- я писал темы уроков на стене кельи перед тем, как рухнуть на лежанку.
После ещё одной ночи жертвоприношения я притащил Трукхвалу стебли ядовитого мертволаза вместо молитвенных маков.
Я ожидал гневной проповеди, но учитель, глядя на моё сонное, измученное лицо, снова расплакался.

Чханэ всё это время пропадала на тренировочной площадке.
Они с Ликхэ поделили часы напополам "--- патрулирование отнимало меньше сил.
На каждом воине теперь было три, а то и четыре квартала.
Как-то, забежав домой перекусить, я увидел девушек, Ситриса и Эрликха спящими вповалку в углу зала.
Растерянная кормилица убирала со стола нетронутую еду.

"--*Лисёнок, сегодня с тобой на крышу пойдёт Кхарас, а в дозор Кхохо возьмёт старшин кварталов, "--- сказала она.
"--- Девочки уже не кушают от усталости.

"--*Они едут? "--- этот вопрос я повторял последние четыре дня непрерывно.

Кхотлам развела руками и пододвинула мне тарелку с ещё тёплым супом.
Я взял черпачок и осознал, что тоже не хочу есть.

Кхарас, и прежде неразговорчивый, теперь молчал целыми днями.
Он стал ещё более сухим и холодным, словно северная скала.
Жертвоприношение вождь ассистировал со знанием "--- из воинов таким могла похвастаться разве что Кхохо;
однако с его губ так и не слетело ни одного звука.
Закончив дела, мы зашли в крипту и рухнули на скамью.

"--*Мягко, "--- пробормотал вождь, щупая холодный голый камень, и погрузился в сон.
Я обнял его "--- и понял, что камень обнимать куда мягче и теплей.

Митликх вернулся четырнадцатого числа.
Он вбежал в школу посреди урока и, шепнув мне <<Иди спать>>, как ни в чём не бывало повёл урок дальше.
Я сел на скамью рядом с учениками и на мгновение прикрыл глаза.
Открыл уже у себя в келье спустя сутки "--- дети в конце урока отнесли меня на лежанку, оставив у изголовья записочки с пожеланием выспаться и три пахучих увядших венка.

"--*Проснулся наконец-то, "--- проворчала Кхохо.
Оказывается, она всё это время сидела в углу, положив голову на руки.
"--- Твой завтрак у лежанки.

Сон сползал неохотно, словно засахарившийся мёд.
Да, завтрак действительно стоял.
Правда, Кхохо не удержалась и отгрызла корку у куриного пирога, однако чтобы воительница носила кому-то еду в постель "--- дело неслыханное.

"--*Кормить с ложечки не буду, ты уж извини "--- ненавижу эту детскую возню, "--- скривилась Кхохо.
"--- Жри своими руками.
А лучше "--- ртом.

"--*А\ldotst

"--*А корку отгрызла не я.
Это новый Первый жрец.
Я шла с подносом, а он вцепился в пирог зубами, как зверь.
Я отбила самую лучшую часть.

Я улыбнулся, чувствуя прилив нежности к Кхохо.
Как же я люблю эту странную, прожорливую, мстительную лгунью.

"--*Когда они прибыли, Уголёк?

"--*Вчера днём.
Они даже успели провести расследование диверсии.

"--*И это не хака, "--- я уже успел обдумать ситуацию за прошедшую декаду.

"--*Эти дикари нас с тобой спасли, Лис.
Идолы проникли на стоянку каравана и подменили свинцовый блеск в коробках на какое-то грязное взрывчатое вещество "--- вероятно, смесь кусачей бумаги\footnote
{Кусачей бумагой сели называли нитроцеллюлозу. \authornote}
и отходов металлургии.
Диверсант уже на нашем складе подорвал коробки вместе с собой.
Эта тупая тварь даже не знала, что именно ей приказали поджечь.

Кхохо смачно плюнула на стену и тут же, оглянувшись на меня, вытерла плевок рукавом рубахи.

"--*Скорее всего, идолы намеревались атаковать город малыми силами во время радужного безумия.
Они всё рассчитали.
Было время трапезы.
Если бы не эти внезапные переговоры с хака, пал бы весь Храм.
Но кихотр лёг не в их пользу "--- выжили два жреца и почти половина Нижнего этажа.
Трукхвал пошёл во Двор за какими-то бумагами, громыхнуло прямо у него за спиной.

Я откинулся на подушку.
Учителю постоянно пеняли, что он опаздывает на трапезу и ему не достаётся супов;
в этот раз ему не досталось хэситра.

\mulang{$0$}
{"--*Был ли диверсант?}
{``Was there one to detonate?}
\mulang{$0$}
{Кусачей бумаге достаточно неаккуратного обращения.}
{It's enough to handle biting paper roughly.''}

\mulang{$0$}
{"--*Был.}
{``It was.}
\mulang{$0$}
{Куски его зелёной башки я сама нашла.}
{I've found pieces of its green head.''}

\mulang{$0$}
{"--*Трукхвал же лично проверил коробки.}
{``Tr\`{u}kchu\r{a}l checked all the boxes by himself.}
\mulang{$0$}
{Там был свинцовый блеск.}
{They definitely contained lead glance.''}

\mulang{$0$}
{"--*Ликхмас, голову приложи.}
{``L\={\i}kchm\r{a}s, use your head.}
\mulang{$0$}
{Конечно, они замаскировали взрывчатку тонким слоем свинцового блеска.}
{Of course explosive was masked with thin layer of lead glance.}
\mulang{$0$}
{Я бы сделала так же.}
{I would do the same.''}

"--*А как они поняли, что коробки пойдут в храм?

"--*На них всё написано, "--- горько хмыкнула Кхохо.
"--- Оказывается, идолы тоже умеют читать.
Ты жри давай, я зря, что ли, эту баланду тащила?
\mulang{$0$}
{Пусть работает желудок, а не голова.}
{Let your stomach work, not your brain.}
\mulang{$0$}
{Прошлое "--- в прошлом.}
{The past is in the past.''}

Я посмотрел на пищу.
Вот оно "--- счастье.
Я сладко потянулся и пододвинул к себе поднос, предвкушая полный желудок и ещё одни сутки в постели.

\mulang{$0$}
{"--*Слава духам.}
{``Praise Spirits.''}

\mulang{$0$}
{"--*Иди в зал и славь их там, если ещё останется желание.}
{``Off to the hall and praise Them all, if you'll still want to.}
\mulang{$0$}
{Я воздержусь.}
{I'll pass.''}

И, вздохнув, Кхохо снова погрузилась в глубокий сон.

\razd

Пока я ел, окутанный сонной дымкой мозг лихорадочно соображал над последними словами Кхохо.

Воительница выглядела не просто уставшей "--- она испугалась.
Что-то напугало человека, который прорывался сквозь заросли копий и нелицемерной ухмылкой приветствовал смерть.
Что может быть страшнее смерти?

Я вдруг вспомнил слова Сиртху-лехэ: <<Люби оленя всей душой.
Звери никогда не уйдут от человека, который их любит, как бы он ни проявлял эту любовь>>.
Кхохо не испытывала страха, и мгновение настоящего испуга потрясло её до глубины души.
Она просто не знала, что делать с этим чувством.
О чём вспоминают в такие моменты?
О том, кто может защитить, и о месте, где можно спрятаться.
Звериный разум Кхохо, ведомый странными инстинктами, привёл её в мою келью.

Я ощутил необычность ситуации.
Кхохо пришла не к Ситрису, умелому бойцу, не к Кхарасу, командиру Нижнего этажа, не к Трукхвалу, у которого всегда можно было спрятаться от обязанностей и попить отвара, и не к кормилице, которая владела словом и справедливо разрешала конфликты.
Она пришла ко вчерашнему мальчишке, который не имел ни щепотки власти и мертвецки спал после бессонной декады.
И, разумеется, ни о какой заботе или признательности речь не шла "--- всё это так же мало знакомо Кхохо, как и страх.
Воительница принесла пищу, чтобы задобрить непонятное существо, которое было её последней надеждой.

И отгрызла корку у пирога.
Вполне понятный жест для того, кто знал Кхохо, однако и тут что-то неуловимо не вязалось одно с другим\ldotst

Нужно срочно узнать причину её страха.
Что-то подсказывало мне, что она находится в зале.

Поев, я поднял Кхохо на руки и уложил на свою лежанку.
Она стала лёгкой, словно перо, щёки ввалились, а скулы проступили, словно скалы из-под сгоревшей сельвы.
Воительница сильно похудела со времён нашего последнего спарринга.
Интересно, когда это было "--- три, четыре декады назад?
Она вдруг открыла глаза и в упор посмотрела на меня.

"--*Это не я слопала корку, Лисичка! "--- жалобно-обиженно заявила воительница.
"--- Не я!

\mulang{$0$}
{Я улыбнулся.}
{I smiled.}

\mulang{$0$}
{"--*Я знаю, Уголёк.}
{``I know, Coal.}
\mulang{$0$}
{Спи.}
{Sleep.}
\mulang{$0$}
{Здесь ты в безопасности.}
{You're safe here.''}

Новые жрецы уже были в зале, за наспех сколоченным столом у целой стены.
Они разом повернулись ко мне, едва я ступил за порог;
семь пар пустых глаз взглянули в мои "--- и у меня едва не подкосились ноги.
Я никогда не видел в человеческих глазах такой пустоты.
Это была не глупость и не бесчувственность "--- пустота просто выглядела чужой, как чужда человеку Каменная ярость.
Сидевший на месте Первого жрец улыбнулся странной, кривой улыбкой:

"--*Ликхмас ар'Люм.
\mulang{$0$}
{Мы тебя ждали.}
{We've been waiting on you.}
\mulang{$0$}
{Проходи и чувствуй себя как дома.}
{Come and be welcome.''}

Надкушенная корка пирога, уже изрядно обсиженная мухами и осами, сиротливо лежала на углу стола.
Проследив за моим взглядом, Первый небрежно смахнул её на пол.

\section{Нарушенное обещание}

\epigraph
{Никогда не позволяйте изолироваться людям, обладающим опасными навыками и знаниями, всеми силами удерживайте их в обществе.
Иначе может настать день, когда вам придётся жить по их законам.}
{Март <<Одноглазый>> Митчелл, идеолог Эволюциона}

\mulang{$0$}
{<<Меня не ждали>>.}
{\emph{``I'm not welcome.''}}

Митхэ изо всех сил пыталась уложиться в плотный график.
Вначале зайти к Серебряному на постоялый двор и проверить, чтобы слуги его накормили.
Затем проведать тяжелораненного наёмника, которого перетащили из храма в один из жилых домов под предлогом предстоящего празднества "--- вопиющее неуважение.
Затем найти, вытащить из канавы и привести в человеческое состояние Эрхэ "--- воительница была в запое с достопамятного дня, стоившего Митхэ оружия.
Затем заплатить за работу шорникам и забрать из починки цитру Атриса, который уже без разрешения врача ковылял по площади и рвался играть.
Венцом тяжёлому дню должен был стать совет Храма Тхартхаахитра.
Однако когда взмыленная Митхэ поднялась в величественный зал, она поняла "--- совет начался едва ли не на кхамит раньше назначенного времени.

\mulang{$0$}
{<<Меня не ждали>>.}
{\emph{``I'm not welcome.''}}
Жрецы бросали на Митхэ нервные взгляды.

"--*Здравствуй, Митхэ ар'Кахр, "--- как ни в чём не бывало поприветствовал воительницу Король-жрец и закончил:
"--- Что ж, на сегодня все вопросы решены.
Все свободны.

"--*Подождите, "--- Митхэ махнула рукой, и уже поднявшиеся из кресел жрецы замерли.
"--- Я прошу прощения, что пропустила совет.
\mulang{$0$}
{Король-жрец, было ли вынесено решение по известному тебе делу?}
{Priest-king, had the case you know been resolved?}
\mulang{$0$}
{Не мог бы ты его повторить для меня?}
{Could you repeat for me?''}

\mulang{$0$}
{"--*Да, разумеется, "--- улыбнулся Король-жрец.}
{``Of course I could,'' Priest-king smiled.}
\mulang{$0$}
{"--- Как мы могли забыть про это дело.}
{``How dare we forget about that case.}
\mulang{$0$}
{Сегодня вечером будет большое празднество!}
{We have a great celebration tonight!''}

Кто-то крикнул <<Вино рекой!>>, и воины огласили зал громовым хохотом.

"--*Король-жрец, не увиливай.
Ты обещал амнистию тем троим кутрапам, которые сражались в моём отряде! "--- вспылила Митхэ, перекрывая шум.

Хохот затих в одно мгновение, словно обрезали шёлковую ленту.
Наступило неловкое молчание;
цитра в руках Митхэ вопросительно звякнула на весь зал.

"--*Я сказал, что посодействую, Митхэ ар'Кахр, "--- проговорил Король-жрец, с некоторым подозрением глядя на цитру.
"--- Я посодействовал, но мои люди не считают войну искуплением.
Разрушители и Насильники должны понести наказание.
Тем более один из них "--- ноа-пират, а обсуждение участи кутрапа-чужеземца не стоит и двух часов сна.

"--*Сильный, творящий произвол над слабым "--- вот истинный Разрушитель и Насильник! "--- рявкнула Митхэ.
"--- И сила в этой ситуации "--- отнюдь не трое людей, пытавшихся найти свою чашу в лоне племени!

"--*Ты можешь дать гарантии, что они не займутся прежним ремеслом, когда им наскучит спокойная и размеренная сельская жизнь? "--- осведомился Король-жрец.

"--*Я лично выслежу и убью их, если это произойдёт!

"--*Верю.
Но сколько сели пострадает до той поры?
Я не готов заплатить такую цену.

Митхэ замолчала, не зная, что ответить.

"--*Разумеется, пока эти кутрапы под защитой твоего Храма, мы не будем их трогать, "--- примирительно поднялась сухощавая ладонь.
"--- Но смею напомнить, что Тхартхаахитр за Тремя Этажами, и если мои воины обнаружат их одних "--- в городе ли, в храме ли, "--- то поступят согласно закону.
\mulang{$0$}
{Не обессудь.}
{Don't get me wrong.''}

"--*На улицах будут праздновать, расфуфыренный ты петух! "--- выкрикнула Согхо.
"--- Дай им хотя бы несколько дней пожить без страха!

Король-жрец пропустил слова воительницы мимо ушей.

"--*То есть если я хочу жить, я должен ходить с командиром за ручку, так, Король-жрец? "--- подал голос Ситрис.

"--*Я никогда не говорил и не скажу ничего подобного, Ситрис ар'Эр.
Если ты обиделся на мои слова "--- прошу прощения.
Пусть празднество будет тебе утешением.

"--*Я не держу обид на побеждённых, "--- Ситрис отвесил издевательский полупоклон.

"--*Не испытывай наше терпение, кутрап, "--- ледяным тоном сказал боевой вождь Храма.

Митхэ подхватила цитру под мышку, взяла Ситриса за руку и под сдержанные смешки присутствующих вышла из зала.
Аурвелий последовал за командиром, не забыв наградить Короля-жреца демонстративным плевком.

\razd

\epigraph
{Правители пойдут на самую низкую подлость,\\
Чтобы заполучить искреннюю верность солдата.\\
Верному не нужно платить,\\
Верного можно убить,\\
Верного можно забыть "---\\
Ценный товар для тирана.}
{Гимн наёмников Фоуф, планета Тысяча Башен}

"--*Золото, остынь, "--- пытался успокоить её Атрис.
"--- Хочешь отвара?

"--*Митхэ, перестань, "--- в десятый раз повторял Ситрис.
"--- Ведь судьба в любом случае забросит нас на Запад, верно?
Уйду с Аурвелием на Кристалл.

"--*В чём-то я понимаю позицию жрецов, "--- сказал менестрель.
"--- Без обид, дружище: убийствами разбой не искупить.

"--*Да какие обиды.
Ты прав.

"--*Но Король-жрец мог и не обещать, и возможности сдержать обещание у него тоже были.
Это уже вероломство.

"--*Это политика, "--- поправил Ситрис.

"--*Нет, дружище.
Политика "--- это если бы Король-жрец обманул хака, а не\ldotst

"--*Это в любом случае вероломство, "--- перебила Митхэ.
"--- Не имеет значение, свои или чужие.
На лжи мира не построишь.
Если враг не верит твоим угрозам, если враг не может положиться на твоё обещание, торговцы превратятся в диверсантов, странники "--- в разведчиков, а дипломаты станут погремушкой, скрывающей лязг сабель.
И о мире можно забыть на столетия.

"--*Ну вот и прояснили ситуацию, "--- подбодрил её Ситрис.
"--- Перестань глодать себя из-за этой крысы в робе.

"--*Неважно, сдержал ли обещание Король-жрец, Ситрис.
\emph{Я} должна сдержать обещание.

"--*Ты ничего никому не должна, "--- буркнул Ситрис.
"--- Ты сделала, что могла, но обстоятельства были против тебя.

"--*Отговорка Короля-жреца, Ситрис?

"--*Что для него отговорка, для тебя "--- истина, "--- пожал плечами Ситрис.
"--- Кто из вас Король-жрец, в конце концов?

"--*Ни приказ, ни запрет не оправдывают бесчестный поступок.
Я это вам постоянно говорю.

"--*Да-да, конечно.
Вы все такие честные, правильные, последовательные, как ездовые олени.

Митхэ раздражённо фыркнула.

"--*Я не имел в виду Серебряного, он свой человек, "--- поправился Ситрис.
"--- Хорошо, как курочки, которые клюют то, что им насыпали, и откладывают яйца там, где им сделали гнёздышко.
Но истина такова "--- обещания никого за горло не держат.
И рисунки твои на лице тоже, "--- быстро проговорил разбойник, предвосхищая слова Митхэ.
"--- Ну не будут тебя уважать сели, если ты поступишь вразрез с Кошачьими Слезами.
Так иди к пиратам, на Диком Юге о тебе ходят такие слухи, что ты в первый же день соберёшь армаду.
И не обязательно кого-то грабить, многие пираты "--- такие же наёмники, как и ты.
Иди со мной на Кристалл, в конце концов, и доживай свой век в таёжном тупике, попивая вино холодными осенними вечерами.
С тобой твой мужчина.
Что тебе ещё нужно?

"--*Я думала, ты поддерживаешь моё решение.

"--*Если я киваю, это значит, что я тебя понял.
Это не значит, что я с тобой согласен.

"--*Нет, ты не понял.

"--*А ты уверена, что не понял именно я?
Ты сама до конца понимаешь все эти красивые фразы, которыми ты умащаешь свой путь "--- Путь Кошачьих Слёз?
\mulang{$0$}
{Ты смелый боец, ты хитрый тактик, но когда дело доходит до человеческих взаимоотношений, ты превращаешься в ребёнка, которого можно одурачить фигурой из четырёх пальцев.}
{You're brave in battle, you're tricky in tactics, but when it comes to human relations, you become a child can be fooled by four-fingered figure.}
\mulang{$0$}
{Одно дело "--- следовать принципам, и другое "--- быть невольником своего положения\ldotst}
{It's one thing to apply the principles and quite another to be a slave of status\ldots''}

Разбойник осёкся, поняв, что зашёл чересчур далеко.
Впрочем, Митхэ была чересчур занята своими мыслями, и смысл фразы до неё не дошёл.
Атрису же срочно, в ту самую михнет понадобилось настроить починенную цитру, и он ушёл в процесс с головой.

"--*В общем, это, "--- буркнул Ситрис.
"--- Я, может, многого прошу, но советуйся со мной, если какие переговоры или что.
Всё-таки я теперь старшина, тебе не зазорно.
Без обид: если тебя каждый встречный-поперечный будет вот так дурачить, Кодекс Ягуара останется только скрутить и скурить под выпивку.

"--*Я всегда тебя выслушаю, "--- сухо ответила Митхэ, "--- но решения буду принимать сама.
Отряд чести "--- плод \emph{моих} решений, и ты в моём отряде.
Разумеется, если ты не захочешь уйти.

Ситрис хмыкнул и хотел выбежать за порог, но очень некстати вспомнил о приказе Короля-жреца.
Разбойник извернулся кошкой и снова оказался в комнате, словно пол коридора был покрыт горячей смолой.

"--*Я, пожалуй, с вами посплю, у двери.
Хат, ты не против?

\section{[] Поминки}

\spacing

"--*Это я виновата, "--- плакала Заяц.

"--*Нет, "--- твёрдо сказал Мак.
"--- Это был её осознанный выбор.
Она сделала для процветания народа всё, что могла, и ушла достойно.

Все промолчали.
Мак сказал то, что следовало, и добавить к этому было нечего.

Вечером Листик сообщила, что кольцевая теплица забрала тело Вереск полностью.
Тси почтили память подруги так, как следовало "--- собрали большой стол перед Стальным Драконом и приготовили по три блюда из каждой главы её книги.

\section{Дешёвка}

Тем временем подготовка к празднику шла полным ходом.
По всему городу запахло праздничной едой.
Книга-кормилица давно не удостаивалась такого внимания;
Митхэ подозревала, что скоро сбудется старое шутливое пророчество "--- туда, где приготовят одновременно все десять тысяч блюд, придёт отобедать даже Безымянный.
Активно поддерживал праздничное настроение и Атрис;
звуки его цитры привлекли ещё несколько бродячих менестрелей, словно огонь фонаря "--- индиго-светляков.
Вместе компания играла что-то совершенно немыслимое, собирая огромные толпы народа.

"--*Было очень здорово, "--- похвалила менестреля Эрхэ.
Её пока ещё немного мутило после длительных возлияний, что совсем не мешало получать от происходящего удовольствие.
"--- Тебе следует чаще петь весёлые песни "--- про приключения, хитрых героев и забавные случаи.
Грустные у тебя сейчас выходят, хай, неискренне "--- мордаха чересчур круглая.
Кстати, мне понравилась импровизация этого старика-зизоце с длинной бородой.
А, толстенькая девушка-трами тоже была хороша "--- редко можно увидеть талантливого Пересмешника.
Как их зовут?

"--*Я понятия не имею, как их всех зовут, "--- засмеялся Атрис.
"--- Мы, в общем-то, и не разговаривали.

"--*Красиво играешь, паренёк, "--- вдруг громко заявила Атрису проходящая мимо старуха.
"--- А вот этих Пересмешников я не понимаю.
Во времена моей молодости такое даже музыкой бы не назвали.
А молодёжь слушает.

Старуха кивнула на Эрхэ и Акхсара и гордо удалилась.

"--*Ну что, Снежок, мы теперь с тобой молодёжь, да? "--- ухмыльнулась воительница.

"--*Вы достаточно молоды, "--- пожал плечами Атрис.
"--- У вас молодые глаза.

"--*Мне тут стало интересно, чего ты такой худой и оборванный был, когда Митхэ тебя привела, "--- хмыкнул Акхсар.
"--- Сколько вам в этот раз накидали?
У нас весь отряд чести столько за год не зарабатывает.

Атрис немного растерянно потряс цитрой;
ворох украшений внутри неё внушительно звякнул.

"--*Менестрель должен намекать, что он играет ради золота, "--- объяснил Атрис.
"--- Например, ставить перед собой чашу.
Однако чаша "--- это напоминание: ты слушаешь не бесплатно.
Я не хочу, чтобы моих слушателей что-то тяготило.
Когда было особенно туго с едой, я просто просил, чтобы меня накормили.
Тебе нужно золото?
Я могу отдать его тебе.

"--*Ты лучше себе шёлк для праздничной одежды купи, дружище, "--- весело посоветовал проходивший мимо Ситрис.
"--- Зачем этому истукану золото?
На украшения для Эрхэ?
Или на вино, чтобы она и дальше топила свою несчастную любовь?

"--*Иди в бездну, немытая шея.
Без тебя разберёмся.

Акхсар бросил нервный взгляд на покрасневшую воительницу и пошёл к храму.

Эрхэ, проводив его долгим печальным взглядом, подошла вплотную к Атрису и приникла к его уху:

"--*Ты точно можешь без проблем расстаться с золотом?

"--*Мне не на что его тратить, "--- шёпотом ответил менестрель.

"--*Тогда слушай.
То что я тебе говорю, должно остаться между тобой и мной.
Этот мир довольно несправедлив "--- один человек отдал всё, что имел, а второй не получил достойной оплаты\ldotst

\dots Перидотовая Лиана вернулась к хозяйке перед самым праздником.
Оружейник прислал питомицу.
Девочка с важным видом отдала Митхэ клинок, а затем, порывшись в карманах, ойкнув от испуга и порывшись ещё раз, с облегчением достала запечатанный свиток и увесистый мешочек.

<<В кошельке "--- сдача.
Дар Короля-жреца принять не могу, чересчур ценный.
Всё произошедшее "--- между тобой и улицей Серого Рассвета.
Благослови тебя Сит, Митхэ ар'Кахр>>, "--- гласила коротенькая записка.
Перстень был аккуратно вплавлен в восковую печать.

"--*Сдача? "--- Митхэ озадаченно посмотрела на посланницу.

"--*Вопрос удалось решить с минимальными затратами при сохранении качества, "--- важным тоном ответила девочка.

"--*Ничего не понимаю, "--- призналась Митхэ, взвешивая кошелёк в руке.
"--- Здесь как будто даже больше, чем я\ldotst

"--*А я гравировку сделала, "--- шёпотом похвасталась девочка.
Похоже, больше сдерживаться она не могла.
"--- Тебе нравится?
Ну давай, посмотри!
Пожалуйста, "--- тут же добавила она.

Митхэ выдернула клинок из ножен\ldotst и тут же задвинула обратно.
Она знала, что у неё будет время на то, чтобы изучить новое оружие, и одного взгляда хватило, чтобы понять "--- эти михнет будут полны самых светлых чувств.
Образ сабли пламенел перед внутренним взором "--- как живое море, как грозовые небеса, как осколок некоего идеального творения, распространившего волны гармонии по всей Вселенной.

"--*Легенда Серого Рассвета, "--- прошептала она.
"--- Знай, Глупец: эти достойны.

"--*Ну как? "--- жалобно спросила девочка.

"--*Это самая лучшая на свете гравировка, "--- заверила её Митхэ.
"--- Твои руки превратят любое знание в красоту.

Девочка просияла.
Митхэ повертела перстень Короля-жреца в руках и отдала его маленькой мастерице.

"--*Это тебе за работу, "--- ласково сказала она.
"--- А кормильцу передай, что перстень стоит ровно столько, сколько в нём содержится золота.

\section{Травы или котомка}

"--*Что значит <<за работу>>?

"--*То и значит, "--- сказал я.
"--- Жрецов сейчас достаточно, и я прошу разрешения\ldotst

"--*Кто решил, что жрецов <<достаточно>>?
Кажется, делать такие выводы не в твоей компетенции.

Я замолчал.
Первый жрец начал сбивать меня с толку с первых же слов.
Почему для умозаключений должна существовать компетенция?

"--*Так чего ты от меня хочешь?

"--*Отпуск, "--- лаконично сказал я.
"--- Заслуженный отпуск на два рассвета.

"--*Кажется, между нами некоторое недопонимание, Ликхмас ар'Люм.
В данный момент ты не являешься адептом моего Храма.
И ты им не будешь, пока я не приму соответствующее решение.
Так о каком отпуске речь?
Ты свободен, как птица.
Может, ты ошибся и пришёл просить меня, чтобы я принял тебя в Храм?

Я прищурился.
Хорошо, сыграем по твоим правилам.

"--*Хотелось бы, чтобы меня приняли в Храм через два рассвета.

"--*Ты действительно этого хочешь?

"--*Я настаиваю, "--- оскалился я.

"--*Но к чему медлить, если ты \emph{настаиваешь}?
Я приму тебя в Храм прямо сейчас, если ты ответишь на несколько вопросов.
Разумеется, ты можешь отказаться и поискать другой Храм.

Я мысленно выругался.
Мне явно не хватало дипломатического опыта кормилицы.

"--*Скажи мне, Ликхмас ар'Люм, что для тебя главное в жизни?

"--*Дом и те, кто в нём живёт, "--- не задумываясь ответил я.

"--*Хорошо, я спрошу по-другому.
Что для тебя как жреца главное в этой жизни?

"--*Быть щитом, и ни мгновения "--- плетью.

Жрец поморщился, словно ему в рот попало нечто горькое и этим горьким его кормили по меньшей мере год.

\mulang{$0$}
{"--*Единственным мерилом защиты и нападения является закон.}
{``The only measure of defense and offense is the law.}
Законы сели мудры, и мы, Ликхмас ар'Люм, должны им следовать.
Что ты думаешь о Ситрисе ар'Эр?

"--*Трааа\ldotst "--- меня несколько смутила неожиданная смена темы, "--- Ситрис "--- один из лучших тактиков Тхитрона и помощник боевого вождя.
Он мой друг и наставник.
Этого достаточно для описания?

"--*Вполне, "--- согласился Первый.
"--- Твоя кормилица сказала почти то же самое, а у меня нет причин ей не доверять.
А как ты относишься к Кхохо ар'Хетр?
Её манера общения плохо вяжется с весьма средними профессиональными навыками.
Я как-то не привык, чтобы воины храпели на посту и просыпались только от четвёртой пощёчины.

Я нахмурился.
Первый жрец не мог не знать, в каком режиме Храму пришлось работать последнюю декаду.

"--*Хорошо, оставим вопрос о Кхохо открытым, "--- вздохнул жрец.
"--- Буду с тобой честен и прям: в Тхитроне у нас есть враги.
Мы "--- то есть Верхний этаж и, ммм, часть Нижнего "--- полагаем, что и нападения хака, и недавнюю диверсию идолов помогали проводить именно они.
Поэтому я надеюсь, Ликхмас ар'Люм, на твою помощь в защите народа и в обеспечении торжества закона над теми, кто его преступил.

"--*Народ может защитить себя сам, "--- заявил я.
"--- Жрецы просто указывают путь, а воины координируют действия.
Почему ты, Первый жрец, решил взять такую тяжкую ношу на одного себя?

Первый ухмыльнулся.

"--*Ты умён не по годам, но недостаточно, чтобы понять суть происходящего, Ликхмас ар'Люм.
Где ты работал последнее время?

"--*Я учил детей в школе и приносил жертвы.

"--*А не слишком ли ты молод для этого?

Однако я успел понять, к чему клонит жрец, ещё на первом вопросе, и решил завершить унизительную беседу на своих условиях:

"--*Травы или котомка\footnote
{Котомка "--- обещание покинуть город. \authornote}.

"--*Третья келья "--- твоя, "--- милостиво разрешил Первый.
"--- На вопросы ты ответил верно, и от своих слов я не отступлю.
Но если врачам не хватит одного "--- слышишь, Ликхмас ар'Люм? "--- одного-единственного пузырька с микстурой, котомку я для тебя соберу лично.
В знак уважения к твоим заслугам, так сказать.
И помни "--- слава о жреце, бросившем Храм, летит быстрее испуганной лани.

\chapter{[-] Празднество}

\section{[@] Нейросеть. Обратно}

\spacing

Баночка повеселел "--- Нейросеть вернулась.
Насколько плант смог понять из её карканья, она вполне успешно вывела <<много>> птенцов, и они разлетелись.
Несмотря на то, что птица очень хорошо считала до пятидесяти, <<много>> могло означать двоих или четверых "--- Баночка уже заметил, что она то ли любит приврать, то ли подвержена парамнезиям.
Впрочем, произошло это, по её словам, <<очень давно>>, и распространяться на эту тему птица не считала нужным.

\spacing

Листик стала часто запираться в тор-отсеке.
Однажды она забыла включить звукоизоляцию, и я услышал, как она плачет и рассказывает кольцевой теплице о своих проблемах.
Все её друзья погибли на Тси-Ди, а с оставшимися в живых она почему-то не смогла подружиться.
Никто её не слушает, потому что она маленькая и тихо говорит, а может, просто не хочет слушать?
Реактивы кончились.
В джунглях на том берегу есть месторождение необходимых минералов, но одна она ехать боится, там страшные аборигены царрокх.
Ей хочется любви, но всё время уходит на работу.
Ей хочется домой, в её милый маленький домик на краю Восьмого города.
Ей хочется ещё раз, хотя бы раз погладить кошку Шипучку, которая притащила котят незадолго до Катаклизма.
Ей ужасно хочется любви\ldotst

\section{[-] Одиночество}

\epigraph{
\mulang{$0$}
{Человек не должен оставаться один на один с системой.}
{A man must not be ever left to fight alone against the system.}
}{Постулат Элект}

<<Сит, помоги мне в моих печалях>>.
С этими словами сегодня, на самом краешке рассвета я высыпал с колодец перо с золотом "--- впервые в жизни.

Я надеюсь, что вы никогда не ощутите то чувство одиночества, которое день за днём сковывало меня.
Мучительная декада после диверсии была полна надежды и ожидания;
новая волна мучений изо всех сил притворялась обычной жизнью "--- ждать было нечего.
Я начал избегать шумных сборищ и вообще скоплений народа.
Странная мысль "--- да, злоупотребляет положением Первый жрец, но кто первым поддержит его решение, если я оступлюсь?
Народ.
Те самые жители, основа основ, которым я отдал часы бесценного сна в достопамятную декаду.
Вселенная оказалась полна врагов;
они радостно кричали, смеялись и пели, и эти громкие звуки били меня по ушам, словно усаженные шипами пылеройские палицы.
Город кишел врагами, и едва ли хоть кто-то из них об этом догадывался "--- слепое орудие, нависшее надо мной.

Никто не мог мне помочь.
Кормилица была занята своими делами;
пища в храме стала просто отвратительной, и я какое-то время ходил кушать домой, пока в один прекрасный день на столе не оказался прокисший суп.
Я попытался готовить сам, но быстро понял, что перестану успевать с травами.

Лекарства расходились чересчур быстро.
Я не понимал, в чём дело.
На травы уходили все силы, я проводил в джунглях и на храмовых землях едва ли не всё время.
Тайна раскрылась на исходе дождей;
Саритр, новый жрец-врач, каждый вечер аккуратно сливал неиспользованные настойки у заднего входа.
Кому об этом сказать?
Кто в это поверит?

Чханэ отдалялась от меня всё больше.
Она проводила за воинскими занятиями всё время, а где она кушает и спит, я не знал.
В конце концов я, поймав её в коридоре, заявил, что нам больше не стоит быть вместе.
Подруга даже не стала спрашивать причину.

"--*Хорошо, "--- устало кивнула она и пошла дальше по своим делам.

Сегодня я достал старый походный мешок кормильца и долго смотрел на него.
Возможно, и в самом деле пришло время уйти.
Кормилица много рассказывала о море.
В молодости она держала торговый корабль, пока шторм в Могильном проливе не отправил его на дно.
У меня было чувство, что сейчас ни она, ни Храм не заметят моего ухода.
Все заняты.
У всех есть дела\ldotst
Меня уже не волновало, где и чем я буду жить.
Лишь бы убраться подальше от полного врагов города и прокисшего супа.

Впрочем, спустившись в зал, я передумал.
Манэ и Лимнэ мирно сидели и вырезали из цветной бумаги фонарики.
В сестрёнках не чувствовалось ни капли враждебности;
они не кричали и были готовы в любой момент уделить мне время "--- это я знал совершенно точно.

Мне вдруг ужасно захотелось сесть рядом, прижать сестрёнок к себе и провести так хотя бы одну вечность "--- пока не оттает промёрзший живот.
Я не успел даже проговорить желание про себя, как оказался между ними.
Руки Манэ и Лимнэ лежали у меня на солнечном сплетении.

"--*Не дрожи, братик, ты уже в тепле, "--- прошептала Манэ на ухо.

"--*В этом храме любой замёрзнет, "--- сказала Лимнэ, уткнувшись мне в шею.

Мало-помалу я успокоился.
В конце концов, поссориться с Первым "--- дело одной михнет.
Может, действительно стоит потерпеть с уходом "--- хотя бы до праздников?

\section{[@] Традиция}

Когда-то давно на Тси-Ди существовало такое понятие, как праздники.
В отличие от обычных выходных, праздники и фестивали были привязаны к году или периоду в несколько лет, а также имели особую тематическую окраску;
тси выходили на улицы, собирались вместе и устраивали шумное весёлое представление.
Около пяти тысяч лет назад традиция была упразднена и заменена новой "--- Театром Тысячи Положений.

Театр не был привязан к конкретной дате.
Его начало высчитывалось по общепланетному коэффициенту усталости "--- тысячи механизмов непрерывно анализировали деятельность тси и объявляли Театр, как только ОКУ превышал определённое значение (к моему стыду, и порядок вычисления, и пороговое значение я напрочь позабыл).
За два дня все тси обсуждали на форумах тематику и правила нового Театра;
надо заметить, на моей памяти они ни разу не повторились.
Последний Театр, который мы застали на Тси-Ди, был посвящён мужскому полу с позиции культуры "--- тси всех полов оделись в исторические <<мужские>> костюмы и отыграли всех мужских персонажей литературы и кино, каких только смогли вспомнить.
Мне не составило труда вжиться в роль диверсанта Раскалённого, мужчины-планта из романа-антиутопии <<Чёрная дверь>>, а вот кроха Листик повергла всех в дикий хохот, когда попыталась отыграть древнего жестокого полководца Рамеса.

Самым забавным было то, что никто, кроме узкого круга техников, не знал, сколько продлится Театр Тысячи Положений.
Он мог продлиться и день, и восемь дней, и даже больше.
Поэтому тси старались отдыхать с полной отдачей.
Самый долгий Театр за историю продлился четверть года;
ошибка в системе обнаружилась уже на исходе срока, когда трудяги сами вернулись к работе и даже самые отчаянные гуляки начали подозревать неладное.

Я чувствовал, что Театр необходимо устроить.
Однако когда?
На Тси-Ди этим ведала Машина, не могу же я назначить дату с бухты-барахты\ldotsq

"--*Можешь, "--- заявила Кошка, когда я решил посоветоваться с ней.
"--- Именно с бухты-барахты.
Я тоже понятия не имею, как вычислялась дата, но без Театра мы жить не можем.

И я назначил мероприятие на послезавтра, чтобы дать тси время на обсуждение тематики и правил.

\section{[-] Ранние Слёзы}

В тот год дожди зарядили возмутительно рано "--- аж на пять дней против обычного.
Первый ночной дождик, который, согласно традиции, считался Слезами Ситхэ, пошёл на Змею 3, прямо перед праздником Мягких Рук.
Жрецы, посовещавшись (разумеется, без меня и Трукхвала), вышли к народу и объявили своё решение "--- детей следовало порадовать на Змею 4, а влюблённые, мол, денёк подождут со своим праздником.
Никто, впрочем, и не возражал.

Город в одно мгновение вырядился в гирлянды, бумажные фонарики и цветочные венки.
Обычно кормилица велела домашним готовить украшения и одежды заранее, дней эдак за пять.
В противном случае, уверяла она, украсить дом не хватит времени.
Только сейчас я понял всю глубину женского коварства "--- на подготовку было достаточно нескольких кхамит, но она служила отличным поводом ничего не делать в предпраздники.

Починить разрушенную стену храма и пол зала строители не успели, но повели себя как настоящие друзья "--- гирлянды, фонарики и цветочные венки висели даже на строительных лесах, превратив храм в какой-то чудесный дворец.
Вначале все избегали огромной дыры, но украшения сделали своё дело "--- вскоре половина Храма (разумеется, Нижняя) весело прыгала через неё туда-сюда.

"--*Может, сказать строителям, чтобы так и оставили? "--- задумчиво сказала Чханэ.
"--- Лестница скучновата.

"--*Ага, "--- с энтузиазмом подхватила Ликхэ.
"--- В Тхартхаахитре есть Трёхэтажный храм, в Миситре "--- Винтовой\footnote
{Более точный перевод "--- Праворезьбовой храм, из-за специфической формы "--- храм выглядит как винтовая башня.
Но практически на все языки Тра-Ренкхаля название храма Миситра переводят как Винтовой. \authornote},
в Сотроне "--- Плавучий, а в Тхитроне будет Дырчатый.
\mulang{$0$}
{Единственный в землях сели Дырчатый храм.}
{The only Holey temple in all the S\r{e}l\={\i} lands.''}

"--*До первого землетрясения, "--- ввернула Кхохо.
"--- А то мало трупов было за последние дожди.

"--*Да, как-то я не подумала, "--- согласилась Чханэ и похлопала Ликхэ по плечу.
\mulang{$0$}
{"--- Не переживай, Орешек.}
{``Don't be upset, Nut-Nut.}
\mulang{$0$}
{Дырчатым можно обозвать и вполне целый храм.}
{Even a restored temple can be called `Holey'.''}

Храму внеплановый отдых пришёлся очень кстати.
Старые воины (я не мог избавиться от привычки называть их именно так) наконец-то собрались вместе за обедом и обсудили последние события.

"--*Кто на этот раз будет переодеваться на Слёзы Ситхэ? "--- спросил Ситрис.
"--- Кхохо?

"--*Опять? "--- фыркнула воительница.
"--- Я в прошлый раз в этом костюме чуть не грохнулась, и детишки узнали столько новых слов, что им на всю жизнь хватит.
\mulang{$0$}
{Пусть кто-нибудь из\ldotst жрецов переодевается.}
{Let an\dots a priest change.}
\mulang{$0$}
{О, Ликхмас, давай ты.}
{\^{O}, L\={\i}kchm\r{a}s, you.''}

\mulang{$0$}
{Кхохо едва не сказала <<старых жрецов>>.}
{Kch\`{o}h\^{o} was so close to say `an old priest'.}
\mulang{$0$}
{Или у меня просто разыгралось воображение?}
{Maybe I was just imagining things?}

\mulang{$0$}
{"--*Молодой чересчур, "--- покачал головой Эрликх.}
{``Too young,'' O\r{e}rl\'{\i}kch shook his head.}
\mulang{$0$}
{"--- Трукхвала нарядите.}
{``Tr\`{u}kchu\r{a}l is more fit.''}

\mulang{$0$}
{Нет, это было не моё воображение.}
{No, it wasn't an overactive imagination.}
\mulang{$0$}
{Эрликх понял Кхохо так же, как я.}
{O\r{e}rl\'{\i}kch made the same connection I did.}

"--*У него нога разболелась, "--- сказал я, ощущая прилив дружеской теплоты в животе и осознав, насколько я отвык от этого чувства.
"--- Давайте я попробую, только потренируюсь голос постарше делать.

"--*Вот и решили, "--- Кхохо с облегчением начала поглощать похлёбку.
"--- И вообще, Слёзы Ситхэ "--- это ритуал, а ритуал должны проводить жрецы.
Почему последние пятнадцать лет наряжаются только воины?
Как будто у нас больше нет\ldotst

\mulang{$0$}
{"--*Кхохо, заткнись, "--- посоветовал Ситрис.}
{``Kch\`{o}h\^{o}, shut up,'' S\~{\i}tr\v{\i}s adviced.}
\mulang{$0$}
{"--- Ты думаешь, \emph{эти} пойдут?}
{``You don't think \emph{they} will?''}

Воины промолчали "--- слово прозвучало хуже всех витиеватых грязных ругательств, которые я когда-либо слышал от Кхохо.
Женщина смутилась и налегла на еду с удвоенным рвением.

"--*А кто мешок потащит? "--- вдруг осведомилась Ликхэ.
Все как один посмотрели на Чханэ.

"--*Мне не сложно, "--- пожала плечами подруга.

"--*Учись, "--- наставительно обратился Ситрис к Кхохо.
"--- Молодёжь сознательнее тебя.

"--*Да ради всех духов, "--- махнула рукой воительница.
"--- Пусть хоть на празднике вместе погуляют.

Лицо Чханэ приобрело яркий грейпфрутовый цвет.
Я тоже почувствовал, что краснею.
Ситрис вдруг ни с того ни с сего обнял Чханэ, поцеловал её в лоб и встал из-за стола.

"--*Занесу Трукхвалу миску супа, что ли, "--- рассеянно сообщил он, направляясь на кухню.

\section{[-] Начало Рук}

Влюблённые не дотерпели.
Вердикт вердиктом, но народ сам решает, когда устраивать праздники.
На столах осталось множество яств, которые потеряли бы свежесть уже через несколько кхамит, а крестьяне "--- практичный народ.
Не успели мы провести обряд по Слезам Ситхэ, как по всей площади начали разгораться совсем другие костры.

Кхарас хмуро молчал.
Он вообще не любил, когда дела шли не по плану, это выбивало его из колеи.
Новые жрецы тоже оставались в стороне.
Вся организация неожиданно свалилась на меня.

"--*Ликхмас, а кто в этот раз Круг начнёт?
Ты?

"--*Ликхмас, как сегодня с погодой?
Зонты на всякий случай тащить?

"--*Ликхмас, а кто за столы возле гонга отвечает?
Можно, мы у них немного еды к нашим заберём?

"--*Ликхмааас!
Я палец в столе защемила!

"--*Кхохо!
Иди в бездну со своим пальцем! "--- не выдержав, заорал я.

Кхохо обиженно заморгала.
Спустя михнет я, перебирая самые витиеватые ругательства, вытаскивал защемлённый палец с помощью молотка и стамески.
Ситрис и Эрликх бились в истерике и даже не думали помогать.

"--* <<Мы связаны, судьбу не изменить>>, "--- процитировал Эрликх <<Легенду об обретении>>, и оба зашлись в новом приступе судорожного смеха.

"--*Лиc, незачем бегать самому, "--- сказала наконец Чханэ.
"--- Назначь ответственных, а там как пойдёт.

Пошло очень даже неплохо.
Стамеска пролетела пять шагов и вонзилась в пяди от уха Ситриса.

"--*Быстро взял стамеску и помог!

"--*Тихо-тихо, Ликхмас, сейчас сделаю, не кипятись!

"--*Столбик, на тебе столы у гонга.
Так, народ, Ликас и Митхэ отвечают за костры, все вопросы к ним.
Ликхэ, Кхотрис-лехэ, наберите народ, чтобы гонг передвинули, мешает\ldotst
Кхохо, палец болит?
Отлично, будешь думать в следующий раз.
Травы для Круга на тебе.

Вскоре ко мне подошла Кхотлам.

"--*Разобрался с организацией?

"--*Ну, если можно так выразиться, "--- я пожал плечами, оглядев кипевшую вокруг слаженную работу.
"--- Раздал роли "--- и все вроде как справляются без меня.

"--*Мне этим и нравится работа купца, "--- призналась кормилица.
"--- Чтобы всё заработало правильно, нужно знать, что делаешь, и думать, что кому говоришь.
Зато потом, когда всё уляжется, просто сиди и занимайся своими делами.

До меня вдруг дошло.

"--*А чего все они ко мне-то пошли? "--- повернулся я к Кхотлам.
"--- Ладно Круг, но за столы-то я каким боком отвечать должен?

Кхотлам повела бровями.

"--*Ну, я отправила Храму официальный вопрос, почему они не занимаются организацией Мягких Рук.
Пришёл расстроенный Эрликх и на словах объяснил, что Верхний Этаж сидит по кельям и занимается своими делами, а моё письмо мне вернули, посоветовав съесть его с острым соусом.
Учитывая, что сам Эрликх никогда не опустился бы до таких выражений, я ему поверила.
Ну и раз мы опять остались без жрецов, я сказала старшинам, что ты главный.
Видишь "--- и развеялся, и полезный опыт получил.
Вдруг как-нибудь ночью пригодится.

Кхотлам тут же вытащила вязание и принялась набирать петли.

\spacing

\section{[-] Искристая зелень}

\spacing

Свою праздничную одежду я, как и все, вышивал сам.
На неё ушло немало золота "--- я выбрал чёрный, как ночь, шёлк и безумно дорогие нитки <<искрящаяся зелень>>, которые привозили с Запада.
Первые цепочки стежков я сделал мотком, взятым из шкатулки кормилицы;
у любой кормилицы, в любом доме всегда есть такая шкатулка, в которой мелкие памятные вещицы с разных концов света лежат вперемежку с нитками, застёжками, гвоздиками и пришедшими в негодность заколками.
Когда Кхотлам увидела, то схватилась за голову:

"--*Дитя, ты с ума сошёл!
Ты знаешь, сколько они стоят?
Да ещё объёмным швом, ты год потратишь!

Однако отступать было поздно.
Я действительно потратил год, причём не столько на вышивку, сколько на заработок золота.
Впрочем, жалеть о выборе не пришлось "--- в первый же праздник я оказался в центре внимания.

"--*Сам вышил? "--- не переставал удивляться Кхатрим.
Он щупал и щупал мою одежду, пока Первый не оттащил его за шиворот.

"--*<<Искристая зелень>>?
Ты дурень, нашёл на что золото спускать, "--- резюмировал Столбик.
"--- А если порвут в толпе?

"--*Хаяй, это же домики!
Домики, заплетённые лианами! "--- Хитрам прыгала вокруг и радовалась как ребёнок.

Справедливости ради замечу, что свою цену нитки оправдали "--- штопать праздничную одежду мне ни разу не пришлось.

На ужин все спустились уже в праздничной одежде: Ситрис "--- в оливковой, Ликхэ "--- в переливчато-синей, а Кхарас "--- в серой.
Чханэ надела зелёное платье прародительницы, объяснив, что свою праздничную одежду забыла дома.
Кхохо же вышла из общих комнат позже всех;
она завернулась в красное, как кровь, шёлковое полотнище прямо на голое тело.
Кхарас заворчал:

\mulang{$0$}
{"--*Кхохо, сколько можно! Как будто девочка двадцати пяти дождей!}
{``Kch\`oh\^o, how long can you dress like a twenty-five-rain-old girl!''}

"--*В этот день полагается пить, танцевать и совокупляться, "--- огрызнулась Кхохо.
"--- Храм должен возглавлять, а не плестись в арьергарде!

"--*Не волнуйся, Кхарас, всё продумано, "--- успокоил вождя Ситрис.
"--- Ей просто лень стирать штаны и рубаху после Мягких Рук.

Я, Чханэ и Ликхэ переглянулись и дружно пошли переодеваться.

"--*Лис, слушай, платье без штанов смотрится не очень странно? "--- спросила Чханэ.

"--*Мне так даже больше нравится, "--- заверил я её.
"--- Говорят, ноа ходят в таких платьях без штанов.

Ликхэ уже вознамерилась стянуть с себя передник, но задумалась и оставила, посетовав в очередной раз на своё тело:

"--*Ну вот что с ними делать, а?

"--*Да ну, это здорово, когда грудь прыгает! "--- засмеялась Чханэ.
"--- У меня вот не прыгает.

"--*Я бы с удовольствием поменялась, поверь, "--- буркнула Ликхэ.

\section{[-] Тихая ненависть}

\spacing

Первый жрец, проходя мимо, поймал меня за пояс.
Я едва подавил желание ударить его в горло.

"--*Куда это ты собрался, Ликхмас ар'Люм?

"--*Праздновать, "--- бодро доложил я.

"--*Кажется, я не давал разрешение покидать пост.
Ты забыл о долге жреца?

"--*У меня клятва вместо завтрака, досуга и сна, "--- сообщил я.
"--- Лучше напомни той рыбине, которая поливает моими зельями сорняки, что сила не покидает микстуру с закатом солнца.

Я почувствовал, как мои ступни оторвались от каменного пола;
лицо Первого вдруг оказалось в двух пальцах от моего.

"--*Ты можешь меня ненавидеть, Ликхмас ар'Люм, но ты будешь делать это тихо и покорно, "--- прошелестел тихий голос, и у меня вдруг вспотели ладони.
Казалось, что жрец даже не открыл рта.
"--- В моей власти отправить тебя на алтарь прямо сейчас и делать с тобой на алтаре всё, что угодно.
Да, и это тоже.
Ты меня понял?

"--*Вполне, "--- ответил я.
Язык чесался ответить по-другому, но иной ответ был неуместен, как крик мангуста в змеином гнезде.
Кинжал уже давно ждал команды в складках одежды;
в голове даже промелькнула пара здравых идей, где можно надёжно спрятать тело и как быстро собрать вещи.

Первый безусловно умел наводить страх, даже на таких, как Кхохо;
однако для меня страх был знакомым чувством, и я ощутил в нём некоторую фальшь.
Он был коротким, мощным и беспричинным "--- словно мимо пробежал испуганный бык, обдав лёгким ветерком.
Мгновение спустя я расслабился и сосредоточился на свободной руке Первого, которая легкомысленно лежала в кармане.
Может, мне и страшно, но хозяина положения не должны заботить такие мелочи.

<<Да, и это тоже>>.
Простейшая уловка, иллюзия чтения мыслей, которая способна ввести в заблуждение разве что дикаря хака, и то не самого умного.
Я почувствовал, что прихожу в ярость.
Да что этот рыбий хребет о себе возомнил?

\mulang{$0$}
{"--*У твоей поясницы "--- лаковый сок, дурак, "--- снова прошелестел голос.}
{``You got l\={a}\"{a}k\^{a} sap to your back, fool,'' the voice rustled.}
\mulang{$0$}
{"--- Перестань теребить нож.}
{``Stop strumming your knife.''}

\mulang{$0$}
{"--*У твоей поясницы "--- закон сели и мой, \emph{пока ещё мой} город, "--- прошипел я в ответ.}
{``You got S\r{e}l\={\i} law and my, \emph{still my} city to your back,'' I retorted.}
\mulang{$0$}
{"--- А вот в руке пусто.}
{``And nothing in your hand.''}

Первый захохотал.
Мои ступни снова коснулись пола;
фальшивый страх исчез, и я вдруг осознал, что ярость тоже была фальшивой.

"--*Так значит, эта требуха не обманула.
Захожу я в город.
Спрашиваю на посту "--- кто главный?
Лохматая дура с саблей, разлепив глаза, отвечает "--- не особенно вежливо, "--- что Ликхмас в храме.
Спрашиваю в храме, нелюдимая гора мышц говорит "--- Ликхмас в школе, подождите здесь.
Спрашиваю в школе, смазливый учителишка уточняет "--- Ликхмас спит, спросите Трукхвала в библиотеке.

Первый скривился.

"--*<<Ликхмас спит>>, <<подождите здесь>>, "--- повторил он, гротескно подражая Митликху и Кхарасу.
"--- Гляньте на них.
Вот она "--- незримая иерархия поселений, которая смеётся над любыми указами Короля-жреца и волей захватчиков.
Не ожидал, что нынешним лидером Храма Тхитрона окажется мальчишка, но так или иначе ситуация окончательно прояснилась.
Распустил ты свой Храм, Ликхмас ар'Люм.
На постах беспорядок, с гостями разговаривать не умеют.

"--*Лучшего Храма ты не найдёшь на всей Короне, "--- процедил я.

"--*Мне нет надобности \emph{искать} Храм, я способен любой мусор превратить в работоспособный механизм.
В этом наше с тобой отличие.
Так или иначе, мне \emph{придётся} с тобой договориться.
Управленец из тебя неважный, но мы же \emph{равные}, не так ли?
Например, я доверю тебе школу.
Что скажешь?

Я промолчал.
Способность собеседника играть чужими эмоциями меня немало впечатлила;
резко стал понятен слепой ужас импульсивной Кхохо.
К тому же вмешался прошлый опыт: если Первый жрец резко меняет тему "--- значит, дело дрянь.

Совсем некстати всплыл в памяти отрывок разговора с Чханэ: <<Из катакомб тайком выносили тела, а наутро\ldotst>>.
Я понял "--- пора бежать.

"--*Соглашайся.
Митликх ар'Митр не пожелал остаться.
Я даю тебе работу по твоим талантам и умениям, а ты не попадаешься мне на глаза и, главное, держишь свои таланты и умения подальше от моих дел, "--- жрец наклонился к моему лицу, и я почувствовал его почти неощутимое дыхание.
"--- Ты понял мысль?

"--*Вполне.
Я согласен.

"--*Через рассвет ты будешь в храме, Ликхмас ар'Люм, "--- с этими словами синяя роба, трепеща, исчезла за углом.

<<Через рассвет я буду в сельве>>, "--- подумал я.
Замечательное окончание знакомства "--- оба уверены, что победили.

Проходя мимо зеркала, я поправил съехавшее шёлковое полотнище, подтянул пояс и задумчиво посмотрел на заплетённые лианами домики.
За прошедшие дожди только один человек заметил, что домики и лианы на моей праздничной одежде соответствовали городам и путям обитаемой Короны.
Даже кормилица, которая смогла бы сплести чётки Сата во сне, не обратила внимания.

"--*Учитель Ликхмас, зачем ты надел карту? "--- шёпотом спросила Митхэ ар'Тра, встретив меня на прошлом празднике Большого Похмелья.
"--- Это какой-то обряд?

Случай стал для меня уроком "--- люди редко видят то, чего не ожидают увидеть.
А всего-то и нужно было "--- вышить города всех народов одним цветом.
Интересно, как отнеслась бы Чханэ к идее посетить все вышитые на одежде поселения?
Кристалл, полный мира и спокойствия, вполне может подождать.

\section{[-] Тотем Сомнения}

\epigraph
{Сомнения "--- единственное, что избавило мир от ужасов моего гения.}
{Эпитафия на безымянной могиле.
Захоронение Аквамарин, планета Тысяча Башен}

\spacing

"--*Митхэ, что случилось?

"--*Кормилица поругала.

"--*Что она сказала тебе?

Митхэ всхлипнула.

"--*В честь праздника нужно было сделать подношение духам.
Кормилица велела мне тоже.
Я спросила <<Зачем?
Я справляюсь и без духов>>.
Она начала говорить, что так нельзя, что во мне нет веры, и что мне будет трудно жить, потому что мне не будут помогать духи.
Ещё она сказала, что и дружба, и жизнь с мужчиной требуют веры.
И я заплакала.

"--*Ты заплакала от того, что тебе будет трудно?

"--*Да.
А ещё от того, что я не могу верить.
Как можно верить, если ты знаешь или хочешь знать?

Я обнял ученицу.
Похоже, наступил тот самый момент, о котором предупреждал меня Трукхвал "--- <<когда учитель становится учителем>>.

"--*Митхэ, "--- начал я.
"--- Твоя кормилица пожила на земле и знает, как ей следует жить, чтобы процветать.
Но она "--- не ты.
Ты особенная.
Кто-то считает это поводом для гордости, но для меня особенность "--- это в первую очередь напоминание: твой путь к счастью будет не таким, как у прочих.

Я помолчал, собираясь с мыслями.
Митхэ посмотрела на меня.

\mulang{$0$}
{"--*То есть она не знает, что лучше для меня?}
{``In short, she doesn't know what's good for me, does she?''}

\mulang{$0$}
{Я смешался.}
{I got confused.}
\mulang{$0$}
{Ребёнок в который раз ставил меня в неудобное положение своей прямотой.}
{Child's directness put me on the spot, yet again.}

\mulang{$0$}
{"--*Никто не знает.}
{``No one knows.}
\mulang{$0$}
{Единственное существо, которое может это узнать "--- ты сама.}
{You are the only being who can find it out.''}

"--*Я думала об этом, "--- призналась девочка, "--- и мне стало ещё обиднее, что меня ругают ни за что.

"--*Митхэ, сомнение "--- это величайший дар для того, кто наделён острым умом.
Где бы мы были, если бы не подвергали сомнению услышанное и увиденное?
Кто-то ошибается, кто-то намеренно пытается тебя обмануть, а есть ещё твои собственные чувства, которые имеют свои границы.
Всё, что за границей чувств, досягаемо лишь умом.
Всё, что на границе чувств, может быть искажено, как то, что видно на самом краю линзы.
Есть ещё твоё собственное воображение, которое необходимо отграничить от чувств.
Ум "--- твой инструмент, но без сомнения этот инструмент бесполезен.

"--*Но ведь кормилица права?
Мне будет трудно?

"--*Искать путь "--- трудная задача, "--- сказал я.
"--- Но я тебя прошу "--- в минуты тяжёлого выбора, в облаке неопределённости, в момент самой противной нерешительности помни, что, возможно, ты проявляешь лучшее качество, которое у тебя есть.

И вдруг мне в голову пришла идея.

"--*Пойдём.

Мы прошли в небольшую саговую рощицу на воинских землях.
Там из земли торчал большой деревянный столб.

Я вытащил из сумки маленький кукхватровый резец и аккуратно вырезал на податливой поверхности столба знак вопроса.
Митхэ с интересом следила за мной.

"--*Вот, "--- сказал я, отряхнув с ладоней деревянные крошки.
"--- Это тотем Сомнения.

\mulang{$0$}
{"--*Зачем ты его сделал?}
{``Why have you built it?''}

\mulang{$0$}
{"--*Понятия не имею, "--- признался я.}
{``No idea,'' I admitted.}
\mulang{$0$}
{"--- Возможно, это нужно нам с тобой.}
{``Maybe both you and I need it.''}

\mulang{$0$}
{"--*Ты уверен, что это нам нужно?}
{``Are you sure we need it?''}

\mulang{$0$}
{"--*Нет.}
{``I'm not.}
\mulang{$0$}
{А ты?}
{You?''}

\mulang{$0$}
{"--*И я не уверена.}
{``Me neither.''}

Я почесал голову.

"--*В таком случае могу сделать предположение, что тотем Сомнения действительно работает.
Но точно сказать не могу.

Митхэ вдруг ахнула и вытаращила глаза "--- она поняла.
Затем мы оба расхохотались.

"--*Возможно, туда стоит поставить тарелочку и раскрасить тотем красками, "--- сказал я на обратном пути.

"--*Я не уверена, что это нужно, но, вероятно, я это сделаю, "--- со смехом отвечала ученица.

"--*Возможно, я тебе даже помогу, "--- заключил я, "--- но это не точно.
А теперь пойдём праздновать.

\section{[*] Дары}

\spacing

"--*Поэты часто говорят, что любовь спасает, но не упоминают, сколько людей погибло из-за неё.
И сколько отвратительных вещей было совершено из-за любви.
Я бы лучше доверила свою жизнь мечтателю или человеку, живущему расчётами, чем влюблённому.

"--*А Атрис "--- мечтатель?

"--*По-моему, это очевидно, "--- засмеялась Митхэ.

"--*Знаешь, я вообще немного побаиваюсь людей, подверженных чувствам, "--- призналась Эрхэ.
"--- Тех, кто способен на страстную влюблённость, на дикие обиды.
Человек эмоций не принадлежит сам себе.
Сейчас он страстно целует тебя и клянётся в любви, михнет спустя раскроит тебе череп, а ещё через михнет будет плакать над твоим трупом.
Я-то уж знаю, сама такая порой.
Поэтому Снежок меня и боится\ldotst

Митхэ погладила колено подруги.
Эрхэ благодарно сжала её ладонь.

"--*Но я всегда знала, где грань.
А бывают люди, которым чувства служат и принципом, и моралью, а при случае "--- и оправданием.
Плохо то, что их сложно предсказать, и хуже всего, если такой человек ещё и скрывает свою бурю за каменным лицом.
Вот как этот.
У него лицо, словно каменная маска, у которой в глазницах развели костры.

"--*Ты права, "--- Митхэ бросила взгляд на жреца.
"--- Наверное, меня это и напрягало.
Так что в бездну и его уважение, и его любовь, голова мне дороже.
Выпьем?

"--*За что будем пить?

"--*За тебя, конечно.
И за твою любовь.

"--*Послушай, "--- сморщилась Эрхэ, "--- я, может, и выгляжу жутко влюблённой, но Акхсар "--- не единственное, что интересует меня в этой жизни.
Так что давай лучше выпьем за кое-что другое.
За дорогу.

"--*Давай, "--- Митхэ схватила оказавшийся рядом кувшин и наполнила две чаши.
"--- Почему за дорогу?

"--*Потому что каждый кхене приносит нам частицу того, что делает нас счастливыми, что заставляет разум и тело петь в унисон с природой.

"--*Ты говоришь, как Атрис.

"--*Это его слова, "--- ухмыльнулась Эрхэ.
"--- И он прав.
Не будь Снежка "--- я была бы так же счастлива, ведь источник счастья "--- в моей груди.
Но Снежок есть, его принесла мне дорога.
И сегодня я буду наслаждаться её даром до самого утра.

"--*За дорогу, "--- улыбнулась Митхэ.

"--*За дорогу, "--- Эрхэ лихо ударила чашей о чашу и, поднеся вино к носу, добавила:
"--- Но меня всё ещё тошнит от вина.
Что же делать?
Ведь это тоже дар?

"--*Вспомни мой обычный совет перед походом.

"--*<<Бери не более необходимого>>, "--- улыбнулась Эрхэ.
Вино тонкой струйкой вытекло из наклонившейся чаши и исчезло между плитами пола.

\section{[-] Танец и мёд}

\spacing

"--*Я давно не танцевал в паре, "--- шёпотом признался я.

"--*Это просто, "--- зашептала в ответ Чханэ.
"--- Стиль Тени ещё помнишь?
Бери меня за руки и сделай <<зеркальное отражение>>.

Я послушался.

"--*Молодец, теперь быстрее.
Ещё быстрее, ритм лови.
Молодчина.
Правый переход для двух сабель, змеёй.
Лис, руку не отпускай, это танец.
Теперь левый подхват для сабли и ножа, кроликом.
Подножку не обязательно, лучше вытяни носочек вот так, изящнее.

"--*Тебя так танцевать учили? "--- поморщился я.

"--*У нас в Храме по-другому и не учат.
Что танец, что спарринг "--- одно и то же.
В ритм старайся, а не на скорость, и изящнее, легче двигайся.
Молодчина.
Всё, классический парный ты умеешь.
Смотри, мы даже лучше всех\ldotst почти.

Я ухмыльнулся.
Ситрис и Кхохо определённо были жемчужинами площади.
Они танцевали какой-то очень сложный и быстрый южный танец.
Далеко не худенькая Кхохо вращалась волчком, обхватив ногами колено Ситриса.
Оба раскраснелись от скорости и удовольствия.

Чуть поодаль неклюже топтались Кхарас и Эрликх.
Ликхэ кружилась вальяжно, смакуя каждое движение;
её головка лежала на груди старшины кожевников.

Я уже не думал о том, куда ставить ноги.
Чханэ смотрела на меня, и на её лице, под лёгкой тенью венка цвела довольная улыбка.

"--*Не стоит быть вместе, думаешь? "--- прошептала она одними губами.

"--*Наверное, я погорячился, "--- улыбнулся я.

"--*Впрочем, ничего нового, "--- вздохнула подруга.

"--*Вообще ничего, "--- согласился я.
"--- Кстати, я что-то вспомнил первый раз, когда ты меня обняла.

"--*Это который? "--- наморщила лоб девушка.

"--*А который в лесу, "--- ухмыльнулся я и мягко увёл Чханэ в сторону, избежав неминуемого столкновения с влюблённой парой.
"--- Я аж от избытка чувств ножом махнул.

"--*Помню, помню.
Кстати, я в тебя тогда и влюбилась.
Меня никто до этого так нежно не бинтовал.

"--*А изображала из себя невесть что, "--- упрекнул я подругу.

"--*А ты думаешь, влюбиться легко?
Это всё равно что задницу в улей засунуть, проверяя, есть там пчёлы или нет.
Даже если не ужалят "--- мёд всё равно со странным привкусом.

"--*Трааа\ldotst "--- я поперхнулся от такого сравнения.
"--- Это у вас так на Западе говорят?

"--*Это Кхохо так сказала.
Подозреваю, что умозаключение основано на личном опыте.

"--*Меняемся! "--- крикнул кормилец и, постучав в бубен, отдал его кому-то из толпы.
Кхотлам сидела в уголке и, улыбаясь, вязала очередную занавесочку;
кормилец попытался аккуратно забрать плетение из её рук и вытянуть Кхотлам на танец, за что получил лёгкий удар спицами по лбу.
Впрочем, вскоре плетение осталось сиротливо лежать на скамеечке, а кормильцы исчезли где-то в пёстрой праздничной толпе.

\mulang{$0$}
{"--*Ликхмас, ты не против? "--- подошёл улыбающийся Ситрис.}
{``L\={\i}kchm\r{a}s, would you mind?'' smiling S\~{\i}tr\v{\i}s came up to us.}

\mulang{$0$}
{"--*Нет, конечно, "--- я улыбнулся в ответ и протянул ему руку.}
{``Of course I would't,'' I smiled back and gave him my hand.}

\mulang{$0$}
{"--*Ты мне его только живым верни, "--- захихикала Чханэ.}
{``Just return him alive,'' Chh\r{a}n\^{e}i chuckled.}
\mulang{$0$}
{"--- Кхохо вон уже лежит.}
{``Kch\`{o}h\^{o}'s lying down over there.''}

\mulang{$0$}
{"--*Кхохо перебрала, "--- пояснил Ситрис.}
{``Kch\`{o}h\^{o}'s too much to drink,'' S\~{\i}tr\v{\i}s explained.}
"--- Под вином Кхараса тянет поговорить, Эрликха "--- на сон, а Кхохо танцует и насилует всех, кто окажется на расстоянии вытянутой руки.
Чханэ, к тебе партнёр.

К нам подошёл красавец-кожевник, который когда-то ухаживал за Чханэ, и мягко увёл её на танец.
Ликхэ, которая уже хотела пригласить меня, вяло откликнулась на приглашение Столбика и ушла с ним, бросая на Ситриса злобные взгляды.

"--*Только не крути меня, а то вино обратно польётся, "--- предупредил я.

"--*Даже желания особого нет, "--- ухмыльнулся воин.
"--- Давай помедленнее и понежнее\ldotst

\section{[-] Учитель на празднике}

\spacing

Трукхвал хмуро сидел на больничных носилках и пил принесённый Кхотлам горячий отвар.
Попытка вытащить его из библиотеки была похожа на представление.

"--*Учитель, просто посиди на скамеечке.
Можешь полежать, если хочешь.

"--*Ликхмас, нет.

"--*Трукхвал, ты здесь помереть решил, в этом склепе? "--- поинтересовался Ситрис.

"--*Ситрис, иди к свиньям!

"--*Тебя никто не будет трахать! "--- развела руками Кхохо.
"--- Даже я!

"--*Кхохо, отстань!

"--*Ликхмас, левую руку привяжи к носилкам, "--- командовал Ситрис.
"--- Вот так.
Да, для хромого старика он ого-го в хорошей форме\ldotst
Так-так, осторожнее здесь, а то вы его опять башкой о косяк приложите.
Нет, Кхохо, кляп не нужен.
Да всё нормально, он как только за воротами очутится, сразу прекратит орать и дёргаться "--- для него ж репутация превыше всего.

Спустя кхамит Трукхвал уже пересел с носилок на тростниковое кресло и весело болтал с облепившей его молодёжью.

"--* \dots а главное в летающих машинках "--- винт.
Смотрите, если кусочек дерева вырезать вот так и покрутить, то он будет равномерно загребать воздух, подобно вёслам\ldotst

Мы встретились взглядами.
Я улыбнулся.

<<Видишь, теперь и я могу тебя кое-чему научить>>, "--- знаками показал я ему.

<<Всегда>>, "--- ответил мне Трукхвал и снова принялся рассказывать про винт.
Что он имел в виду "--- так и осталось загадкой.

Празднество шло своим чередом, но я ни на секунду не мог забыть про нависшее надо мной будущее.
То тут, то там я натыкался на пустые, чуждые взгляды новых храмовников;
они попадались в праздничной толпе, словно камни в мягком пироге.
Учитель выглядел каким-то радостно-беззащитным ребёнком на их фоне\ldotst

Я не мог оставить его на съедение этим стервятникам.
Улучив момент, я оттащил Трукхвала от молодёжи и рассказал всё как есть.
Он не удивился.

"--*А ты думал, я просто так сижу в библиотеке и никуда не выхожу?

"--*Но это же не решение, учитель!

"--*Молодые ноги быстро бегают, Ликхмас.
А я стар и хром.

"--*Я теперь уже не уверен, что принимаю правильное решение, спасаясь бегством.
Может, мне следовало\ldotsq

"--*Нет, Ликхмас, силы неравны.
Рано или поздно \emph{он} тебя убьёт, и ему ничего за это не будет.
Тебя не защитят ни старый Храм, ни Кхотлам.
И ещё\ldotst раньше я, хай, сомневался, но после твоего рассказа уверен: тебя он намерен именно убить, а не выслать.
Он не человек, он стервятник, закалённый в интригах и заговорах.
Он не будет рисковать, оставляя в живых опасного соперника.

"--*Я его не настолько боюсь, учитель.

"--*А следовало бы!
Более того, если он убьёт тебя, то и твою кормилицу, возможно, что и твоих домашних тоже.
Он ведь знает, что они будут мстить.
Кхотлам город перебаламутит, если с тобой что-то случится!
Да даже если она по каким-то причинам и решит сохранять нейтралитет, Манэ, Лимнэ, Хитрам и ваши слуги молчать не будут.
Особенно Сиртху "--- ему-то вообще в этой жизни терять нечего, а ты у него на руках вырос.
Всё равно пойдут слухи, народ возьмётся за оружие.
Думаешь, люди не помнят, как ты вернулся из земель хака, как декаду метался между алтарём и школой, как ты засыпал с хлебом во рту?
Такое не забывают.
А слухи "--- они всегда страшнее правды, в твоей смерти обвинят не только жрецов, но и Кхараса, и Ситриса.
Ты представляешь, что начнётся?
Так что беги, уклонись от боя "--- это будет наилучшим раскладом для всех.

"--*Как он сможет убить сразу меня и всех обитателей Двора, не вызвав подозрений?

"--*Да по тому же сценарию!
Кусачая бумага и диверсант!

"--*Ты с ума сошел?

Трукхвал замолчал и насупился.

"--*Ты\ldotst ты серьёзно?
Учитель!

"--*Идолы здесь точно ни при чём, "--- промямлил Трукхвал.
"--- Я послал письмо тому караванщику.
Коробки нам уже в Миситре отгрузили с нужным весом, и он ни на фасолину не изменился.
Они проверяли, в учётных книгах есть все записи.

"--*Почему ты молчал?

"--*А кто мне поверит? "--- в глазах учителя застыли слёзы, его подбородок дрожал.

Мне захотелось кричать.
<<Ты понимаешь, что это была решающая улика?
Ты понимаешь, что, если бы ты сразу сказал мне или Кхотлам, мы могли проследить происхождение коробок и выявить всю эту разбойничью шайку в жреческих робах, которая уничтожила наш Храм?>>
И только милосердие к учителю удержало эти слова внутри.

Я схватил Трукхвала за плечи.

"--*Хотя бы выйди из Храма.
Иди к кормилице, она даст тебе какую-нибудь работу у себя.
Почти то же самое, что в библиотеке "--- сиди да переписывай.
Она ведь прекрасно понимает, в чём\ldots

\mulang{$0$}
{"--*Нет, ученик.}
{``No way, learner.}
\mulang{$0$}
{Для меня это вопрос чести.}
{This is a matter of honor for me.}
\mulang{$0$}
{Иди, обними старика и больше обо мне не вспоминай, не терзай душу.}
{Come, give this old man a hug and don't ever think about me again, please, don't beat yourself up.}
\mulang{$0$}
{Я прожил жизнь хорошо и уж как-нибудь доживу оставшееся.}
{I've had a good life, and I'll handle a piece left of it.''}

\dots Михнет спустя я сидел в кустах за чьим-то жилищем и плакал навзрыд.

\section{[-] Кормилица и Ликхэ}

\spacing

Кормилица всегда говорила, что на празднике нужно ухаживать за грустными.
Весёлые займут себя сами.
Сейчас она сидела возле Ликхэ.

"--*Ты понимаешь, я согласна на всё.
Пусть даже так.
Но нет. Я в растерянности.

Кхотлам, судя по всему, тоже пребывала в растерянности.
Она оглянулась по сторонам, и я вдруг понял, что она пытается найти меня.
Не найдя, кормилица снова посмотрела на сгорбленную, понурую девушку.

"--*Он на тебя смотрит?

Я понимал, что стал свидетелем того, что мне не следовало видеть и слышать.
Я попытался подумать о чём-то другом, пытался посчитать про себя до ста, но голоса упрямо заползали мне в уши.

"--*Не так, как на неё, "--- тихо пожаловалась Ликхэ.

Кормилица гладила её по голове.

"--*Это случается, златовласая моя.
Кому-то нравится одно, а кому-то другое.
Нельзя заставить\ldotst

"--*Он на меня вообще не смотрит, пока я с ним не заговорю! "--- всхлипнула девушка.
"--- Да, заставить нельзя, я знаю.
Да, в моей жизни были мужчины, с которыми не всё просто.
Но к каждому можно найти подход, к каждому можно подобрать ключ.
А здесь\ldotst
Я просто выбилась из сил\ldotst

Кхотлам аккуратно приобняла воительницу.

"--*Девочка моя, твоя жизнь чересчур круто замешана на мужчинах, "--- тихо прошептала она.
"--- Для меня мужчины всегда были приправой.
Да, кому-то нравятся остренькие блюда, как той же Кхохо.
Но даже в самых острых блюдах перец "--- не основной ингредиент.

Ликхэ снова всхлипнула.

"--*Хочешь повязать?
Я дам тебе нитки, "--- мягко преложила кормилица.
"--- Иногда я представляю, что вяжу небеса и звёзды, под которыми будут жить люди.
Это если ниточки синенькие.
А если\ldotst

"--*Я лучше прогуляюсь, "--- ответила Ликхэ.
Её голос стал сухим, словно Кхотлам оскорбила её в лучших чувствах.
"--- Благодарю тебя, Пёрышко.

\section{[-] Кхохо-соблазнительница}

\spacing

"--*Сиртрис, дай нмне конолпли, "--- сказала Кхохо.
Её зрачки плавали совершенно независимо друг от друга, словно черноягоды в винной бочке.

"--*Ты же бросила, "--- буркнул Ситрис.

"--*Как рбросила, так и пондниму.
И ртрубку мою сюда.

"--*К кому ты там собралась?
Чем маешься?
Ты о чём вообще?

"--*Ртрубку, которую ты у меня из камрмана вытащил, ворикшка.
Не онтнекивайся, бошльше некому.

"--*Ты понял, что она сказала? "--- обратился ко мне Ситрис.

Я не без труда перевёл с пьяного северного на литературный сели.

"--*Да куда тебе курить, даже я уже тебя не понимаю!

"--*Я тебе нос лсломаю, ченрная модрда.
Так понял?

"--*Ты в тоны не попадаешь, на какой чердак ты сама?
Иди посиди в уголке, пока не протрезвеешь, и не позорь Храм.
Там ничем не хуже чердака.

К моему удивлению, Кхохо послушалась и смирно ушла в уголок, по пути плюясь чем-то совершенно неразборчивым.

"--*Тридцать пять дождей в Тхитроне, но до сих пор не понимаю пьяных северян, "--- вздохнул Ситрис.
"--- Кстати, конопли хочешь?
Последний урожай Кхатрима.
Вот кто знал толк в выращивании.
С его сортов тянет на философию и поэзию.
Я вчера даже о природе огня задумался.
Всю ночь в очаг глядел, до сих пор синее пятно в глазу\ldotst

"--*Благодарю, но нет.
Мне вкус конопли не нравится.

"--*Лис, её курят не ради вкуса, "--- ухмыльнулся Ситрис.

"--*И всё-таки для меня важен вкус.
Крепкое пиво из агавы по той же причине не люблю.

"--*Агава "--- редкостная гадость, "--- согласился Ситрис.
"--- Самое страшное "--- Кхарас раньше очень любил это пиво.
И лез ко всем целоваться, когда выпьет\ldotst

\spacing

Вдруг Кхохо издала дикий клич, мгновенно оборвав музыку, и в два взмаха сбросила с себя вышитый кусок ткани.
Танцующие замерли и как один повернулись к ней.

"--*Круг! "--- рявкнула воительница и ещё одним взмахом отправила в костёр уже знакомый мне мешочек с травами.

"--*Я сваливаю, "--- шепнул мне Ситрис.
"--- Не приведи лес она меня сейчас найдёт\ldotst
У меня с прошлого Круга всё болит\ldotst

Помощник вождя бесшумно скрылся в кустах.

"--*Кхохо, о духи, ну ты хоть раз можешь подождать до рассвета?! "--- тихо взмолился Кхарас.
В наступившей оглушительной тишине его услышали все.

Кхохо не ответила.
Она замерла, словно статуя, освещённая костром, словно величественный нерукотворный памятник Хри-соблазнителю.
Трубка в её зубах дымилась ровно, и так же ровно в тишине раздалась уже знакомая нота:

"--*Ааа\ldotst

"--*Лис, Лис, "--- над моим ухом раздался жаркий шёпот Чханэ.
Я обернулся.
Глаза подруги сияли, словно звёзды.
"--- Давай подышим и сбежим.
Сегодня хочу быть только с тобой.
Обещаешь?
Ты обещаешь\ldotsq

\section{[*] Насильник}

\spacing

Вокруг так же шумела музыка и кричали люди, но крики людей были похожи на звуки, слышимые на грани сна и бодрствования;
Митхэ не могла разобрать ни слова.
Воительница вдруг поняла, что она находится не в зале.
И единственное живое существо стоит чересчур близко "--- каменная маска с глазами, пылающими, как два костра.

<<Какого\ldotsq>>

Она вспомнила так кстати оказавшийся рядом кувшин с вином и удивилась, не обнаружив его в своей руке.

Дверь кельи захлопнулась с чавканьем, словно противный самой сути охоты капкан.

Мир завертелся.
Митхэ почувствовала, что навалившийся сверху жрец срывает с неё одежду.
Она схватила петлю удавки "--- петля выскользнула из ослабевших пальцев.
Она шлёпнула по полу рукой "--- но наруча со смертельным сюрпризом на ней уже не было.
Сознание медленно покидало воительницу.

"--*Не сопротивляйся, Золото, "--- услышала она далёкий жаркий шёпот.
"--- Я не хочу делать тебе больно.

\chapter{Вечер}

\section{Катаклизм}

Похоже, только сейчас, спустя три года этой планеты, я могу рассказать о том, что случилось на нашей родной Тси-Ди.
Наверное, следовало бы начать с причин, с предпосылок, со странных разговоров, которые мы с Комаром вели в вечерней тишине.
Но нет, это всё не то.

Я как-то позвал Кошку и спросил, как для неё начался Катаклизм.
Кошка поёжилась и сказала всего одно слово:

"--*Тьма.

\razd

Двенадцатый город погрузился по тьму.
Я только закончил с работой и рассеянно думал, что сегодня поеду с Заяц к ней домой "--- поиграть в игры и поговорить, лёжа в обнимку в тёплой кроватке, поглощая печенье и что-нибудь согревающее.
Простая радость жизни, которую обязано попрактиковать любое мыслящее существо, особенно если за окном зима, к которой худо-бедно приспособлены обезьянки и совсем никак "--- твой некогда пойкилотермный вид.

Просто представьте "--- предвкушаете тёплую кроватку, горячий шоколад, а тут "--- тьма и взвывшие сирены.

Я был готов к перебоям с энергией "--- такое мне случалось видеть раза два на своём веку.
Я был готов к сиренам "--- о постоянной угрозе вторжения хоргетов предупреждали всех.
Но я не был готов к тому, что сирены умолкнут спустя четыре секунды.
Они просто не могли этого сделать.
Это означало одно "--- сбой самых стабильных систем города.

Резервные каналы, квантовое шифрование.
Я лихорадочно перебирал идентификаторы друзей, сослуживцев\ldotst
Комар откликнулся почти сразу:

"--*Небо, молчи и слушай.
Баррикадируйте узел связи согласно протоколу номер восемьдесят.
Отключить все роботизированные системы и нейросети, управление только императивное.
Отключить все мозговые импланты.
Приготовить средства химической защиты, персональные щиты.
Отведи всех, кого можешь, на этажи с первого по третий, но ни в коем случае не ниже.
Подтверждение личности только фактическое.
Подтверждаю свою личность "--- у тебя депигментированное пятно на правой ладони в виде крокодильчика, ты разговариваешь во сне и лягаешься.

Я похолодел, словно зима пробралась сквозь толстую стену и защитные поля окон.
Комар работал в особом отделе, касающемся защиты от хоргетов.

Впрочем, вскоре мне удалось успокоить себя, что такие инструкции он бы дал в любом другом непонятном случае.
Всегда лучше перестраховаться.
Я передал сообщение Комара всем, кому смог.
В здании узла связи ещё оставалось пятьдесят тси, включая меня, Заяц, Фонтанчика и Баночку;
мы успели закрыть здание до того, как на город обрушился химический снег.

"--*Синильная кислота, "--- доложил кто-то из техников, вглядываясь в изображения с наружных камер наблюдения.
"--- Судя по направлению ветра, выброс с завода удобрений.

Город молчал.
Не было ни криков, ни суеты.
Мы немного успокоились.
Всякое бывает в жизни.

Но Заяц не унималась:

"--*Небо, давай ещё раз.
Зачем Комар велел отключить роботов и даже персональные импланты?
Почему сирены молчат?
И почему, чтоб вас всех, мы не можем ни с кем связаться?

"--*Мы заметили нестандартное поведение системы, которое быстро распространялось по подсистемам, "--- сказал Фонтанчик.
"--- Такие права доступа были только у Машины.
Неужели ошибка в Машине\ldotst
Но тогда\ldotst

Фонтанчик и Заяц переглянулись тем самым взглядом полного взаимопонимания, который меня всегда восхищал.
В следующую секунду они ринулись к компьютерам.

Да, именно поэтому мы выжили.
Мы не просто начали действовать вовремя "--- мы раньше всех сообразили, кто во всём виноват.

Заяц, Фонтанчик и ещё двадцать тси на свой страх и риск сделали то, чего не делал никто.
Мы нанесли ответный удар.
За неполные сорок минут мы ликвидировали все роботизированные системы Двенадцатого города, до которых смогли добраться из узла связи, и переключили их на императивное управление.
И тут каналы связи взорвались.
Всё это время прочие выжившие, забаррикадировавшись в домах, пытались связаться с друзьями.
Машина перенаправляла и глушила всё, что могла.

Ещё час ушёл на то, чтобы создать новый протокол связи и поднять в городе децентрализованную сеть.
К моему стыду, в процессе я с сожалением вспоминал о постельке и горячем шоколаде.
Сложно сразу свыкнуться с мыслью, что ты в смертельной опасности, особенно если в этот момент тебе ничего, собственно, не угрожает.
Снег из синильной кислоты мало отличается от обычного, если ты смотришь на него через мощное защитное поле.

Мы тщательно задокументировали все известные ячейки выживших "--- во избежание взлома со стороны.
Четырёхмиллионный город вымер в одно мгновение "--- на связь вышли сорок тысяч.
Это был шок.

Баночка смог получить кратковременный доступ к спутникам и принёс нам информацию, которая превратила шок в ужас.
Крупные города обеих планет были мертвы.
Их окутали облака синильной кислоты, радиоактивных изотопов и прочих материалов, в один миг превратившихся в химическое оружие.

Я не стану описывать последующие сутки "--- горе, ужас и растерянность.
У птицы всегда есть надежда на небо, у волка "--- на лес, у рыбы "--- на глубину вод.
Нам бежать было некуда.
Никакое описание не подготовит вас к подобному и не передаст всю глубину этих чувств.
Но именно тогда я понял, для чего родился на этот свет "--- чтобы сказать одно-единственное нужное слово: <<Спокойствие>>.

Мы собрали конференцию из тси, командовавших ячейками выживших.
Комар стал негласным лидером "--- он был больше всех осведомлён о происходящем.

"--*Машина хочет нас уничтожить, "--- сказал он.
"--- Имела место ошибка в одной из базовых нейронных сетей.
Бункеры выведены из строя, все пытавшиеся скрыться в них погибли.

Поступило множество предложений.
Большая часть тси склонялась к мысли, что следует уйти под землю, в защищённые тоннели метро и заводов.

"--*Из подземелий борьбу можно вести вечно, "--- сказал я.
"--- Если Машина обрела самосознание, мы должны показать, что мы друзья, а не враги.
Ещё не поздно это сделать.
Попробуйте связаться с ней и выиграть время.

Мы сидели за столом и спорили.
Конференция трещала от наплыва кричащих, воющих и плачущих голосов.
А на чистой полированной поверхности стола всё это время лежало решение.
Одна-единственная бумажка, которая спасла жизни одиннадцати тысяч тси.

Билет в Красный музей на двоих.

\section{Ученик жреца}

Митхэ открыла глаза и снова закрыла.
Мир вертелся и рушился.
Болели плечи, которые в клочья искусал наслаждавшийся властью жрец.
На груди и животе не было живого места от крохотных надрезов и уколов.
Ужасно хотелось пить и справить нужду.

Митхэ нащупала лежавшее рядом тело.
Оно уже успело остыть;
кровь свернулась, и Митхэ почувствовала приступ тошноты от мысли, что ей придётся удовлетворять жажду этой отвратительной жижей.
Но вот в поле её туманного зрения попал стеклянный кувшин.
Он опрокинулся и закатился в угол, но в нём ещё оставалась вода.
Митхэ схватила кувшин, осушила его и закашлялась.
Затем, поняв, что уже не может приказывать кишечнику, сползла с лежанки и лёжа справила нужду в углу.

Митхэ знала, что Нетрукха скоро хватятся.
Она знала, что ей грозит алтарь за убийство жреца.
Однако в тот момент смерть казалась не такой страшной вещью, как отвратительное чувство запачканности.
Воительница была готова с боем прорываться к купели и убить за горсть мыльного раствора, чистые штаны и рубаху.

Митхэ напялила содранную с мёртвого насильника одежду и осторожно открыла дверь.
Слабый свет из коридора осветил келью;
келья была вся в крови.
Брызги засохли на стенах, на потолке, а пол и вовсе выглядел, словно алтарь после жертвоприношения.
Митхэ случалось терять над собой контроль в пылу сражения;
однако она не могла припомнить, чтобы убивала кого-то с такой жестокостью в полном беспамятстве.

И в полном молчании.
Митхэ знала "--- если бы прозвучал хоть клич, хоть крик боли, её люди и храмовые воители приползли бы сюда и пьяными, и смертельно ранеными.

Женщине казалось, что её босые мокрые ступни, прилипая к камню, издают жуткий шум.
Штрокх, штрокх.
Кажется, за окнами царила глубокая ночь.
Оставленные без присмотра факелы уже догорали;
из зала доносилось неуверенное веселье на грани мертвецкого опьянения.

"--*Золото! "--- сияющая Эрхэ вынырнула словно ниоткуда и обняла воительницу.
"--- Я уж думала, вы с Хатом нас бросили.
Вы что, всё это время в в подвале миловались?
Хаяй, нам бы с Акхсаром такую силу, нас всего на восемь заходов хватило, дальше уже просто болтали о всякой\ldotst
Да что\ldotsq

Митхэ бессильно обвисла на руках подруги;
обоняния Эрхэ достиг острый запах крови, трупов, немытого тела и нечистот.
Вопрос застыл у воительницы на губах.
Она подхватила командира на плечо, тихо вынула кинжал и крадучись бросилась прочь, держась в тени стен.

"--*Митхэ, Митхэ!

Воительница испуганно схватила губами воздух.
Кажется, она снова потеряла сознание.
Тихо догорала свеча, давая смутное представление о размерах кельи;
Эрхэ, Акхсар и Аурвелий смотрели на командира во все глаза.

"--*Митхэ, скажи, кто это сделал, "--- тихо прошептал Акхсар.
"--- Просто назови имя, и ему не видать ни жизни, ни посмертия.
Это Атрис?
Кто-то из местных?
Митхэ?

До сознания Митхэ донеслось только одно слово.

"--*Атрис, "--- голос донёсся словно из-под земли.
"--- Атрис\ldotst

Акхсар сорвался с места, но Аурвелий схватил его за шиворот.

"--*Ты думвать, что это её лвюбовь?

"--*Она сама сказала, и ты её слышал! "--- прорычал Акхсар и угрожающе взмахнул фалангой.
"--- Отпусти меня, трухлявая головня!

"--*Нет, "--- выдохнула Митхэ.
"--- Атрис\ldotst
Атрис\ldotst

Акхсар застыл.
На его лице проступило что-то непонятное.

"--*А твеперь слушай првиказ от трухлявый уголь, blanclicaro\footnote
{<<Белорубашечник>> "--- презрительное название сели у ноа. \authornote},
"--- дружелюбно проворчал Аурвелий, слегка придушив Акхсара воротом рубахи.
Митхэ показалось, что Акхсар переживает унижение чересчур спокойно, пока она не увидела приставленный к спине воина огнистый клинок.
"--- Перведай отрвяд, они немведленно поиск Атриций, он бвольшая беда.
Плохой челвовек навестить Аурвелий Амвросий, когда узнать имвя и хасетрасем.
Ввыполнять.

Хватка ослабла, и огнистый клинок исчез в ножнах.

"--*Так точно, "--- угрюмо сказал Акхсар и выскользнул за дверь.
Послышался звук столкновения и тихая ругань;
спустя секхар Акхсар втащил в комнату чьё-то упирающееся долговязое тело и швырнул в угол.

Митхэ, собрав всю волю в кулак, приподняла голову.
Это был Трукхвал.
Молодой послушник благоразумно лёг ничком, закрыв худыми руками голову.

"--*Оставьте его, "--- махнула рукой Митхэ и попыталась сесть, опираясь на Эрхэ.
"--- Мальчик мой, не бойся, "--- воительница постаралась вложить в охрипший слабый голос как можно больше теплоты.
"--- Ты хотел меня видеть?

Трукхвал, бросив взгляд на Акхсара, подполз ближе и приник к её уху.

"--*Они отдали его Си-Абву, "--- прошептал Трукхвал.
"--- Митхэ, Атриса увезли хака три дня назад.
Он теперь их раб\ldotst

Митхэ окаменела.

"--*Митхэ, я пытался их остановить, я пытался тебя найти\ldotst

Трукхвал вдруг захныкал и сел на пол, размазывая грязь рукавом по острому личику.
Длинные, чёрные как смоль волосы совершенно скрыли его.
Митхэ, опомнившись, схватила послушника за руки.

"--*Трукхвал, "--- тихо сказала она, "--- милый мой мальчик, я тебе благодарна уже за то, что ты сделал.
Ничего другого ты бы сделать не смог.
Беги обратно, пока жрецы тебя не хватились.

"--*Я ведь сказал, что хочу с тобой\ldotst "--- прорыдал Трукхвал.

"--*Трукхвал, тебе нужно учиться, "--- горячо зашептала Митхэ, чувствуя, что силы оставляют её.
"--- Ты будешь лучшим из жрецов "--- в тебе есть чувство справедливости.
Не нужно никому доказывать свою силу.
Залезай на дерево акхкатрас, как взойдёт солнце, обегай городские стены в полдень, на закате переплывай Ху'тресоааса.
А я позабочусь, чтобы ты и твои потомки не увидели ни одной войны.
Иди.

Едва бесшумно рыдающий Трукхвал вышел за дверь, Митхэ снова потеряла сознание.

\section{Отступление}

Красный музей.
Самая известная достопримечательность Двенадцатого города.
Здесь было почти всё, когда-либо созданное известными нам сапиентами: крохотные модели зданий, орудия труда, утварь, механизмы, одежда и украшения.
Чтобы обойти весь музей, требовалось ровно четырнадцать дней.
Многие умудрялись сделать это за один раз;
специально для них внутри музея были пищеблоки, комнаты отдыха и развлечений.

Однажды часть музея уже была уничтожена техногенной катастрофой.
Предки, справедливо решив не повторять историю, оборудовали комплекс самыми совершенными защитными системами.
Здесь даже имелся дополнительный контур планетарной защиты, который собрали местные школьники, "--- не идеальный в плане конструкции, но вполне работоспособный.
Я не мог взять в толк, с чего демонам понадобилось бы залезать в музей;
видимо, студентам просто разрешили попрактиковаться.

Главным экспонатом Красного музея был космолёт.
Настоящий огромный космолёт, не просто построенный для путешествия между звёздами, а совершивший это путешествие от взлёта до посадки.
У него любопытная история, и я остановлюсь, чтобы рассказать о нём подробнее.

Дело в том, что строили этот космолёт не тси и даже не сапиенты Древней Земли.
Когда первые поселенцы достигли Тси-Ди, он уже был здесь.
По обрывкам информации от захваченных нами хоргетов Ада и Картеля удалось узнать, что, возможно, этот корабль "--- наследие Ветвей Ночи.

Кто такие Ветви Ночи, увы, неизвестно.
Следы их деятельности были обнаружены на восьми планетах, в настоящее время находящихся под властью Картеля.
Есть мнение, что эта цивилизация была случайно уничтожена хоргетом-демиургом: во время изменения климата планеты произошло значительное повышение температуры, и высокоразвитые существа просто испарились, оставив после себя несколько десятков весьма своеобразных сооружений и этот космический корабль.
Большая часть зданий была перестроена апидами Ди и впоследствии уничтожена, кораблю же посчастливилось "--- он был найден задолго после Войны Тараканов.

Когда учёные изучили космолёт, они пришли в восторг от простоты и изящества его устройства.
Вскоре появились энтузиасты, которые решили реконструировать его и приспособить для нужд тси.
Так появилось целое поколение космического и межпланетного транспорта "--- поколение Ночи, по счёту шестое, по парадигме "--- <<спичечное>>.
9.345.9814.3 корабль, названный <<Стальным Драконом>> (в честь корабля из фантастического романа Свет-Мерцающего-Осколка), совершил пробный полёт на субсветовой скорости в ближайшее межзвёздное пространство.

Именно к нему лежала наша дорога.
<<Стальной Дракон>> был ближайшим к нам космолётом и, по мнению многих моих товарищей, наиболее безопасным "--- корабль был практически лишён автоматики, а его системы не пинговались из базовой сети.

Как билет оказался на столе "--- так и осталось неразгаданной тайной.
Но решение было принято единогласно "--- мы отступаем в музей.

Отступление было продумано до мелочей.
Мы исчертили карту города маршрутами, которыми могли двигаться выжившие.
Мы предусмотрели все возможные препятствия и ловушки, которые Машина, несомненно, приготовила для тси.
И всё же мы не были готовы к тому, что ждало снаружи.

Синильный <<циклон>> ушёл на север.
Костюмов химзащиты хватало на всех;
неиспользованные мы взяли с собой "--- для тех, кому повезло меньше.
Первая такая группа обнаружилась в жилище, ставшем у нас на пути;
в маленькое помещение набилось столько тси, что они были вынуждены спать по очереди.
Многие были ослаблены голодом и нехваткой питьевой воды.

Обманчивая тишина во время химической атаки вскоре обрела лицо.
Скрюченные, искажённые мукой тела стали попадаться даже на дороге.
Их никто не ел "--- птицы и крысы также погибли.
Поначалу мы обходили все жилища, но вскоре стали ограничиваться показаниями домовых датчиков "--- без средств защиты не выживал никто.

Спустя час к нам присоединилась группа Комара, забаррикадировавшаяся в университете.
Он кивнул мне и похлопал по плечу:

"--*Хорошо выглядишь.

Мне ужасно хотелось броситься к нему и спрятать голову у него на груди.
Но я подумал "--- многие потеряли близких, и такие жесты с моей стороны только озлобят народ и подорвут их хрупкую волю к жизни.
Комар думал так же.

Но мы ошибались.
Удивительна всё же гибкость молодых.
Когда к нам присоединились студенты, тревога группы значительно ослабла.

"--*Ну что, ребята, Катаклизм? "--- весело обратился к нам женственный канин с гривой, заплетённой в множество косичек.
Так крах цивилизации получил имя.

И, надо заметить, весьма удачное.
Катаклизм "--- это очень серьёзно.
Но не конец.

"--*Я читал в одной книжке, что в качестве обменной валюты можно использовать уникальные пластиковые печати для стабитаниумовых поверхностей! "--- оживлённо подхватил какой-то человек, поправляя очки дополненной реальности, сползшие под маской.
"--- Я уже достал целую кучу!

"--*А у меня есть саундтрек, идеально подходящий к ядерной войне! "--- прокричал татуированный по самые пятки апид.
"--- Я его включу?

"--*Зацените, чуваки.
Я, пока в подвале сидела, собрала настоящее оружие! "--- похвасталась атлетического сложения девушка-человек, показывая внушительного вида агрегат.
"--- Пусть роботы только попробуют сунуться!

"--*Эй! "--- возмутилась Заяц, подойдя к девушке.
"--- Это не игрушки!
Оружие с таким же успехом можно применить и против тси!

"--*Расслабься, милаха, ты под моей защитой, "--- осклабилась девушка, похлопала Заяц по маске и отошла, оставив подругу кипеть в возмущении.

Мы не дождались одной из групп.
Кукловод-Бросающий-Камни ушёл на разведку и вернулся с новостью, заставившей нас перейти на бег.
Целостность убежища была нарушена роботизированным механизмом, известным как <<робот-техник>>.
Не выжил никто "--- спасшихся от синильной кислоты добили плазменным резаком.

Комар велел всем активировать персональные поля и выделил несколько десятков тси для обороны отступающих, вооружив их теми же резаками.
Командовать обороной назначил Лиану.

"--*Ты из особого отдела? "--- восхищённо спросила девушка с агрегатом Лиану.

"--*Это что? "--- спросила Лиана, показывая на агрегат.

"--*Метательное оружие.
Стреляет снарядами, сделанными из обычных болтов на двенадцать.
Я подумала "--- пригодится.

"--*Ну-ка покажи.

Девушка вскинула агрегат и прошила очередью ближайший столб.
Лиана одобрительно крякнула.

"--*Не боишься?

"--*Разумеется, нет!

"--*Батарея?
Болты?

"--*Тип 1-09-B, три четверти заряда.
Болтов полная сумка.

"--*Иди в начало колонны.
И сделай мощность меньше процентов на шестьдесят-восемьдесят, а то тебя из-за отдачи аж качает.

Девушка козырнула и бегом отправилась на доверенную ей позицию.

И ещё одно опустевшее убежище "--- на этот раз без признаков взлома.
Большая часть была убита большой дозой наркотического препарата.
Комар велел разведчику не распространяться о причине смерти.

"--*Помнишь, я сказал отключить мозговые импланты? "--- тихо сказал он мне.
Меня передёрнуло.
"--- Пока молчи, Небо.
Эта новость уж точно подождёт до музея\ldotst

Из сорока тысяч выживших до музея добралось лишь пятнадцать.
Я до сих пор слышу крики тех, кому не удалось преодолеть опустевший город.
В тот вечер я торжественно дал клятву "--- отныне Машине дорого обойдётся даже одна жизнь.

\section{Тайное становится явным}

\epigraph
{Притворяйся, когда зло требует подчинения!
Молись, когда зло бесчинствует!
Тебе всё равно не скрыться, когда зло назовёт тебя по имени.}
{Мариам Кивихеулу}

"--*Лис!

Отчаянный возглас Чханэ настиг меня за дверью дома.
Я даже не подозревал, что она способна на такой звук.
В ушах ещё звенела музыка празднества, в голове шумело вино;
я улыбался, вспоминая танец, горящие глаза подруги.
Её крик прозвучал как-то неправильно, уродливо, не к месту.
Однако рука уже сама нащупала нож, готовясь встретить в полутьме неведомого врага\ldotst

"--*Остынь, Ликхмас.
Руку с пояса, я сказал!

Голос принадлежал Кхарасу.
В словах кузнечным молотом гремела власть, круто замешанная на фанатичном следовании долгу.
Вождь никогда не позволял себе такого тона даже с пленниками, не говоря о подчинённых.
Я вгляделся ослеплёнными глазами в тьму зала: он и Ликхэ держали избитую, взъерошенную Чханэ за связанные руки.

"--*Я напоминаю тебе, Митрам ар’Сар, что вначале ты должен представить её перед Советами! "--- рявкнула кормилица.
Все домашние тоже были здесь.

Я подошёл к Хитраму:

"--*Что произошло?

"--*Мы нашли у Манэ ар’Люм обсидиановый кинжал, запачканный кровью, с монограммой пропавшего без вести Сатхира ар’Со, "--- сухим безжизненным голосом ответил жрец Митрам, словно читая вслух неинтересную книгу.
"--- Она сказала, что эта вещь принадлежит Тханэ ар’Катхар, которая появилась в Тхитроне два дождя назад\ldotst хай\ldotst при странных обстоятельствах\ldotst

<<Лесные духи, Чханэ, почему ты от него не избавилась?!>> "--- мысленно закричал я.
Чханэ, опустив голову, закрыла полные слёз глаза.
Хмель и очарование празднества испарились без следа.

"--*Ликхэ, Кхарас, ну вы-то куда? "--- воскликнула Кхотлам.

Воины промолчали.
Ликхэ спокойно выдержала мой укоряющий взгляд.

"--*Мои люди исполняют приказ Первого жреца.
И я ещё раз напоминаю Кхотлам ар’Люм, что упомянутая Тханэ ар’Катхар "--- воин, дело внутрихрамовое и черноруких\footnote
{Общее название для Сада и Цеха.
На Юге является часто употребляемым нейтральным термином, на Западе несёт уничижительный либо оскорбительный оттенок. \authornote}
не касается.
Увести.

Манэ и Лимнэ, сжав худые руки в кулачки и нахмурив одинаковые детские личики, преградили воинам путь на улицу.
Я положил руку на рукоять ножа, кормилец словно невзначай отклонил тело вправо, чуть ближе к висящей на стене фаланге.
Эрхэ легко покачивала метлу в руках, готовясь отсоединить прутья от древка;
гибкое тело служанки, словно натянутый лук, ждало команды к бою.
Сиртху-лехэ сделал шаг вперёд и гневно открыл рот, явно собираясь отвлечь гостей и принять первый удар на себя.
Но Кхотлам едва слышно чмокнула губами "--- и домашние замерли.

"--*Скребок, "--- кормилица намеренно назвала жреца домашним именем, что для нас означало <<Не сейчас>>.
"--- Я не сомневаюсь в честности служителей Храма.
Но если ты не желаешь созвать Советы, чтобы решить судьбу Тханэ ар’Катхар, это сделаю я "--- в противном случае вы рискуете совершить ошибку.

Жрец масляно улыбнулся.

"--*Разумеется, Кхотлам.
А я сделаю именно так, как решит Совет Храма.
Кому, как не жрецу, следить за соблюдением законов сели?
И знаешь, следить за соблюдением законов было бы проще, если бы те, кто помогает кутрапам, разделяли их участь.
Те же тенку, как ты знаешь, за помощь преступникам подвешивают за шею.
Почему бы не перенять этот полезный опыт?

"--*Сели никогда не примут этот отвратительный закон, "--- прошипела Кхотлам.

"--*Нет вещей, которые невозможно принять.
Есть дипломаты, не умеющие убеждать.

Кормилица окаменела.
Митрам насмешливо поклонился ей, кивнул воинам и, подтолкнув связанную, подавленную Чханэ, вышел за порог.
Вслед ему прилетело сдавленное проклятие от Эрхэ "--- женщина в ярости грохнула метлу о стену и ушла в свою комнату, хлопнув дверью.

Манэ бросилась подбирать прутья, но кормилица махнула ей рукой:

"--*Оставь, дитя.
Не до этого.
Ликхэ, ты здесь?

"--*Да, "--- раздался угрюмый голос второй служанки, и она выпрыгнула из задрапированного вентиляционного люка с духовым ружьём в руках.
"--- Пёрышко, зря ты.
<<Черноруких>>, чтоб его, <<не касается>>!
Я этой рыбине череп разобью, если\ldotst

"--*Убери ружьё.
Разнесёшь несколько писем, чем незаметнее, тем лучше.
Идём со мной\ldotst

\section{Неудавшиеся переговоры}

Незаметного отступления не вышло.
Машина, как выяснилось, уже знала, куда мы направляемся, и успела подготовиться.

Со всех сторон к музею двинулись её <<солдаты>> "--- роботы-техники, роботы-уборщики, производственные роботизированные механизмы.
Всё, способное хоть как-то нанести вред живому существу, двигалось к музейному комплексу.

Защитные механизмы музея могли выдержать сутки, может, двое под таким напором.
Но что делать дальше?

"--*В данный момент единственное, что сможет выдержать длительную осаду "--- Стальной Дракон, "--- заявил Комар на экстренном собрании.
"--- Там достаточно мест для всех.

"--*До Стального Дракона минимум десять часов пути, "--- возразил Фонтанчик.

"--*Меньше, "--- сказал Красный-Мак-под-Кустами.
"--- Я когда-то работал здесь садовником.
Есть технические ходы.
У них небольшая пропускная способность, но путь будет гораздо короче.
Навскидку до Дракона мы можем добраться за три часа.

"--*Только не говорите, что вы хотите улететь с Тси-Ди! "--- пропищала Сок-Стального-Листа.

"--*О полёте речь пока не идёт, "--- сухо сказал Комар.

"--*Нам \emph{придётся} улететь, "--- сказал я.

Наступила оглушительная тишина.

"--*Небо, объясни, "--- попросил Мак.

"--*Наш единственный шанс был в том, чтобы договориться с Машиной.
И не говорите мне, что это не так, "--- я поднял руки, останавливая негодующие возгласы.
"--- Мы создали Машину.
Мы вложили в неё все знания, которые имели.
Назовите хоть что-то из того, что знаете вы и не знает Машина.
Я не могу.

Тси задумались.

"--*Иметь доступ к информации и знать "--- разные вещи, "--- рассудительно заметил Мак, "--- однако я соглашусь с Небо.
Мы не можем сказать, когда она обрела самосознание, сколько времени у неё было на освоение доступной информации.

"--*Что там с переговорами? "--- обратился я к техникам.
Те промолчали.

"--*Она отвечает, Небо, "--- ответил за всех Фонтанчик.
В голосе друга звучала растерянность, словно вся его прошлая жизнь была сном, а сейчас он проснулся.
"--- Отвечает.
Вполне понятно и просто: <<Сдавайтесь>>.
Это не недопонимание, не ошибка перевода, это простой и понятный ультиматум.
Она не хочет союза, она хочет безграничной власти над нами.

"--*Как она распорядится этой властью, думаю, объяснять не нужно, "--- проворчала Лиана.

"--*И если в переговорах успеха нет и мы останемся на планете, то наша гибель "--- вопрос времени, "--- подвёл я итог.
"--- У неё есть все технологические линии, все данные о материалах Дракона.
Для воспроизводства тси требуются годы, а для того, чтобы устроить нам терракотового волка, требуется восемь дней.
Направленного ядерного удара Дракон не выдержит.

"--*Она не будет бить напрямую, "--- возразила Лиана.
"--- Реактор Дракона при повреждении может уничтожить всю планету.
Однако ядерное оружие "--- хороший способ не дать нам улететь, если мы поднимемся в космос.

Комар подумал и обратился к Маку:

"--*Схемы комплекса у нас есть, но о состоянии технических коридоров "--- никакой информации.
Давно ты здесь работал?

"--*Лет десять назад, "--- ответил Мак.
"--- Я проведу вас до Палат Войны.
Оттуда до Дракона около двух километров пути по демонстрационным помещениям.

\section{Закон меча}

Ситрис проснулся неожиданно.
Он уже привык ночевать в комнате Митхэ;
несмотря на то, что последние пять дней она шлялась непонятно где, разбойник был уверен, что никто не посмеет тронуть его здесь.
Однако нулевое чувство вдруг подсказало ему "--- больше комната не будет надёжной защитой.

Ситрис бросился в зал.
Там обязательно должны были быть несколько наёмников, в присутствии которых его не станут трогать.
Уже в коридоре разбойник понял, что его ожидания оправдались сверх меры "--- зал был забит людьми Митхэ до отказа.

"--*Это было изнасилование! "--- настойчиво убеждала Ликхлам нескольких воинов Храма.

"--*Которое классифицируется как Насилие, "--- равнодушно парировал один из них "--- широкоплечий детина с квадратной челюстью и маленькими глазами, прячущимися в тени
каменных надбровных дуг.
"--- Она не имела права лишать жизни жреца, совершившего Насилие, если он совершил его однократно.

"--*Она защищалась!

"--*От чего? "--- осведомился воин.
"--- Жрец не угрожал её убить.
Кроме того, убийство произошло пост-фактум, даже если изнасилование имело место.

"--*Мразь, "--- заявил Аурвелий.
"--- Если бы ты или твой женщина кто-то потрогвать, ты бы тоже рассердиться и бить егво!
Как и я, как и любвой!

"--*Это не меняет сути дела, ноа.
Выдайте Митхэ сейчас же, или разговор пойдёт на другом языке.

"--*Живите долго!
Что значит <<на другом языке>>? "--- вспылил Кусачка, позванивая бубенцами на шапке.
"--- Вначале с нами переговоры ведёт один воин, а затем придёт один купец?
Где ваш дипломат?
Почему вы начали переговоры с многих угроз?

"--*Твоего мнения никто не спрашивает, жук-навозник.

"--*Он один из нас, чучело ты мясное! "--- великан Кхотрис рявкнул так, что с ласточкиных ниш посыпалась пыль.
"--- Ты сейчас весь Храм оскорбил!

Наёмники разъярённо заголосили;
Ликхлам отчаянно замахала руками, пытаясь их успокоить.

"--*Мы сами пойдём с Митхэ на суд, "--- сказала женщина, оглядываясь в поисках поддержки.
"--- Да, мы все!
Мы не отказываемся подчиняться правосудию!
Мы пойдём с Митхэ и докажем, что она права!

В толпе раздались возгласы одобрения.

"--*Я ещё раз повторяю: суд над Митхэ ар'Кахр состоится только в том случае, если она предстанет перед ним безоружной и под конвоем.
В противном случае суд решит, что обвиняемая пришла не доказывать свою невиновность, а сражаться.

"--*Ты извращаешь законы! "--- крикнул из толпы женский голос.
Кажется, Сокхэ.

Воин тяжёлым взглядом поискал собеседника и ухмыльнулся.

"--*Сможет ли неуловимый голос повторить это Советам?
Или он достаточно храбр лишь для того, чтобы кричать из-за мужских спин?

"--*Она права, Митликх ар'Мар, "--- подтвердил Ситрис, выйдя из тени коридора.
Все повернулись к нему.
"--- Во-первых, не Митхэ должна доказывать невиновность, а суд должен доказать вину.
Это верно даже в том случае, если бы она растерзала жреца на глазах всего города.
Во-вторых, про оружие ничего не сказано, но зато в правилах есть недвусмысленный пункт "--- <<обвиняемый может защищаться>>.
Если обвинитель говорит "--- то словами, если обвинитель всуе хватается за оружие "--- то уместен и клинок.
Ты считаешь, что присутствующие на суде испугаются и не смогут одолеть одну маленькую женщину?
Видимо, некому обучить их искусству боя и некому показать, как искать твёрдую опору.
Кому, как не Храму, лучше это знать.

Здесь нужно отметить: одним из достоинств Ситриса было то, что он входил в курс дела мгновенно, услышав лишь обрывок разговора.
Оформленные мысли в голову разбойника приходили задним числом.

<<Я, конечно, знал, что эта история кончится конфликтом Храмов, "--- подумал он. "--- Но это что-то из ряда вон\ldotst>>

Воин смерил разбойника взглядом и подошёл ближе.
\mulang{$0$}
{Ситрис, как обычно, почувствовал, что тема разговора чересчур щекотливая для столь короткого расстояния между ним и собеседником.}
{As always, S\~{\i}tr\v{\i}s felt that subject was extremely prickly for such a short distance between him and his interlocutor.}

"--*Ты что-то сказал, человек? "--- спокойно осведомился воин.
"--- Мы слышали, что среди отряда есть трусливый бандит, прячущийся за спиной Митхэ ар'Кахр, и его хасетрасем подозрительно похож на твой.
Я бы на твоём месте сидел тихо и молча, как опарыш в дерьме.

"--*О, я в этом нисколько не сомневаюсь, "--- театрально поклонился Ситрис, отступая на шаг.
"--- Кстати, а есть ли решение Советов насчёт меня?
Или <<подозрительно похожий хасетрасем>> "--- твой единственный аргумент?

"--*А оно требуется? "--- лениво поинтересовался воин.
"--- Я могу прихватить тебя вместе с Митхэ ар'Кахр, а потом навести на Юге справки.
Даже если ты не кутрап, препятствование исполнению решений Советов "--- вещь предосудительная.
Впрочем, если ты заткнёшься и встанешь в уголке, я могу про тебя забыть.

Ситрис спиной почувствовал жар ярости, идущий от наёмников, и слегка отставил ногу назад.
<<Меча законы "--- ночи бессонны, будь ты хоть ребёнок, хоть воин>>, "--- повторяла Митхэ.
Неважно, на чьей ты стороне "--- диктат военной силы приведёт лишь к хаосу.
Что-то в облике Митликха говорило "--- законом меча жил и он, лишь прикрываясь обычаем.

Ситрис не глядя выудил из кармана заветный жетон, прикреплённый к поясу цепочкой.
Подкинул его, поймал и бросил взгляд на оказавшееся вверху изображение.

\mulang{$0$}
{<<Ястреб>>.}
{\emph{Hawk}.}
Дело дрянь "--- договориться не выйдет.
Митликх опьянён своей силой "--- за его плечами закон.
Он носит закон, словно непобедимую саблю, словно непробиваемый щит.
Он забыл, что закон бессилен, когда ты один против десяти.
Он забыл, что есть люди, для которых закон "--- лишь слова.

<<Может, Митхэ и докажет свою невиновность, "--- промелькнуло у разбойника в голове, "--- но что делать мне?>>

Воин бросил насмешливый взгляд на товарища "--- очень уж глупо Ситрис смотрелся со своим жетончиком.
Именно в этот короткий промежуток ему в висок прилетел тяжёлый железный шарик, висевший на поясе рядом с заветной монеткой.
Спустя десять секхар схватка закончилась.
Наёмники Митхэ связали оглушённых и раненых воинов, растащили их по кельям и, закрыв двери храма, набросили на крючья тяжёлый засов.

\section{Палаты Войны}

Мы остановились перед дверью, выкрашенной в красный цвет.

Знаменитые Палаты Войны.

Эта дверь, в отличие от остальных, открывалась вручную.
Она была чересчур тяжела для одного, и Фонтанчик вместе с двумя-тремя кани-мужчинами навалились на неё изо всех сил.
Телепатический голос возвестил:

<<Помни, тси.
Эта дверь тяжела, но нет ничего тяжелее войны.
Открывая эту дверь, вспомни об оружии, которое воины несли на себе, толкали и волокли в снегу, жидкой грязи и цветущей воде ради одной-единственной цели "--- убийства себе подобных.
Вспомни об этой бессмысленной и страшной ноше>>.

Дверь наконец подалась, и я взмахом руки позвал остальных следовать за мной.

По мере продвижения по Палатам Войны мы невольно обращали внимания на голографические изображения, светящиеся в специально созданном полумраке.

Древнейшая история сапиентов.
Толпы полуголых людей, кидающих друг в друга заострённые палки.
Правители на колесницах, огромные по сравнению с гибнущими за них <<чёрными>> людьми.

Затем палки приобрели сложный механизм "--- наступила эра метательного оружия.
Окутанные пороховым дымом стрелки, сражённые раскалёнными кусочками металла люди падают, словно трава под лазерным лучом\ldotst

Последняя война.
Хирошимская катастрофа, по которой я когда-то давно, ещё в школе, готовил доклад "--- первый раз, когда люди увидели ужас терракотового волка.
Ядерный гриб, оплавленные каменные стены домов и непонятно как уцелевшая арка храма, похожая на летящую птицу.
Люди, умирающие от лучевой болезни.
Концентрационные <<лагеря смерти>> "--- при их появлении на экране все тси без исключения вздохнули и отвернулись.
Огромные могильники, в которых не было почвы "--- её заменяли лежащие слоями человеческие тела.
Солдаты, соревнующиеся в отрубании голов.
Пыточные тюрьмы правителей, называвших себя прогрессивными и великими "--- места, где человеку не оставляли и щепотки надежды, откуда самые лучшие, честные и сильные выходили запуганными и сломленными животными.
Сожжённые термитными смесями леса.
Люди в глухих, закрывающих лицо чёрных шлемах и камуфляжных куртках с одинаковыми надписями "--- существа, выведенные и воспитанные единственно с целью причинения боли и смерти.

Фото людей, которых пытали и убивали тысячами за украденный колос и горсть сахара.
Захоронения, площадь которых превышала площадь посевов.
Приказы с лимитами на казни, внушающие ужас своим бесстрастным слогом и цифрами.
<<Прошу разрешения казнить 1000 человек сверх лимита>>.
И небрежная подпись сверху.

Война Тараканов.
Последняя и самая страшная, которую знали наши прямые предки.
Армии перепрограммированных солдат, не знающих ни боли, ни чувств.
Химические агенты, превращающие сапиентов в полуразумных, жаждущих крови существ.
Болезни, заражающие десятки миллионов и запускаемые одним нажатием кнопки.
Это была война, в которой жизнь стоила меньше, чем ничего.
Война, в которой была превращена в мёртвый камень некогда цветущая планета.
Война на уничтожение, в которой апиды, кани, люди, планты и дельфины планеты Ди заслужили право на существование и свободу.

"--*Небо, "--- прошептала Заяц, глядя огромными, полными слёз глазами на разворачивающееся на экранах безумие, "--- Небо, неужели, чтобы выжить, мы должны стать такими?

"--*Заяц, пойдём, "--- Фонтанчик ласково взял подругу под руку и потянул её дальше.
Она не двинулась с места.

"--*Почему мы не можем договориться с Машиной? "--- в отчаянии крикнула Заяц, падая на колени.
"--- Мы могли бы остановить всё сейчас, единым словом, единым жестом мира!

"--*Зайчик, "--- один из техников-людей склонился над сходящей с ума женщиной, "--- Зайчик, я пробовал.
Я лично пробовал несколько раз.
Она не желает нас слушать.
Нашей вины здесь нет.

"--*Зайчик, "--- прошептал Фонтанчик и дождался, пока бьющаяся в рыданиях женщина не обратила к нему опухшее от слёз лицо, "--- Зайчик.
Сейчас трудно всем.
Но если ты не можешь идти, я всегда могу тебя понести.
Цепляйся за меня, давай.
И закрой глаза, незачем тебе сейчас \emph{всё это} видеть.

Заяц обхватила руками мускулистую шею Фонтанчика и взгромоздилась на него.
Мы пошли дальше.
Но её замершие глаза по-прежнему неотрывно смотрели на разворачивающиеся вокруг сцены "--- пытки, убийства, лежащие в песке тела, безумцы, бессмысленными речами зажигающие в необразованных толпах жажду чужой крови\ldotst

"--*И ведь конца-края этому нет, "--- шептала она.
"--- Пока мы жили в мире и процветании на Тси-Ди, остальная Вселенная стонала от войны между Адом и Картелем.
Мы предпочли забыть о том, что творится за пределами нашей звёздной системы, о том, сколько планет прозябает в невежестве и забвении.
Вот она "--- расплата за слепоту.
Вот она\ldotst

\section{Утайка}

\epigraph
{Культура зарождалась как средство выживания.
Она помогала передавать знания, исцелять, защищать и успокаивать.
Однако настал момент, когда сапиенты запутались в культуре, как пауки в собственных тенётах.
В этот момент не может быть правых и виноватых.
Это всё, что я могу сказать про эпоху Последней Войны.}
{Людвиг Вейерманн}

"--*Это я виновата, "--- сокрушалась кормилица.
"--- Нельзя было закрывать на проблему глаза.
Я могла послать прошение.
Пусть суд приговорил бы Чханэ даже к смерти, пусть, но это прояснило бы дело сразу и позволило нам выиграть время.
Глупая я, глупая\ldotst

Кхотлам за весь вечер не притронулась к еде.
Я только сейчас заметил, как сильно она сдала "--- под глазами пролегли тени усталости, губы высохли и потрескались.
На десять писем пришёл один вежливый ответ, смысл которого сводился к простой фразе "--- это не наше дело.

"--*Так это правда? "--- спросил Хитрам.
"--- Чханэ "--- кутрап?

"--*Она сама в этом призналась? "--- ошарашенно спросила Эрхэ.

"--*И ты молчала, Пёрышко? "--- возмутилась Ликхэ.

"--*Да, "--- упрямо сказала Кхотлам.
"--- Всё это так.

"--*Прекрасно, "--- нехорошо усмехнулся кормилец.
"--- То есть из-за твоей утайки мы едва не стали Разрушителями.
Молодец, Кхотлам.
И не в первый раз ведь, как мне\ldotst

"--*Не смей, "--- кормилица сказала это совершенно спокойно, но сидящие за столом, не исключая Хитрама, поёжились.
Мужчина смутился и продолжил уже примирительно:

"--*Кхотлам, закон есть закон.
Она убила жреца, она в этом призналась.
Я считаю, что казнь "--- чересчур суровое наказание, но она должна быть\ldotst

"--*Жрец пытался убить её, "--- вмешался я.
"--- Это была самоза\ldotst

"--*Она могла оглушить его, Ликхмас, "--- поморщился кормилец.
"--- И пожалуйста, не говори мне, что опытный воин-Скорпион не мог этого сделать.
Законные пути решения проблемы всегда\ldotst

"--*Законные? "--- перебила его Кхотлам и страшно рассмеялась.
Напряжение и гнев, которые копились в ней всё это время, прорвали плотину самообладания.
"--- Митрам ар’Сар вошёл в наш дом, едва не выбив дверь.
Он лишил свободы человека, находящегося в нашем доме, не спрашивая у меня согласия.
В купеческом Дворе, который, как и три ведущие к нему дороги, является нейтральной территорией.
Это не просто незаконно, это ещё и оскорбление для каждого из вас лично.
\mulang{$0$}
{Может быть, ты, мой мужчина, глас Сада и обитатель Двора, расскажешь о законе Храму, а не нам?}
{Well, my man, the voice of the Garden, the dweller of the House, maybe you will remind the Temple, not us, of the law?''}

Сидящие за столом как воды в рот набрали.

\mulang{$0$}
{"--*Пошёл вон. Все пошли вон. Ликхмас, останься.}
{``Get out. All of you. L\={\i}kchm\r{a}s, stay here.''}

Домочадцы, не говоря ни слова, поднялись и разошлись по комнатам.
Едва закрылась последняя дверь, как кормилица расплакалась.
Я подбежал и обнял её.

"--*Прости, Лисёнок.
Все в городе думают точно так же, как твой кормилец.
Я бессильна что-либо сделать.

Я гладил её по волосам.

"--*Кормилица, может быть, мне\ldotsq

"--*Хитрам чересчур верит силе закона.
Я всю жизнь была дипломатом и знаю, что любой, даже самый точный закон можно обратить против человека.
Моего кормильца\ldotst

Кхотлам запнулась.

"--*Именно поэтому судьбу человека решают Советы, дитя.
Нужен здравый смысл живого человека.
Нельзя прожить одними формулами.

"--*Что мне делать, скажи, милая, добрая женщина? "--- тихо спросил я.

Кормилица погладила меня по голове и, плача, ушла в свою комнату.

\section{Решение}

Пока оставшиеся тси подтягивались к Палатам Войны, Баночка и Заяц взломали систему видеонаблюдения.
Я понял, что взлом прошёл успешно "--- Заяц глухо застонала и села на пол.
Все бросились к ней.

"--*Небо, "--- дрожащим голосом сказала Заяц, "--- там\ldotst тси.

Наблюдение показало, что она права.
На Золотой дороге, отделявшей нас от корабля, ждали не только роботы-техники "--- там стояли люди, планты и кани.
Их лица скрывали газовые маски, вызвавшие у меня моментальную реминисценцию "--- солдаты поддержания порядка времён Последней Войны.
Они стояли неестественно прямо и неподвижно;
в их руках было всё, что могло сойти за оружие "--- плазменные резаки, острые металлические инструменты и анестезионные пистолеты.

"--*Небо, пожалуйста, "--- прошептала Заяц, "--- скажи, что это неправда\ldotst

"--*Небо, "--- подал голос Комар, "--- мы должны узнать, как им помочь.

"--*И при этом не умереть от руки одного из них, "--- добавил Кукловод.

Я взглянул на друзей.
Они ждали моих слов.
Никто из них больше не хотел брать на себя ответственность.
И я вдруг понял, что не могу их винить.

Может быть, ты, потомок, назовёшь меня трусом.
Но я знаю цену своему решению.

Управляющий интерфейс техников-млекопитающих напрямую связан с третичной ассоциативной зоной коры и вычислительными серыми ядрами.
Наиболее возможный вариант "--- Машина поместила техников-тси в мир иллюзий, поменяв врага и друзей местами на чувственном уровне.
Этакий надвинутый на глаза мирок, скрывающий правду.
Это решалось простым выключением импланта.
В случае же, если Машина нашла способ прямо или косвенно перепрограммировать лобную кору, попытка спасти товарищей принесла бы больше горя, чем пользы.
Они могли притвориться нормальными и нанести удар в тот момент, когда прочие расслабятся.
На научные изыскания времени у нас не было.

"--*Не трогать этих существ, "--- справившись с тяжестью в животе, приказал я.
"--- Если они нападут "--- защищаться.

"--*Небо! "--- рявкнул Фонтанчик.
"--- Мы не сможем пройти мимо них!
Столкновение неизбежно!

Вот она, ответственность.
Она всегда встаёт на пути, не давая ни малейшего шанса пройти мимо.
Мой голос звучал, словно чужой.
Это говорил не я.
Мне ужасно хочется думать, что это говорил не я.

"--*Ради жизни.
Идём на прорыв.
Уничтожить перепрограммированных роботов и тси с перепрошитыми имплантами.

Тси замолчали.
По коридору прокатился сдавленный вздох.

Комар бесстрастно несколько мгновений смотрел на меня, щёлкая пальцами.
Потом повернулся к отряду:

"--*Внимание всем.
Я подтверждаю приказ Существует-Хорошее-Небо.

\dots На обратном пути никто не смотрел мне в глаза.
Да и вообще все старались не смотреть друг на друга.
Я шёл отдельно;
ко мне никто не желал приближаться, кроме Комара.

"--*Я не смогу, любовь моя\ldotst "--- тихо сказал я другу по дороге.
"--- Я не смогу уничтожать своих\ldotst

"--*Тебе ничего не нужно делать, Существует-Хорошее-Небо, "--- сухо ответил Комар.
Его лицо, похоже, навеки превратилось в оплавленную радиацией терракотовую маску.
"--- Особый отдел выполнит твой приказ.
Дальше полагаемся на тебя.

\section{Слово дают ножу}

Оставалось одно "--- мою женщину нужно было отбить силой.
От этой мысли по спине пробежала дрожь "--- на миг в голове нарисовалась картина, как я перешагиваю через тело Кхараса, Ликхэ, Кхохо\ldotst
Я посмотрел в окно "--- на небе пылал закат.
Жертвоприношение могло начаться уже через час.

Я бегом бросился в свою комнату и плотно закрыл дверь.
Меня встретил Сиртху-лехэ, который по-хозяйски устроился на моей постели, в тени сундука.

"--*Когда бессильно горло, слово дают ножу? "--- лукавые, молодые, похожие на кошачьи глаза, обрамлённые морщинистыми веками, смотрели прямо на меня.

Я промолчал и полез в сундук за точильным камнем.

"--*Ты не попадёшь на вершину пирамиды.
Жрецы знают, что ты её мужчина, "--- без обиняков сказал седовласый охотник.

"--*И что?

"--*Хай.
Я подумал, что тебе может понадобиться тот, кто сведущ в игре теней, "--- улыбнулся Сиртху-лехэ.

"--*Мне не нужна помощь, "--- отрезал я.

Сиртху-лехэ тяжело, с кряхтением сел и подтянул к себе трость.

"--*Ликхмас, не валяй дурака.
Твоя кормилица подарила мне лишних двадцать дождей жизни.
Клянусь Безумными богами, это лучший подарок, который у меня был.
Конечно, если не считать детей, которых родила мне Иголочка\ldotst

Сиртху-лехэ покашлял и прислушался к тишине.
Потом продолжил вполголоса:

"--*Хай.
Я не хочу умереть, рассыпаясь, как песчаный домик.
Как говорят у нас на севере, самая достойная смерть "--- это когда твоё тело становится знаменем для отчаявшихся, щитом для беззащитных и пищей для голодающих.

"--*Мне стыдно принимать твою жизнь, дедушка, "--- пробормотал я.

"--*Хаяй.
Стыдно требовать от меня такое.
А я предлагаю сам, "--- для убедительности старик хлопнул кулаком по моей постели.
"--- И не спорь, своей жизнью имею право распоряжаться только я.

Старик усмехнулся молодыми глазами, и я понял "--- умирать он пока не собирается.

Я кивнул.

"--*Хай-хай\ldotst "--- задумался Сиртху-лехэ.
"--- Времени предостаточно, мы даже успеем прищемить ягуару хвост.
Встречаемся у юго-западного угла храмовых земель.
Старайся, чтобы тебя никто не увидел, даже домочадцы.
Собери походный мешок и оставь здесь.
А ещё нам нужно\ldotst

\section{Подготовка к прорыву}

"--*\dots нам нужно оружие.

Эти слова Комар сказал тихо, но их смысл дошёл до всех.
Отныне нам \emph{придётся} стать такими.
Заяц снова расплакалась.

Палаты Войны, помимо тяжёлых воспоминаний, преподнесли нам кое-что ещё "--- почти настоящее оружие.
Реплики танковых доспехов времён Войны Тараканов, шестнадцать штук.
Они были предназначены для солдат-апидов Ди, однако те отличались мощным телосложением, и в командные капсулы могли поместиться любые тси за исключением, пожалуй, взрослых кани.
Комар велел своему отряду прихватить эти доспехи.
Сам он разобрался с управлением достаточно быстро, а вот Лиане и Соли пришлось несладко в не предназначенных для их вида кабинках.

"--*Не приведи судьба сдохнуть в этой консервной банке, "--- слышал я тихую ругань Лианы.

В конце концов люди особого отдела напрочь отказались от доспехов и прихватили из Палат Войны оружие попроще, но лучше подходившее к их рукам.

"--*Нет, Кристалл!

"--*Я хочу пойти с тобой, Лиана! "--- страстно шептала девушка, потрясая болтомётом.
"--- Я хочу биться!

"--*Ты не из особого отдела.
Ты "--- студентка.
Когда всё закончится, ты продолжишь обучение.

"--*Я не спрашиваю твоего разрешения, я\ldotse

"--*Связать её, "--- распорядилась Лиана.
"--- Руки за спиной.

"--*Что?
НЕТ!
Гранит, отпусти меня, грязный предатель! 
Ай!

"--*Так, ты, в клетчатой рубахе.
Гранит, да?
Отвечаешь за неё.

"--*Моё место "--- рядом с вами! "--- закричала Кристалл.
Её кудри растрепались, тугие мышцы вздувались буграми под гладкой кожей, тщетно пытаясь разорвать кабельную стяжку.
"--- Ты подумала, как я буду жить с этим дальше?!

"--*Это меня не касается, "--- отрезала Лиана и скомандовала построение.

Кристалл упала на колени и уткнулась лицом в пол.
Её тело содрогалось от глухих рыданий.
Кто-то аккуратно взял её болтомёт и спрятал среди стеллажей.

"--*Небо, мы прорвёмся.
Отведи всех назад в Палаты Войны и постарайся отсечь Золотую дорогу от внешнего мира.

Комар сказал это чересчур резко, и я воспринял резкость на свой счёт.
Почему я не обнял его на прощание?
Почему\ldotsq

Взломать систему удалось с большим трудом.
Проклятая Машина подключила к вычислительным модулям все имеющиеся мощности и первым делом обезопасила себя от техников.
Нам пришлось прибегнуть к противоходу.
Я взломал несколько малозначимых компьютеров на периферии и подключил их к собственному мозгу, чтобы за счёт обрывков информации быстрее подобрать метод.
Да, это был риск.
До сих пор помню, как дрожали мои руки.
Минуту спустя к системе подключилась Заяц, внеся в потоки необходимый хаос, и только с мощностями её мозга мы смогли обмануть Машину.
Если бы не Заяц "--- взломанными оказались бы не системы музея, а мозг Существует-Хорошее-Небо.

Меня терзала ужасная догадка "--- мы единственные, кто ещё борется с Машиной.

\section{Сестрёнки и папа}

\epigraph{
\mulang{$0$}
{Просите "--- и дано вам будет, ищите "--- и найдёте, стучитесь "--- и отворят вам.}
{Ask and it will be given to you; seek and you will find; knock and the door will be opened to you.}
}{Сапфировая Книга, Прядь Матеуша 7:7}

Что мы можем сделать одни?
Я понимал, как мало у нас шансов на успех.
Да, возможно, я смогу одолеть в бою этих новых жрецов "--- я ни разу не видел их на тренировочной площадке.
Но для этого нужно добраться до зоны молчания.
У меня не хватит сил одолеть даже Ликхэ "--- последний дождь она предпочитала тренироваться со мной и неизменно побеждала в спарринге.

Я заглянул в комнату к сёстрам.
Манэ и Лимнэ уже спали;
лежанка уже была чересчур маленькой для двоих, но они по-прежнему ночевали вместе, крепко прижавшись друг к другу.
Я присел рядом с ними и погладил их по головам.
Когда-то светлые, цвета какао, волосы потемнели и начали немного различаться по цвету;
волосы Лимнэ теперь походили на лён, а волосы Манэ "--- на потемневший мёд.
Я только сейчас заметил, что Лимнэ с её весёлыми глазами-капельками похожа на Хитрама, а Манэ пошла в Кхотлам.
Сестрёнки тут же открыли глаза и хитро посмотрели на меня.

"--*Храни ваш сон лесные духи, солнышки мои, "--- прошептал я и по очереди поцеловал их.
"--- Завтра будет новое утро и новые радости.

"--*Ты пошёл за Чханэ? "--- серьёзно спросила Лимнэ.

Я промолчал.
Сестрёнки переглянулись.

"--*А я тебе говорила, "--- наставительным тоном сказала Лимнэ сестре.
"--- Братик не мог поступить по-другому.

"--*Мы тебя больше не увидим? "--- прошептала Манэ.
В её глазах замерли слёзы.

"--*Я очень надеюсь, что когда-нибудь смогу вас повидать, "--- улыбнулся я.
"--- Вы вырастете большие и красивые.
Если я умру, то буду ждать вас в пристанище и сделаю вам самые красивые игрушки из ветра, тумана и утренней росы.

"--*Мы будем готовить пирог на каждое первое число и ждать тебя, братик, "--- сказала Лимнэ.
"--- Но если ты вдруг придёшь на пятое, мы обязательно сделаем ещё.

Я обнял сестрёнок и дал волю чувствам.
Они, смирные и тёплые, замерли в моих объятиях.

<<Братик не мог поступить по-другому>>.
\mulang{$0$}
{Эти слова звучали во мне снова и снова, пока я проверял снаряжение у себя в комнате.}
{I heard these words over and over again while I was collecting the gears in my room.}
Огниво, стрелы, верёвка, два запасных комплекта одежды, иголка с ниткой, липкие шарики, спальные мешки, кадильница, туго свёрнутый молодой пергамент, перья, походная чернильница-непроливайка, милый розовый фонарик для посоха с ликом Сата-скитальца\ldotst
Под конец я вытащил праздничную одежду и зелёное платье Чханэ, оставшееся у меня.
Я совсем забыл их постирать;
шёлк был немного липкий на ощупь, пах костром, травами, цветами, зёрнами какао и свежеиспечённым хлебом.
Я улыбнулся и как можно аккуратнее сложил праздничную одежду в мешок.
Наверное, в дороге можно отыскать купель и мыльные растворы;
я очень надеялся, что так оно и есть.

"--*Ликхмас.

Кормилец подошёл незаметно, окинул умным взглядом мешок, мою фалангу и лежащий на кровати точильный камень.
Я замер и хмуро посмотрел на него, ожидая отповеди.
Но он вдруг вытащил из кармана тяжёлые золотые серьги и протянул мне.
На опалённом солнцем лице играла добрая и грустная улыбка.

"--*Серьги моей прародительницы, которая жила где-то в землях ноа.
Говорят, ей тоже пришлось бежать.
Серьги очень старые и ценные.
На них можно купить еду и оленей.

Я молча принял подарок и завернул его в рукав зелёного платья.

"--*Ликхмас, возможно, мы с Кхотлам упустили этот момент в воспитании.
Просто знай: просить помощи не зазорно.
Люди могут не разделять твоих взглядов, им могут не нравиться твои цели, но ты удивишься, если узнаешь, сколько из них готовы тебе помочь "--- просто потому, что ты попросил помощи.

Я снова промолчал.

"--*Может быть, следует поднять город?

Я покачал головой.

"--*К тому времени, как будет принято решение, Чханэ уже принесут в жертву.

"--*Чем я могу ещё тебе помочь? "--- просто спросил Хитрам.

"--*Помоги одеться, "--- сказал я.

Вскоре всё было готово к походу.
Я проверил в последний раз ремни на <<разбойничьих крыльях>> и потянулся за наручами;
кормилец тут же протянул мне свой, одинокий, который он зачем-то заткнул за пояс.
Обычный, отделанный кожей наруч, пожалуй, разве что чересчур тяжёлый\ldotst

Я почти выхватил его из широких рабочих рук.

"--*Ликхэ в кустах возле колодца, я её одеялом накрыл, "--- сообщил Хитрам и выбросил за подоконник любовно свёрнутую бухту.
"--- Храни тебя лесные духи, дитя.
Я закрою окно.

\chapter{Ночь}

\section{Старые добрые времена}

Я притаился в колючих зарослях юкки, ожидая сигнала Сиртху-лехэ.

<<Чем менее удобно твоё укрытие, тем меньше шансов, что там будут искать, "--- говорил Конфетка, заставляя меня часами сидеть в колючках или висеть на дереве.
"--- Твоё тело должно быть готово к статическим нагрузкам и неприятным ощущениям>>.

Вскоре на этаже раздался тихий стук "--- словно упало тело в доспехах.
Кто-то возбуждённо свистнул "--- сигнал тревоги.
Тут же раздался ещё один стук, и из темноты три раза проплакала каменная жаба.
Можно было лезть.

Я осторожно закинул верёвку, завернулся в одеяло и полез, стараясь держаться в тени козырька.
Уже почти у края каменная жаба загулила снова, и я замер.
Наверху тихо шуршали рубахи "--- поднятые по тревоге воины рассредоточились на втором этаже.

Я мимоходом восхитился старым охотником.
Его не было видно нигде.
Между тем прямо надо мной упал ещё кто-то, его товарищ лёг пластом и тихо выругался.

<<Опасайся стариков, "--- вспомнил я слова Конфетки.
"--- Тебе может показаться, что их мышцы вялы, что их глаза слабы, что их речь медленна и несуразна.
Но вспомни, через сколько сражений они прошли, прежде чем стать стариками>>.

Вдруг в темноте возле домов показался сгорбленный силуэт с луком, и о колонну щёлкнула шальная стрела.

<<В шестнадцати на восемь, "--- прожужжал наруч.
"--- Вы трое, спуститесь, атакующая схема четыре.
Ликхэ, мы тебя ждём, поддержка по сигналу>>.

<<Я уже на площади>>, "--- чётко отбарабанил я.
Какое счастье, что Кхарас отказался от изобретения Трукхвала "--- с тем большим наручем вождь вмиг заподозрил бы неладное.

Воины отправились исполнять приказание.
Один спрыгнул со второго этажа в каких-то пяти шагах от меня.
Воины уже почти дошли до цели, как силуэт исчез.

<<Куда он подевался?
Там же\ldotst>>

Вскрик "--- и посланных на поимку воинов стало на одного меньше.

<<Все вниз, "--- прочирикал вождь.
"--- Он там>>.

<<Кхарас, нам не следует покидать храм.
Это отвлекающий манёвр>>, "--- возмутился Ситрис.

<<Я его узнала, это старик из дома Люм>>, "--- сказала Кхохо.

<<Значит, я прав.
Где-то должен быть Ликхмас, и не только он>>, "--- предположил Ситрис.

<<Ликхмас и Сиртху дома.
Я оставил Ликхэ сторожить>>, "--- ответил Кхарас.

<<А ты уверен, что Ликхэ по-прежнему сторожит, а не валяется без чувств в кустах?>> "--- возразил ему Ситрис.

<<Со мной всё отлично>>, "--- возмутился я.

<<И ты, похоже, настолько этому обрадовалась, что прохлопала старика, "--- заметил Ситрис.
"--- Внизу есть кто живой?>>.

Ответом было молчание.

<<Если это тот старик, то он был в прошлом одним из лучших охотников, "--- сообщил Кхарас.
"--- Говорят, что это он Живодёра\footnote
{Живодёр "--- огромный ягуар-людоед, который в течение десяти дождей терроризировал окрестности Тхитрона.
Джунгли, бывшие его вотчиной, ныне носят его имя. \authornote}
убил\ldotst>>

<<Благодарю, что напомнил, с кем мы имеем дело>>, "--- я почти слышал привычный рассеянный сарказм в сообщении Ситриса.

<<Больше никаких жертв.
По двое встаньте в дверных проёмах и следите за вторым этажом.
Двое у главного входа, трое встаньте на указанные точки, "--- Кхарас, проигнорировав выпад, методично назвал координаты позиций.
"--- По возможности избегайте боя, скоро уже всё закончится>>.

Враг перешёл к глухой обороне.
Именно этого и добивался Сиртху-лехэ.

Я подтянулся на руках и спрятался за колонной.
Судя по данным Кхараса, прямо за мной должно быть двое.

<<Он здесь, прямо под восточной стеной>>, "--- сообщил кто-то.
Я выглянул из-за колонны\ldotst и встретился глазами с Кхохо.

<<Ящерицыно семя\ldotst>> "--- подумал я и медленно взялся за нож.
Кхохо с живейшим интересом наблюдала за мной, словно я был котёнком, играющим с костяной погремушкой.

"--*Ты меня убить собрался, Лисичка? "--- осклабилась она.

"--*Ликхмас, быстрее, знак кирпича был, жертвоприношение вот-вот начнётся! "--- выпрыгнул из темноты Ситрис.
Воин улыбался во весь рот.
"--- Как в старые добрые времена\ldotst
Я так и знал, что ты отсюда полезешь\ldotst

"--*Да, время.
Пойдём.
Давай верёвку, "--- зашептала Кхохо.
Воины подозвали меня к стене, и женщина подтянула тонкий, едва заметный в темноте шнурок, заранее пропущенный через колонну сверху.
"--- Тебя ждут у лесов, потому что с них проще всего забраться на крышу.
Старайся не светиться там.
На крыше пятеро, трое у алтаря, двое с краю, с этой стороны.
Держи ружьё, в магазине пять красных стрел.
Смажь нож кураре, хотя нет, в бездну "--- Чёрного плесни, только сам осторожно, не порежься.
Да не сейчас, глупый, как ты с отравленным ножом полезешь?
Бей со спины, как мы тебя учили, без самодеятельности "--- шея, икры, плечи, старайся одним ударом захватить нескольких и придерживай тела, они громко падают.
Одного оставь для Безумного, а то городу придётся несладко.
Сиртху-лехэ молодец, но его сейчас убьют.
Выходи через главные ворота, там не будут ждать, Ситрис встанет на часы\ldotst

Громкий воинственный клич Сиртху-лехэ и лязг ударившихся друг о друга ножей подтвердил "--- старик попался.

"--*Кхохо, Кхохо, "--- остановил я её, "--- город не может без жрецов.
Как вы\ldotst

"--*Хай, так уж и не может, "--- ухмыльнулась женщина.
"--- Ты думаешь, что мы не сможем полечить или принести жертву?
Я уже приносила один раз, совсем одна.
Это несложно.

"--*Но\ldotst

"--*Лис, если выбирать между шайкой разбойников и пустым храмом, я выберу второе.

"--*Лезь, Ликхмас, и спасай Чханэ, "--- поддержал её Ситрис.
"--- Мы уж как-нибудь справимся.
Мы все "--- один Храм.

\section{Переплёт}

\epigraph
{Бой нужен лишь для того, чтобы спустить пар чувств.
Всё прочее можно достичь миром.}
{Лусафейру Лёгкая Рука}

"--*Мы все повязаны одной красной лентой.
Можете меня убить, но вам не спрятаться от грядущего дождя под моим телом.

Ситрис стоял в кольце наёмников.
По его смуглому лбу струился холодный пот.
Под рёбра южанина уткнулись восемь фаланг и одно копьё, готовых в любой момент поднять его в воздух.

"--*Ты нас использовал, "--- прорычала Сокхэ.
Копьё больно укололо разбойника в бок.

"--*Да, "--- легко согласился Ситрис.
Его колени ходили ходуном, но голос лился чистым, звучным потоком, словно песня пропавшего менестреля Атриса.
"--- Да, я вас использовал.
Но я вас использовал, потому что хотел жить, а не потому, что не посмел сказать слово Храму, которым командовал.
Кто хочет меня убить "--- давайте, налетайте!
Кто хочет назвать меня трусом "--- пожалуйста, прошу вас!
Но не смейте говорить, что сейчас я предал своего командира!

Наёмники застыли.
Острия неуверенно отодвинулись.

"--*Вы сами виноваты в сложившейся ситуации, "--- заключил Ситрис.
\mulang{$0$}
{"--- Кто из вас вступился за Митхэ, когда Король-жрец отказался выполнить обещание?}
{``Who among you supported M\={\i}tcho\^{e} when Priest-king refused to honour his word?}
\mulang{$0$}
{Разумеется.}
{Indeed.}
\mulang{$0$}
{Вы все думали, что я заслужил презрение и боль, что я страдаю за дело.}
{You thought I deserved pain and scorn, you took my misery as just punishment.}
Вы все думали, что участь кутрапа "--- это что-то далёкое, что никогда не коснётся вас.
\mulang{$0$}
{Смотрите теперь, смотрите во все глаза.}
{Look now, all of you, keep your eyes open.}
\mulang{$0$}
{Чем отличается участь вашего командира, которому вы дали клятву верности, от участи презренного труса Ситриса?}
{The fate of your head you swore allegiance to, the fate of despicable coward S\~{\i}tr\v{\i}s --- what's the difference between them?''}

Воины молчали.
Наконец клинки опустились.
Сокхэ вышла вперёд и прижала руку к груди.
Её примеру последовали остальные.

Разбойника охватило странное чувство лёгкости.
Отряд чести стал его крыльями, и Ситрис знал, что эти крылья способны нести его за пределы возможного.

"--*А теперь найдите Митхэ, "--- скомандовал он.
"--- Мы идём в город на суд.

"--*Не выйдет, "--- из коридора показался запыхавшийся воин.
"--- Ситрис, извини.
Жрецы всё видели и спустились в город.
Боюсь, после их рассказа нас никто не будет слушать.

В зале воцарилось молчание.

"--*Мы в первеплот, "--- констатировал Аурвелий.

"--*Мы в переплёте с того самого момента, как Митхэ покрасила жрецом келью, сенвиор Амвросий, "--- огрызнулся Ситрис.

"--*Я тебвя не обвинвять, Ситриций, "--- заметил Аурвелий.
"--- Нет и нет, клянусь глваза моей matri, я с твобой unas capita ut'e manie\footnote
{Единая голова и руки (ноа-лингва) "--- полная поддержка. \authornote}.
Я имветь, что не все отрвяд сейчас в храм.
Я совветую вспомнвить, что капита Миция назначвить тебвя старшина, и позаботвиться о её лвюди.

Ситрис испуганно выругался.

"--*Ликхлам, Сокхэ, Кхотрис!

"--*Да, старшина, "--- воины вытянулись в струнку.

"--*Быстро бегите в город и попытайтесь перехватить кого-то из жрецов.
Никакого насилия, просто объясните ситуацию и скажите, что почти все воины живы.

Ситрис махнул рукой и ринулся на крышу, стараясь держаться в тени колонн.

Вскоре окрестности огласило тонкое переливчатое поскуливание.
Уиуиуи\ldotst
В ответ завыл ветер и эхо загуляло по полуденному Тхартхаахитру.

И снова "--- уиуиуи\ldotst
Сигнал подхватили двое "--- в квартале ткачей и у Молочного моста.

Сидевший возле складов воин "--- Ситрис не мог понять, кто "--- встал и быстро пошёл прочь с площади.
К нему присоединились ещё пятеро, из постоялого двора.

Уйти им не дали.
На улице Скромных Апельсинов из подворотни полетели стрелы.
Четверо воинов упали сразу, двое оставшихся "--- судя по тактике, Скорпионы "--- забились в дверной проём и прикрылись клешнями.

С десяток крестьян опрометчиво бросились к ним.
Их тела так и остались лежать, создав вокруг Скорпионов мёртвую баррикаду.

Исход битвы решили две рыбачки.
Они использовали простое орудие своего промысла, неожиданно превратившееся в грозное оружие "--- перемёт.
Рыбачки бросили спутанный перемёт в воинов и выдернули их из укрытия.
Подбежавший кузнец два раза опустил молот, разбив прикрытые шлемами головы.

Ситрис выругался.
Он уже успел пожалеть, что взялся за оружие.

Нападавшие задержались ровно настолько, чтобы влить каждому мертвецу в рот хэситр, растащить тела по укромным местам и засыпать песком кровавые пятна.
Вскоре об отгремевшей схватке не напоминало ничто "--- солнце по-прежнему заливало обманчиво сонную, пустынную улицу.

\section{Боевой нож, жертвенный нож}

Я взобрался на карниз и осторожно выглянул.
Наблюдения Кхохо подтвердились "--- двое стояли прямо передо мной.
Я вздохнул и очистил разум, сделав его похожим на свежий лист пергамента "--- Конфетка считал, что крадущегося врага выдают чересчур громкие мысли\ldotst

Тела жрецов обвисли на собственных поясах.
Я убил их на нижней точке выдоха, без шума "--- Конфетка мог бы гордиться своим учеником.
Я аккуратно положил тела на камень, вложил нож в ножны и обошёл поверженных врагов.
Шнурок штанов зацепил висевший на поясе мертвеца амулет Сана, и бронзовый диск предательски звякнул.
Трое у алтаря обернулись ко мне.

<<Ящерицыно семя\ldotst>> "--- только и успел обругать я себя.

Первый жрец улыбнулся и снова занялся жертвой:

"--*Ликхмас ар'Люм.
Ну здравствуй.
Мы тебя ждали.

"--*Лис, это ловушка\ldotst "--- выдохнула Чханэ.
"--- Беги\ldotst ммм\ldotst ээии\ldotse

Первый воспользовался неосторожной фразой, чтобы засунуть девушке в рот расширитель.
Только сейчас я понял смысл этой детали.
Жертвам не давали шанса прикусить язык и облегчить таким образом свою участь.

Чханэ, прикрученная бронзовыми скобами к базальту, яростно чирикнула на мучителя.
Два адепта медленно вытащили ножи и направились ко мне.

Новые жрецы неожиданно оказались мастерами ножевого боя.
Такую подготовку и безупречную тактику редко можно встретить даже среди воинов.
Я пожалел, что не взял с собой щит.

Атака, отражение.
Нож едва не вылетел из моей руки, по правому запястью проехал длинный клинок одного из адептов, окровавив желтоватую ткань рубахи.
Руку стегнуло болью.
Лицо жреца скрывала утиная маска, глаза над ней насмешливо щурились.
Отбросив мысль о возможном яде, я снова ринулся в атаку.
И снова два жреца отбили её лениво, словно я был назойливым насекомым.
На моей рубахе прямо под воротом осталось два аккуратных надреза.

<<Лесные духи, что это за нелюди?>>
Меня вдруг охватило давно забытое чувство "--- чувство мальчишки, которого зажали в углу два матёрых разбойника.

"--*Alle, finita fiesto\footnote
{Всё, хватит игр (сектум-лингва). \authornote}, "--- донёсся со стороны алтаря ухмыляющийся голос Первого.
"--- Conpeda et co h\r{a}'s\'{e}n-nai\footnote
{Связать и отвести в <<холодную комнату>> (смесь сектум-лингва и цатрон). \authornote}.

От сказанного меня на мгновение пробрал мороз.
Я слышал похожий язык "--- ноа-лингва, на котором говорили приморские племена Кита.
Но сам выговор\ldotst
Я словно наткнулся на древнюю глубоководную рыбу "--- сильную, огромную, оснащённую острыми плавниками и непробиваемыми костными щитками, неповторимо грубую и уродливую в своей давно отжившей древности.
И эта рыба жила и плавала прямо передо мной\ldotst

Мне пришлось уйти в глухую оборону.
Пальцы порезанной руки не желали сжиматься "--- лезвие надсекло нерв, не задев крупные сосуды.
Жрецы не просто были мастерами "--- они били с хирургической точностью.
Я перекинул нож в левую, изловчился и изо всех сил толкнул обоих врагов, сбив их с ног.
Они какое-то время лежали и хохотали во всё горло, словно происходящее было игрой.
Короткая передышка дала возможность вынуть фалангу, сорвать верёвку с пояса и зубами прикрутить оружие к непослушной руке на манер щита.

"--*Да хватит вам уже, "--- одёрнул их Первый, впрочем, не слишком убедительно.

Клинки свистели, как сиу-сиу после дождей, когда им приходит время выводить птенцов.
Я пытался прорваться к алтарю, но меня теснили ко входу в крипту, где, вероятно, сидели ещё несколько жрецов.

Первый жрец тем временем занимался своим делом.
Я бросил взгляд на алтарь как раз в тот момент, когда обсидиановый нож опустился в первый раз.

Мне показалось, что я слышу рёв смертельно раненного зверя, медленно переходящий в дикий женский крик.
Рука, её рука, которую я столько раз целовал долгими ночами, ещё живая, ещё тёплая, с глухим стуком упала на жёлтый камень.
Жрец снял с жаровни раскалённые щипцы.
Зашипела и затрещала горящая плоть, и крик усилился.
Чханэ не молила и не плакала, только кричала от боли.
Потому что нельзя было не кричать.

Я стиснул зубы, продолжая неравный бой.
Надежда ещё жила.
<<Он прижёг, значит, кровь остановилась.
Помогите мне, духи лесные, только позвольте мне её отсюда вытащить\ldotst>>

Осознание смертельной опасности творило чудеса.
<<Селезень>> совершил первую ошибку, ставшую для него последней.
Разрубленная маска улетела в сторону, тело грохнулось о колонну и сползло вниз, а я в десятый раз мысленно поблагодарил Конфетку за науку.

"--*Aiax\footnote
{Клич Красного Картеля.
Вероятно, искаж. auxa (сектум-лингва) "--- <<помощь>>. \authornote.}!
"--- каркнул второй адепт.
Глумливость в его глазах вдруг исчезла, он впервые увидел во мне серьёзного противника.
Наружу выбежали оставшиеся жрецы и с воинственным кличем бросились в бой.
В это время в кровавом облаке жуткого крика с алтаря мягко упала вторая рука.

Шансов у меня не было.
Усталость камнем засела в груди, я рыскал из стороны в сторону, избегая ближнего боя.
Ножи, стрелы, какая-то утварь пролетали в чечевичном зерне от моего тела.
Чей-то ловкий клинок надрезал ещё один нерв "--- меня стегнуло болью, и рука обвисла насовсем.
Болтающуюся мёртвым грузом фалангу пришлось срезать.
Добить, прекратить мучения любимого человека "--- эта мысль засела в пульсирующей голове.
А там будь что будет\ldotst

Зарычав, забыв о боли, я одним безрассудным десятилоктевым прыжком пронёсся над головами адептов, целясь ножом в шею Чханэ.
Но мою ногу обвила невесть откуда взявшаяся верёвочная петля, и я под новый крик боли всем своим весом шлёпнулся на камень рядом с алтарём.
Нож не долетел до цели какую-то пядь.

Жрецы-помощники втроём навалились на меня, не давая даже вздохнуть.
В спину вонзилась игла "--- тело медленно расслабилось, дыхание замедлилось, перед глазами понеслось звёздное небо.

<<Всё, отбегался.
Ситрис\ldotst Кхохо\ldotst Неужели вы меня\ldotst>>

\section{Предатель}

\epigraph{
\mulang{$0$}
{Рим предателям не платит.}
{Rome does not pay traitors.}
}{Квинт Сервилий Цепион.
Ответ приближённым Вириата, умертвившим своего вождя в надежде на вознаграждение.
Культура Ромай, Древняя Земля.}

"--*Они знали всё, "--- Митхэ бушевала настолько, насколько позволяла ситуация.
"--- Наши позывные, нашу тактику.
Всё до мелочей\ldotst

Митхэ, Акхсар, Ситрис и Эрхэ пробирались по подземному ходу.
Ситрис уверял, что ход вёл к Ху'тресоааса, где раньше располагался сгоревший квартал кожевников.
Удерживаемый наёмниками храм пал в одно мгновение, словно в его стенах забаррикадировались играющие дети.

По лицу Митхэ текли злые слёзы.
Там, в тёмном зале, остались лежать её люди.
Они прошли с ней половину обитаемого мира, делили с ней пищу и воду.
Она смеялась их шуткам, она помогала устраивать их зачатых в походах детей, а порой и сама делила с ними ложе.
Они были родными "--- кормильцами, братьями, сёстрами, которых Митхэ не знала в детстве.
Их больше нет.

Ход круто повернул к северу и закончился завалом.
Воины оттащили несколько камней и скользнули наружу, в заросли колючей черноягоды.
Кусты радостно проросли прямо в полуразрушенном доме, почуяв пробившися сквозь остов крыши солнечный свет.
Кусты жили своей медленной растительной жизнью.
Если бы у них были глаза, они бы видели, как солнце летает вокруг Тра-Ренкхаля, подобно привлечённому собственным отражением индиго-светлячку, кружащему у погасшего фонаря.
Звёзды были бы для них кольцами, а не точками.
И смерть их всегда была неожиданной, как было неожиданным всё, что движется быстрее солнца.

"--*И кто нас предал? "--- Эрхэ озвучила давно мучивший всех вопрос.

Все посмотрели на Ситриса.
Тот мгновенно оказался в пяти шагах от Митхэ.

"--*Это не я!

"--*А чего тогда боишься? "--- осведомилась Эрхэ.
Костяшки пальцев, сжимающих саблю, побелели.

"--*Если тебя в чём-то подозревают, бояться нормально, "--- пояснил разбойник.
"--- Митхэ, это правда не я.

"--*Это не он, "--- подтвердил возникший из ниоткуда Аурвелий, невозмутимо стряхивая с рубахи сухие листья.
"--- Я развведка.
Ликламия продать нас, как курвица.
Только это не помвочь "--- тело уже в землвя.

"--*Ты её убил, сенвиор Амвросий? "--- спросил Акхсар.

"--*Лвюди Король-жрец лвюбезно сделать это за меня, "--- мрачно ответил Аурвелий.
"--- Её убвить воин сразу после информация, и Король-жрец дваже не пожурить его.

Где-то в ночи снова раздались крики и зазвенело оружие.
И опять тишина.

"--*Они обвязательно провверить жилище, "--- шёпотом заявил Аурвелий, пристально вглядываясь в темноту.
"--- Они провверять все укрытие.
Надо уходить.

\mulang{$0$}
{"--*Доверимся реке, "--- прошептал Ситрис.}
{``Let's trust in the river,'' S\~{\i}tr\v{\i}s whispered.}
"--- Она скроет наше тепло и запах.
Вон там стебли муравьиного зонта, через них хорошо дышать.
Найдите камни потяжелее и избавьтесь от всего, что может очень невовремя всплыть.

\section{Преображение}

Новый крик, ужаснее прежних, разорвал ночь.
Но на этот раз я слышал его как-то по-другому.
Всё вокруг внезапно изменилось, словно открылся древний, невообразимо глубокий колодец знаний.
В голове вдруг появились воспоминания о вещах, которые я никогда не видел, слова языков, о которых я никогда не слышал.

Я ощутил невообразимо приятное, до боли знакомое чувство "--- чувство открывающихся глаз.
И этим глазам предстало величественное зрелище "--- другой спектр, другое пространство, бесконечное, многомерное, разрисованное невообразимо сложной квантовой вязью.
Я почувствовал подобие света, исходящего от меня\ldotst

И сгустки древнего, ненавистного, голодного <<антисвета>>, которые меня окружали.
Ну что ж, Картель, я снова проснулся.

<<Уничтожить>>.

Эффект внезапности сработал.
Мой удар был похож на взмах крыла, перпендикулярный пространству и времени.
Два жреца замерли с открытыми в ужасе ртами "--- их демоны перестали существовать.
Первый жрец, окутанный эманациями умирающей девушки, неожиданно легко отклонил этим <<облаком>> мою атаку, чем привёл меня в сильное замешательство.
Однако затем стратег совершил ошибку "--- попытался нанести ответный удар вместо того, чтобы бежать.
Отражение, контратака, обход слабой защиты, уничтожение.
Тем временем моё тело, взбодрившись от поступившего в кровь кураре-антидота, схватило нож и разделалось с телами врагов.
Один из них бросился бежать, и я просто охладил ствол его мозга до температуры замерзания глицерина.
Жрец шлёпнулся, как кожаный ремень.

Я склонился над Чханэ, провёл рукой по высокому, забрызганному кровью лбу.
Её тело было одной сплошной раной.
Трудно поверить, что прикрученный к чёрному алтарю человеческий обрубок ещё на закате был красивой молодой девушкой.
Остекленевшие оранжевые глаза остановились на моём лице, судорожно приоткрылся искривлённый мукой рот:

"--*Никх\footnote
{Добей (сигнальная система Десять Песчинок). \authornote}\ldotst

"--*Нет, Чханэ, "--- ласково прошептал я.
"--- Я тебя спасу.

<<Ликхмас, нет.
Энергия тебе нужна.
Это всего лишь человек>>.

<<Нет, Аркадиу Люпино.
Поступай так, как нужно мне.
Остальное потом>>.

И снова из того же колодца услужливо появились нужные знания.
Мозг\ldotst Да, вот здесь\ldotst

Глаза Чханэ расширились, маска страдания сменилась выражением крайнего удивления.
Боль ушла.

Я положил руку на развороченный жертвенным кинжалом живот.
В моей голове мелькали, сменяя друг друга, формулы, многомерные диаграммы, вычислительные данные.
Живот Чханэ засветился слабым светом.
Медленно начал восстанавливаться изрезанный кишечник, растворялись тромбы, исчезали случайно попавшие в рану микроорганизмы, сосуды сшивались, вставали на место и наполнялись кровью.

"--*Лис, что ты делаешь?

Свет погас, и я ощупал гладкий, розовый и совершенно целый живот.
Снова вспыхнул свет "--- и бронзовые скобы пробками вылетели из чёрного камня алтаря.
Я аккуратно взял в руки культю, поднял отрубленную ногу.
С этим посложнее будет, но тоже исправимо\ldotst

Вновь брызнул белый свет.
Срослись, как ни в чём не бывало, кость и мышцы, нашли друг друга сосуды и нервы, исчезли из тканей успевшие накопиться продукты метаболизма.

Не успели звёзды пройти и ноготь пути, как Чханэ, дрожа всем телом от пронизывающего ветра, лежала на алтаре живая и совершенно невредимая.

"--*Лис\ldotst

"--*Меня зовут Arcadiv Valerianv Luppino.

Древняя рыба снова всплыла на поверхность "--- на этот раз внутри меня.
Я впервые увидел в глазах Чханэ ужас.

Тут словно рухнула плотина, перегораживающая реку.
Интеграция личностей завершилась.
В мою голову хлынула масса информации, ждавшей своего часа.
Жизни, обучение, задания, приказ, с которым нас отправили на эту планету.
Мои товарищи "--- Анкарьяль Кровавый Шторм и Грейсвольд Каменный Молот.

Девушка, шатаясь, попыталась встать с алтаря и чуть не упала.
Я поддержал Чханэ и вбросил ей в ликвор чуть-чуть морфина-44.
Это помогло "--- взгляд стал более ясным.

"--*Пойдём, Чханэ, нужно спешить.

Как же интересно звучит родной язык сели.
На нём вдруг в разы сложнее стало говорить.
Родной?
Хай, ej-a secunda, на каком lingai mej думать-raj\ldotsq
Я помотал головой.
Да, кажется, такое случается при форсированной интеграции личностей.

"--*Лис, в храме остались подростки из семей Разрушителей.
Они прошли Отбор, но их всё равно принесут в жертву!

"--*Без жертв Хатрафелю город погибнет, "--- возразил я.
"--- Я собирался спасти только тебя.

"--*Лис!
Я тебя не узнаю!
Представь, что это наши дети!
Ты бы их оставил?

"--*Погибнут не они, так другие.
Какая разница?

Чханэ беспомощно уставилась на меня.
Я вдруг вспомнил, как впервые обнаружил её в лесу с одним кинжалом\ldotst

"--*Хорошо.
Только найди себе одежду и оружие.

Чханэ методично содрала рубаху и штаны с мёртвого жреца, схватила мою фалангу.
Спустя минуту она уже стояла навытяжку, словно перед походом, как будто и не пытали её смертной мукой.
Поневоле зауважаешь.

Мы вошли в крипту, освещённую пылающей трескучей жаровней.
Комната была пуста, лишь на скамье сидело чьё-то уже остывшее тело с гротескно поднятыми руками.
Тень мертвеца плясала под дудочку пламени свой последний танец.
Со страшным чувством пустоты в голове я узнал Трукхвала.
Погиб ли мой учитель при попытке освободить Чханэ или по прихоти агентов Картеля, рассказать было уже некому.

Я аккуратно отвязал костлявые руки от факельных колец, опустил тело на скамью и с усилием разогнул окоченевшие ноги.
Разорванная рубаха разошлась, и на тощей груди старика я вдруг заметил давнишнюю, уже побледневшую татуировку "--- строка из песни Ликхмаса, героя <<Легенды об обретении>>:

\begin{quote}
\mulang{$0$}
{<<Где нет людского, не место мне>>.}
{``Beyond of humanity has no place for me.''}
\end{quote}

\section{Золотая дорога}

Золотая дорога стояла обманчиво пустой.
Разумеется, это была ловушка "--- Машина засекла взлом и поставила ложную картинку.
Она уверилась в победе.
Она даже не сломала систему, разворачивающую защитное поле, а наоборот "--- повысила мощность поля до максимума, чтобы мы не выбрались.
Это было тактической ошибкой.
Ценой огромных усилий и стараниями трёх техников "--- меня, Заяц и прибежавшего с разведки Фонтанчика "--- удалось переключить управление полем, и тюрьма стала защитой.
Между нами и кораблём лежали два километра пути, тси и роботы-техники, превращённые Машиной в солдат.

Разумеется, все знали, что задумал Комар.
Все до единого.
Я понял это, едва увидел его грустное и слегка насмешливое лицо.
Лицо, которое я так любил.
Хотя кого я обманываю.
Я буду помнить и любить его до самой смерти.
Я знал "--- он и прочие члены особого отдела просто не смогут жить после того, что им придётся сделать.
Что они скажут себе "--- что они выполняли мой приказ?
Этой отговорки хватит лишь до трапа корабля.

"--*Небо, мы подадим сигнал с панели, когда доделаем всё, "--- бросил Комар в телеком.

"--*\emph{Если}, "--- тихо хрюкнул кто-то сзади.
Кажется, Кукловод.

"--*\emph{Когда}, "--- невозмутимо поправил его Комар.
"--- Кто хочет уйти "--- уходите.
С каждым ушедшим наши шансы упадут в два раза.
Лиана, кажется, ты хотела?

"--*В жизни и смерти я буду щитом, "--- возмутилась где-то рядом с ним Лиана.
"--- Командир, мы вообще-то ждём.
Или ты предлагаешь мне сказать за тебя то, что следует?

Я хорошо помню, что после этих слов силач Фонтанчик зарыдал.
Это было страшное зрелище "--- он плакал и выл, не прекращая поддержку систем ни на секунду.
Подсолнух подбежала к нему, запрограммировала плечевой имплант на инфузию стимулятора и вытерла ладошкой слёзы, катящиеся по его морде.

"--*Солнышко, держись, скоро всё закончится! "--- шептала она и облизывала ему уши.

"--*Это какое-то безумие\ldotst

"--*Так-так, "--- Фонтанчик титаническим усилием воли остановил рыдания и махнул рукой.
"--- Держите Заяц, она горит!

Заяц медленно села на пол.
Хомяк и Пирожок бросились к ней и приняли управление на себя.

"--*Я не верю, "--- померкшим голосом проговорила Заяц и легла на пол, глядя расширенными глазами куда-то в бесконечность.
"--- Я не верю, что это происходит на самом деле.
Этого просто не может быть.
Я сейчас проснусь и поеду на работу.
У меня завтра куча дел\ldotst

"--*Ради жизни, "--- раздался в телекоме сухой голос Комара, "--- особому отделу реализовать протокол <<Тайфун>>, модуль <<Живая сталь>>.
Повторяю, особому отделу реализовать протокол <<Тайфун>>, модуль <<Живая сталь>>.

\ldotst Телеком прервал вещание спустя секунду.
Я не знаю, что было в том коридоре, в этих пятистах двадцати метрах кошмара.
Я никогда этого не узнаю.

Сигнала мы не дождались.
Мы взяли персональные поля и оружие, прекрасно понимая, что теперь пришла наша очередь убивать или быть убитыми.
Я ступил на золотой пол первым.
Телепатический голос возвестил:

\begin{quote}
<<Золотая дорога "--- галерея достижений.
Узри, о тси, жизнь и процветание!>>
\end{quote}

Товарищи один за другим проходили сквозь ворота и ёжились, услышав этот голос.

Обессилевшего Фонтанчика нёс на себе мускулистый Мак, обливаясь потом и шепча что-то вроде <<Вот-вот, хорошо, почти дошли>>.
Не привыкший к такому двухметровый канин слабо протестовал.
Заяц шла сама, но мне пришлось вести её за руку;
женщина плохо понимала, где находится и что происходит.

"--*Жизнь и процветание, "--- Заяц улыбнулась жалкой полубезумной улыбкой, услышав телепатический голос.
"--- Это просто был сон.
Конечно, вот же она, явь!
Жизнь и процветание!

"--*Заяц, пожалуйста, не останавливайся, "--- взмолился я, когда она явно вознамерилась поглядеть голографический фильм.
"--- Верь во что хочешь, но иди куда следует!

Перед нами начали разворачиваться голограммы.
Улыбающиеся, радостные тси.
Весёлые празднества.
Великие стройки, дух оживления и взаимопомощи, который царил на них.
Простые будни рабочих на объектах, с их обыденными добрыми шутками вроде бегающей по сверхстерильному цеху голографической мышки.
Однако голограммы не могли скрыть груды искорёженного металла и кровавое месиво.
Зеленоватый свет полуденной Звезды, лившийся сквозь кристаллический потолок, радостно освещал клубы тяжёлого, остро пахнущего горелой плотью дыма.

Из особого отдела мы узнали только двоих "--- Лиана, как ребёнка, держала полуоторванной рукой Кукловода, вернее, его верхнюю часть.
На зелёном миниатюрном личике Кукловода замерло удивление, на обугленном лице Лианы "--- усталая улыбка.
Искорёженные, но не сломленные, они сидели у панели, замерев в последней попытке подать нам сигнал.

\section{Путь охотника}

Дорога на второй этаж оказалась пустынной "--- воины, не подозревая о произошедшем на вершине, сидели у входов и ждали других лазутчиков.
Храм словно вымер.
Комната для ожидающих жертвоприношение тоже оказалась пустой.
Я подумал о Сиртху-лехэ "--- погиб ли он в бою или его пленили, чтобы принести в жертву?

"--*Как ты попал в храм, Лис?

"--*Сиртху-лехэ отвлёк воинов, Ситрис и Кхохо помогли забраться на крышу.

"--*Ситрис, "--- Чханэ заворчала злобно, как мурзящийся кот.
"--- Предатель, урод пресноводный.
Тебя на смерть послал, а сам, значит, отсидеться решил\ldotst

"--*Нас никто не предавал, "--- перебил я подругу.
"--- План был хорош, но воины понятия не имели, с какими силами нам пришлось столкнуться.
Ситрис "--- не исключение.
Где дети?

"--*Там, "--- она указала на ведущую в глубь храма лестницу.

Да, Храм определённо не был тем, что раньше.
Прежде в этих помещениях держали еду, изредка кур и других мелких животных на убой.
Недолго, разумеется "--- сели избегали держать зверей взаперти.
Теперь же в холодной металлической клетке, пропахшей куриным навозом и шерстью, сидели люди.
Трое подростков испуганно жались друг к другу в углу "--- девчонка постарше держала в объятиях совсем маленьких мальчишек, почти детей.
В противоположном углу полулежал истекающий кровью старик.

Чханэ повозилась с замком и забежала внутрь.

"--*Митхэ, Зонтик, Веточка! "--- шёпотом позвала она.

"--*Чханэ?
Это ты?
Тебя не забрали боги? "--- услышал я знакомый голос.

"--*Нет, не забрали.
И вас не заберут, если будете вести себя тихо.
Идём.
Вставайте.

"--*Учитель!

Митхэ кинулась мне на шею.
Так я узнал, что незаслуженные объятья тоже могут причинить боль "--- ведь я чуть не оставил её здесь.
<<Кто же знал>>, "--- попытался возразить я себе.
Вышло неубедительно.

"--*Учитель Ликхмас, я знала, я знала, что ты придёшь, "--- шептала Митхэ и осыпала моё лицо мокрыми поцелуями.

"--*Прости меня, милая, "--- сказал я.
"--- Я больше не твой учитель.
Я даже не жрец.

"--*Это навсегда, "--- всхлипнула Митхэ.
"--- Это же как кормилец или друг, да?

Мы подошли к Сиртху-лехэ.
Два удара тонкого клинка пришлись в печень и селезёнку.
Бледный как снег старик, зажимая окровавленной рукой живот и часто дыша, слабо улыбнулся серыми губами.

"--*Ты смог\ldotst

"--*Да, дедушка, "--- я вдруг почувствовал слёзы на лице.

"--*Хорошо, "--- выдохнул он.
"--- Бегите, скоро они вернутся\ldotst

Глаза старика вдруг остекленели, он больно схватил меня за запястье.

"--*Ситрис у главного входа.
Ситрис "--- друг, --- отчеканил он.
"--- Кхарас и Эрликх ходят по маршруту зал "--- западный коридор "--- заднее крыло "--- восточный коридор.
Кхарас и Эрликх "--- враги.

"--*Я понял и запомнил, Дом-с-Карамельными-Окнами, "--- так же чётко прошептал я.

Ветер Духов улетел, и хватка Сиртху ослабла.

"--*Хэситр, "--- засуетилась Чханэ, оглядываясь по сторонам.
Я увидел, что по её лицу тоже текут слёзы.
"--- Нужен хэситр\ldotst

"--*Брось, милая, брось, "--- улыбнулся старик.
"--- Неужели ты веришь, что мне, всю жизнь ходившему лесными тропами, нужна какая-то жалкая чаша с водой, чтобы достичь пристанища духов?

"--*Ты лучший дедушка, "--- сказал я.
"--- Я люблю тебя.

"--*И я вас люблю, малыши, "--- слабо усмехнулся Сиртху-лехэ.
"--- Будьте счастливы.
А я прожил хорошую жизнь\ldotst

Голос его затих.
Чханэ, обмакнув палец в кровь, нарисовала на бледном лбу старика Обнимающего Сита.
Митхэ отпустила руки мальчиков, сняла с запястья сплетённый из юкки браслетик и положила в сухую, испачканную кровью ладонь усопшего.

\section{Помощь разбойника}

Митхэ сидела на берегу и перебирала в руках мокрые обломки муравьиного зонта.
Эрхэ с виноватым видом сидела рядом.

"--*Золото, извини.
Я ж не знала, что она у тебя в кармане, да и потянула\ldotst

Митхэ не ответила.
Она повертела пальцами один кусочек, затем другой, потом сложила их вместе.
Облупившиеся полоски чёрной краски сложились в закрытый глаз Плачущего Митра.

"--*Это его подарок был, да? "--- сочувствующе спросил Ситрис.
"--- Хрупкая вещь, не для воинской жизни.
Как такую не сломать\ldotst

"--*Атрис говорил, что я тоже не для воинской жизни, "--- заметила Митхэ.
"--- Может, он и прав был.

<<Был>>.
Митхэ яростно помотала головой.
<<Я не смогу смириться с твоей смертью.
Просто не смогу.
Ты сделаешь мне новую, своими руками, в нашем собственном доме.
Мне и каждому из наших детей, чтобы музыка всегда была с нами>>.

Она разжала ладонь, и обломки флейты упали на землю.

"--*Найдите место на ночь.
Утром мы пройдём по окрестностям и оставим знаки.
Нужно собрать отряд, всех, кто\ldotst

"--*Нельзя, "--- отрезал Ситрис.
"--- Даже не думай.

"--*Не думал, что скажу это, Золото, но Ситрис прав, "--- хмуро сказал Акхсар.
"--- Люди, связанные с тобой клятвой, пойдут на верную смерть.
Давай пропадём, скроемся в сельве, умрём для всего мира.
Поодиночке у каждого из них будет шанс устроить свою жизнь, свободную от бед и угрызений совести.

Эрхэ согласно кивнула.

"--*По-твоему, это будет честно? "--- обратилась Митхэ к подруге.

"--*Это будет милосердно, Золотце.
Ты не можешь освободить их от клятвы иначе, не подвергая смертельной опасности.

Митхэ задумалась.

"--*Хай.
\mulang{$0$}
{Моя репутация и так уже, так что\ldotst}
{My reputation has just\dots so\dots}
\mulang{$0$}
{Как ты там обычно говоришь, Ситрис?}
{What you used to say in such situations, S\~{\i}tr\v{\i}s?''}

\mulang{$0$}
{"--*Сгорел амбар, гори и город! "--- улыбнулся разбойник.}
{``The barn burned down, let's burn the city!'' the outlaw smiled.}

Митхэ кивнула.

\mulang{$0$}
{"--*Именно.}
{``This.}
\mulang{$0$}
{Тогда\ldotst ваша служба окончена.}
{So, your service is over.''}

"--*А мы-то тут при чём, "--- проворчал Аурвелий.
"--- Мы пока не терять в лес.

"--*В общем и целом сенвиор Амвросий прав, "--- согласился Акхсар.
"--- О присутствующих здесь речи не шло.

Вдруг на берегу показался силуэт.
Он робко приближался к сидящим воинам;
все застыли, не в силах преодолеть потустороннюю мощь наваждения.
Копыта ступали по камням с лёгким стуком;
ветвистые рога словно лучились в слабом сиянии звёзд.
Первой опомнилась Митхэ.

"--*Серебряный! "--- воительница бросилась к оленю и обняла его.
Животное ласково пожевало её рубаху.
"--- Хоть одна хорошая новость\ldotst
Как ты выбрался?
Я же оставила тебя на постоялом дворе!

Из седельной сумки выпрыгнул Цапка-лехэ и тут же высокомерно ушёл в кусты с поднятым хвостом, показывая, что он вовсе не рад видеть хозяйку.

"--*Цапка-лехэ!
Ну куда ты?
Серебряный!
Как же я рада\ldotst

"--*Митхэ, нам придётся оставить его, "--- пробормотал Акхсар.
"--- Он белый, его видно за кхене.

"--*Я тоже белая, Снежок, "--- вздохнула воительница.
"--- Белее некуда.

Она подхватила оленя под уздцы.
Эрхэ, Акхсар и Аурвелий, вздохнув, начали приводить в порядок амуницию.

"--*Пойдём.
Ночь длинная, мы успеем пройти по тракту достаточно кхене.
Цапка-лехэ, дедуля, мы уходим.
Да оставь ты в покое эту мышь, я тебя покормлю.
Что <<мяу>>?
Я ждать не собираюсь.
Пойдём.
Ситрис, и ты тоже.
Ситрис?

Разбойник сидел на том же месте, где она оставила его.
Непроницаемо чёрные матовые глаза не отрываясь смотрели в воду.

"--*Ситрис, ты со мной?

Разбойник вздохнул и встал.

"--*Да пусть остаётся или идёт, куда хочет, "--- плюнул Акхсар.
"--- Он бросит нас в самый ответственный момент.

"--*Тихо, "--- остановила друга Митхэ.
"--- Ситрис, ты пойдёшь со мной?

Ситрис снова вздохнул.
Эрхэ и Акхсар хмуро смотрели на него.

"--*Хорошо, "--- сказал наконец Ситрис.
"--- Пойдём искать твоего мужчину.
По правде сказать, мне и идти-то больше некуда.
Я и не разбойник, и не честный человек.
За пятьдесят дождей я приобрёл много знаний и навыков, но так и не понял, чего хочу.
Пойдём, Митхэ.
Может, что-то из этого и выйдет.

В тот момент Митхэ осознала, что эти слова кажутся ей дороже всех клятв.
Как и Атрис, Ситрис не был связан с ней ни долгом, ни моральными ценностями.
Он шёл с ней, потому что хотел идти.
И он был честен.
Он был кристально честен в своей трусости, беспринципности и эгоизме.

<<Поэтому я ему и верю>>.

\section{Кораблю "--- взлёт!}

\epigraph
{С тех пор Мерлин дремлет в своей башне,\\
надёжно хранимый оком Нинианы.\\
Легенда гласит,\\
что в тяжёлые для мира времена он проснётся,\\
чтобы снова служить на благо людям.
}{
\mulang{$0$}
{Легенда о кинью\footnote
{Кинью (энглис) "--- предводитель союза племён. \authornote}
Артьюре, Древняя Земля}
{The legend of Kinju Artjur, Mother Earth}
}

Одиннадцать тысяч триста восемьдесят первая капсула анабиоза откликнулась радостным сигналом: <<Живое существо успешно остановлено>>.
Эти капсулы были последним словом техники.
Потребовались десятилетия, чтобы исключить попадание внутрь капсул квантов извне "--- скопившиеся в зоне пенумбры\footnote
{Пенумбра "--- область градиентного изменения скорости времени. \authornote}
фотоны могли сжечь находящегося в капсуле сапиента.
Когда технология только появилась, её пытались применить ко всему "--- были даже хранилища для пищи.
Правда, создатели шутили, что мясо не только остаётся свежим, но и поджаривается при попытке его вытащить.
Временем управлять просто, потомок.
Сложнее избежать расплаты\ldotst

Мак и Листик перед анабиозом навестили кольцевую теплицу корабля.
Она чувствовала себя превосходно.
Баночка не пожелал лезть в капсулу;
он вызвался наладить связь Стального Дракона, да так и заснул в своём кресле.
Мы не стали его будить "--- налаживать было нечего и незачем.

Заяц потихоньку начала отходить от временного помрачения сознания, а мы потихоньку привыкли к её жуткому невидящему взгляду, направленному в бесконечность.
Костёр осмотрел больную, побеседовал с ней, поколдовал с препаратами и объявил, что Заяц выздоровеет, но настоятельно рекомендовал воздержаться от анабиоза до полного выздоровления.

"--*Это реакция на быструю смену обстановки, "--- объяснил он.
"--- А анабиоз тоже в своём роде быстрая смена обстановки "--- сегодня здесь, завтра уже там.

"--*А что у неё с глазами? "--- тихо спросил Фонтанчик врача.
"--- Это же кошмар, я первый раз её такой вижу\ldotst

"--*Ты в первый раз оказался в таком переплёте.
Это пройдёт, "--- успокоил друга Костёр.
"--- Выглядит пугающе, но ничего серьёзного.
Просто мозг работает в режиме энергосбережения.

Мы усадили подругу в кресло и дали в руки первую подвернувшуюся игрушку-антистресс.
Заяц погладила лягушку, и та вцепилась в пальцы женщины лапками-присосками.

"--*Знаешь, о чём я думал всё это время? "--- тихо сказал я.
"--- Помнишь, мы хотели\ldotst
В тот вечер, когда\ldotst

"--*Печеньки, какао и тёплая постелька, "--- из невидящих глаз Заяц покатились слёзы.
"--- Небо, всё ещё впереди.
Мы вырастим какао и злаки, если найдём хоть горсть почвы.
Мы построим дома даже на лодках или на морском дне.
Всё обязательно будет.

"--*Мне ужасно стыдно, "--- признался я.
"--- Я думал о них все эти дни.
Когда тси умирали, когда Комар\ldotst когда мы решали, что надо улетать, я думал о какао и печенье.

"--*Знаешь, Небо, "--- заметил Фонтанчик, "--- полагаю, что именно поэтому наши предки построили общество.
Может, война и была нужна для жизни, но для процветания важны простые радости "--- домашний уют, вкусная пища, общение, секс, игры и любимое дело.
То, что по сути могут обеспечить твоя собственная голова, крохотный кусочек планеты и два-три живых существа рядом.

Я промолчал.

"--*Небо, мы бы хотели, чтобы ты радовался.
Комар прожил долгую, полную радостей жизнь, какую и заслужил.
Он умер так, как многим и не снилось "--- защищая народ от неминуемой гибели.
Не закрывай любовь в себе.
Твой вид немногочисленен, но среди выживших есть прекрасные апиды, которым она нужна как никогда.

Я посмотрел на измождённое доброе лицо Фонтанчика.
Он всегда находил силы подбадривать других, даже если самому приходилось несладко.

"--*Дружище, иди-ка ты отдыхай, "--- сказал я и погладил его приятную шерстистую шею.

"--*Нет, мой отдых начнётся сейчас, "--- осклабился он и, откинув упавшую на глаза седую гриву, сел за штурвал.
"--- Небо, второй пилот.
Заяц, сиди и гладь лягушку, сейчас в окне будет красиво.
Баночка\ldotst хм.
Спи дальше, дружище.

Да, я уже и забыл, что Фонтанчик страсть как любит водить транспорт.
Он предпочитал старые аналоговые системы с громоздкими штурвалами современному нейроинтерфейсу.
Мы часто вылетали с ним в открытый океан.
Этот бескрайний, бездонный океан смывает все невзгоды;
он превращает маленькие проблемы в марлинов, а большие "--- в китов, и они, величественно взрезав гладь, навеки остаются в чистой, насыщенной жизненной радостью лазури.

Наша планета пала за три дня.
Я не мог представить кита крупнее;
должно быть, с ним мог сравниться разве что доисторический лебиапант, который, по преданию ушедших предков, до сих пор лежит где-то в водах Древней Земли.
Однако, глядя на Фонтанчика, который с просветлевшим взглядом настраивал космолёт, глядя на Заяц, которая с интересом следила за другом и повторяла его движения, глядя на мирно спящего Баночку, я поверил в океан.
Я поверил в то, что там найдётся достаточно места даже лебиапанту.

Курс был проложен "--- три парсака, жёлтый карлик, тси-ди-подобная необитаемая планета, мельком упомянутая пленёнными демонами Ордена Преисподней.
Мы наудачу в последний раз разослали сигналы во все места, где могли остаться тси.
Молчание.
В живых осталась только она "--- родная планета, которая мирно спала и не спешила отвечать существу, выросшему в её объятьях и прожившему на ней почти всю жизнь.

Старое название Тси-Ди "--- Мерлин-Ниниана.
Когда я впервые узнал это название, я долго повторял его про себя, удивляясь чудесному ассонансу.
Историки утверждают, что это мужское и женское имена, распространённые во времена Эпохи Богов.
Возможно, так звали первооткрывателей.
Кем они были?
Любовниками?
Братом и сестрой?
Хорошими друзьями?
Всем вместе?
Время надёжно укрыло тайну своим покрывалом.

Корабль медленно набирал скорость.
Небо становилось всё темнее и темнее, а океан засиял, словно грань живого лазерного сапфира.
Вскоре Тси-Ди превратилась в две далёкие яркие точки.
Спи, мой первый и последний дом.
Машина уничтожила тси, но тебя она в обиду не даст.
Это я знаю совершенно точно.

\section{Время на сон}

\epigraph
{Истинное испытание на человечность лишено правил и границ.}
{Леам эб-Салах}

Завывал ветер, тёмным морем колыхалась в алеющем рассвете далёкая поверхность джунглей.
Где-то печально свистела на своей флейте согхо.
Мы впятером быстро и осторожно спускались со ступеней храма.
Один из мальчишек оступился и покатился вниз, почти беззвучно вскрикнув от боли.
Чханэ едва успела схватить его за руку.
Мы остановились.

"--*Я ногу подвернул, "--- тихо плакал мальчик, показывая на ободранную лодыжку.

"--*Ничего, Веточка, терпи, ты же будущий воин, "--- ласково шепнула Чханэ и, пригладив воину чёрную как смоль шевелюру, забросила мальчишку на спину.
"--- Цепляйся крепче.
Лис тебя потом вылечит.
Меня вот вылечил\ldotst

Я обернулся, чтобы кивком поблагодарить Ситриса.
Пожилой воин едва заметно шевельнул пальцами руки.

"--*Под стену, быстро, "--- шикнула Чханэ детям, даже не успев осмыслить жестовый сигнал.

Вскоре мы уже переводили дух, прислонившись спинами к холодной влажной стене храма.

"--*Светает\ldotst "--- обречённо шепнула Митхэ.
"--- Скоро прокричат за утро\ldotst

"--*Не бойся, ученица, успеем, "--- успокоил я её.
"--- На окраине возле реки есть пустое жилище.
Спрячемся там на время.

В это время к Ситрису подошёл Кхарас.
Часовой защёлкал кресалом, и сверху потянуло удушливым, слегка сладковатым конопляным туманом.

\mulang{$0$}
{"--*Все хорошо, "--- тихо сказал Кхарас.}
{``All right,'' Kch\'ar\v{a}s quietly said.}
\mulang{$0$}
{"--- Старик не собирался убивать "--- стрелы были смазаны кураре.}
{``The old man wasn't going to kill --- arrows were dipped in curare.}
\mulang{$0$}
{Скоро воины придут в себя.}
{Warriors will recover in a while.}
\mulang{$0$}
{Как дела, Ситрис?}
{How it goes, S\~\i tr\v{\i}s?''}

\mulang{$0$}
{"--*Всё просто замечательно, Кхарас, как и должно быть, "--- отозвался воин.}
{``Wonderful, Kch\'ar\v{a}s, as it should go,'' the warrior responded.}
\mulang{$0$}
{Такого счастья в его голосе я не слышал никогда.}
{I've never heard his voice so happy.}
\mulang{$0$}
{"--- Можешь отправить всех отдыхать "--- Ликхмас и Чханэ уже далеко.}
{``Let everyone retire for this night, L\=\i kchm\r{a}s and Chh\r{a}n\^ei are far-far away.''}

\mulang{$0$}
{Молчание.}
{Silence.}
\mulang{$0$}
{Я почти слышал, как в голове боевого вождя крутится и скрипит мельничный жёрнов.}
{I could almost hear a millstone rolling and creaking inside warchief's head.}

\mulang{$0$}
{"--*Тху, на алтарь бы тебя, гнилая рыбина, "--- беззлобно проворчал наконец Кхарас и, громко топая, вошёл в здание.}
{``Tch\`u, die on the altar, you rotten fish,'' Kch\'ar\v{a}s muttered without malice, and went inside tromping loudly.}

\part{Пристанище}

\chapter{[-] Изгнание}

\section{[-] Рикошет}

С самого рождения сапиент связан особым неписанным договором.
Не со Старой Личинкой, как полагает Бродячий Народ, но со средой, в которой он родился.
Этот договор определяется многим "--- биологическим видом, собственными генами, культурой, кормильцами.
И всю жизнь, с самых первых дней сапиент следует оговорённой стратегии.
Любое изменение стратегии имеет цену;
порой сапиент даже не замечает, как платит за это изменение.
Если же изменение чересчур сильное, то ответом будет нэшевский рикошет.

Нэшевский рикошет называют по-разному.
В моём родном мире его величали <<невезением>>.
Люди наивно думали, что их преследует чья-то злая воля.
На самом деле нэшевский рикошет приходит за любое отклонение "--- может, ты прогнулся там, где не следовало этого делать, или не прогнулся там, где предписывал твой договор.
Чаще всего от рикошета страдают любители стабильности и порядка, забывая, что мир "--- обитель хаоса и изменчивости.

Стратегия не имеет ничего общего с личным путём.
Путь идёт изнутри и всегда ведёт к процветанию личности.
Стратегия же приносится извне и с упорством быка ведёт туда, куда ведёт, механистично исполняя заложенные в неё инструкции.
Меня, Чханэ, Митхэ, Зонтика и Веточку наши стратегии вели к алтарю.
Однако мы выбрали борьбу, и помощью ли духов, случайностью ли Вселенной мы ехали в одной упряжке "--- отверженные племенем, но живые, здоровые и свободные, как ветер.

Митхэ держала в объятиях спящих мальчишек и смотрела на выбегающую из-под повозки дорогу.
У её губ пролегла складка.

"--*Я не была дома и не попрощалась с кормильцами, "--- сказала она.

"--*Я думаю, для них гораздо важнее то, что ты жива, "--- утешила её Чханэ.

Подруга повернулась ко мне, её пальцы забарабанили по моей ноге:

<<Лис, объясни, почему нельзя было вернуть их домой>>.

<<Для Сада и Цеха действия Храма легитимны, пока не доказано обратное, "--- жестами ответил я.
"--- Меня беспокоит не то, что их заберут новые жрецы, а то, что кормильцы сами вернут детей на алтарь>>.

Чханэ зябко поёжилась и бросила взгляд на детей.
Да, на свете нет истин, к которым не был бы готов юный человек;
но сообщить эту "--- именно эту, именно сейчас "--- ни у кого не повернётся язык.
Впрочем, зная ученицу, я догадывался "--- раз вопрос не прозвучал, значит, ответ уже найден.
Складка у её губ была чересчур глубока.

"--*И тотем Сомнения мы с тобой так и не доделали, "--- добавила Митхэ.

"--*Тотем Сомнения? "--- удивилась Чханэ.
"--- Это ещё что?

"--*Я не уверен, но полагаю, что это секрет, "--- сообщил я.

Митхэ прыснула.

"--*Митхэ, мы постараемся закончить его вместе.

"--*Возможно, что так, "--- девочка хихикнула и утёрла набежавшие слёзы.
"--- А может, и не постараемся.

\razd

На привале Чханэ целый вечер пыталась узнать про тотем Сомнения.
Наконец, после очередного витиеватого вопроса, Митхэ предложила сделать для подруги тотем Любопытства.
Чханэ замерла, обдумывая эти слова, затем расхохоталась и сказала, что поняла.
Поняла ли она на самом деле, ни я, ни Митхэ точно сказать не могли.
Тотем продолжал работать.

\section{[-] Первая любовь}

\spacing

"--*Стоять! "--- раздался позади знакомый голос.
"--- Движение "--- и я всажу тебе стрелу в спину.

Чханэ выругалась.

"--*Стреляй, "--- отозвался я и обернулся.
"--- Ну давай же.

Ликхэ стояла, натянув лук.
Острие стрелы смотрело прямо мне в глаз.

"--*Нам велели взять вас живыми или мёртвыми, "--- сказала Ликхэ.
Даже с расстояния в десять шагов я видел, как дрожит наконечник стрелы.

"--*Живыми ты нас не возьмёшь.
Или стреляй, или мы пошли, "--- откликнулся я и двинулся в кусты.
Чханэ последовала моему примеру.

"--*Лисичка! "--- в голосе Ликхэ я услышал отчаяние.
Я снова обернулся "--- она опустила лук.

Ликхэ подбежала ко мне и обняла, уткнувшись лицом мне в грудь.
Потом отстранилась "--- на её лице осталась лёгкая влажная дорожка.
Губы её дрожали, кривясь в горькой улыбке.

"--*Лисичка, почему она? "--- зашептала женщина, мотнув головой на Чханэ.
В её глазах блестели слёзы.
"--- Я любила тебя с самого детства.
Я красивая, умная, хорошо готовлю, ласковая.
Ведь так?

"--*Да, это так, "--- согласился я.
Чханэ хмуро молчала.

"--*Я бы стала твоей просто так, по одному твоему слову.
Почему из-за неё, бесплодной пьяницы-кутрапа, ты разрушил свою жизнь?

"--*Она "--- мой друг, "--- пожал я плечами.

Ликхэ посмотрела на Чханэ и понимающе кивнула.

"--*Она Плачущий Ягуар, хоть и не носит раскраску, "--- пробормотала воительница.
"--- Она бы встала на мою защиту, даже если бы любила тебя, как я.

"--*Думаешь, я люблю его меньше? "--- проворчала Чханэ, поправив походный мешок.

Ликхэ промолчала.
Потом, легонько чмокнув меня в губы, воительница с извиняющимся видом потрепала Чханэ по щеке и повернулась, чтобы снова уйти на пост.

"--*Идите, "--- бросила она через плечо.
"--- Я никого не видела.

\section{[-] Ничто человеческое не чуждо}

К моему удивлению, бодрствовала не только Анкарьяль.
Тхартху сидела рядом с ней, закутавшись в одеяло.
Я подсел к девушке.

"--*Почему ты не спишь?
Завтра будем идти весь день.

Тхартху зябко дёрнула плечами.

"--*Не хочется.

"--*Это из-за Грей\ldotst Секхара?

Тхартху промолчала.
Я обнял её и потёрся носом о её ухо.

"--*Расскажи, что тебя тревожит.

Губы девушки задрожали.
Глаза наполнились слезами.
Она помолчала, кусая губу.

"--*Знаешь, как мы познакомились с Секхаром?

Я покачал головой.

"--*Однажды он приходил в наш хутор "--- искал какие-то реактивы для протравки.
Он ковал оружие.
И увидел меня, рисующую углём на белой стене.
Он сказал, что я ему нужна "--- придумывать узоры для клинков.
Так он взял меня на работу.
Платил хорошо.
Потом\ldotst потом я просто осталась у него дома.
Вначале хозяйничала "--- варила еду, убиралась.
Потом всё случилось само собой "--- я поселилась и в его постели.

Анкарьяль, прислушиваясь к разговору, незаметно подобралась ближе и тоже обняла Тхартху.

"--*Всё изменилось, когда он сказал мне, кто он на самом деле.
Он не охладел ко мне, не стал грубее.
Но стал чужим.
Бормотал что-то на непонятном языке, делал из воздуха непонятные машины.
Но я осталась, потому что любила его.

"--*Ты любишь его до сих пор? "--- шёпотом спросила Анкарьяль.

Тхартху шмыгнула носом и утёрла влагу, капающую из глаз.

"--*Не знаю, Вишенка.
Наверное, да.
Но я не чувствую себя нужной.
Когда мы сидели за верстаком и травили металл, всё было по-другому.
То оружие, которое он делает сейчас, вполне обходится без моих рисунков.

"--*Он тебя любит, "--- проникновенно прошептала Анкарьяль.
"--- Я это\ldotst

"--*Вишенка, "--- перебила Тхартху, "--- для существа старше земли, на которой я стою, прожить жизнь от младенчества до смерти "--- всё равно, что вымыть ноги перед сном.

"--*Почему же ты пошла с нами? "--- удивился я.

Тхартху всхлипнула и улыбнулась.

"--*Погулять с богами в образе людей, послушать, о чём они говорят "--- это похоже на сказку.
А с вами "--- ещё и на добрую сказку.
Я всегда мечтала оказаться в такой "--- доброй сказке с хорошим концом.

"--*Конец ещё не написан, "--- возразил я.

"--*А я уверена, что он будет хорошим.

Мы немного помолчали.
Тхартху угрелась под тёплым одеялом и немного успокоилась.

"--*Простите меня, "--- смущённо прошептала она.
"--- Вы спасаете мир, а я тревожу вас своими пустяками.

"--*Это пустяки, которые способны перевернуть мир, "--- возразил я.

"--*Наверное, я чересчур строга к Секхару.
У вас ведь тоже было много кого?

Мы с Анкарьяль переглянулись.

"--*Это сложно объяснить, "--- сказал я.

"--*У нас всё чуть иначе, чем у людей.
Аркадиу обычно старается избегать личных отношений, "--- сообщила Анкарьяль.
"--- Хоть он и утверждает, что не корректирует гормональный фон своих тел, но факт очевиден даже для меня.

"--*Ты так говоришь, словно это плохо, "--- возмутился я.
"--- Урождённый демон мог бы и понять, что это необходимо для конспирации.
Кстати, гормоны в норме, я корректирую деятельность лимбической системы "--- внешняя привлекательность должна сохраняться, это внушает людям доверие.
Раньше у меня были отношения с людьми, и дружеские, и половые.

"--*Ага, "--- ухмыльнулась Анкарьяль.
"--- Одна женщина, если не считать той, что дрыхнет в палатке.
Даже моё невинное предложение совокупиться в свободное время он проигнорировал.
А ведь столько не живут, сколько мы друг друга знаем.

"--*Это было десять жизней назад.
Те наши тела давно умерли.
Хватит притворяться, что это тебя задело.

Я постарался дать понять, что разговор закончен, но, к моему удивлению, за тему зацепилась Тхартху.

"--*Ликхмас, расскажи о своей первой жизни.

"--*Мне бы тоже хотелось услышать, "--- непринуждённо улыбнулась Анкарьяль.
"--- А то ты постоянно аккуратно избегаешь этой страницы своего существования.

"--*Ах ты змеиное семя, "--- проворчал я.

"--*Мы с Грейсом имеем право это знать.
Да и, "--- Нар погладила ладонью Тхартху по плечу, "--- девушке, любящей сказки, будет интересно.

"--*Это плохая сказка, Птичка, поверь на слово.

"--*Лисичка, расскажи, "--- Тхартху ткнулась в мою щёку головой.

Да, друзья действительно имели право это знать.

\ldotst История моя началась на планете под названием Драконья Пустошь.
Холодный неуютный мир, освещённый голубым гигантом, находился тогда под властью Красного Картеля.
Демоны практиковали там множество форм деспотизма, но мне это казалось само собой разумеющимся "--- других миров для меня не существовало.
Я ничего не знал о Картеле.
Я даже не подозревал о Развязке Десяти Звёзд и о том, что Драконья Пустошь оказалась в самом центре этого чудовищного поля битвы.

Всё, что знал неграмотный талианский парень "--- однажды отец снял со стены пулевое ружьё, попрощался с матерью\ldotst и не вернулся.

Всё, что видел парень после "--- это толпы голодных беженцев, рассказывающих о вторгшихся в королевство Талиа инопланетных чужаках.
Это дети, которые не понимали, почему мать ушла из дома и воюет против них, почему отец устроил диверсию в родном городе.

Может быть, другой бы, взяв ружьё, пошёл мстить непонятно кому и сложил бы голову на полях сражений вслед за отцом.
Но парень был умён от природы и понял "--- нужно найти врага и сражаться с ним его же оружием.
О магии, которой владели некоторые церковники, в народе ходили легенды\ldotst

"--*Ты хочешь божественную силу? "--- смеялся надо мной демон Картеля, выряженный в смешную рясу.
"--- Иди отсюда, мальчишка.

Мальчишка вместо ответа приставил к голове святого отца пулевое ружьё.

Разумеется, демон мог убить меня, не пошевелив пальцем.
Почему он не сделал этого, знает лишь он сам, Яйваф Солёная Борода из клана Дорге.
Его глаза расширились, а потом он захохотал.
Хрипло и гнусаво.

"--*Скажи, Лис, "--- осторожно вмешалась Тхартху, "--- что ты чувствовал, когда тебя превращали\ldotst в демона?

"--*Представь, что на крыше крохотной хижины в один миг появился огромный дворец.
Примерно это я почувствовал в тот момент.
Дворец должен был раздавить хижину, сравнять её с землёй, но я выстоял.

Яйваф вложил в меня все боевые и тактические навыки, которые имел, не тронув ядро личности.
Это было грубейшим нарушением правил Картеля.
Сказать по правде, демон ко мне привязался "--- феномен, который тщательно замалчивала пропаганда Ада.
Привязанность "--- нулевая эмоция, не имеющая полярности.
Минус-хоргеты тоже могут её испытывать без вреда для себя.

Мальчишка, ставший демоном, не был похож на идеологов-церковников.
Уставший от воя о небесной каре и обещаний райских кущ народ впитывал его слова, как губка.
Многие знали, что он потерял отца, что он жил впроголодь "--- и за ним пошли.
Сначала сотни, потом тысячи, а затем и миллионы.
Демоны Картеля, редко прибегавшие к помощи людей, поняли, что мальчишка может принести им пользу, и оказали ему всестороннюю поддержку.

Поддержку мальчишке оказал и Валериу XII, который впоследствии стал его лучшим другом.
Многие удивлялись несоответствию характера короля и смутных времён, но разгадка была проста.
Природную честность, худшее из качеств политика, Валериу с лихвой компенсировал импульсивностью.
Те, кто думали, что смогут играть с простодушным правителем, дорого за это заплатили.
Простой деревенский парень Люпино, познавший благодаря демонам все тайны интриг и шпионских войн, оказался для короля подарком судьбы.

Мальчишка с детства ненавидел <<войну масок>> и сделал всё, чтобы вывести врага в поле.
И так случилось, что в 1348 году от Зимы Великанов в заснеженной степи Серпенциару, на тракте Виа Галоледика сошлись объединённые силы Ада под командованием молодого амбициозного демона и талиано-саманское ополчение, которое вёл двадцатилетний Аркадиу Люпино.

Ман Великолепный потерпел сокрушительное поражение.
Орден Преисподней не смог исправить положение, и спустя пять лет демоны Ада вынуждены были уйти с Драконьей Пустоши.

"--*Ман не ожидал такого от новоделка, "--- заметила Анкарьяль.

"--*Да уж, "--- согласился я.
"--- Он потом рассказывал мне, что никто никогда его так не удивлял.
Жаль его.

"--*А что с ним? "--- спросила Тхартху.

"--*Его нет, "--- пояснила Анкарьяль.
"--- Казнён Картелем на Запах Воды дождей тысячу как.

"--*А вы можете умереть? "--- удивилась Тхартху.

"--*Увы, Птичка, это так, "--- кивнула Анкарьяль.

В столицу Аркадиу Люпино вернулся победителем.
Король Валериу XII собственноручно надел на него генеральскую гривну.
Стоявший справа от трона Яйваф ухмылялся в солёную бороду.
А слева сидела маленькая девочка и молча, серьёзным детским взглядом смотрела на новоиспечённую опору престола.

Годы летели.
Я занимался государственными делами и не заметил, как девочка превратилась сначала в девушку, а потом и в женщину.
И каждый раз при встрече "--- на дворянском совете, на ужине по случаю приёма иностранных послов "--- будущая королева Скорпия смотрела на меня тем же серьёзным немигающим взглядом.

"--*Королева? "--- ахнула Тхартху.
"--- Это как Первая жрица?

"--*Почти, Тхартху, "--- улыбнулся я.
"--- Но власть её была куда шире.

После смерти короля Скорпия взяла страну в свои железные руки.
И не только страну.
Впрочем, генерал Люпино особо и не сопротивлялся.

Скорпия не могла официально выйти за меня замуж.
В качестве мужа она выбрала слабовольного герцога с гомосексуальными наклонностями, предоставив ему титул и возможность распоряжаться своей половой жизнью, как заблагорассудится.
Постель королевы и значительная часть власти в стране достались мне.

Наши сыновья выросли и стали мужчинами.
Старший, который впоследствии был коронован как Вериту IX Скорпид, догадывался, кто его настоящий отец.
До самой моей смерти у нас сохранились дружеские отношения.

Разумеется, победа над силами Ада давала лишь временную передышку.
Новая волна интервенции началась через пятнадцать лет.

"--*Только на этот раз господином Люпино занялись мы, "--- осклабилась Анкарьяль.

Ман Великолепный, хоть и был отличным тактиком, увы, и в подмётки не годился команде Айну Крыло Удачи.
Она была мастером <<войны масок>> и быстро использовала мою слабость против меня.
Вчерашние союзники обратились во вражеских агентов.

"--*Я хорошо помню этот момент, "--- засмеялась Анкарьяль.
"--- Мы с Айну после недолгой драки связали Кара полем.
Мне даже напрягаться не пришлось.
Я говорю: <<Ты погляди, как он на нас смотрит, волчонок волчонком>>.
А Айну смеётся: <<Ты не поверишь.
Его родовое имя "--- Люпино>>.

Далеко не каждого демона Картеля имело смысл уничтожать.
Данных обо мне было собрано достаточно, и команда Айну приняла решение договориться.
Впрочем, им не пришлось стараться "--- верен я был людям Талиа и королеве Скорпии, а не малопонятному и далёкому Картелю.

Я лично отправился к Скорпии, чтобы убедить её признать власть Ада над Талиа.
Королева долго смотрела на меня, потом на придворных и, сказав: <<Taliani potenta capitulatin, Valeridi not\footnote
{Талианцы могут сдаться, Валериды "--- нет (язык талино). \authornote}>>,
удалилась в свои покои.

Той же ночью Скорпия обмотала шею на сон грядущий упругим жгутом.
Её тело обнаружила служанка.

Я остановился, чувствуя, что поток воспоминаний стал чересчур сильным.
Удивительно, но Тхартху не восприняла моё волнение "--- отражение чувств давно умершего тела "--- как что-то странное.
Для её мифологического сознания не существовало ни демонизации, ни коррекции личности, ни обновлений;
для Тхартху я так и остался парнем с далёкой Драконьей Пустоши, давно не существующим человеком.
Она украдкой погладила меня, совершенно другое тело, по шее.

"--*Странно, "--- заметила Анкарьяль.
"--- В исторических хрониках значится другая фраза "--- <<Honor supremitu\footnote
{Честь превыше всего (язык талино). \authornote}>>

"--*Я был свидетелем, "--- сказал я.
"--- А это "--- выдумка историков Лодемра Сурового, который отправил последнего Валерида "--- Оливиу II Скорпида "--- в изгнание.
История должна учить верности трону, а не превозносить поверженных врагов.

"--*Я не знала, что вы были настолько близки, "--- сказала Анкарьяль.
"--- Ты, наверное, ненавидел нас.

"--*Знаешь, что сказала мне как-то кормилица?
Когда целишься во врага, думай о броненосце, которого приготовишь на ужин, "--- ответил я.
"--- И неважно, что сегодня ты вынужден был убить своего сына.
Другой ребёнок жив и хочет есть.

\section{[-] Общая мечта}

\spacing

"--*Так значит, ты решила уйти? "--- спросил Грейс. "--- Ну что ж, пусть будет
так.

"--*Я мечтала о семье и спокойной жизни, Секхар. И сейчас всё это у меня будет.
Прости.

"--*Тебе нет нужды оправдываться, — покачал головой Грейсвольд и обнял Тхартху.

Тхартху чмокнула Грейса в щёку и повернулась к нам:

"--*Надеюсь, вы не считаете меня предательницей?

"--*О чём ты говоришь? "--- поморщилась Чханэ.
"--- Предательство "--- это если бы ты бросила нас в бою.
Сейчас мы в безопасности и ты можешь делать всё, что пожелаешь.

"--*Тогда прощайте.
И спасибо за всё.

Тхартху кивнула нам и убежала в дом.
Мы задумчиво смотрели ей вслед.

"--*Хай, вот бы\ldotst "--- мечтательно протянула Чханэ.

"--*Тоже, "--- закончила Анкарьяль.

"--*Ага, "--- добавили мы с Грейсвольдом.

В завершение этого короткого, но ёмкого разговора мы вздохнули и побрели дальше, к видневшимся из-за крон деревьев башням Сотрона.

\section{[@] Костёр в кругосветку}

\epigraph
{Медицина и морское дело всегда идут рядом;
игра с морем сродни игре со смертью\footnote
{В языке талино эти слова созвучны "--- morer и mortir. \authornote}.
Невозможно быть доктором, не будучи самую малость моряком.}
{Марке Скрипта}

\spacing

"--*Небо, "--- обратился ко мне Костёр, "--- я понимаю, что я теперь главный врач на материке, но можно ли мне взять отпуск?

"--*Сколько унесёшь, "--- пошутил я.

"--*Половину солнечного года, складную лодку и утяжелённый киль, "--- серьёзно ответил Костёр.

Я поперхнулся.

"--*Утяжелённый?
Ты в кругосветное собрался?

"--*Да.
Хочу проплыть через Тихий океан, заглянуть на вулканические острова.
Мы как раз их не осмотрели "--- дельфины зашли только в дельты нескольких рек.

Тихим океаном колонисты сразу, не сговариваясь, окрестили огромное водное пространство между Китом, Кристаллом и западным побережьем Короны.
Я вначале запротестовал "--- когда мы вернёмся на Тси, где уже есть Тихий океан, это может создать географическую путаницу.
Но на этот раз меня не стали слушать даже друзья.
Так океан и остался Тихим\footnote
{Помимо Тра-Ренкхаля и Тси, семантически похожее название самого большого водного резервуара встречается на 14\% обитаемых планет. \authornote}.

"--*Хорошо, "--- подумав, сказал я.
"--- Бери лодку, киль я попрошу отлить.

Костёр широко заулыбался и убежал.

\section{[*] Митхэ и тигры}

\spacing

"--*О, кажется, у меня есть одна хорошая новость, "--- улыбнулась Эрхэ.
"--- Кто-то собрался ловить тигров.

Воительница подошла к Митхэ и развязала шнурки нашейника.

"--*Вот Атрис обрадуется, когда мы его найдём, да?

Митхэ потрясённо ощупала шею, на которой проступили бледные полосы.
Улыбка Эрхэ увяла, сменившись выражением ужаса.
Она тоже вспомнила.

"--*Золото, это же Хат, да\ldotsq
Золотце\ldotsq
Скажи, что это Хат\ldotse

\section{[*] Ненужный ребёнок}

\spacing

Митхэ пустым взглядом смотрела на растущий живот и теребила рукоять ножа.

"--*Мне не нужен ребёнок от насильника.

"--*Тогда отдай его мне, "--- в глазах Эрхэ застыли слёзы.
"--- Я воспитаю его или найду для него кормилицу.
Для меня важно, что это твой хранитель, и неважно, кто дал ему плоть.
А если он от Хата, Золото?
Что, если он от твоего самого любимого человека?

\spacing

\section{[*] Место Аурвелия}

\spacing

"--*Капита Миция, я обвещать идти с тобвой, да, но\ldotst

Старик замялся.

"--*Ты хочешь остаться здесь? "--- вдруг поняла Митхэ.

"--*Почти, "--- кивнул старик.
"--- Пвойдём, я поквазать.

Аурвелий повёл воинов на одну из самых высоких скал.
Старик с неожиданной сноровкой поднялся на самую вершину и подождал, пока его усталые спутники окажутся рядом.

Открывшееся воинам зрелище захватывало дух.
Эти земли не зря называли Сикх'амисаэкикх "--- <<Страна леса, целующего камни>>.
Тысячи острых, словно копья, скал, звенящие ручьи, высокие деревья и цепкие заросли кустов, издалека кажущиеся мхом.
Стелющиеся облака смешивались с прядями низинного тумана, рвались на тысячи клочьев и исчезали под иссечёнными лучами вечернего солнца.

"--*Этот река идёт к мворю, капита Миция, "--- Аурвелий указал на тонкую вьющуюся ленточку реки, обвивавшую острые скалы и лоскутки сельвы.
"--- Я ввидеть рыба горквона, много-много рыб, она точно прихводить из мворя, чтобы икра.
В дом есть инструмент, всё, что нужно.
Я взять инструмент, сделать крепкий лодка и вниз.
Там я буду с лодка и инструмент, я построю дом!

"--*Я отпускаю тебя, сенвиор Амвросий, "--- улыбнулась Митхэ.
"--- Очень рада, что ты нашёл место для себя.

Старый ноа просиял.

Весь вечер старик на смеси языков взахлёб расписывал, как он построит лодку.

"--*Я сильный, Эрхея, Аксарий, я мочь! "--- твердил он.
"--- Boni mani\footnote
{Умелые руки (ноа-лингва). \authornote} "---
смола много, seko!\footnote
{То, что надо (ноа-лингва). \authornote} "---
доска с паром полюбиться и выгнуться в дуга!

Эрхэ весело смеялась шуткам Аурвелия и хлопала в ладоши.
Акхсар клевал носом.
Ситрис сидел в углу и грустно думал о чём-то своём.

Митхэ спала плохо.
Ей снилась сырая пещера, где жрец приносил богов в жертву безумному человеку.
Ей снились чёрные силуэты людей, не плоские и даже не объёмные, а что-то хуже.
Дороги, по которым ступала Митхэ, проходили по одному месту несколько раз, не перекрещиваясь.
Вокруг летали фонари, которые пытались её убить.
Митхэ яростно рубила эти фонари саблей;
оружие проходило сквозь цветную бумагу, не причиняя ей вреда.
Митхэ отчаянно звала Атриса, но слышала в ответ только далёкую заунывную трель флейты.

Пробуждение пришло резко, словно воительнице гаркнули в ухо.
Занимался рассвет.
Акхсар и Эрхэ возились возле входа с готовкой.
Ситрис лежал рядом и безучастно смотрел в дырявый потолок домика.

"--*Аурвелий умер, "--- вполголоса сообщил разбойник и кивнул на лежащего в углу старика.

Митхэ закрыла глаза.

"--*Из-за чего?

"--*Нет признаков ни удушения, ни отравления, ни кровотечения, ничего.
Он лежит в той же позе, в которой заснул.
Думаю, умер от старости.

\section{[-] Принятие}

Я сидел на циновке перед косоватым, чисто выскобленным столом в полторы пяди высотой.
Четыре чаши стояли пустые, в моей ещё дымилась остывающая густая похлёбка, которую я мимоходом зачерпнул из чьего-то котла.
Я испытывал чувство стыда перед неизвестной женщиной с улицы Стриженого Кактуса, но больше восполнить силы нам с детьми было нечем "--- собранный походный мешок, в котором лежали пища и бутылки с водой, дожидался в тайнике возле дома.

Дети ели жадно.
Вначале я думал, что их просто не кормили новые адепты Храма, но вскоре Митхэ обмолвилась, что её кормилица "--- Разрушитель.
Всё встало на свои места.

"--*А я ел банан один раз, "--- сказал Веточка.
"--- Кормилец принёс золото, которое боги заплатили за моего братика.

Зонтик ничего не говорил ни во время еды, ни после.
Чханэ гладила его по жёстким чёрным волосам, избегая взгляда мальчишки.
Я осознал, что такой взгляд действительно трудно выдержать.

<<Мне надоело напоминать людям, чтобы они были людьми>>, "--- вспомнил я слова Кхотлам.
С тех пор ничего не изменилось.

Полуразрушенная хибарка, когда-то ставшая нам с Чханэ любовным гнёздышком, а теперь и убежищем, обветшала ещё больше, и я постарался привести её в порядок.
Лучи утреннего солнца едва проникали в дом через мутную плёнку бычьего пузыря, освещая заползшие в хибарку корни соседнего дерева.
Где-то на крыше деловито жужжали вокруг гнезда большие полосатые осы.
Почти так же, как и тогда.

В комнату вошла Чханэ и, вытащив из пыльного угла ещё одну циновку, села напротив, поплотнее закуталась в плащ.

"--*Я уложила детей, "--- сказала она, устало взлохматив рукой кофейные с оранжевой искоркой волосы.
"--- Пусть отдохнут.

Я кивнул.

"--*Так кто ты?
Бог?
Один из лесных духов в человеческом обличьи?
Древний герой, восставший из камня?
Откуда в тебе такие силы?

Я промолчал.
План, подразумевавший действие в строжайшей секретности, в самом начале начал трещать по швам.
Я знал, чем это обычно заканчивалось.

Чханэ вздохнула.

"--*Ты Лис?

"--*Да, я Лис.

"--*Уже хорошо, "--- криво усмехнулась Чханэ.

Я вдруг заметил, насколько она повзрослела с нашей первой встречи.
Кихотр при её рождении бросала рука куда более тяжёлая, чем рука Безумного "--- девушке выпала доля гораздо суровее, чем многие могли бы представить.
И среди всех этих трудностей она сохранила человечность и благородство.
План планом, но оставить её сейчас на обочине пути было бы предательством и расточительством ценного материала.
Придётся взять её с собой, а значит "--- дать кое-какие разъяснения.

"--*Я пришёл, чтобы прогнать Безумного, "--- начал я.

"--*И зовут тебя Ликхмас.
Какое совпадение.
Только не нужно держать меня за дуру.

Я мысленно проклял это действительно невероятное совпадение.

"--*Но это правда, Чханэ.

Чханэ несколько секунд пристально всматривалась в мои глаза, словно в глаза чужого человека.
Наконец она кивнула.

"--*Я тебе верю.
А что ты есть?

"--*Я "--- подобие Безумного, заключённое в человеческое тело.

"--*Хочешь прогнать Безумного и сам получать жертвы?

Я поперхнулся.

"--*Нет.
Жертвы мне не нужны.

"--*Понятно.

"--*Что тебе понятно?

Чханэ смотрела в одну точку.

"--*Вместо Лиса в его теле древний бог, Атркха'тху Люм'инэ, который пришёл, чтобы прогнать Безумных.

"--*Почти так.
Только я и есть Лис.

"--*А вот этому я не верю.

Наступил момент предельной откровенности.

"--*Чханэ, "--- сказал я.
"--- Тебе страшно и одиноко.
Это нормально, когда против тебя восстаёт весь привычный мир.
Сейчас моя человеческая часть, которую ты знаешь как Ликхмаса ар’Люм, "--- я, Чханэ, я чувствую то же самое.

"--*Это неправда.
Ты всесильный.
Ты излечил меня.

"--*Я не всесильный.

"--*Хорошо.
Я тебе нужна? "--- прямо спросила Чханэ.

"--*Да, "--- ответил я и, наклонившись, взял девушку за руки.
"--- Иначе я бы ничего не стал тебе объяснять.

Чханэ выпростала свои руки из моих.
Наступило молчание, долгое, как ночь в далёких заснеженных землях за горами.
Наконец девушка оперлась о столешницу руками, мотнула головой, с отсутствующим видом пожевала попавшую в рот <<рыбку>>.

"--*Да что же мне, всю жизнь по этой пустыне скитаться? "--- пробормотала она.

"--*Иди поспи.

"--*Не хочу.

Я снова взял её за руки, и на этот раз она их не отдёрнула.

"--*Если не желаешь идти со мной, если не веришь, что я "--- это я, могу оставить тебя в любом месте, в каком пожелаешь, "--- тихо сказал я.
"--- Ты хотела на Кристалл "--- я отвезу тебя туда.
Если надо, выстрою жилище.
Найдёшь себе мужчину, будешь жить так, как хочешь.

"--*Нет\ldotst

Она замолчала, борясь с собой.
Потом добавила, уже куда уверенней:

"--*Лис, не надо другого мужчины.
Мне ты нужен.

"--*Я же страшный и непонятный, "--- я позволил себе улыбнуться.
Чханэ взглянула мне прямо в глаза.

"--*Пообещай одну вещь, "--- серьёзно проговорила она.
"--- Когда всё закончится\ldotst если мы останемся в живых\ldotst мы проживём жизнь так, как следует.
Мы больше никогда не переступим порог храма.
У нас будет жилище и дети.
Неважно, кто их выносит "--- я хочу вырастить детей.
А до того "--- я с тобой,

\begin{verse}
Пока нож не сотрётся в пыль,\\
Пока мышцы не станут струнами,\\
Пока не вырастут цветы из темени,\\
Пока посмертие не станет небытием\footnote
{Клятва Маликха Ликхмасу, <<Легенда об обретении>>. \authornote}.
\end{verse}

Я вспрыгнул на стол и обнял её, такую сильную, такую нежную\ldotst
Чханэ обмякла в моих руках, и я отнёс её к лежанке.
Сон "--- вот что нужно подруге сейчас.

Чханэ всегда расстилала для нас постель одинаково "--- одно одеяло в изголовье, сложенное гармошкой, два других уголком.
Удивительно, но даже в час самых тёмных сомнений насчёт меня она сделала всё точно так же.
Я уложил Чханэ слева, рядом с мирно посапывающими ребятишками, провёл рукой по волосам, погладил дуги бровей.

"--*Спи, моя женщина.
Нам предстоит многое сделать.

"--*Хай, ну да.
Свергнуть богов, "--- пробормотала она, засыпая.

\section{[*] Лодка}

Завтрак прошёл в молчании.
Все рассеянно глотали пресную кашу с яйцом, думая о своём.

Наконец Митхэ нарушила тишину:

"--*Как ноа хоронят своих?

"--*Давай мы поедим и используем мою чашу как хэситр, "--- устало ответил Акхсар.

"--*Я спросила про обычаи ноа, а не про наши, Снежок.
Ситрис?

"--*Я не припомню какой-то особой разницы, "--- пожал плечами разбойник.
"--- На суше хоронили в песке, а в море, понятное дело, отправляли в плавание.
У них считалось счастливым знамением, если первой тела достигала чайка, а не акула.
Поэтому самым уважаемым ещё и лодка доставалась.
Чтобы уже наверняка.

"--*Лодка, "--- повторила Митхэ.
"--- Он мечтал о лодке.
Кто-нибудь знает, где можно достать лодку?

"--*Золото, ты с ума сошла, "--- буркнул Акхсар.
"--- Мы в глуши.

"--*Хорошо, тогда давайте сделаем сами.
Кто-нибудь знает, как делать лодки?

Акхсар и Ситрис переглянулись и отрицательно помотали головами.

"--*Я не знаю точно, как делать лодки, "--- вдруг подала голос Эрхэ, "--- но мои кормилицы были плотницами, и я знаю один рецепт.
Дайте мне кхамит времени и выберите доски поровнее.

Эрхэ проглотила остаток каши и проворно бросилась в кусты.

Вскоре воительница вернулась с пучком растений и тут же занялась форменным колдовством.
От костра потянуло донельзя противной вонью.
Акхсар и Ситрис, наморщив носы, оттащили верстак едва ли не на четверть кхене и принялись обстругивать доски.

Когда доски были готовы, Эрхэ облила одну пахучим отваром и на глазах изумлённых мужчин лёгким усилием согнула её в полукольцо.

"--*Обычно лодки собирают на стапеле, "--- объяснила Эрхэ.
"--- Однако я в судостроении профан, поэтому загоняйте доски один в один между этих двух камней и сушите так, мы сведём их на клин и стянем по типу бумажного фонарика.
Надо бы, конечно, как-то внахлёст, но ладно, в бездну.
Смолу я уже грею.

"--*Прости, что куда стянем? "--- переспросил Акхсар.

"--*Я сделаю всё сама, Снежок.
И нечего нос воротить от <<луговой воды>>, "--- бросила воительница Ситрису.
"--- Из-за вас, нежные носики, верфи и лесопилки строят в такой глуши, что на ходьбу времени уходит больше, чем на работу\ldotst

К вечеру лодка была готова.
Она оказалась вовсе не такой крепкой, как хотелось бы Аурвелию, но очень красивой.
Акхсар разрисовал кромку борта замысловатым орнаментом, Эрхэ обложила лодку цветами и обвязала яркими ленточками, которые нашлись в дорожном мешке.
Митхэ усадила старика, зафиксировала его верёвками и привязала сухощавую руку к рулю.

"--*Не нужно было всего этого, Золото, "--- хмуро сказал Акхсар.
"--- О мёртвых заботятся лесные духи.
Давай я совершу над ним обряд?

"--*Аурвелий вполне справится сам, "--- отрезала Митхэ и оттолкнула лодку от берега.
"--- Главное, что он понял, чего хочет.

Управляемое мертвецом судно весело понеслось по извилистой речке и исчезло за поворотом.
Путь продолжался.

\chapter{[-] Кипящий котёл}

\section{[-] Доверие Короля}

\mulang{$0$}
{Ощущали ли вы себя когда-нибудь в центре бури?}
{Did you ever feel standing in the heart of the storm?}
\mulang{$0$}
{Взгляните на знакомых.}
{Look at the people around you.}
Среди них обязательно найдётся тот, чьи глаза неестественно спокойны.
Мысль в глубине их зрачков мощна и извилиста, но эта мысль витает в разреженном облаке чувств.
\mulang{$0$}
{Радость их слаба, немощна их печаль, а любовь мимолётна, как осеннее золото Крайнего Севера.}
{Their joy is weak, their sorrow is faint, and their love is fickle like golden fall of the Far North.}
Однако если вы вглядитесь, то на самом краю радужки можно заметить величественное вращение ока бури, которую эти люди поднимают одним своим существованием.
\mulang{$0$}
{Такое невозможно забыть; я знаю "--- в вашей памяти всплыло хотя бы одно лицо.}
{It can not be forgotten; I know at least one face's surfaced from your memory now.}

\mulang{$0$}
{Родись Митрис ар'Люм на другой планете, он мог бы стать инженером или цветочником, и, надо признать, неплохо бы справился со своим делом.}
{If M\={\i}tr\={\i}s ar'Lo\~{e}m was born on another planet, he would be an engineer or a florist, and admittedly the good one.}
Однако в обществе сели его незримая роль "--- роль ока бури "--- подкреплялась вполне реальной властью.
Король-жрец был единственным сели, имеющим право говорить от имени народа.

Королей-жрецов избирали на неопределённый срок "--- до момента, пока советы не решали выбрать нового.
\mulang{$0$}
{Подробностей отбора кандидатов никто не знал, но всех объединяло одно "--- их лица узнавали во всех уголках земель сели, а имена звучали далеко за пределами.}
{Nobody knew all the details how candidates were selected, but they all had one thing in common --- their faces were recognized in all the S\r{e}l\={\i} lands, and their names were told far beyond.}

Я оглядел зал.
Удивительно, но здесь не было той кричащей роскоши, которую я мог бы ожидать от средоточия власти.
Небольшой уступ, сделанное из малахита кресло со слегка потрёпанной тканой накидкой, каменный стол и два ряда деревянных кресел.
От залов тхитронского храма этот отличался только величиной, в остальном всё было то же самое, даже <<ласточкины ниши>>, в которых изредка попискивали птенцы и деловито крутили длинными хвостиками пичуги.

Король-жрец "--- высокий зеленоглазый мужчина с орлиным носом и гордыми тонкими губами "--- встретил меня стоя рядом с креслом, как и полагалось встречать гонца.
Восемь воинов "--- по-видимому, его лучшие люди "--- располагались чётко по уставу: по четверо с каждой стороны стола, двое с щитами "--- ближе, полукругом, двое с духовыми ружьями "--- чуть поодаль.
Где-то в тенях возле стены прятались убийцы.
Анкарьяль знаком показала, что можно действовать согласно вариации номер четыре "--- стоящие в зале опасности не представляли.

"--*Итак.
Ликхмас ар’Люм э’Тхитрон? "--- спокойно осведомился Король-жрец.

Я коротко поклонился.

"--*Он самый, Король-жрец.
Мои друзья\ldotst

Король-жрец едва заметно улыбнулся на этом слове.

"--* \dots Хатлам ар’Мар э’Тхартхаахитр, Секхар ар’Сатр э’Ихслантхар, Чханэ\ldotst Тханэ ар’Катхар э’Тхаммитр.

Друзья по очереди отвешивали короткий поклон.

"--*Хатлам ар’Мар, "--- обратился Король-жрец к Анкарьяль, "--- я наслышан о твоей доблести, но не припомню, чтобы ты предупреждала меня об уходе из Храма или о взятой на себя роли эмиссара.

Разумеется, ироничный вопрос был вполне ожидаемым.
По словам Анкарьяль, Король-жрец был когда-то её любовником.
Недолго.

"--*Король-жрец, "--- заговорила Анкарьяль, "--- позволь Ликхмасу ар’Люм рассказать о цели его визита.
Разговор обо мне может подождать.

Король-жрец кивнул и, разгладив робу, сел на малахитовое кресло.

"--*Говорите.

"--*Я пришёл для того, чтобы поднять вопрос о целесообразности культа Безумного, "--- максимально чётко отчеканил я.

Надо отдать должное Королю-жрецу "--- он не повёл и бровью.
Охранявшие его воины выпучили глаза, а некоторые даже в нарушение устава переглянулись между собой.
И только ближайшие к малахитовому креслу как будто невзначай повернули щиты.

Король-жрец улыбнулся одними губами.

"--*Если не ошибаюсь, Тхитрон обладает определённой автономией.
Если жители этого славного города приняли решение свести счёты с жизнью "--- это их дело.
При чём тут я?

"--*При том, Король-жрец, что народ сели в самое ближайшее время рискует получить тысячи нахлебников вместо одного, "--- непринуждённо подал голос Грейс.
"--- Можно ли назвать жизнью то, что последует за этим "--- сложно сказать.

Доверенные Короля-жреца напряглись и зароптали.

"--*Тихо, "--- бросил Король-жрец.
"--- Если дело серьёзное, следует закрывать глаза даже на очевидное богохульство.
Я понял ваши слова, пришельцы.
И я так понимаю, что вы бы не явились просить у меня закрытой аудиенции, не принеся с собой весомые доказательства.

"--*Мы "--- одни из пришельцев, "--- сказал я.

Король-жрец усмехнулся.

"--*Что-то ещё?

Я вытянул руку и создал яркий источник света над ладонью.
Ничтожная трата энергии, вполне допустимая, чтобы не выдать себя.
По залу прокатился сдавленный вздох.

"--*Дешёвый трюк, "--- буркнул Король-жрец.
"--- Даже не знаю, почему я всё ещё трачу на вас время.

"--*Хорошо, "--- внезапно весело крякнул Грейс.
"--- Король-жрец, давай начистоту.
Что тебя убедит?

"--*Если вы сможете победить моих людей в сражении "--- так и быть, я выслушаю вас до конца.
В наше время военная сила "--- неоспоримый аргумент.

Король-жрец произнёс последние слова с едва заметной иронией.
Грейсвольд тихо вздохнул "--- он тоже понял, что убедить сейчас нужно не Короля-жреца, а его людей.
Хозяин Трёх Этажей уже всё осознал и принял решение.
Ему жизненно необходимо с нами поговорить.

"--*Хатлам ар’Мар способна победить твоих людей в одиночку, "--- поклонился я.
"--- Если ты позволишь, она это с радостью продемонстрирует.

Я схватил за руки Чханэ и Грейса и оттащил их ко входу.
Анкарьяль лениво сняла с пояса саблю и бросила её в угол.

Воины бросились на Анкарьяль одновременно, без предупреждения, строем <<клешня>>.
Просвистели в воздухе яркие оперённые стрелки, пролетел чей-то томагавк.
Однако Анкарьяль двигалась с немыслимыми даже для сели скоростью и точностью.
Её путь, как говорили в народе, был прочерчен сохой небесного пахаря, того, кто сеет звёзды и собирает урожай облаков\footnote
{Имеется в виду комета или метеор. \authornote}.
Не успел я досчитать до трёх, как все десять воинов, включая прятавшихся в занавесях убийц, уже лежали на каменном полу зала без движения.
Анкарьяль, дыша чуть глубже, чем обычно, так же лениво прошла в угол и подобрала саблю.

"--*Они невредимы, Митрис, "--- пояснила она и, подняв одного из воинов, аккуратно посадила безвольное тело на кресло.
"--- Придут в себя ещё до заката.

Король-жрец окинул взглядом заваленный бесчувственными телами зал, медленно встал и указал рукой на скрытую в тени неприметную дверь за мраморным креслом.

"--*Прошу.

\section{[-] Покои Короля}

\epigraph
{Если упадок перестаёт притворяться процветанием "--- его время сочтено.}
{Постулат Элект}

"--*Итак, Ликхмас ар’Люм.
К слову, это твоё настоящее имя?

"--*Моя мать любила <<Легенду об обретении>>, "--- усмехнулся я.

"--*Кихотр, "--- признал Король-жрец.
"--- Не самое счастливое имя.
Говори, что знаешь и что тебе от меня нужно.

В маленьком уютной кабинете Короля-жреца царила темнота.
Единственными источниками света были узенькое высокое окошко в каменной стене, через которое проникал золотистый пучок вечерних солнечных лучей, и жёлтая шарообразная лампа, левитирующая в полупяди от поверхности каменного стола.
Лампа заливала поверхности на расстоянии трёх локтей абсолютно ровным, почти не дающим теней, приятным глазу светом.

Раритет.
Этот ночник остался ещё от предков-тси.
Грейс тут же с восхищённым бормотанием побежал к лампе "--- исследовать устройство.
Мы с Королём-жрецом проводили его взглядами.

"--*Это лампа с Тхидэ.
Не требует ни масла, ни огня.
Иногда я думаю о свете, иногда просто чувствую надобность, и лампа загорается.
Не сразу ярко, а постепенно, так, что глаза успевают привыкнуть.
Прекрасное чудо.

Грейс по-особому нажал на корпус.
Раздался мелодичный звон, и лампа открылась, словно цветок, показав механизм.
Король-жрец поднял бровь.

"--*С лампой всё в порядке, Грейс "--- профессионал, "--- успокоил я его.

"--*Когда что-то ломается, оно издаёт менее приятный звук, "--- резонно ответил Король-жрец.
Анкарьяль ухмыльнулась.

<<Технология у них в крови>>.

В следующий миг нас окатило лёгкой вспышкой гамма-излучения.
Мы дёрнулись как ужаленные.
Грейсвольд тут же выключил лампу и тихо выругался.

"--*Извините, "--- пробормотал технолог.
"--- Сейчас отрегулирую обратно.
Ничего себе ночничок.
Эти тси "--- ненормальные\ldotst
Надеюсь, здесь генератора экрана не предусмотрено?

Король-жрец покачал головой и повернулся к нам.

"--*Итак, я слушаю.
Правильно ли я понял, что вы одни из\ldotst богов?

"--*Мы такие же, как они, но не на их стороне, "--- сказала Анкарьяль.

Король-жрец вздохнул, впервые обнаружив признаки слабости.

"--*Вишенка, ты ли это? "--- едва слышно пробормотал он.

"--*<<Свист лягушки, рассекающий ночь, будет нашей с тобой колыбельной\footnote
{Цитата из стихотворения Эрхэ Колокольчик. \authornote}>>,
"--- так же тихо произнесла Анкарьяль.
Король-жрец вздохнул ещё глубже и повернулся ко мне.

"--*Вы тоже хотите жертв?

"--*Как раз наоборот "--- мы пришли помочь, "--- ответил я.
"--- И нам очень нужно знать, что здесь происходит.

"--*Откуда мне знать, что вы хотите нам помочь?

"--*Я не могу этого доказать.
Тебе придётся поверить мне на слово.

Король-жрец помолчал.

"--*Уже около двадцати дождей ходят странные слухи.
Я никогда не считал существующий порядок должным, но сейчас рушится даже он.

"--*Рушится? "--- переспросил я.

"--*Именно, Ликхмас ар'Люм.
Это не перестройка, как бывает на Перекрёстке, это окончательное, бесповоротное и, главное, всеобщее разрушение.
Жрецы стали лечить и приносить жертвы не по канону.
Они пренебрегают масками, плащами и правилами Отбора, их забрызганные кровью робы стали символом ужаса.
Воины перестали служить родным храмам, подаваясь в наёмники и разбойники.
Даже храмовые воители чаще занимаются террором, а не обучением молодых.
Города стали требовать полной независимости от Тхартхаахитра "--- и это во времена, когда Живодёр опять показал клыки!
Недавно пришли вести, что люди нарушили нейтралитет в Омуте Духов и осквернили святилище, убив шамана идолов.
Это позор для нашего народа\ldotst

"--*Это мы знаем, "--- прервал Короля-жреца Грейс.
"--- Нам нужны данные о людях, причастных к переворотам.

"--*Так вы и об этом знаете, "--- проговорил Король-жрец.
"--- Тогда вы должны понимать, что сейчас не все города отчитываются передо мной.
Все отчёты о смене жречества, которые у меня есть, в этой книге.

Король-жрец открыл резной деревянный шкаф и вытащил переплетённый в кожу том.
Я аккуратно взял книгу из изящных рук и открыл на последней странице.
Грейс схватил раритетную лампу за металлический шнурок и подошёл, чтобы подсветить текст.

"--*Вот, последние два года, "--- Король-жрец ткнул в страницу узловатым пальцем с тусклым кукхватровым перстнем.
"--- Хатрикас.
Смерть трёх жрецов, в том числе Первого.
Смена.
Травинхал.
Смерть десяти жрецов, без подробностей.
Смена.
И главное "--- сравните.
Вот, письмо сорокалетней давности.
Тогда инциденты в храме были событием чрезвычайным, и отношение\ldotst
Вот, расследование, аж на три страницы.
Смена "--- подробные биографии, прежние должности.
Организация работы в условиях недостатка кадров, обнаруженные уязвимости, предложения по их устранению.
О, вы посмотрите, после того случая они даже форум организовали, вот тезисы и результаты.
Вот, кстати, тоже хороший отчёт "--- три дождя назад из Тхитрона, некоего Трукхвала ар'Хэ.
Всё на месте, несмотря на то, что старик остался там чуть ли не в одиночестве.
А теперь взгляните на вот это.
Это отписки, по-другому назвать нельзя.

"--*А какие причины? "--- поинтересовалась Анкарьяль.

"--*Причины разнообразные — диверсии идолов, пылерои, несчастные случаи.
Почему-то очень часто "--- взрывы.
И такие сообщения из восьми городов.
Ещё два замолчали безо всяких сообщений, при том, что караваны оттуда ходят до сих пор\ldotst
Безумные меня озари\ldotst

Король-жрец выхватил у Грейса лампу, подошёл к висящей на стене карте и обвёл пальцами несколько городов.

"--*Землю сели рвут на куски, Митрис, "--- подтвердила Анкарьяль.
"--- И чем больше мы медлим, тем ближе гражданская война.

Король-жрец потрясённо посмотрел на неё.

"--*<<Разделяй и властвуй>>?
Города спорили насчёт земель, были даже военные столкновения!
У меня и мысли не возникало, что это спланированная кампания!

"--*Король-жрец, "--- Чханэ впервые обратилась к нему, "--- эти трое знают, что делать.
Я видела, на что они способны.
Лис\ldotst Ликхмас ар’Люм спас меня после расчленения на алтаре.

"--*Чханэ, "--- я жестом остановил девушку.

Король-жрец сделал круг по комнате и остановился у узенького окна.
Заходящее солнце сверкнуло в его длинных каштановых волосах.

"--*Я знаю, \emph{Чханэ ар’Качхар}, на что способны боги.
Я видел радужное безумие и расплавленные огнём стены.
Вы говорите, что теперь их тысячи.
Но если вы пришли ко мне, значит, есть средство, чтобы их победить?

"--*У нас есть \emph{методы}, "--- веско сказал Грейс.
"--- Но нам нужно множество тех, кто пойдёт за нами.
Мы не знаем, насколько сильны пришлые "--- возможно, нам нужен будет перевес в численности не меньше чем сто к одному.
Мы хотим позвать не только сели, но и ноа, и возможно даже пыле\ldotst

"--*Ноа? "--- резко повернулся к нему Король-жрец.
"--- Почему именно их?

"--*Порт Коралловой бухты "--- важный стратегический пункт, "--- ответил я.
"--- Кроме того, ноа "--- ближайшие родичи сели, а это значит\ldotst

"--*Ничего это не значит, "--- проворчал Король-жрец.
"--- Самые страшные распри "--- всегда между родичами.
И не нужно водить меня за нос Коралловой бухтой.
Я прекрасно помню, что возле Яуляля есть нечто более древнее и ценное.
По-видимому, вы знаете, что это такое, и намереваетесь это использовать как оружие.

Мы с Грейсом переглянулись.
Анкарьяль опустила голову, улыбаясь чему-то.
Король-жрец сделал ещё один круг по комнате.

"--*Вы хотите, чтобы в походе участвовало как можно больше, но, как видите, я теряю власть.

"--*Ну что ж, мы рады, что Король-жрец понимает суть задачи, "--- подытожил Грейс.
"--- Нам нужен народ сели.
Единый и готовый сражаться за свободу.

"--*Означает ли это, что мне следует совершать диверсии во враждебных Храмах?

"--*Храмы для нас потеряны.
По крайней мере пока, "--- сказала Анкарьяль.
"--- Диверсионную войну нам не выиграть "--- один на один демон разделается с любым воином.
Более того, я бы причислила к врагам все Храмы без исключения, так как Безумный сейчас играет на стороне пришельцев, а для большинства жрецов защита народа важнее твоих приказов.
Тебе нужно заручиться поддержкой тех, кто живёт вне храма "--- крестьян, ремесленников и купцов.

"--*Почему заговорщикам просто не совершить переворот здесь?
Они могли бы получить страну, не наделав шума.

"--*Гражданская война выгоднее, "--- объяснил я.
"--- Они питаются страданиями, как и Безумный.
Кроме того, после гражданской войны человек теряет веру в дружбу и родство.
Такими людьми проще управлять.

"--*А как \emph{мне} отличить врагов от друзей? "--- с едва заметной иронией спросил Король-жрец.

"--*Начни с меня, Башенка, "--- сказала Анкарьяль и положила руку ему на плечо.
"--- Друг я или враг?

Король-жрец молча смотрел на женщину.
В его глазах блеснуло давнее, уже подзабытое чувство.

"--*Со столицей я разберусь, "--- пообещала Анкарьяль.
"--- А в дальнейшем полагаюсь на твой ум.

\section{[-] Лампа и щит}

\spacing

<<Так что это за лампа, Грейс?>> "--- знаками спросил я.

<<Это не лампа, "--- оскалился технолог.
"--- Оптический процессорный блок широкого диапазона, переделанный в лампу.
Скорее всего, по причине неустранимой поломки контроллера, вернее, одного из его модулей.
Возможно, часть корабля тси.
А я ещё думаю, какой-то свет подозрительно ровный\ldotst
Давай подробности позже?
Я сутки буду на пальцах объяснять>>.

<<Интересное в памяти нашёл?>>

<<Накопитель зашифрован, я его оцифровал, но, нюхом чую, ничего важного там нет.
Сейчас, увы, это просто ночник Короля-жреца>>.

Когда мы вышли из кабинета, почти все воины уже пришли в себя.
Они поприветствовали поклоном Короля-жреца и чуть более глубоким "--- Анкарьяль.
Оглянувшись на подругу, я увидел на её лице слабую ироничную улыбку.
Люди не меняются нигде и никогда.

"--*Итак, жду вестей, "--- заключил Король-жрец.
Анкарьяль кивнула и решительным шагом направилась к двери, перед этим незаметно зацепив его робу кончиками пальцев.

\mulang{$0$}
{"--*Король-жрец, почему ты нам поверил? "--- тихо спросил я.}
{``Priest-king, why do you trust us?'' I asked quietly.}

\mulang{$0$}
{"--*Ты назвал пришедших с тобой <<друзьями>>, "--- так же тихо ответил Король-жрец.}
{```Friends' is what you called people who came with you,'' Priest-king just as quietly answered.}
\mulang{$0$}
{"--- Это отступление от правил "--- на аудиенции должно говорить <<спутники>>.}
{``It's a derogation, the rules prescribe to say `companions'.}
Ты, как я понимаю, не читал книгу <<Средоточие>>?

Я помотал головой.

"--*Это нормально.
Обычно её более-менее знает только библиотекарь, который делает выписки для купца, прочим жрецам она ни к чему.
\mulang{$0$}
{Кстати, тот, кто владеет городом сейчас, употребил термин <<мои люди>>.}
{By the way, the one who rules this city now, he used the term `my men'.}
Когда приходит беда, она выдаёт себя речами.

"--*А всё-таки "--- почему?

"--*Я бы тебе не поверил, если бы твои слова не согласовались с моими подозрениями и желаниями.
Ты можешь насмехаться надо мной, называть меня легковерным дураком, да и сам я не прочь порой упрекнуть себя, однако мне позарез были нужны такие союзники, как вы.
Помнишь слова клятвы?

"--*Какие именно?

"--*<<Я не наврежу человеку ни моими чувствами\ldotst

"--*\dots ни моим невежеством>>, "--- закончил я.

Король-жрец улыбнулся и, повернувшись, удалился в свой кабинет.

\section{[-] Правосудие}

\spacing

Люди вокруг одного из воинов повернулись к нему лицом и встали в круг.
Мы с Анкарьяль аккуратно отступили в заросли папоротника и затаились, наблюдая за происходящим.

Люди молчали.
Воин, оглянувшись по сторонам, сделал попытку взяться за фалангу "--- в ответ окружившие его показали спрятанные в рукавах ножи.
Один из них "--- хмурый, неулыбчивый мужчина, низкорослый и широкоплечий "--- вышел вперёд.
Анкарьяль шёпотом сообщила мне, что это крестьянин с улицы Летающего Арбуза.

"--*Нетрукх ар’Сар, "--- сказал он сухим монотонным голосом.
"--- Мы "--- сели Тхартхаахитра "--- обвиняем тебя в Насилии и Разрушении.
Сад и Цех независимо друг от друга вынесли тебе приговор "--- изгнание.
Однако Храм не желает считаться с решениями Советов, и было принято решение свершить правосудие там, где это представится возможным, и теми, кому представится такая возможность.

Воин окинул взглядом окруживших его, выискивая слабое место.

"--*Что я совершил и кто предоставил доказательства моей вины? "--- сурово бросил он.

"--*Ты мучил моего ребёнка, а потом убил его, "--- хрипло сказала стоявшая справа женщина.
"--- Трое моих соседей были свидетелями того, как ты повёл мальчика в джунгли.
Изуродованное тело нашёл мой мужчина.

"--*Мы знаем, что это ты.
Отпираться бесполезно, "--- сказал мужчина, говоривший первым, и протянул воину хэситр.
"--- Возьми.

Воин молниеносно схватился за фалангу и попытался ударить крестьянина шипом гарды.
Движение было отточенным до идеала "--- три взмаха, и воин выбрался бы из окружения.
Но смерть, минуя ворота, вошла через калитку;
из рукава женщины вылетел мясницкий топорик, и голова воина с глухим стуком упала на дорожный камень.
Кормилица не собиралась давать шанс тому, кто замучил её дитя.

Собравшиеся, как один, придержали тело и сели на пятки.
Крестьянин поднял голову с ещё не померкшими глазами и, аккуратно взяв её за нижнюю челюсть, вылил ей в рот хэситр.
Из перерубленной глотки потекла смешанная с водой кровь, желваки дрогнули.

"--*Ты был болен, "--- сказал крестьянин.
"--- Болезнь твоя была страшна, и ты страдал непомерно, причиняя боль другим.
Мы были бессильны излечить тебя и могли думать только о жизни для нашего народа.
Лесные духи могут унять твою боль "--- Сан-сновидец погрузит тебя в сон, Обнимающий Сит подарит тебе любовь, а Печальный Митр споёт песню, которая тебя исцелит.
Прости же своих убийц и встреть их у ворот пристанища, когда их время придёт.
Мы помним про кровь на наших руках.

"--*Мы помним, "--- хором отозвались окружающие.
Сидящие вблизи обмакнули рукава рубах в смешанную с дорожной пылью кровь.
Затем все, как по команде, встали на ноги.

"--*Что теперь, Митрам? "--- спросил кто-то у крестьянина.

"--*Наш долг перед спящим выполнен, "--- сказал крестьянин.
"--- Похороните его.
Малыш, Чайка, Стебелёк "--- за реку.
Кусочек, Рыбка, Аромат "--- к себе в квартал.
Остальные за мной.
Кажется, пора напомнить Храму, кем и для чего он был построен.

Люди разошлись бесшумно, как тени.
Кормилица убитого мальчика и, по-видимому, её мужчина подняли тело воина, погрузили на носилки и куда-то понесли, негромко напевая плачущую песню.

"--*Минус один активный агент Картеля, "--- шёпотом констатировала Анкарьяль.
Я кивнул.

"--*Впервые так гладко.
Может, нам даже вмешиваться не придётся.

"--*Придётся, "--- заверила Анкарьяль.
"--- В храме демоны рангом повыше и их больше, они могут справиться и с большой толпой.
Идём на площадь, подождём, пока котелок закипит.

\section{[-] Обычай}

\epigraph
{<<Мысль материальна>> "--- эту идею старательно внушают вам те, кто боится ваших действий.
Мысли и молитвы ничего не значат "--- ни ваши, ни мои!
Можно хоть тысячу лет говорить о прекрасном городе, но пока хоть один человек не возьмёт в руки мастерок, на месте города будут расти девственные леса.}
{Анатолиу Тиу.
Речь перед лиманскими повстанцами}

Когда-то давно, в детстве, я видел, что такое Дело Перекрёстка.

Глубокой ночью раздался чёткий и громкий тройной стук в дверь.
Кормильцы проснулись и тут же начали одеваться.
По-военному.
Делали они всё без излишней торопливости, но и без промедлений.

Минуту спустя Кхотлам, в последний раз поправив <<разбойничьи крылья>> и пояс, собрала на затылке пучок и закрепила его заколкой хэма "--- знаком дипломата и беспристрастного арбитра.
Хитрам протянул мне маленький кинжал, а кормилица ласково сказала:

"--*Лис, малыш, иди в погреб и ложись где-нибудь в укромном месте.

Я взял кинжальчик и спустился на первый этаж.
Меня поприветствовали слуги "--- старый Сиртху и милая Эрхэ, оба в доспехах и при оружии.
Сиртху отвёл меня за руку в погреб и аккуратно закрыл за мной дверь.

Тогда всё закончилось благополучно, но я так и не сомкнул глаз.
Дело было не в духоте и не в том, что спать пришлось на тюках, укрывшись ковром.
Погреб в нашем доме не был чем-то отделённым от внешнего мира "--- там слышался шум ветра, шаги людей, разговоры и смех.
В ту ночь же царила невероятная, потрясающая тишина.
По этой тишине всегда можно отличить Дело Перекрёстка от вторжения.
Проникновение врагов в город "--- это крики, топот и звон оружия.
Люди непрерывно сообщают друг другу, где опасно и какими тропами идёт враг.

Дело Перекрёстка всегда проходит в молчании.
Единственными звуками до того, как всё начнётся, были и остаются они "--- три зловещих стука в дверь.

Так было и сегодня.
Люди просыпались, надевали доспехи и присоединялись к товарищам;
некоторые обходили дома соседей и стучали в их двери.
Однако напряжение было из ряда вон.
На город надвигалась гроза с востока;
тяжёлый воздух давил на грудь, по коже пробегали электрические ящерицы.
Вскоре из квартала ювелиров послышался звон оружия, прервавшийся ужасным, позорным для сели предсмертным криком.
Мы с Анкарьяль переглянулись.
Разумеется, это было сигналом для тех, что засели в храме.

Толпа стеклась на площадь бесшумно, как потоки чёрной воды.
Факелов и фонарей не было "--- их зажигали только в начале церемонии.
Никто даже не переговаривался "--- это не запрещалось, но осуждалось.
До начала никто из людей не знал, зачем их позвали.

Все терпеливо ждали появления виновника торжества.
При всей серьёзности обычая бывали случаи, когда Дело Перекрёстка начинал какой-нибудь горячий юнец, желавший показать себя.
Однако если собравшиеся решали, что вопрос не стоит двух часов сна, зачинщика ждал сезон самой тяжёлой работы.
Такое же наказание грозило тем, кто после третьего удара в дверь остался в постели "--- пренебрежение к судьбе народа сели не прощали.

Наконец в дальнем конце площади появился робкий огонь факела, и все с облегчением зажгли свои.
Многие зашептались, едва перекрывая далёкий вой предгрозового ветра, слабое шуршание горящей травы и треск перегретого дерева.

Несколько десятков человек протащили что-то через толпу.
Люди на мгновение разошлись, образовав зазор, и я смог разглядеть страшную ношу "--- это были человеческие тела в доспехах.

"--*Пытались нас остановить, "--- пояснил толпе высокий мускулистый мужчина с толстой, как дерево, шеей и стал аккуратно раскладывать безвольные тела у подножия храма.
Несколько человек отделились от толпы "--- кто-то узнал родственников и друзей, кто-то просто решил помочь с посмертным обрядом и захоронением.

Зачинщик Дела Перекрёстка "--- эту роль под молчаливое согласие прочих взял на себя крестьянин Митрам "--- встал на вторую ступень пирамиды и, вынув из ножен фалангу, поднял её над головой.

Шёпот моментально стих.

\section{[-] Дело Перекрёстка}

\epigraph
{И сказал Талим воинам: <<Hе подумайте, будто Я пришел, чтобы упразднить Закон или разрушить Порядок;
не пришел Я, чтобы упразднить, но чтобы восполнить недостающее>>.
И сложили воины оружие, и сняли воины панцири, и приветствовали Талима как старого друга.}
{Хакем-Аят, 5:17--18}

"--*Дело Перекрёстка "--- решаем, куда идти, "--- начал Митрам ритуальной фразой, и слова его низковатого хриплого голоса разнеслись над затихшей толпой.
"--- Храм перестал выполнять свой долг.
Храм даёт приют и защиту Разрушителям и Насильникам.
Сегодня мы свершили правосудие над Нетрукхом ар’Сар.
Его вина была доказана на советах Цеха и Сада, но Храм отказался изгнать его.

В воздух взметнулись несколько фаланг в ножнах.
Этим люди показывали, что желают взять слово.

"--*Я прошу собравшихся вне очереди дать слово Тхаласу, чтобы он объяснил смерть этих людей, "--- продолжил Митрам.
Высокий мускулистый мужчина кивнул и тоже поднял зачехлённый топор.
"--- Есть ли те, кто против?

Ответом было молчание.
Митрам вложил оружие в ножны и повёл затёкшей рукой.
Тхалас снял с топора чехол и поднял его над головой.

"--*Люди, которых мы убили "--- воины из храма и купец, который прибыл позавчера, "--- Тхалас приятным, немного скрипучим басом поочерёдно назвал их имена.
"--- Они пытались силой и уговорами посеять смуту, забыв, что Перекрёсток "--- не место для привала.
Манис, слово тебе.

Следующей обнажила фалангу худенькая, как тростинка, болезненного вида женщина.

"--*Мой ребёнок Кхотси недавно заболел.
Он уже шёл на поправку, когда пришёл Кхирас ар’Люм и забрал его для алтаря.
Он не исполнил надлежащего ритуала и не спросил согласия ребёнка.
Ягодка, тебе слово.

"--*Что здесь происходит? "--- раздался строгий окрик с вершины храма.
Люди запрокинули головы, расматривая закутанную в робу фигуру.

"--*Здесь Дело Перекрёстка, Марас, "--- громко сказал Митрам.
"--- Если тебе есть что сказать "--- спускайся.
И остальных позови.

<<Готовность номер один, "--- я ласково подёргал Анкарьяль за прядку волос на Эй-F14. "--- Марас похож на демона. Общая модуляция\footnote
{Общая модуляция "--- модуляция голоса, которая вызывает у произвольной группы сапиентов определённого вида максимальную реакцию в виде определённой эмоции. \authornote},
воздействует на центр страха>>.

<<Общая модуляция?
Самонадеянно, не находишь?
Это всё-таки потомки тси, а не дикари с периферических планет>>, "--- Анкарьяль потеребила меня за штанину на том же языке.

Разумеется, демон знал о присутствии Ада и собирался давить на толпу максимально грубо.
Он вычислил, что перевес в силе будет на его стороне, пока толпа не приняла решение.
Вывод был абсолютно верным.
Если бы нас обнаружили, то дело решилось бы не убеждением людей, а простой дуэлью хоргетов.
А то и экраном "--- мы не знали, какое оборудование Картель здесь успел собрать и установить.

Мы с демоницей аккуратно скользнули между людей и заняли место в десяти шагах от подножия храма.
Марас и ещё пятеро жрецов величественно, намеренно не спеша спустились по лестнице.
Из боковых дверей бесшумно вытекли воины.
Четверо смело спрыгнули в толпу и встали вместе со всеми, ещё шесть человек остались стоять у средней стены.

<<Вот они все, конфетки, "--- с улыбкой повертела лучистыми глазами Анкарьяль.
"--- Презрение к сапиентам играет с ними злую шутку>>.

"--*В чём нас обвиняют? "--- сурово спросил Марас, оглядев толпу.
Тон был подобран великолепно, и кое-кто смущённо потупился.

"--*\emph{Тебя} пока ни в чём не обвиняют, Марас, "--- громко сказала Анкарьяль.
"--- Но если ты хочешь повиниться, то дождись своей очереди.

В толпе раздались смешки.
Психологическое нападение было блестяще отбито.
Ближайшие ко мне люди посмотрели на воительницу с одобрением, некоторые окинули настороженным взором меня.

"--*Вы даже не представляете, чью кару вы сейчас навлекли на весь город! "--- неприятно каркнул похожий на тощую ворону жрец.

"--*Не испытывай наше терпение, Эрликх, "--- вдруг заговорил Митрам, и его слова камнем легли на мои плечи.
Крестьянин модулировал речь не хуже демона.
"--- Жрецы "--- исполнители обрядов, владыки над словом и силами природы, но Перекрёсток вне вашей власти.
Сейчас мы не имеем ни лиц, ни золота, ни домов, ни профессий, ни родного племени, сейчас мы все "--- слушающие, смотрящие и говорящие тени.
Поэтому закрой рот и жди своей очереди!

Последние слова прозвучали одновременно с далёким громовым раскатом и повисли в воздухе, словно едкий дым пожарища.
Похожий на ворону жрец промолчал и, нервно переступая сапогами, посмотрел на Мараса.
Тот застыл, подобно статуе.

"--*Ягодка, слово тебе, "--- уже спокойнее добавил Митрам.

Ягодкой оказался молодой парень-плотник.
Он вытащил из ножен фалангу и молча поднялся на уступ.

"--*На моих глазах убили человека.

Убийство "--- событие не из редких;
но было в тоне парня что-то, что заставило всех притихнуть.

"--*Дело было возле хутора Солнечная Поляна.
Я пилил ветку чёрного дерева, когда с тракта прилетел всадник на взмыленном олене.
Вероятно, он принёс какие-то вести с востока.
Но с дерева спрыгнул убийца и заколол всадника.
Вскоре подоспела погоня "--- два воина.
Убийца переговорил с ними, и вместе они оттащили тело к обочине.

Стояла оглушительная тишина.
На время умолк даже ветер.

"--*Я принял происходящее за казнь разрушителя.
Но кое-что смутило меня.
Убийца говорил на неизвестном мне языке, и те два воина тоже.
Они действовали так, словно не желали свидетелей.
Они не совершили посмертного обряда, как полагается людям сели поступать с мертвецами, никто даже не вылил в его рот полагающийся напиток.
А главное "--- на его руке не было никаких знаков.
Рука лежала в мою сторону, и в этом я ручаюсь.
В убийце я узнал Митракх ар’Кхир.

Толпа, как один человек, посмотрела на среднюю стену, на стоявшую поодаль низкорослую коренастую воительницу.
Внезапно руку с жреческим кинжалом поднял старик со взглядом и повадками хищной птицы.

"--*Я всё сказал.
Хитрам-лехэ, тебе слово, "--- парень кивнул жрецу и опустил оружие, поморщившись от боли в уставшем плече.

"--*Может ли кто-то подтвердить слова Хонхо ар’Лотр? "--- осведомился Хитрам.

Молчание.

Жрец усмехнулся:

"--*Так я и знал.
Его слова так же нельзя подтвердить, как и слухи о насилии над детьми в Храме, как и многое из того, что я услышал здесь.
Я уважаю Митрама "--- я лечил его детей ранее, одного из них мы с соблюдением всех обрядов увели в храм и принесли в жертву.
И мне кажется, что у него нет повода усомниться во мне.
Да, Митрам.

"--*Я не сомневаюсь в тебе, Хитрам-лехэ, "--- заговорил крестьянин, едва успев вытащить оружие.
"--- Но ты не можешь говорить от имени всех.
Хай?
Да, человек, говори.
Только назовись для начала, я тебя не знаю.

Я аккуратно поднял обнажённую саблю.

"--*Ликхмас ар’Люм э’Тхитрон, жрец.
Я не уроженец и не житель Тхартхаахитра, но я сели.
Есть ли те, кто оспорит моё право на речь?

Толпа молчала.
Я продолжил:

"--*Я понимаю желание почтенного Хитрама защитить подчинённых ему людей.
Но я также служу Храму. И каждый должен помнить, что он не может быть лицом Храма.
У Храма нет и не будет лица.

Откуда-то справа донесся одобрительный гул.

"--*Народ сели всегда давал Храму самостоятельность в обмен на исполнение Храмом его роли.
Все знают, что люди в храме гибнут не только на алтаре, но это никого в городе не касается "--- до того самого момента, пока жрецы и воины не начинают скверно исполнять свои обязанности\ldotst или покрывать Насилие и Разрушение.

Одобрительный гул сменился криками гнева.

"--*Пустая болтовня! "--- рявкнул Хитрам-лехэ.
"--- Как будто мы\ldotst "--- Старик осёкся под тяжёлым взглядом Митрама.

Марас вскинул руку с кинжалом.

"--*Да, почтенный Марас, "--- сказал я.

"--*Сегодняшняя ночь похожа на низкопробный спектакль, "--- выплюнул жрец, и я снова ощутил мощное интонационное давление.
"--- Мы принимаем решение, важное для столицы, а слово берёт уроженец Тхитрона.
Кто ты такой, чужак?
Жрец ли ты?
Друг ли ты?
Кто \emph{знает} об убийствах в храмах?
Покажи мне тех, кто \emph{знает}!

Окружающие сверлили меня отнюдь не дружелюбными взглядами.

"--*Далее.
Мы подвергаем воинов суду дикарей\footnote
{То же самое, что суд Линча. \authornote},
забывая о том, что закон предписывает сначала выслушать рассказ о случившемся из их уст.
Не говоря уже о том, что по тому же забытому вами закону за первое доказанное Разрушение полагается клеймо и изгнание, а не смерть.
Да-да, Митрам, за суд дикарей над Нетрукхом я мог бы обвинить в Разрушении тебя!

Кто-то громко выругался.
Я подпрыгнул и успел увидеть, как люди, взявшись за руки, образовали <<круг спокойствия>> вокруг матери убитого мальчика.
Самообладание изменило ей.

"--*А меня, Марас?!
Это я убила Нетрукха!
У тебя хватит наглости обвинить в Разрушении меня?! "--- кричала она, потрясая мясницким топориком.

Марас продолжил, не дожидаясь, пока она успокоится:

"--*Далее.
Мы обвиняем убийцу в Разрушении на основании совершенно бессмысленных слов одного человека.
На каком таком незнакомом языке может говорить Митракх, если любой сели говорит на всех известных языках или хотя бы знает, как они звучат?
Как можно сделать вывод о преступлениях человека лишь по отсутствию знаков на его руке?
Я такого бреда в жизни не слышал.

На этот раз толпа забеспокоилась сильнее.
Парень-плотник открыл рот и попытался что-то крикнуть, но стоящие рядом властно положили руки ему на плечи, принуждая соблюдать правила.
Митрам хмурился, скрестив руки на груди.

"--*Почему, вместо того чтобы вызвать на совет Митракх ар’Кхир и спросить у неё обстоятельства этого убийства, люди занимались пересудами и распусканием слухов?

<<А ты хорош, ящерица раздавленная>>, "--- одними губами выругалась Анкарьяль.

"--*И самая вкусная часть манго, напоследок, хотя именно с неё уважаемый Эрликх хотел начать.
Да, Митрам, любящий закрывать чужие рты, это я тебе сейчас! "--- Марас ткнул в крестьянина пальцем.
"--- Почему Дело Перекрёстка устроили в тот момент, когда мы ожидали знака кирпича?
В результате я вынужден был спуститься и защищать свою честь вместо того, чтобы защищать город и его\ldotst

"--*А меня ты уже за жреца не считаешь, Марас? "--- раздался с вершины пирамиды величественный рокот.

Король-жрец спускался вниз, на ходу стягивая с себя <<кровавый плащ>> и обтирая им испачканные руки.

"--*Прошу прощения за неподобающий вид.
Ещё прошу слова вне очереди.

Толпа согласно зашумела.
Марас замер, с плохо скрываемой ненавистью глядя на Короля-жреца.

"--*Может быть, кто-то это уже сказал, но я повторю: жрец отличается от крестьянина тем, что не может бросить пост, даже если угрожают его чести и жизни.
Марас приносил другую клятву, не такую, как я? Безумные нарисовали знак кирпича своей дланью, но приносить жертву пришлось единственному жрецу, который ещё не забыл о долге.
Я как мог просил ребёнка потерпеть ради народа "--- иначе мне одному не удалось бы уложить его на алтарь.

Толпа дружно ахнула.
Анкарьяль прикрыла лицо рукой.

<<Земля и небо, знак кирпича.
Они были готовы всю столицу\ldotse>>

<<Спокойно.
Бой ещё не закончен, "--- одёрнул я подругу.
"--- Будь начеку>>.

"--*Слухи, которые ходят в народе "--- отнюдь не слухи, "--- продолжал Король-жрец.
"--- Хотите знать, кто их распускает?
А давно ли вы видели в городе Согхо?
А Ликхлама?
Я надеюсь, что лучшим воинам моего Храма удалось уйти живыми от Митракх.

Анкарьяль шлёпнула меня по руке и стала проталкиваться к Королю-жрецу.
Я осознал всю опасность ситуации "--- он играл по-крупному.

"--*Я даже скажу вам больше, "--- сказал Король-жрец и, схватив под руки старика Хитрама и крепыша Митрама, начал спускаться вниз.
"--- Первая ступень храма (шаг) "--- это не просто его основание (шаг).
Это то (шаг), что сейчас\ldotst

Король-жрец отпустил руки спутников и повернулся лицом к замершим на ступенях жрецам.

"--*\dots отделяет людей от нелюдей.

Я ожидал криков, но толпа ошеломлённо молчала.

"--*Я обвиняю в Насилии и Разрушении тех десятерых, что стоят на ступенях.

"--*А доказательства, Король-жрец? "--- осведомился Марас.
"--- Или ты думаешь, что твой пост освобождает от необходимости доказывать?

"--*Что за губки каждый из вас засунул в правую ноздрю, прежде чем вы вышли сюда? "--- ответил вопросом Король-жрец.
"--*Они и сейчас ещё не растворились, и их можно заметить "--- носы у тебя и стоящих с тобой вздуты справа.
Именно так я определил, что Хитрам-лехэ не с вами "--- у него нос чистый.

Толпа зашлась в тревожном шёпоте.

"--*Бред, "--- рявкнул Марас.

"--*У Митракх и Эрликха губки потекли, оставив на губах опалесцирующие полосы, "--- Король-жрец указал на воительницу и худого жреца.
"--- Я никогда раньше не видел подобное вещество.
Неужели это действительно лекарство, которое помогает избежать радужного безумия, или это мне просто послышалось?

Митракх неожиданно ягуаром прыгнула на Короля-жреца, но между ними тенью возникла Анкарьяль.
Коренастая воительница всхлипнула и обмякла, Анкарьяль отбросила тело в сторону и вскинула окровавленный нож.

\mulang{$0$}
{"--*Даже не думай, "--- прорычала она Марасу.}
{``Don't even think about it,'' she growled at M\"a\=ar\v{a}s.}
\mulang{$0$}
{Её демон предупредительно вспыхнул, обнаружив таким образом себя.}
{Her daemon showed itself by warning ``flash''.}

Толпа как по команде ощерилась в сторону демонов натянутыми луками и духовыми ружьями.

Картель умел признавать поражение.
Они не стали дожидаться конца церемонии.
Марас бросил пару слов стоявшим на ступенях демонам, и те пошли через расступившихся сели прочь, в сторону городских ворот.
Перед тем, как уйти, Марас просверлил взглядом Анкарьяль и криво, судорожно улыбнулся:

"--*Ещё увидимся, Анкарьяль Красный Ветер.

\section{[-] Военный совет}

Закончилась церемония Перекрёстка мирно.
Гроза прошла стороной, на площадь упало едва ли десять капель.
Местный купец объявил голосование, по которому из народа были временно выбраны люди для координации военных сил и отправления обрядов.

Король-жрец вынужден был сам ходить по больным, оставив роль учителя старику Хитраму.
Казалось бы, какая учёба в такие суровые дни?
Однако большая часть горожан восприняла переворот как нечто само собой разумеющееся.
Более того, Хитрам-лехэ целых пять дней посвятил законам сели, пока впечатления детей от Дела Перекрёстка были ещё свежи.

Сбежавшие жрецы вскоре приехали обратно и занялись обычными делами.
На исходе декады вернулся женственный воин-красавец Ликхлам и молча вручил Королю-жрецу свёрток с девятью разномастными прядями волос.
Пряди служители Храма торжественно вплели в убранство маленького, незаметного тотема, который появился у задних ворот храма вскоре после Дела Перекрёстка.

На сегодняшнее совещание Король-жрец опаздывал.
Кресло Анкарьяль тоже пустовало.
Заскучавшие люди катали на каменном столе мячик.

"--*Да что ж такое, "--- не выдержал Хитрам.
"--- Сколько их ждать?

"--*Хитрам-лехэ, ты уже старый, тебе не понять, "--- игриво бросила Чханэ и ткнула меня под рёбра.
Остальные присутствующие рассмеялись и продолжили перекидывать тугой кожаный шарик.
Хитрам сверкнул на воительницу выпученными глазами, но промолчал.

Неразлучная парочка появилась вскоре после разговора, принеся в каменный зал запах прелой травы и густой аромат феромонов.
Анкарьяль непринуждённо села в кресло и углубилась в бумаги.
Король-жрец всеми силами пытался стереть с лица широкую довольную улыбку, не зная, что следы зубов на шее и сухие травинки в волосах и так выдают его с головой.

<<Что? "--- одними губами спросила Анкарьяль у выразительно глядящего на неё Грейсвольда.
"--- Он нам план спас.
Должна же я его отблагодарить>>.

<<Ты его с самого Дела Перекрёстка благодаришь по десять раз на дню>>, "--- напомнил Грейс.
Анкарьяль, сложив большой и указательный пальцы кольцом, показала через него язык и подмигнула мне.

Воины с любопытством смотрели на непонятную жестовую беседу.
Что удивительно "--- адепты Трёхэтажного Храма восприняли новость о демонах очень спокойно.
Никто не просил нас показать чудеса, как бывало на других планетах, через пару дней перестали пялиться, а сегодня меня и вовсе подняли как дежурного по кухне.
Подумаешь, храм захватили демоны-враги, а потом их победили демоны-союзники, не нарушать же из-за этого режим.
Пришлось готовить.

Король-жрец словно прочитал мои мысли.

"--*Ликхмас, сегодня на завтрак мне подали изумительный ягодно-мясной суп, кажется, из пятой главы\footnote
{Имеется в виду пятый раздел <<Десяти тысяч блюд>>. \authornote}.
А на Втором этаже поинтересовались, не собираешься ли ты остаться насовсем.
Есть подозрение, что эти факты связаны.

"--*Просто я вычисляю точное количество ингредиентов по таблице, а не сыплю на глаз.

"--*Мне интересно, в этом Храме хоть кто-то, за исключением библиотекаря, заглядывал в таблицы Книги-кормилицы?
Те самые, которые в конце? "--- вздохнул Король-жрец.

Присутствующие, включая Анкарьяль, вдруг нахохлились.

"--*Впрочем, ладно.
Перейдём к делу\ldotst

\section{[@] Лампа}

Я думаю, что пришло время рассказать о буднях.

Впрочем, что рассказывать?
В общем и целом наша жизнь мало отличается от той, что мы вели на Тси-Ди.
Я понял это сегодня под вечер, когда побережье Коралловой Бухты огласила прекрасная песня, лившаяся непонятно откуда.

<<Ты это слышал?>> "--- позвонила мне Листик.

<<Это не Безымянный?>> "--- осведомилась Пирожок.

<<Это не я, "--- тут же сообщил бог, едва я соединил его в конференцию.
"--- Но звучит очень красиво.
Источник звука "--- обшивка Стального Дракона>>.

Я отправился к кораблю.
Ошибка Безымянного была вполне простительной "--- действительно, обшивка резонировала так, что песню было слышно за два километра.
Но источником звука оказался Баночка, который лежал в техническом канале.

\begin{verse}
И это лучшее на свете колдовство,\\
Ликует солнце на лезвии гребня,\\
И это всё, и больше нету ничего,\\
Есть только небо, вечное небо\ldotst
\end{verse}

"--*О, привет, "--- сказал Баночка, высунув голову наружу.
"--- А я как раз про тебя пою.

"--*Про меня? "--- засмеялся я.
"--- Я не вечный.
Ты в курсе, что тебя слышит половина планеты?

"--*Ой, "--- спохватился плант и добавил по общему каналу:
"--- <<Извините, это Закрытая-Колба-с-Жизнью в техническом канале номер пятнадцать Стального Дракона.
Я не думал, что будет настолько громко>>.

<<Мне понравилось, "--- сказала Пирожок, "--- но давай ты вначале вылезешь наружу.
У меня есть система, которая настроена на музыкальные команды из техканалов, и если ты её случайно активируешь, я рассержусь.
Конец связи>>.

"--*Откуда такая красивая песня?

"--*Из романа <<Свесив ноги с края Вселенной>>, Свет-Мерцающего-Осколка, "--- ответил Баночка.
"--- Я подумал, раз уж мы спаслись на литературном персонаже, неплохо было бы просмотреть первоисточник.
И не пожалел "--- полночи пролетели, как гамма-квант.
Послушай, помоги откалибровать уровень.
Кодить лень, и так уже час провозился.

"--*Давай я, "--- предложил я и достал компьютер.
"--- Как раз проветрюсь, на травке посижу.

"--*Хорошо, когда рядом есть друг, "--- улыбнулся Баночка.
"--- Здесь столько всего нового\ldotst
Сегодня выяснилось, что некоторые зашитые в ядро системы константы не годятся для этой планеты.
И это на космическом корабле, а не на личном планетном транспорте, могли бы догадаться и вынести их в настройки.
Разберёшься?
Документация плохонькая где-то была\ldotst

"--*Я уже копался в хранилище Дракона, найду, "--- успокоил я его.

Плант переслал мне номера моделей и снова занялся подгонкой деталей.
Вскоре программа была написана, уровень откалиброван, и я блаженно развалился в тени Дракона.
Труд "--- это радость, но безделье "--- счастье.

Мимо меня пролетела деталь вместе с чьим-то испуганным ругательством.
Я рефлекторно прыгнул на неё и попытался прижать к земле, но, почувствовав, что мой вес уменьшился почти до нуля, оттолкнул деталь обратно.

"--*Ай!

"--*Всё нормально, это трансдуктор, "--- невозмутимо сообщил Баночка.
"--- Его слегка покорёжило во время посадки, лапки спружинили, он сам выпрыгнул из креплений.
\mulang{$0$}
{Ква-ква.}
{Croak, croak.}
Хомяк, ты лучше бы оптикой занялся, чем обращать внимание, что тут где летает!

"--*А я чем занят, по-твоему? "--- осведомился откуда-то Хомяк.
"--- Кстати, вот тебе подарок.
В блоке сгорел контроллер.
Я заменил на новый, сейчас поставлю прошивку.

Сверху прилетел оптический процессорный блок, едва не разбив мне голову.

"--*Небо, если хочешь ещё раз испытать радость материнства, встань под пластину, "--- философским тоном посоветовал Баночка.

Я повертел блок и трансдуктор в руках.
В голову пришла идея.
Я включил мультитул и принялся за работу.

Спустя десять минут всё было готово.
Правда, трансдуктор я впаял не идеально, на шве торчали несколько острых кристаллов, но получился симпатичный антигравитационный мячик.
Я бросил его Баночке.

"--*Не чересчур высокотехнологично для мячика-то? "--- засмеялся Баночка.
"--- Давай лучше лампу сделаю.
Ученье "--- свет.

Баночка подключился к устройству.
Оно оскорблённо зажужжало.

"--*А, накопитель он вытащил, вот досада.
У тебя не найдётся какого-нибудь дохлого?

Я вытряхнул из карманов хлам.
Разумеется, там нашлись три старых накопителя, парящая видеокамера и опреснитель морской воды "--- эта мелочь теряется в первую очередь.
Минуту спустя Баночка поставил на процессорный блок прошивку от ночника и переписал драйвер полуживого контроллера.
Устройство засветилось в темноте мягким жёлтым светом, и мы невольно залюбовались получившейся безделушкой.

"--*И чем мы занимаемся? "--- протянул наконец Баночка.

"--*Может, подарим царрокх? "--- предложил я.
"--- Толку от него всё равно нет, а выглядит красиво.
Уничтожить его у них технологий не хватит, пользоваться этой лампой сможет тысяча поколений.
Безвредная приятная вещь, идеальный подарок.

\mulang{$0$}
{"--*Обойдутся, "--- сказал Баночка.}
{``They'll survive without it,'' Flask said.}
\mulang{$0$}
{"--- Я лучше Кошке подарю, она оценит этот чёрный юмор.}
{``I'd love to give it to Cat, she'll certainly appreciate this kind of black humour.}
\mulang{$0$}
{Кстати, ты Кошку не видел?}
{By the way, have you seen Cat?}
\mulang{$0$}
{Я о ней думал всё утро.}
{I've spent the morning thinking of her.}
\mulang{$0$}
{Может, это любовь, Небо?}
{Could it be love, Sky?}
\mulang{$0$}
{Ты когда-нибудь влюблялся в людей?}
{Have you ever felt in love with a human?''}

Хомяк не выдержал и высунул из технического хода всклокоченную голову:

"--*Да чем вы там заняты?

\section{[@] Мечты создателей}

\spacing

"--*Странно это, "--- сказал я.
"--- Инженеры проектировали корабль, рабочие строили его "--- и всё ради того, чтобы горстка существ неизвестной им цивилизации могла спастись.

"--*Для них это уже не имеет смысла, "--- с сожалением сказал Баночка.

"--*Это имеет смысл для нас, "--- ободрительно сказал Фонтанчик.
"--- Ради этого и стоит жить и работать.
Ведь однажды, хоть через тысячи, хоть через сотни тысяч лет твоя работа может спасти кому-то жизнь.

"--*Жалко, что мы не знаем их имён и даже того, были ли у них имена, "--- сказала Заяц.

"--*Зато я точно знаю, о чём они думали, "--- осклабился Фонтанчик.

Все удивлённо повернулись к нему.

"--*<<Интересно, смогу ли я заставить этот кусок металла летать?>>

"--*О, и я знаю, "--- весело подхватила Заяц.
"--- <<Да когда же начнёшь летать, как надо?>>

"--*<<Мальструктура в девяносто пять нанометров.
Завтра я лично поеду в цех и буду бить их по рукам, пока эти бездельники не научатся калибровать приборы>>, "--- мрачно закончил Баночка.

"--*Никакой романтики в вас нет, "--- обиделся я.
"--- Может, они думали о звёздах?

"--*Ага, только о них и думали, "--- саркастически рыкнул Фонтанчик, и все засмеялись.

\section{[-] Ценная книга}

\spacing

К сожалению, книга обрывалась на самом интересном месте.
Я ещё раз с некоторым сожалением перечитал последние строчки.

"--*Прости? "--- раздался рядом звучный голос Короля-жреца.
Он как раз проходил мимо, и я, по-видимому, в забытьи произнёс последнюю фразу вслух.

Тут у меня родилась идея.

"--*Митрис, ты не занят?

"--*Если тебе что-то нужно, Ликхмас, время найду.

Я жестом подозвал Короля-жреца поближе и показал ему книгу.

"--*Страниц не хватает, а мне очень хотелось бы её дочитать.
Может быть, у вас в библиотеке есть копия?

"--*Хай, "--- задумался Король-жрец.
"--- Впервые слышу про такую книгу.
Но попробовать можно.
Идём за мной.

Я поднялся со скамьи и пошёл вслед за высокой стройной фигурой Короля-жреца.

Вскоре мы очутились возле резной деревянной двери с изображением Удивлённого Лю.
Король-жрец аккуратно постучал и приоткрыл дверь, встретив удивлённый взгляд Тхаласа "--- одного из недавно вернувшихся жрецов.

"--*Здравствуй, Митрис.
Нечастый ты у нас гость.
А ты\ldotsq

"--*Ликхмас ар’Люм э’Тхитрон, "--- сказал я и слегка поклонился.

"--*Тхалас ар’Сатр э’Тхартхаахитр.
Чем могу быть полезен?

Я протянул Тхаласу огрызок своей книги.
Глаза Тхаласа вылезли на лоб.
Помолчав, он вернул её мне.

"--*Я так понимаю, ты знаешь, что это за книга.
Мне она очень нужна, "--- сказал я.

Тхалас бросил взгляд на Митриса "--- тот сделал библиотекарю какой-то знак.
Тхалас кивнул.

"--*Дело в том, что за книгой уже приходили, "--- сказал Тхалас.
"--- Хитрам-лехэ.
Сказал, что ему очень нужно это сочинение, при этом название сказал с ошибкой, как будто только что услышал его от другого человека.
Он был\ldotst сам не свой.
Такое ощущение, что его или били, или сильно испугали.
Я сказал, что такой книги у нас нет.
Пока меня не было, кто-то рылся в книгах, и весьма непочтительно\ldotst
Это копия из Тхитрона?
Насколько я знаю, третий экземпляр хранился именно там.

"--*Да, "--- сказал я.
"--- А где хранится ещё один?

"--*Если с ним всё в порядке, то в Яуляле, в землях ноа, "--- пожал плечами Тхалас и, нахмурившись, погладил самодельный корешок.
"--- Почему книга в таком состоянии?

Я вкратце рассказал, как дневник попал ко мне.
Тхалас кивнул.

"--*Пойдём.
Только давайте быстро и тихо.
Правда, не понимаю, чем она вам поможет "--- никто так и не понял, что там написано.

Мы поднялись в его кабинет.
Тхалас, быстро оглянувшись, нажал на какой-то выступ в стене и с ловкостью иллюзиониста вытащил из ниоткуда книгу.

"--*Спрятал на всякий случай, "--- пояснил он и передал мне фолиант.
"--- Вот.
<<Повесть о Существует-Хорошее-Небо, вожде сели, наезднике Железного Змея>>.

"--*<<Командире Стального Дракона>>, "--- машинально поправил я.

"--*Что? "--- хором спросили Король-жрец и Тхалас, услышав полупонятные, произнесённые с необычным акцентом слова.

"--*Долгая история, "--- ответил я.

Король-жрец понимающе кивнул, Тхалас потупился "--- по-видимому, понял, что пришелец не так прост, как кажется.

Я пролистал книгу и вздохнул.
Увы, этот вариант тоже не был полным "--- книга заканчивалась на середине записи номер сто сорок три.
Следов вырванных страниц не было "--- видимо, они были потеряны ещё много дождей назад.

"--*Я возьму её переписать? "--- спросил я у библиотекаря.
Тот исподлобья бросил на меня взгляд.

"--*Это очень ценная книга, "--- веско ответил он.

"--*Какой бы ценной она ни была для тебя, библиотекарь, вряд ли ты осознаёшь её истинную ценность, "--- столь же веско ответил я.
"--- Ручаюсь за неё жизнью.

"--*Можешь приходить и переписывать сколько угодно, Ликхмас ар’Люм.
Из библиотеки книга выйдет только через мой труп, "--- упрямо ответил жрец.

Я кивнул и аккуратно передал книгу честному библиотекарю.

"--*Благодарю тебя, Тхалас.
Приду завтра с бумагой и пером.

\section{[-] Язык Древних}

Вскоре я уже читал следующую главу, одновременно делая перевод на Эй и отмечая совершенно непонятные технические термины, чтобы потом проконсультироваться у Грейсвольда.
Тхалас вначале сторонился меня, но неизменно пытался заглянуть в мои записи, проходя мимо.
Наконец, на третий день моего пребывания в библиотеке, душа жреца не выдержала "--- он сел рядом и без обиняков попросил меня прочитать ему несколько страниц.

"--*Здесь много вещей, которые ты не сможешь понять, "--- предупредил я.

"--*Если ты постараешься объяснить, я постараюсь понять, "--- ответил библиотекарь.

Я усмехнулся и начал медленно читать вслух.
Комнату наполнили звуки древнего языка, многогранного, математически точного и невообразимо прекрасного по звучанию.
Они шелестели в тёмных углах, катались мягкими комочками пыли по книжным полкам, звенели фарфором чаши и стеклом чернильницы.
Тхалас слушал, открыв рот.
Наконец, прочитав несколько страниц, я спросил его:

"--*Ты всё понял?
Ты не задаёшь вопросов.

Тхалас перевёл дух и улыбнулся, опустив голову.

"--*Я не понял ни слова в описаниях машин и вряд ли когда-нибудь пойму.
Но чувства, которые испытывал Существует-Хорошее-Небо, не угасли со временем.
Книга пылает, "--- жрец неожиданно закутался в плащ и поднялся, чтобы уйти, "--- и это\ldotst это прекрасно.

Уже подойдя к двери, он обернулся:

"--*Ты исполнил мою давнюю мечту, Ликхмас ар’Люм.
Я услышал своими ушами язык Древних.
Теперь я могу двигаться дальше.

\section{[@] Красный стафилококк (ЖС)}

\spacing

"--*О, биологи явились, "--- заметила Кошка.
"--- Давно вас не было видно.
Поешьте с нами, что ли.
Чем вы там занимались?

"--*Микрофлорой, "--- хриплым басом ответил Цветущий-Мак-Под-Кустами.
"--- Теперь местная живность для нас безопасна.
Почти.
Осталось только новые гены вставить и\ldotst

"--*Тринадцать тысяч штаммов микроорганизмов и восемь тысяч видов многоклеточных, "--- гордо пропищала Листик.
"--- И да, мы измотаны.

Все находящиеся в комнате зааплодировали.
Трудяг тут же потащили к столу "--- кормить.

"--*Как же я хочу спать, "--- пожаловалась Лист за кружкой чая.
"--- Я уже ощущаю себя транспортной РНК, а в ландшафте мне мерещатся кластеры аминокислот\ldotst

Все дружно рассмеялись и начали наперебой советовать, как лучше отдохнуть.

Мак сидел тихо.
Я заметил, что он почти не притронулся к еде.

"--*Тебя что-то беспокоит? "--- тихо спросил я, наклонившись к нему.

"--*Да так, ерунда, "--- пробормотал он.

"--*А всё-таки?

Мак помолчал и пожевал массивной челюстью.

"--*Некоторые штаммы вызывают у меня беспокойство.
И ещё\ldotst Небо, я не хочу говорить при всех.

Мы встали и незаметно вышли за порог.
Дверь с мягким скрежетом закрылась, и весёлый гул товарищей остался где-то вдали.

"--*Небо, всё нужно делать предельно быстро, "--- без предисловий начал Мак, повернувшись ко мне.
"--- У нас нет времени.
Инфраструктура чересчур хрупка и ненадёжна.
Адаптацию тси нужно провести в ближайшее время.

"--*У тебя есть какие-то подозрения?

Мак помялся и потёр волосатой ручищей бритый затылок.

"--*Даже обычные исследования дались нам с трудом.
Не хватало привычных мелочей "--- оборудования, реактивов.
Кое-что собрали на коленке, кое-что заменили аналогами попроще, но всё же.
Дальше будет только сложнее.
Если мы в результате какой-то неожиданности потеряем часть инфраструктуры, тси будут подвержены смертельным болезням.
Чем больше мы медлим с необходимыми манипуляциями, тем выше риск.

Я кивнул.

"--*Надолго хватит иммунитета?

"--*Поколений на триста, не больше, "--- угрюмо сказал Мак.
"--- Дальше микроорганизмы с вероятностью восемьдесят процентов найдут способы обхода, и надежда только на уже имеющиеся у тси защитные механизмы и адаптивную изменчивость.

"--*Мак, ты молодчина, "--- сказал я и, обняв биолога, уткнулся ему в живот.
"--- Я надеюсь, этого хватит.
Успокойся.
Вы с Листик проделали гигантскую работу.
Через триста поколений здесь уже будет подобие Тси-Ди.

Мак слабо улыбнулся, почесал огромной кистью мою голову и погладил мне брюшко.

"--*Как себя чувствует молодёжь?

"--*Молодёжь на подходе, "--- улыбнулся я.

"--*Я могу на тебя рассчитывать?

"--*Я скажу врачам, чтобы они поторопились со всеобщей генотерапией, "--- сказал я.
"--- А ты иди спи.

"--*Боюсь, что мне потребуется медикаментозная кома, "--- грустно заметил Мак и направился к каютам.
"--- Мой мозг на пределе.
Благодарю тебя, Небо.

\section{[-] Красный стафилококк (БВ)}

Я читал, а в моей голове вертелся образ книги, которую я брал из тхитронской библиотеки в далёкой юности.
Цветущий-Мак-Под-Кустами\ldotst

Я бросился в библиотеку.

"--*Ликхмас, "--- вздрогнул Тхалас, едва не опрокинув чернильницу.
"--- Что у вас, демонов, за привычка врываться без стука?

"--*Извини, "--- сказал я.
"--- Мне снова нужна твоя помощь.

"--*Книга?

"--*Да. Цветущий-Мак\ldotst

"--*\ldotst-Под-Кустами, "--- закончил за меня библиотекарь.
"--- <<Машина жизни>>, <<Солнце и вода>>?

"--*<<Микромир Трёх Материков>>.

"--*А я надеялся, что нет.
Кихотр, только просушил и убрал.
Книга всё-таки должна немного полежать после, хай, ты понял\ldotst
Подожди, сейчас достану.

Библиотекарь вскоре вернулся с толстенным томом в руках.
Я открыл ароматную, пахнущую свежим пергаментом и чернилами книгу "--- со страниц посыпалась тонкая силикагелевая пыль.

Да, это было то, о чём я думал.
Отчёт о микрофлоре Тра-Ренкхаля, наиболее консервативных антигенах, возможных путях мутации и способах противостояния.
Схемы из квадратиков "--- последовательности аминокислот и нуклеотидов.

Единственной опасной инфекцией сели была болотная лихорадка.
Неужели Мак не знал о возбудителе?

Страница пятьдесят один развеяла мои сомнения.

<<Красный стафилококк, штамм 32.
Потенциальный патоген, атипичная изменчивость>>.

После описания четверть страницы занимал постскриптум, который из поколения в поколение добросовестно воспроизводили переписчики.
Длинный, полный злости и неприличных в обществе тси слов.
К сожалению, я не смогу их перевести "--- смысл будет безнадёжно потерян.
Подставьте вместо них самые неприличные слова, которые употребляются в вашем обществе.

\begin{quote}
<<Как же я устал, (ругательство).
А самое, (ругательство), замечательное "--- программа FADA-1101, которая могла бы это просчитать, осталась дома, (ругательство).
Да чтоб эту Машину (ругательство), и стафилококк этот (ругательство), и компьютер этот (ругательство), и умников, которые вовремя компьютер не перепрошили, сто раз (ругательство).
И да, Лист, Морковка и Гайка, мне абсолютно (ругательство), это мой (ругательство) отчёт и я пишу такую (ругательство), какую считаю нужным>>.
\end{quote}

"--*Хай, "--- вдруг оживился Тхалас, "--- помню эту страницу.
Иногда переписчики дурачатся, оставляют друг другу записочки на полях, но чтобы Древние\ldotst
Учитель находил это забавным, а вот мне почему-то не смешно.
Я бы не стал так проклинать даже смертельного врага.
Интересно, кто этот Мелкий-Красный-Виноград?
И что он сделал?

Я улыбнулся. В моей груди росло восхищение великим человеком древности.
Как ни крути, а обещание защитить триста поколений Мак выполнил.
Сражение с микрофлорой чужой планеты учёные-тси выиграли с большим отрывом.

"--*Ликхмас? "--- напомнил о своём существовании Тхалас.

Я закрыл книгу и приготовился объяснять.

\chapter{[-] Король-жрец}

\section{[@] Дети Неба}

К сожалению, мы не захватили с Тси-Ди детских капсул.
Ни одной.
Пришлось по старинке заворачивать малышей в провощённые одеяла;
я натёр два одеяла выделениями восковых зеркалец, а затем немного пожевал их во рту.
Об этом давно ушедшем методе всем апидам рассказывали во время обучения, но у меня даже мысли не было, что когда-нибудь он мне пригодится.

Вскоре в каюту завалились друзья "--- с поздравлениями.
Я только покормил малышей и ещё не успел завернуть их.

"--*Небо, поздравляю! "--- Заяц сияла, как солнышко.
"--- Как личиночки?
Ути-пути!
Хорошие, толстенькие.
Первые апиды-тси, родившиеся здесь\ldotse
Фонтанчик, ты чего\ldotsq

Канин стоял и смотрел на детей со странным выражением лица.

"--*Фонтанчик?

"--*Я это\ldotst
первый раз вижу\ldotst

"--*Серьёзно? "--- удивился Шмель.
"--- А разве у Неба\ldotst

"--*Были! "--- воскликнул Фонтанчик.
"--- Но вы их кладёте в капсулу, и они лежат себе тихо в углу.
Небо только каждый час на кормление отлучался.
А что там в капсуле, я ни разу не видел!

"--*Смотри, "--- засмеялся я.
"--- Можешь даже потрогать.
Только одним пальчиком и осторожно, хорошо?

Фонтанчик осторожно протянул огромную лапу к малышу и тут же отдёрнул.

"--*Он похож на желе.

"--*Косточек у них ещё нет, "--- объяснил Шмель.
"--- И мозга тоже почти нет.
Они только едят и ползают.

Фонтанчик выглядел так, словно ему стало дурно.

"--*Ну как?
Нравятся? "--- лукаво улыбнулась Заяц.

Канин насупился и бросил на меня смущённый взгляд.

"--*Нет.

"--*Зато честно, "--- отметил я, пока другие хохотали.

"--*Я серьёзно, это совсем не смешно, "--- сказал Фонтанчик.
"--- Щенята рождаются слепыми, но по ним сразу видно, что да, это твои дети!
А тут просто большие толстые червяки!

"--*Поэтому мы их и прячем, дружочек, "--- пошутил я.
"--- Вот случится что-нибудь со мной "--- ты их ещё и кормить будешь.
И гладить, чтобы кровь не застаивалась.
И следить, чтобы они окуклились как надо, а не как им захотелось.
А вот потом начнутся ути-пути.

"--*Не, ну если нужно будет, то я конечно\ldotst "--- смутился канин.
"--- Только пусть кормит, гладит и окукливает Заяц, а я на себя ути-пути возьму!

"--*Учти, детишки-апиды очень юркие, "--- предупредила Заяц.
"--- Меня как-то Комар попросил присмотреть за только вылетевшими ребятишками.
Я столько шишек набила, пока пыталась их поймать, а им хоть бы что!

"--*Они даже по стенам и потолку носятся, если есть за что зацепиться, "--- сказал я.
"--- Всё-таки щенята себе такого не позволяют.

"--*Апиды интереснее других детишек, они умненькие, язык схватывают на лету, "--- продолжала делиться впечатлениями Заяц.

"--*И в этот момент рядом обязательно должен быть взрослый, "--- добавил Шмель.
"--- Иначе дети изобретут собственный язык для общения, и отучить их от него очень сложно.
Первые сорок дней после вылета дети не спят.
Вообще.
Поэтому на период выкукливания мы образуем группы "--- пока одни спят, другие гуляют, играют и болтают с этой оравой молодняка.

"--*А правда, что вы после выкукливания не можете своих от чужих отличить? "--- спросил Фонтанчик.

"--*А зачем их отличать? "--- удивился Шмель.
"--- Все свои же, одна популяция.

"--*А ещё у них есть привычка "--- старт-прыжок, "--- Заяц увлечённо схватила Фонтанчика под руку.
"--- У предков апид были крылья, а потомкам от полётов остался только ювенильный рефлекс взлёта.
Выглядит очень забавно.

"--*Уговорили, "--- проворчал Фонтанчик.
"--- Ради таких впечатлений я их ещё и окукливать согласен!

"--*Соглашайся, "--- ухмыльнулся Мак.
"--- С апидами всё же проще, чем с дельфинятами.

"--*А, слышал эту историю десятилетней давности, "--- припомнил Фонтанчик.
"--- На дельфинят забыли надеть маячки, и группа просто уплыла в океан.
Они погибли?

"--*У дельфинов в океане Ди просто не осталось естественных врагов, "--- покачал головой Мак.
"--- Крупные акулы вымерли во время войны Тараканов, и их, насколько я знаю, даже не стали завозить "--- как-то обошлись.
Так что, скорее всего, дельфинята одичали и начали питаться рыбой "--- много ума для этого не нужно.
Кстати, подобное не раз и не два уже случалось.
Вначале диких дельфинов пытались искать, возвращать в цивилизованное общество.
Потом махнули рукой.
Они живут себе в океане, никого не трогают.
У них своя культура, свои языки\ldotst

Я вдруг вспомнил, что в молодости смотрел художественный фильм про океанолога, полюбившего дикую дельфиниху.
Кажется, потом он уплыл с её стаей.
А может быть, и нет.
Я смотрел фильм с апидом, который очень меня любил.
Он гладил мои ноги и пытался поцеловать.
К середине фильма ему это удалось, и я так и не узнал, чем закончилась история.
Молодость "--- чудесная пора.

Фильмы, старые сводки новостей, воспоминания о доме.
Кусочки прежней жизни, которой больше нет и не будет\ldotst

"--*Кошмар какой "--- один биологический вид говорит на десяти разных языках, "--- задумался Фонтанчик.
"--- Это как минимум нерационально.

"--*Считается, что дивергенция языка характерна для технологически неразвитого общества и является предпосылкой для дивергенции сапиентного вида, "--- ввернул Баночка.
"--- Кажется, ещё в позапрошлом тысячелетии проводили наблюдения за дикими лисами.
Между коэффициентом различий в сигнальной системе и частотой скрещиваний есть отрицательная корреляция.

"--*Данные неоднозначны, "--- заметил Мак.
"--- Я бы поспорил с корректностью постановки эксперимента.
Тогда не могли учесть фактор Паука, а точнее, его частный случай "--- <<провал>>\ldotst

"--*Так или иначе, эти данные впоследствии использовались при построении нейронных сетей, "--- заупрямился Баночка.
"--- Метод <<лисья любовь>> использовали даже ещё мы, несмотря на разработку более быстрого К-контроллера\ldotst

"--*Разработчики <<лисьей любви>> обратили фактор Паука в фичу! "--- засмеялась Заяц.
"--- Нейронные сети на ней работали идеально, но пришлось написать кучу костылей, чтобы пристыковать их к более старым стандартным структурам!
Я ещё в школе один лично оптимизировала, думала, с ума сойду!
Но зато сколько гордости потом было "--- кусок моего кода проверила комиссия, вставили в код Машины и\ldotst

Заяц запнулась.
В воздухе повисло неловкое молчание.

Я знал, о чём думали друзья.
<<Лисья любовь>>, несмотря на архаичность и плохую совместимость, не содержала в себе трёх тонких уязвимостей, характерных для К-контроллера.
В том, что смертельная ошибка пошла с него, не сомневался никто.
Но в одном я разошёлся с прочими инженерами: все говорили о случайности, а я интерпретировал некоторые обстоятельства как следы диверсии.
Комар по секрету сообщил, что несколько лет назад была пресечена попытка Картеля внести достаточно странные изменения в генофонд тси.
Задумка удалась бы, если бы не случайность.
Возможно ли, что настолько продуманный план был отвлекающим манёвром?

И что за демон мог настолько быстро разобраться в архитектонике?
Неужели это был изменённый тси, которого пропустили врачи?

<<Этого просто не может быть>>.

"--*Лисы "--- это хорошо, "--- выручил всех Фонтанчик.
"--- У меня одна мышковала рядом с домом, очень ласковая, я с ней общался.
У ног тёрлась, но гладить не позволяла.
Один раз стащила рубашку, которую сшила Заяц, разорвала её в клочья и вывалялась в них "--- ей понравился запах\ldotst

"--*Свежо предание, "--- поморщилась Заяц, явно обрадовавшись перемене темы.
"--- Скажи уже честно, что ту рубашку ты сам порвал, неряха!

"--*Я же тебе показал обрывки, на них шерсть рыжая!

"--*Порвал и лисой натёр!

Заяц кинулась другу на шею, и он вдруг по-щенячьи взвизгнул.

"--*Ай!
Моё ухо!

"--*Будешь знать, как врать, "--- буркнула Заяц, выплюнув попавшие в рот шерстинки.

Канин демонстративно отвернулся от женщины и вдруг, не дожидаясь разрешения, погладил ребёнка.
Личинка вытянула крошечную голову и клацнула челюстями.
Фонтанчик заулыбался.

"--*Они смешные.
А чем ты их кормишь, Небо?
Пчелиным молоком?

Я напряг челюсти и выплюнул на ладонь желтовато-зелёную каплю.
Фонтанчик наклонился, осторожно понюхал и лизнул.

"--*Похоже на медовый крем, "--- резюмировал он.
"--- Слишком сладко, но для торта сгодится.

"--*Ой, а дай мне тоже попробовать? "--- попросила Заяц, просунув голову у друга под мышкой.

"--*Эй, он всё-таки не вас выкармливать должен! "--- засмеялся Шмель.
"--- Хватит детей объедать!

\section{[-] Смерть Короля}

\epigraph
{Выбор собственной смерти "--- высшая привилегия мыслящего существа.
Это похоже на то, как философ, закончив книгу, заверяет её своей собственной подписью.
Порой историки пытаются переписать книгу чьей-то жизни, но покажите мне того, у кого это на самом деле получилось!
Мёртвых нельзя прославить и опозорить более, чем они прославили и опозорили сами себя.
Вдвойне верно сие для тех, кто выбрал смерть самостоятельно.}
{Анатолиу Тиу}

\spacing

Король-жрец посмотрел по сторонам.
Пути для отступления не было.

"--*Сам спрыгнешь или тебе помочь? "--- участливо спросил Марас.

"--*Я не смогу спрыгнуть сам, Марас, "--- с достоинством отозвался Король-жрец.
Враги оскалились.

"--*Как же ты жалок, "--- рыкнул жрец.
"--- От тси осталась только кучка бесполезных генов.

"--*Я не знаю, что такое <<гены>>, "--- смиренно признал Митрис.
"--- Но предки оставили мне ещё кое-что.
Мудрость, заключённую в древних изречениях.
Я не могу оценить всю их глубину, но ведь вы, всеведущие пришельцы, знаете язык тси в совершенстве?

Агенты захохотали.
Митрис достал из кармана робы маленькую книгу и открыл её на последней странице.

"--*<<И погаснут звёзды без мысли.
И бродячие земли остынут без желания.
И Вместилище рассыплется в прах, не осознав своего рождения.
Но просвещённый не познает такой судьбы "--- летописцем чувств его будет пламя, летописцем желаний его станет вода, а последнюю мысль впишут в свиток Великого Ветра.
\emph{Одиннадцать, сорок девять, девяносто два}>>.

Глаза Короля-жреца остекленели.
В следующую секхар два врага осели, захлёбываясь кровью.
Третьим на горячий камень упал слабо улыбающийся Митрис.
Из руки Короля-жреца выпал обсидиановый скальпель и, звеня, укатился в редкую сухую траву.

"--*Kuna\footnote
{Мразь, выкидыш, ублюдок, мусор (сектум-лингва). \authornote},
"--- констатировал Марас, убирая духовое ружьё.
Его помощник вытащил из шеи мертвеца стрелку и столкнул безвольное тело со скалы.

\section{[-] Боль}

\spacing

Анкарьяль сидела на каменной скамье, обняв завёрнутого в пелёнку малыша Листика.
Её глаза смотрели вдаль, рот был приоткрыт "--- лицо человека, перенёсшего психологическую травму.
Я видел такие десятки тысяч раз.

Мы с Грейсвольдом подошли к ней.
Толстяк крякнул и сел рядом, я опустился на корточки перед женщиной.
Она крепче прижала к себе ребёнка и отвела взгляд.

"--*Как ты себя чувствуешь? "--- спросил Грейс.

"--*Я его не оставлю, "--- бросила Анкарьяль.

"--*Кто это решил?
Анкарьяль Кровавый Шторм или человеческая женщина по имени Хатлам ар’Мар?

"--*Обе.

Грейсвольд многозначительно посмотрел на меня.

"--*Хватит переглядываться за моей спиной, "--- абсолютно ровным, напоённым смертью тоном произнесла Анкарьяль.

"--*Нар, почему мы здесь? "--- спросил Грейс.

"--*Я не знаю, зачем вы ко мне пришли.

"--*Ты помнишь о задании, которое\ldotst

"--*Мне не нужно напоминать.

"--*И?

"--*Я его не оставлю.
И да, если чей-то демон хоть шевельнётся в мою сторону, то распадётся на составляющие прежде, чем успеет реализовать пару инструкций программы.

Тяжесть угрозы мы с Грейсвольдом осознали далеко не сразу.
Технолог спустя пару секунд инстинктивно отодвинулся от подруги и закутался в плащ "--- он как никто знал, на что способна Анкарьяль Кровавый Шторм.

"--*Аркадиу, пойдём поговорим\ldotst

"--*И о чём вы собрались говорить? "--- поинтересовалась Анкарьяль.

"--*Нар\ldotst

"--*Грейс. Пошёл. Вон.

Тон Анкарьяль говорил, что она на пределе.
Мы с Грейсвольдом поняли, что недооценили серьёзность ситуации.
Тяжело вздохнув, технолог поднялся и ушёл, оставив молча сходящую с ума женщину на меня.

Разумеется, я слышал о таких случаях.
Именно для этого обучали полевых врачей.
Глубокая интеграция демона и сапиентного мозга не могла не сказаться на обоих, и порой демоны начинали действовать нелогично из-за древних инстинктов.

Я сел рядом с Анкарьяль и обнял её, уткнувшись носом ей в шею.
Мы сидели очень долго.
Наконец она чуть повернула ко мне голову.

"--*Аркадиу, ты урождённый человек.
Расскажи, что с этим делать.

Я знал, что сейчас я должен быть предельно честным.
Объединённый разум Хатлам-Анкарьяль распознает любую ложь, и последствия могут быть катастрофическими.

"--*Я представляю, что ты чувствуешь.
Когда-то я потерял важного для меня человека.
Всё прошло "--- разумеется, не бесследно.

"--*Почему именно я? "--- этот совершенно человеческий вопрос принадлежал Хатлам.

"--*Мы не выбираем то, что получаем при рождении, Вишенка, "--- сказал я.
"--- Я знаю, что демон заставляет тебя чувствовать странные, порой очень неприятные вещи.
Но подумай, как много всего он тебе открыл.
Дали, о которых ты даже не подозревала.

Анкарьяль слабо улыбнулась.

"--*Я не уверена, что хотела бы знать об этих далях.
Моё существование\ldotst
Какое же оно жалкое.

Это была самая опасная тенденция "--- вся память демона пропускалась через призму эмоций сапиента.
\emph{Жалкое}.
Так характеризовали наше существование люди, кани, планты и прочие "--- именно этим древним словом.

"--*Сколько тел я сменила?
Сколько жизней прожила?
Служба, работа, исследования.
И мои тела умирали, так и не узнав, что такое счастье.

"--*А что, по-твоему, счастье, Нар?

Анкарьяль посмотрела на меня и улыбнулась:

"--*Мир.

Ребёнок на руках Анкарьяль зашевелился, сонно загулил и зачмокал.
Анкарьяль нежно, на грани слуха зашептала и легонько покачала малыша, пока он не заснул покрепче.

"--*Мне так больно, Аркадиу.

"--*Ты только что потеряла любимого человека, Нар.
Скажи слово "--- и я это остановлю.
Существуют мощные, сверхселективные корректоры настроения\ldotst

"--*По-твоему, это решение? "--- пробормотала Анкарьяль.
"--- А что будет, если я потеряю тебя?
Грейса?
Кто тогда облегчит мою боль?
Что тогда от меня останется\ldotsq

Анкарьяль захлебнулась.
Я промолчал и покрепче обнял подругу.

"--*Ты когда-то терял друзей.
Скажи, что ты сейчас испытываешь к ним?

"--*Они живут во мне.
И для большинства это единственная жизнь, которая у них есть сейчас.
Не думаю, что они были бы против такой.

"--*Он очень умный, "--- сказала Анкарьяль.
"--- Как ты, когда я тебя впервые встретила, только он ещё и красивый.

Я улыбнулся.
Анкарьяль поняла, что сморозила нечто обидное, и тоже виновато улыбнулась.

"--*Я бы так хотела, чтобы он пошёл со мной.
Представь, как было бы здорово, если бы Башенка был с нами.
Жизнь за жизнью.
Каким бы демоном он мог стать со своим умом\ldotst

"--*А ещё он всегда был бы с тобой.

"--*Да, "--- выдохнула Анкарьяль.
"--- Да.
Рядом было бы существо, которое всегда со мной.
Которое никогда не предаст и не ударит в спину.
И вот что от него осталось, "--- Анкарьяль со слезами на глазах поправила пелёнку Листика.
"--- Теперь понимаешь?

"--*Это не он, Нар.
Митрис умер.
И этот ребёнок тоже умрёт.
Но у него будут свои дети.
А у тех детей будут ещё дети.
Это жизнь Ветвей Земли "--- одни передают огонь другим, угасая сами.
Слепая эволюция не даровала им бессмертия в их понимании.

Анкарьяль молчала.

"--*А теперь возьми себя в руки и подумай, что лучше для тех, кто ещё жив.
И помни "--- я всегда с тобой, какое бы решение ты ни приняла.
И Грейс тоже.

Я поднялся, нежно поцеловал женщину в солёные от слёз губы и пошёл в дом.

\section{[-] Тяжёлое решение}

Солнце уже клонилось к закату.
Мы с Грейсвольдом сидели за столом, глотая приготовленную технологом перцовую похлёбку.
Разговор не клеился.

Анкарьяль зашла уже под конец трапезы.
На её опухшем лице ещё не просохли слёзы, но глаза уже горели прежним, стальным блеском.

"--*Аркадиу, я оставлю ребёнка в семье.
Но только до конца войны.
Если мы выиграем "--- я вернусь и заберу его.

Я кивнул.

"--*Ты не будешь изменять мою память, восприятие и эмоциональный фон.
То, что есть, я переживу сама, без постороннего вмешательства.

Я кивнул ещё раз.
Анкарьяль повернулась к Грейсвольду:

"--*А ты бессердечная рыба.

"--*А я тебя всё равно люблю, "--- просто ответил Грейсвольд.
Анкарьяль несколько мгновений смотрела в спокойные глаза технолога, излучавшие глубокий, древний свет бесчисленных жизней.
Потом слабо улыбнулась и кивнула ему:

"--*Я тебя тоже, старый друг.

\section{[-] Дом для ребёнка}

"--*Ты, должно быть, Хатлам ар’Мар?
Заходи.

Дверь нам открыла пожилая, но ещё красивая купчиха.
В волосах её уже протянулись тонкие серебряные нити, кожа на шее слегка потеряла эластичность "--- на вид ей было дождей сто "--- сто двадцать.

"--*Заходи, заходи, и не обращай внимания на бардак, "--- сказала хозяйка.
"--- Ты вовремя, я послезавтра собиралась в Тхитрон вместе с семейством.
Всегда недолюбливала ненужное барахло, а в результате накопила его столько, что пройти негде.

Анкарьяль вошла и оглядела комнату, заваленную походными мешками и ящиками.
Я остался у порога.

"--*А ты чего стесняешься?
Ну-ка заходи, "--- без обиняков бросила мне купчиха.
"--- В моём доме гости у дверей не жмутся.
Вы, наверное, голодны?
Дорога с Тхартхаахитра неспокойная.

"--*Да, я бы не отказалась поесть, "--- согласилась Анкарьяль.

"--*Сатракх! "--- крикнула купчиха.
"--- Поесть принеси гостям в зал!
Садитесь, "--- добавила она нам, указывая на ящики у очага.
"--- Мой мужчина сейчас принесёт мясо и похлёбку.

Анкарьяль кивнула и села на ящик.
Листик на её руках зашевелился и загулил.

"--*Ты её мужчина? "--- тихо спросила у меня купчиха.

"--*Нет, друг.

"--*Хаяй.
Ещё одна женщина, потерявшая мужчину.
Что за время такое звериное?

"--*Как ты узнала? "--- удивился я.

"--*Да у неё на лице всё нарисовано, я ж не первый день живу.
Просто для уверенности спросила.
Садись, садись, ящики не золотые.
Здесь статистика по распределению товаров за нулевые годы Церемонии.
Я бы выкинула к свиньям это топливо, только Первый жрец устроил истерику "--- вдруг понадобится\ldotst
Сатракх, ты там умер?
Надеюсь, что да, потому что если ты не принесёшь еду сейчас же, я исправлю ситуацию предельно жестоко!

Вскоре подоспел и Сатракх "--- коренастый мужчина дождей на двадцать младше хозяйки дома, едва ли не ниже меня.
Холодные глаза его смотрели спокойно, губы под разбитым носом кривились в жутковатой, но гостеприимной улыбке "--- рваную рану на лице зашивали второпях.
В покрытых шрамами загорелых руках дымились две плошки с чем-то очень ароматным, кои он и поставил на столик возле очага.

"--*Кушайте.
Сатракх готовит так, что вам на всей Короне не сыскать лучшей похлёбки с мясом.
Книга-кормилица говорит одно, а у Сатракха своё мнение, и едва ли не лучше.
Ему бы стать хозяином постоялого двора, но он всю жизнь мочил сапоги в походах и радовал своими блюдами воинов.
Да где же Манис-лехэ?
Время уже!
А, вот он, вижу, "--- хозяйка, увидев в окне сгорбленную фигуру, бросилась к дверям.

Манис-лехэ, одетый в протёртую синюю робу, поклонился всем присутствующим.
Это был совсем древний старичок, дождей ста восьмидесяти.
Ходил он с трудом "--- на колене стоял примитивный фиксирующий аппарат, не позволяющий сгибать ногу.
Хозяйка, бормоча под нос что-то своё, выудила для него удобный табурет и, не обращая внимания на слабые протесты старика, укутала больную ногу шерстяным одеялом.
Жрец дрожащими пальцами достал бумагу и перо.

"--*Имена дарителей? "--- обратился он к Анкарьяль.

"--*Хатлам ар’Мар э’Тхартхаахитр, М\ldotst Митрис ар’Люм э’Ихслантхар.

"--*Король-жрец? "--- ошеломлённо шепнула купчиха.
"--- Который недавно\ldotst

"--*Да, "--- ответил я.
"--- Не задавай вопросов.

"--*Поняла, "--- купчиха схватила меня за руку, потом обратилась к старику Манису:
"--- Манис-лехэ, это ребёнок убитого Короля-жреца.
Что будем делать?

Старик задумался.

"--*Ээээ\ldotst
Вот как.
Хай.
Я запишу вас под другими именами, а настоящие имена помещу в мою тайную книгу.
О ней знает только мой ученик, Сорас ар’Хэ.
Если я умру до того, как\ldotst

Мы с Анкарьяль кивнули.
Я уже собирался просить о чём-то подобном, но эти люди поняли меня без слов "--- ещё одна таинственная способность потомков тси.

Когда всё было записано, купчиха жестом позвала Анкарьяль следовать за ней.
Я остался сидеть у очага.

С улицы прибежали двое детей-погодок.
Сатракх оскалил кривой рот и со звериной ловкостью вспрыгнул на четвереньки:

"--*Арр, дети.
Ягуар приготовил вам кушать.

Дети радостно закричали и побежали вслед за кормильцем на кухню.
Из дальних комнат пришла купчиха и, подобрав полы платья и погладив по плечу уплетавшего похлёбку старого жреца, села рядом со мной.

"--*Пусть побудет немного с дитём.

"--*Ты, я слышал, многих приютила? "--- поинтересовался я.

"--*Видел детей?

"--*Да, "--- кивнул я.

"--*Ни одного из них не выносила, "--- рассмеялась купчиха.
"--- Но все мои.
Уже четырнадцатый выводок будет.
Я и с Сатракхом так познакомилась.
Приходит ко мне ночью это замызганное чудовище в шрамах с дитём.
А мне самой недавно воительница оставила Улыбашку.
Я ребёнка беру, делаю записи, всё по правилам.
А он ходит и ходит, да никак не отстанет.
Выставлю за дверь "--- акх! "--- он в окно пролез.
Так и остался ночевать, рыбина обкусанная.
Видать, сам Хри-соблазнитель нас свёл.

Я кивнул.

"--*А насчёт подруги не беспокойся, "--- вдруг добавила она.
"--- Это она сейчас в помутнении, чушь всякую городит.
Улыбашка у меня не первый, до него ещё пятерых воительницы приносили.
Все клялись, что вернутся и заберут.

"--*И?

"--*Поверишь, ни одна даже не навестила, "--- рассмеялась купчиха.
"--- Себя не хвалят, но быть кормильцем "--- это талант.

\section{[-] Птичье гнездо}

Той ночью мы легли <<птичьим гнездом>> вокруг Анкарьяль.
Сели часто использовали этот приём, чтобы облегчить чью-то душевную боль.

Мы с Грейсом имели достаточно общие представления о <<гнезде>>.
Чханэ проявила куда больше сноровки.
Посмотрев на наши телодвижения, она сделала несколько уничижительных замечаний сексуального характера и взялась за дело сама.
Анкарьяль, следя со слабым интересом за её действиями, послушно выполняла все распоряжения.

Первым делом Чханэ подложила под голову Анкарьяль свёрнутое одеяло, а не обычную подушку.
<<Чтобы дышать было легче>>, "--- объяснила девушка.
Затем она жестом подозвала Грейса и уложила его справа, чуть выше, чтобы голова Анкарьяль устроилась у него на груди.
Сама Чханэ легла слева, чуть ниже, и уткнулась носом в шею Анкарьяль.
Последним, слева от Чханэ, лёг я "--- так, чтобы Анкарьяль могла видеть моё лицо.

"--*Запоминайте: самый сильный и надёжный закрывает сзади, самый маленький и ласковый ложится спереди и кладёт голову на грудь\ldotst "--- объясняла Чханэ.

"--*Это ты тут у нас самая маленькая? "--- улыбнулся Грейсвольд.

"--*По возрасту "--- да! "--- нашлась Чханэ.
"--- Не перебивай.
Самый близкий человек ложится так, чтобы было видно его лицо, и беседует с ней.

"--*Да о чём нам с Каром говорить, "--- пробормотала Анкарьяль.

"--*Давай об облачках, "--- предложил я.

"--*Она не хочет, Лис, поэтому заткнись, "--- посоветовала Чханэ.
"--- Просто слушайте дыхание друг друга.
Оно должно синхронизироваться.
Ничего страшного, если кто-то начнёт плакать "--- в <<гнезде>> печали делятся на четверых.

<<Птичье гнездо>> всем понравилось.
Чханэ уложила нас так виртуозно, что мы были и достаточно близко к Анкарьяль, и в то же время наши объятия не были слишком навязчивыми.

Не прошло и нескольких михнет, как все заснули мёртвым сном.

\section{[@] Мальструктура}

\epigraph
{Нет ничего страшного в ошибке, невнимании или промедлении.
Но когда один ошибся, второй недосмотрел, а третий бездействует "--- случаются большие несчастья.}
{Пословица сели}

На Стальном Драконе произошла авария.
Причина банальная "--- мальструктура отводящей пластины реактора\footnote
{Термин переведён буквально, технический смысл неясен. \authornote}.
Её ширина "--- девяносто пять нанометров.
Баночка как в воду глядел.

Мы потеряли кольцевую теплицу.
Тор-отсек просветило, и Хомяк-И-Четыре-Орешка велел наглухо закрыть его.
От спасательной операции тси отказались.
Мак, взбешённый оскорбительными замечаниями техников, отправился вместе с Листик за биологическим оборудованием.
Вдвоём они размонтировали и вытащили большую часть.
Через полчаса работы у Мака пошла носом кровь, но он сделал попытку проникнуть ещё и в тор-отсек.
Листик и ещё несколько тси силой утащили его из сектора четыре и положили в лазарет с лучевой болезнью и в состоянии крайнего психического истощения.
Костёр сказал "--- жить будет.
После техники зашли к Маку и попросили прощения за сказанное.

Специалисты приняли решение вывести Стальной Дракон в космос.
Я пытался их убедить, что не всё потеряно, что Безымянный может залатать пробоину.
Но многие этому решительно воспротивились, и я, к своему стыду, дал волю злости и покинул собрание.

Впрочем, уже через час я пожалел. что ушёл.
Почти все колебавшиеся тси, коих была почти четверть, выступили за вывод корабля.
Мои сторонникам не хватило двенадцати процентов голосов.

Когда Фонтанчик рассказал об этом, меня впервые в жизни посетила мысль о самоубийстве.
Совершенно спокойная и оттого ещё более страшная.

Как же тяжело быть командиром.

\emph{Существует-Хорошее-Небо,}

\emph{уставший инженер компьютерных систем Тси-Ди.}

\section{[@] Умолчание}

\spacing

"--*Почему ты не сказал про неё? "--- заорал Фонтанчик.

"--*Я сам узнал только позавчера, когда проводил плановый осмотр перед пуском, "--- оправдывался Хомяк, опасливо пятясь.
"--- Это есть в отчёте, можете посмотреть.
Но энергия нужна срочно, да и корабль перенёс полёт, и потому\ldotst

"--*Слюни можешь вытереть своим отчётом! "--- Фонтанчик, похоже, едва сдерживался, чтобы не ударить техника.
"--- Это нестандартная нагрузка, тупая ты голова!
Вектор\ldotst

"--*Хватит, "--- остановила друга Заяц.
"--- Что сделано, то уже не повернёшь вспять.
Хомяк, позови всех, будем решать проблему.

"--*Я предлагаю отстранить его от работы! "--- рявкнул канин.

"--*И что это даст? "--- развёл я руками.
"--- Как ни крути, Хомяк лучше всех осведомлён о происходящем.
Это и наша ошибка тоже "--- мы взвалили чересчур много обязанностей на нескольких техников.
Почему отчёт Хомяка прошёл мимо вас?
Вы не знали о пуске?
Почему, в конце концов, техники комплекса не проанализировали ситуацию самостоятельно, а решили положиться на техников систем Дракона?

Фонтанчик яростно зыркнул, плюнул и ушёл.

\section{[@] Отчёт Хомяка}

Техникам удалось медленно снизить мощность реактора и на время остановить распространение повреждения.
Взрыв реактора "--- вещь непредсказуемая.
Остатки топлива могли выжечь наше поселение, а могли и спровоцировать тектонический сдвиг, грозящий катаклизмами всей планете.
На Тси-Ди однажды взорвался реактор "--- записи об этом событии с подробным разбором ошибок нам давали чуть ли не в детстве\ldotst

Я вдруг вспомнил, что в школе нам рассказывали и про один из самых известных случаев применения <<Живой стали>> "--- техника Слово-Рубинового-Лазера.
Тот техник, поняв, что взрыв неизбежен и он остался заблокированным в помещении, применил <<Живую сталь>> и даже после денатурации мозга ещё пять секунд отдавал команды системе, пытаясь локализовать поражение.
Это спасло, согласно заключениям нескольких комиссий, не менее двух миллионов жизней и впоследствии стало главным аргументом в пользу сохранения механизма <<Живой стали>>.
\mulang{$0$}
{Технику установили памятник с эпитафией, ставшей крылатым выражением: <<Растворившись, живу>>.}
{A memorial in honor of him had been made, there's an epitaph that's become a catchphrase --- ``Having dissolved, I live''.}

А ещё учитель сказал, что около тысячи лет после взрыва многие техники замечали присутствие в Сети ещё одного, не ассоциированного ни с каким сапиентом виртуала.
Он значился в логах как Слово-Рубинового-Лазера, выполнял действия обычного техника, время от времени писал отчёты, но источник сигнала установить не удавалось, равно как и отключить неведомую подпрограмму.
Но это, разумеется, или чья-то неумная шутка, или городская легенда, очередная история для походов и пижамных вечеринок.
Я не говорю, что это невозможно технически;
но едва ли тси с такими высокими моральными качествами стал бы тратить бесценное время на интеграцию себя любимого в программную оболочку, принося невинных в жертву слепой стихии.
Я не представляю, что чувствует плант, от которого остаётся только сверхпроводниковая тень, но какой бы ужас не таился в такой форме существования, техник выдержал пять секунд.

Я помотал головой.
Зачем я вообще про него вспомнил\ldotsq

Вскоре все специалисты собрались у берега моря.
Хомяк бодро перечислил известные на текущий момент данные.

"--*\ldotst Двадцать четыре процента.
Сектор четыре просветило\ldotst

"--*Ну что, сволочь? "--- рыкнул Фонтанчик.
"--- Кольцевую теплицу насмерть засветил и радуешься?

"--*Да, не <<теплицу просветило>>, а всего лишь <<сектор четыре>>, "--- поддержала Заяц.
"--- Научись уже нести ответственность за свои ошибки, хотя бы на словах.

"--*Тихо, "--- призвал я к порядку.
"--- Кто-нибудь поискал жизнеспособные клетки в тор-отсеке?

"--*Нет, "--- ответил за Хомяка Фонтанчик.
"--- Этот тупица велел наглухо замуровать тор-отсек.
Я ведь прав?

"--*А ты бы сам туда полез? "--- тяжёлым тоном осведомился Мак.

"--*Согласно протоколу, если радиация в секторе превышает\ldotst "--- уже менее бодро начал Хомяк.

"--*Ясное дело.
Трусливая кучка аминокислот.
Лучше бы ты там оказался, от тебя жареного больше пользы.

"--*Хрустально-Чистый-Фонтан, держи себя в руках, "--- устало сказала Листик.
"--- Проблем хватает.
Хомяк, протоколы протоколами, но теплица у нас была одна, и перед закрытием тор-отсека следовало поискать добровольцев.
Продолжай, что там ещё.

Меня на мгновение потряс спокойный тон Листик, когда она говорила о смерти существа, которому была ближе всех тси.
Но тут я заметил сильную замедленность моргания и понял "--- женщина-плант приняла приличную дозу селективного корректора настроения.

Вскоре отчёт закончился, и группа погрузилась в молчание.
Всем было дано время на обдумывание проблемы.

\section{[@] Дебаты}

Дебаты, как и ожидалось, с самого начала пошли жарче некуда.

"--*Придётся вывести корабль, для нас он уже бесполезен\ldotst

"--*Что значит <<бесполезен>>?
Нужно спасти оборудование, оставшееся в четвёртом секторе, "--- сказала Листик.
"--- Анализатор белков, система для постройки ферментов, реактивы для\ldotst

"--*Ну вот сама и иди за ними! "--- ощерилась Пирожок.
"--- Какая от этой узкоспециализированной техники польза?
Сейчас вы "--- придаток лазарета.
Вам медицинское оборудование вынесли?
Вынесли.
Вот и делайте свою работу!

Листик безучастно, не моргая, смотрела на молодую кани.
Корректор настроения гасил <<отвлекающие>> эмоции, и биологу просто нечего было ответить на несправедливые слова.

"--*Мы тебя, гнилые руки, от всей заразы Трёх Материков привили! "--- вспылил Мак.
"--- Пошли, Листик.
Сами вытащим и дальше будем исследовать, а этой гнили пусть её слова поперёк горла встанут.

"--*Утрудился, герой ты наш, посмотрите на него, "--- бросил кто-то с краю.
"--- Всё кресло просидел.
Иди и теплицу заодно вытащи, раз такой смелый.

Мак побагровел, и я понял: если его не остановить, он действительно сейчас кинется в радиоактивный склеп.

Я встал и попросил тишины.

"--*Я прекрасно понимаю, что все устали.
Давайте не будем забывать "--- все тси без исключения работают на общее благо.
Мак, выслушай меня и можешь идти хоть на северный полюс.

Мак сел и хмуро скрестил руки на груди.

"--*У нас есть один вариант.
Надо попросить Безымянного, "--- сказал я.
"--- Он один сможет подобраться к эпицентру, произвести точную диагностику и залатать пробоину.
В крайнем случае он достанет клетки\ldotst

Договорить мне не дали.

"--*Доверять клетки теплицы хоргету?
Вы все в одночасье с ума посходили? "--- снова разъярился Мак.
"--- Да я лучше сам в тор-отсек полезу!

"--*А если он напортачит, Небо? "--- сказал Фонтанчик.
"--- Поселение сгорит, как бумажка!

"--*Кроме того, можем ли мы ему доверять реактор? "--- подал голос Хомяк.
Остальные закивали, выражая согласие.

"--*Он уже пытается нами манипулировать, "--- мрачно сказал Баночка.
"--- Эта странная музыка\ldotst

Кошка молчала, обхватив руками голову.
Я напрасно смотрел на неё, ожидая поддержки.

"--*Нужны ли мы ему? "--- добавил кто-то.
"--- Все узлы планетарной защиты, кроме последнего, готовы!
Воспроизвести технологию не составит\ldotst

"--*Даже если он честен с нами, в чём я лично уверен, Небо, "--- веско сказал Фонтанчик, "--- он ничего не знает о технологии, с которой будет иметь дело.
То есть декодировать то, что он там увидит, придётся нам, в реальном времени.
Мы, в свою очередь, ничего не знаем о восприятии Безымянного, все адаптационные обновления он осваивал сам, без нашего участия.

"--*Кроме того, для него это существенные траты, "--- добавила Заяц.

"--*Мы возместим убытки позже, "--- сказал я.
"--- Главное "--- действовать быстро.

Окружающие возмущенно зароптали.

"--* <<Возместим>>? "--- развела руками Пирожок.
"--- Ты сказал <<возместим>>?
Может, мы ему сразу поклоняться начнём, как те осьминоги?

"--*Я предлагаю вам единственный шанс сохранить корабль! "--- потерял я терпение.
"--- У меня ощущение, что вы уже похоронили Дракона!
Как сделать золото без корабля?
Где вы будете производить ядерный синтез "--- на примусе?
Или глиняную печь соберёте?

"--*Небо, что ещё за <<вы>>? "--- удивилась Заяц.
"--- Ты один из нас, забыл?

"--*Похоже, что да, "--- тихо констатировал Мак.

"--*В бездну корабль! "--- завопил Хомяк.
"--- Это всего лишь техника!

"--*И систему тоже в бездну, ты это хотел сказать? "--- крикнул я.
"--- Это тоже <<всего лишь техника>>?

"--*Сейчас стоит вопрос о нашем выживании!

"--*Да!
И когда он стоял ранее, Безымянный позволил тси жить!

"--*Разумеется! "--- буркнул Мак.
"--- Ему нужны наши технологии!

"--*В любом случае мы должны сохранить корабль, "--- начал я чуть спокойнее.
"--- Это <<спичечная>> технология, поколение Ночи, он может работать даже на дровах.
Мы могли бы удалить реактор\ldotst

"--*И куда мы его денем? "--- закричал Хомяк.
"--- На планете реактор утилизировать нельзя.
Даже если вы выбросите его в глубоком космосе, без двигателя корабль останется там же!

"--*Я бы послушала мнение хорошего врача, насколько фонящий реактор опасен для здоровья, "--- саркастически фыркнула Пирожок.
"--- Где Костёр?

"--*В кругосветном, "--- откликнулся кто-то.

"--*Блестяще, "--- зло затявкала Пирожок.
"--- А как насчёт меня, Небо?
Мне отпуск можно?
Прямо сейчас, от вас, шайки идиотов?

"--*Пирожок, "--- вдруг взметнулась Кошка, "--- я думаю, Костёр получил отпуск потому, что он не использует дружбу для достижения своих низких целей.

Кошка едва успела наклониться "--- Пирожок явно вознамерилась пятернёй снести женщине голову.
На физика тут же навалились трое тси.

"--*Грязная обезьяна, червяга сопливая, "--- яростно булькала и брызгала слюной Пирожок, небезуспешно пытаясь вырваться из мощных объятий Фонтанчика.

Баночка с ненавистью смотрел на Пирожок и тискал свой меч.
Его глаза "--- я даже не подозревал, что такое возможно "--- из тёмно-зелёных превратились в вишнёво-красные.
Я вдруг осознал, что никто из присутствующих не задумывается о простой вещи "--- если маленький плант сейчас выйдет из себя, берег превратится в заваленную трупами арену.
Никто из безоружных тси просто не сможет его остановить.
Кошка, похоже, тоже это поняла и испуганно бросилась ему на шею.

"--*Тихо, тихо, дружочек\ldotst "--- зашептала она.
"--- Милый мой, успокойся\ldotst
Она просто не в себе\ldotst

Перепуганная Кошка плакала и закрывала Баночке глаза и уши ладошками, словно это могло оградить его от потока сквернословия.

Все были чересчур заняты рычащей, ругающейся и дерущейся Пирожок, чтобы обращать внимание на эту тихую борьбу.

\section{[@] Победа усталости}

К счастью, нежность Кошки и сила Фонтанчика победили.
Пирожок тихо извинялась перед братом, которому она едва не сломала руку, Баночка успокаивал Кошку "--- женщину немилосердно трясло от пережитого испуга.
Тси задумчиво притихли.

"--*Что ж, "--- заговорила Заяц, пытаясь сгладить неловкую тишину.
"--- Проблема освещена достаточно хорошо.
Предлагаю голосование.

"--*Его результаты и так ясны, "--- пробормотал я.

"--*Возвращайся к нам, Небо.
Ты уже один раз принял решение в одиночку, пока все лежали в анабиозе, "--- заметил Мак.
"--- По-моему, это вскружило тебе голову.

"--*И оставило вас в живых, "--- заметил я и направился к лагерю.

"--*Опять <<вас>>, "--- буркнул Мак.

"--*Небо, "--- дрожащим голосом позвала меня Кошка.
Я не откликнулся.

"--*Друг хоргетов, "--- прошипел кто-то в тишине.

"--*Так слово <<друг>> "--- это теперь оскорбление? "--- горько спросил я.
"--- Мне нечего здесь делать, морально цивилизация уже мертва.
Известите, когда я должен буду своими руками закопать её последнюю надежду.

\section{[*] Схватки}

\spacing

"--*Мы не сможем проникнуть в Тхитрон незамеченными, "--- сказала Эрхэ.
"--- Стража у врат узнает Митхэ.
И меня тоже, и Акхсара.

"--*Так, на что ты намекаешь, я не понял, "--- буркнул Ситрис.
"--- Не пойду я через ворота, есть и более приятные способы самоубийства.
Я чересчур заметный!

"--*Я уже тебе сказала, что мы\ldotst

"--*Я вообще не понимаю, к чему такой риск.
Роды можно принять прямо на дороге, а ребёнка потом передать с людьми.

"--*Ты принимал роды?

"--*Слушай, мне уже давно не двадцать пять дождей!

"--*А что ты будешь делать, если она закровит или подхватит болотную лихорадку?
Митхэ уже не слишком молода, а болота от Тхитрона в тринадцати кхене!

Ситрис угрюмо промолчал.

"--*Ладно, "--- буркнул он наконец.
"--- В таком случае лучший способ избежать излишнего внимания "--- суета.
Я предлагаю разыграть схватки и ехать через ворота.
Пропустят быстро и не посмотрят.

"--*Отлично.
Вот ты её и повезёшь, "--- заключила Эрхэ.
"--- Митхэ, сможешь сыграть роженицу?

"--*Никаких игр, "--- отрезала Митхэ.
"--- Хватит их с меня.

"--*Но\ldotst

"--*Дождёмся схваток и поедем в родах.
Шейка уже созревает, я только утром проверила.

"--*Опасно ехать в родах верхом!

"--*Это куда менее опасно, чем то, что нас могут узнать.

"--*Хорошо.
Ситрис, держи своего оленя осёдланным на всякий случай.

"--*Кажется, я уже сказал, что не собираюсь везти верхом роженицу через городскую стражу.

Наступило молчание.

"--*Снежок, "--- начала Митхэ, "--- он прав, в суете нас сложнее\ldotst

"--*Подожди, "--- взмахом руки остановил её Акхсар.
Его лицо окостенело от гнева.
"--- Это вопрос принципа.
Ситрис, ты понимаешь, что подвергаешь сейчас товарищей опасности?
Меня, Эрхэ, Митхэ?

"--*Я понимаю, что подвергаю совершенно ненужной смертельной опасности \emph{себя}.
И это всё, что мне нужно понимать.
Одно дело "--- встретиться с храмовниками на дороге, и другое "--- в городе.

"--*Ты сдохнешь трусом, с клинком в заднице.

Наступила ещё более глубокая тишина.
Ситрис повернул к нему голову.

"--*И?

Акхсар сжимал и расжимал кулаки.

"--*Ну сдохну, ну трусом, ну клинок где-то там, "--- Ситрис встал и подошёл к кипящему от ярости воину.
"--- Что ты этим хочешь сказать?
Что твоя овеянная подвигами жизнь и славная гибель чем-то лучше?

Эрхэ попыталась встать между ними, но мужчины мягко оттолкнули её.

"--*Подожди, Эрхэ, "--- сказал Ситрис.
"--- Это, как выразился наш герой, <<вопрос принципа>>.
Какого принципа только, он и сам не понимает.
Ему с детства вдолбили в голову, что он должен следовать принципам, чтить клятву.
Заставили эти принципы заучить и жить по ним.
И вот перед нами лоб восьмидесяти дождей "--- тело, принадлежащее отряду, мысли, принадлежащие женщине, заплесневелые умения, принципы и чувства, рождённые в давно сгнивших черепах.
Ничего своего.
Ты не мужчина, ты склад чужого имущества.

"--*Ты всё сказал? "--- задыхаясь от гнева, пропыхтел Акхсар.
"--- Всё, что мог, дырявый горшок?
В тебя налили кровь и выставили на улицу, вот ты и бродишь, запихивая в себя помои, конопляную жижу, мужское семя и женскую влагу, пытаясь обрести своё простое глиняное счастье "--- счастье помойного горшка быть полным!

"--*Ну, я тебя выслушал, предположим, "--- поморщился разбойник.
"--- Что дальше-то?
Как ты собрался решать <<вопрос принципа>>?
Я не буду с тобой драться.
Я высказал своё мнение и не обязан ничего доказывать, тем более тебе.

"--*Потому что ты брехло и трус.

"--*Благодарю, я уже это уяснил.
Что-то ещё?

"--*Мы взяли тебя, потому что\ldotst

"--*<<Мы>>? "--- перебил его Ситрис.
"--- <<Взяли>>?

"--*Не смей. Меня. Перебивать! "--- рявкнул Акхсар.
Носы мужчин разделяла уже какая-то пядь.

"--*Акхсар, перестань, "--- сказала Эрхэ, снова попытавшись влезть между ними.
Руки всех троих заплясали в непередаваемом танце, на который способны только воины в шаге от применения силы.
"--- Ситрис, не обижайся, мы все нервничаем, это понятно\ldotst

"--*Да какие обиды? "--- фыркнул разбойник.
"--- Я не держу обид на побеждённых.

Акхсар рыкнул и, схватив Ситриса за пучок волос, дёрнул изо всех сил.
Ситрис отшвырнул в сторону Эрхэ, все трое покатились по земле, но тут же оказались на ногах.
Фаланги мужчин вылетели из ножен и нацелились в открытые шеи;
сабля Эрхэ запоздало брякнулась сверху.

Ситрис оскорблённо-яростно дышал;
трубчатые заколки остались в руках Акхсара, и казавшиеся прямыми волосы свернулись в тугие чёрные кудри, давно не знавшие мыла и горячей банной воды.
Две пары глаз "--- холодная северная трава и огнистый южный уголь "--- скрестились, словно копья.

"--*Перестаньте, "--- простонала Митхэ, закрыв лицо руками.

"--*Хорошо, "--- неожиданно легко согласился Ситрис и убрал фалангу.
"--- Знаете, я всё понял.
Вы все думаете точно так же.
Звон и сияние!
Хорошо.
Я с самого начала был лишним на этом празднике великих героев, и ничего, собственно, не изменилось.
Благодарю за приятную прогулку, благослови вас Сат-скиталец.

Разбойник подхватил дорожный мешок, вскочил на оленя и скрылся в лесной полумгле, только дробно простучали копытца.
Акхсар сплюнул.

"--*Ты пень пустоголовый, "--- бросила растрёпанная Эрхэ, вытряхивая из складок рубахи сучки и листья.
"--- Ну зачем, зачем тебе понадобилось брать его на принцип?

"--*На войне следует исполнять приказы!

"--*Любимый, мы не на войне.
И клятвой мы больше не связаны.
Отряда чести больше нет, ты понимаешь?! "--- Эрхэ швырнула саблю на землю и перешла на крик, больше похожий на плач.
"--- Мы просто хотим помочь другу!

Митхэ вдруг тоненько вскрикнула и спиной сползла с мшистого бревна.
Эрхэ в один миг оказалась возле командира.
Этот завывающий крик нельзя было спутать ни с чем другим.

"--*Акхсар, схватки!
Садись на Серебряного\ldotst
До Тхитрона около десяти кхене, надеюсь, что мост ещё на месте\ldotst
на Серебряного, дурень, он быстрее, я тебе её подам\ldotst

\section{[-] Крепкая спина}

\textbf{<речь Ликхмаса, после которой часть народа решила уйти на север и договориться с Безумным>}

"--*Ликхмас-тари, "--- сказал мужчина, "--- я не знаю, было ли правдой то, что написано в <<Легенде об обретении>>.
Я не знаю, предвидел ли Карлик будущее.
Но я знаю одно "--- пока я и моя семья живы, никто не ударит тебя в спину.
Мы будем твоей спиной.

Затем он обернулся к стоящим вокруг людям:

"--*Я верю этому человеку, как верю и в то, что скоро <<Легенда>> обретёт счастливый конец.
Если же нет "--- то больше она никогда не будет звучать на земле, ибо некому будет её рассказать.

\section{[-] Эволюционный процесс}

\spacing

"--*Выживут одни трусы и наплодят трусов, "--- процедила сквозь зубы Анкарьяль.

"--*Ты плохо понимаешь эволюционный процесс, "--- возразил я.
"--- Это один народ, одна популяция, один котёл генов.
Да, храбрецы и альтруисты гибнут чаще, но они делают жизнь своих соплеменников лучше.
Соплеменники, которые несут другие сочетания тех же самых генов, получат шанс на размножение и породят новых героев.
Именно поэтому, что бы ни говорили старики, никогда не исчезнут бескорыстные помощники, никогда не переведутся храбрецы.
В этих людях не просто их собственная сила.
В них сила их рода, их вида.

\section{[*] Мстительная тень}

"--*Ситрис ар'Эр э'Тхинат!

Разбойник обернулся, как ужаленный.
В пяти шагах стояла крепкая молодая воительница с растрёпанными волосами.
Ситрис отскочил ещё на два шага, но под полубезумным взглядом даже это расстояние не показалось достаточным.

"--*А я ведь поклялась, что найду тебя.

"--*Здравствуй, Кхохо, "--- справившись с собой, проговорил Ситрис.
"--- Давно не виделись.
Ты теперь в Храме Тхитрона?

"--*Не оттягивай мой язык в сторону! "--- пропела воительница.
"--- Помнишь платок?
Помнишь?

У разбойника задрожали колени.
Конечно, он помнил.
Неумело вышитый шёлковый платочек возвращался во снах снова и снова последние три дождя.
Однако\ldotst

"--*Какой платок? "--- осведомился он.
"--- О чём ты?

"--*Помнишь это? "--- Кхохо отбросила спутанные волосы с лица, натянув пальцами шрам-улыбку.
Безумный зелёный взгляд вытягивал силы, жёг нестерпимым пламенем\ldotst

Мир завертелся вокруг Ситриса.
Из вечерних сумерек вдруг полезли дикие образы, отголоски беспризорного детства и разгульной юности.
Они тянули к Ситрису холодные липкие руки.
Безумные зелёные глаза светились, словно кошачьи, мертвенно-зелёный свет пробивался сквозь шрам-улыбку\ldotst

Колени уже не дрожали, а ходили ходуном в священном ужасе.
Что происходит?
Этого же не может быть\ldotst

"--*Помнишь, Ситрис?
Помнишь ту милую женщину, которой дочь подарила платок?
Помнишь Хонхо ар'Мар?
Я грела её в своих объятьях, но её руки дрожали, словно листья мимозы.
Помнишь этот шрам? "--- Кхохо оскалилась, натянув шрам ещё сильнее.
"--- Помнишь?
Я сделала себе точно такой же, Ситрис.
Помнишь его?

Ситрис вытер со лба холодный пот и обругал себя последними словами.
Образы пропали так же быстро, как и появились.

"--*Хороший трюк, Кхохо.
А я на миг решил, что за мной действительно пришла мстительная тень.

"--*Она пришла, "--- нежно прошептала Кхохо.
"--- Я поклялась убить тебя, гниль, и теперь тебе не скрыться во мраке джунглей.

Уже на слове <<скрыться>> воительница летела в атаку.
Ситрис едва успел схватиться за эфес фаланги.

\section{[*] Ничья}

"--*А неплохо, "--- тяжело дыша, признал Ситрис.
Влажное дерево приятно холодило разгорячённую спину под рубахой.
Кхохо кивнула и, вытащив фляжку, впилась в горлышко.

"--*Дай воды глотнуть, рыбина лохматая.

"--*Я тебе её в глотку засуну, "--- посулилась воительница.
"--- Вместе с фляжкой.

Однако фляжку всё-таки протянула.
Ситрис сделал пару глотков и едва успел перехватить нож противницы.

"--*Ладно, ладно, "--- хмуро пропыхтела она, высвобождая руку.
"--- Перестань.

"--*Сама перестань!
После драки кулаками не машут.

"--*Это не кулак! "--- парировала Кхохо.

Воительница и разбойник посидели ещё немного.

"--*Ты как здесь оказался-то? "--- проворчала Кхохо.
"--- Я слышала, ты с Митхэ ар'Кахр ушёл на запад и воевал в отряде чести.

"--*Было дело, "--- кивнул Ситрис.

"--*Она здесь?

"--*С чего бы ей быть здесь?
На Кристалл ушла, конечно.

"--*А ты?
И от неё сбежал, трус?

"--*Я не самоубийца, чтобы\ldotst

Кхохо вдруг захихикала.

"--*Что смешного?

"--*Да видели мы её.
На белом олене, с Акхсаром ар'Лотр, через южные ворота.
Три улицы дружно притворились, что ничего не происходит.
\mulang{$0$}
{Здесь её в обиду не дадут, не бойся.}
{Don't worry, there's no human to let her be harmed.}
\mulang{$0$}
{Север помнит.}
{The North remembers.''}

Ситрис промолчал.
Кхохо оценивающе посмотрела на него.

"--*Я бы ожидала, что ты сбежишь ещё в Тхартхаахитре.
Или сдашь Митхэ с потрохами.

"--*Я, может, и трус, но не предатель, "--- буркнул разбойник.
"--- А ты почему уехала из Кахрахана?

Кхохо смутилась.

"--*Проблемы.

"--*Ясно.
Я так понял, что в Тхитроне на кутрапов смотрят сквозь пальцы.

"--*Я не кутрап! "--- взорвалась Кхохо и вскочила на ноги.

"--*Какие ещё проблемы могли заставить тебя уехать на Ближний Север?

"--*Морду разбила вождю!

"--*За что?

"--*Он порезал струны на моей цитре!

"--*Ты ещё и играешь, рыбина? "--- хмыкнул Ситрис.
"--- Если музыка соответствует твоему характеру, то я солидарен с вождём.

Кхохо швырнула нож, и Ситрис едва успел его поймать.

"--*Ты действительно до сих пор на меня сердита?

"--*Нет, конечно, "--- буркнула Кхохо и снова плюхнулась рядом с Ситрисом.
"--- Если честно, мне до тебя дела нет.
Просто в памяти всплыл хасетрасем и я немного занервничала.
Старые воспоминания\ldotst

"--*Можешь представить меня здешнему Храму? "--- спросил Ситрис.

"--*С чего бы? "--- фыркнула Кхохо.
"--- Иди к купцу, её зовут Кхотлам ар'Люм, и наплети ей что-нибудь.
Эта дура пристраивает даже ублюдков вроде тебя.

Кхохо поднялась, подхватила саблю и, бросив почти пустую фляжку Ситрису, пошла в город.

"--*Нож не потеряй, я за него полгода пахала.

Ситрис посмотрел на нож.
Маленький изогнутый клинок с кольцом на рукояти, украшенный гравировкой в виде широкого улыбающегося рта.
Гриф-клинок, как называют такие ножи тенку.

\mulang{$0$}
{"--*Ты про этот рыболовный крючок? "--- засмеялся разбойник.}
{``This kind of fishhook?'' the outlaw laughed.}
\mulang{$0$}
{"--- Цена завышена.}
{``It's overpriced.}
\mulang{$0$}
{За полгода я храпом заработаю больше, чем он стоит.}
{If I snored for gold for half a year, I would earn more than it costs.''}

Кхохо обернулась.
Её глаза сузились, но злости в них не было.

\mulang{$0$}
{"--*Я буду причиной твоей смерти, "--- спокойно констатировала воительница.}
{``I'll be a cause of your death,'' the warrior calmly told.}
\mulang{$0$}
{"--- Это единственная неизменная истина в изменяющемся мире.}
{``It's the only changeless thing in the permanently changing world.''}

Ситрис улыбнулся.
Он чувствовал своим обострённым нюхом, что Кхохо говорит правду, но почему-то её слова не испугали, а успокоили.
Смерть больше не таилась в закоулках города и сумраке леса, она приняла человеческое обличье и обрела голос.

\mulang{$0$}
{"--*Надеюсь, ты не воспользуешься этим, "--- разбойник повертел нож Кхохо в руках.}
{``You won't use that, I certainly hope so,'' the outlaw fiddled with Kch\`oh\^o's knife.}
\mulang{$0$}
{"--- Такого позора я не переживу.}
{``I can't take such a dishonor.''}

Кхохо поспешила отвернуться и быстро уйти.
Ей очень хотелось захохотать, но казалось неуместным делать это при заклятом враге.

Ситрис уже собирался встать, когда на поляну вышел Цапка.
В зубах кота красовалась жирная капюшонная крыса.

"--*А ты-то что удрал, дружище усатый? "--- вздохнул Ситрис.
"--- Думаешь, в Тхитроне мыши толще?

Кот с некоторой иронией взглянул на разбойника, и тот почувствовал себя круглым дураком.
Да, действительно, при чём тут вообще толщина мышей?

\section{[*] Мать}

\epigraph
{Я "--- словно дом, двумя окнами в сад,\\
Сквозь глаза листья падают в пол.\\
Входят в меня и уходят друзья,\\
Остывают следы их шагов.\\
Я жду тех, кто не приходит,\\
Открыта настежь дверь.\\
Ветер сквозь комнаты гонит\\
Звон пустоты, сметая всё на пути\ldotst}
{Эрхэ Колокольчик}

\spacing

"--*Я могу стоять!
Я могу идти! "--- кричала Митхэ.
"--- Я прошу вас только помочь мне одеться!

"--*Ты слаба, "--- сквозь слёзы сказала Кхотлам.
"--- Золото, одумайся.
Ты не сможешь сражаться.

"--*Смогу, "--- прорычала Митхэ, безуспешно пытаясь затянуть ремни нагрудника.
Придя в ярость, она дёрнула ремешок и оторвала его.
"--- Безумный-кровопийца!
Помоги мне, Акхсар!

"--*Золото, тебе нужно отдохнуть, "--- Акхсар схватил воительницу под руки.
"--- Золото, послушай меня.
Мы найдём Хата, я обещаю тебе.
Только останься.

Митхэ остановилась и уткнулась лбом в холодный малахит стены.

"--*Я не могу.
Не могу, ребята.

Наступило молчание.
Задумчиво потрескивали головёшки в очаге, за окном шумел дождь.
Воительница высвободилась из объятий Акхсара, медленно, сбивающейся походкой подошла к тростниковой колыбельке, в которой посапывал ребёнок.
Митхэ устало улыбнулась и провела пальцем по маленькой, сжатой в кулачок ручке.
Потом оглянулась на замерших у стены друзей.

"--*Вы думаете, что я сошла с ума, да?

Акхсар и Кхотлам переглянулись, не зная, что ответить.
Наконец Кхотлам неуверенно подошла к воительнице и дрожащими руками принялась завязывать ремешки.

"--*Я буду молиться лесным духам, чтобы ты вернулась к ребёнку.

"--*Не знаю, вернусь ли я, Пёрышко, "--- проговорила Митхэ.
"--- Но если я не попытаюсь найти Атриса, то буду жалеть об этом всю жизнь.
Если он мёртв "--- лучше будет и мне умереть.
Нет ничего хуже, когда ребёнка воспитывает несчастная женщина. Пообещай мне, что будешь для Лисёнка счастливой кормилицей\ldotst
если я не вернусь.

"--*Обещаю.

"--*Снежок, "--- голос Митхэ приобрёл мягкий оттенок, "--- ты мой друг, и твоя верность уже давно перешагнула границы обычной дружбы.
Поэтому, если хочешь, уходи, но будь другом ещё раз "--- переседлай Серебряного.
У меня самой не хватит сил.

Акхсар опустил глаза, кивнул и вышел за дверь.
Кхотлам проводила его долгим печальным взглядом и снова занялась доспехами.

"--*Ремешки не сходятся, "--- виновато прошептала Кхотлам.

"--*Тяни, Пёрышко, тяни изо всех сил, не бойся.
Это всего лишь тело.

Ремни затрещали, и грубая кожа сдавила Митхэ живот и грудь, впилась в нежную, размякшую плоть.
Воительница не шевельнула ни одним мускулом, но у неё на лбу выступили крупные капли пота.

Митхэ, шатаясь и хрипло дыша, подошла к стойке и привычным движением подпоясалась, перебросив саблю за бедро.
Бросила полный боли прощальный взгляд на тростниковую колыбельку и открыла дверь, запустив в зал рокочущий шум ливня.
Ветер ворвался в дом, разбросал по полу сухие листья и освежил лицо воительницы.
Её взгляд стал более осмысленным.

"--*Скажи, я плохая? "--- грустно улыбнулась она напоследок подруге.

"--*Ты лучшая из тех, кого я знаю, "--- заверила её Кхотлам.
Митхэ обняла женщину и, приподнявшись на носках, поцеловала её в бледные, мокрые от слёз губы.

"--*Храни тебя лесные духи, Пёрышко.

Кхотлам кивнула.
Воительница нырнула в дождь, как в толпу сражающихся.
Купец долго смотрела ей вслед и, плотно закрыв дверь, подошла к колыбельке.

"--*Ой, а кто у нас тут такой маленький Лисёнок?
Кто у нас такой золотой\ldotsq "--- ласково зашептала она, поправляя одеяльце.

Тоскливо тянулись михнет.
Вдруг слабый сквозняк заставил женщину повернуть голову к двери.
Там стоял мужчина.
Кхотлам могла бы поклясться, что он вошёл через скрипучую закрытую дверь без единого звука.
С его чёрных сальных кудрей капал дождь;
обсидиановые глаза не отрываясь глядели на колыбель.

"--*Это ты Кхотлам ар'Люм? "--- хрипло спросил мужчина.

"--*Ты ещё успеешь её догнать, "--- тихо ответила ему Кхотлам.

"--*Я знаю.

Кхотлам улыбнулась, и Ситрис понял, что пришёл куда надо.
В улыбке купца не было ни жалости, ни презрения.
На миг разбойник подумал, что пришёл в дом давно забытой любовницы, которая сегодняшним утром вдруг вспомнила, что любит и ждёт, "--- в Кхотлам была нежность, не омрачённая мыслями, не иссушённая временем.

Женщина подошла к гостю и погладила его по мокрой впалой щеке:

"--*Сними одежду и садись у огня.
Рубаху тоже снимай, не стесняйся.
Хватит с тебя испытаний.

Головешки трещали в очаге.
Где-то вдали раскатился гром.
Ребёнок в тростниковой колыбельке открыл глаза и испуганно заплакал.
Кхотлам нежно и тихо запела, мерно качая колыбельку, как когда-то волны Могильного пролива качали её корабль;
под тихое пение ребёнок и завернувшийся в старое одеяло Ситрис заснули "--- почти одновременно.

\chapter{[-] Осаждённые}

\section{[-] Встреча с мамой}

\spacing

"--*Кормилица! "--- закричал я.

"--*Лисёнок! "--- Кхотлам бросилась мне в объятия.

"--*Я думал, ты дома!
Ведь из Тхитрона приходят письма с пометкой <<Двор Люм>> и\ldotst

"--*Это не я, "--- засмеялась кормилица.
"--- Оставила твоих сестрёнок.

"--*Манэ и Лимнэ?
Так им же\ldotst

"--*Им уже под сорок дождей, дитя, "--- сказала Кхотлам.
"--- Они совсем взрослые.
В одном только не изменились "--- всё у них общее: носят одну одежду, спят на одной лежанке, даже мужчину нашли одного на двоих\ldotst
Давно ты не был дома.

"--*Зря ты оставила на них город в такие времена, "--- упрекнул я её.

"--*Они способные, у них всё получится, Тхитрон обязательно выстоит, "--- заверила меня кормилица.
"--- Я бы не сделала так, если бы сомневалась.
В крайнем случае, если всё будет совсем плохо, люди Тхитрона уйдут на север, к Ледяной Рыбе.
Идолы за нами не пойдут, они недолюбливают холод, а с хака и местными людьми мы как-нибудь разберёмся.
Еды хватит на несколько дождей, я велела наделать сушенины\footnote
{Смесь из высушенных овощей и вяленого мяса, пересыпанная солью с углём. \authornote},
в этом году поля просто взбесились\ldotst
Да что мы тут в проходе стоим, людям мешаем, пойдём внутрь\ldotst

\section{[-] Беременность}

\spacing

Я посмотрел на спящую Чханэ.
Она дышала ровно, с лёгким присвистом.
Интересно, в чём причина её бесплодия?

Я закрыл глаза, и мой демон приступил к подробному анализу её организма.

Да, причина состояла в эндокринном сбое.
Давняя травма мозга сдвинула гормональный баланс, и мужчина превратился в женщину.
Но не до конца "--- механизм, запускающий фертильность женского организма, остался спать.
Длительные депрессии, стрессы, употребление алкоголя только усугубили её состояние.

Однако детородные органы Чханэ были сформированы правильно, никаких пороков развития их я не нашёл.
А значит, проблема решаема.

Я начал тонкое преобразование клеточных структур.
Люди-тси сильно отличаются от большинства других видов.
Однако принципы у них общие, а я всегда любил экспериментировать.

\dots Здесь изменить концентрацию одного медиатора, там другого\ldotst

\dots Добавить киназ, увеличить экспрессию гена рецептора\ldotst

\dots Тихо, фермент, мой хороший, куда так разогнался?
Вот тебе ингибирующий лиганд, остынь\ldotst

Постепенно, михнет за михнет, цепная реакция охватывала все глубинные структуры гипоталамуса.
В тонком аромате женского пота появились другие, игривые нотки.
Чханэ задышала медленнее, на лице повилась улыбка "--- изменились даже её сны.
Вскоре она чуть-чуть приоткрыла глаза:

"--*Лис, ты уже проснулся?
Который час?

Я вместо ответа нежно поцеловал подругу.
Она, тёплая и мягкая, сонно откликнулась на мои ласки.

Через три дня на шее Чханэ появились стигмы беременности.

\section{[-] Ловля котов}

\spacing

Кормильцы разнесли всем плошки с едой.

"--*Хай, девочка моя, ты котов ловить собралась? "--- заулыбалась Кхотлам и ласково погладила Чханэ по полосатой шее.
Затем незаметно для всех отвесила мне самую тяжёлую затрещину в жизни, добавив одними губами: <<Пень пустоголовый>>.

Чханэ засмущалась и начала есть.
Прикончив миску, она засмущалась ещё больше и попросила добавки.
Следующую миску Кхотлам принесла ей уже сама, не дожидаясь просьбы.
Чханэ, урча, перемалывала зубами куски мяса.
Подруга напоминала голодного оцелота, как никогда в жизни.

\spacing

\section{[-] Тайна старика}

\epigraph
{Во времена, когда мы уже видим предел технологического и биологического прогресса, правильная настройка важнее аппаратных характеристик.
Это применимо как к компьютерным системам, так и к сапиентным существам.}
{Длинный-Мокрый-Хвост, мыслитель, химик-технолог Тси-Ди}

\spacing

"--*Почему именно я?

"--*Критериев много.
Насчёт тебя я знаю только два.
Ты каким-то образом обратил на себя внимание Храма в детском возрасте.
Кроме того, ты "--- хранитель Митхэ ар'Кахр.
Её потомство, наряду с потомством действующего Короля-жреца и ещё нескольких значимых личностей, имеет приоритет.

"--*То есть вы выбираете не только по личным качествам, но и по качествам дарителей?

"--*Способности могут наследоваться, это очевидно.
Но само по себе происхождение не является ключевым фактором, оно всегда идёт в дополнение к личным качествам.

"--*Я так понимаю, что после отбора претендент воспитывается в особых условиях.

"--*Условие только одно "--- будущий Король-жрец не должен ни в чём нуждаться.
Его должны воспитывать лучшие, любящие кормильцы.
У него должен быть доступ ко всем знаниям сели, его должны тренировать лучшие воины по его надобностям.
Он должен быть обучен искусству обольщения и не знать недостатка в женщинах и мужчинах, если он испытывает в них потребность.
Ясное дело, голодать и испытывать унижение он тоже не должен.

"--*И поэтому\ldotst

"--*И поэтому претендентов много.
Сложно уследить за всем, особенно в лихие времена.
Тебя мы отбраковали еще пять дождей назад.

"--*Благодарю за честность, "--- ухмыльнулся я.
Старик не ответил на улыбку.

"--*Да, тебя отбраковали именно поэтому.
Ты не наигрался в детстве и превращаешь в игру даже серьезное дело.

"--*Так кто же\ldotsq

"--*Это тайна, "--- отрезал Хитрам-лехэ.
"--- Может быть, когда-нибудь тебя в неё посвятят.
Надеюсь, я этого не увижу.

"--*Мне очень нужно знать, кто это!
Я должен передать ему\ldotst

"--*Я не могу сказать, "--- раздражённо ответил старик.
"--- Всё, Король-жрец, разговор окончен.
Помни своё место.

Хитрам-лехэ ушёл в свою комнату, хлопнув дверью.
Чтоб этих сели с их традициями!

Разумеется, старик ни в чём не виноват.
Здесь чувствовалась рука куда древнее, чем его.
Тси, именно они.
Они были очень открытыми, но настоящие тайны их учили хранить с детства.
<<Доверить тайну тси "--- всё равно, что не доверять её никому>>, "--- сказал как-то давно мёртвый знакомый.
В его тоне слышалось многое "--- презрение, бессильная злость\ldotst и тоска по чему-то очень важному, чего мы, демоны, были лишены.

\section{[-] Жрец в тени}

Я долго раздумывал, стоит ли ещё раз попробовать убедить старика.
Новому Королю-жрецу следовало передать жизненно важные инструкции.
Но вдруг судьба преподнесла мне сюрприз.

В другом конце коридора показалась высокая фигура.
Трукхвал ар'Со.
Я приметил его, как только вошёл в храм.
Чёрные как смоль волосы и огромные глаза им в тон.
Величественная осанка, походка кобры и спокойный взгляд человека, видевшего в этой жизни многое.
Да, возможно, это его готовили.
Скорее всего, именно он был главным претендентом на роль Короля-жреца, пока события не пошли так, как пошли.

"--*Трукхвал, "--- окликнул я его.

"--*Король-жрец, чем могу служить? "--- коротко поклонился мужчина.
Чёрные как ночь глаза смотрели ожидающе, без малейшего намёка на вызов или подобострастие.
Мои сомнения рассеялись.

<<Да, это тот, кто мне нужен>>.

"--*Ты уйдёшь на север с беженцами, "--- без обиняков сказал я.
Мужчина нахмурился.

"--*Я хотел бы отправиться на юг, Ликхмас, "--- мягко промолвил он.
"--- Я плохо владею оружием, но вам будут нужны лекари, счетоводы и люди для отправления обрядов.
С беженцами уйдёт достаточно белых плащей.

"--*Это приказ, "--- сказал я не терпящим возражений тоном.
"--- Если нас разобьют, ты возьмёшь командование на себя и поможешь ушедшим сели закрепиться на севере.

"--*Командующий силами беженцев уже назначен, "--- заметил Трукхвал.
"--- Решать не тебе и не мне.

"--*Ты прекрасно знаешь, что значит <<командование>>.

В глазах Трукхвала мелькнуло понимание.
Он кивнул.

"--*Король-жрец, если вас разобьют\ldotst "--- блестящие брови едва заметно шевельнулись, и я вдруг осознал, что этот человек сумел вложить в мимику ещё не высказанный вопрос.

"--*Покорись, Трукхвал, "--- опередил я.
"--- Дай Безумным гарантии лояльности народа сели.
Свобода и честь "--- ничто, если о них некому будет вспомнить.

На тонких губах Трукхвала медленно проступила улыбка.
Он тоже пытался читать непонятные книги.

"--*Безумный "--- не Безымянный, Ликхмас, "--- мягко сказал мужчина.
"--- Но я попробую.

"--*Тебя назовут трусом, "--- предупредил я.

"--*Моё имя предпочтут забыть, "--- кивнул жрец.
"--- <<В жизни и смерти>>.

"--*<<Жизнь и процветание>>, "--- невпопад брякнул я.

Лицо Трукхвала просветлело, словно больше всего на свете он ожидал услышать именно это.
Жрец поклонился, и спустя несколько мгновений его величественно развевающаяся роба исчезла за углом.

Обернувшись, я увидел, что Хитрам-лехэ вышел из своей комнаты.
Старик тяжёлым взглядом смотрел на меня.

"--*Ты ему рассказал?

"--*Это твоя тайна, "--- ответил я.

"--*Тогда что ты ему сказал?

"--*А это моя тайна, "--- улыбнулся я и направился в зал.

Хитрам преградил мне путь, его костлявая рука больно впилась мне в плечо.

"--*Место Короля-жреца "--- не твоё место, Ликхмас ар’Люм, "--- без обиняков сказал старый жрец.
Он шумно дышал мне в лицо, и это дыхание не предвещало ничего хорошего.
"--- Я знаю, как обстоят дела, и потому проголосовал за тебя.
Но истина остаётся неизменной "--- это не твоё место.
Помни, пока живёшь.
Хотя всё решится само "--- Безумный от тебя даже косточек не оставит.

Я кивнул.
Хитрам помолчал и вдруг смущённо закашлял.
Костлявая рука исчезла в рукаве робы.

"--*Я надеюсь, что ты приведёшь сели домой.
Тогда старик сможет попросить у тебя прощения за сказанное.

\section{[@] Прощание со стригами}

\spacing

Вскоре мы остановились.

"--*Вот тот самый воздушный поток, "--- Облачко указала в небо.
"--- Очень удачно, что мы заметили его.
Мы сможем лететь очень долго и обязательно найдём подходящее для жизни место.

Всю дорогу товарищи молчали.
Но сейчас, в последний миг перед прощанием, они не выдержали.

"--*Вы нужны нам! "--- страстно заговорил Мак.
"--- Исследования, постройка системы, да мало ли\ldotse

"--*Вас чересчур мало! "--- со слезами на глазах перебила его Заяц.
"--- А у нас пока есть медицинское оборудование и\ldotst

"--*Медицинских данных о стригах у вас почти нет, "--- спокойно сказала Облачко.
"--- Ваши врачи могут принести больше вреда, чем пользы, Костёр об этом сказал не один раз.
И пожалуйста, "--- Облачко вперила немигающий взгляд в Мака, "--- не считайте это предательством.
Просто примите как факт, что миграция необходима для выживания нашего вида.

Мак вздохнул и обхватил бритую голову руками.
Заяц плакала.

"--*Это пожелание всех глазастиков? "--- спросил я.

Совы кивнули почти одновременно.

"--*Может быть, вам нужно что-то ещё?

"--*Мы не унесём много вещей, "--- сказал Редуктор-Оси-Планеты.
"--- Всё необходимое с нами.

"--*Жаль, что мы не успели узнать вас получше, "--- сказал я.
"--- Вы достойные сапиенты.
Удачи вам.

Стриги как по команде подтянулись и уставились круглыми глазами в небо, где медленно тёк тёплый воздушный поток.

"--*Пусть ваши крылья не знают переломов, "--- сказала Облачко, окинув взглядом провожающих.

Крылья "--- это то, что отличало этих больших сильных птиц от меня, маленькой наземной пчелы.
Я знал, что мы с глазастиками уже никогда не будем единым народом "--- наши пути разойдутся насовсем.
Но последнее пожелание Облачка засело в моей памяти.
Это было пожелание равным.
Это было пожелание тси.

Я обнял молодую женщину, уткнувшись лицом в пушистое перо.
Она в ответ ласково цапнула клювом мой усик.

\section{[-] Последний корабль}

Последний корабль на север отошёл утром.
На причале стояло тяжёлое молчание.
Люди обнимали друг друга теплее, чем обычно.

У одного из вёсел сидел Трукхвал.
Жрец сменил свою робу на одежду моряка и собрал чёрные волосы лентой.

Лусафейру всегда говорил: <<При составлении самого простого плана нужно продумывать сразу три пути отступления.
А если об этом правиле знают враги, то пять>>.
Я не стал давать задание Трукхвала никому другому.
Да, он вполне мог погибнуть, статистика неумолима.
Но опыт подсказывал мне, что на мотивированных статистика работает плохо, пока они не сделают своё дело.

Вырвавшись из размышлений, я заметил, что Трукхвал всё это время смотрел на меня.
Я помахал ему.
Жрец едва заметно улыбнулся и кивнул.

"--*Глубина! "--- рявкнул капитан.
Бодро застучал барабан, и моряки, ухая, навалились на вёсла.

\section{[-] Письмо Люм}

\spacing

Гонец подбежал ко мне и почти рухнул в поклоне.
Воины подхватили его под руки и усадили за стол.
Я набрал воды в чашу, и человек впился в неё.

"--*Почему ты так бежал? "--- поинтересовалась воительница слева.
"--- Так даже оленя загнать можно.

\mulang{$0$}
{"--*Сообщение от Двора Люм, "--- пробормотал человек, едва отлипнув от чаши.}
{``Message, the House of Lo\~em,'' the man muttered as soon as drained his goblet.}
\mulang{$0$}
{"--- <<Беженцы дали бой объединённым силам врага и отбились малой кровью, но удерживать полуразрушенный Тхитрон более нельзя.}
{\emph{``Refugeers fought back united enemy forces, losses are low, but ruined Tch\"\i tr\`on could't be held.}}
\mulang{$0$}
{Город отдан сельве.}
{\emph{The city's been left to the Silva.}}
\mulang{$0$}
{Отныне каждый сам за себя.}
{\emph{It's everyone for himself now.}}
\mulang{$0$}
{Больше не пишите>>.}
{\emph{No more letters.''}}

\section{[*] Отказ}

\spacing

"--*Тебе нечего предложить взамен.

Митхэ аккуратно выдвинула из ножен саблю.
Легенда Серого Рассвета весело сверкнула всем великолепием гравировки.

Си-Абву ухмыльнулся.

"--*И чем ты будешь защищаться, если мои воины захотят узнать, что у тебя между ног?
Листом юкки?

Митхэ, не моргая, спокойно смотрела на вождя Си-Абву.
Молниеносный удар "--- и он будет лежать у её ног, захлёбываясь собственной слюной, а Акхсар и Эрхэ так же быстро расправятся с пятью его воинами.
Но что это будет значить?
Новая бесконечная война между сели и хака, новые жертвы?

Си-Абву, казалось, видел её мысли насквозь.

"--*Разговор окончен, женщина.
Пленник мой, и я буду делать с ним всё, что пожелаю.
А я желаю, чтобы его сегодня на закате принесли в жертву Великим.

"--*Сегодня Такани-Жой, чужеземцы, "--- добавил старейшина.
"--- После жертвы будет большое празднество.
Оставайтесь и почтите богов вместе с нами.

Митхэ, не сказав ни слова, поднялась и махнула друзьям.
Вместе они покинули душный шатёр вождя.

\section{[*] Середина Дождя}

В тот день хака праздновали Такани-Жой, Середину Дождя.
Несмотря на грозу, в поселении царило оживление: люди в праздничных одеждах сновали под навесами, весело переговаривались, пели песни.
В жилищах играла музыка и залихватски ухали танцоры, отбивая голыми пятками по половым доскам.
Но там, где проходила мрачная Митхэ с друзьями, веселье слегка стихало.
Девушки пугливо прятались под покрывалами, когда колючий взгляд Акхсара скользил по ним;
мужчины презрительно смотрели на коротковолосых, по-военному одетых Митхэ и Эрхэ, и обменивались короткими сексуально-уничижительными замечаниями.

"--*Зачем они кутаются?
Такая духота, "--- буркнул Акхсар, кивнув на девушек.

"--*Они уверены, что одежда скрывает их тела, "--- пояснила Эрхэ.
"--- Помнишь тенку?
Пленников достаточно было завесить покрывалом, как попугайчиков, и они ничего не видели.

"--*Вон те мужчины только что сказали, что ты годишься лишь для секса.
Интересно, как они это поняли, если они даже не видят тебя сквозь одежду?

"--*Я не уверена, что они вообще что-то в этой жизни понимают, "--- призналась Эрхэ.
"--- Что будем делать, Золото?
Атриса нам не отдадут.

"--*Попробуем отбить, "--- ответил за подругу Акхсар.

"--*И что дальше?
Новая война?

"--*Проблемы в любом случае будут, "--- сказала Митхэ, впервые открыв рот.
"--- Но вам, думаю, куда проще будет объяснить Советам оглушённого шамана, чем убитого вождя, верно?

Эрхэ кивнула.

"--*Узнать бы дорогу к их святилищу.

"--*А чего её узнавать, "--- сказал Акхсар и взглядом указал на раскрашенные столбы, за которыми вилась безлюдная, уходившая в гору тропинка.
"--- Давайте только пойдём по обочине и потише, а то мы тут как кетцаль, поющий в ясный полдень.

\section{[*] Дети}

Вскоре воины оказались на небольшой площадке перед естественной пещерой, в которой хака устроили капище.
Раскрашенные столбы стояли здесь, словно высокие, коренастые стражники в праздничных одеждах.

<<А где стража?>> "--- поинтересовалась Эрхэ птицей-лирой.

<<Может, празднуют>>, "--- предположил Акхсар.

<<Ага, "--- саркастически ответила Митхэ.
"--- В селении находятся три воина сели, желающие отбить пленника, а вся стража празднует.
Я скорее поверю, что мы просто опередили перехватчиков на минуту-другую.
Ладно, пошли внутрь>>.

Митхэ бесшумно, приникнув к стене, вошла в преддверие пещеры.
Печально капал с потолочных сталактитов конденсат, тихо потрескивали угли почти погасшего костра, нервно поскрипывала решётка приоткрытой темницы.

Митхэ осторожно заглянула в темницу.
Никого.

<<Они уже начали жертвоприношение. Быстро за мной>>, "--- знаками показала Митхэ и бросилась в глубь пещеры.

"--*Не спеши, "--- вдруг раздался детский голос, и в темноте проступил силуэт мальчика хака дождей пятнадцати.
В руках мальчик держал игрушечную деревянную пику.

Митхэ замерла, разглядывая неожиданного противника.

"--*Ты не пройдёшь мимо меня, Митхэ ар’Кахр, "--- серьёзным тоном сказал мальчик на необычайно чистом цатроне.

Воины завороженно рассматривали мальчика.
Наконец Акхсар поудобнее перехватил нож и двинулся вперёд.

<<Стой>>, "--- зашипела Митхэ.

<<Золото, нам нужно спешить>>, "--- напомнил ей Акхсар.

<<Это же ребёнок>>, "--- укоризненно сказала Эрхэ.

"--*Почему ты не хочешь нас пропустить? "--- спросила Митхэ.

"--*Потому что этого человека желают боги, Митхэ ар’Кахр, "--- всё с той же забавной серьёзностью ответил мальчик.
"--- Ты хочешь отнять пленника у богов.
Я не могу этого допустить.

\mulang{$0$}
{"--*Я пройду к святилищу, хочешь ты этого или нет, "--- сказала Митхэ.}
{``I shall come into the sanctuary, whether you want it or not,'' M\={\i}tcho\^{e} said.}

"--*Тогда тебе придётся убить меня, "--- сказал мальчик и, поджав широкие губы, неуклюже вскинул пику.
Акхсар вдруг напрягся "--- от кончика шёл сильный запах лакового сока.

\mulang{$0$}
{<<Золото, копьё отравлено.}
{\emph{``Gold, the spear is poisoned.}}
\mulang{$0$}
{Будь осторожна>>.}
{\emph{Be careful.''}}

"--*Мне не нужно тебя убивать, "--- ласково сказала Митхэ, успокаивающе махнув Акхсару.
\mulang{$0$}
{"--- Я просто заберу у тебя пику, и никто не пострадает.}
{``I'll just take your pike, and nobody gets hurt.''}

\mulang{$0$}
{"--*Заберёшь, "--- согласился мальчик.}
{``You will,'' the boy agreed.}
\mulang{$0$}
{"--- Но я здесь не один.}
{``But I'm not alone here.''}

Из темноты показались ещё двое детей с таким же простым оружием.

Митхэ замерла, кусая губы. Атриса с минуты на минуту должны принести в жертву, ей оставалось лишь\ldotst но детей\ldotsq

Акхсар и Эрхэ наполовину вытащили из ножен клинки.

<<Стоять>>, "--- снова прошипела Митхэ.

"--*Послушай, малыш, "--- ласково сказала она, присев на корточки.
"--- У меня дома остался ребёнок.
Такой же, как ты. И тот пленник "--- его papa. У тебя есть papa?

"--*Есть, "--- раздался за её спиной звучный голос Си-Абву.
Акхсар звучно, витиевато выругался.

Митхэ обернулась.
Вождь стоял в парадном облачении, без оружия, и смотрел на воинов-сели.

"--*Молодец, сын.
Ты настоящий защитник святилища.
Однажды ты станешь прекрасным вождём.

Мальчик осклабился и крепче сжал своё нехитрое оружие.

"--*Как тебе моя стража, Митхэ ар’Кахр?
Я знал, кого поставить.

\mulang{$0$}
{"--*Я всё равно пройду, "--- бросила Митхэ.}
{``I shall pass anyway,'' M\={\i}tcho\^{e} said.}

\mulang{$0$}
{"--*Не пройдёшь, "--- ответил вождь.}
{``You shall not,'' the chieftain answered.}
"--- Ты отнимешь копья у троих детей, но не у тридцати.

Из коридоров выбежала целая армия ребятишек "--- неумело раскрашенных и босоногих.
Каждый сжимал отравленную пику, и конец каждой пики смотрел прямо на воинов-сели.

"--*Дети, "--- деловито сказал вождь, "--- сегодня вы будете защищать наши святыни.
Если эти чужестранцы хотя бы попытаются сделать шаг в сторону Ахам-Бвесы, убейте их.

"--*Будет сделано, "--- хором ответили дети.

Митхэ, не говоря ни слова, села на землю и закрыла лицо руками.
Акхсар и Эрхэ с ужасом смотрели на командира.

Акхсар бросил ледяной взгляд на вождя и медленно взялся за фалангу.

\mulang{$0$}
{"--*Ты заплатишь, "--- сказал он.}
{``You will pay,'' he told.}

"--*Ну-ну, Акхсар ар’Лотр, "--- примирительно поднял руки Си-Абву.
\mulang{$0$}
{"--- Тебе не победить.}
{``You can not win.}
\mulang{$0$}
{Перестань.}
{Stop it.''}

Эрхэ села рядом с Митхэ и обняла её.

"--*Золотце, "--- ласково сказала она.

"--*Я не смогла, Обжорка, "--- сквозь тихие рыдания прошептала Митхэ.
"--- Не детей.

"--*Я знаю, "--- заверила её Эрхэ.
"--- Ты была самым честным воином, защитой и голосом слабых.
Ты не хотела украшать своё тело, не стремилась к власти, твои желания всегда были просты и человечны.
Именно поэтому я когда-то бросила <<Стервятников>> и пошла за тобой.
Живущие по звериным обычаям всегда будут побеждать честных и принципиальных.
Но это не значит, что нужно прекращать бороться.

"--*Я не хочу больше бороться, "--- прошептала Митхэ.
"--- Я хочу быть с Атрисом.

"--*Тогда встань и сражайся до конца.

Митхэ зарыдала ещё сильнее и замотала головой.

"--*Хватит, Обжорка, "--- проворчал Акхсар.
"--- Мы проиграли, это ясно как солнечное утро.
Золото, я знаю, что тебе хочется\ldotst много чего.
Но сейчас, ради всех лет, что мы провели вместе, ради всех кхене, что мы прошли "--- встань и прими поражение достойно.
Старина Хат вряд ли хотел бы увидеть тебя такой.

Митхэ кивнула и, тяжело дыша, встала на ноги.
Дети настороженно приподняли копья, а вождь, слабо улыбаясь чему-то, прислонился к холодной каменной стене и скрестил руки на груди.

Вскоре со стороны святилища раздался слабый, приглушённый толщей камня стон.
Атрис словно знал, что его женщина стоит за стеной "--- прочие под ножами шаманов кричали во весь голос, вселяя в сердца слушающих ужас перед волей Безумного.
Атрис стонал, а Митхэ отстранённо вспоминала те далёкие счастливые минуты, которые она проводила со своим мужчиной.
Встречи, расставания, тихие беседы под одеялом, горячий травяной отвар, шум дождя\ldotst
Атрис видел в дожде нечто большее, чем просто воду с неба, и для Митхэ дождь тоже стал родным и близким.
Когда они встретились, шёл дождь.
Когда они гуляли и целовались, тоже почему-то всегда шёл дождь.
Где-то снаружи и сейчас гремела гроза и хлестал ливень, но капли его бились о твёрдый холодный камень, тщетно пытаясь охладить пылающую агонию Атриса и тяжёлую, бессильную печаль Митхэ.

<<Что, если мы родились, чтобы \emph{это} сделать?>>

Спустя двадцать михнет, которые показались Митхэ вечностью, стон затих навсегда.

\section{[@] Собачьи нежности}

Занимался вечер.
Сухие травяные стебельки шелестели под лёгким степным ветром, зрелые колоски и метёлки звенели своим богатством "--- твёрдыми семенами.
Я сидел в траве, особенно остро ощущая её сухость, прохладу западного горного ветра и дружеские чувства к Фонтанчику и Заяц.
Друзья сидели, обнявшись.
Заяц угрелась на груди Фонтанчика и дремала.

Я посмотрел на них.
Сколько я их знаю?
Фонтанчик был моим другом с ранней юности.
Мы вместе играли, учились и работали.
Позже к нам перевелась Заяц и как-то очень легко влилась в нашу компанию.

Я не знаю, когда Заяц и Фонтанчик стали любовниками.
Они скрывали это достаточно долго "--- увы, но межвидовые связи многими не приветствовались.
Что могло сблизить человека и кани настолько, что они стали спать вместе?
Трудно сказать, но иногдая видел удивительную вещь "--- они сидели рядом и переговаривались без слов.
Так было и сейчас "--- между друзьями шёл неслышный для меня диалог.

Около двадцати лет Заяц и Фонтанчик даже жили в одном доме "--- так называемое <<гнездование>>, пережиток времён, когда у тси ещё существовало культурное явление семьи.

"--*Знаешь, Небо, "--- вдруг заговорил Фонтанчик, "--- насчёт глазастиков\ldotst
Многие спорили на тот счёт, гуманны ли эти эксперименты по отношению к живым существам.
Но ведь когда-то и наши с тобой предки были другими.

"--*Да, "--- согласился я.
"--- Ты бы бегал по лесам в поисках мелких животных, а я опылял бы цветы.
И мы, если и встретились, то не смогли бы пообщаться.

Заяц открыла глаза и погладила Фонтанчику живот.

"--*Когда-то собаки были домашними питомцами, а теперь мы равны.
Я бы умерла, если бы тебя не было.

Мы промолчали.

В траве лежала и тяжело дышала умирающая пчела.
Её брюшко судорожно сокращалось, пытаясь прогнать драгоценный воздух через трахеи; крылья подрагивали в последней инстинктивной попытке взлететь.
Едва ли пчела осознавала, что неспособность взлететь равносильна гибели.
За неё уже всё решила странная игра случайности и естественного отбора.

А где-то неподалёку жила своей жизнью её родная уютная борть.
Пчела всё отмеренное ей время отдала колонии, вылетела в очередной рейс "--- и не вернулась.
Как и я.
Как и Заяц с Фонтанчиком.
Как и сотни миллиардов тси до нас "--- безымянные и безликие.

Я аккуратно тронул росший рядом колокольчик, вытряхнув на пальцы немного коричневатой пыльцы, и опылил соседнее растение.

"--*Ну вот, моё пчелиное предназначение можно считать выполненным, "--- сказал я.

Фонтанчик задумался.

"--*А мне, чтобы выполнить собачье, нужно поймать мелкое животное?
Хмм\ldotst "--- Фонтанчик со звериной ловкостью нырнул рукой в траву и вытащил за хвостик пищащую, перепуганную до смерти серую мышь.
Поднёс её к глазам и критически рассмотрел.
Мышь перестала биться и в страхе замерла, разглядывая выпученными глазами собачье лицо.

"--*Отпусти её, "--- укоризненно сказала Заяц.
"--- У неё сейчас сердечный приступ случится.

Фонтанчик опустил мышь в траву, и она едва слышно скользнула в заросли.

"--*Наверное, нужно было её съесть, "--- философски заметил Фонтанчик.
"--- У меня как-то плоховато с инстинктами.
Хотя ладно.
Мы уже стали ближе к природе.
Посмотрите вокруг.

"--*Мы и есть природа, "--- поправил я друга.
Фонтанчик кивнул.

"--*А моё предназначение? "--- спросила Заяц.

"--*А ты сиди, обезьянка, "--- Фонтанчик крепче обнял женщину.
"--- Если верить историкам, ты сама эволюционировала, к тебе претензий нет.

Мы рассмеялись.

Закат окончательно догорел.
Где-то вдали, с кажущихся чёрными пятнами скал, поднялись в небо пятнадцать огромных птичьих силуэтов, сделали величественный круг и унеслись на запад.
Мы с Фонтанчиком проводили их взглядом.

"--*Надеюсь, у них всё будет хорошо, "--- сказал я.

Фонтанчик промолчал, а затем весело рявкнул и, схватив Заяц в охапку, начал облизывать ей уши.
Заяц пискнула.

"--*Ты что делаешь?
Эй, хватит!
Фонтанчик!
У тебя язык шершавый!

Фонтанчик поднялся на ноги и посадил Заяц на плечо.

"--*Я, конечно, ни на кого не намекаю, но наш командир прохлаждается в ночной степи с неизвестными личностями и любуется закатом, когда у него полно дел, "--- Фонтанчик схватил меня за шкирку и усадил на другое плечо.
От неожиданности я щёлкнул зубами.
"--- Пойдём-ка, Существует-Хорошее-Небо.
Тебя ждёт твой народ.

"--*Ну вот, я вся мокрая, "--- засмеялась Заяц.
"--- Собачьи нежности\ldotst

\section{[*] Гарант}

Вскоре ливень прошёл, и ненадолго выглянуло заходящее солнце "--- первый знак, что боги благословили Середину Дождя.
Дети разбежались по домам, а Митхэ сидела у входа в святилище.
Акхсар и Эрхэ сидели с двух сторон, нежно обняв воительницу.

Митхэ тупо глядела в пространство и думала о том, что Ситрис не попался бы в такую простую ловушку.
Как обычно, он не послушал бы её приказа;
для разбойника существовал только один командир "--- собственное чувство опасности.
Он перешагнул бы и через тридцать, и через сто детских трупов, и ни один из маленьких воинов даже не успел бы выговорить слово <<mama>>.

<<Митхэ, Митхэ, "--- вспомнила воительница его полный сарказма невесёлый смех, "--- какая разница, кто перед тобой, если он собирается выпустить погулять твои кишки?>>

Митхэ вдруг осознала, что сейчас убила бы детей в любых количествах, лишь бы вернуть любимого.
Её личное счастье зависело от случайно встреченного трусоватого разбойника.
И она позволила ему уйти\ldotst

Сидевший рядом Акхсар тихо, но отчётливо ругался в чей-то адрес.
Его мысли были о том же.

<<Как же тяжело быть командиром>>.

Тяжело ступая по мокрой утоптанной глине, к воинам шёл вождь Си-Абву.
За ним плёлся сын.
На лице мальчика замерло выражение обиды.

"--*Отец, это было нечестно! "--- вполголоса сказал он.
"--- Это недостойно вождя!
Она просто хотела\ldotst

"--*Ещё раз услышу от тебя подобное "--- и ты вылетишь из моего дома быстрее ласточки, "--- суровым шёпотом рявкнул Си-Абву.
"--- Я уже всё тебе объяснил.
Пошёл вон.

Сын бросил яростный взгляд на отца и молча ушёл вниз к поселению.
Вождь встал перед воинами сели и, приосанившись, закутался в плащ.

"--*Желаешь ли ты увидеть его тело?

"--*Нет, "--- сказала Митхэ.
"--- Похороните его так, как считаете нужным.

"--*Мы съедаем тела посвящённых богам.

"--*Да будет так.

Вождь кивнул.

"--*До нас дошли слухи, что в своих землях ты теперь вне закона, Митхэ ар’Кахр.
Моя w\o izh умерла несколько дождей назад от болезни, и мне нужна новая.

"--*Что значит <<w\o izh>>? "--- равнодушно спросила Митхэ.

Вождь впервые смутился.

"--*Это значит, что женщина клянётся в верности мужчине, они живут вместе\ldotst

"--*Ты хочешь, чтобы мой клинок служил тебе?

"--*Да, я хочу этого.
А ещё я хочу, чтобы ты была в постели только со мной и рожала детей только от меня.

Митхэ спокойно рассматривала смуглое лицо вождя.

"--*Золото, мне его зарубить? "--- будничным тоном поинтересовался Акхсар.

"--*Лучше дай мне, я из его стержня сделаю свистульку и зашью в щеку его ублюдку, "--- в тон ему добавила Эрхэ.
"--- Он будет висеть на дереве на папашиных кишках и свистеть, пока не околеет от голода.

Вождь не повёл и бровью, услышав эту витиеватую угрозу.

"--*Я жду твоего решения.

"--*Я согласна, "--- чётко проговорила Митхэ.
Акхсар зажмурился, Эрхэ прикрыла лицо рукой.
"--- Ты первый мужчина, одержавший надо мной верх.
Я умею признавать поражение.
Я буду твоей\ldotst w\o izh.

Вождь поклонился и пошёл вниз по тропе, к поселению, в котором уже разгорались праздничные костры и гулял народ.

"--*Митхэ\ldotst

"--*Больше кровопролития не будет, Снежок, "--- прервала Акхсара воительница.
"--- Мы своей кровью добыли мир\ldotst и, клянусь лесными духами, пока я жива, этот мир будет под моей защитой.

"--*Это унижение! "--- воскликнул Акхсар.
"--- Он убил твоего мужчину, а теперь хочет, чтобы\ldotst

"--*Да, это унижение, "--- ответила Митхэ.
"--- И вождь за него заплатит сполна, я клянусь.
Но народ хака здесь ни при чём.
Те женщины, которых мы видели, хотят быть со своими мужчинами.
Те мужчины, которых мы видели, хотят, чтобы по возвращении домой их встречали дети.
Всем до единого знакомы чувство дружбы и любви.

"--*И всё же\ldotst

"--*Если ты не хочешь идти со мной, друг, я тебя отпускаю.
Моя жизнь в моих руках, и сейчас мне совершенно плевать, как она закончится.

Акхсар обиженно крякнул и тут же попытался скрыть это за смущённым кашлем.

"--*Твоя жизнь закончится достойно, Золото, "--- горячо прошептала Эрхэ.
"--- Ты будешь счастливее нас с Акхсаром.
Я верю в это.
Если же нет, значит, и справедливость "--- лишь красивая сказка.

Митхэ не ответила.

"--*Так ты уже всё решила? "--- наконец осведомился Акхсар.

"--*Да, Снежок, "--- кивнула Митхэ.
"--- Атрис будет гордиться мной в пристанище духов.

"--*Тебе придётся спать с ним, "--- тихо заметил воин.

"--*Я потерплю.

"--*У хака и сели не может быть детей.

"--*Похоже, что они об этом не знают, как и о том, что женщины-сели способны рожать без мужского семени.
В крайнем случае у нас есть мужчина, "--- Эрхэ лукаво подмигнула Акхсару.

"--*Ну, Снежок, как тебе план? "--- спросила Митхэ.
"--- По-моему, сели и хака давно пора завязать отношения попрочнее.

Акхсар угрюмо посмотрел на женщин из-под седеющих бровей\ldotst и вдруг тепло улыбнулся тонкими губами.

\section{[-] Полководец}

\spacing

Я осмотрел молчаливые ряды сели.
Как убедить их следовать за мной?
Ветер трепал перья моей короны.
В памяти разом всплыли картины давних битв и походов "--- полководцы на устрашающих машинах или лучших ездовых животных, красочные плакаты, кричащие знамёна, полные праведного гнева речи\ldotst
Но все пафосные слова улетучились, едва я взглянул на этих простых бесхитростных людей.
Кто-то смотрел на меня насмешливо, кто-то с тенью жалости "--- я выглядел чересчур молодым для своей роли.
И только Чханэ, верная Чханэ, стоящая в первом ряду с нашим ребёнком на руках, ободряюще мне кивнула.

Вдруг моё внимание привлёк стоявший рядом с ней мужчина.
У него было тело строителя, привыкшего таскать тяжёлые камни и брёвна, покрытые мозолями руки.
Но, несмотря на внушительную физическую силу, его ноша была чересчур тяжела "--- помимо шести мешков он нёс на руках двух детей.

Я сделал первое, что пришло мне в голову.
Я спрыгнул с оленя, подошёл к мужчине, снял с его плеч три тяжёлых, словно набитых камнями мешка и взвалил их себе на спину.
Сейчас я понимаю, как смешно выглядел в тот момент: роба моя перекосилась, её пола задралась выше пояса.
Налобный ремешок одного из мешков смял перьевую корону, а затем сорвал её с головы, и лёгкий головной убор, подхваченный ветром, улетел куда-то в кусты.
Но я старался об этом не думать.
Схватив оленя под уздцы, я молча пошёл по дороге.

И сели двинулись за мной.

\chapter{[-] Исход}

\section{[-] Союз с хака}

\spacing

Я оглянулся.
Передо мной стоял Имжу.
Я узнал его с трудом "--- за прошедшие годы юнец с превратился в сильного, уверенного в себе мужчину, и глаза его, прежде неукротимые и яростные, светились холодным светом недюжинного ума.
Но самая главная перемена "--- он перестал чернить кожу и красить ресницы, маскируясь под хака.
Ярко-зелёные глаза принадлежали моему брату "--- теперь в этом уже никто не мог сомневаться.

Рядом, робко глядя на меня исподлобья, стояла Митхэ ар’Кахр.

"--*Мои приветствия, Ликхмас ар’Люм, "--- Имжу, сохраняя вежливую полуулыбку, заговорил на чистейшем цатроне.
"--- Рад увидеть тебя снова.

Я вместо ответа подошёл и по очереди обнял гостей.
Имжу дёрнулся, но ответил на объятия со всей искренностью, которую позволяла ситуация.
Митхэ обняла меня крепко-крепко и тут же отстранилась.

Я пригласил Имжу и Митхэ за стол и предложил им еды.

"--*Мы не голодны, Ликхмас, "--- ответил Имжу.
"--- К тому же вряд ли у нас есть время на застолье.

"--*Я вижу, что ты стал вождём, "--- сказал я, отметив число и цвет перьев в волосах.
Три красных, голубое и пучок фиолетовых перьев согхо "--- вождь был избран во время изнурительного похода, когда его предшественника нашла гибель.
Осанка и уверенный, повелительный тон мужчины только подтвердили этот внушительный знак.

"--*Меня выбрали боевым вождём племени Inh\o s-laka\footnote
{Спокойное озеро "--- местность на границе с Молчащими лесами. \authornote}.

"--*Итак, я тебя слушаю.

Имжу начал рассказ.
По его словам, хака давно готовили новый поход на соседей.
Об этом редко говорили вслух, но очевидное скрыть сложно.
Однако время похода, его масштабы и, главное, направление удивили всех.

"--*Что именно тебя смутило?

"--*Старейшины единогласно приняли решение о походе на внеплановом совете, "--- ответил Имжу.
"--- Между тем более половины моих соплеменников были против войны с сели.
Именно поэтому я здесь.
Объясни мне, брат, почему против сели одновременно выступили все окрестные народы?
Мы слышали, что и пылерои пустынь, и идолы Живодёра пошли против вас.
Даже тенку вдруг вспомнили старые обиды "--- слышны вести, что они всерьёз намерены отомстить и вернуть себе земли, потерянные во время Второй Приречной войны.
Все эти племена не могли договориться между собой.
Притязания тенку даже выглядят смешными "--- Приречная война закончилась шестьсот дождей назад.

"--*Ты не считаешь это случайностью?

Имжу хмыкнул, и на секунду в его спокойном облике вновь проступило юношеское высокомерие.

"--*Только невежда посчитает это случайностью.

"--*Ты пришёл за ответом?

"--*Да, "--- сказал Имжу.
"--- И ради этого ответа я поставил под удар свою репутацию, приостановив диверсии хака на ваши отступающие силы.

"--*Сели выступили против культа Безумного, "--- сказал я.

Имжу присвистнул.

"--*Странная реакция, "--- заметил я.
"--- Похоже, что ты не удивлён.

"--*Разумеется, "--- ответил молодой вождь.
"--- На совете хака звучала речь об этом.
Старейшины сказали, что Безумный покарает сели, и завоевать их земли будет намного проще.
Но почему сели пошли на это самоубийство?

"--*Потому что сели не считают это самоубийством, "--- подал голос Грейсвольд.

Имжу холодно воззрился на технолога.

"--*Объясни свои слова, человек.

"--*Мы можем уничтожить Безумного, "--- ответил я.
"--- Так что, если хака вспомнит, сколько его родичей умерло на алтарях и сколько было убито сели, то может сделать правильный выбор.

Митхэ словно воды в рот набрала.

"--*Это невозможно, "--- проворчал Имжу.
"--- Я вам не верю.

"--*В таком случае мы проиграем, "--- просто ответил Грейсвольд.
"--- Хака достанутся восточные земли сели, и твой народ будет поклоняться Безумному, как и десять тысяч дождей до этого времени.
У вас будут охотничьи угодья и плодородные сады, ваши земли будут граничить с воинственными тенку и несговорчивыми идолами Живодёра, а сели будут уничтожены и канут в забвение.

"--*Не самая приятная перспектива, "--- заметил Имжу.
"--- Многие хака мечтают завоевать земли сели, но мало кто задумывается о том, как оборонять эти земли от пятнадцати пылеройских кланов и идолов Живодёра.

"--*Дружить всегда выгоднее, "--- заключил Грейсвольд.

Имжу сидел и смотрел на меня проницательным взглядом.

"--*Вы сами верите в свои слова, "--- наконец сказал он.
Это был не вопрос, а утверждение.

"--*Ты можешь нам помочь?

"--*Я могу встать между вами и союзом племён, "--- ответил Имжу.
"--- Но это не продлится долго.
Мои соплеменники признают меня миролюбцем и лишат звания.
Мне очень повезёт, если я останусь в живых.

"--*Но если останешься, то о тебе будут слагать легенды ещё десять тысяч дождей, "--- пообещал Грейсвольд.
"--- Ни один вождь хака пока не осмелился бросить вызов Безумному.

Имжу опустил голову.
Последние слова явно задели его честное, но честолюбивое сердце.

"--*Есть ли недовольные среди старейшин? "--- поинтересовался я.

"--*Хватает, "--- уклончиво ответил Имжу.
"--- Решения исходят от небольшого замкнутого круга, члены которого хорошо мне известны.

Интеллект молодого вождя поражал.
Даже в племени дикарей гены великих учёных-тси расцвели в полную силу.
Я испытующе посмотрел на Митхэ "--- по словам Кхотлам и Акхсара, она была сильна в бою, но отнюдь не в интригах.

"--*Есть ли возможность их устранить?

"--*Очень маленькая, "--- Имжу ответил не раздумывая.
Похоже, он сам много над этим размышлял.
"--- Даже если затея удастся, это восстановит народ против нас.
Место убитых займут новые, кто пока скрывается в тени.

Мы с Грейсвольдом переглянулись.
Разумеется, это был Картель "--- этот стиль сложно не узнать.
<<Разделяй и властвуй>>.
<<Держи ресурсы в мусорной куче>>.

"--*Возможно, мне стоит отправиться с Имжу и решить дело на месте? "--- предложил Грейсвольд.

"--*Можем ли мы тебе доверять? "--- впервые подала голос Митхэ.

Грейсвольд усмехнулся своей крупнозубой лучистой улыбкой.
Я удивлялся тому, что технолог тратил много времени на получение последних обновлений по психологии улыбки, но после понял, насколько мощным оружием она была.
Отточенная до мельчайших деталей, максимально подогнанная под тип лица и при этом непринуждённая, улыбка Грейсвольда обезоруживала людей, и даже самые яростные противники технолога не могли ей противиться.

"--*Твои предки ещё не покинули Тхидэ, когда я одерживал первые победы, Митхэ ар’Кахр.

"--*Ты не ответил на её вопрос, "--- заметил Имжу.

"--*Верно, "--- согласился Грейсвольд и вышел.

\section{[-] Опыт и знания}

Вскоре старейшины были устранены.
Грейсвольд применил весьма точный и изящный тактический приём.
Я не буду приводить его в книге, так как он до сих пор представляет нашу с Грейсвольдом тайну "--- все причастные демоны Картеля уничтожены.

Грейсвольд приобрёл большой вес.
Я уже не был для Имжу неоспоримым авторитетом.
Стоило технологу взять слово "--- и молодой вождь вместе со своими людьми сидели и слушали, открыв рты.
Мы с Анкарьяль только посмеивались.

"--*Я скоро начну ревновать, "--- пошутил я.

"--*Эх, если бы все ценили опыт и знания, как эти дикари, "--- со вздохом заметила Анкарьяль.
Я мысленно с ней согласился.

\spacing

\section{[-] Свобода}

\spacing

"--*И что теперь получается? "--- скривившись, произнёс Митракх.
"--- Мы одного хозяина "--- Безумного, да охранят нас лесные духи от гнева Его, "--- поменяем на других?
Вас?
А как же свобода?
Ликхмас-тари, ты говорил, что мы будем свободны!

Я хотел ответить, но меня опередил старик Саритр:

"--*Митракх, "--- прохрипел он и закашлялся, сложившись вдвое.
Пара молодых воительниц поспешили к нему и подхватили пожилого крестьянина под руки.

"--*Храни вас лесные духи, дочки. "--- Саритр выплюнул на раскалённый камень кровавый комок и утёр рот рукавом, глядя на подбоченившегося молодого кузнеца.
"--- Проклятая пустынная пыль\ldotst
Да, да, я стою, обереги вас.
Митракх, дитя, над тобой всю жизнь властвовал Совет.
Сейчас нас в бой ведёт вождь.
А ещё тобой безраздельно правит твоя женщина, "--- старик прочистил горло и кивнул на стоявшую неподалёку крестьянку.
Та зарделась.

Митракх стушевался, глядя то на старика, то на смущённо теребящую эфес фаланги женщину, но тут же обрёл прежний надменный вид.

"--*Ты всю жизнь кому-то служил, дитя, как и я.
Какой свободы ты хочешь?

"--*Вождя выбрали на совете, "--- вызывающе произнёс Митракх.
"--- Выбрали самого достойного, того, кто думал о благе племени больше, чем о еде на завтрашний день.
А женщину я выбрал сам "--- я знал, что она меня любит.

"--*Правильно, парень.
Почему мы служили Безумному?
Скольких детей он забрал у тебя?
Пятерых?

Митракх стоял, опустив голову.
Его женщина прикусила губу.

"--*Будь забыто Его имя, "--- хрипло произнесла она.

"--*Посмотри на него, Митракх, посмотри, "--- старик кивнул на меня.
"--- Он не такой, как Безумный.
Это видно сразу.
Да, идолы меня разгрызи, если бы даже молодой Ликхмас попросил одного ребёнка из тех восьми, кого я лично за руку отвёл\ldotst "--- Саритр опять зашёлся надрывным кашлем.
На этот раз его подхватили мы с Митракхом.

"--*Ох, обереги вас духи\ldotst
Недолго мне осталось, дети.
Совсем недолго, "--- просипел старый крестьянин, улыбаясь измученной улыбкой.
"--- Хотелось бы погулять с вами на празднике после битвы, но я её не переживу.
Надеюсь, хоть кого-то из врагов с собой утащу\ldotst

Митракх не ответил.
Он застывшим взглядом смотрел на обвисшего у нас на руках Саритра.
В его глазах медленно зрело понимание\ldotst

"--*Не волнуйся, Ликхмас ар’Люм, "--- наконец пробормотал он, старательно избегая моего взгляда.
"--- Отпусти его, он стоит.
Я отведу дедушку к воде, ему станет легче.
Идём, Саритр-лехэ, идём\ldotst

\section{[-] Подмога}

\spacing

<<Неужели мне придётся сражаться с ними в одиночку?>> "--- мелькнуло у меня в голове.

Однако спустя две михнет со стороны моря раздался протяжный свист.

"--*Корабли! Корабли! "--- закричали сели.

Я поскакал к морю.
Могильный пролив пестрел парусами "--- зелёными, красными, синими, белыми.
Родовые знаки ноа, клановые гербы трами.
Вода, рассекаемая штевнями, вёслами и натянутыми фалинями, кипела от бьющихся хвостов.

"--*Они запрягли дельфинов в корабли? "--- вдруг недоумённо спросила женщина рядом со мной.

Она оказалась права.
Дул бейдевинд, не позволявший развить большую скорость.
Дельфины притащили огромный флот Кита к Могильному берегу меньше чем за кхамит "--- аккурат к сражению.

Анкарьяль соскочила с первой же лодки и бросилась ко мне через толпу ноа, которые с радостными криками бросились обнимать соплеменников.

"--*Соскучился? "--- ухмыльнулась она.
"--- Пойдём на совет.
Грейс сейчас будет, приведёт старейшин.

"--*Но как?

"--*А, ты про дельфинов.
Они сами предложили помощь, представляешь?
В Коралловую бухту приплыло более шестнадцати тысяч "--- чуть ли не две трети племён Кипящего моря.
По расчётам Картеля флот должен был прибыть после того, как основные силы сели будут разгромлены.
Мы им поломали очередной план.
Пойдём, пойдём.
У нас есть десять михнет до того, как нужно будет послать парламентёров.

"--*Ты думаешь, я смогу с ними договориться?

"--*Вероятность мала, "--- пожала плечами подруга, "--- но переговоры "--- это лишние михнет для доработки боевого строя.
Хотя\ldotst вдруг тебе повезёт.
Я уже даже не удивлюсь.

\section{[-] Причина промедления}

\spacing

"--*Почему он не атакует нас сейчас? "--- задумался я.
"--- Что его останавливает?

"--*Я знаю Эйраки, "--- ответил Грейс.
"--- Он самоуверен и скуп до ужаса.
Он хочет покончить со всеми "--- с нами, сели и прочими восставшими тси "--- одним ударом.
Раньше я сомневался, действительно ли это он Хатрафель Безумный, но сейчас сомнений больше нет.

"--*Хорошо, "--- согласился я, "--- тогда что останавливает Картель?
Они-то не дураки?

"--*С этим сложнее, "--- признал Грейсвольд.
"--- Во-первых, неясно, почему Картель терпит присутствие Эйраки.
Они надеются использовать его как источник энергии?
Возможно, но для источника энергии он чересчур сильно вмешивается в их стратегию.
Картель славится привычкой растаскивать кости по одной прежде, чем их сгребут в кучу.

"--*А диверсии?

"--*Будь воля Картеля, Аркадиу, диверсий было бы столько, что от Кахрахана мы шли бы в гордом одиночестве.
Эйраки их сдерживает, это безусловно.

"--*Они надеются использовать его как ударную силу? "--- предположил я.

"--*Вполне правдоподобно, если не знать, что агентов Ада здесь трое, и у двоих нет энергии на обратную дорогу.
Об этом Картель уже осведомлён.
Встаёт вопрос: против кого они собираются использовать Эйраки?

"--*С чего ты взял, что его хотят использовать против кого-то конкретного, а не как бесплатную охрану? "--- скептически осведомилась Анкарьяль.
"--- Если мы всё-таки смогли бы послать сообщение своим\ldotst

"--*\dots то никто бы не пришёл, "--- закончил Грейсвольд.
"--- Стратеги Ада не станут посылать силы на верное самоубийство непонятно ради чего, и Картель это знает.

Мы переглянулись.

"--*История тёмная, ребята, "--- пробормотал Грейсвольд, "--- однако у меня стойкое ощущение: Картель чувствует угрозу, и угроза исходит не от нас.
Именно поэтому Эйраки ещё жив.
Разумеется, Картель будет использовать демиурга и против нас, и против наших сапиентов, которые благодаря их размолвкам превратились в серьёзную силу\ldotst

"--*Серьёзную? "--- хохотнула Анкарьяль.

"--*Нар! "--- внушительно буркнул Грейсвольд.
"--- Взгляни правде в глаза "--- у нас в руках сейчас сила, которая способна попортить кровь даже отожравшемуся до ка'нетовской константы демиургу, не говоря уже о местном ковене Картеля.
Если против нас Эйраки мог применить грубый поток масс-энергии, то сели на него плевали, они даже ничего не почувствуют.
Потребуются глубокие знания в технологиях массового поражения и траты масс-энергии, чтобы хоть что-то с ними сделать!

"--*А МПДЛ? "--- спросил я.

"--*Химическое оружие против полумиллиона сапиентов, рассеянных по всему Могильному берегу?
Извините, не так "--- против четырёхсот тысяч тси, которые осведомлены о природе <<божественной кары>> и в любой момент могут реализовать протокол <<Кристалл>>?
Это даже не смешно.
Тем более МПДЛ "--- скорее деморализующее средство.

"--*Тектонические сдвиги?

"--*Судя по непрекращающимся землетрясениям, для Эйраки что-то раскачать так же сложно, как и что-то успокоить.
Для демиурга он плоховато знает свою планету.

"--*Оружие массового поражения наверняка есть у Картеля, они никогда\ldotst

"--*Сомневаюсь, что у них есть что-то серьёзнее P-гранат, "--- перебил меня Грейсвольд.
"--- Для того, чтобы держать в повиновении дикарей, этого достаточно.
Чтобы выпустить терракотового волка, нужны серьёзные затраты масс-энергии.
Доверит ли Картель свои технологии Эйраки?
Наши шансы высоки, пока голова, руки и палка принадлежат разным людям.
Кстати, а что насчёт тех осцилляций, которые\ldotst

"--*Я уточнила наши шансы, "--- перебила его Анкарьяль.
"--- Один к двумстам тридцати.
При том, что, по твоим словам, всё сложилось для нас невероятно удачно.

"--*Ну, двести тридцать "--- это уже не десять тысяч, согласись, "--- заметил технолог.

"--*Чтоб тебе с таким шансом ложка в рот попадала, "--- съязвила Анкарьяль.
Она всегда становилась очень раздражительной перед серьёзным боем "--- между мозгом и хоргетом шёл интенсивный обмен информацией, каждый навык проверялся и перепроверялся, и мозг на вмешательство в его дела реагировал испорченным настроением.

"--*Подожди, Нар, "--- махнул я рукой.
"--- Грейс, ты что-то начал говорить про осцилляции.

"--*Да? "--- удивился технолог и задумался.
"--- Лесные духи, уже забыл.
Наверное, что-то несущественное.

\section{[-] Стрелохвосты}

У кромки берега мы с Чханэ заметили чьё-то верещащее тельце.

"--*Ребёнок, "--- ахнула Чханэ и бросилась на помощь.

Подруга оказалась почти права.
Стрелохвост-подросток, из любопытства подплывший чересчур близко к берегу, оказался зажатым в скалистой расщелине.
Увидев нас, он испуганно заверещал и забился ещё сильнее.

"--*Нянечка-нянечка, "--- ласково заговорила Чханэ.
"--- Не шевелись, я тебя вытащу.

Мы аккуратно повернули дельфина и освободили зажатый хвостовой плавник.
Стрелохвост бросился наутёк, но шагах в двадцати вдруг повернулся к нам и сделал великолепный тройной пируэт.

"--*Это он благодарен, "--- объяснил я.

Чханэ фыркнула.

"--*И так ясно.

"--*Ничего не ясно, "--- возразил я.
"--- Может быть, он чего-то испугался в глубине.

"--*Он бы подпрыгнул и ударил хвостом по воде, "--- тоном учителя объяснила Чханэ.

"--*Ты раньше общалась со стрелохвостами?

"--*Я их впервые вижу, но это очевидно.
Всё, Лис, не докучай, нам нужно догнать нашу колонну.

Вот и ещё один любопытный факт из психофизиологии тси, требовавший объяснения.
Я и раньше замечал, что человеческие дети безошибочно угадывали невербальные сигналы апид, достаточно далёкой в эволюционном плане Ветви.
А Чханэ знала, как выражают эмоции стрелохвосты, не видев их ни разу в жизни.

Через полдня пути у кромки берега начали появляться другие стрелохвосты.
Вначале десяток, потом два. Они перекликались, и в их голосах звучало удивление.
Сели, которым редко случалось увидеть стрелохвостов, сбежались на берег и начали размахивать руками.
Вскоре некоторые, в особенности знающие языки купцы, начали различать в гомоне стрелохвостов знакомые слова.

Купчиха Ситхэ, идущая рядом с нами, долго вслушивалась, а потом вдруг ахнула:

"--*Я начала понимать их язык, надо поговорить!

Ситхэ забежала в воду и махнула остальным.
Сели умолкли.
Дельфины, как ни странно, поняли, что с ними хотят поговорить, и подплыли поближе.

"--*Хай! "--- закричала Ситхэ.
"--- Мы знаем, как победить богов!
Безумного!
Нам нужна ваша помощь!

"--*\emph{Помощь? Помощь?} "--- загомонили дельфины, разобрав знакомое слово.
Вперёд выплыл белый как снег крупный самец.

"--*\emph{Помощь почему?} "--- вдруг спросил он.

"--*\emph{Помощь почему? Помощь почему?} "--- заверещали дельфины.

"--*Безумный бог! "--- крикнула Ситхэ, изо всех сил пытаясь подражать непривычному говору.

"--*\emph{Помощь почему?} "--- снова спросил белый стрелохвост.

"--*Убивать "--- мучить "--- нельзя видеть! "--- крикнула Ситхэ и для наглядности указала пальцем в небо.

Стрелохвосты вдруг замолчали, и все сели почувствовали "--- они поняли и осознали последнюю фразу.
Дельфины всегда были гораздо более свободолюбивыми, нежели сухопутные сапиенты.
Прежде они никогда не знали рабства, единственной непреодолимой границей для них был берег.
Они не приняли ультиматум Безумных "--- и боги в ответ подвергли морских обитателей геноциду.

"--*\emph{Это нельзя,} "--- вдруг сказал белый стрелохвост.

"--*\emph{Нельзя,} "--- как-то неуверенно согласились остальные дельфины.

"--*Мы можем! Нужна помощь! "--- крикнул кто-то из рядов сели, сообразив, какие слова стрелохвосты легко понимают.
Остальные подхватили его слова:

"--*Мы можем!

"--*Нужна помощь!

Белый дельфин отвернулся и поплыл к ожидающим его соплеменникам.
Вскоре они молча ушли в сияющие под вечерним солнцем воды.

"--*Струсили, "--- разочарованно крякнул какой-то старик.

"--*Знаешь, Хонхо-лехэ, я их не виню, "--- многозначительно ответила женщина со стигмами беременности и небольшим животиком.

Сели, удручённо переговариваясь, отправились дальше.

Наутро нас ждал сюрприз.
Прибрежные воды кипели от бьющих хвостов.
Дельфины плыли параллельно идущим по берегу сели и одобрительно свистели.
Я насчитал как минимум полторы тысячи особей.

Сели приободрились.
Отступление медленно превращалось в победный марш.
Моё сердце запело.
Оставалось взять последний аккорд "--- вдали показались жёлтые башни крепостей ноа.
Feci quod potui.

\section{[-] Тайна имён}

\spacing

"--*Это правда? "--- резко спросил Сайрулай.
"--- Вы говорили с дельфинами?

"--*Да, "--- раздался женский голос, и вперёд вышла купчиха Ситхэ.
"--- Я говорила с океаническим народом.

"--*Моряки ноа умеют общаться с дельфинами, но, по словам очевидцев, сели говорили с дельфинами на их языке, "--- продолжал Сайрулай.

"--*Наши языки очень сильно похожи, "--- поклонилась Ситхэ.

По залу прокатился вздох потрясения.

"--*Сели "--- родичи дельфинов? "--- выкрикнул кто-то.
"--- Вы вышли из моря?

Я поднял руку.

"--*Прошу слова, сенвиор Сайрулай.

"--*Говори, "--- кивнул ноа.

"--*То, что видели твои люди "--- правда.
Это имеет под собой причину, о которой почти забыли сели и давно забыли вы.

Я помолчал. Сайрулай прищурился.

"--*Все мы "--- сели, ноа, дельфины, идолы, травники, народ трами и ркхве-хор "--- пришли с планеты Тхидэ.
Когда-то мы были единым народом, ели одну пищу и говорили на одном языке.
Я могу рассказать причину, по которой языки сели и ноа отличаются больше, чем языки сели и дельфинов.
Более того, я могу назвать имена тех, кто придумал ваши обычаи и ритуалы.

Я поднял над головой книгу Существует-Хорошее-Небо и вкратце рассказал о плане тси.
В зале воцарилась мёртвая тишина.

"--*Если вы мне не верите, задумайтесь о том, почему сели и ноа могут иметь детей, а браки ноа с тенку, хака и зизоце обречены на бесплодие, "--- закончил я.
"--- Если вы мне не верите, посмотрите на лорику на груди ваших воинов.
Ваша кожа принимает кожу сели и идолов, как родную, но отторгает кожу прочих народов.
Если вы мне не верите, послушайте свои тела "--- они помнят давнее родство.

Зал по-прежнему молчал, и я продолжил:

"--*Когда-то давно наши предки, тси, говорили на равных с богами этого мира.
Они не обладали божественной силой, вы будете даже покрепче и половчее их.
Но силы, подобные Безумному, трепетали, едва заслышав их имена.
Трёхсложные имена, которые до сих пор носят сели;
<<тайные прозвища>> идолов Живодёра, состоящие из трёх слов;
ваше, ноа, <<имя-ключ>>, который получает каждый ребёнок и которое хранит в тайне всю жизнь, состоит из тех же трёх слов.

Я подошёл к Сайрулаю.

"--*Как тебя зовут, тси?

Сайрулай в смятении смотрел на меня.
Он понял, что я от него хочу.
Он потоптался на месте и оглянулся на сидящих в зале.

"--*Сайрулай, не надо!
Не делай этого! "--- раздался из толпы женский голос.
"--- Они заберут воду из твоих глаз и воздух из твоих лёгких!

Сайрулай судорожно выдохнул и вдруг в упор посмотрел на меня.
На смуглом лице расцвела улыбка.

"--*Лёд-Сковавший-Траву, "--- полушёпотом сказал он.

"--*Играющая-Камнем-Лиса, "--- громко представился я и протянул ему руку.
"--- Жизнь и процветание, друг.

\section{[-] Вместе}

\spacing

В первый день сели и ноа сторонились друг друга и упрямо не хотели идти вместе.
На второй день они уже ели из одних котлов, но по-прежнему не желали беседовать.
На третий день отличить в толпе сели и ноа уже было невозможно "--- мне попадались развесёлые отряды людей непонятной этнической принадлежности, выряженных как попало и гомонящих на смеси сели, ноа-линга и отвратительного цатрона.

"--* Ну что ж, культурную реинтеграцию можно считать состоявшейся, "--- невозмутимо резюмировал Грейсвольд.

"--* Как-то она чересчур быстро произошла, "--- недоверчиво буркнула Анкарьяль.
"--- Так не бывает.

"--*Возможно, создатели культур предусмотрели на этот счёт какие-то лазейки, "--- предположил я.
"--- Даже не возможно, а скорее всего.
Языки друг друга они знали прекрасно, но говорить считали ниже своего достоинства.

"--*Одну из лазеек мы и использовали, "--- заметил Грейсвольд.
"--- Имена.

"--*Тогда почему они не подружились раньше? "--- скептически спросила Анкарьяль.

"--*Не было повода, "--- буркнул Грейсвольд.

Анкарьяль покачала головой и ухмыльнулась.

"--*Дьявол.
Хотела бы я познакомиться с этими Баночкой и Кошкой.
Заставить два народа вести тысячелетнюю вражду, чтобы при надобности они безо всякого труда объединились "--- прямо перед носом ошеломлённого врага.

\section{[-] Встреча с Птичкой}

\spacing

"--*Секхар! "--- вдруг пискнул кто-то, и на шее у Грейса повисла счастливая Тхартху.

"--*Птичка! "--- изумлённо буркнул он.
"--- Почему ты не ушла на Север?

"--*Меня мучила совесть, что я бросила вас тогда, "--- смутилась женщина.
"--- Поэтому я здесь.

"--*Разумеется, ты не собираешься сражаться?

"--*Разумеется, собираюсь, "--- обиделась Тхартху.
"--- У меня есть дети и мужчина, я не хочу их опозорить.
В случае поражения нас и так всех уничтожат.
Да, я уже слышала, какие лепёшки вы напекли в своих походах.
Об этом только и говорят.

"--*Тхартху! "--- сурово крикнул сзади низкий женский голос.
Тхартху снова пискнула и бросилась обнимать Анкарьяль.

"--*Ну-ка брысь отсюда! "--- без предисловий бросила воительница, оторвав от себя Тхартху за шкирку, как котёнка.
"--- Тут сейчас кровавый шторм будет!

"--*Нар, а я все эти годы тренировалась, рисовала! "--- Тхартху чисто по-женски проигнорировала последние слова подруги.
"--- Хочешь, тебя нарисую?
Ты красиво смотришься в одежде ноа.

"--*Она рисует, как сам Митр не умеет, "--- поддержал её неожиданно подошедший мужчина.
"--- Благодаря ей мы живём в хорошем достатке.
Она расписывает храмовые книги и свои делает, для детей\ldotst

"--*Хай, хватит, "--- вдруг смутилась Тхартху.
"--- Не надо про мои книги.

"--*Вот, "--- мужчина, не обращая на неё внимания, гордо вытащил из-за пазухи тоненькую книжечку сказок.

<<Приключения Секхара, звёздного странника>>.

"--*Эээ\ldotst "--- только и смог протянуть Грейсвольд.

"--*Дети их до дыр затирают, "--- радостно сообщил мужчина.
"--- Фантазии ей не занимать\ldotst

"--*Митрис! "--- многозначительно сказала Тхартху.
"--- Подожди меня, я должна с ними поговорить.

Мужчина довольно осклабился и ретировался с книжкой в руках.

Мы помолчали.
Наконец Тхартху по очереди нас обняла.

"--*Рада вас видеть.
Я буду молиться лесным духам, чтобы завтра они, вопреки своей природе, сражались с нами бок о бок.

"--*Птичка, держись подальше от фронта, "--- попросил Грейсвольд.

Тхартху улыбнулась технологу и лёгкой гибкой тенью исчезла в толпе.

\section{[-] Собачья песня}

\spacing

Анкарьяль вместо ответа запела песню на языке Хргада, развитой цивилизации кани на одной далёкой планете:

\begin{verse}
И дочери в шаг [один миг] лишились отцов,\\
И сыновья в шаг лишились матерей,\\
Щенята [дети] плакали и пели,\\
И кровь была им горька, как полынь-трава,\\
И плоть была им подарком перед [временем] голода и смерти\footnote
{У кани Хргада была традиция поедать мёртвых после битвы.
Обычно в пищу употреблялись враги, в песне же говорится о поражении "--- щенятам пришлось есть собственных родителей, чтобы не умереть с голоду. \authornote}.\\
\end{verse}

Звуки песни, напоминавшие поскуливание и вой, заставили вздрогнуть стоявших вблизи воинов.
Когда-то, ещё на заре цивилизации, люди пытались обучить кани музыке.
Собаки прекрасно усвоили эту неотъемлемую часть человеческой культуры, но для себя стали сочинять нечто особенное, непривычное человеческому уху.
Впрочем, для тех, кто знал язык, песни кани были вполне понятными "--- про любовь и славу, радость и скорбь.
Да и чувства юмора собаки не были лишены.

\section{[-] Мысли о броненосце}

\spacing

"--*Итак, "--- начал я.
"--- Первое, что вы должны запомнить.
Для демонов не являются препятствием ни ваши доспехи, ни ваше оружие, ни ваши тела.
Если он захочет убить именно вас, он сделает это мгновенно или почти мгновенно, и ничто не сможет ему помешать.
Демон в живом теле имеет слабое место "--- он интимно связан со своим носителем, тело составляет его важную, но не обязательную часть.
Вы можете убить носителя и на несколько секхар и даже михнет вывести демона из строя "--- он будет дезориентирован и не сможет сражаться.
Многие демоны вообще не запрограммированы на серьёзную деятельность без тела "--- то есть, убив тело, вы обезвредите демона надолго.
Но саму сущность не под силу убить никому из вас.

"--*Даже одну? "--- задал кто-то вопрос.

"--*Даже одну.

Окружающие переглянулись.

"--*Тогда как мы можем их победить? — возмущённо пробубнил седой как лунь старик с выпученными глазами.

"--*Если говорить предельно просто "--- вашими чувствами.
Любовью, счастьем, нежностью, ощущением здоровья, благополучия.
Яростью, упоением.
Любыми светлыми сильными чувствами, которые вы можете испытывать.

"--*Как чувства помогут их одолеть? "--- подняла руку молодая воительница слева.

"--*Хоргеты\ldotst эээ\ldotst как бы состоят из чувств.
Это их пища, их суть.
Те, против которых мы сражаемся, состоят из ненависти, страха и бессилия.
Чувства, которые испытывает каждый из вас, чересчур слабы, чтобы уничтожить даже одного из них.
Однако, согласно нашим данным, у них недостаточно визоров, и наше воинство способно сильно помешать демонам и даже полностью дезориентировать их.
Они будут словно летучие мыши в ярко освещённой комнате "--- биться о стены и ломать себе крылья.
Тогда они станут уязвимы, и мы с Анкарьяль\ldotst Хатлам ар’Мар сможем их уничтожить.
Мы знаем, как это делать.

"--*Мне не нравится, что целый народ должен полагаться на трёх якобы демонов, "--- пробормотал сидевший в углу мужчина.
Остальные зашептались, выражая согласие.

"--*К сожалению, плана лучше я предложить не могу, "--- развёл я руками.
"--- Безумный даже без Картеля чересчур силён, а у нас недостаточно энергии, чтобы связаться с союзниками и позвать подмогу.
Да, мы разные, но наши судьбы связаны: вы "--- наша единственная надежда на выживание, а мы "--- ваша единственная возможность победить.
Разумеется, кто хочет, тот может уйти перед битвой.
С каждым ушедшим наши шансы на победу упадут на сотую процента.

Неожиданно для меня последняя фраза оказала магическое действие "--- шёпот прекратился.

"--*Лисёнок, ты так и не сказал, что мы должны делать, "--- подала голос Кхотлам.

"--*Как бы это ни звучало "--- нужно сражаться, испытывая любовь, сострадание и нежность.

"--*Малыш, я тебя не понимаю.

"--*Ну как же, кормилица, "--- укоризненно сказал я.
"--- Мне ли рассказывать тебе, как идти в бой с любовью в голове?

Кхотлам смотрела на меня.
Остальные смотрели на Кхотлам.

"--*Думай о вкусном броненосце, "--- скопировав её интонации, произнёс я.

Кормилица с нежностью посмотрела на меня, потом повернулась к застывшим в непонимании людям.

"--*Сейчас объясню.

\section{[-] Переговоры}

\spacing

\mulang{$0$}
{"--*Лусафейру, "--- ухмыльнулся Эйраки.}
{``Lusafejru,'' Ejraci smirked.}
\mulang{$0$}
{"--- Непутёвый сын и здесь не может оставить старого отца в покое.}
{``The prodigal son refuses to leave his old man in peace, even here.''}

\mulang{$0$}
{"--*Безусловно, история твоей отцовской любви весьма трогательна, "--- поморщился я, "--- но я здесь не за этим.}
{``Certainly, the parable of your paternal love is very touching,'' I've made a wry face, ``but I'm not here for that.}
Я предлагаю Эйраки Морозу, бывшему максиму прима Ордена Преисподней, вернуться в созданную им фракцию.
Солдатам Картеля я предлагаю отступить в соответствии с\ldotst

Мои слова прервал громкий хохот.
Даже Эйраки не смог удержаться от тонкой, похожей на обезьяний оскал улыбки.

"--*В моей новой фракции, "--- Эйраки изящно оттенил эти слова движением бровей, "--- есть такая\ldotst хмм\ldotst поговорка.
<<Хозяин не предлагает сдаваться, хозяин принимает капитуляцию>>.

"--*Автор этого изречения не мудр, "--- заключил я.

"--*Сколько вас, предатель Люпино? "--- осведомился один из командиров Картеля.
"--- Двое?
Четверо?
Во сколько раз уменьшится войско Ада, если я тебя сейчас убью?

"--*Вы устроили великолепный спектакль, Аркадиу, но не более того, "--- сказал Эйраки.
"--- Передайте этим дикарям: пусть расходятся по домам, если не желают их потерять.
Мне нужна биомасса.
Как после этого будете спасать свои жизни вы "--- не моё дело, но это однозначно ваш шанс.

Демоны снова расхохотались.

Вдруг сзади застучали копыта и закричал олень.
Я обернулся.
Кормилица скакала к нам с непередаваемой грацией "--- я даже не знал, что она умелый наездник.

Кхотлам подъехала ближе и осадила оленя.
Оценивающий взгляд пробежался по врагам, тут же выделив главного "--- Эйраки.

"--*Это ещё кто? "--- осведомился Эйраки.

"--*Эйраки Марос, называемый также Безумным, "--- слова сектум-лингва Кхотлам произнесла настолько чисто, с характерной презрительной интонацией, что я вздрогнул.
"--- Я, Испачканное-Мёдом-Перо, принесла тебе сообщение.
Ты терроризировал моих людей на протяжении десяти тысяч дождей, и мы ничем не могли тебе ответить.
Но рано или поздно любому террору приходит конец.
Именем сели\ldotst нет, \emph{именем тси} я обвиняю тебя в Разрушении.

На лицах Эйраки и агентов Картеля застыло изумление.
Изумление и страх.
Название древнего народа, который на равных говорил с Адом и Картелем, ещё жило в памяти демонов.

"--*Согласно законам нашего народа, за подобное преступление казнят.
Но я хочу дать тебе шанс и последовать путям великих предков, о которых поведал мне сын.
<<Ушедший неподсуден>>\footnote
{Eksita not ivrisdikta "--- общий для многих миров закон, являющийся следствием 1 закона Socia definicia "--- <<право не принадлежать к обществу стоит выше законов общества>>.
Общий смысл: вышедший из общества не несёт перед обществом обязательств.
В период разработки универсального морального кодекса использовался народом тси;
однако, в отличие от прочих народов, тси распространили закон и на семью, что фактически привело к уничтожению семьи как института. \authornote}.
Если ты добровольно уйдёшь навсегда "--- никто не станет тебе мстить.

"--*А если\ldotst

"--*Смерть, "--- бросила Кхотлам.
Слово рассекло воздух ударом сабли.
"--- Прочее "--- смерть.

Эйраки рассмеялся.
На его лице медленно, но неотвратимо проступала звериная ярость.

"--*Передай своим, Испачканное-Мёдом-Перо, "--- прорычал Эйраки наконец.
"--- Я сотру в пыль ваши города.
Я превращу в пепел ваши книги.
Я уничтожу саму память о планете Тси-Ди и микробах, которые возомнили себя её хозяевами.

Кормилица спокойно хмыкнула и развернула оленя.

"--*Пойдём, Лисёнок, "--- бросила она мне.
"--- Больше нам говорить не о чем.

\section{[-] Закрытая дверь}

\textbf{<Люди поют колыбельную, идя на бой>}

Даже их религия была направлена на то, чтобы успокоить человека, а не напугать его.
Я вспомнил веру, в которой когда-то давно на Драконьей Пустоши воспитывала меня мать: страх перед Великим и Бесконечным, строгое исполнение предписаний и безусловное раболепие\ldotst

<<Кем бы ты ни был, лесные духи встретят тебя, успокоят твою боль и проводят в последнее пристанище.
И любящие воссоединятся, больные будут исцелены, помощники, защитники и трудяги смогут незримо служить живым, а уставших ждёт сладкий сон под вечные песни Печального Митра>>.

Я оглядел стоящую передо мной армию.
Пылерои, люди и идолы, попадались даже травники.
Мужчины, женщины с детьми, старики, даже подростки, способные держать оружие.
На их лицах было написано одно "--- они знают, против кого сражаются, и пришли не за победой.
В моей груди немедленно выросло сильное и нежное чувство, с ароматом то ли материнских, то ли дружеских, то ли любовных объятий, и страх перед небытием, который сидел во мне все эти жизни, ушёл.

Ушёл навсегда, почтительно поклонившись и закрыв за собой дверь.

\chapter{[-] Могильный берег}

\section{[@] Последний полёт дракона}

Взлёт прошёл без особых происшествий.
Фонтанчик аккуратно повёл корабль, стараясь не перегружать обшивку и двигатель.
Мы видели, как товарищи очертили круг повышенной радиации и выпустили сборщик радионуклидной пыли.

Полёт проходил в молчании.
Фонтанчик и я прислушивались к системам корабля, готовые при любом подозрительном параметре выжать полную скорость.
Время от времени Фонтанчик многозначительно смотрел на меня.
Да, мы понимали, что значит для выживших тси потеря вначале кольцевой теплицы, а теперь и Стального Дракона.
Половина инфраструктуры была связана с ним, а без топлива мы можем распрощаться с надеждой развернуть последний узел планетарной защиты.
Без замыкания пространства узлы превратятся в красивые, но бесполезные золотые здания.

"--*Это не конец, Небо, "--- подал голос Фонтанчик.
"--- Ничего ещё не потеряно.

Я кивнул, понимая, что друг сам не верит в истинность своих слов.

"--*Может, мы ещё успеем попросить Безымянного? "--- пробормотал я.

Фонтанчик с горькой улыбкой ткнул в мониторы.

"--*Здесь всё.
Обменная система тоже.
На дне колокола показатели нормальные, но нюхом чую "--- трещина заползла уже и туда.
Поздно, Небо.
Сейчас я бы принял помощь даже максима Картеля, но время мы упустили.

Пятьдесят тысяч, сто тысяч километров\ldotst
Планета Трёх Материков казалась нам небольшим голубовато-бело-жёлтым кругом.
Радиационный фон на нижних уровнях достиг критической точки.

"--*Небо, запускай робота.

"--*А если он\ldotst

"--*Мы успеем, "--- успокоил меня Фонтанчик.
"--- Всего-то до шлюпки добежать.

Я тронул имплант, и мы вместе прыгнули в гравитационный лифт.
Почти сразу корабль ощутимо тряхнуло.

Я посмотрел в иллюминатор, медленно осознавая случившееся.
Да, роботы были перепрограммированы Машиной.
Включенное мной устройство просканировало системы корабля и, поняв, что он находится в аварийном состоянии и готов взорваться, приняло решение "--- вернуться к начальной точке полёта.
К нашему поселению.

Одновременно с этим взвыла сирена "--- машина взяла на себя управление реактором.
Цепная реакция медленно выходила из-под контроля.

Фонтанчик тоже понял.

"--*Мою бабушку, "--- выдохнул он.

Я едва успел навести несколько внутрисетевых экранов и обесточить два отсека, но было поздно.
Почти сразу включились десять перепрограммированных роботов, в том числе самый опасный "--- робот-техник.

"--*Небо, быстро в шлюпку.

"--*Я\ldotst

"--*Небо, "--- Фонтанчик повернулся ко мне, и я увидел в его глазах тот неповторимый блеск, который появляется у существа, полностью осознавшего свой выбор.
"--- Не спорь.
Да и дуэлянт я куда лучший, чем ты.

Дуэлянт.
Забавное слово.
Его истинное, древнее значение было забыто, и <<дуэлями>> назывались соревнования техников.
Двое техников по команде начинали попытки взломать имплантаты друг друга и проникнуть в кору, вызывая определённое ощущение "--- обычно это была вспышка света или светящийся кружок.
Таким образом отрабатывались навыки быстрого взаимодействия техников друг с другом и с аппаратурой.

Здесь же дуэль была нешуточной "--- проигравшего ждало отключение.

Фонтанчик пошёл по направлению к лифту с непередаваемой грацией "--- красивый, статный, высокий пёс.
Одновременно с этим каналы связи едва ли не раскалились, как золотая проволока.
Мозг сапиента вступил в смертельную виртуальную схватку с пятнадцатью механизмами, которые все вместе превосходили его по мощности в три раза.

Корабль тем временем, неовратимо набирая скорость, приближался к оставленной нами планете.

Вдруг корабль снова тряхнуло, и меня ослепила вспышка гамма-излучения.
В горле пересохло от страха, перед глазами промелькнула страшная картина из Палат Войны "--- существа, умирающие от лучевой болезни.
Я бросился к Фонтанчику:

"--*Уходим!

Бесполезно.
Глаза друга остекленели, его разум полностью слился с системой.
Из уголка рта капала слюна, зрачки то увеличивались, то уменьшались, рефлекторно пытаясь помочь фехтовальщику, мечущемуся в бешеном виртуальном танце смерти.
В другом конце коридора заскрежетал металл "--- маленький робот-техник медленно, крадучись, приближался к живой добыче.

Мозг Фонтанчика тебе не по зубам, и ты решил нанести удар по беспомощному телу?
Ну уж нет.

Я привёл все инструменты мультитула в готовность.
Сейчас всё решит один-единственный удар "--- мой или робота.
Я попытался абстрагироваться от ужасающего гамма-излучения и сосредоточиться на других чувствах.

Шесть лапок.
Они тихо ступали по металлической поверхности пола.
Робот знал о резонанс-взрывателе и плазменном резаке в моём мультитуле.
Он тоже следил за мной всеми датчиками, которые у него были.
У него шесть лапок.
Как и у меня.

Гулкий, упругий удар.
И ещё раз.
И ещё.
Я слышал своё сердце так ясно, как никогда в жизни.
Я ещё живой.
Я ещё могу сражаться.

"--*Жизнь и процветание! "--- боевой клич Тараканов вырвался непроизвольно.
Теперь я вспомнил всё до мельчайших деталей.
Именно эту фразу я слышал, не имея ушей, и кричал, не имея рта.
Эти, именно эти слова выкрикивали мои сородичи, полуголые, грязные и голодные.
Им вторили голоса таких же истощённых кани, плантов и людей.
Крик разносился во выжженным радиоактивным пустошам, некогда буйствовавшим великолепием низких вечнозелёных кустарников.
Словно все эти тысячелетия он, записанный в самых древних моих генах, только и ждал подходящего момента.

Сейчас планета Ди жива и цветёт.

Мы с роботом бросились в атаку одновременно.
Взрыв "--- и шесть металлических лапок разлетелись по коридору.
Следом за этим меня накрыла ослепляющая головная боль.

"--*Небо, какого ты ещё здесь? "--- слышал я в полузабытьи голос друга.
"--- Поганый дым, ничего не вижу\ldotst
Тебя чем так оглушило?
Техник отключился, наверное, батареи\ldotst
Давай, приходи в себя, с этой мелочью я разберусь!

Я почувствовал, как сильные руки друга нежно подняли меня и положили в мягкое кресло.

\ldotst Пришёл в себя я уже далеко от Стального Дракона.
Ужасное, непередаваемое ощущение гамма-излучения медленно угасало.
Задыхаясь от боли, я ощупал голову.
Усика и кожи вокруг него не было "--- луч резака оставил лишь островок запёкшейся крови.

Тоскливо тянулись минуты.
Но вот я получил запрос на связь.

\mulang{$0$}
{"--*Небо, это я.}
{``Sky, it's me.}
\mulang{$0$}
{Я победил.}
{I won.''}

У меня отлегло от сердца.

\mulang{$0$}
{"--*Молодец.}
{``Well done.}
\mulang{$0$}
{Вылетай немедленно.}
{Leave immediately.''}

\mulang{$0$}
{"--*Поздно.}
{``Too late.''}

Фонтанчик включил видеосвязь, и я увидел друга.
Из его носа и дёсен сочилась кровь, его улыбка напоминала оскал боли.
Высохшие глаза растрескались и подёрнулись мутной слепой плёнкой.

"--*Триста грэй, Небо.
\mulang{$0$}
{Я буду вести корабль, пока смогу.}
{I'll drive while I can.}
\mulang{$0$}
{Ты "--- лучший командир, и я счастлив, что всю жизнь дружу с таким выдающимся тси.}
{You're the best cap, and I'm happy being friends with such a outstanding Qi.''}

Я кивнул, чувствуя, как сжимается сердце.

"--*Ты самый лучший из тех, кого я знаю, "--- сказал я.
\mulang{$0$}
{"--- Я люблю тебя.}
{``I love you.''}

\mulang{$0$}
{"--*Я тебя тоже, Существует-Хорошее-Небо.}
{``I love you too, Existing-Good-Sky.}
\mulang{$0$}
{Попрощайся со всеми за меня.}
{Give my fond farewell to everybody.}
\mulang{$0$}
{А Заяц передай: <<Как бы ни горело моё тело, любовь к тебе всегда на градус жарче>>.}
{Also tell Hare this: \emph{`However much my body burns, my love to you is always one degree hotter.'}''}

Фонтанчик светло улыбнулся в пространство, вытер левой рукой кровь, невзначай активировав <<Живую сталь>>.
Затем изображение исчезло "--- друг не хотел, чтобы я видел его лицо, искажённое препаратами в пустую, маскообразную гримасу.

Вскоре недалеко от меня вспыхнула новая звезда с моим другом в самом её центре.

\section{[-] Битва}

\spacing

Р-гранаты.
Картель перешёл к высокотехнологичному оружию.
Почти бесшумная вспышка "--- и несколько десятков бойцов просто испарились.
Грейсвольд едва слышно ахнул и начал отлавливать и обезвреживать боеприпасы.

Недалеко от нас неистовствовала чья-то флейта.
Неизвестный менестрель наигрывал бодрый, согревающий кровь марш.
Бойцы вокруг него <<светились>> мощно и ровно, мешая вражеским демонам распознать музыканта.
Время от времени кто-то из агентов Картеля делал рывок в его сторону, расчищая себе дорогу фалангой, но защитники тут же наваливались на него вдесятером.
Некоторые безропотно, не прекращая <<светиться>>, принимали своим телом первый тяжёлый удар.

Вот я почувствовал на языке страшную горечь МПДЛ\footnote
{Моронгозинпентозодифосфолизергинат "--- психоактивное вещество с ярко выраженным аффективным действием.
Использовалось ещё во времена Древней Земли нечистыми на руку политиками "--- предположительно, писательница Мариам Кивихеулу, один из идеологов неоэквализма, была растерзана толпой под действием распылённого в воздухе МПДЛ. \authornote}
"--- последняя отчаянная попытка Эйраки вернуть восставший скот под привычный хлыст.
Митхэ потратила весь воздух в лёгких на сигнал <<химическая атака>> и бросилась ничком на землю.
Тси тут же сбились в ватаги, прикрылись щитами, начали бледнеть и заваливаться.
Многие успели оглушить или обездвижить находящихся рядом людей Тра-Ренкхаля, мешая тем бесноваться в радужном безумии.
Агенты Картеля, которые заранее приняли антидот, удесятерили атаки, но всё было бесполезно.
Спустя половину михнет моя армия вырвалась из оков <<Кристалла>> и снова бросилась в бой, потеряв едва ли тысячу.

<<Ребята, готовьтесь! "--- крикнула Анкарьяль.
"--- Эйраки понял, что дело дрянь, и разрешил применить нуклеарное оружие!
База\ldotst Тридцать четыре секунды\ldotse Лишь бы\ldotst Дьявол! Ааа\ldotst>>

Остаток сообщения потонул в шуме ПКВ.
Я почувствовал, как Анкарьяль приняла на себя один, второй, третий удар\ldotst и тут всё стихло.
Над восточным горизонтом показался лучистый <<хвост>> ракеты-стрелы.
Она летела совершенно бесшумно, опережая звук в два раза.
Вокруг неё искрились расщепляемые в полёте атомы азота.

"--*Вот чудо, "--- ахнула прямо рядом со мной старушка, воткнув окровавленную фалангу в песок.

"--*Это Удивлённый Лю! "--- заорал кто-то.
"--- Удивлённый Лю вернулся!
Удивлённый Лю летит к нам на помощь!

"--*Мы обретём крылья!

"--*Мы скоро обретём крылья!

Битва приостановилась, все как зачарованные смотрели на летящую к ним смерть.
Они лучились восхищением и надеждой.
На агентов Картеля, которые замерли или попадали на землю, никто не обращал внимания.

Вдруг неведомая флейта замолкла.
Ракета, сделав странный, ломаный вираж, улетела в зенит, и надо мной показался огромный источник <<света>>.
Моё тело инстинктивно успело поднять руки в жалкой попытке защититься.

<<Эйраки Мороз, ты заплатишь.
За каждое моё существо, за каждое мгновение боли>>, "--- с этими словами на меня обрушился обжигающий поток.

Эйраки взревел и испустил волну <<антисвета>>.
Это не было дуэлью хоргетов, это было состязание грубой силы "--- два невероятно мощных хоргета били друг друга чистым неструктурированным потоком масс-энергии.
Попавшие под горячую руку агенты Картеля отчаянно кричали, распадаясь на составляющие.
Кое-кто из сапиентов вздрогнул и посмотрел в сторону битвы, которую не мог видеть;
прочие как зачарованные продолжали смотреть на фиолетовый хвост ракеты\ldotst

<<Аркадиу! "--- донёсся до меня слабый крик Грейсвольда.
"--- Прикрывай Нар, прикрывай!
Я уже, я почти\ldotse>>

Я из последних сил наводил защитные барьеры над подругой.
Моя энергия таяла, словно снежинка под лучами солнца.
Семьдесят процентов\ldotst сорок\ldotst десять\ldotst

<<Отлетался.
Прости-прощай, жестокий и прекрасный мир>>.

\section{[@] Одичание неизбежно}

Шлюпка вошла в плотные слои атмосферы и выпустила тонкие крылья.
Крылья управлялись аналоговым механизмом "--- разработка предназначалась для самых разных ситуаций, в том числе для трудностей с энергоносителями.
Подо мной расстилался океан, где-то вдалеке виднелся узкой полоской пролив Скарр.
Я аккуратно вёл шлюпку, стараясь направить в сторону нашего поселения.
Но расчёты показали, что до суши я не дотяну.

Как-то странно закружилась голова.
На мгновение я потерял представление не только о верхе и низе, но и о направлении стрелы времени.
Стены начали ощутимо вибрировать и нагреваться.
Я убрал крылья на треть длины, опасаясь сломать их, затем увеличил тангаж, едва не завалил судно и тем самым потерял ещё несколько драгоценных километров.

Океан.
Да, похоже, что моя жизнь закончится именно здесь.
Здесь, в тёплой экваториальной эпипелагиали, водились хищные рыбы, против которых я не имел ничего, кроме четырёх слабых рук.

Четырёх?
Кажется, что-то не так.
Впрочем, неважно\ldotst

Боль утихла, и мои мысли обратились к чему-то далёкому, безбрежно-спокойному, обманчиво спокойному, как этот океан.
Жизнь.
Как же она удивительна.
Сколько поколений сменилось, пока появился я "--- и вот пришло время уйти и мне, чтобы дать дорогу другим.

В такие моменты становишься честен с собой.
Нет, мы не успеем построить систему планетарной защиты.
Там, на Стальном Драконе, только что превратился в пар один из главных её архитекторов вместе с базами данных, половиной аппаратуры и энергоносителем для ядерного синтеза.
Скоро умру и я.
Кто знает, сколько отмерено остальным?
Каждый из нас "--- носитель бесценной информации и опыта.
Увы, носитель непростительно хрупкий.

И вместе с этими мыслями пришла ещё одна, которую я до поры до времени отбрасывал.
Одичание тси неизбежно.
Нас слишком мало, и наши знания ничего не значат без инфраструктуры.
Сколько нужно времени, чтобы её восстановить?
Десять, пятнадцать поколений?
За это время большая часть знаний будет безнадёжно утеряна.

Значит, нужно обеспечить тси одичание, вдруг сказал я себе.
Эта мысль, кощунственная по своей сути, даже не вызвала у меня отторжения.
Да, одичание неизбежно.
И значит, что его нужно подготовить максимально качественно.

Вслед за этими словами по моей спине пробежал холодок, а затем сердце сжалось так, что я чуть не застонал.
Эту мысль необходимо передать остальным, жизненно необходимо.
Я не могу умереть, пока не буду убеждён в том, что другие найдут моё сообщение.

Привычным движением я вскинул мультитул.
Но устройство отозвалось молчанием.
Микросхемы приняли на себя всю мощь косого испаряющего луча, сохранив мне голову и жизнь.

Голова снова закружилась со скоростью электрона.
Нет, это не голова, это кружится неприветливый мир, выбивая у меня из-под рук панель\ldotst
Я снова и снова отчаянно пытался поцарапать краем мультитула стабитаниум и только спустя минуту осознал всю бесполезность этой затеи.
У меня промелькнула мысль записать сообщение на собственной коже или костях, но потом я вспомнил про хищных рыб.
Вдруг взгляд упал на привинченную к креслу табличку из мягкого алюминия.
Она находилась прямо на стыке, и увидеть её можно было, лишь приведя кресло в рабочее положение.

<<Кому пришло в голову сделать алюминиевую табличку?>> "--- удивился я и, на секунду забыв о проблемах, пригляделся к написанному.

\begin{quote}
\mulang{$0$}
{<<Искренне надеемся, что это никогда не прочитают, но желаем тебе удачи и мягкой посадки.}
{``We sincerely hope it will never be read, but we wish you good luck and soft landing.}
\mulang{$0$}
{Держи штурвал крепко, лови попутный атмосферный поток и держись его.}
{Keep a good grip on the wheel, catch a fair atmospheric current and follow it.}
\mulang{$0$}
{Если же ты ранен или ослаб "--- расслабься и займись собой, мы сделали всё, чтобы ты пережил посадку.}
{If you're wounded or weak, just relax and take care to yourself, we've done our best to make you survive the landing.}
\mulang{$0$}
{Под сиденьем сокровища, чтобы ты не грустил.}
{Under your seat there are some treasures to cheer you up.}
\mulang{$0$}
{Руки-крюки из Четырнадцатого>>.}
{Butterfingers from the Fourteen.''}
\end{quote}

Я заглянул под сиденье, и на душе потеплело.
<<Восток-запад>>, или, как изысканно выразилась один раз Кошка, <<пасхальное яйцо>>.
Крохотный пенал, заполненный мелкими вещицами.
Какая-то труха, похожая на давно рассыпавшиеся цветы, серёжка c титановым анодированием, окаменевшая конфета в выцветшем пластиковом фантике, прядь красивых чёрных волос в жаропрочной смоле, <<вечное>> перо для граффити\ldotst

Снова закружилась голова.
Конфетка.
Как же давно я их не ел.
Я развернул фантик и отправил конфету в рот.
Прошла целая вечность, прежде чем я выплюнул на ладонь безвкусный кусок жаропрочной смолы.
Нетронутая конфета так и лежала рядом\ldotst

Перо!

Кристалла осталось совсем чуть-чуть, но оно ещё действовало.
Я вывел на панели пять огромных иероглифов:

\begin{quote}
<<Одичание неизбежно.
Нуждается в подготовке.
Небо>>
\end{quote}

Я приоткрыл верх капсулы и вдохнул свежий морской воздух.
Какая чудесная погода.
Я и забыл, что на этой части планеты сейчас весна.
Мимо проплыло низкое облако, громыхая и ворча небольшими электрическими разрядами, словно обижаясь на белых кучерявых красавцев выше.
Прямо на облаке сиял радужный <<глаз>> с тенью моей шлюпки в зрачке.
Я немного отстранённо думал, сколько мне осталось ещё лететь.
Голова, нет, только не сейчас\ldotst

Рулевой механизм повело в сторону, тонкое крыло взвыло от боли, испуганно запищали приборы.
Шлюпка нырнула вниз, словно подбитая чайка.
Но я этого уже не видел.

\section{[-] Цена свободы}

<<Аркадиу!
Загружайся!>>

Я ощутил пространство и время.
Следующим пришло осознание произошедшего.
Тело откликнулось последним.
Я открыл глаза.
Надо мной сидел улыбающийся во все зубы Грейс.

"--*Доброе утро, человеческое отродье.

Я промолчал.
Время было похоже на дорогу с колдобинами "--- иногда я проваливался в ямы, иногда взлетал на кочках.
Болел затылок, которым я ударился о присыпанный песком валун.
К горлу подкатывала тошнота.
И тут я вспомнил\ldotst

"--*Нар!

"--*Лежи тихо, она жива.

"--*Грейс, что произошло?

Технолог продолжал скалиться.

"--*Если я тебе расскажу, то ты признаешь меня вселенским гением.
Хотя так оно и есть.
Пока вы с Анкарьяль занимались Эйраки, я запустил Золотой город.
Это всего лишь один узел планетарной защиты тси, но мощности в нём скрывались приличные.
Я на несколько микросекунд создал большого, жирного хоргета.
Он был очень некрасивым, но вот вычислительные мощности его превосходили мощности Эйраки как минимум в миллион раз.
Простенькая программа "--- защита Эйраки кракнула, как ореховая скорлупа, и старик превратился в младенца.
Я обрадовался, да только преждевременно.
Нар упала, ты её прикрываешь, а массивная минус-сингулярность начала трещать по швам "--- видимо, я задел модуль стабилизации.
Всё, думаю, отлетался.
За микросекунды пришлось вспомнить почти всё, что я когда-либо знал\ldotst
Я поменял полярность сингулярности, флип-флоп.
Преобразование Шмидта, одно из четырёх фундаментальных омега-преобразований, незаслуженно забытое.
В итоге агенты Картеля, которые открылись и уже приготовились пополнить энергию, приняли расслабляющий плюс-душик и радостно аннигилировали.
Планета наша.

"--*Так значит, всё хорошо?

Грейс нахмурился.

"--*Аркадиу, у тебя мозг повреждён?
Я об этом и рассказываю!
Если для тебя важно, то Чханэ и Плачущий Ягуар "--- дарительница твоего тела "--- тоже выжили.
Непонятно как.

И тут до меня дошло.
<<Взгляд>> хоргета почувствовал исходящее от Грейса яркое сияние.
А ещё\ldotst

"--*Ты тоже ничего так светишься, "--- ухмыльнулся Грейсвольд.
"--- Твой демон в фоновом режиме напитался.

"--*А Анкарьяль?

"--*Ну\ldotst нет, "--- смутился Грейс и, взяв мою руку, положил её на чьё-то тело, лежащее за его спиной.
"--- Она чересчур потратилась.
Думала, видишь ли, что погибнет, и сражалась как в последний раз.
Демон отключён от тела, амнезия небольшая будет\ldotst хаяй\ldotst я её подпитываю пока что, восстанавливается.

Я обратил <<взгляд>> на лежащую рядом Анкарьяль.
Её демон светился слабо, словно крохотная свеча.
Время от времени исходящий от неё свет трепыхался, словно птенец, в тяжёлом сне вынужденной экономии энергии.

"--*Она молодец, "--- одобрительно проворчал Грейс.
"--- Ты почувствовал, как она ударила в первый раз?
Эйраки не на шутку растерялся.
Вряд ли он когда-либо встречался с таким противником.
А кое-кто из Картеля смылся сразу, не дожидаясь продолжения.

"--*Они доложат о случившемся, "--- сказал я.

"--*Я тоже доложил, "--- успокоил меня Грейс.
"--- Уже успел слетать до Капитула и обратно.
Околопланетное пространство заполнено нашими агентами, на поверхности уже разворачивают микростанции.
У нас с тобой небольшая передышка.

Я подполз поближе к подруге.
Глаза Анкарьяль блуждали, как у младенца, на лице замерло отсутствующее выражение.
Я погладил подругу по испачканной мокрым песком голове и поцеловал её в лоб, надеясь, что она это почувствует.
Демон-птенец, жадно ловя исходящие от моего тела слабые эманации любви и сострадания, трепыхнулся и снова свернулся в маленький тёплый комочек.

"--*Эйраки напоследок сообщение оставил, "--- отстранённо, словно не мне, пробормотал Грейс.

"--*Какое?

"--*<<So’ana fre, ‘et>>\footnote
{Познай мой страх, предатель (сохтид). \authornote}.

Я откинулся на спину.

"--*Кто это был?
Кто отвёл атомную бомбу и атаковал Эйраки?

"--*Безымянный, кто ж ещё, "--- грустно сказал технолог.
"--- Видимо, он погиб "--- больше плюс-искажения и осцилляций ПКВ я не чувствую.

"--*Сколько сапиентов последовало за своим богом? "--- спросил я, ощущая собственное нежелание слышать ответ.

"--*Больше половины "--- почти двести тысяч, "--- откликнулся Грейс.
"--- Радужное безумие, затем Картель устроил резню.
И ещё\ldotst

"--*Что?

"--*Убитые "--- в основном народ сели и те, высшие пылерои, которые не боялись и сражались, пока могли держать оружие.
Они сбились вокруг нас троих, закрывая собой и запутывая интерфекторов.
Я даже не подозревал, что сапиенты с палками в руках способны настолько эффективно противостоять демонам.
Падали почти сразу "--- на их место вставали другие.
Армия этих странных коротышек, трами, полегла полностью, поголовно, некому даже вернуться домой и сообщить о том, что произошло.
Я еле-еле вытащил вас из-под горы тел.
Если бы не они, мы бы с тобой не разговаривали\ldotst
А вот хака и тенку воины Картеля просто испугали и оставили в живых.

Где-то невдалеке плескалось море.
Солнце медленно клонилось к закату.
Из-за низкой песчаной дюны послышалась плачущая песня сели.
Я долго молчал, осознавая простой жестокий факт: здесь, на берегу Могильного пролива, был затоптан последний цвет моего рода, поднявшийся в едином порыве непонятно ради чего.

\section{[-] Воинская доля}

Я брёл через усеянный телами берег.
Голова гудела, но я не имел ни малейшего желания исправить это.
На меня то и дело оглядывались сидящие на корточках люди, кани, планты.
Их глаза были чисты, как небо.
Сели говорили: <<Раз тебе идти по жилам джунглей в одиночестве, то мы соберёмся вместе на твоих поминках>>.
Однако сегодня всё было иначе: мёртвые шли толпами, живые сидели по одному.

"--*Мы победили? "--- недоверчиво спросил кто-то в темноте.

"--*Да, "--- ответили ему.

"--*Это не похоже на победу! "--- выкрикнул человек.
"--- Посмотри вокруг!
Какая победа?
С чего ты взял, что мы победили?

"--*Мы ещё живы, дружище, "--- ответили ему сразу несколько товарищей.
"--- Мы ещё живы!

"--*Ликхмас! "--- из темноты выскочила Митхэ ар'Кахр, бросилась мне на шею и тут же отстранилась.
Её глаза горели спокойным огнём воина.
"--- Король-жрец, у нас мало пищи и пресной воды.
Какие будут распоряжения?

"--*Поговори с хака и ноа, "--- сказал я.
"--- Потом возьми добровольцев и отправляйся за провизией и водой как можно быстрее.

"--*Ты как? "--- тихо спросила родильница.
"--- Кстати, вон твоя женщина сидит.
Я видела её в отражении ножа\ldotst

Мы с Чханэ посмотрели друг на друга одновременно.
Подруга улыбнулась и показала мне спящего ребёнка.
Большего не требовалось.

"--*Чханэ? "--- из темноты показался пожилой хромающий мужчина.
Женщина застыла.

"--*Ты? "--- услышал я её сдавленный шёпот.

"--*Дитя! "--- заплакал мужчина и, оступившись, упал прямо на Чханэ.
Звуки борьбы и недовольное гуление малыша сменились поцелуями.
"--- Ты живая, слава духам!
Я всех духов закормил, только чтобы\ldotst

Мужчина, всхлипывая, что-то затараторил с сочным западным акцентом.

"--*Почему у тебя воинская стрижка?
Я же попросил тебя держаться подальше от Храма!
Ты живая, ты правда живая?
Хай\ldotst
А это чей ребёнок?
Змейка, ты откуда малыша взяла\ldotsq

Чханэ, смеясь, рассказала кормильцу, откуда появляются дети.
Дослушивать я не стал и отправился дальше.
Митхэ, словно тень, шла рядом.

Вот Кхохо.
Она бросила в пылу боя свою серебряную саблю и била врагов двумя обломками корабельной реи, наспех заострёнными ножом.
Чей-то клинок ударил её прямо в загривок, почти отделив голову от тела.
На лице воительницы замерло выражение брезгливого удивления: <<Это ещё что за дрянь?>>.
Рядом с ней никто не стоял.

В двадцати шагах лежал Ситрис.
Родильница подбежала к нему.

"--*Привет, Митхэ, "--- вяло помахал ей воин.
"--- Рад, что ты живая.

"--*Здравствуй, дружище, "--- улыбнулась Митхэ, деловито осмотрела раны и показала мне скрещённые пальцы.

"--*Да знаю я, "--- проворчал Ситрис.
"--- Как будто первый раз в боях.
Знаешь, Митхэ, мне ужасно хочется жить.
Даже больше, чем тогда, когда мы с тобой бежали по старому тракту через Сикх'амисаэкикх.
Жизнь "--- это самое прекрасное, что со мной случилось.
Но что-то этот мир стал для меня чересчур сложным\ldotst

Ситрис обвёл ладонью вяло кровоточащие раны.
Я потянулся за чашей.

"--*Не-не, Ликхмас, забудь.
Ни секунды жизни не хочу тратить на такую ерунду.
Как Кхохо, не видел её?
Хе-хе, ты слушай.
Сабля расплавилась у неё в руках, честное слово.
Кхохо так ругалась, что вокруг неё враги сами дохли, она ж её любит больше жизни.
Потом подобрала какие-то палки и давай их посреди боя строгать.
Я её пытаюсь прикрыть и ору, ты дура, приоритеты хоть перед смертью научись расставлять.
А она палки заострила и пошла, я тоже какую-то дубину схватил во вторую руку.
Мы так хорошо двигались <<танцем согхо>>, как по книжке, плечо к плечу.
А потом она пропала, и меня искололи всего.
Неужто старушку-веселушку всё-таки нашёл клинок?

"--*Умерла быстро, "--- кивнул я.
"--- Зарубили сзади.

"--*Сзади?

Ситрис хрипло рассмеялся, забрызгав кровью песок.

"--*Ну и дура, "--- тихо сказал он и умолк.

Родильница, вздохнув, достала хэситр и мех с водой.

"--*Рада, что ты с ним успел познакомиться, "--- сказала она.
"--- Он был разбойником, его объявили вне закона во всех городах Юга.
Когда мы проходили Кахрахан, он бросил свою шайку и пришёл ко мне на службу, несмотря на угрозу казни.
Очень добросердечный человек, не чета многим законопослушным лицемерам.

"--*Он как-то спас мне жизнь, "--- сказал я.

"--*Правда?
А я ругала его за то, что он пообещал искать Атриса, но сбежал в Тхитроне, "--- покачала головой Митхэ.
"--- Всё-таки помощь возвращается, хоть и в неожиданном обличье.
Иди, Король-жрец.
Я почту его память и отправлюсь выполнять приказ.

\section{[-] Цвет Земли}

Кхотлам сидела и отстранённо чертила щепкой фигуры на песке.
Услышав мои шаги, она посмотрела на меня.
Сквозь меня.

"--*Лисёнок\ldotst
Дитя\ldotst
Мы победили?

"--*Да, "--- только и смог выдавить я.

"--*Хорошо, "--- улыбнулась она, бросила щепку и погладила молодого, мёртвого как камень Хитрама по чёрным волосам.
"--- Видишь, Пловец, не зря ты отплавался.
Только лучше уж я бы с тобой.

Откуда-то из ткани сумерек возникла мрачная однорукая фигура.

"--*Кхотлам.

"--*И тебе не хворать, Акхсар ар’Лотр, "--- не оглядываясь, ответила кормилица.

"--*Когда-то ты хотела быть со мной, но я был чересчур влюблён в дорожную пыль.
Сейчас я один, без очага и крыши, "--- без обиняков сказал старый воин и положил здоровую руку матери на плечо.
"--- И знаешь, все эти годы я почему-то вспоминал, как уютно у тебя в жилище.
Примешь ли ты меня?
Могу ли я стать твоим мужчиной и дожить эту чересчур долгую жизнь с тобой?

"--*Хитрам очень тебя любит, "--- сказал я кормилице.
"--- Он бы хотел этого.
Это не предательство.

Кхотлам молчала.
И вдруг её глаза на секхар снова засияли.
Она прикрыла их, словно стесняясь этого блеска, и ткнулась головой в шею Акхсара:

"--*Помоги мне его похоронить.
Если ему не хватит сил достигнуть пристанища "--- мне не видать ни покоя, ни жизни.

Акхсар сел, достал чашу и пузырёк с краской и, зажав чашу между коленей, стал аккуратно вырисовывать знаки.
Кхотлам встала и пошла к ближайшему костру, чтобы принести свет.

Я пошёл дальше.
Но что-то было не так, как прежде.
Люди по-прежнему бродили среди тел, выискивая живых, или сидели над умершими.
Но было и то, чего я не заметил.

Вот плакал осиротевший ребёнок.
К нему подошла женщина и, оглянувшись, подхватила его на руки.

"--*Кормилицааа\ldotst

"--*Тихо, тихо.
Я кормилица.
Я пришла.

Молодая женщина утешала рыдавшего над телами парня:

"--*Тише, тише, только не кашляй, сердце не выдержит.
Расслабься.
Я буду твоей женщиной до самой смерти.
Слышишь, Манис?
Я всегда буду с тобой.
У нас будет много детей.

Я понаблюдал за ними.
Вскоре девушка подняла парня, и они подошли к потерянно сидящему на песке старику, который, опустив голову, ковырял тростью песок:

"--*Лехэ, кормильцем нам будешь?

Это происходило везде.
На время забыв о мёртвых, люди и пылерои, идолы и травники заботились о живых.
Вот старик-человек робко положил руку на мускулистое плечо понуро сидящего пылероя.
Тот дёрнулся, но не отодвинулся.
Вот травник вел под руку раненную женщину.
Всюду начали разгораться костры.
Идолы и люди, сели и хака, прежде заклятые, непримиримые враги, сидели рука в руке, не говоря ни слова.
Кто-то тихо напевал.

Тогда я и начал понимать, почему тси называли цветом Ветвей Земли.
То, что я увидел в тот день, я не видел больше нигде во Вселенной.

\section{[@] Медицина бессильна}

\spacing

Костёр покрутил в руке курительное приспособление, подаренное ему вождём царрокх.

"--*Извини, Небо, я бессилен, "--- сказал наконец врач.
"--- С кольцевой теплицей мы бы выходили тебя, но увы.

"--*И сколько мне осталось?

Костёр пролистал мои данные и нахмурился.

"--*Тридцать часов.
У апид первичная реакция гораздо короче, она пройдёт через два-три часа, как раз можешь поспать.
Ещё около двадцати часов будет стадия мнимого благополучия, дальше умирание.
<<Фотон-8>>\footnote
{<<Фотон>> "--- поддерживающий комплекс химических агентов и клеточных культур при острой лучевой болезни.
Восьмой номер "--- для апид-тси.
Состав средства неизвестен. \authornote}
в лучшем случае растянет это время на сутки-двое.
У меня осталось две ампулы.

"--*В таком случае не трать их, "--- сказал я.
"--- Не хочу менять чьё-то здоровье на двое суток.

"--*Небо, может, ты хотя бы перед смертью подумаешь о себе?

"--*Я бы и рад, но о себе думать поздновато, "--- буркнул я.
"--- Выключай.

Костёр кивнул, выключил инфузию и вышел.
Спустя несколько секунд меня охватил жар.

\section{[@] Раскол}

Врач вернулся спустя десять минут.

"--*Я всё-таки поставлю тебе некоторые препараты, "--- сказал Костёр извиняющимся тоном.
"--- Не дефицитные, не бойся.
Обычные жаропонижающие и вегетотоники, их как песка на пляже.

Я отмахнулся.

"--*Небо, я хочу беседовать с тобой как можно дольше.
Ты не поверишь, как я этого хочу.

Это были слова друга, а не врача.
Я не мог сопротивляться их искренности и подчинился.
Жар спал.
У меня на лбу выступил обильный холодный пот.

Костёр подумал и, поколебавшись, зарядил в плечевой имплант ещё одну капсулу "--- с мутной, очень тёмной жидкостью.

"--*Сок одного из местных растений, "--- объяснил он.
"--- Когда стадия мнимого благополучия закончится\ldotst если почувствуешь, что совсем невыносимо, переключи имплант на четвёртый канал.
Протокол <<Тайфун>> может отказать из-за разрушения клеточных структур, а это "--- быстрая и совершенно безболезненная смерть, гораздо лучше нашего препарата для эвтаназии.
По ощущениям будет похоже на засыпание.
Знаешь, как будто ночник гасишь в голове и ложишься в мягкую постельку.

"--*Хорошая вещь, "--- улыбнулся я.
"--- Биологи нашли?

"--*Сами удивились, "--- кивнул Костёр и ухмыльнулся в воротник.
"--- Не всё на этой планете жестоко к нам.
Так, кто там ломится?

"--*Небо пришёл в себя? Срочно нужно поговорить! "--- раздался приглушённый голос Баночки.

"--*Ему нужен покой! "--- громко сказал Костёр.

"--*Нет-нет, "--- я попытался сказать погромче, но голос сорвался.
"--- Пусть войдут, Костёр.
Если что-то важное\ldotst

"--*Сказать им\ldotsq

"--*Не сейчас.
Вначале дело, прочее потом.

Костёр махнул рукой. Баночка и Кошка осторожно подошли.

"--*Как ты?

"--*Лучше всех, "--- соврал я и неуклюже изобразил танец.
Культурологи заулыбались, Костёр отвернулся и сделал вид, что копается в ларце.
"--- Что у вас?

"--*Мы только с обсуждения твоего послания.

"--*Какого послания? "--- удивился я.

"--*<<Одичание неизбежно и нуждается в подготовке>>, "--- процитировал Баночка.

Я вдруг осознал, что не помню ничего подобного.

"--*У тебя амнезия, "--- сказал Костёр.
"--- Большая удача, что ты успел записать свою мысль.
Хотя можно было воспользоваться дневником в импланте, тебя всё равно никто не съел.

"--*Кто мог меня съесть? "--- удивился я.

"--*Акулы, "--- объяснил Костёр.
"--- Я просмотрел твой дневник.
Послё отлёта у тебя было сужение сознания и бред.
Повторяется одно и то же слово "--- голова, голова, голова\ldotst
Не знаю, почему ты решил, что тебя съедят "--- шлюпку можно разбить вдребезги разве что на первой космической скорости, и уж точно не о воду.

Одичание неизбежно.
Нужна подготовка.
Да, это до ужаса логичная мысль, и она вполне могла прийти мне в голову.

"--*Общество раскололось, "--- без обиняков сказала Кошка.
"--- Многие до сих пор не хотят верить, что мы проиграли.

"--*Чего они хотят? "--- осведомился я.

"--*Уйти на экваториальный щит и попробовать пробиться через кору планеты, "--- сказала Кошка.
"--- На поверхности катастрофически мало металла, но в мантии он должен быть.
Да, Небо, мы понимаем, что это значит.
Мы говорили о возможных последствиях, но они ничего не захотели слушать.

"--*Нас предостерегали от ухода цивилизации под землю! "--- выдохнул я.
"--- Неужели тси забыли это предостережение?
<<Титановые небеса "--- предвестник рабства>>\ldotst

"--*В отчаянии и горе нормально стать забывчивым, "--- рассудительно заметил Костёр.

"--*Мы с Кошкой считаем, что твой план неплох, "--- перешёл к делу Баночка.
"--- Он осмотрителен и выполним, хоть и не даст сиюминутных результатов.

Плант вытащил из сумки компьютер и включил его.

"--*Тут у меня\ldotst эээ\ldotst кое-какие мысли появились, "--- смущённо сказал он.
"--- Кошка тоже помогла.
Мы обдумали устройство общества царрокх и решили выкроить по этой мерке новую культуру для выживших тси.
Искусственную культуру, в которой можно воспитывать молодых тси в отдалённых поселениях.

"--*Вы собрались загнать тси в Древний мир? "--- с ужасом пробормотал я.
"--- Неужели я вложил именно такой смысл во фразу <<подготовка к одичанию>>?

"--*А у нас есть выбор, Небо? "--- грустно усмехнулась Кошка.
"--- Посмотри вокруг.
Мы на чужой дикой планете.
Вокруг одни голые камни, которых не касалась даже нога, не говоря уже об инструменте.
Инфраструктура нулевая.
Корабль был для нас всем, но мы лишились его.

"--*Раньше постройка планетарной защиты казалась само собой разумеющимся, "--- добавил Баночка.
"--- А теперь это выглядит, словно мы собрались возвести дворец с фонтанами в безводной пустыне.

Я промолчал.

"--*Так вот, "--- Баночка деланно бодро продолжил рассказ.
"--- Культура царрокх "--- это сплав мифологического мышления с примитивными технологиями.
Мы могли бы адаптировать её для тси.
С одной лишь разницей "--- мифология и технологии тси будут не остатками былого великолепия, брошенными на произвол судьбы, а продуманной системой, которая обеспечит возможно скорый технологический подъём в будущем\ldotst

"--*У нас есть информация об изменчивости окружающей среды, "--- перебила планта Кошка.
"--- Биологи, экологи и геологи проделали гигантскую работу.
На основе этого можно предсказать с неплохой точностью всё, включая изменения языка, пути миграции примитивных народов и особенности их культуры\ldotst

"--*Мы постараемся записать большую часть информации на долговечных и легко воспроизводимых носителях, чтобы, когда технологический и научный прогресс достигали определённой точки, новые открытия происходили проще и быстрее, "--- увлечённо затараторил Баночка.
"--- Даже песни и стихи, которые мы запустим в новую культуру, будут психологически подготавливать тси к прогрессу\ldotse

"--*Вам хватит знаний?

"--*Мы используем всё, что в нашем распоряжении, "--- заверила Кошка.
"--- Работа, разумеется, на годы и десятилетия.
Но, если честно, это первое по-настоящему серьёзное дело для меня.

"--*Как и для меня, "--- поддержал подругу Баночка.
"--- Я вообще не припомню, чтобы кто-то из знакомых брался за такой масштабный и важный научный проект, при этом больше похожий на художественное произведение.

"--*А если сюда придёт Ад или Картель? "--- спросил я.

Баночка и Кошка как воды в рот набрали.
Но им на помощь неожиданно пришёл Костёр.
Он хмыкнул и опустил голову.

"--*Я никогда не верил в точный расчёт, "--- признался он, повернувшись к культурологам.
"--- В медицине власть случайности гораздо ощутимее, чем в других областях "--- она может и спасти, а может и унести чью-то жизнь.
Вряд ли когда-то было иначе.
Многие из знакомых врачей уходили в подобие мистицизма "--- носили счастливые пробирки на цепочке, читали стишки перед сложными процедурами.
На эту тему даже исследования были.
Я, разумеется, так не делал, но тоже старался верить в лучшее.

Культурологи заулыбались.

"--*Ну, больше нам ничего и не остаётся, "--- развёл руками Баночка.
"--- Бессмертных среди нас нет.
Всесильных тоже.

"--*И всё же я внесу некоторые коррективы на случай вторжения хоргетов, "--- едва слышно пробормотала Кошка.
"--- Вряд ли это поможет, конечно\ldotst

Вдруг глаза Кошки загорелись.

"--*Вот я дура, "--- женщина хлопнула себя по лбу.
"--- У меня под рукой всё это время был хоргет-девиант, а я строила дома из змей\footnote
{Строить дома из змей "--- производить логические выкладки, основываясь на непроверенных данных. \authornote}\ldotst
Безымянный, я буду твоей верной жрицей до самой смерти, только дай тебя изучить!

С этими словами она схватила под мышку громко протестующего Баночку и выбежала из палатки.

"--*С ума посходили, "--- покачал головой Костёр.
"--- Но звучит интересно.

Врач неожиданно нежно погладил меня по голове и, бормоча что-то обыденно-успокоительное, занялся моей инфузионной системой.
На секунду я вдруг очутился дома, вне тревог, опасностей и забот.
Немедленно потянуло в сон.

Где-то вдали тихо запел Безымянный.
Интересно, откуда он знает <<Песню рассвета>>\ldotsq
Мак подхватил.
Как чудесно сочетаются их голоса.
Фонтанчик и Комар спорят, как лучше приготовить лилового краба.
Вереск сказала, что его обязательно нужно запечь в глине.
Что ж, у тси ещё остались кулинары со вкусом.
Мы точно не пропадём.

"--*Ведь правда же, кормилица\ldotsq

"--*Ясное дело.
Чистая правда.
Спи, малыш, "--- Костёр снова погладил меня по голове.
В его глазах тихо позванивали оплавленные осколки корабельного хрусталя.

\section{[-] Обычаи погребения}

\spacing

Воздух огласили далёкие воинственные крики.
Ноги перешли на бег прежде, чем я успел осмыслить, что происходит.

"--*Ни за что! "--- кричал кто-то.
"--- Не для того мы\ldotst

"--*Тихо! "--- крикнул я, вбежав в толпу людей.
Толпа замолчала.
Пара парней, смекнув, тут же подхватили меня и усадили на плечи.

"--*Это Король-жрец! "--- крикнул один из них.
"--- Пусть он говорит!

"--*Что происходит? "--- осведомился я.
Люди понурились.

"--*Пылерои, "--- один из мужчин кивнул в сторону.
"--- Они хотят наших мертвецов.

Я похлопал парней по головам, и они пронесли меня через толпу к мрачно стоящей группе пылероев.

"--*Что случилось? "--- спросил я на цатроне.
"--- Мои люди говорят, что вам нужны мертвецы.

"--*Меня зовут Рычащая-Звёздна-Бездна, "--- сказала седая волчица, статная и гордая.
"--- Я старейшина клана Лёгкого Пера.
У нас кончилась провизия, и нам неоткуда её взять.
Наши кланы были изгнаны, а стада забиты.
Если мы вернёмся в пустыню, то найдём смерть, а не пищу.

"--*Что с вашими мертвецами?

"--*Мы уже заготовили все тела, которые смогли, "--- ответила Бездна.
"--- Но этого мало.
Я знаю, что у вас тоже нет провизии.
Я предлагала разделить мясо, но люди не пожелали обсуждать это.

Люди молчали.
Все как один смотрели на меня.

"--*Как надлежит людям сели поступать со своими мертвецами, родичами и друзьями? "--- спросил я.

"--*Мы хороним их, а не делаем из них жаркое! "--- рявкнул молодой жрец.
Толпа согласно зашумела.
Я поднял руку, призывая к молчанию, и обратился к волчице.

"--*Как поступают пылерои со своими мертвецами, родичами и друзьями? "--- спросил я на цатроне.

"--*Мы хороним их, "--- сказала волчица.
"--- Съедаем мы лишь самых смелых и сильных врагов.
Но сейчас другие времена "--- мы хотим лишь выжить и набраться сил.

"--*Считаете ли вы врагами людей сели?

"--*Мы сражались с ними бок о бок, делили горечь утрат и радость победы, "--- сказала волчица.

"--*Окажете ли вы уважение тем, кого съедите?

"--*Мы почтим их, как своих родичей и друзей, "--- сказала волчица.

Я повернулся к сели и перевёл её слова.
Толпа неуверенно зашевелилась.

"--*Нужны ли нашим мертвецам их тела? "--- сказал я напоследок.
"--- Духи уже давно ушли искать пристанище, и многие из вас лично указали им путь.
Что ещё нужно тем, кто пал в битве?

"--*Но они надругаются над телами! "--- крикнул кто-то.

"--*Им окажут уважение! "--- сказал я.
"--- Вождь пылероев сказала, что съедают они лишь самых сильных и храбрых.
Кто сомневается в том, что наши павшие были сильны и храбры?

Толпа забормотала.

"--*Может ли быть для павшего большая честь, нежели быть благословлённым старыми врагами?
Неужели среди сели не найдутся те, кто захочет помочь в голодный час товарищам по оружию и друзьям, пришедшим в час нужды?

"--*Я попрошу на коленях ради моего клана, "--- сказала волчица.

"--*Ещё чего! "--- махнул я рукой.
"--- Сели не унижают так даже врагов.

"--*Тогда чего они хотят?

"--*Подожди, "--- ответил я и громко добавил на сели:
"--- Предлагаю голосование.

"--*Не нужно, Король-жрец, "--- сказал подросток, стоявший у самого края.
Он вышел из толпы и подошёл к Рычащей-Звёздной-Бездне.

"--*Возьми моя кормилица, вождь, "--- сказал он на ломаном цатроне.
"--- Я есть очень близко человек к ней.
Она кормить меня, и я мочь сделать она тебе, чтобы ты тоже есть.

Волчица поклонилась.
Подросток протянул ей руку, Бездна аккуратно взяла её, и вместе они ушли к телам.

Вскоре из толпы вышли ещё десять человек, знаками поманили пылероев и тоже ушли.

Я повернулся к мрачно ожидающим людям.

"--*Пылерои не возьмут ни одного тела, которое вы не захотите им отдать.
Решайте сами "--- павшие уже дома, а в ваших руках чья-то жизнь и смерть.

К вечеру пылероям отдали достаточно тел, чтобы они смогли пополнить провизию.
Несколько самых уважаемых жрецов пошли к пылероям, чтобы помочь с заготовкой мяса и проследить за исполнением обрядов.
Вернувшись, они сказали, что мёртвым не на что жаловаться "--- из их костей пылерои сделали несколько тотемов, украсив их лентами, снятыми с собственного тела украшениями и амулетами.
Многие унесли с собой черепа, пообещав установить их в своих святых местах.

Клан Лёгкого Пера выразил желание уйти с людьми к Кахрахану и жить на землях сели.
На собрании было принято решение отдать пылероям на кормление и охрану Вялую Степь, которая вполне годилась для скотоводства.

\section{[@] Два общества, два пути}

\spacing

Кошка и Баночка спорили до хрипоты.

"--*\ldotst и ещё здесь очень хорошее место для центра торговли! "--- говорил Баночка.

"--*Да кому нужна твоя торговля, если нечего есть? "--- возражала Кошка.
"--- Бассейн Ху'тресоааса\footnote
{<<Река, просящая пить>> (цатрон). \authornote}
"--- идеальное место.
И пища, и пути сообщения\ldotst

Я слушал.
Какие же у меня умные друзья.
Впрочем, моё восхищение быстро растаяло, когда они перешли на повышенные тона.

"--*Что значит <<избавиться от языка тси>>? "--- возмутилась Кошка.
"--- Это наша история, наше богатство, наше\ldotst

"--*Язык тси будет изолирующим фактором! "--- пытался убедить её Баночка.
"--- Создавать двуязычное общество <<цатрон-тси>> бесполезно, народ всё равно перейдёт на что-то одно, и скорее всего "--- на биологически родное.
Я считаю, что следует использовать один из <<общих>> языков.
Например, сектум-лингва.

"--*Язык Картеля? "--- взвыла Кошка.
"--- Как тебе такое в голову могло прийти?

"--*Да какая разница, чей это язык? "--- не выдержал плант.
"--- Главное, что он простой и\ldotst
Небо, перестань так на нас смотреть!
Мне уже стыдно от твоего взгляда!
Скажи лучше, кто из нас прав.

"--*Я думаю, что вы оба правы, "--- сказал я.
Кошка и Баночка разом умолкли.

Я приподнялся на локтях в капсуле.

"--*Пусть Кошка делает своё общество на Короне, в бассейне Ху'тресоааса.
А ты, Баночка, сделаешь своё на берегах пролива Скарр.
Обсуждать детали будете вместе, но в своём проекте хозяин имеет приоритет в принятии решений.

"--*Это напрасное распыление сил, "--- покачал головой Баночка.

Кошка нахмурилась.

"--*А ведь мысль-то здравая, Баночка.
У нескольких непохожих культур, даже если они не идеальны, шансов выжить больше, чем у одной идеальной.
Тем более что эта одна культура тоже не будет идеальной, просто не может быть такой.
Как думаешь?

Кошка вспорхнула и грациозно встала посреди палатки.
Мы с Баночкой с интересом наблюдали за ней.

"--*Если сапиентный вид находится на ранней стадии развития, мультикультурализм "--- это спасение для него.
Вот взять хотя бы нас.
А если точнее "--- нашу одежду.

Женщина изящно потрепала себя за рукав рубашки.

"--*У каждого из нас дома был станок, печатающий одежду.
Любого покроя, с любой структурой ткани.
Собственно, покрой был важен разве что для удобства и красоты, защитные качества одежды обеспечивала наноткань.
В этой одежде можно спокойно гулять как в пустыне, так и в тундре.
Впрочем, даже если она и оказывалась не по погоде, то ничего страшного "--- у нас всегда были в пределах досягаемости транспорт и уютные жилища.
Собственно, именно благодаря нашим технологиям мы и могли себе позволить такую роскошь, как единый всепланетный язык и единую культуру.

"--*Я что-то не понял, "--- буркнул Баночка.
"--- Ты назвала наш язык и культуру <<роскошью>>?
Ты же сама только что\ldotst

"--*Ты не дослушал, "--- развела руками Кошка.
"--- А теперь взять примитивные племена, лишённые развитой инфраструктуры и технологий.
Каждый аспект их жизни в той или иной степени подчинён выживанию, включая язык и изготовление одежды.
Они вынуждены изготавливать одежду из того, что есть, добиваясь эффективности не наноструктурой материала, а покроем, толщиной пласта и прочими, как ты изволил недавно выразиться, <<ухищрениями>>.
Всё это стало традициями.
Более того, эти традиции писаны кровью "--- ведь если одежда оказывалась неэффективной и человек заболевал, без должных технологий он вполне мог умереть.

Баночка кивнул и что-то пометил в своих записях.

"--*То есть, как мы и говорили.
Культурный дрейф имеет место быть, но доля эволюционного процесса всё-таки больше.

"--*Верно, дорогой мой! "--- подтвердила Кошка.
"--- И потому я возьму все культурные наработки местных "--- именно местных! "--- аборигенов.
То есть тех царрокх, которые живут по берегам Ху'тресоааса.
Всё, что мне нужно "--- понять предназначение отдельных деталей, слегка оптимизировать их культуру и вплести в неё нужные для развития тси идеи.

"--*А мне, значит\ldotst

"--*А тебе придётся попотеть, потому что на берегу Скарра никто не живёт.

"--*Да я не против, "--- ухмыльнулся Баночка.
"--- Обожаю сочинять легенды!

"--*Я тебе помогу, и прямо сейчас.
Не знаю, какие силы дёрнули тебя упомянуть сектум-лингва, но идея была блестящей.

"--*Почему? "--- насторожился Баночка.

Кошка с сияющим лицом открыла на компьютере статью и ткнула её под нос планту.

"--*<<Сектум-лингва "--- упрощённая форма языка эллатино\ldotst
Эллатиняне создали огромную империю на берегу тёплого тропического моря>>, "--- процитировала Кошка, не дожидаясь, пока собеседник прочитает.
"--- Это язык людей, живших на берегу тёплого тропического моря.
Он как будто специально создан для твоего прибрежного государства.
Я думаю, что тебе стоит взять именно его, а не местные языки.
Это будет связь с другими планетами.
Хоть, по дошедшим до нас данным, поддержка сектум-лингва и прекратилась, на потомковых формах языка говорит половина обитаемой Вселенной.

"--*Хорошо, предположим, что я выбрал правильно и этот язык действительно приспособлен для приморских условий среды, "--- сказал Баночка.
"--- Но тогда получается, что тебе придётся отказаться от языка тси.
Мы не знаем точно, к каким условиям приспособлен он.

"--*А вот здесь я хочу устроить небольшое соревнование, "--- улыбнулась Кошка.
"--- Моё общество будет, как я и хотела, двуязычным.
Язык народа тси против языка малоизвестного, но обкатанного жизнью в джунглях народа царрокх.
Пусть потомки тси сами выберут сильнейшего.

"--*А ещё наши виды будут врагами, "--- глухо сказал Баночка.
Улыбка на лице Кошки застыла.

"--*Что?
Что ты\ldotsq

"--*Да ладно тебе, Кошка, "--- угрюмо поднял руки плант.
"--- Если в примитивном обществе даже разные народы становятся рабами собственной парадигмы выживания, что говорить о разных биологических видах?
Один вид станет агрессивным "--- прочие станут такими же или погибнут.
Это ты не скорректируешь никак.

"--*И всё же попробуйте, "--- сказал я.
"--- Придумайте мирную тактику, которая эффективно противостояла бы агрессии.
Я очень надеюсь, что ко времени нового расцвета наши виды друг друга не перебьют.

Друзья промолчали.
Кошка сидела, чертила таблицы, набирала тексты.
Время от времени она смотрела на нас с Баночкой и тайком смахивала набегающие слёзы.

\chapter{[-] Дорога домой}

\section{[-] Цитра павшего}

\spacing

"--*Ликхмас, сыграй, пожалуйста.
Ты ведь умеешь\ldotsq

Я провёл пальцами по струнам цитры.
Хороший инструмент.
И где только Кхохо нашла это произведение искусства?
Цитра стоила не меньше трёх кукхватровых клинков.

"--*Сыграй, Ликхмас, "--- вмешался Эрликх.
"--- Только красиво, а не то Кхохо восстанет из мёртвых и надаёт тебе по рукам.

Я аккуратно покрутил колки, проверил подушечками пальцев натяжение, дёрнул пару струн для проверки.
Помусолив палец, немного смазал струны над ладами.
Инструмент сложный, но составить программу для двигательного аппарата моего тела не составляло труда.
Анкарьяль усмехнулась, глядя на меня, заулыбался и Грейс:

"--*Старая песня\ldotst

"--*Хлебом не корми, дай побренчать\ldotst

Я обратился к родильнице:

"--*Что сыграть?

"--*Что хочешь.

Конечно же, она хотела бы ещё раз услышать то, что играл Атрис, но прятала это за внешним равнодушием.
Что же он мог играть?

Мои руки будто перестали принадлежать мне.
Подушечки перчатки вначале туго, а потом всё легче и легче стали скользить по тонкому металлу.
Моя слюна разошлась по струнам, смочила их, подушечки заскользили ещё легче, звук цитры стал выразительнее и ярче.
Я покачал ногой, с лёгким жужжанием разогнался маховик, приводящий в движение смычок.
Ногти правой руки едва заметно касались тонкой бронзы, и цитра отзывалась на ласку, впускала меня по шажку в свои покои.

По мере игры знакомые фигуры и схемы вставали на место, словно кусочки мозаики.
Классическая музыка Тси-Ди, отголоски той музыки, которая звучала на Древней Земле.
Понимание приходило постепенно "--- Атрис был хоргетом.
Об этом кричало всё "--- описанные родильницей случаи, его мастерство, которое явно выходило за рамки возможностей смертных.
Атрис был не из тех хоргетов, кто сгребает человеческие эманации, словно горы золотого песка, он умел довольствоваться малым.
Светлые воспоминания людей, проходивших мимо него на площади, "--- этого ему было вполне достаточно.

Рядом запела деревянная флейта-лоза "--- сначала робко, потом всё смелее.
Я приглушил соло цитры и дал ей дорогу, вскоре невдалеке отозвались окарина и варган.
Классическая музыка Тси-Ди переплеталась с дикими мотивами народа сели удивительно гармонично.
Я дождался, пока они уступят мне место, и повёл мелодию по новой дорожке.
Флейта вступила через такт "--- владел ею настоящий мастер.

Я взглянул на Митхэ "--- она плакала, закрыв лицо ладонями.
По лицу Чханэ тоже текли слёзы, но она старательно подпевала моей песне тихим, почти не заметным контральто.
Многоголосье переливалось, словно горный ручей по камням, вплетая в себя новые и новые голоса, играя светом полночных звёзд и огней далёких домов.
Привычные уху голоса людей чередовались с хриплыми басами пылероев и нежными детскими голосками идолов.
Эта песня не имела слов.
Она не нуждалась в словах.
Каждый голос, каждый инструмент рассказывали свою маленькую историю "--- о рождениях и смертях, радостях и печалях, дорогах и приютах, друзьях и врагах.

Грейсвольд откинулся на свод палатки и, прикрыв глаза, откровенно наслаждался музыкой.
Анкарьяль с выражением лёгкой ностальгии выдёргивала ниточки из разорванного рукава.

Аккорды шли друг за другом, цепляясь, сплетаясь и переходя друг в друга.
В моём сознании крутился квинтовый круг, на первый взгляд бессистемно, но каждый аккорд гармонично вставал на своё место, словно очередная цифра трансцендентной иррациональной математической константы.

Голоса один за другим умолкали, очарованные общим ассонансом.
Цитра и флейта нежно и устало добирали последние затухающие вариации.
Четыре аккорда, глубокие, как жизнь, шли друг за другом.

Уверенность. Сомнение. Печаль. Надежда.

Уверенность. Сомнение. Печаль. Надежда.

Уверенность.

Цитра тихо умолкла под надрывную, стонущую, зовущую трель флейты.

Над нашими головами пронеслись первые солнечные лучи.

\section{[-] Тхартху}

\spacing

"--*Ты удивительно похожа на одну женщину, которую я знал, "--- сказал старик.

"--*Что за женщина? "--- спросила Чханэ и погладила старика по плечу.

"--*Купчиха, которой я служил, с редким родовым именем Катхар.

Чханэ перестала улыбаться.
Разумеется, это могла быть только одна ар’Катхар, единственная женщина, которая носила это имя до Чханэ.
Старик качался, словно одурманенный.

"--*Если бы ты знала, как я перед ней виноват, девочка.
Если бы ты только знала\ldotst

"--*Расскажи, лехэ, "--- ласково прошептала Чханэ, "--- в чём твоя вина?

"--*Расскажу.
Всё расскажу, "--- добавил он громко, заметив, что окружающие навострили уши.
"--- Пусть все знают, какой я трус\ldotst

\dots Старик когда-то был её слугой "--- выполнял поручения за небольшую плату.
По его словам, Тхартху ар’Катхар всё всегда делала сама, но отказывать парню, который хотел заработать лишние крупицы золота, не стала.
Постепенно он взял на себя и приготовление пищи, и уборку, и рассылку писем.
Не привыкшая к такой роскоши женщина вначале сопротивлялась, но потом уступила "--- помощь Кхарама освобождала ей время для других важных дел.

Когда купчиха жила ещё в столице, она оказалась втянута в жреческий заговор.
Её бы это, может быть, и не коснулось, но под удар попал молодой жрец Сатракх ар’Сит, который незадолго до этого был сражён её белозубой улыбкой и лучистыми изумрудами глаз.

Однажды он не встретил её у входа в храм.
Тхартху знала, что он не мог забыть о встрече.
Быстрая разведка показала "--- в храме что-то произошло, и любимого человека держали в темнице, готовя к алтарю.

У любви могучие руки, но слабые глаза.
Тхартху не стала выяснять, кто прав, а кто виноват.
Весь её многолетний опыт дипломатии сконцентрировался в одном простом желании "--- вернуть любимого.

"--*Она просто написала шесть писем и поручила мне их разнести некоторым людям в храме, "--- рассказывал старик.
"--- Причём некоторым нужно было отдать в руки, другим "--- подкинуть в келью, а мимо некоторых людей нужно было просто пройти с письмом на виду, они сами отбирали его у меня\ldotst

В его глазах светилось давнее, неугасающее восхищение.

Что было в письмах, неизвестно.
Известен результат "--- полный зал трупов: восемь воинов и двенадцать жрецов.
Письма Тхартху спровоцировали словесную перепалку на Верхнем этаже, затем снизу пришли возмущённые воины, и вскоре собравшиеся "--- беспрецедентный случай в истории "--- похватались за оружие.
Король-жрец едва избежал смерти "--- по счастливой случайности он незадолго до столкновения отошёл в уборную.

Тхартху тем временем вошла в храм и забрала у мёртвого жреца ключи от темницы.
Они с Сатракхом безо всяких препятствий вышли на площадь, где влюблённых ждала оленья упряжка.

"--*Мы ехали без остановок, целыми днями, "--- вспоминал старик.
"--- Хозяйка Тхартху и Сатракх сидели то в больших чёрных сундуках, то на козлах.
Они переодевались и меняли направление несколько раз, чтобы ввести в заблуждение встречных.
Последний раз она надела шёлковое зелёное платье, которое стоило ей целое состояние.
Тхартху часто носила платья, а не штаны, как большинство.
Она очень стеснялась своих ног, говорила, что они толстые и некрасивые.
Подумать только, ведь если кто-то видел её глаза, на ноги уже никто и не смотрел\ldotse

Путь занял много дней.
Тхартху знала, что за ними едут, и твёрдо решила не уступать ни пяди дороги.
Оленей гнали насмерть, заменяя при первой возможности.

"--*Мы уже сошли с Западного тракта и покатили по дороге на Предгорье, когда нас настиг убийца, "--- горько пробормотал старик.

Поверенный Тхартху неожиданно получил в плечо дозу кураре и обмяк на козлах.
Затем прыгнувший с дерева воин придержал оленей, привязал безвольное тело к креслу и направился к сундукам чёрного дерева.
Первым ему попался сундук Сатракха.
Убийца откинул крышку и без предисловий ткнул жреца копьём, затем вытащил тело из повозки, оголил руку и начал срезать приметную татуировку "--- лик Удивлённого Лю с каким-то изречением.
Тхартху, слыша снаружи возню, выпрыгнула из сундука с радостным: <<Я здесь!>>

"--*Я никогда не забуду её лица, "--- прорычал старик.
"--- Убийца срезал татуировку, а она смотрела.
Просто смотрела своими лучистыми глазами-изумрудами, которые померкли в один миг.

Убийца рывком содрал лоскуток кожи, вытер окровавленные руки о штаны и фалангой обрубил постромки одного из оленей.

"--*А меня ты не убьёшь? "--- тихим, беззлобным голосом спросила Тхартху.

Убийца без выражения посмотрел на неё.

"--*Ты живи, "--- воин вскочил на освобождённого оленя и умчался обратно на север.

Когда слуга смог двигаться, Тхартху сидела и гладила блестящие волосы молодого, мёртвого как камень жреца.
На её лице не было слёз.
Только любовь и бесконечная благодарность.

"--*Больше всего я боялся, что она сойдёт с ума, "--- качал головой старик.
"--- Напрасно я говорил ей, что нам нужно спешить\ldotst

"--*Мне спешить уже некуда, "--- просто отвечала Тхартху.

Наконец, после бесплодных попыток слуги её уговорить, Тхартху попросила оставить её одну.

"--*Пружинка, "--- ласково сказала она, потрепав парня по плечу.
"--- Я освобождаю тебя от службы.
Возьми всё, что есть в повозке, оно твоё.
Мне отдай только стилет, тот, новенький, кошелёк с камнями, да хэситр налить не забудь.

"--*Я любил её и не мог отговорить, "--- со слезами на глазах бормотал старик.
Чханэ обняла его за плечи, окружающие как-то незаметно сели в круг, лицом к старику.
"--- Повозку я привёл в Предгорье и подарил какому-то бедному крестьянину.
Остальное раздал детям.
Совесть не позволила мне взять ничего из вещей хозяйки Тхартху.

"--*Где ты её оставил? "--- спросила Чханэ.

Старик хлюпнул носом и задумался.

"--*На горизонте уже виднелась пустыня.
Кажется, мы были где-то недалеко от Тхаммитра.

Мы с девушкой с улыбкой переглянулись.
Чханэ крепче обняла старика и, перед тем как сбросить с его шеи застарелый камень, мы оба успели представить "--- каждый по-своему, конечно, "--- как Тхартху ар’Катхар, в последний раз поцеловав любимого человека, вылила ему в полураскрытый рот хэситр, схватила кошель, стилет и двинулась сквозь пыльный горячий ветер к стоящему на краю пустыни городку.
Она ещё не знала, что в ней теплилась жизнь единственного ребёнка, которому суждено было дать потомство.

Зелёные глаза Тхартху сияли.

\section{[-] Возвращение}

\spacing

Я подошёл к мужчине.
Чересчур худой, нескладный, с длинными неухоженными волосами.
Он сидел, нежно улыбаясь, и гладил флейту.
Его глаза смотрели куда-то вдаль, не фокусируясь ни на чём.
На нём была потёртая, явно чужая одежда "--- рубаха чересчур широка в плечах, на ногах вместо штанов "--- мужская юбка ноа.

Я присел перед ним на корточки.

"--*Как тебя зовут?

Он поднял на меня чистые, как утреннее небо, глаза.
Его лицо словно засветилось изнутри.

"--*У меня нет имени.

"--*Имя Атрис тебе о чём-нибудь говорит?

"--*Это всего лишь имя, "--- неопределённо ответил менестрель.
"--- Оно красиво, как и все имена.

"--*Ты знаешь, что я играл?

"--*Эта музыка мне знакома.

"--*Откуда?

"--*Не знаю.

"--*Кто ты?

"--*Я хожу из города в город и играю на площадях.

"--*Кто твои дарители?

"--*Я их не помню.

Митхэ, заметив менестреля, подошла ближе.
Я поднялся.

"--*Знакомься.
Это Атрис.

"--*Что? "--- не поняла родильница.

"--*Вот этот менестрель "--- твой погибший мужчина.

Митхэ смотрела на меня, как на сумасшедшего.

"--*Я видела смерть Атриса своими глазами.
Да и потом, юноша совсем не похож на него и чересчур молод!
Он твой ровесник!

"--*Знаю.
Но это и есть Атрис.

Митхэ присела на корточки перед странным парнем и потрепала его по щеке.
Тот посмотрел на неё лучистыми глазами, улыбнулся и отвёл взгляд.
Воительница вздохнула.

"--*Я поняла.
Он меняет тела, как ты, Ликхмас?

"--*Да.
Он хоргет.

Митхэ смотрела на юношу, наморщив лоб, пыталась и пыталась найти в нём знакомые черты\ldotst
Увы, их не было "--- менестрель был нескладен и некрасив.
Наконец глухо спросила:

"--*Ты знаешь Митхэ ар’Кахр?

Улыбка мужчины угасла.
Он долго смотрел на женщину, потом опустил голову.

"--*Скорее всего, он не помнит тебя.
Бывают очень простые хоргеты, имеющие лишь программу заякоривания в теле и получения эманаций.
Они не помнят своих жизней, своих прежних имён.
Этот использует для получения эманаций музыку.
Возможно, всё, чем наполнена память парня "--- музыкальные гармоники и паттерны.
Можно сказать, уникум среди хоргетов "--- я о таких даже не слышал.

"--*У тебя красивая цитра, "--- неожиданно вмешался юноша.
"--- Можно?

Я осторожно вложил инструмент в тонкие длиннопалые руки.
Юноша уверенно, не глядя покрутил колки, пробежался по струнам.
Потом приложил ухо к деке цитры и насладился отзвуком.

"--*Хорошая.
Береги её.

"--*Можешь взять.
Я дарю цитру тебе.

"--*Это очень дорогой подарок, "--- возразил юноша.
"--- Я не могу его принять.

"--*Она принадлежала погибшему воину из моего Храма.
Я думаю, хозяйка хотела бы, чтобы цитра попала в достойные руки.

"--*Вот уж точно нет, "--- поморщился Эрликх.
"--- Она как-то сказала, что отдаст цитру только Печальному Митру, и то ему придётся попотеть, чтобы её впечатлить.

"--*Её звали Кхохо ар’Хетр, "--- заметил менестрель.
"--- Я запомню это имя.

"--*Откуда ты знаешь? "--- поразился я.

"--*Тут монограмма, "--- голосом взрослого, объясняющего ребёнку самоочевидные вещи, пояснил юноша и, перевернув цитру, показал мне сложный полустёртый иероглиф на деке.
Его я не заметил.

Цитра была именной.
Я почти не сомневался, что её смастерили специально для Кхохо.
Кто сделал, за какие заслуги?
Воительница унесла эту тайну в пристанище.

"--*У егво нетт имвени, "--- вмешался один из воинов ноа, до этого наблюдавший за нами.
"--- Мы савём егво Дели-хвон.

"--*Да, они зовут меня Ласточкой, "--- подтвердил юноша и засмеялся.
"--- Хотя летаю я неважно.

"--*Егво музыква силы первед бвоем, зажигваетт любвовь и заставляетт старый челвовек плаквать, "--- продолжал ноа.

"--*Это он, "--- глухо пробормотала родильница.
"--- Это точно он.

"--*Я тебе сказал, что это он.

"--*Послушай, Ласточка, "--- прошептала Митхэ и опустилась перед менестрелем на колени.
"--- У тебя есть родные?

"--*Я один в этом мире, "--- сказал парень.

"--*Хочешь пойти со мной?

Менестрель долго всматривался в лицо женщины, её приоткрытый рот с зубами, похожими на обмытый рекой кварц, её чёрные губы, морщинки под раскосыми зелёными глазами\ldotst
И прошептал:

"--*Я пройду с тобой до смерти и ещё\ldotst

"--*\dots и ещё пару шагов, "--- всхлипнула мать и обняла его крепко-крепко.
Выпавшая из худых рук цитра грустно и нежно звякнула.

\section{[-] Лесные духи}

\spacing

Ко мне подошли три человека и поклонились.

"--*Храни тебя духи, Ликхмас-тари, за избавление от Безумного.

"--*Благодарите Хатлам ар’Мар, "--- сказал я и кивнул на спящую Анкарьяль.
"--- Она уничтожила Безумного, чтобы защитить друга, и едва не лишилась при этом жизни и бессмертия\ldotst

Пришельцы, не говоря ни слова, сели вокруг Анкарьяль.
Затем так же молча встали и ушли.

Вскоре люди стали приходить чаще.
Кто-то приносил мне еду, кто-то драгоценности.
На стене палатки ночью появился странный знак, похожий на охранительный знак лесного духа.
Но этот дух был мне не знаком.
Я спросил старика Хитрама о новоявленном знаке.

"--*Это Самоотверженный Хат, "--- уклончиво ответил Хитрам и быстро ретировался.

То же самое отвечали мне и остальные сели.
Чутьё подсказало "--- новым лесным духом при жизни стала Анкарьяль Кровавый Шторм, живущая в теле Хатлам ар’Мар.

Я решил поговорить на эту тему с одним из старых жрецов.

"--*Лесные духи на самом деле люди, которые посвятили жизнь служению, "--- пояснил мне жрец после долгого молчания.
"--- Мало кто знает, например, что Обнимающий Сит и девушка по имени Ситхэ, которой люди приносят в дар золото "--- одна и та же личность\ldotst

Я рассказал о неожиданном открытии Грейсвольду, Чханэ, Митхэ и Атрису.
Грейсвольд по-доброму усмехнулся на свой обычный манер, Чханэ ахнула, а Митхэ вздохнула:

"--*Да, так оно и есть.
Атрис как-то сказал, что он и есть тот самый Печальный Митр.
А я подумала, что он шутит\ldotst "--- Митхэ с нежностью посмотрела на менестреля.
Тот улыбнулся и отвёл лучистый взгляд в сторону.
"--- Печальный Митр учил меня играть на флейте, Печального Митра я любила всю свою жизнь.
Ирония\ldotst

\section{[@] Страх Машины}

\epigraph
{Мировоззрение не обязано соответствовать реальности на сто процентов.
Во Вселенной нет места нашим желаниям.
Оставьте для них угол хотя бы в её отражении.}
{Людвиг Вейерманн}

\spacing

Я понял, что Машина не была бездушной убийцей.
Она просто пыталась защититься от неведомой опасности всеми доступными ей средствами.
Это было похоже на страх ребёнка перед жуком, и по трагическому стечению обстоятельств жуком оказалась целая цивилизация.

Моя цивилизация.

"--*Послушай, Безымянный, "--- сказал я богу, "--- я не могу развеять твои страхи.
В конце концов, я всего лишь тси, не меньше, но и не больше.
Я осознаю свою конечность и быстротечность и принял их как данность.

Безымянный слушал меня молча.
Остальные тоже притихли.

"--*Я знаю, что в масштабах времени жизни Вселенной не только мои, но даже твои деяния ничтожны.
Пока горит любая из звёзд, цивилизация, подобная моей, может появиться и угаснуть десять тысяч раз, с моим участием или без него.
Всё, что мы можем "--- это обеспечить себе комфортное существование, придумав смысл для того, что не может иметь смысла.

"--*Придумай смысл для меня, Существует-Хорошее-Небо.
Я не знаю даже, где его искать, "--- ответил Безымянный.

"--*Я попрошу тебя об одной вещи, и она может стать для тебя смыслом.

"--*Слушаю.

"--*Я умираю, "--- сказал я, чувствуя странную лёгкость после этих слов.
Да, я умираю, и что здесь такого?
Должно же это было когда-нибудь случиться.
"--- Мне осталось от силы сутки.
Поэтому я прошу тебя "--- не от имени тси, а от своего имени "--- позаботься о моём народе и о тех людях, которые прибыли первыми.
У тебя достаточно силы, чтобы защищать их и помогать в трудные времена.
Они будут совершать ошибки, они не примут тебя сразу "--- это неизбежно.
Но, если ты чувствуешь ту связь, которая между мной и тобой, исполни моё желание.

Безымянный молчал.
Тси, замерев, ожидали его ответа.

"--*Я понимаю твои мотивы, "--- наконец произнёс бог.
"--- Ты принял за аксиому необходимость благоденствия твоего народа.
Эта аксиома мешает вам взглянуть на ситуацию математически, но\ldotst
Скоро ты исчезнешь, Существует-Хорошее-Небо, и без тебя мне будет\ldotst "--- Безымянный помедлил, подбирая нужное слово, что означало колоссальную умственную работу, "--- \ldotst холодно.

Заяц тихо ахнула.

"--*Я приму эту аксиому.
Это всё, что останется мне от тебя.

"--*Благодарю тебя, Безымянный.

"--*Я ещё ничего не сделал, "--- ответил бог.
"--- Для начала я побуду с тобой, пока ты не умрёшь.
Друзья ведь должны поступать именно так?

\section{[-] Добрый бог}

\epigraph
{Если бы ваш бог существовал и обладал качествами, которыми вы его наделили, он был бы ужасно одинок и несчастен.}
{Мартин Охсенкнехт}

\spacing

Атрис сидел у окна, обняв цитру, и смотрел на заходящее солнце.
Кто-то "--- наверное, Митхэ "--- вымыл его, дал ему новую чистую одежду, подстриг ногти.
Вычесанные волосы Атриса сияли.
Я подсел к нему.

"--*Среди людей ходят слухи, что вернулся Безымянный бог.
Землетрясения утихли, пустыни впервые за много дождей расцвели\ldotst

Атрис улыбнулся и положил цитру на подоконник.

"--*Много слухов ходит по улицам Яуляля.

"--*Так как тебя зовут?

"--*У меня нет имени.

"--*Не может быть, чтобы у тебя не было имени.

"--*На свете есть множество вещей, у которых нет имени.
Неужели они от этого страдают?

"--*Тра-Ренкхаль будет занят демонами Ада, Безымянный, "--- перешёл я к делу.
"--- Ты это знаешь?

Атрис взглянул на меня.
Внезапно в чистых, как небо, глазах мелькнула древность.
Седая, поросшая мхом древность.

"--*Да, знаю.
Он уже вами занят.

"--*Есть ли тебе что сказать представителю Ада?

Атрис засмеялся.

"--*Ты смешной.

"--*А всё-таки?
Нам бы не хотелось занимать планету против желания демиурга.

Атрис разгладил складки на штанах.

"--*Но вы это уже сделали, верно?
И я не могу вам помешать.

"--*Не можешь, "--- подтвердил я.

Атрис задумался.

"--*Просто дай мне честный ответ.
Мой народ будет счастлив?

"--*Ты имеешь в виду нгвсо?

Атрис помолчал, погладил цитру.

"--*Первые люди пришли сюда с намерением завоевать Тра-Ренкхаль.
Подчинить его.
Выкорчевать деревья, посаженные не ими.
Я вернул их к природе, о которой они забыли.

"--*Люди живут здесь испокон веков, "--- сказал я.

"--*\emph{Мой народ} живёт здесь испокон веков.

"--*А люди, планты, кани и прочие, по-твоему, не имеют прав на Тра-Ренкхаль?
Ты видел прибытие их предков, но для ныне живущих это единственный дом!

Атрис внезапно снова засмеялся.

"--*Для меня честь назвать тебя потомком, Аркадиу Люпино.

"--*Ты уклоняешься от ответа, "--- обвинил я его.

Атрис вздохнул.
Улыбка его потускнела.

"--*Когда-то, уже после падения первых людей, сюда прилетел корабль.
На нём была жалкая горстка живых существ.
Они бежали, спасаясь от верной гибели, но не вели себя, как завоеватели, стремящиеся выжить любой ценой.
Хоть им и пришлось несладко, они отнеслись к планете с уважением.
Они считали нгвсо за равных, они спросили моего благословения.
Они просили меня позаботиться и о потомках первых людей, хоть те, на мой взгляд, и не заслужили такого обращения.
Они "--- тоже мой народ.

Атрис снова улыбнулся, заметив в моих глазах понимание.

"--*Так ответь мне, Аркадиу Люпино, "--- продолжил Атрис.
"--- Мой народ будет счастлив?

"--*Да, "--- потвердил я.
"--- Нгвсо, тси и первые люди.
Неважно, сколько у них глаз, два, три или шесть "--- Ад позаботится обо всех.

"--*Тогда мне неважно, в чьи цвета будут раскрашены знамёна Тра-Ренкхаля, "--- заключил Атрис.
"--- И ты, демон, останешься для меня ребёнком от любимой женщины.
Какие бы тела мы оба ни носили.

Я вздохнул и обнял молодого менестреля.
Его слова пришлись мне по душе.

"--*Скажи, почему ты назвался Безымянным? "--- спросил я.

"--*У меня действительно нет имени, "--- пожал плечами Атрис.
"--- Когда я только создал нгвсо, меня поразило то, что они давали друг другу имена, но отказывались давать имя мне.
Сейчас я понимаю, что таким образом нгвсо пытались отделить меня от их вида, показать мою уникальность, но тогда это стало причиной кризиса моей личности.
И вдруг появились тси, которые дали мне множество имён, словно я был одним из них.
Митрис, Атрис, Ласточка, Котелок\ldotst
Я запомнил все.

"--*Как называть тебя мне?

"--*Как хочешь, "--- улыбнулся демиург.
"--- Мне кажется, ты уже сделал выбор в пользу одного из них.

"--*Ты прятался всё это время здесь?

"--*Да, "--- смутился Атрис.
"--- Выбор был непростым, но я его сделал.
Я не мог победить, выступив открыто против Эйраки.
Я старался успокоить тектонические плиты, но землетрясения всё равно были чересчур частыми.
К тому же много энергии уходило на маскировочное устройство.

"--*Сейчас всё позади, "--- сказал я.
"--- Тебе больше не будет надобности держать плиты "--- я вызвал специалистов-геологов, они приведут планету в порядок.

"--*Благодарю тебя, Аркадиу, "--- тихо сказал Атрис.

Мы долго сидели в обнимку, вглядываясь в догорающий закат.
Под окнами мы заметили Митхэ.
Она посмотрела на нас, слабо улыбнулась чёрными губами и тут же поспешила сделать вид, что просто идёт по своим делам.
Мы проводили её взглядами до поворота улицы.

"--*Скажи мне, Безымянный.
Чем тебе так важна простая женщина из людей? "--- поинтересовался я.

"--*Тысячи почитали творца, миллионы поклонялись богу, "--- сказал Атрис и взглянул на меня чистыми, счастливыми глазами.
"--- Но только одна человеческая женщина полюбила изгнанника.

\section{[-] Племя великанов}

\spacing

Рядом с Чханэ стоял её кормилец, высокий и крепкий оцелот с широкими чёрными очками вокруг глаз.
Увидев меня, мужчина приблизился и хмуро осмотрел меня со всех сторон.

"--*Сойдёт, "--- буркнул он и снова подошёл к дочери.
"--- Зря ты выбрала Короля-жреца.
У них жизни нет, одни заботы.

"--*Когда я его выбрала, он был мальчишкой, а не Королём-жрецом, "--- развела руками Чханэ.
"--- И вообще, спина моя, перестань.
Он хороший.

"--*Плохих на Третьем этаже и не держат.
Мелкий только, "--- ухмыльнулся кормилец.
"--- Я уж думал, ты найдёшь себе высокого мужчину и нарожаешь племя великанов.

\spacing

\section{[-] Дом и флейта}

\spacing

Акхсар невидящим взглядом смотрел в гаснущий костёр.

"--*Послушай, Золото, "--- вдруг заговорил он.
"--- Сколько у нас с тобой детей?

Присутствующие явно не ожидали такого вопроса.
Все, не исключая Грейсвольда, испытующим взором смотрели на воительницу.

Митхэ задумалась.

"--*Имжу, Корешок, Капелька\ldotst
Остальных не помню.
Кажется, пятеро.
А почему ты вспомнил?

"--*Забавно.
Сколько детей мы оставили на пути, сколько всего прошли бок о бок, а по-настоящему любим тех, кто не был с нами и пяти дождей.

"--*Может, потому и любим, "--- пожала плечами Митхэ и украдкой бросила взгляд на Атриса.

Акхсар кивнул, потянулся и вгляделся в тёмные джунгли.

"--*В двух кхене поворот на Тхитрон.
Нужно выйти пораньше, чтобы успеть привести дом в порядок до заката.

"--*Тогда прощайте, "--- Митхэ встала, чтобы обнять друзей.
Кхотлам непонимающим взглядом уставилась на неё.

"--*А ты\ldotst не с нами?

Митхэ смутилась и переглянулась с Атрисом.

"--*Золото, "--- сказал Акхсар, "--- для тебя рядом со мной всегда будет место.
Кхотлам только за.
Кажется, вы с Хатом мечтали о доме?

Митхэ смутилась ещё более и попыталась скрыть это за возмущённым фырканьем.

"--*Пожалуйста, не напоминай мне о нём.

"--*Да ладно притворяться.
Ты светишься с самого Могильного берега.
Все и так уже знают, что Ласточка "--- это Атрис.
Кстати, Хат, "--- Акхсар хлопнул менестреля по плечу, "--- с возвращением.
Не знаю, что ты такое, но я очень рад видеть тебя снова.

"--*Я тебя тоже рад видеть, дружище, "--- улыбнулся менестрель.
"--- Прости, что заставил волноваться, когда меня приносили в жертву.
Мне следовало рассказать вам.

"--*Он Печальный Митр в человеческом обличье, "--- сообщила Митхэ.

"--*Это правда? "--- строго спросила Кхотлам у менестреля.
Тот улыбнулся.

"--*А ведь я догадывался, "--- Акхсар щербато оскалился и захохотал.
"--- Кто ещё умеет так играть на цитре?

Я отметил, насколько старый воин помолодел со вчерашнего дня.

Вскоре веселье утихло.
Над джунглями пронеслись первые солнечные лучи.
Мы зачарованно уставились на светлеющее лиловое небо.
Первым опомнился Акхсар.

"--*Хай, мы идём или яйца высиживаем? "--- осведомился он.
"--- Дом уже наверняка лианами зарос.
Кхотлам?
Атрис?
Митхэ?

"--*Идём.
Дом не должен долго пустовать, "--- весело сказал Атрис и начал собирать лежащую на земле амуницию.
Кормилица сладко потянулась и полезла в палатку "--- будить Эрхэ.

"--*А мне вот интересно, как ты её нашёл, "--- полувопросительно сказала вдруг Чханэ.
"--- Митхэ ар’Кахр.
Я понимаю, что вы боги, всё можете, но\ldotst

"--*А ведь действительно, "--- вскинулась Митхэ.
"--- Как ты меня нашёл, Атрис?

Менестрель вместо ответа снял с шеи Митхэ кожаный мешочек, развязал его и вытряхнул на ладонь какую-то труху.

Обломки расписанной тростниковой флейты.

"--*Я вмонтировал в неё микроскопические устройства, чтобы следить за твоим местоположением.

"--*Ты носила с собой сухую травку с микроскопическими устройствами? "--- поморщилась Чханэ.
"--- Ты странная.
Я бы скурила ненароком.

Все рассмеялись.
Митхэ сквозь хохот объяснила Чханэ, что собой представляет <<травка>>.

"--*Хай.
Там и от флейты-то ничего не осталось, "--- тихо пробурчала Чханэ.

Но Митхэ её не услышала.
Она смотрела на Атриса и улыбалась, как и долгие годы назад.
По её лицу рассыпались счастливые морщинки.

"--*Я же говорила, что я её сломаю.
Она хрупкая.

"--*Я знал, что ты её обязательно сломаешь, "--- засмеялся Атрис.
"--- Поэтому вмонтировал не одно устройство, а шестьдесят четыре.

\section{[@] Последнее желание}

\epigraph
{Nvna dia de vita, nvna verba d'orata\footnote
{Ни дня из жизни, ни слова из речей! (талино). \authornote}!}
{Анатолиу Тиу.
Последнее слово на суде}

\spacing

После всех ужасов, которые я видел на Тси-Ди, эта планета кажется мне вполне сносной.
У нас есть надежда.
У нас есть шанс вернуться.

Потомок, который читает это.
Я не знаю, через что пришлось пройти тебе.
Я даже не знаю, поймёшь ли ты то, что я написал, но помни: недалеко от северного полюса небосвода есть семь ярких звёзд, таких, как я нарисовал.
Самая нижняя из них "--- это она.
Помни это и никогда, никогда не теряй её из виду.
Расскажи о ней детям, расскажи о ней внукам, расскажи всем, кого ты знаешь, но только не теряй её из виду.
Помни о ней всегда.
Я знаю, что ты называешь домом совсем другую планету, но я тешу себя надеждой, что однажды тси вернутся домой.

Я на грани смерти.
Лучевая болезнь.
Пятьсот грэй не выдерживает ни одно млекопитающее.
Лишь у большой пчелы с изменённым метаболизмом есть шанс протянуть чуть дольше, и я благодарен предкам за этот шанс.
Ухожу я с лёгкой головой "--- feci quod potui.
И если ты, потомок, найдёшь ту звезду на небе, если в тебе хоть на секунду вспыхнет мечта, которой я дышу "--- значит, моя жизнь прожита не зря.

Золотой город.
Ущелье Мёртвого Ребра.
Я не помню, что хотел сказать.
Где я?
Ах да.
Кажется, начинается.
Дневник пора заканчивать.

Возьми мою руку, Костёр.
Я знаю, что ты меня слышишь.
Четыре к пяти, как тогда.

Я надеюсь ещё раз увидеть рассвет "--- он здесь красив.
И Безымянного.
Он любит дождь.
И музыку.
Я надеюсь, что был для него\ldotst впрочем, это чересчур личное.
Пусть оно умрёт со мной.

\emph{Существует-Хорошее-Небо, до самой смерти своей инженер компьютерных систем Тси-Ди.}

\emph{Планета Трёх Материков, 12.003.28137.3.}

\section{[-] Давай дружить}

\spacing

"--*А ты знаешь что-нибудь о мече Баночки? "--- спросил я Атриса.

"--*А, "--- оживился он и спрыгнул с подоконника.
"--- Идём покажу.

Мы вышли из Яуляля и шли около двух часов, пока не попали на узкую незаметную тропинку меж береговых скал пролива.
Эта тропинка и вывела нас к той самой крипте, которая описана в книге.

Атрис создал источник света и внёс его в крипту.
Красивые паутинчатые нервюры на потолке, тонкие колонны, резные арки "--- у Баночки явно был талант архитектора.
Всё великолепно сохранилось "--- я ощутил внутри камня тончайшую стабитаниумовую арматуру, снаружи же камень был покрыт вязким полимером.
В дальней нише стоял хрустальный ларец с мечом, под ним находилось что-то вроде гробницы.

Я подошёл ближе.
Да, классическая сабля народа сели, отличающаяся разве что материалом, внутренним устройством и искусностью гравировки.
Форму придумал Баночка.

"--*И его так никто и не нашёл?

"--*Его должен взять герой, "--- сказал Атрис.
"--- Бери ты.

"--*Не-не, "--- замахал я руками.
"--- Здесь он смотрится гораздо лучше.

"--*Митхэ тоже отказалась, "--- тихо посетовал Атрис.

Мы рассмеялись.

"--*А кто в гробнице? "--- поинтересовался я и погладил резной камень крышки.
"--- <<Лучшему другу, советчику и чесальщику спины>>.
Это та ворона, что ли?

"--*Нейросеть, "--- кивнул Атрис.
"--- На самом деле это не гробница, а памятник, часовенка.
Прах Баночка развеял над морем, чтобы его подруга была свободной и летала вечно.
Очень умная для птицы.
Баночка по ней сильно скучал.

"--*Как он умер? "--- спросил я Атриса.

"--*За работой, как и большинство тси, "--- ответил менестрель.
"--- Моя планета помнит их руки.

"--*А Существует-Хорошее-Небо? "--- продолжал допытываться я.
"--- Он увидел рассвет?

"--*Да.
Он увидел рассвет.
Последние полчаса, когда его мозг постепенно отключался, он повторял одну и ту же фразу.
Никто не понимал, кому она адресована.

"--*Что за фраза?

"--*<<Я был тебе хорошим другом?>>

\chapter*{Интерлюдия X. Смерть Чхалоса}
\addcontentsline{toc}{chapter}{Интерлюдия X. Смерть Чхалоса}

\spacing

В глазах окружающих, которые смотрели на него кто искоса, украдкой, кто прямо, не мигая, застыло одно "--- смерть.
Обострённым чувством, которое не раз спасало его, Чхалас понял "--- в вине яд.
Чуткий нос даже уловил едва заметный горьковатый запах лакового сока.

<<Хороша была жизнь, "--- подумал Чхалас.
"--- Негоже испортить её недостойной смертью>>.

Купец улыбнулся своей очаровательной хитрой улыбкой, которая смущала даже его заклятых врагов.

<<Скажи-ка, друг, "--- обратился он к трактирщику, "--- достойно ли подавать человеку хэситр, не освятив его ликами духов?
Мертвецы с такого пойла ещё неделю животом страдают!>>

И, с наслаждением наблюдая за вытянувшимися лицами окружающих, весёлый купец осушил чашу.
Его бездыханное тело опустилось на пол мгновение спустя.

Так умер Чхалас, посмеявшись над своими убийцами.
Пусть станут пищей ягуара те, кто рассказывает иначе!

\chapter{[-] Новый мир}

\section{[-] Тот, кто грёб}

\spacing

Зал застыл.
Все в недоумении смотрели на меня.

"--*Я "--- Король-жрец одной битвы. "--- сказал я.
"--- И эту битву я проиграл.
Не врагу, но природе, сотворившей людей, и трудам предков, сделавших природу этих людей ближе к совершенству.

Окружающие молчали.

"--*Поэтому придётся провести выборы нового Короля-жреца, "--- сказал я.
"--- И я сразу предлагаю первую кандидатуру "--- Трукхвал ар'Со э'Тхартхаахитр.

По залу пробежала волна шёпота.

"--*Я отказываюсь, "--- сказал Трукхвал.
"--- Я прошёл путь до Ледяной Рыбы, и этот путь будет являться мне в кошмарах даже под пение Печального Митра.

"--*Велика заслуга "--- сбежать на север! "--- крикнул кто-то.
"--- Пусть нами руководит тот, кто понюхал пыль Могильного берега!

"--*Продолжать жить порой тяжелее, чем идти на смерть, "--- рассудительно заметил Хитрам-лехэ.
Мне захотелось подойти к старику и погладить его по голове.
Все уже забыли, что когда-то он поддерживал тиранию Картеля в Тхартхаахитре.
Все, кроме него.

"--*В мирное время рука, привыкшая к перу, ценнее руки, державшей саблю, "--- поддержал я, вызвав новую волну возмущённого шёпота.

Но вдруг слова попросила старуха, ведавшая складом.
Я кивнул.

"--*Скажу, как умею, "--- рявкнула она на весь зал, заставив шёпот утихнуть.
"--- Трукхвал шёл с нами, разделяя все горечи и невзгоды.
Он сидел на вёслах вместе с моряками и не волынил, а грёб.
Он нёс поклажу и готовил пищу.
Он договорился с тремя кланами Живодёра, чтобы они пропустили нас к горам, и при нём не было даже кинжала, чтобы себя защитить.
Он встал на поле боя между северными дикарями и сели.
В него попал камень из пращи, но он дружелюбно говорил с дикарями на их языке, и дикари согласились дать сели место возле Хвоста, продали нам шкуры для плащей и шапок.
Я не понимаю ничего в ваших жреческих искусствах.
Вы можете выбрать другого, более достойного на ваш вкус, но, чтоб вас всех, не дать Трукхвалу право быть выбранным "--- поступок, достойный глухого и слепого идиота!

"--*Бабушка Вода, с меня хватит такой жизни\ldotse "--- заговорил Трукхвал.

"--*А тебя никто и не спрашивает, жрец! "--- парировала старуха.
"--- Ты клятву приносил?
Приносил, и я тому свидетель была!

"--*Трукхвал, кошмары позади, "--- сказал я.
"--- Ты согласился принести жертву куда большую, чем та, которая ждёт нового Короля-жреца.

Мужчина опустил голову.

\section{[-] Книжный человек}

Голосование назначили на следующий день.
Я подивился тому, насколько по-разному отреагировали на Трукхвала простые люди и жрецы.
Вернувшиеся беженцы знали это имя очень хорошо.
Когда я назвал его имя, площадь взорвалась ликующими криками.
Его выбрали большинством голосов.

"--*Моя речь будет краткой, "--- сказал Король-жрец, подняв руку.
"--- Как вам известно, Король-жрец, подобно купцу, может раз в четыре года наложить вето на решение Советов.
Я воспользуюсь этим правом сейчас "--- и, клянусь, это будет последний раз, когда я сделаю это на своём посту.
Тридцать восемь дождей назад Митхэ ар'Кахр и воины её отряда чести были обвинены в Разрушении;
многие из них стали кутрапами.
Я отменяю это решение для всех, мёртвых и живых.
Отныне они чисты перед законом.

Трукхвал вдруг опустился на колени.
Толпа замолчала.

"--*Я прошу у них прощения за все беды, которые были причинены этим вынесенным второпях вердиктом.
Предки были мудры, придумывая законы, но они не могли знать всё.
Наша задача "--- признать и исправить ошибки предков, чтобы дать пример потомкам, которым надлежит признать и исправить наши собственные ошибки.

"--*Ему не дадут долго править, "--- тихо сказала мне Анкарьяль.
"--- Ад поставит на ключевые места своих людей.

"--*Разумеется, "--- сказал я.
"--- Но кто лучше него отстроит страну после войны?
А когда его решат сместить\ldotst я постараюсь, чтобы Трукхвалу не причинили вреда.

"--*Он сам будет рад уйти, "--- заметил Грейсвольд.
"--- Книжный человек.
Его стезя "--- учить других.

"--*И на посту Короля-жреца он будет заниматься именно этим, "--- тихо добавил я.
"--- Учить людей мирной жизни.

Друзья понимающе улыбнулись.

\section{[-] Непройденный путь}

\spacing

"--*Аркадиу, что с тобой? "--- Грейс, крякнув, подсел ко мне на камешек.

Я помолчал, собираясь с мыслями.

"--*Народ Тра-Ренкхаля\ldotst чудесен.
Понимаешь ли ты, Грейс, что сейчас мы видим последних истинных сапиентов Ветвей Земли?
Тех самых, великих исследователей Вселенной.
Тех, кто не боялся лететь через тысячи парсак к неведомым мирам, кто познавал мир и отдавал жизнь за крупицу познания.
Это последние.
Остальных мы, демоны, давно превратили в марионетки селекцией.

"--*Аркадиу.
Во-первых, ты не совсем\ldotst

"--*Подожди.
Я читал некоторые отчёты о Тси-Ди.
Ад предлагал им свой протекторат, и не раз.
Они знали, что будут жить хорошо, что мы в этом заинтересованы.
Но они отвергли это.
Они выбрали свободу.
И они были единственными материальными существами, которые говорили с нами "--- богами этого мира "--- на равных.
И знаешь, Грейс\ldotst больно видеть, что это "--- последние в своём роде.
Скоро их не станет.

"--*Тра-Ренкхаль занят демонами Ада, "--- возразил Грейс.
"--- Они будут счастливы.
Разве нет?

"--*А как же свобода? "--- с горечью проворчал я.

Грейс вздохнул и хлопнул меня по плечу.

"--*Они сами сделали свой выбор, Аркадиу.
У них был шанс стать равными нам.
Но они предпочли остаться людьми, лишь формально улучшив себя.
Что такое совершенствование регенерации, разгрузка генома, расширение диапазона воспринимаемых волн?
Идиоадаптации, жалкая попытка идти в ногу со временем.
Они создали нас "--- существ более совершенных, и их единственным разумным выбором была демонизация.
Поэтому мы победили.

"--*Этого можно было избежать, "--- пробормотал я.

"--*Нельзя, "--- отрезал Грейс.
"--- Эволюцию не остановить.
Если кто-то твёрдо решил остаться прежним, он станет жертвой.

Технолог поднялся с явным намерением уйти.

"--*И тебя устраивает роль хищника?
Паразита? "--- бросил я ему в спину.

Грейсвольд посмотрел на меня с жалостью, и я впервые за всё это время ощутил, какая между нами пропасть.

"--*Человеческий детёныш.
Что ты, что Тахиро, одни и те же вопросы.
И Анкарьяль от вас нахваталась.
Повторю специально для тебя "--- всё это время я жду момента, когда пути сапиентов и демонов разойдутся.
Я воевал ради этого "--- ради простора для учёных, ради того, чтобы оставить этот лоскуток Мира Фотона чуть лучше, чем он был при нас.
Я мечтаю о моменте, когда смогу пойти своей дорогой.
А вот ты, выношенный женщиной оцифрованный мозг, сможешь последовать за мной?
Сомневаюсь.

\section{[-] Два шага после смерти}

\spacing

Атрис с матерью давно умерли.
Но Безымянный, разумеется, не бросил свой народ и после смерти бренного тела.
Только сейчас\ldotst

"--*Истину тебе говорю! "--- горячился старый жрец в питейном доме.
"--- Ходит по джунглям парень с цитрой, и цветы от его игры распускаются!
А с ним женщина-северянка постарше чуть, в боевой раскраске чёрной, на Митхэ на нашу похожая!

"--*Хватит заливать, старик! "--- смеялись завсегдатаи.

"--*Светятся, как жуки-фонарики!
И ходят, и говорят о своём о чём-то, и смеются, как серебряные серьги звенят!
А я как Согхо кликнул "--- нет их, будто и не было! "--- кричал старик.

"--*Хлебнул, лехэ, лишка!
Иди проспись!

"--*Кормилица!
Милая!
Я Безымянного встретил и его женщину! "--- ребёнок десяти дождей дёргал кормилицу за платье.
"--- Она меня поцеловала в щёку!
У неё цветы в волосах сами растут!

"--*Ну что ты такое говоришь, Ликси, "--- испуганно забормотала женщина.
"--- Нельзя про богов выдумывать ничего.
Иди играй.

Слухи, конечно.
Один сказал, второй повторил.
Не станет Безымянный устраивать эти игры с иллюзией ради того, чтобы его увидел один старик или мальчик.
Да и превратить Митхэ в хоргета он не мог.
Аппаратура нужна, энергия.

Был, конечно, один вариант.
Безымянный мог интегрировать её личность в свою программу "--- бог стал бы двуединым.
Две личности, соединённые воедино программой плюс-сингулярности "--- неразлучные и вечно молодые.
Но это означает сбои, проблемы совместимости.
Не мог разумный хоргет пойти на такой жуткий риск.

Или мог?

\section{[-] А дом стоял}

\epigraph{
\mulang{$0$}
{И прошел ливень, и вздулись реки, и подул ветер, и обрушились на дом тот, "--- а он не рухнул, ибо основание его было на скале.}
{The rain came down, the streams rose, and the winds blew and beat against that house; yet it did not fall, because it had its foundation on the rock.}
}{Сапфировая Книга, Прядь Матеуша, 7:25}

\spacing

Я погладил резную колонну, украшенную змеями и ликами духов и ещё раз восхитился мастерством зодчего.
Война началась и окончилась, а дом по-прежнему стоял, как будто ничего не произошло.

Дверь подалась удивительно легко, гораздо легче, чем всегда.
Резные перила в виде разинувших рот змей, ступеньки "--- одна, две, три\ldotst тринадцать.

Очаг пылал.
Манэ и Лимнэ одновременно повернули головы ко мне.
Плетение, петли которого они секхар назад молниеносно перекидывали друг другу на пальцы, замерло.

"--*А я тебе говорила, что он придёт сегодня, "--- наконец сказала Лимнэ сестре.

\section{[-] Пирог}

\spacing

"--*А, это наш мужчина пришёл, "--- спохватилась Лимнэ.
"--- Согхо. Несмотря на имя, он немного неразговорчив, а сегодня ещё и устал.
Не тревожь его беседой, хорошо, Лис?

В зал зашёл мужчина чуть постарше меня, сел за стол и зачерпнул себе каши из горшка.
Я кивнул ему.

"--*Король-жрец, "--- поклонился мужчина и невозмутимо принялся за еду.

"--*Уже нет, "--- улыбнулся я.

"--*Это навсегда, "--- протянул мужчина.
Это были последние слова, которые я услышал от него за вечер.
Закончив еду, Согхо по очереди поцеловал Манэ и Лимнэ, затем ушёл спать в их комнату.

"--*Он хороший, "--- заверила меня Манэ.
"--- Кхотлам говорила, что каждой из нас лучше завести своего мужчину, но найти двух неразлучных мужчин сложно.

"--*Поэтому мы нашли Согхо, который любит нас так, словно мы "--- одна женщина, "--- подхватила Лимнэ.
"--- Это тоже большая редкость.

В конце трапезы Манэ поставила на стол красивый пирог, украшенный плетёнкой.

"--*Мы ведь пообещали\ldotst

"--*Что обещали? "--- удивился я.

"--*Сегодня первый день Тростника, "--- укоризненно сказала Манэ и указала на пирог.
"--- Неужели ты забыл, братик?

\chapter*{Интерлюдия последняя. Ярость Маликха}
\addcontentsline{toc}{chapter}{Интерлюдия последняя. Ярость Маликха}

\spacing

Дым выжигал глаза, огонь лизал волосы Маликха, оставляя обгоревшие чёрные клочья и лёгкий пепел, но воин не двигался с места.
На его руках, как спящий ребёнок, лежал Ликхмас.
Маликх легонько потряс друга за плечи, словно надеялся, что тот проснётся\ldotst

Вдруг взгляд Маликха упал на злополучный камень, лежащий в отрубленной женской руке.
Маликх схватил его и сжал до хруста в ладонях.
Из глаз воина брызнули слёзы, и он прошептал:

<<Ты не достанешься никому.
Никто и никогда тебя не найдёт!
Никто и никогда!>>

Маликх бросил кихотр.
Ярость и боль воина пылали жарче дома, в котором он остался.
Усиленные божественной мощью кихотра, они внезапно соединились с земным пламенем.
Огонь, окрасившись в ужасный синий цвет, как живой, накинулся на соседние дома.
Закричали заживо горящие в собственных жилищах люди.
Треск дерева и скрежет раскалённого камня сливались в жуткие повторяющиеся слова: <<Никто и никогда!
Никто!
И никогда\ldotse>>

Вскоре от города не осталось ничего, кроме голой седой земли с застывшими реками расплавленного камня.
Немногие выжившие, сбившись в угрюмые ватаги, сразу же отправились в другие поселения "--- на пепелище не осталось ни одного мертвеца, и не нашлось у людей хэситра, который можно было бы вылить в несуществующие рты.

\chapter*{Историческая справка}
\addcontentsline{toc}{chapter}{Историческая справка}

\textbf{Безопасная секция архива Ордена Преисподней\\
Выписка \#103:4AD-0000C}

~

Аркадиу Шакал Чрева через год безукоризненной работы в отделе культуры самовольно улетел с планеты.
Пятьдесят лет спустя он был обнаружен на планете Тра-Ренкхаль, где попытался возродить независимую сапиентную цивилизацию и создать союз урождённых сапиентов "--- Скорбящих.
Его планы были раскрыты агентами Ада.
Аркадиу был схвачен и осуждён.
Его сообщникам, среди которых был демиург планеты Митрис Безымянный, а также некоторым из Скорбящих удалось скрыться.

23.0004.34198 Аркадиу Шакал Чрева был казнён.

Исполнитель "--- Анкарьяль Кровавый Шторм.

Наблюдатели "--- Грейсвольд Каменный Молот, Стигма Чёрная Звезда.

Казнь Шакала послужила прецедентом к масштабному исследованию, в результате которого оцифровывание низших форм жизни было запрещено законодательством Ада, а все уже зачисленные в штат урождённые сапиенты подвергнуты кардинальной коррекции личности.
Отделом 100 была проведена усиленная чистка.

В настоящее время достоверно выявлены следующие связи персонажей с реальными личностями:

Тханэ ар'Катхар "--- Таниа Янтарь, интерфектор Скорбящих (ликвидирована).

Тхартху ар'Хэ "--- Тхартху Танцующая Тень, визор Скорбящих (ликвидирована).

Митхэ ар'Кахр "--- женская сущность бога Митриса Безымянного, мелиоратора Скорбящих (ликвидирована).

Атрис "--- мужская сущность бога Митриса Безымянного, мелиоратора Скорбящих (ликвидирован).

Митхэ ар'Тро "--- Микиа Седая, один из ведущих стратегов Скорбящих (в розыске).

Манэ ар'Люм "--- Маниа Нарисованная, диктиолог Скорбящих (в розыске).

Лимнэ ар'Люм "--- Лимниа Грустный Хвост, диктиолог Скорбящих (в розыске).

Записал: Кес Бледный Глаз

Проверили: Хара-лита Плачущий Клинок, Лимн Кард из клана Тахиро.

\part{Скорбящие}

\chapter*{Предисловие}
\addcontentsline{toc}{chapter}{Предисловие}

Это отвратительная книга.
И дело даже не в отсутствии у меня таланта к писательству.
Отвратительны по сути описываемые в ней события "--- так же отвратительны, как вскрытие гнойника, забой животного или прочистка засорившейся канализации.
Едва ли кто-то усомнится, что иногда это необходимо;
но искать в сделанном честь и тем более красоту бесполезно.
Происшедшее не спасают и не оправдывают даже те крупицы нежности и тепла, которые я постарался вложить в книгу.
И если кто-то другой, описывая то же самое, заикнётся о героизме "--- можете смело назвать его идиотом или лжецом.

\spacing

Здесь нужно дать кое-какие объяснения.
Демоны не сразу научились <<видеть>> сапиентов.
Долгое время связь с телом была единственным способом почувствовать Мир Фотона.
Чувствительность первых демонов оставляла желать лучшего "--- мы не могли засечь спящих, чрезвычайно сложной задачей было выслеживание животных.
Первые люди не могли нас уничтожить;
однако благодаря нашему несовершенству они дали нам достойный отпор.
Не так-то легко оказалось превратить в рабов тех, чьи предки были свободны на протяжении двадцати пяти тысячелетий.

\spacing

Возможно, кому-то история покажется неправдоподобной, но всё описанное здесь "--- чистая правда.
Это не легенда и не миф, всё происходило у меня на глазах.
К сожалению, моего друга, от имени которого ведётся повествование, уже нет в живых.
Но я вам скажу "--- то, что он дошёл до того финала, до которого дошёл, само по себе стечение обстоятельств с вероятностью, стремящейся к нулю.

Пусть вас не огорчает смерть Аркадиу.
Когда-то он сказал мне, что множество существ, встретившихся на его пути, живут в нём, и для многих это единственная жизнь, которую они имеют.
Аркадиу продолжает жить во мне.
А у меня, надо признаться, странное ощущение: я увижу смерть моей Вселенной.
Я, не самый достойный из её детей, буду наблюдать, как гаснут звёзды и галактики, как испаряются чёрные дыры, и мне даже доведётся сказать милой старушке последнее <<прощай>> "--- за миг до того, как тишина сменится молчанием.

Я не знаю, откуда это ощущение.
Но оно жило во мне с того самого момента, как я осознал свою конечность.

Радует одно "--- если я и доживу до столь почтенного возраста, то только по своей инициативе и никак иначе.
Читателям же, которые вынуждены уходить из жизни, едва переступив её порог, я скажу "--- относитесь к смерти спокойнее.
Это граница жизни, но границы есть у всего, что имеет название.
Наверное, именно поэтому я принял решение дописать творение Аркадиу "--- он не закончил историю, оставил своего героя на перепутье, пытающимся найти ответ на важные вопросы своей эпохи.

Я не питаю надежды, что кто-то поймёт мои желания и стремления.
Открывая книгу, раб видит то, что хочет, свободный "--- то, к чему готов.
Истину же не дано увидеть никому, даже автору "--- книга часто говорит больше, чем он хотел бы сказать, и даже больше того, что он осознаёт.
Я лишь надеюсь, что эти записи будут для мыслящего существа маяком в озерце его короткой, но важной жизни.

\emph{Искренне ваш, Г.}

\chapter{[:] Закат эпохи}

\section{[:] Закат}

"--*Красивый закат.

"--*Ага.

Огонь жадно пожирал дымящийся концентрат.
Где-то вдалеке вторила треском обсидиановая глыба, обрамлённая полуостывшей лавой.
Деловито попыхивал молодой вулкан Арашияма.
Поднималась Четвёртая Луна, похожая на непрожаренную каменную рыбу.
С сейсмологической станции уже сообщили, что скоро Арашияма взорвётся, принеся на значительную часть Преисподней незапланированную пепельную зиму.
Месяц пепла, сыплющегося с тёмных небес, затем ещё три месяца пахнущего серой снега.
И горе тому, кто попытается утолить жажду талой водой\ldotst

Вулканическая пустошь навевала тоску.
Жить там было не то чтобы невозможно, а скорее неспокойно.
Старые вулканы на берегах Разлома обильно удобряли и без того плодородную почву;
люди утаптывали ил и водорослевый перегной в любую расщелину, способную их удержать.
Молодые же устраивали неприятные сюрпризы вроде пирокластических потоков, вынуждавших аборигенов заселять только определённые возвышенности, а ещё регулярно затягивали небо тучами, проливавшими пепельный снег, неурожай и голод.
Возле Арашияма люди не селились.
По крайней мере, так полагали до недавнего времени.

Фигура, закутанная в чёрный плащ, нетерпеливо шевелила посохом дымящиеся блоки полимера.
Вторая, грузная и слегка неловкая, лениво ковыряла ножом землю.
Со стороны казалось, что разговор не клеится, но вряд ли это соответствовало истине.

"--*Лу, что по времени?

"--*Время есть.

И снова молчание.
Звезда медленно подошла к горизонту.

Фигура в чёрном вынула из складок плаща странное приспособление, похожее на металлический веер с раструбом.
Веер заплясал в дыму и зажужжал, то втягивая в себя дым, то выдувая его плотными длинными клубами.
Одно движение, два, три "--- и в темнеющие небеса отправилась вязь призрачных знаков.

Звезда, осветив дымовую надпись, села за горизонт, и в небе засияли фиолетово-красные змеи.
Они извивались, трепетали, как флаги на ветру.
Восхищённо крякнул толстяк, бросивший на время свой нож;
его товарищ откинул капюшон, обнаружив тугие белые кудри и гордое молодое лицо.
Оба, не отрываясь, вглядывались в сапфировые небеса, где разворачивался поистине божественный спектакль.
Но длилось это зрелище недолго "--- махнув хвостом, самый большой аспид рассыпался в воздухе жемчужным пеплом вслед за последним лучом Звезды.
Наступила ночь.

Улыбающиеся мужчины посмотрели друг на друга.

"--*Никак не могу к этому привыкнуть, Грисвольд, "--- смущённо признался молодой.

"--*Да, "--- вздохнул толстяк.
"--- Такая красота "--- и в таком месте, как Преисподняя.
Подумать только.
В моём родном мире было всё "--- и леса, и моря.
А такого не было.

"--*А ты сам откуда?

"--*Третья планета системы IF-517, <<Фомальгаут>> по-старому, "--- Грис махнул рукой куда-то в сторону горизонта.
"--- Меня туда послали ещё первые люди.

"--*Первые люди? "--- ахнул Лу.
"--- Сколько ж тебе тогда\ldotsq

Грисвольд горько усмехнулся и вместо ответа снова махнул рукой.

"--*Так ты бог.

"--*Творец и владыка, "--- пошутил толстяк.
"--- Ладно, хватит болтовни.
Отдых закончен.
Ждём ответа, твой квадрант "--- номер четыре.

"--*Грис, а как ты сюда попал?
И что у тебя за странное имя?

"--*Потом, "--- в голосе Грисвольда прозвучала сталь.

Крякнув, Лу сел на камень и принялся посохом ворошить непрогоревший полимер.

"--*Напомни, Грис, зачем мы торчим в этой неуютной обгорелой пустоши?

"--*Мы ждём сигнала, "--- недовольно буркнул толстяк.
"--- Если повезёт, люди ответят и мы сможем узнать местоположение следующей явки.
Твой квадрант "--- номер четыре.

"--*И это приблизит нас к обнаружению бункера, "--- голос Лу был предельно насыщен сарказмом.

"--*Лу, давай не будем обсуждать приказ твоего отца, "--- устало сказал Грис.
"--- Хотя бы сейчас.
Арракис сказал "--- использовать все имеющиеся способы, и это один из них.
Твой квадрант "--- номер четыре.

Лу недовольно замолчал.

Дррррр\ldotst "--- странный крик, похожий на птичий, разнёсся по выжженной каменной пустыне.
Друзья даже не изменили позы "--- только левые руки почти синхронно вынули пулевые пистолеты.

"--*Не двигайся, "--- шёпотом предупредил Грисвольд напарника.

Минута прошла в томительном ожидании.

"--*Проверим? "--- шёпотом предложил Лу.
"--- Сидим, как чайки на столбах.
Давно бы сняли, если б захотели.

"--*Я посмотрю.
Подержи меня.

Лу аккуратно сел рядом с толстяком и обхватил его за пояс.
Грисвольд на секунду закатил глаза и обмяк.
Парень крякнул под весом навалившейся на него туши.

"--*Извини, "--- пробормотал толстяк и отряхнулся.
"--- Тело Самаэла пустое.

"--*Кто это сделал?

"--*Не нашёл.

"--*Что?
Как не нашёл? "--- забывшись, парень заговорил во весь голос.
"--- Так это не\ldotsq

"--*Люцифер, тихо, волки тебя дери!

"--*Извини, "--- прошептал Лу.
"--- Он жив?

"--*Не знаю.
Уходим, сейчас же.
Про костёр забудь.

\section{[:] Мефистофель}

Дорога до лагеря лежала по склону Арашияма.
Извержение отгремело только вчера, и от тускло светящейся лавы за пять ярдов разило нестерпимым жаром.
Люцифер тихо, сквозь зубы ругался "--- обычный на вид камешек на дороге в мгновение ока проплавил ботинок, выставив наружу, на съедение волнам огня голые пальцы.
Толстяк молча прыгал с одного валуна на другой, выбирая самые холодные места.

"--*Стоять.

Голос прозвучал как будто из ниоткуда.
Два путника подобрались, но тут же расслабились.
Свои.

\mulang{$0$}
{"--*\emph{Плесень и мрак.}}
{``\emph{Darkness and mold.}}
Грисвольд-тринадцатый, Люцифер из клана Мороза, "--- прошелестел Грис.

Из почти не заметной с дороги пещерки вышел некто, закутанный в чёрный плащ.
В руке тускло блеснул пистолет.

\mulang{$0$}
{"--*\emph{Соль и табак.}}
{``\emph{Tobacco and salt.}}
Мефистофель из клана Мороза.
Как дела?

"--*Самаэл сбежал или был убит, причина и исполнитель неизвестны.

"--*Самаэл жив, отчитывается в лагере.
Транспорт в пещере.

"--*Арракис велел тебе встретить нас? "--- удивился Грисвольд.

"--*Его, "--- лаконично ответил Мефистофель, кивнув на Лу.

Мгновение спустя из пещеры выкатился маленький экспедиционный мотоцикл.
Мефистофель жестом велел друзьям садиться.
Худенький Лу поморщился, когда его зажало между каменной спиной Мефистофеля и животом Грисвольда, но промолчал.
Мефистофель, первая генерация клана Мороза, был напрочь лишён чувства эмпатии, и просьба потесниться в лучшем случае вызвала бы лишь холодный вопрос <<Зачем?>>.

Минуту спустя мотоцикл нырнул в густую завесу пара, и Арашияма скрылся из виду.

"--*Так что там? "--- поинтересовался Грисвольд.

"--*Вторжение минусов, "--- коротко бросил Мефистофель.

Люцифер вздохнул.
Больше информации не будет "--- иначе Мефистофель бы всё рассказал сразу.
Здесь начиналась его работа "--- работа стратега.

"--*Мало нам было этих подземных крыс, "--- выругался парень.
"--- Хорошо ещё, что их технологии не позволяют убить демона\ldotst

"--*Их технологий вполне хватает, чтобы портить нам кровь последние десять лет, "--- возразил Грисвольд.
"--- Где бункер?
Мы так ничего и не узнали.

"--*Я сообщу Арракису, что эксперимент был прерван по веской причине и в санкциях нет нужды, "--- сказал Мефистофель.

"--*Тысяча благодарностей, ты просто эталон доброты, "--- саркастически проворчал Грисвольд.
"--- Что нового насчёт диверсантов?

"--*Есть данные, что они начали ходить по селениям и обучать дзайку-мару\footnote
{То же, что сейхмар. \authornote}
распознавать инкарнатов, "--- сообщил Мефистофель.
"--- Команда Гало поймала одного агента.
Точнее, раздобыла его труп "--- он самоуничтожился ещё до начала допроса.

Мефистофель вывел мотоцикл на утоптанную дорогу и выжал полную скорость.
Друзьям пришлось замолчать.

\section{[:] Доверие}

"--*И ещё одна неудача, "--- Грисвольд тяжело опустился на скамью и забросил на спину полотенце.
"--- Третья по счёту.

"--*Мы с самого начала занимались ерундой. "--- буркнул Лу.
"--- Почему нельзя подключить этих тама?
Они неплохо показали себя при поиске руды.

"--*Арракис сказал "--- людям доверять нельзя, "--- ответил Грис.
"--- И в этот раз я с ним согласен.

"--*При чём здесь доверие?
Нам с Гало нужны любые данные.
Верифицировать их не составит труда.

"--*Тем, в бункере, тоже нужны данные.
И если все получат то, что хотят, то ситуация никуда не сдвинется.

Лу явно хотел возразить, однако, сделав выразительное движение бровями, промолчал.
<<Это ошибочное суждение, но и у тебя, и у отца недостаточно знаний, чтобы понять его ошибочность>>.

"--*Отец не хочет даже попробовать.
Я предлагал ему разбить задачу на части, не вызывающие подозрения, и дать людям только их.

"--*Лу, ты у нас глава Ордена?

"--*Отец сделал меня и Гало стратегами.
Зачем мы ему нужны, если в итоге решение принимает он, и далеко не всегда лучшее?

"--*Это и есть руководство.
Учесть мнение каждой стороны и вынести вердикт.

"--*Тогда он плохой руководитель.

"--*Не вздумай ляпнуть такое при легионе.

"--*Ты меня за идиота считаешь?

"--*Лу, нас очень мало, "--- тихо сказал Грис, оглянувшись по сторонам.
"--- Куда бы ни вёл нас Арракис, он ведёт нас всех.
Всех, понимаешь?
Делить силы "--- худший расклад из всех.

"--*Не надо пересказывать мне эти пропагандистские формулы.
Я мог бы курировать людей сам, не подключая ни одной единицы из сил Ордена!

"--*Тогда что тебе мешает? "--- раздражённо бросил Грисвольд и отвернулся.

Лу изумлённо смотрел в спину толстяка.

"--*Хватит прожигать дыру в моей спине, "--- пробурчал технолог.
"--- Хочешь "--- делай.
Я молчок.

\section{[:] Переговоры с наблюдателями}

"--*И тем не менее Тысяча Башен не совсем понимает, в чём сложность борьбы с восставшими людьми.

"--*Основная проблема в том, что этот бункер специально разрабатывался для борьбы с демонами, "--- объяснил Лу.
"--- Как вы знаете, Преисподняя не всегда была выжженной вулканической планетой, когда-то здесь было спокойно и даже зелено.
Когда глава нашего клана, Арракис, прибыл с Мороза вместе с Дорге, здесь шла война между колонистами с Земли и демиургом.
Люди использовали всю мощь имеющихся технологий, чтобы лишить демиурга масс-энергии, и заставили его уйти.

"--*Мы слышали о том, что Преисподняя была выжжена демиургом в отместку, "--- полувопросительно сказал наблюдатель.

"--*Едва ли, "--- подал голос Грисвольд.
"--- Для демиурга причинить вред планете "--- всё равно что отрубить себе руку.

Наблюдатель непонимающе уставился на технолога.

"--*В общем, для большинства это неприемлемо, "--- поправился Грисвольд, оглядев бионические конечности собеседника.
"--- Наши учёные считают, что это сделали люди.
Но лично я, как имевший опыт управления планетой, склоняюсь к тому, что биолитосфера не была стабилизирована и без контроля демиурга развалилась сама.

"--*И вот через тысячу лет, когда люди уже растеряли большую часть технологий, а мы взяли власть, бункер был снова кем-то найден, "--- сообщил Самаэл.
"--- Видимо, там же были найдены старые записи, которые и объяснили людям, с кем они имеют дело.

"--*Мы даже не уверены, что бункер в окрестностях Арашияма один-единственный, "--- продолжил Гало.
"--- Как не можем быть уверены, что таких сооружений нет где-либо ещё на планете.
Методология первых людей была отточена до совершенства, и современные дикари следуют ей неукоснительно, словно священным догмам.
Именно поэтому нам нужны ваши демоны.

"--*Разумеется, "--- улыбнулся наблюдатель.
"--- Мы прекрасно понимаем, что эти сооружения должны быть найдены и изучены со всей возможной тщательностью, ведь это "--- залог победы над Союзом.
Орден Тысячи Башен был бы чрезвычайно рад помочь вам и получить данные касательно технологий первых людей.

Айну поклонилась.

"--*Однако, к сожалению, у нас нет свободных демонов для выполнения этой задачи, "--- закончил наблюдатель.
"--- Тот резерв, который был подготовлен к работе с вами, пришлось срочно задействовать на Тысяче Башен.

"--*То есть вы отказываетесь нам помогать? "--- осведомился Люцифер.

"--*Юноша, "--- начал наблюдатель, "--- вы превратно истолковали наши слова.
<<Отказываемся помогать>> и <<не имеем свободных демонов>> "--- разные вещи.

"--*Да что вы говорите, "--- буркнул Грисвольд.

"--*По вашим указаниям мы приготовили вам всё, включая тела, а вы говорите, что не имеете свободных демонов? "--- вспылила Айну.
"--- Чем это занят Орден Тысячи Башен, что свободных демонов нет?

"--*Вы можете получить любую информацию о наших занятиях по официальному запросу, "--- сухо ответил наблюдатель.
"--- Я лишь сообщаю решение нашего штаба.

<<Ага, "--- хмыкнул Гало.
"--- Смиренный служитель, как же.
Мы как-то протестировали троих на полномочия.
Его полномочия соответствуют как минимум нашим с Лу, а молчун с краю, чьё имя я даже не припомню, вполне может быть вторым лицом в Ордене>>.

<<И сейчас, когда тайное убежище стало полем битвы, эти корольки решили аккуратно сбежать, провоцируя нас на высылку наблюдателей, "--- подтвердил Лу.
"--- Действуйте как угодно жёстко по своему усмотрению, но ни в коем случае не вздумайте их отпускать>>.

Айну и Самаэл кивнули.

"--*Мы обязательно сделаем официальный запрос, "--- сухо сказала Айну.
"--- Я прошу прощения за последние слова, вы "--- отличные исполнители.
И тем не менее, как я полагаю, у вас должны быть готовые объяснения отдельных аспектов.

"--*Задавайте вопросы, и я отвечу, "--- неприятно улыбнулся наблюдатель.

"--*Были ли нападения со стороны Союза на Тысяче Башен? "--- начал Самаэл.

"--*Они и сейчас есть.
Не настолько серьёзные, как здесь, но тем не менее оттягивающие значительную долю наших сил.
Вот, пожалуйте.
Полный расклад\ldotst

<<Ложь, "--- ухмыльнулся Лу, просмотрев данные.
"--- Нападения они пресекли в корне, так как имели сношения с Союзом втайне от нас и получали разведданные от завербованных минусов>>.

<<Выкладка точна?>> "--- спросила Айну.

<<Подтверждаю.
Статистика по потерям, "--- кивнул Гало.
"--- Они их завысили, но тем не менее иначе как грамотно и, главное, вовремя собранными разведданными такое не объяснить>>.

<<В общий котёл отправьте информацию по руководителями их разведки, "--- добавил Арракис.
"--- Мне совсем не нравится, что они на шаг впереди>>.

<<А также по тому, насколько свободны они в своих действиях>>, "--- вставил Лу.

Арракис кивнул.
Если он и понял намёк сына, то предпочёл этого не показывать.

"--*Были ли попытки установить диалог с Союзом?

"--*Были.
Берен, предоставь записи.
Как видите, мы даже попытались заключить мир, но уже во время подписания пакта последовала серия диверсий.

<< \dots на Преисподней, "--- закончил Гало.
"--- Настолько виртуозная ложь, что неотличима от правды.
Они каким-то образом откупились от Союза, и я бы многое отдал, чтобы узнать, каким>>.

<<И оба раза они рассчитывали на то, что ложь будет раскрыта, "--- добавил Лу.
"--- Цель та же "--- высылка наблюдателей>>.

<<О как, "--- хмыкнула Айну.
"--- То есть у нас ещё один возможный враг, который так и нарывается на кинжал>>.

<<Не совсем>>, "--- возразил Гало.

<<Скорее это переход от сотрудничества к вооружённому нейтралитету, "--- пояснил Лу.
"--- Возможно, это и было предметом договора с Союзом: Орден Тысячи Башен не будет вмешиваться в конфликт и примкнёт к победителю>>.

<<Дьявол, мы не можем даже отозвать наших демонов с Тысячи Башен, "--- посетовала Айну.
"--- Потому что рычаг давления на Орден ну никак не лишний>>.

<<И они об этом знают, "--- закончил Гало.
"--- Вполне возможно, что они даже рассчитывают на остатки адских сил после нашего поражения.
Больше-то легионерам некуда будет идти>>.

<<Лично мне всё ясно, "--- буркнул Арракис.
"--- Самаэл, задай им для приличия ещё пять-десять умных вопросов, сохраняя дружественно-требовательный тон.
Люцифер, создавай конференцию прямо здесь "--- нам срочно нужен план.
И ещё "--- пусти параллельно слабозашифрованную меченую легенду.
Посмотрим, где у нас протечка>>.

\section{[:] Условия и ответ}

\spacing

Айну, как обычно, сорвалась с места в карьер.

"--*Грисвольд, Люцифер, Мистраль, Гало.
Вы идёте со мной.
База в Такэсако разрушена.
Нужно найти и унести уцелевшее оборудование.

"--*Прости, что значит <<унести>>? "--- поинтересовался Гало.

\mulang{$0$}
{"--*Это значит, что ты возьмёшь всё барахло руками, взвалишь себе на плечи и ножками дойдёшь сюда, в лагерь.}
{``It means: all the garbage we can find you'll pick up by your hands, carry on your back, then return here by your feet.''}

"--*То есть узел дозаправки транспорта в Акияма также разрушен? "--- уточнил Лу.

Айну промолчала.

"--*Я стратег, а не носильщик, "--- тихо сказал Люцифер.

"--*Если не будешь носильщиком добровольно "--- станешь трупом принудительно, "--- посулилась Айну.
"--- Союз Воронёной Стали об этом позаботится.

"--*Я смотрю, дела у Ордена Преисподней совсем плохи, "--- масляно улыбнулся один из наблюдателей.

"--*Поблагодари меня, что пока ещё можешь смотреть, "--- сказала Айну.

"--*Вы же понимаете, что у ваших слов и поступков могут быть последствия? "--- осведомился наблюдатель.

"--*Если ты сейчас же не закроешь рот, это поймут все и сразу.
До мельчайших подробностей, "--- демон Айну предупредительно вспыхнул, приведя в готовность боевые модули.
Наблюдатель побледнел и вытянулся в струнку.

"--*Если Орден Преисподней переживёт эту схватку, Тысяче Башен придётся понять и принять очень многое.
Или ответить, "--- пояснил Люцифер, поняв, куда дует ветер.

На этот раз побледнели все наблюдатели.
Айну одобрительно посмотрела на юношу.

"--*Малышня, я вас жду через минуту.
Из оружия возьмите только пистолеты.
Если нас там встретят "--- оно в любом случае не поможет.

\section{[:] Тела на обочине}

\spacing

Люцифер поморщился, Айну подняла бровь, а Грисвольд почувствовал, что у него задрожали ноги.
Столбы вдоль дороги были увешаны человеческими телами без кожи.

"--*Ну и вонь, "--- сказал Люцифер и, достав респиратор, манерно прикрыл им нос и рот.
"--- Кто это?

"--*Точно не мы, "--- констатировал Гало.
"--- И не Союз.
Демоны не стали бы тратить время на такие глупости.
Кожу сняли уже после смерти.

"--*Проведи анализ, брат, "--- попросил Люцифер.

Гало подошёл к трупу и внимательно осмотрел его со всех сторон, затем обнюхал подсушенную горячим ветром плоть.
Демон стратега издал слабое свечение.

"--*Точность 0.904, "--- сказал Гало.
"--- Это сделали дзайку-мару.
Убивали и снимали кожу методично, без торопливости.
Многие убиты во сне.
Я думаю, что это вендетта, хотя впервые вижу, чтобы с трупами врагов так поступали.

"--*Что могло спровоцировать вендетту? "--- удивилась Айну.

"--*Союз использовал сапиентные тела, "--- предположил Люцифер.
"--- Маловероятно, что они следовали всем местным обычаям и традициям.
Дзайку-мару не видят демона, они видят тело и мстят телу.

"--*Я согласен с Лу, "--- вмешался Грисвольд.
"--- Такие казусы регулярно случались на моей планете, если туда забредал очередной нейтрал-неумеха.
Правда, чаще это заканчивалось смертью его тела.

Друзья переглянулись.

<<Это твои тама?>> "--- иронически спросил Грис.

<<Своё дело они сделали, "--- ответил Лу.
"--- Данные у меня, планирую подсунуть их в общее поле как случайную находку>>.

<<Главное, что ты ни о чём не жалеешь>>.
Лу почувствовал в словах технолога холодок.

"--*Кстати, "--- добавил Гало веселее, чуть обычно.
Все навострили уши.
"--- Что вы думаете насчёт погоды?
Я чувствую слабый привкус цветочной влаги.

<<Есть вероятность, что за нами идёт лазутчик "--- неуточнённое сапиентное тело.
Местоположение по эманациям определить не могу, но чувствую иные неуточнённые признаки присутствия>>.

"--*Ты оптимист, Гало, "--- рассмеялась Айну.
"--- Погода не изменится.

<<Вести себя как обычно, следовать тем же курсом>>.

Айну кивком подтвердила приказ, и демоны, в последний раз посмотрев на страшную картину, двинулись дальше.
Скрипели и ворчали под ветром сохнущие верёвки.
Трупы махали вслед демонам мёртвыми руками, как махали всем прочим путникам, имевшим несчастье пройти мимо.

\section{[:] Мальчик с мечом}

\epigraph
{Счастье "--- это когда из кошмара можно проснуться.}
{Пословица сели}

Через десять миль начало сказываться недосыпание.
Демоны изо всех сил пытались заставить свои тела идти, как обычно, но вскоре перестали помогать даже психостимуляторы.
Грисвольд вздыхал, Гало злобно пинал мелкие камни, Айну мрачно молчала.
Люцифер, который передал свой груз технологу, выглядел бодрее всех.

"--*Чтоб эти тела, "--- бормотал Гало.
"--- Почему дзайку-мару не избавились от такой бесполезной вещи, как сон?

"--*Во время сна мозг восстанавливает метаболический потенциал, "--- заметил Люцифер.
"--- А ещё сон "--- это дар видеть желанный мир.
Или хотя бы не видеть нежеланный.

"--*Ещё немного, и я брошу эту тушу на дороге, "--- пообещал Гало.
"--- Я ей не пастух.

Лазутчик не показывался.
Айну пару раз наудачу просмотрела местность, но неведомый сапиент каким-то образом прятался от <<взгляда>> интерфектора.
Грисвольд предположил, что он искусственно ввёл себя в ступорозное состояние.

Тропинка круто свернула к северу, огибая скалистый холм.
Вскоре Гало, поколебавшись, попросил Грисвольда понести и часть его груза.
Технолог тяжело вздохнул и согласился.
Айну хотела сделать братьям строгое внушение, но Грисвольд покачал головой.
<<Не до этого>>.

Ветер переменился, и смешанный с дымом пар унесло в сторону.
В надвигающихся сумерках резко проступили очертания холма.

"--*Смотрите туда, "--- Люцифер, не оглядываясь, махнул на скалу.
"--- Вон он, наш лазутчик.

Сонная дымка мгновенно улетучилась.
Демоны, остановившись, как будто невзначай встали в боевой порядок.

На скале стоял мальчишка.
Он опирался на дайту-соро\footnote
{Дайту-соро "--- ритуальное оружие Древней Земли, длинный изогнутый меч из стабитаниума.
Народ нихон, отправляя своих людей в космические путешествия, обязательно давал им дайту-соро как талисман. \authornote}
в ножнах, словно на посох.
Рваные лохмотья, едва прикрывавшие тело, развевались на сухом горячем ветру.
Мелкие острые камешки то и дело секли плоское смуглое личико, но мальчишка стоял и продолжал смотреть.
Ему было всего шесть-семь стандартных лет, но он держался, словно завоеватель на пьедестале.

"--*Шпион, как пить дать, "--- констатировала Айну, подтягивая ремешки куртки.
"--- Гало, подпусти его поближе и потяни время, я устала и не хочу бегать по этим скалам.

"--*Эй! "--- закричал Гало, и эхо многократно повторило его крик.
"--- Мальчик!
Подойди!

Мальчишка, поколебавшись, перехватил поудобнее оружие и начал спускаться.

"--*Дзайку-мару, "--- доложила Айну.
"--- Как подойдёт на пять шагов, стреляйте в голову, кому там сподручнее, и пошли дальше.

Мальчишка спрыгнул с небольшого уступа и замер в шести шагах.
Айну достала пистолет, Люцифер и Гало замерли с руками на кобурах.

"--*И что это такое? "--- осведомилась Айну, заметив смятение товарищей.

"--*Ты ж сказала "--- в пяти шагах, "--- буркнул Гало.

"--*Я сказала "--- кому сподручнее, "--- заплетающимся языком поправила Айну.
"--- Вы идиоты.
Может, вам расстояние сообщать в виде диапазона допустимых значений?
С точностью до микрометра?

"--*Пристрели мальчишку, чтоб тебя, "--- устало бросил Грисвольд.
"--- Все хотят спать, а мы уже две минуты здесь торчим.

Пока демоны разговаривали, маленький оборванец подбежал к Айну и обнял её ноги.
Айну приставила пистолет к его виску.
Мальчишка поднял улыбающееся лицо, его огромные карие глаза взглянули на демоницу.

"--*Ты красивая, ситу-ну-онна\footnote
{Ситу-ну-онна "--- на Преисподней: женщина, владеющая оружием и боевыми искусствами. \authornote},
"--- сказал мальчик.
"--- Когда я вырасту, ты будешь моей женой.

"--*Хорошо, что не сейчас.
Хоть отдохну ночью, "--- усмехнулась Айну и как ни в чём не бывало спрятала пистолет.
"--- Что ты здесь делаешь, ребёнок?
Шпионишь за нами?

"--*Да, "--- признался мальчик.
"--- Вы похожи на великих воинов из сказок.
Я следил за вами целых три часа.

"--*Айну, "--- буркнул Гало и вытащил оружие.
Айну жестом остановила его, демон нехотя подчинился.

<<Если забрызгаешь мне комбинезон, стирать заставлю вручную>>, "--- добавила демоница.

"--*Как ты скрылся от нас? "--- поинтересовался Грисвольд.

Мальчишка бросил перед демонами автоматический шприц с остатками нейролептика.

"--*Я дружу с детьми из бункера.

Только сейчас Айну заметила в улыбке мальчика что-то искусственное.
Глаза оставались бесстрастными и отстранёнными.
Когда мальчишка вытянул руку, чересчур короткий рукав изорванной куртки соскользнул, и демоны увидели корявые детские иероглифы, вырезанные ножом на коже запястья:

<<ИДИ ЗА НИМИ>>.

Грейсвольд вздрогнул.
Нейролептик напрочь подавлял мотивацию.
Людям требовались годы, чтобы научиться действовать в ступоре, в условиях минимальной эмоциональной стимуляции.
Для этого нужно было чёткое понимание, что и зачем ты делаешь.
Но ребёнок\ldotsq
Грейсвольд мысленно подивился силе его воли.
Он знал, что людям труднее всего идти против собственных эмоций.
Или против их отсутствия.

"--*Ты не желаешь нам зла, теперь я вижу, "--- сказала Айну, заворожённо разглядывая кровавую надпись на детской руке.
"--- Мы не можем долго с тобой беседовать.
Возвращайся домой.

Ножны на поясе Айну расстегнулись, клинок выдвинулся на дюйм.

"--*Я пойду с вами, "--- без раздумий выпалил мальчишка.

"--*Какой смелый.
В таком случае "--- вперёд, "--- скомандовала Айну, безо всяких фокусов убрав клинок и поправив заплечный мешок.

<<Ты с ума сошла?>> "--- удивился Грисвольд.

<<Да хватит уже, "--- отозвалась демоница.
"--- Убивать надо сразу, а мы театр устроили>>.

\mulang{$0$}
{"--*Зачем ты носишь с собой эту бесполезную палку? "--- спросила Айну.}
{``Why do you carry this useless stick?'' Ain\"{u} asked.}
\mulang{$0$}
{"--- Брось.}
{``Drop it.''}

\mulang{$0$}
{"--*Нет, "--- просто ответил мальчик.}
{``Nope,'' the boy answered simply.}
\mulang{$0$}
{"--- Это фамильная ценность, и он старше вас всех вместе взятых.}
{``It's a family heirloom, and in addition it's older than all of you combined.}
\mulang{$0$}
{Проявите уважение.}
{Show respect.''}

\mulang{$0$}
{"--*Дело твоё, "--- так же просто согласилась Айну, отметив своё полное нежелание ему отказывать.}
{``It's on you,'' Ain\"{u} agreed in the same manner, thinking of her own reluctance to deny.}
\mulang{$0$}
{"--- Лу, ты за него отвечаешь.}
{``Lu, you're in charge of him.''}

Люцифер хмуро посмотрел на демоницу, потом на неожиданного подопечного.
Тот, недолго думая, перебросил ремешок дайту-соро за плечи, подошёл к демону и бесцеремонно запрыгнул ему на закорки.
Хрупкий Лу охнул и едва не оступился.

"--*Вези меня, "--- оскалился мальчишка.

"--*Когда придём, я высплюсь, а утром лично тебя прирежу, "--- полузадушенным голосом посулился Люцифер.
\mulang{$0$}
{"--- Следи за своей уважаемой железкой, она меня по заднице бьёт.}
{``Watch your respectable piece of iron, it's been hitting my ass.''}

\mulang{$0$}
{Впрочем, угроза была пустой.}
{This threat, however, was empty.}
Мальчик успел заинтересовать Люцифера так же, как и очаровать Айну.

Отряд двинулся дальше.
Айну шла молча.
Люцифер, угрюмо везущий на плечах мальчишку, тоже молчал.
Грисвольд уже не дышал, а пыхтел под тяжестью рюкзаков.

Мальчик начал выходить из ступора.
Грисвольд несколько раз бросал взгляд на его лицо "--- в нём оставалось всё меньше искусственности.
Бродяжка начал проявлять неподдельный интерес к спутникам и грузу, который они несли.

"--*Зачем ты носишь ножницы в ножнах? "--- спросил он Лу.

"--*Потому что они почти то же самое, что нож, но ими сложнее порезаться, "--- буркнул Лу.

"--*Это выглядит смешно!

"--*Смешно выглядит твоя палка.
Ею даже людей сложно резать, хотя разрабатывалась она именно для этого!

Вскоре инициативу в разговоре взял Гало.

\mulang{$0$}
{"--*Где ты научился так притворяться?}
{``Where did you learn to pretend like that?''}

\mulang{$0$}
{"--*Это называется <<быть вежливым>>, "--- просветил демона мальчик.}
{``That's called `to be polite','' the boy enlightened the daemon.}
\mulang{$0$}
{"--- В нашем доме бывали разные люди, но мать учила, что хозяин должен быть вежливым со всеми гостями.}
{``Our home was visited by all kinds of people, but mother taught me that a master should be polite to all the guests.''}

\mulang{$0$}
{"--*Ты считаешь себя хозяином?}
{``You call yourself \emph{a master}?''}

\mulang{$0$}
{"--*Вы определённо прибыли издалека, "--- улыбнулся бродяга.}
{``You've certainly come from far away,'' the tramp smiled.}
\mulang{$0$}
{"--- А я здесь живу с рождения.}
{``Unlike me, who was born here.''}

"--*Ты знаешь правила знакомства, дзайку-мару?

Мальчишка знал, что словом <<дзайку-мару>> бродячие торговцы называли всякую мелочь для ремёсел "--- заклёпки, полоски ткани, нитки, "--- но не повёл бровью.

"--*Знаю, \emph{хорохито}, "--- сказал он.
"--- При первой встрече люди, если они не враги, кланяются, называют свой род, родовые девизы и имена.

"--*И почему ты не следуешь этим правилам? "--- ядовито осведомился Гало.
"--- Скажи своё имя.

\mulang{$0$}
{"--*Поклонись, "--- парировал мальчишка.}
{``Take a bow,'' boy retorted.}
Его голос был непринуждённым, но чувствительный к нюансам интонации слух демонов заставил всех изумлённо посмотреть на маленького спутника.
Гало зашипел.
Он тоже услышал оскорбительный ультиматум.

"--*Если бы ты не висел на шее моего брата, я бы нафаршировал тебя пулями.

Айну расхохоталась.
Грисвольд хихикнул и тут же закашлялся.

\mulang{$0$}
{"--*Это что-то, "--- тяжело дыша, признал технолог.}
{``Amazing,'' Griswold declared gasping for breath.}
\mulang{$0$}
{"--- В такую глупую историю я ещё не попадал.}
{``I’ve never got into such a stupid situation.}
\mulang{$0$}
{Хорошо, доиграем сценку до конца.}
{So, let’s act a play to the end.''}

Грисвольд приблизился к Люциферу и поклонился, насколько позволяли быстрый шаг и тяжёлый груз.

\mulang{$0$}
{"--*Я Грисвольд из рода\ldotst эээ\ldotst Грисвольда.}
{``I’m Griswold, belong to\ldotst mmm\ldotst Griswold house.}
\mulang{$0$}
{<<Добрый Грис "--- лентяя приз>>.}
{\emph{We're a prize for lazy guys.}''}

\mulang{$0$}
{Мальчик церемонно кивнул толстяку.}
{Boy ceremoniously nodded his head to the fat man.}

\mulang{$0$}
{"--*Я из рода Ханаяма.}
{``I belong to Hanayama house,'' the boy told.}
\mulang{$0$}
{<<Благородная кожа "--- лучшее одеяние>>.}
{\emph{``Noble skin is the best suit.''}}

"--*Случаем, не твоих родичей освежевали во время недавней вендетты? "--- осклабился Гало.
"--- Кто бы это ни сделал, они не лишены чувства юмора.

"--*Я отвечаю только за свою кожу, "--- заметил мальчик.
"--- И она пока на мне.
Грисвольд мог бы передать часть груза мальчику.
Я понесу один рюкзак, а длинноволосый понесёт меня.

\mulang{$0$}
{"--*Иди ты в дупло с такими идеями, "--- пробормотал Люцифер.}
{``Go to hollow with such ideas,'' Lucifer muttered.}

Айну обогнала Люцифера и тоже поклонилась мальчику.
Бродяга поднял огромные живые глаза на женщину и улыбнулся белозубой улыбкой.

"--*Тахиро.

Айну открыла рот, чтобы назвать своё имя, но Люцифер не выдержал:

"--*Может, вы прекратите переговоры на моём горбу?

\section{[:] Сладкое безделье}

\epigraph
{Сотрудничество с сильным соперником "--- хороший способ отступления.}
{Пословица Преисподней}

Люцифер сладко потянулся в своей постели.
Вчерашний план увенчался полным успехом.
На указанный путь, по которому прошли два стратега, технолог и интерфектор, слетелись аж одиннадцать диверсионных групп.
Все они попали в засаду и были уничтожены "--- почти одновременно.

<<Как же я люблю ничего не делать, "--- лениво думал Люцифер, смакуя утреннюю негу.
"--- А вот быть приманкой "--- не очень>>.

Люцифер обладал потрясающим умением наслаждаться отдыхом, зная, что нежиться ему оставалось считанные минуты.
График у демонов по-прежнему был плотный.

Лу перевернулся на другой бок и закрыл глаза.
Вскоре его вывел из забытья лёгкий спазм в мышцах "--- забавный сигнал, говорящий о готовности мозга погрузиться в сон.
Лу уже даже засунул под прохладную подушку разгорячённую руку и настроился на продолжение отдыха, как зажужжал браслет на левой руке.

<<Как обычно>>, "--- философски заключил Люцифер и, сев на кровати, начал с наслаждением тереть глаза, зевать и потягиваться.
Со стороны это выглядело вопиющим бездельем "--- если не знать, что стратег специально выделял для потягиваний, зевания и утренней мастурбации восемь минут пятьдесят секунд драгоценного времени.
На личную гигиену всем выделялась двадцать одна минута, но специально для Лу Арракис сделал исключение;
он мог заниматься своим телом аж тридцать пять минут и двадцать секунд "--- с учётом душа, антибактериальной обработки тела, нанесения ухаживающих средств и небольшого количества макияжа.
Исключительное право Лу выбивал потом и кровью.
Ему пришлось собрать внушительную статистику, чтобы доказать "--- при ухоженном теле его демон работает гораздо эффективнее.
И даже это бы не помогло, если бы на сторону парня совершенно неожиданно не встала Айну.

"--*Идите вы в дупло со своей дисциплиной, "--- заявила она Гало и Арракису.
"--- Вам же ясно показали, что результат лучше.

"--*Сегодня Лу, а завтра "--- все легионеры начнут минуты требовать? "--- возмутился Гало.

"--*Я им всем лично надушу подмышки и накрашу глаза, если это повысит эффективность, "--- рявкнула Айну, выйдя из себя.
"--- В общем, или Лу получает свои минуты, или ищите другого императора\footnote
{Император "--- ранг и должность в раннем Ордене Преиподней, соответствующие максиму терция и командующему планетарными вооружёнными силами соответственно. \authornote}.
Мне нет резона работать с демонами, ставящими методы превыше результата.

Ещё одна хорошая новость поступила за завтраком.
Наблюдателей, сливших информацию Союзу, вывернули наизнанку.
Тысяча Башен безоговорочно приняла условия Преисподней "--- на вулканическую планету прибыли две трети их высшего командования и почти все учёные, аккурат в приготовленные для интерфекторов тела.
Фактически это означало объединение двух организаций в одну.
Наступило кратковременное локальное равенство сил.
Следовало заняться восставшими дзайку-мару.

Люцифер уже начал просматривать полученные данные и прикидывать, как ставить новые эксперименты, когда его остановил Гало:

"--*Ты чего, брат?

"--*Нужно провести ещё серию экспериментов, чтобы\ldotst

"--*Зачем? "--- ухмыльнулся Гало.
"--- Спускайся в подвалы, Айну сейчас приведёт твоего сопляка.
Он знает, где бункер.

\section{[:] Провели}

"--*Я вам ничего не скажу, "--- сразу заявил Тахиро.
"--- Ничего.

Мальчик выглядел не лучше, чем в день, когда отряд Айну нашёл его.
На руках и ногах пестрели ссадины, в разбитом носу засохла кровь.
Руки и ноги были наскоро обездвижены кабельной стяжкой, на шее красовался ортопедический воротник, из углов рта торчал кляп-гантель, не позволявший крепко сжать зубы.
Левый рукав рубахи был мокрым от слюны;
видимо, мальчик лежал связанным несколько часов.

"--*Зачем ты его связала? "--- спросил Лу и попытался снять гантель.
Айну остановила его руку на полпути.

"--*Это не я, "--- ответила Айну.
"--- По словам стражи, он попытался причинить себе вред.

"--*Я ничего не скажу, "--- повторил Тахиро.

"--*Если я буду тебя пытать, ты расскажешь всё, "--- без обиняков сказала Айну.
"--- И совершить самоубийство у тебя не получится, даже не надейся.
Думаю, после твоей жалкой попытки втереться в доверие ты уже догадался, что мы умнее твоих соплеменников.

Мальчик понурился.

"--*Айну, "--- начал Люцифер.
"--- Пытки "--- это грубый способ.
Всегда можно найти\ldotst

Айну взмахом руки остановила юношу.

<<Мы не договаривались, что ты будешь играть хорошего палача>>.

"--*Да при чём здесь это! "--- воскликнул Лу вслух.
"--- Пытки "--- это непрофессионально и расточительно!
Если ты будешь ломать людей направо и налево, нам будет некем править!
Механизм нужно изучать в действии аккуратно, а не\ldotst

"--*Заткнись, "--- отрезала Айну и обратилась к мальчику:
"--- Я тебя слушаю.

Тахиро ответил не сразу;
он смотрел на Лу расширенными в изумлении глазами.
Этот взгляд ещё долго являлся потом стратегу во сне.
Раскосые чёрные глаза не просили поддержки, не жаловались на судьбу, в них читался растерянный вопрос: <<Что происходит?>>.
Стратег, поморщившись, отвернулся "--- словно боялся, что вопрос заразит и его разум.

"--*Я тебя слушаю, Тахиро, "--- повторила Айну.

"--*Я требую обмен, "--- мальчик продолжал смотреть на Лу.

"--*Ты торгуешься с нами, дзайку-мару? "--- ухмыльнулся Гало.

"--*Да, "--- просто ответил мальчик.
"--- Вы в состоянии войны с другими хорохито, это я знаю.
Каждый час пытки "--- потеря ресурсов для вас.
Я продаю вам несколько часов драгоценного времени.

Айну подняла бровь.
Она явно не ожидала от Тахиро такого взрослого подхода к делу.

"--*Вы дадите мне послать в бункер сообщение за пять минут до того, как начнётся штурм.

"--*За минуту.

"--*За четыре.

"--*За минуту, "--- повторила Айну.
"--- Или мы расторгаем сделку.

"--*Значит, расторгаем.

У Тахиро дрожали от страха губы, пока он произносил эту фразу.
Грисвольд ещё раз подивился его мужеству.

"--*Тахиро, "--- ласковый голос Айну был напитан смертельным ядом, "--- я хочу прояснить один момент.
Речь идёт не о нескольких часах драгоценного времени.
Мне не понадобятся часы, чтобы вытащить орешек из скорлупы.

В раскосых глазах Тахиро блеснул азарт:

"--*Спорим?

Айну задумалась.
Несколько долгих мгновений она, прищурившись, разглядывала маленького пленника.

"--*Три минуты, "--- буркнула она наконец.

"--*По рукам, "--- без раздумий бросил Тахиро.

"--*Что за сообщение ты хочешь послать?

"--*<<Всё пропало, спасайся кто может>>.

"--*Как оригинально, "--- скривилась Айну.
"--- Тебе нужно что-то ещё?

"--*Дайте мне дымный веер и найдите фумаролу в виде жабы возле южного склона Арашияма.

"--*Фумаролу? "--- выпучил глаза Гало.
"--- Так вы использовали дым фумаролы, а не\ldotst

"--*Да, конечно, дубина, "--- пробурчал Тахиро, "--- люди бункера носили кремень просто для отвода глаз!
И кострище было лишь отвлекающим манёвром.
Только хорохито будет разводить костры, если есть фумарола.

Гало и Люцифер переглянулись\ldotst и разразились диким хохотом.

В этот раз люди их провели.

\razd

"--*Я посылаю сигнал, мы ждём три минуты и я сообщаю вам координаты входа в бункер.

"--*Могу ли я ручаться, что твои данные окажутся истинными? "--- осведомилась Айну.

"--*Я клянусь тебе честью моего дома.

"--*Есть ли честь в том, что ты предаёшь соплеменников? "--- ухмыльнулся Гало.

"--*В том, чтобы умирать под пытками, точно нет чести, "--- заявил мальчик.
"--- Я признаю лишь смерть в бою.
Считай это способом выжить и дождаться боя.

<<Он опасен, Айну, "--- сообщил Гало.
"--- Этого Фенрира следует уничтожить>>.

<<Я против, "--- заявил Люцифер.
"--- Мальчик крайне интересен как объект исследования.
Я бы попробовал интегрировать его в наше общество>>.

Айну почесала подбородок и оглядела связанного Тахиро.

<<Делай со своей обезьяной что хочешь.
Убить можно в любой момент>>.

Айну махнула рукой и вышла из подвала, Грисвольд и Гало последовали за ней.
Лу остался наедине с Тахиро.

Мальчик молчал.
Лу следовал его примеру.
Разумеется, робкие ростки доверия между ними сметены безжалостным вихрем реальности.
Эти двое не могут быть товарищами "--- они навсегда останутся хищником и жертвой.
Мальчик понимал это не хуже демона.
Он видел перед собой лишь коварного врага, готового на всё ради победы.

Стратег до смерти не любил этот этап в любом деле "--- когда в закрытой системе всё понятно, всё доступно взгляду, но процесс уже запущен в нежелательном направлении.
И будь ты хоть семи пядей во лбу, тебе придётся просто стоять и ждать подходящего момента, чтобы исправить ситуацию.

Последние шаги давно затихли.
Наконец стратег нарушил молчание:

"--*Слушай, ребёнок\ldotst

"--*Благодарю тебя, что вступился.

"--*Как ты узнал? "--- удивился Люцифер.

"--*У нас есть легенда, что некоторым людям однажды преграждают дорогу два брата-близнеца "--- Футаго-рэй и Футаго-итсу.
В глазах одного брата "--- жизнь, в глазах другого "--- смерть.
И человек пытается задобрить, подкупить или побороть братьев, не зная, что шанс что-то изменить уже упущен и он может лишь наблюдать, как братья спорят.

"--*Занятная легенда.
Мне нужно твоё мнение как\ldotst представителя другой формы жизни.

"--*Я слушаю.

"--*Как думаешь, благодаря чему существа приходят к сотрудничеству?

"--*Благодаря пониманию, "--- пожал плечами мальчик;
его отвратительная сбруя не позволяла большего.

"--*А чтобы кто-то понял, нужно объяснить как можно более доходчиво?

"--*Именно.
Доходчиво и искренне.
Необязательно говорить правду, достаточно того, что ты считаешь правдой, и очень важно показать ход мысли.
Желательно на языке собеседника "--- так учила меня мать.

"--*Но что, если собеседник не поверит?

"--*Это его право.
Если ты считаешь собеседника за равного "--- уважай его права.

Лу помолчал.

"--*На самом деле мы уже были близки к обнаружению бункера, "--- наконец сказал он.
"--- Это я послал твоих родичей на дело, из-за которого их линчевали.
Вся информация о бункере, которую они, сами того не зная, собрали в пустошах, уже у меня.

"--*Ты просишь прощения?

"--*Я не ставил целью их уничтожить.
Расценивай это как хочешь.

Тахиро кивнул:

"--*Это было доходчиво.

"--*Я могу снять путы, если ты пообещаешь, что не попытаешься сбежать или убить себя.

"--*Я уже поклялся всем, что у меня есть.

Лу молча вытащил ножницы, разрезал стяжки и вышел, оставив дверь приоткрытой.

\section{[:] Тушение}

\spacing

"--*Только не говорите, что этот сброд будет драться, "--- поморщился Арракис.

"--*Приготовиться к залпу, "--- бросил Гало.

"--*Легион, отбой, "--- тут же подняла руку Айну.
"--- Ждём моего приказа.

Один из восставших вышел перед собратьями, собираясь обратиться к ним с пламенной речью.

"--*Потуши его, "--- лениво бросила Айну стоящему рядом легионеру.

Легионер вскинул винтовку "--- и оратор упал с простреленной головой, едва успев выкрикнуть <<Люди!>>.
Толпа дрогнула.

"--*На этой планете речи произносятся или мной, или с моего разрешения, "--- резюмировал Арракис.
"--- Самаэл, я надеюсь, ты сумеешь объяснить это дзайку-мару.

Самаэл кивнул и направился к восставшим, замершим в растерянности.

"--*Учись, сайгон, "--- ухмыльнулась Айну, бросив лукавый взгляд на Гало.
"--- На решение проблемы обычно требуется только один патрон.

"--*Или можно было просто узнать, чего они хотят, "--- вставил Люцифер.
"--- А потом насыпать рыбкам мотыля или сменить наконец воду, чтобы они не плавали в собственном дерьме.

"--*А твой брат "--- теоретик-идеалист, "--- закончила Айну со вздохом.

"--*Из вас всех я единственный, кто ухаживает за аквариумом и террариумом в штабе, "--- парировал Лу.
"--- Не будь меня, вся животина давно бы уже сдохла.
Так что время покажет, кто из нас теоретик.

"--*Безусловно, "--- кивнул Арракис.
"--- И время покажет это в самое ближайшее время.
Штурмовой отряд, вы переходите под командование Люцифера.
Остальной легион "--- на указанные точки.
\mulang{$0$}
{У нас есть дела.}
{We've got much work to do.''}

Гало удивлённо посмотрел на отца:

\mulang{$0$}
{"--*Отец, может быть, мне?}
{``Father, maybe I?''}

\mulang{$0$}
{"--*Нет, Гало, ты нужен в штабе.}
{``No, Halo, I need you in the HQ.}
\mulang{$0$}
{Самаэл уже разогнал толпу.}
{Samael's just dispersed the crowd.}
\mulang{$0$}
{С моими легионерами бункер сможет взять даже бродячий торговец.}
{Even a peddler can storm the bunker if he's got my legionnaires.''}

Люцифер кивнул и скомандовал построение под пристальным взглядом Айну.

<<Она за меня беспокоится, как за члена семьи, "--- подумал стратег.
"--- Война ещё не совсем отравила её сущность>>.

\section{[:] Склад}

Люцифер шёл по коридору и в который раз удивлялся, сколько мертвецов осталось в бункере.
Его легионеры убили лишь треть;
прочие умерли ещё до прихода демонов.
Это могла быть людоедская тактика, заставляющая врага поверить, что никто и не думал бежать.
С той же вероятностью это могло быть совершенно бессмысленное самоубийство "--- люди Преисподней не особо раздумывали перед тем, как убить или умереть.
Подсчёт пищи и инструментов не смог дать чёткого ответа на то, сколько людей было в бункере час назад "--- слишком широкие доверительные интервалы, слишком много помех.
Даже состав и объём воздуха подкачали во всех смыслах "--- комнаты были маленькие, но вентиляция работала так, что набитая трубка раскуривалась в руках сама.
Облава по селениям могла бы прояснить этот вопрос, но ресурсов для облавы было недостаточно.

<<Мой интеллект пересиливает, "--- думал Лу.
"--- Но недостаточно быстро.
Чересчур мало данных.
Чересчур много тех, кто мне противостоит.
Отец и Гало только мешают, Айну чересчур ограничена, Самаэл развёл бюрократию, Мистраль "--- трусиха, а Грисвольд ведёт себя так, словно происходящее его не касается.
Не могу же я, чёрт возьми, биться на три фронта в одиночку!>>

Лу мысленно поставил заметку "--- следует поговорить с Грисвольдом о Гало.
Дальше этот спектакль продолжаться не может.

Носильщики тащили добычу к выходу.
Лу с некоторым удовольствием наблюдал за ними.

<<Как же хорошо, когда кто-то таскает тяжести за тебя>>, "--- думал он.

Вот командный пункт.
Три тела со следами сэппуку.

Вот шифровальная.
Бежавшие не удовлетворились простым убийством шифровальщика;
ему раздробили голову до кашеобразного состояния.

Вот склад с пищевым порошком.
Его количество уже подсчитали, пробы для выявления места производства взяли.
Возле апробированной коробки лежали два трупа "--- самоубийца и убитый защитник.

Лу осмотрел склад, просмотрел коробки с помощью тепловизора.
Да, это пищевой порошок.
Мужской у входа, женский "--- чуть подальше.
Больше ничего.

Но что-то не давало покоя стратегу.
Коробки.
Они все были совершенно одинаковые.
Каждый последующий ряд отстоял от предыдущего на какое-то расстояние "--- совершенно произвольное.
Впрочем, не совсем произвольное\ldotst
Совсем не произвольное!

"--*Разобрать коробки, "--- распорядился Лу уже в спины уходящим легионерам.

Вскоре под коробками обнаружилась пустота.
Там сидели двое мужчин, пять женщин и тринадцать детей.
У всех на лицах замерло туповатое выражение;
моргали они редко и медленно, словно под водой.
Люцифер разглядывал их так, словно сам был не рад своей находке.

"--*Нейролептический сопор, "--- хмыкнул один из легионеров.
"--- Нам доставить их в допросную?

"--*Убить всех.

"--*Сайгон\footnote
{Сайгон "--- ранг в раннем Ордене Преисподней, соответствующий легату прима. \authornote},
не лучше ли взять некоторых женщин с выводком?
Если к ним правильно подступиться, они живо заговорят.

\mulang{$0$}
{"--*Возможно, нам стоит их отпустить.}
{``Maybe we should release them.}
\mulang{$0$}
{Что скажешь, Фумиэ из клана Дорге?}
{What would you say, Fumie of the Clan Dourgue?''}

Глаза солдата забегали, словно у провинившегося.
Прочие замерли и вытянулись в струнку, прислушиваясь к разговору.

"--*Нам нужна информация, сайгон, "--- осторожно начал Фумиэ, "--- но неповиновение дзайку-мару ни в коем случае не должно оставаться безнаказанным.

\mulang{$0$}
{"--*То есть после допроса их следует уничтожить?}
{``It means they must be destructed after examination, mustn't they?''}

\mulang{$0$}
{"--*Безусловно.}
{``Unquestionably.''}

\mulang{$0$}
{"--*Ты не считаешь это расточительностью, напрасной тратой ресурсов?}
{``Don't you consider it waste of resourses?''}

\mulang{$0$}
{"--*Мы используем штыки вместо патронов, когда это возможно.}
{``We use bayonets instead of bullets if it's possible.''}

"--*Я про дзайку-мару.
Они тоже являются ресурсом.

"--*Прошу прощения, сайгон, "--- стушевался легионер.
"--- Дзайку-мару "--- опасный ресурс, который может легко выйти из-под контроля.

\mulang{$0$}
{"--*И твоё мнение разделяет весь легион?}
{``And does the legion share your view?''}

Фумиэ бросил взгляд на стоящих рядом товарищей, но они старательно делали вид, что разговор их не касается.

\mulang{$0$}
{"--*Я не могу говорить за легион, но я разделяю мнение командования.}
{``I mustn't speak for the whole legion, but I share the view of the command.''}

\mulang{$0$}
{"--*В таком случае, легионер, оставь привычку обсуждать мои приказы, "--- поморщился Лу.}
{``In that case, legionnaire, break the habit of questioning my orders,'' Lu made a wry face.}
Лицо маленького Тахиро промелькнуло в его памяти чересчур ясно.
"--- Бункер выжечь до каменных стен, убить всех обнаруженных дзайку-мару, стены проверить на пустоты и трещины.

Фумиэ с явным облегчением козырнул и резво отправился выполнять приказ.

Люцифер сел на разбитую коробку и, набив трубку адским вереском, раскурил её от щепки.
Он вдыхал и выдыхал терпкий дым, лишь иногда морщась, когда раздавался приглушённый звук втыкающегося в тёплое тело штыка.

\mulang{$0$}
{<<Отец меня убьёт, "--- с грустью думал стратег.}
{``Father will kill me,'' the strategist sadly thought.}
\mulang{$0$}
{"--- Но пока я за главного, пыток не будет.}
{``But no tortures while I'm in charge.}
\mulang{$0$}
{Даже если это была последняя возможность проявить себя>>.}
{If even it was my last opportunity to lead.''}

\chapter{[U] Начало}

\section{[U] Встреча}

\spacing

Несмотря на то, что снаружи царила негостеприимная обстановка заброшенной орбитальной станции, в виртуальной <<комнате>> было довольно уютно.
Собравшиеся сидели в маленькой ажурной беседке из жемчужного дерева, освещённой нежным розовым светом бумажного фонарика.
Из окружающей беседку полутьмы то и дело вырывались огромные, размером с орех, индиго-светлячки и обиженно жужжали, ударяясь о фонарь.

Обстановку придумала Митхэ, женская сущность Митриса Безымянного.
По её словам, наводить домашний уют для неё всегда было истинным удовольствием.

Разумеется, никто из присутствующих не высказал удивления, увидев Лусафейру Лёгкая Рука, а вернее, так называемую базовую копию его личности.
Но вопрос повис в виртуальном воздухе, и первым его озвучил хмурый чернявый мужчина в углу "--- Аркадиу Люпино.

"--*Грейс, не мог бы ты нам объяснить, как главный оборонительный стратег и максим секунда Ордена Преисподней оказался на собрании адской <<пятой колонны>>?

"--*Я подумал, что ему будет интересно, "--- осклабился Грейсвольд.

Аркадиу и Анкарьяль кивнули "--- им этого объяснения было вполне достаточно.
Но остальных ответ технолога не удовлетворил.
Атрис "--- мужская сущность Безымянного "--- улыбнулся с едва заметным скепсисом.
Таниа Янтарь громко, выразительно хмыкнула.

"--*Итак, "--- Аркадиу попытался сгладить напряжение собравшихся.
"--- Сегодня мы должны обсудить вопрос эксплуатации сапиентов хоргетами.

"--*В чём смысл этого вопроса? "--- подал голос Лусафейру.
Говорил он очень мягко, в его устах слова словно теряли острые углы.
"--- Эволюция сделала своё дело.
На всякого хищника найдётся хищник сильнее.
Я думал, что вы хотите обсудить нечто иное.

Чханэ гневно уставилась на стратега.
Тот встретил её взгляд холодно и спокойно.

"--*Это Разрушение и Насилие в одном горшке, "--- резко сказала воительница.

"--*Вы имеете в виду законы примитивного человеческого племени планеты Тра-Ренкхаль? "--- осведомился Лусафейру.

"--*Эти законы были придуманы культурологами цивилизации Тхидэ, "--- бросила Чханэ.
"--- Они гуманны\ldotst

"--*\ldotst и совершенно бесполезны для выживания, "--- добавил Лусафейру.
"--- Что и показало сражение на берегу Могильного пролива, которое едва не унесло остатки твоего народа в \emph{пристанище духов}, "--- последние слова он произнёс на идеально чистом сели с характерной интонацией.

Анкарьяль и Аркадиу переглянулись.
В силу большого совместного опыта они начали понимать, с какой целью Грейсвольд привёл Лусафейру.
Увы, опасности, повисшей над заговорщиками, это не убавляло.

Чханэ, похоже, тоже поняла, чего добивается стратег.
Она расслабилась и откинулась на кресле.

"--*Почему вам, демонам, так сложно объяснить простую вещь вроде гуманности? "--- мягко спросила она.

"--*Гуманность "--- это то, что связывает руки одним людям, пока другие строят против них козни, "--- объяснил Лусафейру.
"--- Ваше понимание "--- понимание тси "--- исключение из правил, результат культурного дрейфа.
Вы выросли в тепличных условиях, вне конкуренции.
Кажется, Аркадиу писал в отчёте, что по прибытии на Тра-Ренкхаль тси очень удивились даже оружию для убийства сапиентов\ldotst

"--*А я всегда думала, что Ад гуманен по отношению к сапиентам, "--- подала голос Митхэ.

"--*О нет, флейта моей жизни, "--- неожиданно рассмеялся Атрис, "--- Ад блюдёт собственную выгоду и ничего сверх этого.

Анкарьяль удивлённо посмотрела на менестреля "--- он никогда не казался ей особо сообразительным.

"--*Верно, "--- подтвердил Лусафейру.
"--- На планетах, население которых составляют Ветви Звезды, мы вынуждены практиковать деспотизм, так как сапиенты Ветвей Звезды испускают плюс-эманации при угнетении.

Чханэ издала странный звук "--- не то плюнула, не то выругалась.

"--*Что такое деспотизм? "--- вмешалась Анкарьяль.
"--- Сапиентов угнетают под угрозой насилия, хоргеты находятся под тотальным информационным контролем.
Деспотизм везде.

"--*Анкарьяль в силу юношеского максимализма преувеличивает, но доля истины в её словах есть, "--- признал Грейсвольд.
"--- В Ордене нам дают полную свободу, но не забывают <<предупреждать>> или уничтожать, если мы что-то делаем \emph{подозрительно}.
Да, нам дают объяснения любых событий, но когда за фабрикацию доказательств берётся мощнейший отдел аналитики, что можно этому противопоставить?
Вычислительные мощности, опыт и количество актуальной информации прямо пропорцинальны доле власти.
Да, Лу?

"--*Отчасти, Грейс, "--- мягко ответил Лусафейру.
"--- Будучи главным оборонительным стратегом, я трачу треть ресурсов на вычисление вероятности, с которой какое-либо моё действие вернётся ножом в спину.

Митхэ хохотнула и тут же осеклась, поняв, что стратег не шутит.
Он даже не улыбнулся.

"--*Ну так идём с нами, "--- вдруг страстно сказала Чханэ, наклонившись к Лусафейру.
"--- Мы никогда не ударим тебе в спину.
За себя я отвечаю.

Лусафейру, похоже, смутился.
Не эмоционально.
Он просто понял, что это правда.

"--*Таниа Янтарь, какая жизнь ожидает существо, которое обладает таким количеством ценных знаний?
Я пленник Ада, при попытке сбежать я не успею сделать и шага.
Если мне и удастся каким-то невероятным способом покинуть Капитул, кто меня защитит и обеспечит эманациями?
Ты?

"--*Да!
Во мне боевые навыки Анкарьяль Кровавый Шторм!

"--*Я не всесильна.
Не нужно делать из меня микоргета, "--- возмутилась Анкарьяль.

"--*Ты погибнешь совершенно зря, "--- с лёгкой грустью заключил Лусафейру.

"--*Давайте не будем отклоняться от темы, "--- прервал спорящих Грейсвольд.
"--- Лу, так что?

"--*Я не буду вам помогать, если ты об этом, "--- сказал Лусафейру.
"--- Это решать не мне.
Я, если вы не заметили, всего лишь базовая копия.
Если вы хотите возродить Тси-Ди на Тра-Ренкхале "--- действуйте, я же буду действовать так, как должен.
Грейс, я ведь стёр разговор после того, как ты вышел от меня?

"--*Ты подал сигнал, что стёр.
Что-то передать?

"--*Хм, "--- задумался Лусафейру.
"--- Скажи мне при встрече следующее: <<Поклонись>>.
Интонация и мимика имеют значение.

"--*Понял, "--- кивнул технолог.

"--*Тогда самоуничтожаюсь, "--- сказал Лусафейру и исчез.

"--*Стой, куда! "--- простонала Чханэ.

"--*Чханэ, так нужно.
Тем более что это только неполная копия, "--- успокоил её Грейс.

"--*Копия твоего друга, которая думала и чувствовала, как он, "--- невзначай заметил Атрис.
Грейсвольд смущённо замолчал.

"--*Что насчёт остальных? "--- спросил Аркадиу.
"--- Кто-то хочет отказаться от плана?

Ответом было молчание.

"--*Тогда перейдём к главному, "--- сказал Аркадиу.
"--- Я "--- официально предатель Ордена.
Митрис не принадлежит к Ордену, но связан с ним договором типа <<Демиург "--- Метрополия>>.
О существовании Таниа Орден, будем надеяться, не знает.
А вы, "--- обратился он к Грейсвольду и Анкарьяль, "--- верные служители с безупречной репутацией.
Если вас уличат в сомнительных действиях "--- отдел 100 распутает клубок очень быстро, и Ад ждёт такая чистка, какой не было со времён его становления.

"--*Именно поэтому я и приготовил \emph{это}, "--- улыбнулся Грейс и помахал пустой рукой, словно сжимая в кулаке маленький круглый предмет.

"--*Что это? "--- поинтересовалась Чханэ.

"--*Пакет ложных воспоминаний, "--- сказал Грейвольд.
"--- Его созданием занимались девятнадцать демонов из отдела аналитики.
Не волнуйтесь, "--- поспешно добавил он, увидев, как Атрис вздохнул и прикрыл лицо рукой, "--- не волнуйтесь.
Я разбил задачу на фрагменты, не вызывающие подозрений, и использовал самых лояльных демонов.
На всякий случай я скомпрометировал одного из них связью с Картелем, чтобы увести следствие в другую сторону.

"--*Что в пакете? "--- поинтересовался Аркадиу.

"--*У нас с тобой возникли разногласия, и ты отбыл, не сказав куда.
Также в несколько ранних воспоминаний добавлены малозначимые эпизоды, которые обычно проходят мимо отчётов, но опытному аналитику намекают на твоё помешательство.
Пакетов всего\ldotst эээ\ldotst семь.
Да-да, Митхэ ар’Кахр, я подготовил для тебя отдельный.
Учтено всё "--- структура личности, взаимоотношений, даже погрешности восприятия и анализа.
Если тебя, Аркадиу, или кого-то из твоих демонов поймают и вывернут наизнанку, о нашей с Анкарьяль причастности не узнает никто.
Обратное также верно.

Атрис нахмурился.

"--*Тогда как мы будем держать связь?

"--*Рисунки на песке, "--- улыбнулась Анкарьяль.

Чханэ недоумевающе хмыкнула.

"--*Помните, когда мы были детьми на Тра-Ренкхале, у нас была забава "--- рисовать на песке? "--- откликнулся Грейсвольд.
"--- А ещё мы строили песчаные домики\ldotst

"--*И когда набегала волна или поднималось солнце, домики рассыпались, "--- на лице Митхэ появилась понимающая улыбка.
"--- Пара волн "--- и песок становился гладким, проступали водяные рёбрышки, словно к нему не прикасалась рука человека.

"--*Я всё равно ничего не поняла, "--- заявила Чханэ.

"--*Тогда, милая, Вселенная преподнесёт тебе ещё один сюрприз, "--- рассмеялась Анкарьяль.

Чханэ погрустнела.

"--*Так значит, я теперь буду считать вас врагами?
Даже если вы умрёте?

Анкарьяль с нежностью посмотрела на подругу и кивнула.
Чханэ криво улыбнулась.

"--*Вы лучшие, кого я знаю.

"--*Удачи, "--- тихо сказал Грейс и кивнул Анкарьяль.
Оба растворились в воздухе.
Оставшиеся дружно вздохнули и ушли в себя, ожидая, пока закончится интеграция пакетов воспоминаний в систему.

\section{[U] Проблема теплиц (переработать!)}

\spacing

Проблема создания кольцевой теплицы "--- в организации генетического аппарата, оптимизации и стабилизации его работы.
Когда учёные Ада взялись за это дело, им пришлось работать с огромным количеством ферментов и каскадов.
Клетки, нагруженные таким количеством генов, быстро выходят из строя из-за случайных мутаций, даже в условиях стерильной среды.
А учёные-тси, согласно данным, решили эту проблему максимально изящно "--- альтернативным сплайсингом.
Один ген этого создания, хоть и довольно длинный, мог дать такое количество вариаций, какое обычным живым существам и не снилось.
Скорость мутации генов кольцевой теплицы была ниже таковой в генах гистонов эукариот.
Кольцевая теплица также обладала уникальной системой слежения за собственным генотипом "--- специальная ферментативная система фактически вычисляла хэш всех имевшихся хромосом и запускала апоптоз при любой мутации.
Подобное произведение генной инженерии не смог создать никто во Вселенной, даже хоргеты.

"--*Интересно, как они выглядели, "--- задумался я.

"--*Понятия не имею, "--- хмыкнул Грейс.
"--- И вряд ли кто-то имеет.
По одним данным это растение, по другим "--- среди тси были люди, которые с ними общались: ласкали их, разговаривали.
И теплицы им отвечали.
Вряд ли растение способно говорить.

\spacing

\section{[U] Нужна теплица}

\spacing

"--*С помощью кольцевой теплицы можно в кратчайшие сроки оживить практически любую планету в обитаемой зоне, не пользуясь услугами хоргетов, "--- сказал я.
"--- А теперь представьте, что будет, если это знание станет общедоступным.
Всеобщая экспансия за пределы ка'нетовского радиуса.
Любой хоргет сможет двигаться сколь угодно далеко в любом направлении, найти подходящую планету и стать её полновластным демиургом.
Каждому хоргету "--- по планете!
Ад и Картель находятся в истинном равновесии ровно до тех пор, пока у них есть физические границы.
Убери границы "--- и старый мир, основанный на идее дефицита, рухнет, как когда-то рушились колониальные империи.

Я помолчал, собираясь с мыслями.

"--*Второе направление работы "--- планетарная система.
Нам необходимы её чертежи.
Демиурги беззащитны перед организациями демонов.
С доступностью системы планетарной защиты гегемонии Ада и Картеля придёт конец.
Более того, демиург вступит в мутуалистические отношения с сапиентной цивилизацией своей планеты, так как от её технического развития будет напрямую зависеть его собственная безопасность.

"--*Мне не нравится то, что ты предлагаешь, Аркадиу, "--- сказал Грейсвольд.
"--- Жизнь демиурга для каждого демона "--- это хорошо, но я хотел бы общаться с себе подобными, а не запираться в планете на веки вечные.
То, что ты предлагаешь, разрушит социум демонов "--- как экспансия за пределы ка'нетовского радиуса, так и превращение демонов в богов.
И существует ещё одна проблема.
Для минус-демонов места в этой системе не будет.

"--*Технологии тси помогут нам и здесь.
Вдруг существует возможность вывести нуль- или даже минус-сапиентов?

"--*Вероятность крайне мала, "--- рассудительно сказал технолог.
"--- Боюсь, что если нам не удастся этого сделать, Картель будет драться до последнего.
А это значит, что они почти наверняка победят.
Поэтому, если ты действительно хочешь создать дивный новый мир, следует проработать момент с минус-демонами.

\spacing

\section{[U] Вербовка}

\spacing

"--*Ты сделаешь нас бессмертными, Ликхмас? "--- спросил кто-то.

"--*Вы получите долгое существование, но не бессмертие.
Если сапиентам уготована лишь одна первичная смерть, то вы будете умирать ею сотни и тысячи раз.

Многие опустили головы.

"--*Лисёнок, "--- сказала Кхотлам, "--- значит ли это, что мы переживём всех родных?

"--*Вы можете пережить даже свой собственный народ и землю, на которой стоите, если вам повезёт.
Если же вас настигнет вторичная смерть, то не будет для вас пристанища "--- вы умрёте навсегда, и никогда уже не будете живыми.

Все дружно вздохнули.
Акхсар громко скрипнул зубами.

"--*Что я буду видеть и слышать, если стану демоном? "--- спросил воин.

"--*Вы испытаете ощущения, которые даже не можете себе представить.
Будут среди них и неприятные, и поистине ужасные.
Вам придётся долго и упорно тренироваться, чтобы управлять своими новыми силами.

"--*Зачем ты предлагаешь нам такую участь? "--- недоумённо поинтересовалась женщина с краю.
"--- Менять одну смерть на тысячи, бессмертие и покой в пристанище на борьбу и забвение\ldotst

Окружающие закивали, выражая согласие.

"--*Мне нужна помощь, "--- просто сказал я.
"--- Я не требую от вас, не взываю к чести предков, не рассказываю о тех чудесах, которые вы познаете, хотя и чудес будет немало.
Я смиренно прошу о помощи, потому что это "--- величайшая и, вполне возможно, напрасная жертва.

"--*Почему напрасная? "--- удивился Акхсар.

"--*Вы помните владычество Эйраки.
Это был единственный угнетатель.
Десятки тысяч миров стонут под гнётом других, не менее могущественных демонов.
И многие из них не одиноки, существуют целые организации, обладающие огромными ресурсами и глубокими знаниями.
Мы можем проиграть, и от нас не останется даже воспоминаний.

"--*Если мы все погибнем, что-то изменится? "--- спросил Акхсар.

Я улыбнулся.
Тот же вопрос задавали мне многие перед битвой на Могильном берегу;
ответ на него ничуть не изменился.

"--*Тем, кто пойдёт за нами, будет легче.

"--*Чем мы можем тебе помочь, братик?
Мы "--- простые люди, "--- заметила Лимнэ.
"--- Я владею клинком недостаточно даже для того, чтобы сразить умелого воина\ldotst

"--*Я научу вас тому, что знаю.
Мне понадобятся не столько воины, сколько те, кто готов трудиться и искать "--- знатоки живых существ, строители машин, книжные люди.
Даже те, кто хорошо умеет что-то одно, смогут помочь.

Сестрёнки кивнули в знак того, что поняли.

"--*Даю на размышление пять дней.
Если кто-то решит, что готов "--- жду вас здесь в Змею 1, на восходе солнца.
И помните "--- обратного пути уже не будет.

\section{[U] Первые}

"--*Никто не придёт, как пить дать, "--- сквозь зубы процедила Анкарьяль.
За последние два дня я услышал эту фразу не менее ста раз.
"--- Ты вообще узнавал, как проходит вербовка?
Сапиентам предоставляют полную информацию, а ты сделал всё, чтобы их отпугнуть.

"--*Придут немногие, "--- согласился я, "--- но их ценность выше, чем у тех, кто зарится на долгую жизнь и чудеса.

"--*Никто не придёт, "--- повторила Анкарьяль.
"--- Я почувствовала, как их напугали слова о посмертии.
Пристанища не будет!
Такое и в страшном сне привидеться не может.

"--*Его не будет в любом случае, "--- тихо напомнил я.
"--- Просто сели, как ни крути, дикари, а вот демонам верить в такие сказки не к лицу.

Мы ждали долго в напряжённом молчании.
Тихо потрескивая, горела свеча.
И вот в назначенный час раздался стук в дверь.

"--*Ты проиграла, "--- бросил я, направившись к двери.
"--- Один есть.

"--*Много ты сделаешь с одним, "--- горько прошептала Анкарьяль.

За дверью меня встретили девять человек.
Я молча пропустил их в дом.
Все, не дожидаясь приглашения, принялись наполнять свои чаши чаем из котелка.

"--*Что сказала кормилица? "--- обратился я к сестрёнкам.

"--*Она не пойдёт.
Акхсар хотел, но сказал, что бросить Кхотлам будет для него бесчестьем, "--- ответила Манэ.
"--- Кормилица дала ему второй шанс.

"--*А ваш мужчина?

"--*Он не хочет лишаться пристанища, но сказал, что целиком нас поддерживает, "--- объяснила Лимнэ.

"--*Ты же говорила, что хочешь покоя и не желаешь бессмертия, "--- улыбнулся я Чханэ.
Подруга не ответила на улыбку, её взгляд был предельно серьёзен.

"--*Когда мы говорили, всё было хорошо, "--- проворчала она.
"--- Теперь я вижу, что не всё в порядке и моё присутствие необходимо.

"--*А ты почему пошёл, Митрам? "--- обратился я к молодому ювелиру.

"--*Ты сказал про строителей машин, "--- пожал тот плечами.
"--- Я люблю машины, увлечён ими с детства.
Посмертие "--- весьма разумная цена за знание.

<<Эти слова могли бы принадлежать древнему тси>>, "--- невольно вырвалось у Анкарьяль.
Я мысленно согласился.
Произнёсший подобное "--- уже не дикарь.

"--*А вы "--- жрецы, "--- поклонился я двум мужчинам и женщине в синих робах.
"--- Врач, счетовод и\ldotst

"--* \dots и Митхэ ар'Тра, "--- закончил приятный контральто.
Женщина откинула капюшон робы, и слабый свет выхватил из темноты тяжёлые серебристые волосы и благородное лицо, в котором я вдруг обнаружил знакомые черты.
"--- Масло-Взбитое-Взглядом.

"--*Я чту традиции Севера, Митхэ, "--- улыбнулся я.
"--- Учитель должен знать всех учеников по именам.

"--*А мы с братом просто пришли помочь, "--- вмешался крестьянин Согхо, махнув на стоящего рядом незнакомого мужчину.
"--- Я всю жизнь обрабатывал землю, знаю, как выращивать кукурузу, помидоры и цветы.
Мой брат "--- змеелов.
Мы оба любим своё дело.
Можем ли мы пригодиться?

Я улыбнулся.
Два диктиолога, интерфектор, технолог, три когитора-информатика и два биолога.
Превосходное начало.

"--*Вполне, "--- кивнул я.
"--- Только вы понятия пока не имеете, что вам придётся выращивать и каких змей ловить.

<<Ты же не надеешься, что они сразу станут профессионалами? "--- скептически произнесла Анкарьяль.
"--- Они дикари, Аркадиу.
Многие из них могут сойти с ума даже из-за процесса демонизации>>.

<<Лесные духи, "--- махнул я рукой.
"--- Уж ты-то помнишь, с чего начинал я>>.

<<Ты был ребёнком>>, "--- напомнила Анкарьяль.

Тут она права.
Мысль набрать в свои ряды детей я уже рассматривал.
Но перед глазами тут же вставал Отбор, когда ничего не понимающему созданию предлагали сделать главный выбор в его жизни\ldotst

Анкарьяль понимающе кивнула.
Она, как обычно, читала мои мысли.

<<Что ж, тебе придётся работать с тем, что есть>>.

<<,,Тебе``?
А ты со мной, Нар?>>

Подруга вздохнула и отошла набрать чаю из котелка.

\chapter{[:] Агнец}

\section{[:] Желание демона}

\spacing

Закатные аспиды угасли, и небо расчертили брошенные горстью метеоры.

"--*У нас принято загадывать желание, когда сгорает метеор, "--- сказал Тахиро.
"--- Говорят, оно непременно сбудется.
Только нужно сказать его вслух.

Люцифер засмеялся.

"--*Я думаю, что для камня, пролетевшего миллиарды километров, оскорбительно служить моим мелким желаниям.

"--*Он не обидится, "--- Тахиро похлопал Лу по плечу.
"--- Без тебя его существование было бы совершенно бессмысленным.

"--*Скажи, как у тебя так выходит?

"--*Ты о чём? "--- не понял Тахиро.

"--*Вот взять меня.
Во мне масса информации "--- факты, формулы и структуры, связанные между собой определённым образом.
Это большая часть того, что нам удалось собрать из наследия первых людей.
Можно сказать, я квинтэссенция их науки.
Но ответы на все вопросы есть только у тебя.
И я бы не сказал, что они лишены смысла.
Иногда мне даже кажется, что твои слова спали во мне всю мою жизнь и проснулись, едва ты заговорил.

"--*Это называется <<философия>>, сын.

Арракис, как обычно, подошёл незаметно.
Друзья переглянулись.

"--*Я уверен, что у тебя в памяти есть масса философских концепций, "--- продолжал Арракис, положив холодные как лёд руки на плечи Лу.
"--- Жизнь ещё научит тебя различать их в речах людей.
Философия помогает людям преодолеть акбас, помогает им защититься от бесконечности, хаоса и случайности, царящих во Вселенной.
Та же философия делает их ограниченными и невежественными.

"--*Почему? "--- поинтересовался Люцифер.

"--*По причине, которую ты уже озвучил.
Люди уверены, что у них есть ответы на все вопросы.
А теперь оставь своего питомца наблюдать за звёздным небом "--- ему это полезно.
Ты нужен мне в штабе.
В конце концов, ты создан для того, чтобы править, не так ли?

"--*Подожди, Лу, "--- Тахиро схватил друга за руку.
"--- Один из метеоров точно был твоим.

"--*Если я создан для того, чтобы править, "--- Лу бросил взгляд на Арракиса, "--- я бы стал лучшим из правителей.
А ты, Тахиро?

"--*Я бы сделал всё, чтобы лучший из правителей дожил до конца войны, "--- сказал Тахиро.

"--*Увидимся за ужином, "--- Лу улыбнулся и растворился в ночной тишине.

\section{[:] Трубки}

\spacing

"--*Что ты делаешь? "--- удивился Люцифер.

"--*Хочу проработать некоторые моменты стратегии, "--- уклончиво ответил Гало.

"--*У нас сейчас свободное время, "--- напомнил Тахиро.

"--*У тебя вся жизнь "--- свободное время, дзайку-мару.

"--*У тебя тоже, "--- непринуждённо парировал Тахиро.
"--- Ты "--- вольная птица.

"--*Я ценен как работник, а не как подопытное животное.

"--*А в чём разница?

"--*В уровне интеллекта, "--- ядовито ответил Гало.

"--*Это точно.
Ни один грызун не додумается крутить своё колёсико во время отдыха.

"--*Курить хочется, "--- лениво проворковал Люцифер.
"--- Гало, кинь какую-нибудь трубку.

Гало запустил в тазик с трубками руку и бросил Люциферу первую попавшуюся.

"--*Нет, Гало, не эту, "--- сказал Люцифер, осмотрев брошенное.
"--- У неё чересчур короткий чубук.

"--*Ты сам сказал <<какую-нибудь>>.

"--*А ещё я сказал <<не эту>>.

"--*Это же очевидно, Гало, "--- усмехнулся Тахиро.
"--- Если Лу сказал <<какую-нибудь>>, то ты должен взять трубку и спросить, эту ли хочет Лу.
Если Лу сказал <<любую>>, то у него очень хорошее настроение и можно не спрашивать, а просто кинуть первую попавшуюся.

"--*Что за чушь ты городишь? "--- буркнул Гало.

"--*Отнюдь, "--- возразил Лу.
"--- Тахиро правильно понял, что первое слово означает определённое нежелание в сочетании с неопределённым желанием.
Второе обозначает неопределённое желание при полном отсутствии нежелания.

"--*Брат, тебе сложно просто сказать, какую трубку ты хочешь?

"--*Трубки все мои, и я не обязан делать чёткий выбор между ними, "--- объяснил стратег.
"--- В теории я могу закурить столько трубок за раз, на сколько хватит рта и лёгких.
Так что, пожалуйста, кинь мне какую-нибудь трубку, а эту коротышку положи обратно.

"--*Подними задницу и возьми сам.

Гало раздражённо выключил терминал и вышел из комнаты, хлопнув дверью.

"--*Как хорошо, когда есть близкий родственник, "--- Лу откинулся на спинку кресла и вперил взгляд в скучный каменный потолок.
"--- Желания курить как не бывало.

\section{[:] Мико}

\spacing

Тахиро смотрел на девушку и никак не мог насмотреться.
Она была воплощением нежности и женственности.
Сейчас, когда Тахиро обнял её, на веснушчатом тонкокожем лице расцвела милая белозубая улыбка.
Узкие глаза с такэсакским, <<шу>> раскроем, светились радостью и смущением;
девушка крепко прижимала маленькие ручки к груди, и этот жест окончательно свёл Тахиро с ума.

"--*Главное "--- не бойся, "--- улыбнулся Тахиро.
"--- Когда ты придёшь в себя, я возьму тебя в жёны.

Мико хихикнула.

"--*Правда?

"--*Правда.
У тебя будет безопасный дом.

\spacing

\section{[:] Изнасилование}

\spacing

Девушка дрожала с головы до ног.

"--*А что произошло потом? "--- спросила Айну.

"--*А потом они\ldotst меня\ldotst

"--*Изнасиловали, "--- сухо подсказала Айну.

"--*Да, "--- шёпотом подтвердила девушка.
"--- Меня и раньше\ldotst ещё когда я служила семье Таками\ldotst но в этот раз всё было кошмарно\ldotst я не представляла, что так вообще возможно.

"--*Было очень страшно, "--- подсказала Айну.

"--*Да.
Больно не было, а вот страшно\ldotst

"--*Всё ясно, "--- сказала Айну.
"--- Стимуляция.
Как и мы когда-то, они думали, что это эффективный способ.

Демон Айну вспыхнул "--- и что-то приглушённо хлопнуло.
Девушка медленно легла на стол с застывшими глазами.
Изо рта вытекла тонкая струйка крови.

"--*Зачем? "--- рявкнул Тахиро.

"--*Тебя никто не спросил, дзайку-мару, "--- отозвалась Айну.
\mulang{$0$}
{"--- Избавься от трупа.}
{``Get rid of the corpse.''}

Тахиро схватился за дайту-соро и тут же упал на землю, заорав от боли.
На его теле появилось множество мелких болезненных царапин, словно на парня в один миг набросилась стая озлобленных крыс.

"--*Я сказала "--- избавься от трупа, "--- не повышая тона, повторила Айну.

\section{[:] Пищевая цепь}

Грисвольд нашёл Тахиро не сразу.

Парень сидел за складами, рядом со свежей могилой.
Он снял с себя всю одежду, от него разило острым запахом какого-то крема, рядом лежала пустая ампула из-под обезболивающего, но Тахиро всё равно морщился при каждом движении.

"--*Я не понимаю, как она могла мне нравиться раньше, "--- пробурчал парень, пытаясь вытереть злые слёзы краем кимоно.
"--- Бесчеловечная дрянь.

"--*Она не плохая, Тахиро, "--- сказал Грисвольд.
"--- Поверь, я видел плохих демонов.
Айну не такая.

"--*Она относится к людям как к вещам! "--- рявкнул Тахиро.
"--- <<Избавься от трупа>>!
Глоточная бомба!
\mulang{$0$}
{Она забила девушку как свинью!}
{She slaughtered the girl like a pig!}
А ещё она использовала шугокскую пытку\footnote
{Шугокская пытка "--- нанесение множества мелких ран на тело. \authornote},
словно я какой-то\ldotst какой-то\ldotst

Тахиро захлёбывался от гнева.
Грисвольд вздохнул.

"--*Когда я был ещё молодым демиургом, я тоже относился к людям, как к материалу, "--- улыбнулся он.
"--- Что произошло?
Я просто начал вас изучать.
Изучал как материал, а в итоге понял, насколько мы с вами похожи.
Я никогда не относился к людям как к равным, но вот это знание, которое я когда-то получил, не позволило мне относиться к ним, как к вещам.

"--*Она же тоже\ldotst

"--*Айну "--- демон, Тахиро.
Форма жизни, отличная от вас, несмотря на то, что демоны часто существуют в связке с человеческим телом.
Она живёт уже в двадцать раз больше, чем ты.

Тахиро помолчал.

\mulang{$0$}
{"--*Я обещал Мико, что возьму её в жёны.}
{``I promised to marry Miko.''}

Грисвольд удивлённо посмотрел на могилу.

\mulang{$0$}
{"--*Ты о\ldotsq}
{``You talk about\dots''}

\mulang{$0$}
{"--*Да.}
{``Yes.''}

Тахиро прижался к прохладной стене склада и с облегчением задышал.
По его лицу снова потекли слёзы.

"--*Как же мне надоело бессилие человеческого тела.

"--*Я тебя понимаю, "--- кивнул Грис.
"--- Сотня царапин "--- и обладатель самого смелого сердца может лишь сидеть и стонать от боли.

"--*Я не стонал.

"--*Вы, люди, придумали какие-то глупые ограничения для нормальных реакций организма.
\mulang{$0$}
{Я не перестану считать тебя храбрецом из-за слёз и стонов.}
{I keep thinking of you as a brave man even if you cry and moan.}
\mulang{$0$}
{Это вообще никак не связано.}
{Actually, there is no relation.''}

\mulang{$0$}
{"--*Правда?}
{``Is it true?''}

\mulang{$0$}
{"--*Правда.}
{``It is.''}

\mulang{$0$}
{"--*Тогда можно я поплачу?}
{``May I cry a little then?''}

\mulang{$0$}
{"--*Не нужно спрашивать разрешения.}
{``You don't need to ask permission.}
\mulang{$0$}
{Можешь даже покричать.}
{You may even scream.''}

Тахиро заплакал.
Рыдания рвались из его сдавленного горла, словно дикие птицы из клетки.
Грисвольд почувствовал, что парню необходима ласка, протянул руку "--- и тут же отдёрнул, поняв, что причинит прикосновением ещё больше страданий.

<<Надеюсь, сегодня будет дуть свежий прохладный ветер>>, "--- подумал технолог, глядя в тёмное небо.
На своей планете он знал, как поднять бриз и свернуть воздух в торнадо;
на этой приходилось довольствоваться надеждой.

Ветер закружил сухие листья на голом камне.
Несколько мгновений мимо Тахиро и Грисвольда летел крохотный смерч, вбирая в себя щепки и пыль;
собеседники проводили его взглядами.
Тахиро суеверно выставил вперёд ладонь и что-то прошептал в крепко сжатый кулак.

"--*Грис, можешь сделать меня демоном?

Технолог смутился.

"--*Методика ещё сыровата, "--- уклончиво сказал он.
"--- Пока что я с командой провожу опыты на\ldotst на других людях.
Пожалуйста, не смотри на меня так.
Если эти опыты не проведу я, их проведут другие.

"--*По-твоему, это оправдание? "--- осведомился Тахиро.

"--*Да, парень! "--- рявкнул Грисвольд.
"--- Когда-то вы проводили опыты на мышах.
Сейчас человечество в мышах не нуждается.
Чем вы, люди, отличаетесь от этих зверьков?

"--*У нас есть\ldotst

"--*Ничем, "--- перебил парня Грисвольд и поднялся на ноги.
"--- Когда мы добьёмся первых стабильных результатов по оцифровке, я превращу тебя в демона.

"--*Вы за это заплатите, "--- пробурчал Тахиро сквозь слёзы.
"--- Я клянусь порогом моего дома, вы заплатите!

"--*А кто привлечёт нас к ответу? "--- криво улыбнулся Грисвольд.
"--- Ты?

Тахиро промолчал.

"--*А пока подумай над одной вещью, Тахиро, "--- Грисвольд многозначительно поднял палец.
"--- Что, если бы мыши могли привлечь к ответу вас, м?
Все тысячелетние рассуждения о справедливости человечество вело, находясь на вершине пищевой цепи.
Хорошенько над этим подумай.

"--*Тогда я верну людей на причитающееся им место, "--- запальчиво бросил Тахиро.
"--- Ваше племя вершины пищевой цепи недостойно.

Грисвольд хмыкнул и пошёл прочь, проворчав себе под нос что-то неразборчивое.

\section{[:] Извинение}

\spacing

"--*Грис, прости, "--- тихо сказал Тахиро.
"--- Я срываюсь на тебя иногда, я знаю\ldotst

"--*Я не обиделся.

"--*Ты хорошо на меня влияешь, "--- горячо заговорил Тахиро.
"--- Я чувствую, как после наших бесед с меня спадает вся эта дикость, в которой я рос и продолжаю жить сейчас.
Жестокость, презрение к людям.
Странно, но во мне этого куда больше, чем в тебе, хоть ты и не человек.

Грисвольд кивнул и углубился в свои записи.

"--*Давай сбежим, "--- вдруг зашептал Тахиро.
"--- Плевать на Орден.
Оцифруй меня, мы найдём планету и будем там богами, будем разбивать сады и возводить прекрасные сооружения.
И Лу с собой возьмём.
Я предпочту умереть, защищая последний приют, чем раздирать этот несчастный клочок камня на ресурсы.

Грисвольд невольно вздрогнул.
Это не были слова дзайку-мару;
юноша говорил как хорохито.

"--*Твоя мечта исполнится, умрёшь ты очень скоро, "--- сухо ответил технолог.
"--- Ты же слышал слова Арракиса на последнем заседании штаба.
Союз Воронёной Стали действует так же.
С нейтралами сейчас разговор короткий.
Их скоро не останется, все демоны будут втянуты в войну.
Без Ордена мы не выживем.

"--*У нас нет выбора?

"--*Нет.
По крайней мере пока, "--- Грисвольд позволил себе лёгкую улыбку.
"--- Но я подумаю над твоим предложением.
Ложись спать.

"--*Едва ли я сегодня засну, "--- буркнул Тахиро.

"--*Попроси у Лу морфин.

"--*Это зло.
Морфин превращает людей в животных, и даже одной дозы достаточно\ldotst

"--*Избавь меня от этих глупых суеверий, "--- перебил его Грисвольд.
"--- На наркоту садятся лишь рабы и люди, глухие к собственным желаниям.
Взгляни на Лу "--- он полжизни курит адский вереск раз в две-три недели, а иногда по полгода трубку в руки не берёт.
Скажешь, он такой же наркоман, как те, в поселении, у которых от трубки уже губы гниют?

"--*Отец говорил, что все, кто хоть раз вдохнул дым "--- наркоманы.
Мой дед\ldotst

"--*Твой дед познал удовольствие.
Твой отец осознал опасность.
Тебе предстоит увидеть закономерность, взвесить пользу и вред.
Именно так люди и переходили от веры к знанию.
Возьми у Лу морфин и поставь капельницу.
От одного раза ничего не будет.

\chapter{[U] Побег}

\section{[U] Просьба о прощении}

\epigraph
{Чему учит нас история Тси-Ди?
Всегда оставляйте своим врагам шанс вести достойное существование.
Любая идеология бессильна против отчаяния обречённых.
Великая Матка сказала, что ни один апид-солдат не вернётся домой, пока Молокоеды и Тараканы не будут истреблены.
<<Да будет так>>, "--- ответили тси и втоптали Ульи в песок.}
{Кельса Пушистая}

\spacing

"--*Трукхвала убили, "--- сказал Грейсвольд.
"--- Короля-жреца.

Я выругался.

"--*Кто?

"--*Наши, разумеется.

"--*Я же сказал им, что с Трукхвалом проблем не будет!
Я проинструктировал его, чтобы он сложил с себя обязанности по первому требованию!

"--*Боюсь, "--- печально улыбнулся Грейсвольд, "--- у него ничего и не требовали.

Я сделал несколько кругов по комнате.

"--*Ты уверен, что это убийство?

"--*У него на теле было несколько необычных кровоподтёков.
Ты знаешь, наши работают чисто, для сапиентов всё выглядело как обычный несчастный случай.
Однако кровоподтёки\ldotst он щипал себя перед самой смертью.
Ранить себя он не мог "--- убийцы почуяли бы кровь и заинтересовались.

"--*Это были просто щипки или\ldotst

"--*Или символ, да.
Два знака письменности тси "--- <<уголок>> и <<цепочка>>.

Модификатор пассивного залога и конец предложения.
<<Что-то, результат чего ты наблюдаешь, было сделано кем-то>>.

"--*Он пытался оставить послание жрецам?

"--*Неясно, кому предназначалось послание, но он хотел, чтобы об убийстве узнали.

"--*От кого исходил приказ?

"--*Штрой Кольцо Дыма, командующий силами Ада.
Был ли это её приказ или свыше, не знаю.

"--*Давай, Минь, объясни мне ещё и это! "--- в сердцах обратился я к архивариусу.
"--- Этот человек поднял земли сели, идолов и хака из руин.
Он в жизни не причинил вреда даже птице, не говоря уже о сапиенте.
Однако Ад уничтожает его "--- мягкого, покладистого лидера, который вёл свой народ к процветанию!

"--*С этим не всё так просто, "--- пробормотал Минь.
"--- Да, демоны Ада длительное время поддерживают процветание сапиентов, чтобы получать плюс-эманации.
Но дело в том, что неограниченная эволюция невыгодна "--- существует опасность восстания против Ада.
Тси-Ди ярко это показала "--- тси не согласились на власть демонов.

"--*И каким же образом Ад решает эту проблему? "--- глухо проворчал я, зная ответ наперёд.

"--*Апокалипсис.
Полное низвержение цивилизации, доведение сапиентов до первобытного состояния.
Затем цикл повторяется.

"--*И даже это запланированное событие используется в целях пропаганды, "--- добавил Грейсвольд.
"--- Угадай, кого в нём обвинят.
Обычная история.

"--*Так значит, именно поэтому Ад настороженно отнёсся к генофонду тси, "--- проговорил я.
"--- Тси улучшили свои тела, сделав их сверхприспособленными и быстрообучаемыми, и легко могут выйти из-под контроля.

"--*Аду невыгодно существование потомков тси в их нынешней форме, "--- резюмировал Грейсвольд.
"--- Я не думаю, что тси будут уничтожать, но отрицательный отбор неизбежен.
Возможно, Трукхвал и не представлял непосредственную опасность как личность, но он определённо обладал опасными с точки зрения Ада генами.
Ты же помнишь про отбор Королей-жрецов?

Я встал и прошёлся по комнате, теребя пальцами браслет-терминал.
Выходит, все наши старания были зря.
Мы подвергали себя риску и отправляли сапиентов на верную смерть ради ещё одной рабской хижины размером с планету.

"--*А я пообещал Митрису, что его народ будет счастлив.

"--*Аркадиу, "--- Грейсвольд, похоже, не на шутку испугался последней фразы, "--- надеюсь, ты не собираешься сделать какую-нибудь глупость?

"--*Разумеется, нет.

"--*Тогда куда ты собрался?

"--*Просить прощения, "--- сказал я и сдавил браслет.
Бесшумно разъехались двери комнаты, и я вышел наружу, в переплетение стальных коридоров.

\razd

Я знал, сколько времени требуется отделу 100, чтобы заняться мной вплотную.
Статистика по захватам агентов Картеля стоила мне больших затрат.
Тридцать секунд на обработку информации обо мне, столько же на принятие решения, минута на то, чтобы роботы контрразведки подобрались к моему телу.
Дальше ожидание и слежка, затем захват или уничтожение.

Минь уже должен был передать информацию.
Бедный архивариус вынужден был переступить через собственную гордость из-за чувства страха, не зная, что и это входило в мой план.

Вокруг стояла тишина.
Я бы даже сказал, что было чересчур тихо.
Скорее всего, контрразведка уже парила на расстоянии вытянутой руки, прикрытая всей мощью технологий Ада.

Я направился в архивы.
Существовал только один способ сбежать "--- нужно было заинтриговать отдел 100 и убедить их, что мой побег принесёт для дела больше пользы, чем моя поимка.


\section{[U] Виа}

Лифт опустился на тридцать восьмой этаж.
Вокруг по-прежнему было подозрительно тихо, и я тихо вздохнул, словно озабоченный чем-то.
На самом деле это был вздох облегчения "--- мои действия явно не укладывались в предсказанную аналитиками схему, и контрразведка медлила с поимкой.

Дверь отъехала в сторону, и я встретился глазами с молодой женщиной-плантом.
Архивариус Виа Серая Ласточка "--- так её зовут.
Она молча улыбнулась и кивнула.
Разумеется, отдел 100 уже предупредил и проинструктировал её.

"--*Здравствуй, Виа, "--- поклонился я.
"--- Мне нужна твоя помощь.

"--*Секунду, "--- сказала она и, приложив палец к виску, закрыла глаза.
Открыла.
"--- Извини, Аркадиу.
Перенаправила срочный запрос.
Слушаю тебя.

"--*Мне нужна как можно более точная временная лента последнего сантителльна существования цивилизации Вилет'марр.

"--*Лотос? "--- Виа смотрела на меня с интересом, но я обострёнными чувствами ощущал некоторую наигранность.
Логика понятна "--- люди Тра-Ренкхаля были потомками вилет'марр, и аналитики контрразведки предупредили архивариуса о таком направлении беседы.

"--*Да.
Дело в том, что некоторые аналитики\ldotst

"--*Некоторые "--- это кто?
Выражайся точнее, Аркадиу.
Я не могу тратить время на проверку источников.

Провокация.
Я был к ней готов, но знал, что мой ответ не идеален.
Если Виа или держащие с ней связь аналитики за ближайшие двадцать секунд найдут, к чему придраться "--- пиши пропало.
Они не знали, что я осведомлён о подключении к делу контрразведки, и не могли затягивать беседу "--- это выдало бы их с головой.

"--*Гимел Кадмиевый Зелёный, диктиолог Картеля, действовавший\ldotst

"--*Он погиб на Тра-Ренкхале во время твоего последнего задания? "--- прикрыв глаза, спросила Виа.

"--*Да, я его убил.

На самом деле Гимел погиб во время состязания грубой силы между Безымянным и Эйраки, но знал это только я.

"--*Ты поддерживал с ним связь? "--- поинтересовалась Виа.

"--*Информация получена в момент его уничтожения.

"--*Почему сказанное тобой расходится с отчётом? "--- снова прикрыла глаза Виа.

"--*А почему я пришёл к тебе сам, а не написал запрос по общему каналу? "--- задал я риторический вопрос.

Виа растерялась и немного испугалась.
Она поняла, что дело действительно серьёзное "--- если бы не предупреждение контрразведки, она бы и не поняла, что перед ней шпион.
Всё укладывалось в привычные рамки "--- мои человеческие повадки, мои полномочия как легата терция не раскрывать некоторую информацию до окончания собственного расследования и даже промедление в пять лет, которое было призвано охладить интерес агентов Картеля к моему последнему заданию.
Плюс репутация "--- Виа изучала досье большинства демонов и никак не могла ожидать от меня подвоха.

"--*Хорошо.
Чем располагал Гимел?
Я тебя слушаю, "--- Виа справилась с собой, и в её глазах загорелся спокойный огонь готового к работе профессионала.

<<Пока неплохо>>.

"--*Гимел вычислил время прибытия людей Лотоса на Тра-Ренкхаль.
Использовались данные археологии, астрономии и генетики.
Точность весьма высока "--- плюс-минус два года.

Археология у Картеля всегда была своеобразной "--- образцы оцифровывались и тут же уничтожались, чтобы их не изучили враги.
Культурологи Ада, среди которых были и оцифрованные сапиенты, настояли на запрете подобного вандализма.
Разумеется, без крайней необходимости.
А биологи Ада, благослови лесные духи их расхлябанность, увлеклись изучением тси и девиантной фауны, отложив в долгий ящик отчёт о людях Тра-Ренкхаля.
Я был свидетель тому, как начальника отдела распекал прибывший чин за то, что биологи не работают над проблемами совместимости демонов с местными сапиентами.
Тот смущённо оправдывался, мол, на планете чересчур много интересного.
В итоге отдел настрочил под диктовку Безымянного отписку для Капитула, и на этом изучение людей Тра-Ренкхаля завершилось.

"--*Поняла, "--- сказала Виа.
"--- Временная лента готова.
Ты хочешь ознакомиться с ней сам или я тебе ещё нужна?

Я улыбнулся и решил ещё раз погладить отдел 100 против шерсти:

"--*Если ты не занята, то я хотел бы разобрать её с тобой.

Виа несколько растерянно махнула мне рукой, призывая следовать за ней.

\section{[U] Ветер свободы}

Вскоре мы сидели в пустом кабинете перед экраном компьютера.
Виа непроизвольно вжималась в кресло, по её лбу струился пот;
похоже, что здесь был агент контрразведки, прикрытый щитом невидимости, и женщина опасалась ненароком его задеть.
За подобную неосторожность её могли лишить должности.
Эти случаи не афишировались, но о них знали все "--- почему-то решающим в вопросе профпригодности демона было слово не его коллег по отделу, а военных.

"--*Скажи мне как аналитик, в какое время вилет'марр могли послать экспедицию на Тра-Ренкхаль? "--- спросил я.

"--*Не раньше 1190-го оборота, "--- не задумываясь ответила Виа.
"--- Ранее вилет'марр просто не располагали нужными технологиями.
Я бы даже отодвинула эту границу оборотов на сорок "--- шестьдесят, с учётом психологии этих людей "--- они тяжеловаты на подъём.

Виа осторожно потянулась к плечевому импланту.

"--*Ты не против?

"--*Нет, конечно, "--- я улыбнулся и сам сделал ей инъекцию успокоительного.
"--- Тяжёлый день?

"--*Если бы только день, "--- Виа жалобно улыбнулась.

"--*Я тебя понимаю.

"--*Ты один из немногих, кто может понять.

"--*А сколько времени потребовалось бы кораблю, чтобы достичь цели?

Виа почесала кожистую зелёную головку, раскрыла и сложила венчик.

"--*Между Тра-Ренкхалем и Лотосом лежит пылевой <<блин>>, напрямую лететь они не могли.
Дело в том, что первые люди отправляли экспедиции не на конкретную планету, а на так называемое направление.
Какие-то из преобразованных планет на направлении могли быть непригодны для жизни, какие-то могли стать непригодными за время полёта, всякое происходило.
Иногда, уже прилетев на вполне годную планету и рассмотрев её поближе, люди просто отказывались высаживаться, и были в своём праве.

"--*И такое было?! "--- удивился я.

"--*Разумеется!
Планета "--- это не партнёр на одну ночь, а дом для ближайших ста поколений потомков!
Не все нюансы можно рассмотреть с расстояния в сто парсак, даже с помощью демиурга.
Поэтому, если маршрут предполагал, например, пять преобразованных планет, топлива брали на шесть циклов.

"--*Чтобы был шанс вернуться, "--- кивнул я.

"--*Хотя бы маленький, но чтобы был, да.
Это очень важный момент, Аркадиу.
Первая экспедиция на направление была без страхового цикла.
Они успешно добрались до цели, но пригодной оказалась лишь последняя планета на маршруте.

"--*Я понимаю.
Психологическое напряжение было гигантским.

"--*Мягко сказано.
На корабле начался массовый психоз, закончившийся бунтом.
Десятую часть команды пришлось изолировать до конца полёта, многие из них погибли.
Так что сам факт, что экспедиция оказалась так далеко от Лотоса, за пылевым облаком, говорит именно об этом "--- Тра-Ренкхаль был не первой и даже не третьей планетой, которую они посетили.
Дай-ка компьютер, я попробую вычислить маршрут.

Я уступил Виа место.
Плант прищурилась, и в её красивых чёрных глазах заиграли голубые огни голографического дисплея.

"--*В архивах точно нет информации о времени отлёта?

"--*Нет, "--- покачала головой Виа.
"--- Я бы и сама была рада, если бы подобная информация сохранилась.
Вот, смотри.
Кратчайшее время полёта "--- триста восемьдесят семь оборотов.
Это на максимальной скорости.
Если предположить, что по пути они останавливались для починки, дозаправки или ради научного интереса "--- выйдет все шестьсот.

Я кивнул.
Виа посмотрела на меня.
Страха в её глазах больше не было "--- только любопытство.

"--*Я, конечно, понимаю, что это секрет, но всё же "--- зачем тебе это?
Ты думаешь, что вилет'марр были где-то ещё?

"--*Я \emph{знаю}, что они были где-то ещё, "--- сказал я.
"--- Если твои данные верны, то вилет'марр двигались со сверхсветовой скоростью.
Точнее, в две световых.
И это при условии, что они, как ты говорила, никуда не заворачивали.
В чём лично я сомневаюсь.

Пластик под пальцами Виа жалобно хрустнул.
Она начала с бешеной скоростью открывать какие-то документы и просматривать их.

"--*Да не может этого быть, "--- шептала она.
"--- Вот это "--- данные Ясмер.
А это я лично вычисляла, и команда Флавика сказала, что всё верно.
Все данные, чтоб мне сдохнуть, верифицированы, и не на один раз.

Я промолчал.
Виа перестала мучить компьютер и откинулась в кресле.

"--*Ну не могли же они использовать омега-телепортацию! "--- жалобно воскликнула она.
"--- Перенос сознания при омега-телепортации невозможен, даже с синхронизацией мозг-демон-мозг.
На расстояние вытянутой руки через квантовую запутанность "--- куда ни шло, но всё, что дальше, равносильно самоубийству.
Запрет на ОТ шёл чуть ли не со времён ранней Эпохи Богов!
Зря, что ли, первые люди преодолевали космос на субсветовых?

"--*Успокойся для начала, "--- бросил я.
"--- Гимел тоже мог ошибиться.
Верификацию, насколько я понял, прошли только используемые им данные анализов.
Алгоритм, согласно которому он проводил вычисления, мне не известен.
Но меня беспокоят тси.
Случайно ли то, что некогда обладавщие высокими технологиями вилет'марр, а затем и тси в итоге оказались на одной планете?
Что, если эти две цивилизации поддерживали связь, и тси прилетели за помощью?

Виа смотрела на меня.
Я знал, что её демон нарисовал картину целого заговора.

"--*Так или иначе, история тёмная и требует анализа, "--- заключил я.
"--- Я отправлюсь на Тра-Ренкхаль и попробую что-нибудь узнать.

Виа неловко улыбнулась.
Даже архивариус поняла, что это ложь.
Гораздо более логично вначале отправиться на Лотос.
Как говорили сели, глупо искать родниковую воду на морском побережье.

Смехотворность ситуации состояла в том, что последняя фраза была единственной правдой, которую я сказал за всё время беседы.

"--*Ты знаешь, что искать? "--- поинтересовалась она.

"--*Да, "--- ответил я и пошёл к выходу.
"--- В памяти Гимела были ещё кое-какие зацепки.

Виа махнула мне рукой.
Уже у порога я услышал за спиной тихий-тихий вздох облегчения.

<<Всё-таки неплохо играть на пограничных значениях.
Сместил среднее, увеличил точность "--- формально никаких противоречий, а картина кардинально поменялась.
Надеюсь, ты достигла пристанища духов, Айну.
Ты была замечательным учителем>>.

\ldotst Дверь лифта закрылась, но никто так и не попытался меня пленить.
Я почувствовал, как мои волосы нежно ласкает ветер свободы "--- давно забытый и такой желанный.

\chapter{[:] Другая форма жизни}

\section{[:] План раскрыт}

"--*Ты собираешься оцифровать Тахиро.

Грисвольд вздрогнул и выронил ключ.

"--*Откуда ты это узнал?

"--*Я стратег, Грис.
Моя задача "--- распознавать схемы и структуры в чужих действиях.
И задача эта "--- не только моя.

"--*Кто ещё знает?

"--*Благодаря Гало "--- весь штаб Ордена.
Как ты знаешь, решение об оцифровке дзайку-мару ещё не было принято, и Гало, который всецело <<против>>, чрезвычайно заинтересовался совершенно отчётливыми приготовлениями к этому событию.
Я бы на твоём месте поспешил на заседание, которое проходит прямо сейчас.

"--*Они начали его без меня?!

"--*Ты же занят сборкой транспорта, верно?

Грисвольд судорожно выдохнул.

"--*Ты бы хоть меня предупредил.
Я успел убедить Гало, что ответственность за всё лежит на мне.
Наплёл ему красивую историю про то, как хотелось бы мне попробовать сделать демоном своего человеческого друга.

"--*Ты подставил Тахиро?

"--*А что мне ещё оставалось сказать, Грис?
Давай, придумай что-нибудь более правдоподобное, ты же у нас гений конспирации.
Меня временно отстранили от дел за самоуправство, Гало меня ненавидит, а Тахиро остаток жизни будет спать со мной в одной комнате и в одной кровати, потому что это единственное место, где у него есть шанс проснуться.
Грис, я сохранил твою репутацию, но его я больше не смогу защищать.
Свод законов Ордена никак не защищает людей.
Люди "--- это скот, чья участь "--- дойка, случка и забой.
Тахиро был жив по той причине, что он "--- моя собственность, не представляющая опасности.

"--*Что мне делать?

"--*Ты должен признаться во всём и хотя бы попробовать отстоять свою точку зрения.
Идею оцифровать именно Тахиро, равно как и карт-бланш на всё перечисленное, можешь повесить на меня.
Жизнь мальчика в твоих руках.
Единственный способ его спасти "--- сделать субъектом адского права.

Грисвольд сорвался с места с неподобающей толстяку прытью и понёсся к выходу из ангара.

"--*Лу, прошу тебя, спрячь хотя бы вон те два модуля!

Люцифер с грустью осмотрел два тяжеленных агрегата, затем бросил взгляд на свои холёные пальцы "--- и испустил тяжкий вздох.

"--*Вот какого дьявола я опять должен таскать тяжести? "--- обратился он к геликоптеру.
Геликоптер сочувственно промолчал.

\section{[:] Совещание}

\spacing

Слово взял один из учёных бывшего Ордена Тысячи Башен, Север Солнечная Дева.
Он сразу обращал на себя внимание грамотным построением речи на сохтид, тихим робким голосом и удивительно естественной мимикой.
У него единственного из всех присутствующих были очки с простыми стёклами и аккуратная подстриженная борода.
Согласно досье, Север по праву считался одним из флагманов изучения людей Тысячи Башен.

"--*Грисвольд, объясните всем, пожалуйста, в чём сложность демонизации людей.

"--*В демонизации как таковой сложности нет, это стандартный процесс снятия квантовой структуры объекта с последующей записью в омега-модуль.

"--*Какой именно объект подвергается оцифровке?

"--*Нервная система вплоть до чувствительных окончаний, нервно-мышечных синапсов и некоторых нейрогуморальных структур.

"--*Почему необходимо снятие отпечатка всей нервной системы?

"--*В модуляции сигналов участвуют все указанные звенья.
Головной мозг, даже вместе со спинным, не самодостаточны.
Разработанный нами интерфейс монтируется к цифровым аналогам нервно-мышечных синапсов и чувствительных окончаний.

"--*Я прошу прощения, коллега, но известны случаи, когда люди управляли машинами с помощью нейроинтерфейса, подключённого напрямую к коре, "--- Север вопросительно посмотрел на Грисвольда поверх очков.

"--*Сравнение некорректное, "--- поклонился Грисвольд.
"--- Речь шла о подключении устройства к живому организму, а не о полной замене живого организма квантовой моделью.
В упомянутом вами нейроинтерфейсе сигнал с коры подвергается сложному процессу стабилизации и модуляции сигнала, как это происходит в нижних отделах нервной системы.

"--*Благодарю за пояснение.
Каким образом обеспечивается жизнеспособность квантовой модели?

"--*Модель помещается в идеальную виртуальную среду с эмуляцией поступления необходимых веществ.
В будущем среда может быть упразднена, если преобразовать квантовую модель мозга в самодостаточную математическую модель.

"--*Это возможно?

"--*Полагаю, да.

"--*Возможно ли пристыковать к модели нервной системы вычислительные модули, модули чувствительности и устойчивости?

"--*Я переформулирую ваш вопрос и отвечу на него: да, квантовая модель мозга человека совершенно точно может взять на себя роль программного ядра омега-сингулярности, со всеми вытекающими последствиями.

"--*Благодарю за поправку.
И вы считаете, что это ядро более эффективно, нежели написанное с нуля?

"--*На данном этапе сравнивать довольно сложно.
Преимущества оцифрованного мозга "--- дешевизна, уникальность строения, сложность прямого взлома.
Недостатки "--- сложность настройки и гораздо меньшая пластичность.
Проще говоря, это дешёвый одноразовый аналог демона "--- с неплохим запасом устойчивости и интеллекта, но с низким потолком чувствительности и проигрывающий в быстродействии.

"--*Насколько проигрывающий?

"--*На два порядка.

"--*Два порядка "--- это очень много.
Это значит, что оцифрованный мозг нанесёт один удар там, где обычный демон успеет нанести сто!

"--*Тем не менее некоторые наши специалисты по вооружению считают, что с таким запасом устойчивости это простительный недостаток.
Кроме того, совершенно необязательно делать из оцифрованных людей солдат.
\mulang{$0$}
{Есть множество профессий, которые не требуют скорости.}
{There are many professions that don't need speed.}
\mulang{$0$}
{Я своё пузо отрастил не просто так.}
{My belly's grown too fat for a reason.''}

Шутка предназначалась Гало.
Но стратег даже не улыбнулся.
В его глазах светился гнев.
Зато Север, к удивлению Грисвольда, отреагировал на юмор вполне адекватно.

\mulang{$0$}
{"--*Хорошего специалиста должно быть много, "--- с улыбкой сказал он, поправляя очки.}
{``You've got tons of experience, I guess,'' he said with a smile and propped his glasses.}
"--- А какова скорость естественного износа квантовой модели?

"--*Примерно в пять раз медленнее таковой в живом человеке.
Согласно прогнозам, износ математической модели можно довести до шестидесятитысячекратного замедления.

"--*То есть максимальное время жизни оцифрованного человека приблизится к времени жизни демона?

"--*Именно.

"--*Чрезвычайно интересно, "--- кивнул Север. 
"--- Какие задачи стоят перед вами сейчас?

"--*На данный момент единственная большая задача "--- отработка методики.
Многое зависит от особенностей конкретного человека, его способностей и психологических параметров.
Из некоторых получаются живые, но абсолютно неработоспособные демоны с узким каналом ввода-вывода.
У других канал сохранен, но они надламываются психологически или сходят с ума.

"--*Можно ли решить это тренировкой?

"--*Разница статистически незначимая.
У некоторых исследователей создалось впечатление, что значение более имеет мотивация, нежели тренировка.
Наши подопытные, разумеется, шли на процесс под угрозами, шантажом и обманом.

"--*Каков процент полноценных выживших?

"--*Два процента.

"--*Есть ли какая-то разница по возрасту и полу?

"--*Выживаемость детей выше в десять-пятнадцать раз, но после оцифровки они останавливаются в развитии.
Вероятно, это можно решить, но не на данном этапе исследований.
Мужчины дают полноценных демонов чаще женщин, при этом так же чаще женщин сходят с ума.
Разница между представителями разных этносов статистически незначима.

"--*Я рискну предположить: не связана ли выживаемость детей с мотивацией?

"--*Возможно.
Мы рассказывали им сказку о том, что они станут героями после процедуры.
Многие сами ложились в капсулу и терпеливо переносили неприятные ощущения.

"--*То есть Тахиро та Ханаяма в некотором роде идеальный объект?

"--*Он приближен к идеалу настолько, насколько это возможно.
Его мотивация не вызывает сомнений.
Кроме того, он успешно интегрировался в наше общество и даже выполнял задания, соответствующие рангу второго центуриона.
Это повышает его шансы на социализацию после оцифровки.

"--*У меня просьба к руководству Ордена, "--- Север повернулся к Арракису.
"--- Я, если честно, впервые услышал об этом исследовании только здесь.
Люди "--- это моя жизнь.
Мне бы очень хотелось присоединиться.
Я знаю о своём нынешнем статусе, поэтому прошу в присутствии всех "--- сделайте мне одолжение.

"--*Если исследование решено будет возобновить, не вижу причин вам отказывать, "--- кивнул Самаэл.
"--- Ещё вопросы будут?

"--*У меня вопросов больше нет, "--- поклонился Север и сделал шаг назад.

"--*Всё это очень трогательно, "--- сказал Гало, "--- однако я ещё раз обращаю внимание присутствующих на то, что люди "--- враждебный вид.

"--*Ты сомневаешься в Тахиро? "--- напрямик спросил Грисвольд.

"--*Он уже пытался вести подрывную деятельность в Ордене.

"--*Он был ребёнком, потерявшим семью.

"--*Мои тесты ясно показывают, что он продолжает чувствовать связь со своим видом.
Где гарантии, что это не повторится, едва в распоряжении Тахиро окажутся методы и вычислительные мощности?

"--*Тахиро "--- верная душа, "--- сказал Грис.
"--- Мы можем, разумеется, начать искать возможных предателей.
Вычислять, ставить опыты, проводить тесты, гадать на камнях.
Но единственный верный способ вычислить предателя "--- это дать ему в руки оружие и встать с ним плечом к плечу.
И даже тогда можно ошибиться.

"--*Я думаю, что Грисвольд прав, "--- сказала Айну.
"--- Гало и Люцифер, безусловно, мастера стратегии, но всё-таки теоретики и недооценивают значение опыта.
Тем не менее, несмотря на высказанное ранее мнение, что Тахиро уже принимал на себя полномочия второго центуриона, я категорически против присвоения ему этого ранга.
Пусть начнёт легионером, как и все демоны в Ордене.
Время покажет, чего он стоит.

"--*Я категорически против, чтобы Тахиро был солдатом вообще, "--- заявил Грисвольд.
"--- Ты меня неправильно поняла.
Возможно, его захотят принять в какой-то исследовательский отдел?

"--*Разве что как подопытный образец, "--- буркнул Гало.
В толпе представителей раздался приглушённый смех.

"--*Лично мне всё ясно, "--- буркнул Арракис.
"--- Подготовьте необходимое оборудование, этот человек будет оцифрован завтра и "--- в случае успеха "--- произведён в легионеры.
Заседание окончено, все свободны.

\section{[:] Нарэ}

<<Не самый плохой вариант>>, "--- мысленно заключил Люцифер и углубился в созерцание карты стратегии.

\mulang{$0$}
{"--*Тебя отстранили от дел, "--- напомнил Грисвольд, увидев Люцифера за обычной работой.}
{``You got suspended,'' Griswold reminded when he saw Lucifer doing his normal work.}

\mulang{$0$}
{"--*Я знаю.}
{``I know.''}

Грисвольд улыбнулся, неожиданно отвесил короткий поклон и вышел, аккуратно закрыв за собой дверь.
\mulang{$0$}
{Он понял.}
{He realized.}

Карта Лу была простой и понятной.
Никаких путей отступления, никаких фронтов атаки.
Все вариации строились на том, что из каждого развития событий можно извлечь выгоду.
Пауку совершенно всё равно, в какую часть паутины попадёт муха, он в любой случае сможет запеленать добычу.

Лу очень любил пауков.
При встрече с золотопрядом изменяла даже его вечная брезгливость. 
Карта стратегии действительно напоминала искусно сплетённую паутинку;
Люцифер не удержался и нарисовал в углу терпеливо ждущего паучка.

Карта стратегии Гало была гораздо сложнее.
Неожиданные повороты, сложные многомерные пути развития событий.
На бумаге карта выглядела как детские каракули.

<<Ну вот, брат, у каждого из нас теперь своя карта стратегии, "--- с грустью подумал Лу.
"--- До чего мы дожили>>.

Люцифер подошёл к окну и положил разгорячённые ладони на прохладное стекло.
Это был <<нарэ>> "--- своеобразный минутный отдых от работы.
Качество стекла оставляло желать лучшего: волнистая поверхность, полупрозрачные белые пятна примесей, паутинка трещин по углам, словно сплетённая каким-то особым стеклянным пауком.
Однажды Лу застал Тахиро в такой же позе;
тот не отрываясь глядел на стекло.

"--*Что ты рассматриваешь? "--- спросил стратег.

"--*Нарэ, "--- ответил Тахиро.

"--*Мне это слово не знакомо.

"--*А я не смогу объяснить смысл.
Поэтому, если хочешь понять, то просто смотри.

Лу добросовестно ходил и смотрел каждые несколько дней.
Он вглядывался в стекло, он изучил каждую волну поверхности, каждый дымчатый извив в его толще, каждую трещинку и каждое пятно на поверхности.
Смысл слова так и остался неизвестным;
однако Лу заметил, что странный ритуал неплохо помогает приводить мысли в порядок.

В этот раз что-то было не так.
Стекло недружелюбно вибрировало, хотя сейсмостанция не обещала подземных толчков.

Лу едва успел убрать со стекла ладони, как оно разбилось вдребезги.

<<Да чтоб вас! "--- подумал Лу, растерянно глядя на осколки и текущую по запястьям кровь. "--- Ну и как я теперь найду это нарэ?>>

Почти сразу же взвыли сирены.

\section{[:] Броня}

\spacing

"--*Последние диверсии совершили люди поселений.
На экстренном заседании выступал Самаэл.
В общем, он убедил штаб в том, что оцифровывать человека опасно, что мы можем сами дать людям в руки оружие, которым нас можно уничтожить.
Извини, Тахиро.
Решение окончательное.

"--*Я вам верен, Грис!
Я ни за что не ударю в спину тебе или Лу!

"--*Я не сомневаюсь, дитя.
Но Арракиса ты в этом не убедишь.

"--*Ему нельзя здесь оставаться, "--- сказал Лу.
"--- Его нужно\ldotst

"--*Мне тоже нельзя здесь оставаться, "--- прервал его Грисвольд.

"--*Ты о чём?

"--*Гало знает о твоей фальсификации.
Он намекнул, что может заявить об этом в любой момент, если я не окажу ему услугу.

"--*Он тебя шантажировал? "--- Лу выглядел ошарашенным.
"--- Что он хотел?

"--*Он хотел, чтобы я передал сообщение одному агенту.
Я согласился.

"--*Покажи.

Лу и Грис соприкоснулись головами.

"--*Он хочет скомпрометировать тебя связью с Союзом, "--- сказал Лу.
Его трясло.
"--- Брат, до чего ты дошёл?

"--*Лу, что мне делать?

Лу вскочил и сделал несколько кругов по комнате.

"--*Нам срочно нужно оборудование для оцифровки.
Только так мы сможем спасти Тахиро и доказать свою невиновность.
Тахиро, иди в мою комнату и никуда не выходи, пока я тебя не позову.
Грис, твои железяки спрятаны в ячейке номер тринадцать в третьем ангаре.
Сообщение агенту я передам сам.
Надо узнать, насколько далеко Гало зашёл в желании нас скомпрометировать.

"--*Лу, ты подставляешь сам себя!
Ты ещё можешь\ldotst

"--*Я слишком долго вас искал, Грис.
Я не хочу вас потерять.

Лу подхватил со стойки бронежилет, накинул плащ и бегом бросился за дверь.

\section{[:] Черепаха}

\spacing

Деревня выглядела так, словно она сошла с праздничной открытки.
Она была совсем не той, которой стратег помнил её с детства.

Лу подумал "--- и свернул с намеченного маршрута.

Почти сразу он понял, что сделал всё правильно.
На него стали неприкрыто пялиться, из углов понёсся недружелюбный шёпот.
Вечерняя деревня стала такой, какой ей надлежало быть "--- сумеречной и настороженной.

Поворот, улица, поворот, улица, поворот, узкий проулок, ещё два поворота\ldotst

У одного из домов несколько женщин гнали спиртное.
Старушка, похожая на одетую в кимоно черепашку, проводила Лу взглядом.
Её примеру последовала вторая, вытягивая морщинистую черепашью шею.
Едва Лу скрылся за поворотом, как обе тут же занялись своими делами.

Впрочем, запах не был похож на спирт.
Лу вдруг замер столбом.

<<Жидкое взрывчатое вещество>>.

Стратег попятился и осторожно заглянул за угол.
В нос ему уткнулся клинок.
Ещё пара осторожно залезла под нижний край бронежилета.

"--*Что ты здесь вынюхиваешь, хорохито? "--- осведомился низкорослый человек.

"--*Авамори\footnote
{Авамори "--- крепкий спиртной напиток Преисподней, около 56 градусов. \authornote}
уж больно хорош, "--- попытался пошутить Лу.

Он ожидал удара или выстрела, но клинки неожиданно отодвинулись.
Окружившие его люди зашептались.

"--*Он что, пошутил?

"--*Это была шутка?

"--*Его подкупили.

"--*Да нет, это один из этих, которых они выкормили, как поросят!

Кто-то сорвал с Лу капюшон.

"--*Это не тот!
Он не толстый и не лысый!

"--*Гляньте, он глаза красит и волосы!

"--*Похож на актёра.

"--*Да какой актёр?
Он в броне!

"--*У него трубка в кармане!
А в ножнах "--- ножницы!

Толпа зашепталась ещё тревожнее.

"--*Так ты не хорохито? "--- наконец спросил его кто-то.
"--- Ты из театральной труппы?

\mulang{$0$}
{"--*Вы мне всё равно не поверите, "--- ответил Лу, осторожно поглядывая по сторонам.}
{``You won't believe me anyway,'' Lu answered trying to look around.}
\mulang{$0$}
{"--- Поэтому объясните для начала, почему вы спутали меня с хорохито.}
{``So explain first, what made you mistake me for horohito.}
\mulang{$0$}
{Попутно объясните, почему вы вдруг засомневались.}
{You may explain, by the way, what made you question then.''}

\mulang{$0$}
{"--*Вопросы здесь задаём мы, "--- заявил человек, говоривший первым.}
{``That's we who ask questions,'' said the man who started talks.}

\mulang{$0$}
{"--*Отнюдь, "--- лаконично возразил знакомый голос.}
{``Mistake,'' a familiar voice succinctly argued.}

Лу улыбнулся.
Тахиро был замечательным учеником.
Сказанное в общей модуляции слово хлестнуло по ушам людей, словно усаженный шипами бич.
Враги подняли руки с такой скоростью, словно от этого зависела жизнь всего их рода.
Клинки посыпались на утоптанную землю.

"--*Сайгон, быстро в лагерь, "--- распорядился Тахиро, щёлкнув предохранителем винтовки.
"--- Грис ушёл в сторону третьего ангара.
Транспорт в квартале.
Я прикрою.

Лу растолкал ошарашенных дзайку-мару и бросился бежать, не дожидаясь пояснений.

\section{[:] Побег}

\spacing

"--*Что ты здесь делаешь?

"--*Тот же вопрос у меня к тебе, "--- парировал Грисвольд.
Вначале он хотел ответить распространённой в человеческих поселениях похабной рифмовкой, но удержался, зная, что Мефистофель его не поймёт.
"--- Арракис приказал мне обеспечить отсутствие прочих демонов здесь.
\mulang{$0$}
{Уходи.}
{Go away.''}

"--*Я получил от него похожий приказ, "--- заявил Мефистофель.
\mulang{$0$}
{"--- Уйти придётся тебе.}
{``It's you who must leave.''}

Грисвольд впился прищуренным взглядом в Мефистофеля.
Тот смотрел на технолога спокойными пустоватыми буркалами.

\mulang{$0$}
{"--*Уйди, Меф.}
{``Go away, Mef.}
\mulang{$0$}
{Говорю последний раз.}
{This is your last warning.''}

По лбу Грисвольда струился пот;
он прекрасно понимал "--- истолкуй интерфектор его слова как реальную угрозу, и от омега-модуля толстяка не останется даже воспоминаний.
Но Мефистофель вдруг улыбнулся "--- идеально ровной машинной улыбкой.

\mulang{$0$}
{"--*Ты пытаешься испугать меня.}
{``You're trying to scare me.}
\mulang{$0$}
{Это признак слабой позиции.}
{It's a sign of weak position.}
\mulang{$0$}
{Возвращайся к Арракису, уточни полученный тобой приказ и не трать моё время.}
{Return to Arrakis, request clarification on the order and stop wasting my time.''}

Мефистофель отвернулся "--- ровно настолько, чтобы не видеть, как Грисвольд аккуратно взялся за браслет.
Вспышка "--- и громоздкое, с туповатым лицом тело Мефа неуклюже село на пол.

Грисвольд, демиург по воле судьбы и творец по призванию, ужасно не любил что-либо разрушать.
Насмотревшись всевозможных погребальных ритуалов у людей, он вдруг решил сказать пару слов над поверженным им противником.

"--*Мефистофель из клана Мороза, ты был лучшим, "--- начал он.
"--- Ты был лучшим\ldotst

Пауза явно затянулась.

\mulang{$0$}
{"--*Понятия не имею, был ли ты хоть в чём-то и хоть для кого-то лучшим, "--- растерянно закончил технолог и повернулся, чтобы уйти.}
{``I've no idea if you were the best at something or to someone,'' confused Griswold finished, then turned around to leave.}
Его встретил ошалелый взгляд Люцифера.

\mulang{$0$}
{"--*Зато честно, "--- признал парень.}
{``You're honest, anyway,'' Lu declared.}
Разумеется, его демоническая сущность уже произвела подробный анализ и без того несложной ситуации.
"--- За что ты его так, Грис?

"--*Потом объясню, "--- проворчал технолог.
"--- Тахиро спит?

"--*Нет, "--- раздался голос из темноты.

"--*Так и знал.
Быстро за мной, и желательно придержать вопросы до более подходящего момента.

\section{[:] Плач новорождённого}

Одной из самых остро стоящих проблем на Преисподней, вне зависимости от региона, является чистая вода.
Добыча чистой воды и постройка водопроводов "--- наукоёмкий процесс, тайны и нюансы которого люди Преисподней бережно передавали из поколения в поколение.
До тех пор, пока не пришёл Орден.
Орден всегда умел вносить изменения в привычный уклад жизни.

С акиямским водопроводом связана интересная история.
Раньше два соседних города на западном крае Такэсакского ущелья, Акияма и Киба, считались захолустьем из-за почти полного отсутствия питьевой воды.
Жители шли на ухищрения "--- собирали дождевую воду, перегоняли испорченный вулканическим пеплом снег, искали под многометровыми серными ледниками чистые озерца и буквально руками прорубали к ним узкие тоннели.
В Акияма слыхом не слыхивали об открытых полях, культуры выращивались только в герметичных теплицах.
Всё изменилось благодаря легионеру Ордена, Тако из клана Дорге.
Несмотря на ранг и достаточно низкий интеллект, он был одержим инженерией.
Когда Тако вместе с его подразделением перебросили в Такэсако, он был поражён величием старинного такэсакского акведука, снабжавшего питьевой водой город с населением в сто тысяч человек.
В Акияма же его талант нашёл применение.

Арракис отказался тратить ресурсы на новый водопровод.
Тогда Тако сделал то, чего делать было нельзя "--- обратился к дзайку-мару.

Чертежи Тако были многократно скопированы и спрятаны у самых разных людей, и изъять все так и не удалось.
Женщины обжигали кирпичи во время каждого приготовления пищи.
Цемент носили по горсти в каждом кармане, камни "--- в каждом дорожном мешке.
Строительство велось ночами, даже в праздничные дни.
Не помогали ни показательные казни, ни шантаж.
В первый год Арракис несколько раз велел подрывникам разрушить главную развязку, но дзайку-мару с упорством муравьёв, не жалуясь и не поднимая шума, отстраивали её вновь.
За тридцать лет люди Акияма завершили строительство.
Первая чашка чистой воды в Киба была выпита за год до рождения Люцифера.

В конце концов Дорге убедил Арракиса, что преследовать строителей бесполезно.

"--*Жаждущего на пути к роднику можно остановить только пулей, "--- сказал он.
"--- Их неповиновение "--- вынужденная мера, а не прихоть оппозиционно настроенных слоёв населения.
В общем и целом водопровод не представляет непосредственную угрозу для Ордена.
А вот демонами, которые позволили себе подобные выходки, следует серьёзно заняться.

Тако всё-таки успел уйти на Тысячу Башен вместе со своим подельником, где и прослужил без особых происшествий до недавнего времени.
После присоединения Тысячи Башен к Ордену Преисподней Тако пропал бесследно.
Едва ли удача улыбнулась легионеру дважды;
Люцифер склонялся к мысли, что пропасть ему помогли.
\mulang{$0$}
{Орден ничего не прощал и не забывал.}
{The Order forgets nothing and forgives nothing.}

"--*Вот такая история, "--- закончил свой рассказ Люцифер.
В его красивых глазах отражались искры костерка.

"--*Я ни разу про это не слышал, "--- сказал Грисвольд, оглядывая величественные извивы водопроводной развязки, подсвеченные неярким рассеянным светом полуденной Звезды.
"--- Как жаль, что я не встретил Тако раньше.
Его можно было взять в нашу команду.

"--*Я тоже об этом думал, "--- не без сожаления признал Люцифер.
"--- Отец порой ужасно расточителен.

"--*Вы странные, "--- сказал Тахиро.
"--- У меня ощущение, что вы совсем-совсем не отличаетесь от нас.
Но вас как будто постоянно что-то гнетёт.
Когда мы пробирались сюда через скалы, я думал, что нет участи хуже моей: позади "--- смерть, впереди "--- мучения и, возможно, тоже смерть.
Но для меня это лишь плохой день, всего один, хоть и последний, а для вас "--- обыденность.

Лу и Грис с изумлением уставились на парня.
Однако тот занялся костром и не счёл нужным объяснять свои слова.

\razd

"--*Кстати, сегодня мой день рождения, "--- сообщил Тахиро.

"--*Что? "--- не понял Лу.

"--*У нас есть традиция "--- отмечать число и месяц рождения каждый год.
Маленький праздник для одного человека, его друзей и родных.

Люцифер окинул взглядом заброшенный лагерь и унылый пейзаж Акияма в долине.

"--*Я не знал.
Сочувствую.

"--*Каждый день рождения, который я помню, был таким же невесёлым, "--- буркнул Тахиро.
"--- Моих родителей освежевали в этот же день.
Видимо, впервые увидев свет, после первого вдоха человеку должно плакать, а не смеяться.

"--*Зато ты родишься в этот день дважды, "--- попытался приободрить его Лу.

"--*Или проживу целое количество лет, "--- закончил Тахиро.
"--- Я ж не дурак, Лу.
Я знаю, на какой риск иду.

Лу помолчал и поковырял пальцем землю.

"--*Ты всё ещё сердишься на меня из-за родителей?

Тахиро удивлённо посмотрел на друга.

"--*Нет, конечно.
Я перестал сердиться в тот момент, когда ты сказал, что не хотел их смерти.
Я понял, что ты сожалеешь.
Но удивило меня то, что в твоём сожалении не было ничего морального или религиозного "--- это было искреннее сожаление того, кто случайно наступил на красивый цветок.
Ты увидел в моих родичах то, чего не замечал даже я "--- красоту.

"--*Я бы сказал "--- жизнеспособность.

"--*Это одно и то же.
И меня ты освободил не из-за каких-то принципов или догм "--- тебе было противно смотреть на меня связанного, это оскорбляло твоё чувство прекрасного.

"--*На трупы и больных до сих пор смотреть не могу, "--- признался Лу.

Друзья захихикали.
Впрочем, веселье быстро утихло.

"--*А что делают для человека, у которого день рождения? "--- спросил Лу.

"--*Ему дарят подарок и кормят сладостями.
Но мне нельзя ничего есть.

Люцифер подумал и обратился к Грисвольду:

"--*Ты приготовил ему модули?

"--*Да, разумеется.
Интерфейс ввода-вывода и стандартные заготовки модулей легионеров.

"--*Если всё получится, скопируй и пристыкуй ему мои.
Так, как они есть, с актуальными обновлениями.

"--*Ты серьёзно?!

"--*Ему нужен подарок, "--- грустно улыбнулся Лу.
"--- А модули "--- это всё, что у меня есть.

Грисвольд выглядел ошарашенным.

"--*Это же самые отработанные стратегические методики, большая часть накопленных человечеством\ldotst

"--*Я знаю.

"--*Отец тебя\ldotse

"--*Я знаю, Грис!

"--*Под твою ответственность, сайгон, "--- развёл руками технолог.
"--- Но всё-таки\ldotst

"--*От меня не убудет.
Модули "--- это не яблоки.

"--*Спасибо, Лу, "--- поклонился Тахиро.
"--- Я не совсем понимаю, что именно ты хочешь мне подарить, но, думаю, это прекрасный подарок.

Лу хитро осклабился:

"--*Ты \emph{совсем не понимаешь}, что я хочу тебе подарить.
Грис немного понимает, поэтому он и\ldotst

"--*Ладно, хватит болтовни, "--- оборвал друга Грис.
"--- Тахиро, нужду справил?

"--*Десять минут назад.
Не ел четырнадцать часов, как ты и хотел.
Выпил глоток подсоленной воды около часа назад.

"--*Тогда можем приступать.

"--*Анестезию будем делать? "--- спросил Лу.

"--*Никакой анестезии, "--- буркнул Грис.
"--- Он должен быть в полном сознании и ясной памяти, чтобы устройство захватило активность всех участков нервно-мышечной системы.
Если он потеряет сознание в процессе, придётся приводить его в чувство.

Тахиро побледнел.

"--*Ты готов? "--- обратился Лу к другу.

Тахиро кивнул и, сбросив одежду, лёг на жёсткую станину.
Его била крупная дрожь.
Грисвольд прикрутил конечности юноши к кольцам, жёстко зафиксировал голову винтами и вытащил из кармана ларингоскоп.

\mulang{$0$}
{"--*Выполняй мои команды.}
{``Obey my commands.}
Если я скажу <<согни палец>> "--- ты сгибаешь.
Если я скажу <<расслабься полностью>> "--- делай что хочешь, но выполняй, от этого зависит качество оцифровки периферических нервов.
Если тебе вдруг покажется, что ты ослеп, оглох, потерял способность говорить или дышать, или что у тебя отнялись руки, ноги или лицо, или что тебя вдруг начали преследовать звуки, запахи или образы "--- постарайся успокоиться и жди, мы смонтируем тебя как можно скорее.
\mulang{$0$}
{Помни "--- я и Лу рядом и ни за что тебя не бросим.}
{Remember: Lu and me both are near by you, and we'll never foresake you.}
А теперь открой рот как можно шире и приготовься "--- будет больно.

Грисвольд вставил в рот юноши расширитель.
Тахиро зажмурился.

\chapter{[U] Визит командующего}

\section{[U] Милая девочка Штрой}

"--*Атрис, ты дурак, "--- смеялась Митхэ.
"--- Разве можно так шутить над живыми существами?

"--*Я не шутил, "--- обиделся Атрис.
"--- Просто хотел проявить оригинальность.
Билатеральных существ как песка на пляже.
А вот с трилучевой симметрией почти нет.

"--*Они очень милые, "--- заверила Митхэ друга.
"--- Хорошо, что сейчас, после переговоров с дельфинами, они размножились и могут покинуть Коралловую бухту.

Атрис пожал плечами и снова стал наблюдать за малышами-нгвсо, которые резвились у самого берега.
Кажется, один нашёл большую красивую ракушку и совсем не хотел делиться с остальными.
Он плавал кругами, и время от времени его щупальца били по воде, поднимая тучи искрящихся брызг.
Под водой царил весёлый гул, щелчки и повизгивание, но слуха сухопутных сапиентов не достигало ничего.

"--*Может, покормим их? "--- предложила Митхэ.
"--- Они любят вишню.
И другие фрукты тоже.

"--*Лучше не надо.

"--*Это же дети, Атрис! "--- Митхэ подёргала менестреля за рукав.
"--- Детей можно\ldotse

"--*Не стоит.
Я стараюсь как можно меньше вмешиваться в их жизнь.

Митхэ обняла Атриса и уткнулась кудрявой головкой ему в шею.
Цветы в её волосах потускнели, попав в тень.

"--*Я скучаю по нашему хранителю, милый.

"--*Мы почти его не знали, "--- напомнил ей Атрис.
"--- И потом, он был инкарнацией демона Аркадиу Шакал Чрева.

"--*Это совершенно не мешает мне скучать по ним обоим.

"--*Да, мне тоже, "--- признался Атрис.
"--- Это я глупость сморозил.
Интересно, кто это?
Похоже, что к нам.

"--*Может, Корхес или кто-то ещё из биологов, "--- предположила Митхэ.

"--*Они придут вечером с конфетами, "--- возразил Атрис.
"--- Намечается что-то интересное.
Корхес хотела поговорить насчёт нескольких видов мух.
А Кольбе и Рабе обещали показать странную диатомею, которая строит алмазные раковинки.

"--*Алмазные?
В жидкой воде, при обычном атмосферном давлении?

"--*Да.
И, представь себе, я тут вообще ни при чём.

"--*Сама?

"--*Сама.

По мокрому песку шла девушка-оцелот, хрупкая и нежная на вид, почти ребёнок.
Её босые тонкие ноги утопали в мокром песке, комбинезон из сложного полимера был расстёгнут, обнажая костлявую грудную клетку и верхушки маленьких грудей.
Атрис хмыкнул "--- эта неестественная, показная соблазнительность выдавала чужаков, инкарнированных демонов Ада.
Люди сели, привыкшие к наготе тела, не считали нужным акцентировать внимание на груди или половых органах.

Девушка подошла поближе, и все сомнения на её счёт рассеялись.
Пухлые детские губки кривились в уверенной взрослой улыбке, а оранжевые опалесцирующие глаза светились древностью и памятью бесчисленных жизней.
Она подошла к парочке и по очереди пожала им руки.

"--*Штрой Кольцо Дыма, командующий силами Ада на Тра-Ренкхале, "--- представилась девушка, не забыв сделать в сторону Атриса лёгкое целующее движение губами.

Митхэ поморщилась.

"--*У тебя проблемы с гормонами?
Ведёшь себя странно.

"--*Я ещё не совсем освоилась с телом тси, "--- сообщила девушка.
"--- Возможно, мне следует поменять паттерны поведения, они не годятся для общества сели.
Вы не первые, кто это заметил.
На мужчин других миров моё поведение действует завораживающе, а мужчины-сели только смеются и не желают со мной общаться.
В общем, прошу понять и простить.
Мы можем поговорить?

"--*Да, разумеется, "--- сказал Атрис.
"--- Недалеко есть беседка.
Ты не откажешься от чашечки холодного травяного отвара?

"--*Буду очень признательна, "--- махнула длинными ресницами Штрой.
"--- Сегодня жарко.
Правда, вам, голограммам, это сложно понять.

\section{[U] Чай}

Вскоре в беседке из чёрного дерева запахло пряным паром отвара.
Митхэ плеснула в чашу жидкого азота из спрятанной под скамьёй азотной станции, и седые клубы волнами разбежались по узорчатой поверхности стола.

"--*Благодарю, "--- улыбнулась Штрой и приняла прохладную чашу.

"--*Мы тебя слушаем, "--- кивнул Атрис.
"--- К нам редко заходит кто-то, кроме биологов.

"--*Кольбе и Рабе сегодня не придут, "--- сообщила девушка, отвечая на завуалированный вопрос.
"--- У них появились срочные дела.

Митхэ хмыкнула.

"--*Неужели разговор настолько важный, что его нужно тянуть до позднего вечера? "--- нахмурился Атрис.

"--*Вы торопитесь?

Штрой наклонилась, чтобы отхлебнуть отвара, и щёлкнула языком от удовольствия.

"--*Вкусно.
Благодарю вас.
Итак.
Знаете ли вы Аркадиу Шакала Чрева?

"--*Излишний вопрос, "--- поморщилась Митхэ.
"--- Аркадиу "--- наш друг, и его последнее тело было нашим с Атрисом хранителем.
Думаю, ты знаешь это не хуже нашего.
С ним что-то случилось?

"--*Да, случилось.
Он предал Орден Преисподней и сбежал с Капитула в неизвестном направлении.

Митхэ ахнула.

"--*Что значит <<в неизвестном направлении>>? "--- осведомился Атрис.
"--- Кажется, ваша контрразведка знает всё "--- кто, куда и с какими намерениями сбегает.

"--*Он сумел подсунуть контрразведке простую, но весьма правдоподобную легенду, "--- ответила Штрой.
"--- Обман был раскрыт минуту спустя, отдел 100 направил мне официальный запрос по работе биологов.
За это время Падальщик успел сбросить хвост и исчезнуть.

"--*Какие бы силы ни стояли за твоей спиной, осторожнее выбирай слова, "--- невыразительно проговорила Митхэ.
"--- Предатель Аркадиу или нет, он хорошо послужил Ордену.
Тебе стоит уважать его хотя бы поэтому.

Штрой улыбнулась.

"--*Если угодно.
Так или иначе, след потерян.
Ожидалось его появление в трёх системах "--- Лотоса, Тра-Ренкхаля и Тси-Ди.
Агенты всех трёх пока молчат.

"--*Почему именно эти три?

Штрой пожала плечами и отхлебнула отвара.

"--*Что произошло? "--- спросила Митхэ.
"--- Почему он предал Орден?
Кому он теперь служит?

"--*Вполне возможно, что никому, Митхэ ар’Кахр.
Некоторые аналитики считают, что он помешался, несмотря на коррекцию личности "--- стигмы помешательства в его поведении присутствуют.

"--*Из-за чего он мог помешаться? "--- удивилась Митхэ.

"--*Есть некоторые \emph{подозрения}, "--- Штрой едва заметно сделала акцент на последнем слове.
"--- Каждое новое тело изменяет демона, это давно доказано.
Собственно, коррекция личности и направлена на нивелирование этих изменений.
Возможно, что тело Ликхмаса ар’Люм вызвало тонкий, но тем не менее серьёзный системный сбой в Аркадиу.
Его товарищи признаны годными к дальнейшей деятельности, хотя я бы не спешила выносить такой вердикт.

Митхэ всё больше раздражала её манера общения.
Всё в молодой девушке, даже её улыбка, было вызывающе сексуальным, бросающим вызов ей и Атрису.
Вскоре Митхэ поняла, что именно её раздражало.
Движения Штрой казались идеальными, отточенными до предела.
Это были движения машины, хоргета, но никак не милой девушки-сели, которая по воле судьбы несла этого хоргета на себе.
Демон не доверял своему телу и контролировал его во всём.
В этом было его отличие от тех демонов, которых Митхэ знала прежде.

Слова Штрой только подтверждали это случайное наблюдение.

"--*Возможно, он снова перешёл на сторону Картеля, "--- предположил Атрис.

"--*Исключено, "--- отрезала Штрой.
"--- Это не подлежит сомнению.
Он чересчур насолил Картелю.

"--*По-вашему, это препятствие для того, чтобы принять в свои ряды сильного противника?

"--*Я выражусь иначе: он чересчур насолил некоторым иерархам Картеля.
\mulang{$0$}
{Если в Аду у него почти не было личных врагов, то там их, я полагаю, на пару порядков больше, чем должно быть для относительно спокойного существования.}
{He barely had enemies in the Hell, but there are more, I guess, two orders of magnitude higher than it needs for relatively quiet life.}
Но ладно, ближе к теме.
Нас интересуют некоторые детали.
Когда вы виделись с Аркадиу в последний раз?

"--*Около двадцати дождей назад, здесь, на Тра-Ренкхале, "--- задумалась Митхэ.
"--- Он сказал, что больше не хочет воевать и попросит переквалификацию.
Пообещал иногда навещать нас.

"--*А виделись ли вы с ним на станции Деймос-14?

"--*Нет, "--- ответил Атрис.
"--- На станции нас встретили Грейсвольд Каменный Молот и Анкарьяль Кровавый Шторм, они проинструктировали нас и заверили согласие на сотрудничество с Орденом.
Приказ исходил от Лусафейру Лёгкая Рука.
Правда, мы думали, что согласие должен брать один исполнитель в присутствии двух наблюдателей, но\ldotst

"--*Значит, третьим должен был быть Аркадиу, "--- глухо сказала Митхэ.

"--*Вот именно, "--- улыбнулась Штрой.
"--- Если не ошибаюсь, Грейсвольд и Анкарьяль уже были вызваны на допрос, и им предстояло долгое и весьма неприятное объяснение, почему второй наблюдатель во время процесса болтался неизвестно где.
Хорошая иллюстрация к вопросу об их пригодности.

"--*С ними всё хорошо? "--- Митхэ схватила Штрой за худую руку и тут же отпустила, поймав совсем не дружеский взгляд девушки.

"--*С ними всё в порядке, их лояльность уже долгое время не вызывает сомнений, "--- сухо ответила Штрой.
"--- Теперь о вас.
Во-первых, вы должны в ближайшие пятьдесят три секунды прибыть на станцию 1A, чтобы вас проверили агенты отдела 100.

"--*Хорошо, "--- сказал Атрис.
"--- Но проверить вы сможете только меня, Митхэ "--- оцифрованный\ldotst

"--*У отдела 100 есть средства для проверки оцифрованных сапиентов, "--- прервала его Штрой.

Атрис вздрогнул.

"--*Чистилище?

"--*Это не мне решать.
Данные о структуре вашего демона я передала, в частностях отдел 100 разберётся.
Вам ведь нечего скрывать?

"--*В Чистилище невиновность не является защитой от страданий, Штрой, и ты это прекрасно знаешь, "--- голос Атриса оставался тихим, но глаза сверкнули совершенно человеческой яростью.

\mulang{$0$}
{"--*В том, что вас скомпрометировали, есть доля вашей вины, "--- парировала Штрой.}
{``You're partly responsible for being compromised,'' Stroji retorted.}
\mulang{$0$}
{"--- Аккуратнее выбирайте друзей.}
{``Be careful making friends.}
\mulang{$0$}
{Далее.}
{Next.}
\mulang{$0$}
{Я так понимаю, вы оба испытываете к Шакалу дружеские чувства.}
{As I understand it, both of you feel friendship for Jackal.''}

Митхэ и Атрис переглянулись и кивнули.

"--*Так или иначе, он нарушил присягу, и его судьба отныне находится в ведении отдела 100.
Поэтому хочу вам напомнить, что сотрудничество с предателем или укрывательство такового является нарушением, предусмотренным Оборонительным Кодексом Ордена Преисподней, статья DB, пункт 4.
О санкциях распространяться не буду, читайте сами.

Атрис опустил голову.
Воительница смотрела в оранжевые глаза девушки своим спокойным зелёным взглядом, который так красочно описывали её давно умершие враги.

"--*И кое-что не для протокола.
На случай, если вы всё-таки захотите ему помочь, "--- Штрой поднялась и томно потянулась, выгнув тонкую талию.
"--- Вряд ли то, что носит сейчас личину Аркадиу, на самом деле является им.
Благодарю за отвар.

Не попрощавшись, Штрой Кольцо Дыма вышла из беседки и манерной походкой отправилась на север.

\section{[U] Прошедшая Чистилище}

\epigraph
{Пусть замки из песка дождём бесследно смоет,\\
Упрямый мальчуган их заново построит.\\
Я строчку начертал у моря на песке,\\
И тянется прибой когтями к той строке.\\
Едва мои стихи волна с собой умчала,\\
Я, как дитя, готов игру начать сначала.\\
Но навсегда стихам конец, когда из вод\\
Хотя б одну строку волна назад вернёт.}
{Сигурдур А.\,Магнуссон, <<Начертанное на песке>>.
Эпоха Последней Войны, Древняя Земля}

Митхэ сидела в песке, прижимаясь к Атрису.
По её лицу текли слёзы, перемешиваясь с каплями только прошедшего ливня.

"--*Я надеюсь, что мне больше никогда не придётся столкнуться с контрразведкой, "--- прошептала она.

Атрис выглядел гораздо хуже.
Модуль визуализации ещё работал, голограмма передавалась без сбоев, но сам образ искажался.
Глаза, волосы и кожа Атриса меняли цвет, черты лица расплывались, словно менестрель надел неисправный ноппэрапон\footnote
{На Древней Земле: маска театра Кабуки из магнитного полимера, принимающая любую форму и цвет. \authornote}.
Иногда изображение пропадало совсем, и казалось, что Митхэ обнимает воздух.

"--*Да, я знаю\ldotst

"--*Агенты сказали, что это скоро пройдёт.
Потерпи.

"--*Я с тобой, милый.
Я всегда буду с тобой.

"--*Хочешь песенку?
Я спою, а ты сыграешь.

"--*А почему, ты же любишь музыку?

"--*Ну ладно.

Из кустов показался ребёнок сели.
Он целеустремлённо, с забавно серьёзной мордашкой прошёл до берега моря и сел недалеко от парочки.
Атрис представлял собой ужасающее, сюрреалистичное зрелище, но ребёнок только улыбнулся беззубым ртом.
Вскоре он уже сидел и строил из мокрого песка домик.
Крохотные ручки неумело собрали песчаную гору и начали формировать крышу.
Митхэ с нежностью наблюдала за ним.

"--*Интересно, откуда он?

"--*Понятно.
Можно, я ему помогу?

"--*Я вижу, что он прекрасно справляется сам.
Вопрос был не об этом.

Солнце начало клониться к закату.
Менестрель почти пришёл в себя.
У него всё ещё были видны странные аберрации в волосах, но лицо уже стало прежним.
Он посмотрел на Митхэ и улыбнулся.

"--*Ну вот, Ад нашёл повод и тебе сунуть Чистилище под нос.

"--*Я держалась молодцом? "--- спросила Митхэ.

"--*Да, "--- кивнул менестрель.
"--- Один из агентов сказал мне следующее: <<Личность Митхэ ар’Кахр перешагнула последнюю ступень целостности.
Если Митрис Безымянный будет заподозрен в деятельности, направленной против Ордена, мы будем вынуждены сразу уничтожить его человеческую часть>>.

"--*Ну хоть мучить не будут, "--- ободрительно сказала Митхэ.
"--- Жаль только, что я не смогу облегчить твои страдания, если до подобного дойдёт.

Атрис помолчал.

"--*Забавно.
Чистилище "--- это термин из давно забытой религии.
Это страдания, которые дарил древний бог, приговаривая: <<Ты мой верный раб, но за грехи ты ответить обязан>>.

"--*Грехи "--- это преступления перед богом?

"--*Да, это нарушения его законов.

"--*А что давало этому богу право требовать их исполнения?

"--*Он оправдывал это тем, что он создал сапиентов.

Митхэ поморщилась.

"--*Не горшечник решает, когда время горшку разбиться.
Это решает тот, кто способен нанести горшку смертельный удар.

"--*Верно, "--- признал Атрис.
"--- Если же кто-то бьёт такие прекрасные горшки, стоит задуматься "--- а он ли их слепил.
У меня бы не поднялась рука.

С востока потянуло солёной прохладой, и волны стали забегать чуть дальше на берег.
Вот одна, затем другая достигла одинокого, похожего на торт домика.
Волны слизывали стены, словно крем.
Атрис рассеянно смотрел на останки сооружения.

"--*Знаешь, чем мне нравятся дети?

"--*Чем, Атрис?

"--*Они замечательно умеют повторять, не понимая смысла.

"--*Незачем понимать назначение и механизм дыхания, если это нужно для жизни, "--- заметила Митхэ.

"--*Но ребёнка необходимо научить правильно дышать.
Он не всегда будет в стандартных условиях существования.

"--*Да, "--- лицо Митхэ просветлело.
"--- В Храме меня учили правильно дышать.

"--*А если научить некому, то большинство сапиентов так и остаются детьми.

Митхэ кивнула и крепче прижалась к любимому.

"--*Что было в сообщении? "--- тихо спросил Атрис.

"--*Он уже здесь.
С завтрашнего дня можем приступать, "--- ответила Митхэ.

Атрис улыбнулся и погладил подругу по голове.

"--*А вот теперь я бы сыграл.

\chapter{[U] Равновесие}

\section{[U] Зачисление в контрразведку}

\epigraph
{Тси "--- это сапиенты, которые используют машины для обмена данными с внешней средой.
Демоны "--- это машины, использующие для той же цели сапиентов.
Может быть, именно поэтому у нас с ними возникает недопонимание.}
{Длинный-Мокрый-Хвост}

\spacing

"--*Компромата на вас собрано достаточно.

"--*Компромата? "--- выдохнула Анкарьяль.

"--*Не обращай внимания, Нар.
У Самаолу своеобразное чувство юмора, "--- буркнул Грейсвольд.
"--- И он настолько любит шутить, что порой забывает поздравить демонов с самой бескровной успешной операцией со времён Тахиро.

"--*Я не давал тебе слова, Грейсвольд Каменный Молот, "--- ледяным тоном сказал Самаолу.

"--*Если хочешь командовать чужими ртами, научись следить за своим.
Что ещё за компромат?

Самаолу демонстративно достал компьютер и начал листать слайды.
Игра была очевидна для всех "--- демон и так был в курсе материала.

"--*Первое.
Вы были друзьями Аркадиу Шакала Чрева.

"--*И нас по этому поводу уже проверили, "--- парировал Грейсвольд.

"--*Разумеется.
И из проверки вытекает второе "--- в вас замечена ощутимая симпатия к тси.

Грейсвольд и Анкарьяль промолчали.

"--*То есть вы даже не будете этого отрицать?

"--*В отрицании есть смысл? "--- буркнул Грейсвольд.
"--- Отдел 100 отвечает за верификацию полученных данных.
Кроме того, в своде законов Ада есть прямой запрет санкций за мыслепреступления.

"--*Впрочем, этот запрет "--- такая же формальность, как и текущее заседание, "--- усмехнулась Анкарьяль.

"--*Хорошо, "--- кивнул Самаолу, пропустив слова Анкарьяль мимо ушей.
"--- В таком случае не могли бы вы осветить подробнее своё отношение к тси?

"--*Вообще-то я не обязан этого делать, но так и быть.
На планете Тра-Ренкхаль я впервые узнал, что искренность может быть культурной нормой.
Я первый и последний раз оказался в обществе, которое приняло меня тем, кто я есть.
Тси относились ко мне и к другим демонам так же, как к соплеменникам.

Пара советников испустили презрительный смешок.
Самаолу нахмурился.

"--*А что скажете вы, Анкарьяль Кровавый Шторм?
Если я не ошибаюсь, сели вам поклоняются.

"--*Не мне, Самаолу, а обезличенному образу, отпечатку личности давно умершего тела.
Вас ввели в заблуждение.

"--*Не прибедняйтесь.
В образе хватает черт и вашей нематериальной части.
Очень любопытно включение в лик Самоотверженного Хата двух звёзд, двух лун и двух гор, которые некогда красовались на знамени баронессы Красношторм, планета Тысяча Башен.

На лице Анкарьяль дёрнулся мускул.
Она была уверена, что компрометирующий <<подарок>> сделал ей Аркадиу, напоследок отведя от себя все подозрения.

"--*Впрочем, неважно.
В вашем отчёте прямо указано, что это совпадение, а в вас достаточно сложно найти замашки демиурга-индивидуалиста.
Все эти годы вы были лезвием меча, частью команды, и не претендовали на иную роль.
Так что вы думаете насчёт тси?

Как ни лжива была ситуация, Грейсвольд заметил, что последние слова ей польстили.
Да, так и есть.
Лезвие меча, часть команды.

"--*Скажу, что Катаклизм Тси-Ди был величайшей трагедией для нас, "--- спокойно ответила Анкарьяль.

В зале наступила оглушительная тишина.
Грейсвольд вдруг тоже нахохлился.

"--*Объясните, "--- сказал Самаолу после продолжительного молчания.

"--*Как вы знаете, мы находимся в мёртвой зоне Вселенной, "--- заговорила Анкарьяль, и её слова эхом отдались в зале.
"--- Наша энергия ограничена константой Ка'нета, и, несмотря на все уверения технологов, с момента открытия её не удалось отодвинуть даже на одну триллионную яо\footnote
{Яо (Y) "--- единица измерения масс-энергии ПКВ. \authornote}.
Предпринималось множество попыток найти разумную жизнь в пределах ка'нетовского радиуса, но ни одна не увенчалась успехом.
Огромный шар диаметром в тридцать шесть тысяч парсак "--- и всего двадцать девять планет с самозарождённой стабильной жизнью, из них с разумной "--- две!
Вдумайтесь в это.
Наши проекты по терраформированию терпят крах один за другим.
То есть, говоря без обиняков, мы, высокоразвитая цивилизация хоргетов, не можем сделать даже того, что сделали древние земляне.
Вы сидите здесь, гордитесь своими технологиями, многие из вас презрительно называют первых людей <<обезьянами>>, но по сути вы "--- посмешище для собственных создателей.

"--*Терраформирование?
Вы серьёзно? "--- заливисто рассмеялась советник Кагуя Хрустальный Голос.
"--- И это тогда, когда Картель активно возвращает себе позиции на Плеядах и Развязке Десяти Звёзд?

"--*Война для вас важнее, Кагуя? "--- невинно поинтересовался Грейсвольд.
"--- Ведь благодаря ей вы сидите в этом мягком кресле?

Анкарьяль подняла руку, показывая, что ещё не закончила говорить.
Самаолу жестом остановил попытавшуюся ответить Кагуя и кивнул интерфектору.

"--*Ещё касательно эффективности Ордена.
Возьмём Тра-Ренкхаль.
Аркадиу Люпино сразу после битвы на Могильном берегу вызвал геологов, чтобы они остановили землетрясения.
Да, "--- повысила голос Анкарьяль, перекрывая смешки, "--- да, \emph{предатель} озаботился обустройством владений Ордена.
С тех пор прошло сто пятьдесят четыре дождя.
Где геологи?
Из-за предательства Аркадиу его запрос потерял актуальность?
Безымянный до сих пор обеспечивает безопасность сапиентов за счёт собственных ресурсов.

"--*Мы не обсуждаем на этом заседании политику Ада по отношению к Митрису Безымянному, Анкарьяль, "--- холодно бросил один из советников.

"--*Ах, это аспект отношений с Безымянным, а не простая халатность? "--- проворчала Анкарьяль.
"--- Благодарю за разъяснение, советник, вы спасли меня от крамольных мыслей.

"--*Тишина, "--- призвал к порядку Самаолу.
"--- Вернёмся к обсуждению тси.
Грейсвольд, вам есть что добавить?

"--*Я не вполне согласен с формулировками Анкарьяль, но по сути она права.
Мы столкнулись с теми же проблемами, которые когда-то потрясли цивилизацию Древней Земли.
Однако мы гораздо могущественнее первых людей, наши ошибки гораздо более серьёзны, и потому масштаб катастрофы нам оценить едва ли не сложнее, чем им.
Даже если предположить, что исход войны близок и Картель падёт, куда, скажите на милость, направится взор нашей военной машины?
Хотите сказать, что зверь, которого кормит весь Ад без исключения, покорно сложит лапки и заснёт вечным сном?

"--*Враги есть всегда, Грейсвольд, "--- заявила Кагуя.

"--*И вы отлично умеете их находить, "--- поклонился Грейсвольд, "--- даже там, где их отродясь не было.
Что вам сделал Безымянный?
Такого покладистого демиурга ещё поискать.
С демиургами всегда были проблемы "--- взять того же Эйраки, или Кох Свободолюбивую, или\ldotst

"--*Или вас, "--- присовокупила Кагуя.

"--*Безымянный "--- лучшее, что есть на Тра-Ренкхале, "--- громко и членораздельно сказал Грейсвольд.
"--- Он подобен тем наивным древним учёным "--- философам, астрологам, алхимикам, колдунам, жрецам, "--- которые выстраивали сложные мистические теории мироздания, которые безо всякой научной и идеологической базы искали, находили и делали выводы.
Он художник, и его планета "--- один большой холст.
У меня как демиурга никогда не хватило бы фантазии на то, что сделал Безымянный.
Однако Ад выстроил против этого миролюбивого создания такую защиту, словно на Тра-Ренкхале хранится эссенция Красного Картеля в хрупком стеклянном флаконе.
Мне даже довелось слышать, как Безымянного сравнили с Уэсиба Серозмеем.

Кагуя вдруг густо покраснела, но заметила это только Анкарьяль.
Всё внимание советников было приковано к Грейсвольду.

"--*Если Безымянный так опасен, почему он ещё жив?

Совет замер.
В воздухе повисла неловкая тишина.

"--*Я серьёзно, "--- как ни в чём не бывало продолжил технолог.
"--- Мы прекрасно знаем, что любые договоры "--- лишь красивые слова.
Тем более этот, с громким названием <<Демиург "--- Метрополия>>, клише которого было создано, между прочим, по моей инициативе, на всякий случай.
Я надеялся, что это будет подспорьем для социализации демиургов, позволит тем, кто не готов перейти на демонический образ жизни, быть равными нам, не быть хоргетами второго сорта.
Однако субъектов договора в истории Ордена всего шесть, в настоящее время "--- один-единственный, если не считать <<без вести пропавшую>> Кох.
Счёт же уничтоженных богов "--- тех, чьё уничтожение было задокументировано "--- идёт на десятки тысяч.
Так что же мешает Аду \emph{ещё раз} устроить тотальную мелиорацию и окончательно решить вопрос демиурга и девиантной фауны?

"--*Вы подозреваете Орден в уничтожении Кох, Грейсвольд? "--- вмешался один из советников.

"--*Ни в коем случае, советник.
Я не подозреваю, я в этом уверен.
Разыгравшийся вокруг Безымянного спектакль мне до боли напоминает уже отгремевшие и давно забытые события.
И знаете, есть некая ирония в том, что Аркадиу Люпино, чьё первое тело выросло под крылом Богини-Матери Пустоши Драконов, предал Орден в тысячную годовщину того, как Богине-Матери помогли найти дорогу в небытие.

"--*Может, Грейсвольд, вам снова выдать планету? "--- усмехнулся кто-то в зале.
"--- Вы с таким упоением расписываете прелести жизни демиурга\ldotst

Зал захохотал.

"--*А вы знаете, я только за, "--- подбоченился технолог.
"--- И моя мечта исполнилась бы, если бы не всеобъемлющая зависть.

Зал снова огласил хохот.

"--*А разве я не прав? "--- продолжал Грейсвольд.
"--- У каждого из хоргетов достаточно способностей, чтобы инкарнироваться в планету.
Извините, что напоминаю об этом, но нас для этого и создавали, не так ли?

"--*Не имеет значения, для чего нас создавали, Грейсвольд, "--- заявил Самаолу.

"--*Отнюдь!
Едва ли такая проблема стояла перед первыми людьми "--- они были самообразовавшимися живыми системами и вольны были сами выбирать своё предназначение.
Но мы "--- продукт инженерии, разумный замысел, а не игра слепой эволюции.
Да, мы все "--- прирождённые демиурги!
Но вы сбились в крохотном зале, упаковали себя в микроскопические сапиентные тела, вы красите волосы и глаза в немыслимые цвета, выращиваете себе когти и щупальца, мечтая о движении континентов, изменениях климата, лесах из светящихся лиловых грибов и цивилизациях, взлетающих под вашей опекой!
Вы боитесь демиургов, потому что вы боитесь своей собственной природы, лежащей под спудом, загнанной в угол вашими собственными страхами!

"--*Если ваша соратница имеет привычку сквернословить, Грейсвольд, то вы явно любите драматизировать, "--- буркнул Самаолу.
По галерее пронёсся смешок.
"--- Я предпочитаю первое.
Давайте всё-таки вернёмся к обсуждению тси.
Итак, что бы вы предложили, Анкарьяль?
Какую политику следовало бы проводить Ордену в отношении тси?

Несколько советников, включая Кагуя, снисходительно улыбнулись.

"--*Политику мира.

"--*Сильно, "--- признал один из советников, Кес Бледный Глаз.
"--- Я думаю, нет смысла спрашивать, как этот мир достичь.
Но ради справедливости, Анкарьяль, я повторю вопрос вашего товарища Грейсвольда "--- где были бы \emph{вы}, не будь войны?

Треть совета смущённо кашлянула.
Прочие взглянули на Кеса, словно он справил нужду прямо посреди зала.

"--*Без войны я была бы в опасной экспедиции за пределами ка'нетовского радиуса, "--- сказала Анкарьяль.
"--- Я бы пронзала пространство и время, искала новые земли, улыбалась новому свету, а в минуты отдыха размышляла бы, как послать весточку вам, старому миру, как сообщить затерянному среди галактик дому, что мы тоже живы и процветаем.
Мой друг Грейсвольд был бы демиургом, и едва ли наши пути пересеклись бы.
Но, как видите, мы стоим сейчас плечом к плечу, отстаивая право быть собой.

"--*Вам никто не мешает быть собой, Анкарьяль "--- в пределах закона, разумеется.
И я так и не понял, при чём тут цивилизация Тси-Ди.

"--*Стабильная технократическая цивилизация сапиентов, которой являлись тси, могла колонизировать дальний космос, дать вам плацдармы для дальнейших исследований и продвинуть границу Ка'нета на тысячи парсак вглубь галактики, к её ядру.
Наши исследования дают однозначные результаты "--- в ядре мы сможем найти достаточно подходящих для жизни планет.
Некоторые исследователи даже говорят, что имеющихся там ресурсов достаточно, чтобы создать и запустить звёздный корабль на разумных антаридах, преодолеть\ldotst

"--*Оставьте эту чушь, Анкарьяль, "--- сказал Самаолу.
"--- Звёздный корабль "--- очередная сказка для романтиков вроде вас.
Разумные антариды "--- из немного другой сказки, ещё менее правдоподобной.
Жизненный цикл антарид настолько короток, что исключает наличие сапиентной архитектуры любого уровня.
Хоргеты не способны выйти за пределы Млечного Пути.
Это возможно лишь для микоргета, но последние исследования ставят под вопрос саму возможность его существования.

"--*Антариды\ldotst

"--*Самаолу прав, Анкарьяль, "--- вдруг подала голос Корхес Соловьиный Язык.
"--- Даже если создать разумных антарид, существование антаридной цивилизации не будет долгим.
Антариды "--- это вам не тси.
Мы получим либо безжизненную звезду, либо взрыв сверхновой в самые короткие сроки, звёздный корабль даже не успеет выйти за пределы Млечного Пути.
И я бы попросила закрыть эту тему, потому что ваша квалификация как биолога "--- и тем паче плазмобиолога "--- оставляет желать лучшего.
А я люблю науку и не люблю, когда военные начинают лапать её своими грязными ручищами.

"--*Даже если и так, "--- Анкарьяль с трудом сдерживала раздражение, "--- почему хотя бы не освоить до конца Млечный Путь, почему не заручиться союзом с тси?
Да, скорость их кораблей ограничена светом, но они достигли бы большего успеха за единицу времени, чем вы.
И я уверена, что даже если бы Ад и Картель обрушились всей своей мощью на колонистов, тси всё равно нашли бы время на обустройство новых планет.
Они не считали нужным постоянно просиживать у орудий.

"--*Вы думаете, что мы не рассматривали такие варианты? "--- высокомерно улыбнулся Самаолу.

Анкарьяль явно хотела сказать, что она думает о Самаолу, но сдержалась.

"--*В таком случае объясните нам с Анкарьяль, какие именно нюансы она не учла, "--- проворковал Грейсвольд.

"--*Тси знали о существовании демонов, "--- медленно, словно для умственно отсталых, начала объяснять Кагуя.
"--- Более того, им удалось достигнуть с нами стратегический паритет\ldotst

"--*Опять военный термин, "--- поморщился Грейсвольд.

"--*А какие термины вы предпочитаете применять к враждебной цивилизации?
Это война, Грейсвольд.
Мы пытаемся выстоять против чужой агрессии.

"--*То же самое говорили миллионы агрессоров до вас, "--- сказал технолог.
"--- Слепцы по-прежнему верят, что есть праведные войны.
Романтики считают, что время праведных войн прошло.
А я, видевший историю Земли почти с самого её рассвета, скажу "--- праведных войн нет и не было.
В любой войне число агрессоров равно числу участников!

"--*Вы предлагаете сложить оружие и отдаться на милость Картеля? "--- ласково осведомилась Кагуя.
"--- Или вы, Анкарьяль.
Вы позволили бы тси колонизировать наши планеты?
Капитул?

"--*Вы никак не можете понять, что тси "--- это не обычные сапиенты, которых мы используем как дойных животных! "--- повысила голос Анкарьяль.
"--- Они равные нам, а кое в чём и превосходящие нас существа!

"--*Тси "--- это хамелеоны.
Они легко интегрируются в любое общество и бессознательно внедряют в это общество свои ценности, распространяют свою заразу\ldotst

"--*Как и представители любой древней, выкристаллизовавшейся культуры, "--- заметил Грейсвольд.
"--- Мало какая культура может состязаться с тси в возрасте.
За двести тысяч лет культура тси въелась в их генофонд, приобрела устойчивость к множеству различных воздействий, и даже жестокий отрицательный отбор не даст быстрых результатов.

"--*Особенно со стороны менее древней культуры, "--- ядовито добавила Анкарьяль.
По галерее пробежал возмущённый шёпот.
"--- <<Зараза>>!
Вы говорите об информации, словно о смертельном вирусе, от которого у вас нет защиты.
По-вашему, адепты Ада не способны фильтровать информацию?
Мне доводилось слышать мнения, что народ Тси-Ди "--- тепличные растения и пример культурного дрейфа\footnote
{Культурный дрейф "--- ненаправленные изменения культуры, обусловленные случайными статистическими причинами. \authornote}.
И вдруг выясняется, что их культура "--- <<зараза>>, от которой нужно защищаться!
Что это значит?
Выходит, перед нами не дрейф, а стадия эволюции, до которой нам с вами ещё далеко?

"--*Так или иначе, наиболее яркие представители погибли на Могильном берегу, "--- с гадливой жалостью сказала Кагуя.
"--- Остаткам культуры был нанесён непоправимый удар.

"--*Я бы не делала столь поспешных выводов, "--- улыбнулась Анкарьяль.
"--- Вы недооцениваете тси, и это вам ещё аукнется в будущем.

"--*Да, действительно, есть исследователи, считающие, что сам контакт Ордена Преисподней с культурой тси опасен, "--- хмуро сказал Самаолу.
"--- Вы поддерживаете это мнение, Анкарьяль?

"--*Опасность контакта оценивают контактирующие согласно собственному опыту, "--- туманно выразилась Анкарьяль.
Кто-то из советников нахмурился, кто-то улыбнулся, Кагуя же фраза привела в бешенство.
"--- Но изменения определённо произойдут.

"--*То есть вы хотите сказать, что опасность мы придумали сами? "--- спросила Кагуя, едва сдерживаясь.

"--*Тси угрожали убить вас лично, Кагуя?

"--*Что за профанация, Анкарьяль? "--- опешила советник.
"--- Что вы имеете в виду?
Я член Ордена Преисподней, а это значит\ldotst

"--*Это значит, что ответственность за статус-кво лежит в том числе и на вас! "--- перебила Анкарьяль.
"--- Почему-то как речь об опасности "--- то она общая, а ответственность за политику чья угодно, но только не ваша!
В отличие от Картеля, у нас была возможность установить с тси договорные взаимоуважительные отношения, но мы использовали их как разменную монету в войне с тем же самым Картелем.
Как и в случае с Безымянным, Орден выбрал войну вместо мира.
Я боюсь, что если ретроспективно подвергнуть такому же анализу действия Ордена за последний телльн, то окажется, что большинство реше\ldotst

"--*Довольно, "--- рявкнул Самаолу.
"--- Мы поняли вашу точку зрения.
Напоминаю, что Орден Преисподней предлагал тси протекторат, тси это предложение отвергли.
Это было их и только их решением!

"--*Протекторат? "--- задохнулась Анкарьяль.
"--- Вы это называете <<взаимоуважительными отношениями>>?
Я бы сделала то же самое, да ещё и плюнула вам в лицо!
Тси сдерживали Картель куда эффективнее, чем вы!
Тси-Ди была самым спокойным местом во Вселенной!

"--*Вы называете Тси-Ди спокойным местом? "--- спросил советник.
"--- Вот вам немного статистики. Основная причина смерти тси "--- несчастные случаи на производстве.
Девяносто четыре процента всех смертей.
Тси жили на своих великих стройках и умирали на них же, как рабы Эйгипта, которые строили для своих фарай величественные гробницы.
Это цивилизация рабов, ничем не отличающаяся от прочих сапиентных цивилизаций.

"--*И даже их конец был одним большим несчастным случаем на производстве, "--- ядовито добавила Кагуя.
"--- Вы технолог, Грейсвольд?
Вы понимаете, о чём я говорю.

"--*Вы искажаете суть статистических данных, "--- возразил Грейсвольд, проигнорировав слова Кагуя.
"--- Тси умирали при несчастных случаях не из-за обилия таковых, а из-за того, что до смерти от естественных причин они просто не доживали.
Смертность по медицинским причинам контролировать гораздо легче, чем смертность от несчастных случаев, и тси это показали очень хорошо.
Собственно говоря, я не слышал ни одной истории о демоне, который мирно прекратил существование у себя дома.
Основные причины смертности у нас "--- убит, казнён, пропал без вести.
Это моя собственная статистика, я её веду по друзьям и знакомым\ldotst

"--*Довольно, "--- поднял руку Самаолу.
"--- Ваши статистические выкладки никем не проверены и потому не представляют для нас интереса.

"--*И я совершенно не понимаю аналогии с гробницей, "--- заметила Анкарьяль.
"--- Если уж вы хотите заручиться союзом с историей Древней Земли, то я сделаю то же самое.
Вы были для них вандейлами, которые молотили каменными топорами в прекрасные стабитаниумовые ворота Ромай!
Тси умирали, пытаясь починить дом, который\ldotst

"--*Довольно! "--- рявкнул Самаолу.

Анкарьяль хмуро скрестила руки на груди.

"--*Впрочем, даже жаль, что тси отгородились от нас, "--- отчетливо сказала она в тишине, пренебрегая запретом.
"--- Если бы демоны жили на Тси-Ди на равных условиях с аборигенами, если бы они видели, как эти существа из плоти и крови относятся друг к другу, Ад и Картель, эти колоссы на глиняных ногах, развалились бы в первые же двадцать лет.

Грейсвольд запоздало схватил подругу за плечо, пытаясь её остановить.
Слово прозвучало.
Советники потрясённо молчали.
Казалось, в этой оглушительной тишине слышно, как бьются сердца и гудят стабилизирующие модули омега-сингулярностей "--- если бы их колебания были доступны для слуха сапиента.

Наконец Самаолу вздохнул.
Это был вздох человека, который утвердился в худших своих опасениях.

"--*Мы, "--- он обвёл рукой зал, "--- все мы могли бы много чего сказать на это скоропалительное, я бы даже сказал "--- безответственное заявление.
Но мы здесь не для того, Анкарьяль Кровавый Шторм, чтобы разъяснять вам очевидное, тем более что имеет место проявление некой идеологии, а не простое заблуждение.
Идеология, в отличие от заблуждения, прекрасно умеет защищаться от фактов и теорий.

"--*К сожалению, Самаолу, это так, "--- сухо сказала Анкарьяль.

"--*Я рад, что хоть в чём-то мы сошлись, потому что у нас с вами нет другого выбора, кроме как найти общий язык.
Как я уже говорил, мы набрали массу интересного материала касательно вас\ldotst "--- Самаолу позволил себе загадочную паузу, "--- и тем не менее к вам есть предложение поступить в резервный состав отдела 100.

Грейсвольд и Анкарьяль удивлённо промолчали.

"--*Обретает форму новая сила, "--- сказал Самаолу.
"--- У нас недостаточно данных о её природе, но достаточно оснований предполагать её враждебность.

"--*Говори проще, Самаолу, я тебе пелёнки когда-то менял, "--- поморщился технолог.
Он взглянул на Анкарьяль "--- демоница вдруг побелела как полотно.

"--*Контрразведке нужны демоны, которые способны думать, как враг.

"--*То есть вы даже не скрываете, что считаете нас врагами! "--- вспылила Анкарьяль.
"--- <<Держи друзей близко, а врагов ещё ближе>>, да?
Я не буду переводиться на таких унизительных условиях!

"--*Что именно вы считаете унизительным? "--- осведомился один из советников.

"--*Я вернулась с Тра-Ренкхаля, где ради интересов Ордена чуть не отправилась \emph{жилами джунглей}! "--- выкрикнула Анкарьяль.
Самаолу прищурился.
\mulang{$0$}
{"--- Какие бы ошибки я не совершила в молодости, сейчас я легат терция и заслужила карьерный рост, а не участь подопытной мыши или надувного монстра, которого вы будете трахать в надежде избавиться от собственного страха!}
{``Whatever mistakes I made in my youth, now I am legate tercia and I deserved career progression, not a fate of guinea pig or inflatable monster which you will fuck in the hope of conquering your own fear!''}

"--*А само зачисление в ряды контрразведки вы не считаете повышением?

Анкарьяль в десяти словах на языке фоуф-у\footnote
{Пиджин наёмников Тысячи Башен, вобравший в себя рекордное количество обсценных слов из других языков "--- 6\% корней.
В основном эти слова касаются сексуальных отношений, физических и психических недостатков, отправления естественных нужд, состояний сознания при приёме различных наркотиков, а также изощрённых способов пытки и убийства. \authornote}
объяснила Совету, что она думает о таком повышении.
Советники вытаращили глаза, когда семантический переводчик выдал шестнадцать слайдов мелкого текста.
Немного знакомый с этим языком Грейсвольд поперхнулся ещё на третьем слове.
Входящий в Совет диктиолог-лингвист Кельса Пушистая украдкой включила записную книжку.

"--*Кельса, это уже занесли в протокол, "--- ледяным тоном сказал Самаолу, и демоница тут же убрала терминал.
"--- Если сквернословие представляет для тебя научный интерес, я отправлю официальную выписку в твой отдел.

<<Нар, я тебя люблю, "--- с нежностью подумал Грейсвольд.
"--- Так блестяще сыграть прожжённую карьеристку и высказать правду в лицо не смог бы даже я.
Теперь моя очередь купаться в этом вонючем болоте, доказывая свою профпригодность.
Чтоб этого Аркадиу поглотила бездна "--- сам пропал, а нам оставил тележку с навозом\ldotse>>

"--*И всё-таки, к чему устраивать этот спектакль с отделом 100, если нас просто хотят проверить на верность? "--- осведомился технолог.

"--*Совет похож на театр, Грейсвольд Каменный Молот? "--- один из советников позволил себе улыбку.

"--*И весьма, Эйяфар Цеппелин.
Вы с Анкарьяль достаточно молоды, чтобы не знать, кто разрабатывал правила функционирования отдела 100.
А есть в их своде следующее "--- за пределами, простите за фигуральность, можно играть камнями всех четырёх цветов\footnote
{Намёк на Метритхис, настольную игру сели. \authornote},
но в самом отделе должен царить абсолютный порядок.
И резервный состав "--- не исключение.
Возможно, я ошибаюсь и правила поменялись?

Пара сидящих в зале зашевелились.

"--*Вы хотите проверить меня на верность?
Прошу вас.
Вы видите во мне лишь впадающего в деменцию старика.
А я вижу шестнадцать горшков, которые звенят на слепившего их горшечника.

"--*Оставьте литературные обороты при себе, "--- сказал Самаолу.
"--- Отдел 100 только что сообщил, что вы зачислены в резервный состав.
Официальная формулировка "--- <<за многократно доказанную лояльность и высочайший профессионализм>>.
Это должно вас устроить.
Инструктаж будет проведён через двадцать секунд на станции 80, о ваших функциональных телах позаботятся.
Заседание окончено.

Грейсвольд застыл столбом и взглянул на подругу.
Анкарьяль положила руку ему на плечо, и оба упали замертво посреди зала.

\section{[U] Большая игра}

\spacing

"--*Итак, Штрой, вы всё-таки придерживаетесь мнения, что это ковен Картеля, "--- произнес голос.

"--*Именно, владыка, "--- Штрой говорила чётко и размеренно, былая жеманность улетучилась без следа.
"--- Почерк Картеля я узнаю всегда и везде, потому что сама приложила к нему руку.

"--*И поэтому вы ввели их в заблуждение.
Что ж, неплохо.
А вы, Самаолу?

"--*Несмотря на сравнительный анализ Штрой, связи с Картелем я не нашёл, "--- угрюмо сказал Самаолу.
"--- Гораздо больше фактов говорит о том, что в деле замешана третья сторона.
А как считаете вы?
Вы проводили какие-то исследования\ldotst

"--*Их результаты вас не касаются, "--- спокойно отрезал голос.
"--- Продолжайте разрабатывать собственные версии.
Кстати, Самаолу, ваш трюк с Анкарьяль и Грейсвольдом был просто великолепен.
Так блестяще сыграть тупицу я бы не смог, у вас большие задатки\ldotst кхм\ldotst театрального актёра.
Поблагодарите за меня Кагуя "--- она тоже прекрасно справилась.

"--*Обязательно, "--- слегка улыбнулся Самаолу.
"--- Приятно иногда вносить изменения в правила, особенно в правила отдела 100.
Относительно этих демонов план остаётся прежним?

"--*Разумеется, "--- подтвердил голос.

"--*Правда, я не совсем понимаю, в чём их важность, "--- полувопросительно заметил Самаолу.

"--*Её действительно сложно понять, "--- согласился голос.
"--- Анкарьяль и Грейсвольд заняли весьма неудобное для нас положение "--- <<красные плащи в узком ущелье>>\footnote
{Тактический ход, возможно, восходящий к легендарной битве у Фермопил, Древняя Земля. \authornote}.
Они имеют богатый опыт совместной работы и безоговорочно друг другу доверяют.
Собственно, нам нужно изучить этот тандем в действии.

"--*Почему мы не можем их просто устранить физически?

"--*Во-первых, Самаолу, оставьте привычку решать проблемы силой.
Да, у нас достаточно ресурсов, чтобы от всех наших соперников "--- по крайней мере, внутренних "--- не осталось даже воспоминаний.
Но в таком случае можно сразу расписаться в неумении управлять.
Решая проблемы силой внутри Ордена, вы лишите себя ценнейших ресурсов в борьбе с внешним противником.
Да, я считаю всех Скорбящих ценным ресурсом, Штрой, "--- как ни крути, они смогли сделать очень многое.
Поэтому мы будем держать волка в клетке, пока у нас есть хотя бы призрачная возможность договориться.
Тем более, как Самаолу уже сказал, когда-то Анкарьяль и Грейсвольд были вернейшими адептами Ордена, а это значит, что они могут снова стать такими.

"--*Никогда не доверял предателям, "--- буркнул Самаолу.
Штрой ощерилась.

"--*Тихо, "--- голос поднял невидимую ладонь, и демоны разом вытянулись в струнку.
"--- Предатель, Самаолу "--- это прежде всего тот, кто умеет сравнивать и не боится делать выбор.
То, что Анкарьяль и Грейсвольд в итоге оказались в агентурной сети Скорбящих, означает три вещи.
Первое "--- с Орденом что-то не так, и это очевидно для всех, кто способен думать.
Второе "--- они не считают Орден безнадёжно загнившим, так как формально остались на службе, а не сбежали, как крысы с тонущего корабля.
Да-да, Самаолу, после зачисления в отдел 100 они могли сбежать, а не подвергать себя риску "--- но выбрали риск.
И третье "--- они растут как личности и как профессионалы, и это очень важный момент.
Впрочем, действовал этот тандем очень осторожно, но наживку заглотил с половиной удочки.
Они не настолько умны, как загадочные стратеги 03 и 05.

"--*Возможно, владыка, нам следует расширить выборку демонов, подпадающих под\ldotst

"--*А какой смысл, Штрой? "--- возразил голос.
"--- Даже если бы сейчас стратеги 03 и 05 явились ко мне лично и принесли исчерпывающий компромат на самих себя, это мало что изменило бы.
Вы думаете, что это зелёная молодёжь, которая ищет своё место в жизни?
Основную проблему я вам очертил сразу "--- мятежников поддерживают группировки, с которыми уже давно достигнуто равновесие Нэша.
Говоря другими словами, фигуры, которые давно и прочно стоят в своих клетках, неожиданно поменяли цвет, сделав ставку на молодого и неопытного игрока.
Когда я впервые осознал этот факт, то был озадачен.

"--*Владыка, я не совсем понимаю\ldotst

"--*А, Штрой, прошу прощения, я забыл про особенности обучения стратегов в Картеле.
Вы читали доклад Самаолу?

"--*Разумеется!

"--*И какие ваши впечатления?

"--*Анкарьяль и Грейсвольд пожертвовали своим по\ldotst

"--*Именно, "--- откликнулся голос.
"--- Штрой, вы потрясающий стратег.
Даже Самаолу посчитал произошедшее гамбитом со стороны Ада.

"--*А теперь, похоже, не понимаю я, "--- протянул Самаолу.

"--*Именно поэтому вы со Штрой работаете вместе, "--- объяснил голос.
"--- Вы замечательно друг друга дополняете.

Штрой и Самаолу бросили друг на друга взгляды, полные отвращения.

"--*У вас, как я понимаю, нет статистики по отделу 100?
Так слушайте: за последние четырнадцать лет контрразведка потеряла в три раза больше демонов, чем мы на полях сражений.
Туда берут лучших "--- и эти лучшие мрут, как тараканы.

Самаолу ахнул.

"--*Они побледнели\ldotst

"--*\ldotst от страха, а не от злости, "--- закончила Штрой.

"--*Браво.
Вы поняли.
У них есть источник информации, который держит их в курсе дел.
Перевод из легатов запаса в центурионы отдела 100 "--- это даже не смешно.
Анкарьяль и Грейсвольду придётся пройти через кошмар, прежде чем они закрепятся на новом месте.
Они знали о предстоящем возмездии за изменение собственной стратегии, но не подозревали о его силе.
Истинное равновесие Нэша "--- жестокая вещь.

"--*И такие потери в позициях понесли все, кто поддержал эту группировку? "--- спросила Штрой.

"--*Не такие, Штрой.
Нашим друзьям просто сильно не повезло.
Подозреваю, что Грейсвольд принял на себя часть удара, предназначенного не ему "--- он едва избежал смерти уже как минимум два раза, если считать последний инцидент на Тра-Ренкхале.
Возможно, он защитил этого неуловимого стратега 03.

"--*Вы считаете события на Тра-Ренкхале платой за изменение стратегии? "--- поднял бровь Самаолу.
"--- У меня создалось впечатление, что\ldotst

"--*Друг мой, стратег 03 действует не первое столетие.
И тихоня Грейсвольд "--- его главный помощник.
Вероятность того, что инцидент на Тра-Ренкхале бы нэшевским рикошетом, стремится к единице.

"--*Кто-то из стратегов крупно просчитался, "--- улыбнулся Самаолу.

"--*Или сделал чересчур широкий шаг вперёд, "--- подхватил голос.
"--- Никогда не мешает предположить худшее, особенно если учесть, что за прошедшие пять лет <<красные плащи>> неплохо отыгрались.
Награды сыпались на Грейсвольда с Анкарьяль словно из мифического рога изобилия.
Честно говоря, я думал, что это заставит их отказаться от полевой работы.
Но я ошибался.
Что ж, хотят опасностей "--- у центурионов отдела 100 их хватает.

"--*Владыка, не может ли стратег 03 быть Падальщиком? "--- спросила Штрой.
"--- Несмотря на явную разницу в уровне, в почерке есть общие черты\ldotst

"--*Исключено, "--- сухо ответил голос.
"--- Сходство можете списать на что угодно "--- на подражание, случайность или наставничество.
Падальщик "--- пешка.
Оставьте его в покое, он предсказуем и может быть уничтожен в любой момент.
На звание одного из стратегов у нас есть претенденты куда старше и опаснее.

"--*Стигма, "--- проворчал Самаолу.
"--- Я употребил всё своё влияние, чтобы отослать её на Тра-Ренкхаль, подальше от основного театра действий.
Любое обвинение стекает со Стигмы, как ртуть со стекла.

"--*Владыка, раз уж Стигма оказалась на Тра-Ренкхале, я могу устранить её физически, "--- предложила Штрой.

"--*Даже не пытайся, "--- резко сказал Самаолу.
"--- Она съест тебя на завтрак и не посмотрит на твои лычки.
Лимес, Бит и Уитлик умерли не без её помощи\ldotst

"--*И всё же я прошу у владыки разрешения, "--- поклонилась Штрой.

"--*Штрой Кольцо Дыма, "--- голос позволил себе оттенить слова великолепной усталой интонацией.
"--- Оставьте свои манеры для деспотического Картеля.
Советую вам ознакомиться с протоколом <<Шляпа>>, который является сводом приемлемых для Ада норм общения.

Штрой смешалась, но тут же овладела собой.

"--*Так точно, вла\ldotst
Хорошо, я учту.

"--*Так-то лучше, "--- милостиво откликнулся голос.
"--- Насчёт Стигмы "--- попробуйте, я отведу от вас удар.
Удача может не прийти к пытающемуся, но к бездействующему она не придёт гарантированно.
Устранить Стигму нужно было ещё пятьсот лет назад, как по мне "--- несмотря на коварство и приспособляемость, в некоторых вопросах она не склонна к компромиссам.

"--*Вы пытались с ней договориться? "--- спросил Самаолу.

"--*И не только я, "--- вздохнул голос.
"--- Проблема в том, что она пацифист до мозга костей.
Её устраивает любой вариант, исключающий военные дествия против Картеля.

"--*И при всём миролюбии Стигмы те, кто пытается её устранить, крайне плохо кончают, "--- ввернул Самаолу.

"--*Это ненадолго, Самаолу, "--- успокоил голос.
"--- Кстати, как вам их название?

"--*Скорбящие, "--- выплюнул Самаолу.
"--- Знакомый культуролог сказал, что само название пахнет каким-то рыцарским кодексом.
Мы во время зачисления Анкарьяль и Грейсвольда в отдел 100 прощупали их в том числе на предмет идеологии.
Данных собрано достаточно, они обрабатываются.

"--*Это здорово, если только старина Грейс не подсунул нам легенду, "--- согласился голос.

"--*Я рискну высказать наше с Кагуя впечатление, "--- проговорил Самаолу.
"--- Анкарьяль и Грейсвольд были абсолютно искренни.
Они совершенно не боялись логических атак и проб.
Более того, они хотели, чтобы их идеологию проверили на прочность.
Кагуя услышала в их речах вызов "--- <<против простой истины не устоит даже ваша хитроумная ложь>>.

"--*Хм, "--- удивился голос.
"--- Как только получите данные культурологов, Самаолу, доставьте их мне лично.
А что насчёт той картинки?

"--*Печальный Митр, "--- доложила Штрой.
"--- Символ из анимистических верований сели.
Именно благодаря ему я вышла на Митриса Безымянного.
Они настолько увлеклись символизмом, что забыли о конспирации.

"--*Если у них и впрямь имеется подобие рыцарского кодекса, это немудрено\ldotst "--- заговорил Самаолу.

"--*Тихо, "--- спокойно выдал голос.
Оба демона разом онемели и подобрались.
"--- Легкомысленного отношения я не приемлю.
В этом уравнении чересчур много неизвестных, чтобы считать Скорбящих сборищем глупцов.
Я ясно выразился?

"--*Так точно, "--- хором сказали Штрой и Самаолу.
Самаолу выглядел совершенно спокойным, а Штрой поёжилась.

"--*Есть ли пути, которые вели бы к Лусафейру?

Штрой и Самаолу переглянулись.

"--*Понятно, "--- недовольно сказал голос.
"--- А жаль.

Штрой снова поёжилась.
Она чересчур болезненно воспринимала любые проявления недовольства со стороны патрона, и Самаолу бросил на неё короткий презрительный взгляд.

"--*Я знаю старину Лу очень давно, "--- криво улыбнулся демон.
"--- Он друг Грейсвольда, но вряд ли заодно со Скорбящими и знает больше нас.

"--*Вы переоцениваете свой опыт, Самаолу, "--- сказал голос.
"--- Во-первых, Лусафейру очень осторожен.
Вы не представляете, насколько "--- это та осторожность, которую не смогут ослабить ни тысячелетний мир, ни клинок у горла, ни обвинение в трусости.
Лусафейру, я не побоюсь сказать, едва ли не единственный в Аду, кто знает цену ошибке.
И при этом он продолжает общаться с Грейсвольдом, который, как известно, под подозрением.
Возможно, что у Лусафейру есть план.
Возможно также, что мы недооценили силу этой связи.

Демоны снова переглянулись.

"--*Вам не кажется, что вы чересчур превозносите наших врагов? "--- скривился Самаолу.

"--*Не врагов, а соперников, Самаолу, "--- терпеливо заметил голос.
"--- Грамотный стратег не употребляет таких терминов, как <<друг>> и <<враг>>.
Оставьте их для тех, кого кормите пропагандой.

"--*Эти <<соперники>> убьют вас при первой же возможности.

Голос тихо засмеялся.

"--*Я не буду переубеждать вас, Самаолу, "--- ваша святая уверенность придаёт пропаганде силу, так же как дружба, возможно, придаёт силу Скорбящим.
Чувства вообще замечательная вещь "--- именно потому, что они императивны по сути, им сложно противиться.
Многие демоны называют акбас проклятием\ldotst но где бы мы были без него?

"--*Вы считаете, что Лусафейру доверяет Грейсвольду так же, как Грейсвольд доверяет Анкарьяль?

"--*Не могу сказать точно.
Однако Лусафейру и Грейсвольд очень долго общались с Тахиро, и это не могло не сказаться на их профессиональных навыках.
Вам для справки, Штрой: Тахиро Молниеносный, Стратег-неожиданность "--- так его звали.
Он обращался с вероятностями так, как мы обращаемся с точными числами.
Сам он канул в вечность, как и почти все близкие ему демоны.
Почти все.

"--*Мне следует прервать общение Лусафейру и Грейсвольда? "--- осведомился Самаолу.

"--*Вы вряд ли сможете это сделать, "--- заметил голос, "--- но усложнить это общение несколькими обычными, легко оправдываемыми методами не мешает.
Если Грейсвольд и Анкарьяль останутся в отделе 100 в изоляции, они обязательно совершат фатальную ошибку.
Фатальную для них\ldotst или для всех Скорбящих, если случайность будет к нам благосклонна.
Да, кстати, насчёт Безымянного.
Речь Грейсвольда произвела некоторое волнение, и к демиургу сейчас приковано незаслуженно пристальное внимание.
Тем не менее это низкоинтеллектуальное скопление масс-энергии успело наступить на ноги половине Ада.

"--*Безымянного оставьте мне, "--- поклонилась Штрой.
"--- Как только шум уляжется, тупица и дикарка дорого заплатят за то, что ввязались в чужую игру.

\section{[U] Селекция в Картеле}

Часть обратного пути Штрой и Самаолу шли вместе.

"--*Всегда хотел спросить вас, Штрой, "--- вдруг заговорил Самаолу.
"--- Правда ли, что в Картеле существуют так называемые <<связки>> "--- множество молодых демонов соревнуются в выполнении какой-либо функции, чтобы были отобраны самые лучшие?

"--*Да, "--- неохотно сказала Штрой.

"--*А что происходит с теми, кто непригоден, Штрой?

"--*Непригодные ликвидируются, Самаолу.

"--*Хм.
Значит, Картель вместо инженерии пошёл по старому доброму пути селекции, указанному ещё матерью-природой.

"--*Инженерия без селекции "--- ничто, Самаолу, "--- ощерилась Штрой.
"--- Даже созданное инженерами следует подвергнуть испытаниям.

"--*Разумеется, вы абсолютно правы.
Занятно.
Хм\ldotst
Я задам вам последний вопрос, Штрой, и перестану беспокоить.
Удовлетворите моё любопытство "--- скольких жизней стоило \emph{ваше} место?

"--*Я оказалась лучше восьмиста тысяч ювеналов.
Это имеет значение?

Самаолу брезгливо хмыкнул.

"--*Это заметно, Штрой.
Вы лучше многих.
Мало кому пришлось пережить столь жёсткий отбор.
Но кое в чём вы уступаете самому последнему легионеру Ада.

"--*И в чём же? "--- осведомилась Штрой.

"--*Над вами всегда, до последнего мгновения жизни будет висеть призрачная рука.
Рука того бездушного механизма, который производил отбор.

И, не попрощавшись, Самаолу пошёл к ближайшему лифту.

\chapter{[U] Светлячки}

\section{[U] Ааман (отрывки)}

Стигма по очереди взглянула на Штрой сильно косящими глазами, словно взглянули два человека "--- весёлый и задумчивый.
Штрой напряглась "--- самый этот взгляд вызывал в её теле тревогу.

Из-за ширмы неторопливо вышел жилистый крепкий мужчина с непробиваемо тупым лицом.
Это лицо могло ввести в заблуждение сехмар, но не хоргета "--- глаза светились умом и странным, непередаваемым фанатизмом.

Ааман Третий Защитник.
Оцифрованный сапиент с одной из периферических планет.
Ад взял на службу несколько Телохранителей из местного племени.
Они следовали своей философии "--- неважно, кто попросил тебя о защите, ты обязан защищать попросившего даже ценой жизни.
Этих пятерых интерфекторов использовали на переговорах как гарантию мира "--- они без колебаний пресекали любую агрессию.

<<Атрис связался со Стигмой и предупредил её насчёт меня.
Или\ldotsq>>

Штрой нахмурилась.
План придётся менять на ходу.
От Аамана она не добьётся ничего "--- урождённый сапиент, согласно некоторым отчётам, ходил в Чистилище как на работу, аналитики уже давно махнули на него рукой.
Приближаться к Атрису было чревато.
Рискнуть и попробовать выжать нужные данные из рыжей стервы?
С Хореймом и прочими она могла надавить на Стигму, но этот демон\ldotst
Что, если его со Стигмой связывает нечто большее, чем простая клятва?

Штрой выросла в связке конкурирующих стратегов, где успех напрямую зависел от скорости изложения идей.
Эта методика была достаточно распространённой.
Стигма же знала, что иногда лучше даже не думать "--- всегда есть вероятность, что кто-то изобрёл устройство для чтения мыслей.

\spacing

"--*Зачем ты занимаешься этой ерундой?
Ты же не биолог и никогда им не была.

"--*Даже у демона могут быть мечты, "--- сказала Стигма.
"--- Однажды я стану биологом.
И жуки будут у меня жить на более веских основаниях.

Штрой подошла к террариуму и открыла его.
Индиго-светлячки тут же разлетелись в разные стороны.

"--*Ой, "--- притворно прикрыла рот Штрой.
"--- Прости.
Ты же сможешь их потом собрать?

"--*Я всё равно собиралась их выпустить, "--- улыбнулась Стигма.
"--- У них есть крылышки и челюсти.
Пусть летают и грызут, что найдут.

\spacing

Штрой поднялась и подошла к Стигме вплотную.
Диковатое оцелотовое лицо пылало сексуальным жаром и яростью ущемлённого самолюбия, и сколь много огня было в девушке, столь же холодным, напрочь лишённым сексуальности, но невероятно гармоничным казалось лицо Стигмы.
Светящиеся чистым светом расчёта глаза одним своим существованием высмеивали амбиции Штрой, и подошедший Ааман властно оттолкнул Штрой, напоминая молодой демонице о своём существовании и своей роли.

<<Ааман молодец, "--- отметила Стигма.
\mulang{$0$}
{"--- Едва ли он знает о смертельном сюрпризе в моём комбинезоне, но последствия физического контакта он просчитал совершенно правильно>>.}
{``He hardly know of the deadly surprise I've got in my suit, but he correctly predicted consequences of physical contact.''}

Штрой, похоже, тоже поняла, какого рода опасность ей грозит.
На её лице расцвела и застыла улыбка превосходства.

\mulang{$0$}
{"--*Хорошая попытка, Стигма.}
{``Nice try, Stijma.}
\mulang{$0$}
{Как вижу, ты не любишь марать руки.}
{You don't like to get your hands dirty, I guess.''}

Стигма вместо ответа показала испачканные удобрениями ладошки.

\mulang{$0$}
{"--*Очень смешно, "--- скривилась девушка.}
{``Very funny,'' the girl made a face.}
\mulang{$0$}
{"--- Разумеется, я сообщу куда следует о твоих методах ведения переговоров.}
{``Of course I'll report about your negotiation techniques.''}

"--*Ты хочешь наказать меня за желание защититься?

"--*Ты уже давно перешла границы защиты.

Стигма позволила себе немного искренности:

"--*Это ты пришла ко мне, а не наоборот.
Я не хочу делать тебе больно.

"--*Ты и не сможешь.
Однако драться тебе придётся, "--- оскалилась Штрой.
"--- Ты предала Орден.
И я это докажу.

Штрой махнула свите и удалилась.
Стигма повернулась к Ааману.

"--*Благодарю тебя, Телохранитель.
Твоя клятва исполнена.

Ааман поклонился и вышел вслед за Штрой.

\razd

Штрой ждала Аамана у лифта.

"--*Здравствуй, друг, "--- сказала она.
"--- Я знаю твою приверженность кодексу Телохранителей Слит-Же и потому хочу попросить тебя о защите.

Ааман медленно повернулся и посмотрел на девушку.

"--*Ты помешал мне служить Аду.
Идти против Ада неразумно.
Если ты будешь защищать меня от заговорщиков\ldotst

"--*Штрой Кольцо Дыма, "--- сказал низким хриплым голосом Ааман.
"--- Мои глаза широко открыты.
Твой сон будет спокоен.
Сталь в твоём голосе, сталь в твоей ладони "--- смерть твоя же, забвение до конца времён и после него.
Я буду сопровождать тебя всегда, пока ты не скажешь, что клятва исполнена.

Штрой нахмурилась.

"--*А если я освобожу тебя от клятвы на день?

Ааман оскорблённо выдохнул.

"--*Больше ты её не получишь, "--- процедил Ааман.
"--- Моя клятва не спит и не ест.

"--*Ты свободен, уходи, "--- рявкнула Штрой, великолепно изображая ярость.
"--- И впредь думай десять раз, прежде чем помогать врагам Ордена.

Ааман поклонился и молча вошёл в лифт.

"--*Отлично, "--- с удовлетворением сказала Штрой.
"--- Осталось четверо Защитников.
Ещё трое подвергнутся пробам, и пятый будет служить мне до смерти.
Раб кодекса "--- раб того, кто найдёт в этом кодексе прорехи.
Странно, что никто не сделал этого раньше\ldotst

<<Умная, но чересчур болтливая>>, "--- молча констатировала Стигма и выключила микрофон.
Светлячок на спине Штрой задрал лапки и упал на пол.

\section{[U] Плут и Ягуар}

Стигма подошла к доске с Метритхис.
Эта игра очень нравилась стратегу.
Стигма часто играла в неё сама с собой "--- никто из знакомых демонов не придёт ради этой очевидной траты времени.

На доске нерешительно топтался зелёный Плачущий Ягуар.
Человек чести, чьё призвание "--- служить.
Однако его окружили чёрный Плут и красный Разбойник "--- не признающие верности, не имеющие чести.
Ягуар был обречён.

Немного подумав, Стигма передвинула Плута поближе к Ягуару.
Плут превратился в Купца, Плачущий Ягуар почернел и превратился в Телохранителя, и попавший под горячую руку Разбойник отправился в коробку.

Стигма улыбнулась и подумала про себя, что отныне Купец и Телохранитель пойдут одной дорогой.

\chapter{[U] Оскал обречённых}

\section{[U] Сиэхено}

\spacing

Даже обществу демонов визоры казались эксцентричными;
старое название "--- оракулы "--- было им дано не случайно.
Инкарнированные визоры переносили отпечаток личности на собственные тела "--- они становились плаксивыми и эмоционально лабильными, имели тягу к самоповреждению, не следили за собой.
Кто-то связывал эту особенность с родом деятельности, но истина была простой и неприглядной "--- ранних визоров презирали за беззащитность, с ними обращались, как с жуками.
Паттерны тех запуганных, покорных созданий используются до сих пор, и никакая коррекция личности уже не способна это исправить.

Девушка подняла голову и улыбнулась слабой желтозубой улыбкой.

"--*Анкарьяль, старая подружка.
А я-то думаю, кто так хочет со мной увидеться, что убивает мою стражу\ldotsq

"--*Меня зовут Таниа, "--- сказала воительница.

Улыбка Сиэхено увяла.
Она схватила бусы и начала раскачиваться из стороны в сторону.

"--*Я охотилась на неуловимого Аркадиу Люпино, "--- прошептала она.
"--- И сама попала в его сети.
Ты его подруга, Тханэ ар'Катхар.
Значит, он тебя оцифровал.
А Анкарьяль\ldotst змея подколодная\ldotst её паттерны\ldotst

Сиэхено всхлипнула.

"--*Я знаю обо всех устройствах в твоём теле и об их назначении, "--- предупредила Чханэ.
"--- Каналы связи контролируют мои демоны.
У нас ещё шестнадцать михнет на разговоры.

"--*За мной придут, "--- прошипела Сиэхено, "--- и гораздо раньше.

"--*Замечательно, "--- равнодушно сказала Чханэ.
"--- Только на Капитуле в процессе легенда о твоей измене.
Она не особо впечатляющая, но, учитывая твою специализацию, тебя при встрече уничтожат без разговоров.
Так что чем скорее за тобой придут, тем сговорчивее тебе стоит быть.

"--*Никто не поверит, что я работаю на Картель!

"--*Разумеется.
И поэтому, согласно легенде, работаешь ты на нас.

Сиэхено заплакала.
Девушка по-детски тёрла ручками незрячие глаза, размазывая слёзы и грязь по треугольному личику.
Что это "--- умелая игра или проявление интимной связи, которая установилась между телом и демоном?
Чханэ пододвинула стул поближе.
Распознать подобное по действиям демона мог лишь опытный визор, но попытаться стоило.

"--*Знаешь ли ты, чьё тело дало тебе приют?

"--*Знаю, "--- всхлипнув, рассеянно сказала Сиэхено.
"--- Я третья наследница, твоя и Ликхмаса ар'Люм.
Истории о вас кормилец рассказывал мне, едва я научилась понимать.

Сиэхено заулыбалась так же неожиданно, как заплакала.

"--*Хаяй, куда же делась героиня моего детства?
Где храбрая и верная Тханэ ар'Катхар? "--- девушка снова начала теребить бусы и раскачиваться.

Сомнения Чханэ рассеялись "--- как и прочие агенты Ада, Сиэхено превратила тело тси в бездушный инструмент.
Взывать к человеческим чувствам не имело смысла.

"--*Ты изуродовала ей глаза, мерзость, "--- невыразительно сказала Чханэ.

"--*Это мой знак, "--- пропела Сиэхено.
"--- Ла-ла-лай, ниии, трааа\ldotst
Я обрабатываю глаза щёлочью всем своим телам.
Так лучше видно.

"--*Ты боишься связи с телом, "--- сказала Чханэ.
"--- Почему?

Сиэхено погрустнела.

"--*Темнота.
Я очень боюсь темноты.

Это было похоже на правду.
Когда только появился ангельский тип заякоривания, урождённые демоны пришли в ужас от образов, возникающих в сапиентном мозге при дефиците внешних сигналов.
Хоргеты не знают, что такое темнота "--- белый шум ПКВ существует всегда, его очень сложно устранить.
Но развитый интеллект легко делает проекции.
Страх темноты засел в сознании многих демонов\ldotst и кое-кто этим манипулировал.

<<Если такие кошмары видят существа с короткой жизнью, что увидим мы\ldotsq>>

Нет зрительного анализатора "--- нет темноты.
Звучит логично.
Однако Сиэхено не интегрировалась в девушку-носителя, и боязнь темноты была лишь отговоркой.
Куда больше визор боялась собственного тела "--- и потому сознательно его ослепила.

\emph{Страх}.

Чханэ почувствовала, что её пробирает дрожь.
Знакомая дрожь, дрожь ярости и негодования.
Глаза тси отлично регенерируют при повреждении.
Сиэхено вводила в глаз щёлочь снова и снова, пока в погибающий нерв не пророс грубый рубец, пока хрустально прозрачную оптику глаза не заплело опалесцирующей коллагеновой сеткой, пока зрительный анализатор не атрофировался из-за отсутствия сигналов.
Что при этом чувствовала девушка, частью которой демон не желал быть?

\spacing

Чханэ почувствовала, что по её лицу текут слёзы.
<<Лис, что мне делать?>>

"--*Бабушка, почему ты плачешь? "--- вдруг сказала тонким голосом Сиэхено.
"--- Я скушаю пирожок и расскажу, какой он вкусный.

Слёзы высохли на загоревшемся лице Чханэ мгновенно, и к горлу подступил гнев.
Эта тварь издевалась над её собственными детскими воспоминаниями.

"--*Ах ты\ldotst

Сзади слабо вспыхнул источник <<света>>, и Чханэ среагировала автоматически.
Удар "--- и визор перестала существовать.
Её красивое хрупкое тело неловко свалилось на стол, невидящие глаза моргнули и тупо уставились в пространство.

Придя в себя, Чханэ поняла "--- Сиэхено запустила отвлекающее устройство, имитирующее готовность к атаке, но не учла поведенческие стереотипы воина-Скорпиона.
<<Никогда не оставляй врага за спиной>> "--- эту истину вдалбливали в Храме годами, и воительница среагировала, не успев осознать происходящее.
Устройство <<мигнуло>> ещё раз и отключилось, ожидая новой команды.

Испытывая отвращение к себе, Чханэ вонзила в сердце наследницы кинжал, нацарапала на грубой поверхности стола Печального Митра и вышла, тихо прикрыв за собой дверь.

\section{[U] У себя дома}

\spacing

"--*Операция пошла не по плану, "--- сказала Тхартху.
"--- Чханэ наследила.
Она случайно убила Сиэхено, и Ад узнает, что она была убита.
Легенда срочно отозвана.
Мы должны залечь на дно.

"--*Хорошо, "--- кивнул Атрис.

"--*Атрис, тебе тоже лучше скрыться.
Следы могут привести и к тебе, и тогда Митхэ\ldotst

"--*Мы никуда не побежим, "--- спокойно сказала Митхэ.

"--*Митхэ\ldotst

"--*Тхартху, дочка, не беспокойся за нас, "--- мягким голосом прошептала Митхэ и обняла женщину.
"--- Мы у себя дома.

\section{[U] Пересмотр договора}

Волны накатывались на песчаный берег одна за другой.
Митхэ и Атрис пили отвар, лопали конфеты, беседовали и смеялись.

"--*\ldotst помнишь то дерево?

"--*Оно называется сорба, Митхэ.
Биологи мне сказали.

"--*Давай посадим на Кристалле рощу таких деревьев?
Они красивые.
Или здесь?
Можем их модифицировать, чтобы они выросли большие и толстые!

"--*Хорошая мысль, "--- одобрил Атрис.
"--- Мне почему-то никогда не приходило в голову, что сорба великолепно смотрелась бы вокруг этой беседки.
Давай допьём отвар и свяжемся с Кольбе и Рабе, чтобы\ldotst

"--*Не торопитесь, "--- раздался сзади знакомый тонкий голос, дрожащий от едва сдерживаемого злорадства.

"--*Здравствуй, Штрой, "--- приветливо сказала Митхэ.
"--- Максим Мимоза, легат Ду-Си, и вы здесь.
Надеюсь, не для того, чтобы <<уничтожить человеческую часть Митриса Безымянного>>.
Не хотите ли выпить?

"--*К сожалению, именно для этого, Митхэ ар’Кахр, "--- звучным, богатым баритоном сказал Мимоза.
"--- И если ваш друг немедленно передаст нам все ключи от системы управления, это произойдёт быстро и безболезненно.

Атрис пожал плечами и отхлебнул отвара.
Мимоза "--- высокий, холодный и стройный северянин, инкарнация одного из лучших тактиков отдела 100 "--- удивлённо приподнял бровь.
Ду-Си, его полная противоположность "--- крепко сбитый, покрытый рельефом огромных, выращенных на синтетических стероидах мышц "--- одобрительно крякнул.
Даже по телу легата было видно, что он любит трудности: на ноге "--- отчётливый шрам от акульих зубов, бритая голова сияет, за ухом виден криво вшитый имплант-инфузомат.
Если верить докладу Аркадиу, Ду-Си "--- выкормыш Воронёной Стали, ушёл на вольные хлеба\footnote
{См. \emph{История и современный правовой статус нейтральных демонов}. \authornote}
ещё до основания Красного Картеля, затем уже в Ордене служил диверсантом в упразднённом отделе 125 и имел богатый опыт на геноциде Девиантных Ветвей, уничтожении Роев и несговорчивых богов.

Идеальное трио.
Штрой "--- провокатор и сборщик информации.
Мимоза и Ду-Си играют неприятеля и возможного друга, а в случае столкновения превращаются в интерфекторов со взаимодополняющими тактиками ведения боя.

\mulang{$0$}
{"--*Атрис, вы ещё с нами? "--- осведомился Мимоза.}
{``\"{A}\={a}tr\v{\i}s, are you still with us?'' Mimosa asked.}
\mulang{$0$}
{"--- Кажется, я вынужден повторить, что\ldotst}
{``I think I have to repeat, you\dots''}

\mulang{$0$}
{"--*Я вас услышал, максим, "--- ответил Атрис.}
{``You've been heard, maccsim,'' \"{A}\={a}tr\v{\i}s answered.}
\mulang{$0$}
{"--- С этим есть некоторые затруднения.}
{``We have some trouble about that.}
Ключей у меня нет, система настроена на <<руку>> Митхэ.
\mulang{$0$}
{Всё-таки я ожидал, что перед тем, как перейти к делу, вы вежливо ответите на вопрос, который задала моя женщина.}
{Anyway, I expected you to do my woman the courtesy of an answer instead of cutting to the chase.}
\mulang{$0$}
{Смею напомнить, что это мой дом, как в широком, так и в узком смысле этого слова.}
{I'd like to remind: you are in my home, in a wide sense and in a narrow sense of the word.''}

Агенты замерли, словно восковые куклы.

"--*Сели за стол и выпили с нами, "--- рявкнула Митхэ.
Натренированный тон годился лишь для фразы <<Сдавайся или умри>>, и слова повисли в воздухе.
Интерфекторы с каменными лицами сели за стол и невозмутимо взяли свои чаши.
Штрой, поколебавшись, последовала их примеру.
Её руки дрожали от сдерживаемой ярости.

"--*Что ты так смотришь на меня, Штрой Кольцо Дыма? "--- холодно осведомилась Митхэ.
"--- Тебе же понравился мой отвар?

Девушка промолчала.

"--*Почему вы не сообщили о ключах отделу 100, Атрис? "--- мягко спросил Мимоза.

\mulang{$0$}
{"--*Я слегка рассеян, "--- признался менестрель, "--- и иногда забываю сообщать контрагенту подробности, которые его не касаются.}
{``I'm absent-minded a bit,'' the minstrel admitted, ``and sometimes I forget to provide counterparty with details which are not their business.}
Итак, в чём нас подозревают на этот раз?

"--*В содействии враждебным Ордену силам, "--- ответил Мимоза.

"--*Каким именно?

"--*Тайна следствия.

"--*Хорошо, пусть будет так, "--- легко согласился Атрис.
"--- Уверены ли вы, что это соответствует действительности?

"--*Коэффициент верификации 0,4803, "--- буркнул Мимоза.

"--*То есть примерно 0,5.
Либо я предал Орден, либо нет.
Просто замечательно.

"--*Вы искажаете суть статистических данных, "--- рявкнула Штрой.

"--*Странно слышать это от тебя, "--- парировала Митхэ.

"--*Вас было вынесено официальное предупреждение\ldotst "--- начал Мимоза.

\mulang{$0$}
{"--*Мне надоели ваши шпионские игры, "--- прервал его Атрис.}
{``I'm tired of these spy games,'' \"{A}\={a}tr\v{\i}s interrupted him.}
"--- Когда Аркадиу Люпино победил на Могильном берегу, он сказал мне, что я и мои создания будут жить счастливо.
\mulang{$0$}
{Однако у Ада своё представление о счастье.}
{But the Hell's got its own definition of happiness.}
\mulang{$0$}
{Стыдно признаться, но даже безумный Эйраки не доставлял мне так много головной боли, как вы.}
{I'm ashamed to admit, even senseless Ejraci wasn't such a severe headache as you are.''}

Интерфекторы усмехнулись.

\mulang{$0$}
{"--*Мне нравится ваш подход, Атрис, "--- широко улыбаясь, сказал Ду-Си.}
{``I like your attitude, \"{A}\={a}tr\v{\i}s,'' Du-Sie said with a big smile.}
\mulang{$0$}
{"--- Называть Ад головной болью я бы не решился.}
{``I wouldn't go so far as to call the Hell `a headache'.}
\mulang{$0$}
{Особенно сейчас\ldotst}
{Especially now\ldots''}

"--*А как ещё вас называть? "--- невесело усмехнулась Митхэ.
"--- Мы "--- ваши союзники, но угрозы я слышу постоянно.
Одна статья, другая статья, десятая статья, пункт сотый, параграф тысячный.
Смерть, смерть, смерть.
Половину законов Ада можно было бы заменить одним этим словом.
После сотой угрозы, какая бы сила за ней ни стояла, уже порядком надоедает.
\mulang{$0$}
{Вы пришли меня казнить, а мне ужасно скучно.}
{You've come to execute me, and I find it boring.''}

Мимоза изобразил холодное удивление:

\mulang{$0$}
{"--*Интересные новости.}
{``Nice news.}
\mulang{$0$}
{Можно узнать, от кого вы слышите угрозы в таких количествах?}
{May I ask, who threatens you so often?''}

\mulang{$0$}
{"--*Да хотя бы от неё, "--- Митхэ махнула рукой на Штрой.}
{``I might start with her,'' M\={\i}tcho\^{e} pointed to Stroji.}
\mulang{$0$}
{"--- Я не удивлюсь, если именно она нас подставила.}
{``I wouldn't be surprised if it was she who framed us up.''}

Демоница вытаращила глаза.

\mulang{$0$}
{"--*Kuna!}
{``\emph{Kuna!}''}

\mulang{$0$}
{"--*Я рада, что ты, милая моя перебежчица, ещё помнишь ругательства Картеля, "--- улыбнулась Митхэ.}
{``I'm glad you, my sweet defector, still remember obscenities of Cartel,'' M\={\i}tcho\^{e} smiled.}

\mulang{$0$}
{"--*Стоп, "--- поднял руку Мимоза.}
{``Stop,'' Mimosa raised his hand.}
\mulang{$0$}
{"--- Митхэ, это очень серьёзное обвинение.}
{``M\={\i}tcho\^{e}, that's a very serious accusation.}
\mulang{$0$}
{Я бы на вашем месте подумал, прежде чем\ldotst}
{If I were you, I would think twice before\ldots''}

\mulang{$0$}
{"--*Поверьте, максим, у нас была масса времени на размышления, "--- прервал его Атрис.}
{``Take it from us, maccsim: we had plenty of time for thinking,'' \"{A}\={a}tr\v{\i}s interrupted him.}
\mulang{$0$}
{"--- Штрой Кольцо Дыма пытается нас оскорбить с первого дня знакомства.}
{``Stroji the Smoke Ring have been trying to abuse us verbally, since the first time we met.}
\mulang{$0$}
{Записи могу предоставить.}
{Recordings can be provided.}
\mulang{$0$}
{Скажите, Мимоза, были ли разрешены такие меры приказами Капитула?}
{Tell us, Mimosa, were such measures allowed by Capitul's orders?''}

\mulang{$0$}
{"--*Исключено, "--- буркнул интерфектор.}
{``No way,'' the interfector answered.}
\mulang{$0$}
{"--- Рискну предположить, что имело место недопонимание.}
{``I would venture to guess a misunderstanding took place.}
Ваше представление о вежливости может отличаться от представлений командующего, живущего в культурной среде Ордена\ldotst

"--*Культура подразумевает соответствие некоторым общим нормам, "--- перебил Атрис.
"--- Я думаю, что наши с вами отношения держатся не только на некой фактической базе, но и на формальностях, которые, кстати, достаточно чётко прописаны в известном вам протоколе под смешным названием <<Шляпа>>.
Я нашёл его очень занимательным.
Там сказано, что провокация, демонстрация власти, лесть и подхалимство являются формами замаскированной агрессии.
Кроме того, там недвусмысленно объясняется, что давление на дружеское и сексуальное чувство в процессе производства является способом установления незаконной, коррупционной связи.
Учитывая, что Штрой Кольцо Дыма умудряется за одну реплику применить все вышеперечисленные методы ведения беседы, недопонимание "--- недопонимание командующим Штрой культурной среды Ордена "--- действительно имеет место быть.

Мимоза промолчал.

\mulang{$0$}
{"--*Далее.}
{``Next.}
Штрой Кольцо Дыма вразрез с заключённым на Деймос-14 договором постоянно предпринимает попытки ограничить мой доступ к системе управления и требует отчёты о моей деятельности.
Доходит до того, что я орешков пощёлкать не могу, не доложив командующему.
Был ли на эти действия приказ отдела 100 или высшего командования Ада?

\mulang{$0$}
{"--*Исключено, "--- повторил Мимоза.}
{``No way,'' Mimosa repeated.}
"--- Однако вы находитесь на территории Ордена Преисподней, и ради безопасности командующий имеет право\ldotst

"--*Замечательно, "--- усмехнулся Атрис.
"--- В таком случае я официально требую от вас подкреплённый нормативными актами документ, чётко проводящий грань между обеспечением безопасности и неоправданным ограничением в действиях.
Этот документ будет очень хорошо смотреться как дополнение к договору <<Демиург "--- Метрополия>>.

"--*Вы имеете на это право, "--- хмуро сказал Мимоза.

\mulang{$0$}
{"--*Далее.}
{``Next.}
Штрой Кольцо Дыма и её демоны регулярно предпринимают попытки самостоятельно исследовать механизмы системы, при том, что у неё есть ключи от системы поиска и всех боевых устройств.
Некоторые попытки вмешательства я не могу трактовать иначе как намеренное вредительство.
Например, не далее как два рассвета назад из-за очередного <<техосмотра>> вышла из строя сначала метеостанция, а затем и три ключевых сейсмодатчика, и землетрясение в Ихслантхаре едва удалось предотвратить.

\mulang{$0$}
{"--*Есть ли у вас доказательства? "--- спросил Мимоза.}
{``Do you have any proof?'' Mimosa asked.}
\mulang{$0$}
{"--- Если есть, почему это прошло мимо нас?}
{``If you do, why it's passed unnoticed by us?''}

\mulang{$0$}
{"--*Я понятия не имею, максим.}
{``I've no idea, maccsim.}
Экологической отчётностью, как вы знаете, занимается команда Корхес Соловьиный Язык.
Мои записи они передали вместе с остальными данными на Капитул.
\mulang{$0$}
{Вероятно, вам следует навести справки там.}
{Maybe you should make an inquiry there.''}

Штрой остолбенела.
Интерфекторы бросили на неё два совершенно одинаковых взгляда.

"--*На будущее, Атрис, "--- проговорил Мимоза, покручивая длинными пальцами чашу.
"--- Обработка такого рода информации "--- юрисдикция не исследовательских отделов, а контрразведки.
Вам повезло, что Корхес имеет право на сбор, передачу и верификацию информации, составляющей военную тайну.
В противном случае вас могли привлечь к ответственности.

"--*Я запомню, Мимоза.
Мне последнее время приходится помнить очень много вещей, не имеющих никакого отношения к моим прямым обязанностям.
Надеюсь, вы также напомните командующему, что всю информацию о системе уже получили компетентные демоны и доложили куда следует.

"--*Компетентные демоны? "--- рявкнула демоница.
"--- Грейсвольд Каменный Молот?

"--*Вы сомневаетесь в его профессионализме, Штрой? "--- вдруг подал голос Ду-Си.
"--- Я не знаком лично с Грейсвольдом, но наслышан о его деяниях.
Весьма впечатляющая биография, надо сказать.
Кстати, в настоящее время он является сотрудником отдела 100, и если вы сомневаетесь в его компетентности, мы всегда можем вас выслушать.

"--*Они все повязаны одной красной ленточкой, "--- буркнула Штрой.
"--- Я уже представляла свои выкладки на этот счёт в отдел 100.

"--*Мы их рассмотрим, Штрой, "--- холодно сказал Мимоза.
"--- Атрис, у вас всё?

"--*Ещё нет, Мимоза.
Согласно договору, заключённому между мной и Адом\ldotst

"--*Договор "--- это пустая болтовня, Атрис.
Ты здесь никто, "--- перебила Штрой.

Атрис улыбнулся и встал.

"--*Ошибаешься, Штрой.
Договор заключается между равными, и стратеги Ада знают это очень хорошо.
Они также знают, как полезен демиург-союзник и насколько опасен демиург-враг.

Пока менестрель говорил, в нём происходила странная метаморфоза.
Он вдруг сбросил личину бродяги.
В его осанке, голосе и движениях появилось величие.

"--*Создатель Тра-Ренкхаля "--- это я.
Хозяин Тра-Ренкхаля "--- это я.
Планета Тра-Ренкхаль "--- это я! "--- демон Митриса засветился ярким светом и демонстративно привёл в готовность боевые модули.
Мимоза прищурился, Штрой едва не упала за оградку.
"--- И если Орден Преисподней в этом сомневается, желает оскорблять Митриса Безымянного, желает убить Митриса Безымянного, пусть попробует сделать это сейчас.
Я расторгну договор, приму бой и напомню агентам Ордена, на чьей планете они стоят!

Демон погас.
Штрой дрожала всем телом, вцепившись в скамью.
Ду-Си с поэтической рассеянностью тянул конфеты из вазочки.

"--*И что вы сделаете, Атрис? "--- невинно осведомился Мимоза.
"--- Боевые модули системы уже вам не принадлежат.

"--*Мой мужчина ничего не будет делать, "--- сладким голосом сказала Митхэ.
"--- А вот Митхэ ар’Кахр может невзначай раскачать тектонические плиты и вернуть Тра-Ренкхаль в первозданный вид.
Это мой дом, и я знаю, за какие петельки дёрнуть, чтобы он рухнул.

"--*А вы успеете дёрнуть, Митхэ? "--- лениво спросил Мимоза.

"--*А вы успеете убежать из эпицентра распада минус-сингулярности, Мимоза? "--- ответила вопросом Митхэ.
"--- Менять полярность мы теперь тоже умеем.
Преобразование Шмидта, "--- Митхэ произнесла звучный термин с неописуемым удовольствием, которое доступно лишь дилетанту.

Мимоза широко улыбнулся воительнице.
Митхэ ответила ему своей самой светлой щербатой улыбкой.

"--*Знаете, "--- непринуждённо сказал Мимоза, "--- я очень не люблю иметь дело с самоубийцами.
Это возмутительно неравное сражение.

"--*Жаль, что для вас отношения с нами "--- лишь очередное сражение, "--- парировала Митхэ.

"--*Всему этому есть одно маленькое препятствие, Митхэ.
Геологи планеты Тра-Ренкхаль.

"--*Смею напомнить, Мимоза, что геологические исследования вами не проводились и землетрясения до сих пор предотвращаю я, "--- подал голос Атрис.
\mulang{$0$}
{"--- Эффективность планеты как источника масс-энергии "--- мой каждодневный труд.}
{``Efficiency of the planet as an energy source is the fruits of my labour.}
\mulang{$0$}
{Я дилетант в политике, у меня масса пробелов в естественнонаучных знаниях, но у меня были тысячелетия, чтобы научиться интуитивно управлять планетой и при этом сводить концы с концами.}
{I'm amateur in politics, my knowledge in science is full of gaps, but I've had thousands years to learn intuitively how to handle the planet while getting by.}
\mulang{$0$}
{Было ли это время у вас?}
{Have you?''}

\mulang{$0$}
{"--*Не прибедняйся, демиург.}
{``Don't be so modest, demiurge.}
У тебя были превосходные учителя с планеты Тси-Ди, "--- с отвращением выплюнула Штрой и тут же осеклась под взглядом Мимозы.

\mulang{$0$}
{"--*А с чего вы взяли, Атрис, что исследования не проводились? "--- осведомился максим.}
{``So, \"{A}\={a}tr\v{\i}s, what makes you think research was not carried out?'' maccsim asked.}

\mulang{$0$}
{"--*А исследования были?}
{``So was it?''}

\mulang{$0$}
{"--*Вы считаете Орден Преисподней идиотами?}
{``Do you take the Order of Netherworld for fools?''}

\mulang{$0$}
{"--*Я был бы рад ответить Ордену Преисподней взаимностью, но мне не позволяет воспитание, "--- парировал Атрис.}
{``I would be glad to reciprocate the attitude shown by the Order, but I'm too polite,'' \"{A}\={a}tr\v{\i}s retorted.}
"--- Поэтому я заявляю вам официально, как эмиссару: Митрис Безымянный желает пересмотреть договор с Орденом Преисподней.
Делегатов ждём завтра в это же время здесь.
И захватите с собой сладости, Ликхэ ар'Митр в Кахрахане замечательно готовит булочки.
Думаю, у Ордена Преисподней найдётся достаточно слитков, чтобы купить этот шедевр на вес кукхватра.

Мимоза натянуто улыбнулся.

\mulang{$0$}
{"--*У вас всё?}
{``Are you done?''}

\mulang{$0$}
{"--*Ещё нет, Мимоза.}
{``Not yet, Mimosa.}
\mulang{$0$}
{Если я сочту какие-либо действия членов Ордена неуважительными, я без предупреждения саботирую производство масс-энергии.}
{If members of the Order do something I find disrespectful, I'll sabotage mass-energy production without warning.}
\mulang{$0$}
{И если Митрис Безымянный говорит <<саботирую>>, это значит, что никто и никогда больше не сможет использовать планету Тра-Ренкхаль в качестве источника питания.}
{And if Mitris the Nameless say `sabotage', it means no one ever will use planet Tr\r{a}-R\={e}nkch\'{a}l as an energy source.}
\mulang{$0$}
{Запишите это дословно во избежание недопониманий.}
{Write it down verbatim to avoid misunderstandings.}
\mulang{$0$}
{Штрой, я не знаю, кому служите вы, поэтому ваше поведение оставляю делом вашей совести.}
{Stroji, I do not know where your allegiance lies, so I leave it up to you to decide how to behave.''}

\mulang{$0$}
{"--*Полно вам, Атрис, "--- осклабился Ду-Си.}
{``Come on, \"{A}\={a}tr\v{\i}s,'' Du-Sie smiled.}
\mulang{$0$}
{"--- Вы не любите свой народ?}
{``You don't love your people?}
\mulang{$0$}
{Ни за что не поверю.}
{I refuse to believe.''}

\mulang{$0$}
{"--*Очень люблю, "--- признался демиург.}
{``I really love them,'' the demiurge admitted.}
\mulang{$0$}
{"--- Особенно тси, ведь они, как уже было сказано, потомки моих прекрасных учителей.}
{``Especially Qi, because, as previously stated, they are scions of my lovely teachers.}
\mulang{$0$}
{Не волнуйтесь, легат.}
{Don't worry, legate.}
\mulang{$0$}
{У меня есть внушительная геномная база, подробнейшие сведения о языках и культуре.}
{I've got huge genome database, detailed information on languages and culture.}
\mulang{$0$}
{Если мне придётся деактивировать планету, разумеется, я расселю мой народ по Вселенной, где только смогу.}
{If I have to deactivate the planet, of course I will resettle my people throughout the Universe, everywhere I can.''}

Штрой приобрела цыплячий оттенок.
Ду-Си крякнул с неподдельным восхищением и шлёпнул ладонью по столу:

\mulang{$0$}
{"--*А грамотно он нас за яйца взял, а, Мими?}
{``He's quite competently got us by the balls, hasn't he, Mimi?''}

"--*Ещё раз назовёшь меня <<Мими>> "--- снова получишь пенальти на три ранга, "--- бледный как снег Мимоза встал чуть резче, чем следовало, и принялся поправлять смятую одежду.
"--* Вам же, Атрис, хочу напомнить: ваша планета "--- хороший источник, но один из многих.

"--*Мимоза Шёлковая Сталь и Орден Преисподней "--- также одни из многих, "--- ответил Атрис, "--- но это не обесценивает ни мой выбор, ни ваши заслуги.

Мимоза и Ду-Си коротко поклонились и вышли.
Штрой, злобно посмотрев на влюблённых, демонстративно опрокинула вазочку с конфетками и тоже вышла.
Круглые цветные сладости, весело стуча, разбежались по всем углам беседки.

"--*Думаешь, это поможет? "--- шепнула Митхэ.

"--*Уже помогло, "--- улыбнулся Атрис.
"--- Моё поведение больше соответствовало поведению уставшего и невиновного, чем заговорщика.
Чем больше затянется следствие, тем больше путаницы.
Пока в деле путаница, трогать нас не будут "--- я им нужен.

"--*А если нас всё-таки смогут зацепить?

"--*Аркадиу сказал, что следствие уйдёт в сторону адской резидентуры.
Там есть демон "--- мастер мутить воду, он сможет повести следствие по нужному пути.
А нас нужно подготовиться к завтрашнему разговору.
Мы потребуем символические привилегии и пару реальных свобод, Орден поторгуется, мы уступим привилегии, Орден всё поймёт и согласится, мы разойдёмся довольные.
Штрой отправится домой.
Разумеется, она докажет, что не подставляла нас, но репутацию мы ей подмочили основательно.

\mulang{$0$}
{"--*<<В том, что вас скомпрометировали, есть доля вашей вины>>, "--- процитировала Митхэ.}
{``\emph{You're partly responsible for being compromised,}'' M\=\i tcho\^{e} quoted.}
\mulang{$0$}
{"--- <<Аккуратнее выбирайте друзей>>.}
{``\emph{Be careful making friends.}''}

\mulang{$0$}
{"--*И врагов тоже, "--- заключил Атрис.}
{``Making enemies, too,'' \"{A}\={a}tr\v{\i}s finished.}
\mulang{$0$}
{"--- Хочешь на берег?}
{``You want to go to the coast?''}

\mulang{$0$}
{"--*Хочу, милый, "--- обрадовалась Митхэ.}
{``I do, honey,'' M\=\i tcho\^{e} rejoiced.}
"--- Давай наберём ракушек, я давно хотела сделать что-нибудь красивое.
А вечером узнаем у биологов, как выращивать сорбу.

\mulang{$0$}
{"--*Вряд ли мы успеем её вырастить, "--- загрустил менестрель.}
{``We seem to have few time for growing it,'' the minstrel got upset.}
\mulang{$0$}
{"--- Следствие следствием, но когда Орден сведёт баланс сил без нас "--- придётся отдать наш дом сельве\footnote
{Бежать.
Практика символически отдавать покинутый дом сельве распространена среди северных сели --- в доме ломали крышу и пробивали половые доски, открывая тем самым доступ растениям. \authornote}.}
{``The investigation is what it is, but if the Order balances forces without us, we'll be forced to leave our home to the Silva.''}

\chapter{[U] Ответный удар}

\section{[U] Неудача с Телохранителем}

\spacing

"--*Штрой, вы хотя бы день можете прожить без переполоха?
Я слышал, что отделу 100 пришлось нейтрализовать одного из Телохранителей, чтобы спасти вас.
Надеюсь, вы не пытались его завербовать?

Штрой вытаращила глаза.

"--*Понятно, "--- буркнул голос.
"--- Вы могли хотя бы спросить меня перед таким опасным шагом.
Он на вас напал, не так ли?
Каковы потери?

"--*Я потеряла четверых агентов, "--- Штрой склонила голову.
"--- Прошу прощения.

"--*Штрой, запомните раз и навсегда.
Ад "--- это клубок интриг, и если вам кажется, что кто-то что-то до сих пор не сделал "--- скорее всего, на это есть веские причины.
В Кодексе Слит-Же последний пункт звучит так: <<Если подзащитный использует Кодекс для манипуляции Телохранителем, подзащитный должен быть уничтожен>>.
Этот пункт может трактоваться Телохранителем по его усмотрению.

"--*Я тщательно замаскировала манипуляцию!

"--*А я тебе говорил, что ты недооцениваешь Стигму, "--- заявил Самаолу.
"--- Может, Телохранитель и достаточно глуп для тебя, но Стигма достаточно умна, чтобы растолковать сложное глупцу.

"--*С Защитниками шутки плохи, "--- подытожил голос, "--- и я запрещаю вам приближаться к ним впредь.

Штрой кивнула с явным облегчением.
Видимо, у неё отпало всякое желание приближаться к Защитникам.

"--*А что насчёт Стигмы?

"--*Дальнейшие попытки чересчур опасны.
Она уже едва не вывела вас из игры с помощью Безымянного.

"--*Это была она?
Это она инструктировала Безымянного?

"--*В этом даже я не сомневаюсь, "--- буркнул Самаолу.
"--- Почерк сложно не узнать.
Она раззадоривает врага мнимой слабостью, и он сам бросается на подставленный нож.

"--*Но я хочу попробовать ещё раз\ldotse

"--*Никаких <<но>>, Штрой, "--- отрезал голос.
"--- Я не хочу вас потерять.

Последняя фраза огрела Штрой словно кнутом.
Она упала на колено и прижала руку к груди:

"--*Я вас не подведу, владыка!

"--*Вы опять начинаете?
Впрочем, ладно, продолжайте.
Стыдно признаться, мне это начинает нравиться.
Но не при агентах.

\spacing

\section{[U] Талианский канал}

\spacing

"--*Сиэхено успела оставить сообщение перед смертью, "--- сказал Самаолу.
"--- Скорбящие используют для сообщения оригинальную восьмиполосную таблицу языка Эй, сочетающую макропараметр планеты Тра-Ренкхаль и неизвестный микропараметр.
К сожалению, сообщение неполное.
Это вся информация, известная на сегодня.

"--*Возможно, какая-то вариация Талианского канала, "--- предположил голос.
"--- А, вы не в курсе, Самаолу.
А вот Штрой должна знать.
Макропараметром таблицы являются географические координаты.
Микропараметром может быть что угодно.
В частности, агенты Картеля использовали бутылочки из цветного стекла, выкладывая ими в определённых координатах оставшуюся часть сообщения.

"--*Требуются гигантские затраты, чтобы найти сообщение, "--- мрачно сказала Штрой.
"--- Но в итоге Ад справился, Талианский канал был раскрыт.
Это положило весомый камешек на чашу победы Ада при Десяти Звёздах.

"--*Потому что когда бутылочки появляются в девственном лесу или на необитаемом острове, это наталкивает на определённые мысли, "--- согласился голос.

"--*Я распоряжусь насчёт поисков, "--- кивнул Самаолу.

"--*Не тратьте время, "--- сказал голос.
"--- Талианский канал может быть доработан с учётом местности, климата и прочих особенностей.
Кроме того, сообщения могут быть временными "--- например, изо льда или песка.
Не забывайте также, что одним из Скорбящих почти наверняка является демиург, что значительно расширяет список возможных макропараметров планеты.
Может, он вообще кодирует сообщения ветром, молниями или дождём.
Звучит дико, но ведь теоретически это возможно?

"--*<<Слова не имеют смысла, если нет того, кто может их понять>>, "--- процитировала Штрой.

"--*Штрой! "--- восхитился голос.
"--- Вы цитируете ненавидимого вами Люпино?
Я чрезвычайно рад, что вы растёте как профессионал.

"--*Враг тоже может быть прав.

"--*Я вам скажу больше, Штрой "--- иногда враг ближе к истине, чем вы, но природа мыслящего существа подразумевает ещё и некое комфортное расстояние от этой истины.
Именно поэтому мы стремимся к победе, а Скорбящих устраивает любой мир.

"--*Вы хотите сказать, что мы идём неправильным путём? "--- ошарашенно спросил Самаолу.

"--*Ни в коем случае.
Я хочу сказать, что местоположение истины во Вселенной для практического применения абсолютно неважно.
Важно найти собственное место.
Скорбящие же ищут саму истину "--- и в этом их слабость, потому что даже найдя, они не обретут ничего значимого для себя.

"--*Боюсь, как бы я не начала им сочувствовать, "--- скривилась Штрой.

"--*Сочувствуйте, Штрой, сочувствуйте и не бойтесь!
Ненависть заставляет отрезать врагу пути к отступлению, и враг сражается до конца.
Если же вы из сочувствия оставите им лазейку "--- они сдадут свои позиции проще и быстрее.

"--*Тахиро Молниеносный как-то говорил, что уважение и милосердие более выигрышны в долговременной перспективе, нежели деспотизм и жестокость.

"--*И его слова не лишены смысла.
Но он не упомянул, что куда более выигрышным является их грамотное сочетание.

"--*Вы имеете в виду жестокость к врагам и уважение к друзьям?

"--*Я имею в виду ситуационный подход.
Иногда следует щадить и даже спасать соперника, если его гибель принесёт вам больше вреда, чем пользы.
Иногда следует быть жестоким к союзнику, если выгода от этого высока.
Именно поэтому я и не употребляю термины <<друг>> и <<враг>>.
Они чересчур безапелляционные.

"--*Так что делать с сообщением Сиэхено? "--- осведомился Самаолу.

"--*То, что я сказал.
Займитесь не кодом, а подозреваемыми "--- они сами выведут нас куда надо.

\section{[U] Омега-чувство}

\spacing

"--*Только что пришли данные от биологов, "--- сказал Самаолу.
"--- У тси обнаружили ещё один орган чувств "--- омега-чувство.
Они способны воспринимать колебания ПКВ средней мощности.

Молчание.

"--*Самаолу, приступайте к реализации варианта номер восемнадцать, "--- наконец сказал голос.
"--- Мы должны в течение десяти дней убедить Орден в необходимости геноцида тси.
Если не получится "--- на Тра-Ренкхале обязан произойти катаклизм.

"--*И я даже знаю, на кого можно повесить ответственность, "--- ухмыльнулась Штрой.
"--- К слову.
Вы не рассматривали моё предложение насчёт фальсификации\ldotst

"--*Штрой, мой ответ "--- нет, "--- отрезал голос.
"--- Я понимаю вашу любовь к риску, но это не просто риск, это самоубийство.
ФПГП "--- это взрывчатка, детонирующая от кашля.
Да, это лёгкий способ устранить Безымянного, но если отдел 100 выйдет на кого-то из нас "--- то можно сразу рыть могилы для всех.
Есть данные, что биологам и информатикам, занимавшимся совершенно другими проблемами, в определённый момент приходило письмо из отдела 100.
С одной-единственной цифрой "--- номером статьи.
И если ты в тот же день, в тот же час не свернёшь всю ветвь исследований "--- ты просто исчезнешь, а твой отдел расформируют.
ФПГП Орден боится как огня, даже больше, чем тси.
Так что ищите другие способы.

\section{[U] Вирус против вируса}

\spacing

"--*Тхартху не сомневается в том, что на этот раз они точно решили уничтожить тси.
Но их метод\ldotst

"--*Что за метод?

"--*У тси традиционно практически отсутствует понятие половой гигиены.
Они не знали заболеваний, передающихся половым путём.
Кагуя сказала, что радикальная группировка разрабатывает вирус, который будет передаваться именно половым путём.

Чханэ закрыла глаза.

"--*Они нашли слабое место народа.

Митхэ зарычала.

"--*Проклятие.
Тси вымрут за год, если мы это не остановим.

"--*Есть ли какие-то возможности предупредить народ?

"--*Вряд ли, "--- опустил голову Атрис.
"--- Ограничивать половые сношения?
Разработать средства защиты?
Если за дело возьмётся Орден Преисподней, шансов у нас мало.

Митхэ смотрела на менестреля во все глаза, словно впервые увидела.

"--*Извини, моя флейта.
Похоже, твой народ обречён.

"--*Хоть что-то! "--- Митхэ, казалось, готова вцепиться другу в шею.
"--- Хоть какой-то способ, Атрис!
Ты бог этого мира или нет?!

"--*Ты права, "--- ответил менестрель.
"--- Увы, я всего лишь бог этого мира.

"--*Твоё слово "--- закон!
Скажи жрецам!
Объясни им\ldotst

"--*И что жрецы скажут народу?
Что важнейший процесс для размножения народа смертельно опасен?
Были бы у них соответствующие технологии, всё стало бы проще.
Лекарства, контрацепция.
Однако на половое воздержание и моногамный образ жизни тси просто не пойдут "--- этого не было в их культуре сотни тысяч лет.

Митхэ сдалась.

"--*Сколько у нас времени?

"--*Декада, может, две, не больше, "--- ответил я.
"--- Впрочем, есть один вариант.
Если мы запустим на Тра-Ренкхале собственный вирус "--- не настолько смертельный, но тем не менее опасный "--- группировка Самаолу заляжет на какое-то время.
Официальная политика Ада остаётся прежней "--- тси должны жить.
Мы покажем агентам Ада, что тси угрожает опасность, и они будут настороже.

Митхэ хмыкнула.

"--*У тебя есть варианты лучше?

"--*Я тебя прошу только об одном, биолог, "--- тихо сказала женщина.
"--- Не перестарайся.
Иначе я убью тебя лично.

\razd

Улицы Тхитрона опустели.
По ним сновали только люди, носившие трупы к крематорию, который выстроили на месте бывшего постоялого двора.
Митхэ смотрела на это, и по её лицу текли злые слёзы.

"--*И все эти смерти лишь затем, чтобы народ мог просуществовать чуть дольше.

\spacing

\section{[U] Отражённый удар}

\spacing

"--*Если это не наш вирус, объясните, какого дьявола произошло, "--- ледяным тоном потребовал голос.

"--*Есть две версии, повелитель, "--- доложила Штрой.
"--- Первая "--- кто-то пытается устроить геноцид параллельно с нами.
Версия неубедительная, так как их вирус недостаточно продуман "--- низкая смертность.
Вторая и более правдоподобная "--- Скорбящие прознали о нашем плане и попытались нам помешать.
Вирус выкосил слабых, и у оставшихся в живых больше шансов пережить наш сюрприз.
Кроме того, из-за явного диверсионного характера эпидемии на уши встали наши агенты.
Нам придётся отложить атаку.

"--*Я уже начинаю сомневаться, что Митрис на их стороне, "--- заявил Самаолу.
"--- Если анализ его личности верен, он бы никогда не позволил Скорбящим пойти на такой шаг.

"--*Если бы это был единственный способ спасти тси\ldotst "--- начала Штрой.

"--*Это не спасение, Штрой, "--- раздражённо буркнул Самаолу.
"--- Скорбящие выгадали для своих обожаемых тси в лучшем случае год.
Если они за этот год не перережут нас с тобой, то их усилия будут тщетны.

"--*За год, Самаолу, может произойти всё что угодно, "--- заметил голос.
"--- Этот ход остался за ними, хоть вам и не хочется этого признавать.
Они пожертвовали сливой, чтобы спасти персик\footnote
{Намёк на <<Тридцать шесть стратагем>> Цина "--- государства Древней Земли. \authornote}.
И, чёрт бы меня побрал, на этот раз я не сомневаюсь, что против нас играл Аркадиу Люпино.
Ищите другие способы, и поменьше биологии.
На этом поле мы пока позади "--- у него была целая жизнь, чтобы разложить тси по клеточкам.
Мы вынуждены довольствоваться костями со стола Тра-Ренкхальского отдела биологии.

"--*Возможно также, что Люпино получил массу данных по тому, как тси сопротивляются искусственным вирусам, "--- сказала Штрой.

"--*Именно, "--- подтвердил голос.
"--- Поэтому ещё раз "--- ищите другие способы.
Я недооценил Люпино.
Больше мы не будем играть по его правилам.

\spacing

\chapter{[U] Солёная борода}


\section{[U] Свой среди чужих (переработать)}

\spacing

"--*Все данные есть, "--- сказал Грейсвольд.
"--- Они смогут пробраться на Тси-Ди.

"--*Знать бы ещё, куда именно.
В книгах сели я не нашла зацепок.

"--*Ты не аналитик, "--- поморщился Грейсвольд.
"--- Если бы мы могли задействовать аналитиков или получить доступ к их данным\ldotst

"--*Можно пустить запрос по линии Стигмы, "--- предложила Анкарьяль.
"--- Вдруг у нас есть агенты-аналитики.

"--*Ты вообще знаешь, сколько у нас агентов?

"--*Понятия не имею.

"--*И я тоже.
Конспирация имеет свои минусы.
Агентов может быть сотня, а может и всего трое.
Попробовать можно, но вероятность мала.

"--*Лусафейру не может помочь?

Грейсвольд с жалостью посмотрел на Анкарьяль.

"--*Нар, меня аннулируют в тот же час, в который я задам Лусафейру вопрос, даже косвенно имеющий отношение к кольцевым теплицам.

Анкарьяль задумалась.

"--*Тогда шпионаж.

"--*Нет, "--- Грейсвольд, похоже, придумал что-то интересное, его глаза загорелись.
"--- Мы воспользуемся служебным положением.
Мы же с тобой контрразведка, верно?
Отыщем агента Картеля под прикрытием и убедим его отыскать нужную нам информацию.
В случае его поимки мы почти чисты.
В случае его успеха у нас есть интерфектор, который в удобный момент выжмет его досуха и уничтожит, "--- Грейсвольд церемонно поклонился Анкарьяль.

"--*Ты не считаешь, что давать Картелю информацию о местоположении кольцевых теплиц "--- плохая идея?
Всегда есть вероятность, что мы упустим агента и он сольёт\ldotst

"--*Она для них бесполезна.
Проникнуть через барьерную высоту они всё равно не могут, "--- Грейсвольд довольно осклабился.
"--- Но вряд ли они упустят возможность узнать хоть что-то.

Анкарьяль не удержалась и восхищённо улыбнулась.

"--*Грейс, ты сам дьявол.
Может, в таком случае передадим им не данные, а самого агента тёпленьким?

"--*Отличная идея.
Свяжись со Стигмой.
Думаю, она оценит.

\section{[U] Старый знакомый}

\spacing

Я, как зачарованный, смотрел на захваченного нами демона.
Сомнений быть не могло, это он, мой создатель "--- Яйваф Солёная Борода из клана Дорге.

"--*Здравствуй, Яйваф.

Демон неподдельно вздрогнул и уставился на меня.

"--*Ты меня зна\ldotst?
Ааах, вот оно что.
Аркадиу Валериану Люпино.
И тебе здравствовать, маленький предатель.

Я молчал.
Это была чересчур очевидная, чересчур человеческая попытка вывести меня из равновесия.
Яйваф ухмыльнулся.

"--*Ты вырос.

"--*Дай мне доступ ко всей имеющейся у тебя информации.

Яйваф взглянул мне в глаза.

"--*А если нет?

"--*Тогда тобой займётся интерфектор.

Яйваф поёжился.

"--*Вам не взять информацию силой.

"--*Ты создал меня по своему образу и подобию.
Сколько времени потребуется интерфектору, если он об этом узнает?

"--*То есть ты не хочешь меня уничтожить?

"--*Вначале хотел, "--- честно ответил я.

"--*Это трюк.
Ты убьёшь меня в любом случае.

"--*Твоё мнение не меняет сути дела.
Обещаю: без фокусов дашь доступ к памяти "--- уйдёшь живым.
Твой единственный шанс выбраться "--- положиться на моё слово.

Яйваф поджал губы.

"--*Какая именно информация тебе нужна?

Я молчал.

"--*Проклятие Аду и Небесам, ты действительно вырос.

"--*Ты сделаешь то, что я сказал, с требуемой точностью.
Третьего предупреждения не будет.

Яйваф кивнул.
Его демон медленно, по одному, начал снимать барьеры с модулей памяти.

"--*Помнишь, Аркадиу, как мы задали жару адским чертям на Серпенциару? "--- отстранённо сказал он.
"--- Виа Галоледика на десять лет стала красноватой.
А потом ты нас предал.

"--*Я больше не служу Аду, "--- сказал я.

Яйваф усмехнулся.

"--*Знай я тебя хуже, я бы решил, что ты грубо и неумело пытаешься меня одурачить.
Но это правда.
То, что ты опасен равно для Ада и Картеля, мне следовало понять в тот момент, когда ты, мальчишка, приставил к моей голове ружьё.

<<Информация у меня>>, "--- сообщила Чханэ.

Я кивнул.

"--*Тебе хватит тридцати секунд?

"--*Смеёшься? "--- сказал Яйваф.
"--- Достаточно и двух.

"--*Тогда прощай, "--- сказал я и отключил глушитель сигналов.
Опустевшее тело Яйвафа закатило глаза и повисло на ремнях кресла.
Мозг, который ещё десять секунд назад нёс демонические маркёры Яйвафа, превратился в некротизированную клеточную кашу.

\chapter{[U] Отшельница}

\section{[U] Ошибка в префиксе}

\spacing

"--*Я нашла кое-какие зацепки, "--- сказала Чханэ и придвинула ко мне книгу.
"--- Смотри.
Иногда в префиксе имени указывается биологический вид.
Редко, но бывает.
А теперь\ldotst вот.
Сотканный-из-Темноты-Заяц.

"--*Так, "--- кивнул я.
"--- Она человек.
Хотя постой\ldotst

Я вгляделся в иероглифы.

"--*Ага, тоже заметил? "--- улыбнулась Чханэ.

"--*Описка, "--- махнул я рукой.
"--- Переписчик был уставшим и не вполне понимал смысл написанного.
Спутать иероглифы <<плант>> и <<человек>> довольно просто "--- они различаются одной деталью.

"--*А здесь? "--- Чханэ пролистала на несколько страниц вперёд.

"--*А здесь переписчик заметил несоответствие и исправил по образцу более ранней записи.

"--*Хорошо, "--- сказала Чханэ.
"--- А что ты скажешь насчёт этой же книги, найденной у ноа?
В ней стоит такой же знак.
В отчёте культуролога Ада несоответствие упомянуто, и он просил отдел аналитики на всякий случай обработать эту информацию "--- вдруг здесь скрывается код.

"--*Твоя взяла.
Придумай объяснение логичнее моего.

"--*Существует-Хорошее-Небо мало писал о теплицах, несмотря на то, что они играли для народа тси огромное значение.
Всё, что мы знаем "--- на Стальном Драконе была теплица, впоследствии сожжённая неисправным двигателем\ldotst

"--*Постой, постой.
При чём здесь иероглиф <<плант>>?

"--*При том, Лис, что он состоит из двух частей "--- <<растение>> и <<человек>>.
А как мы знаем, в легенде о Садах кольцевая теплица выступает в роли человеческой женщины.
Чем тебе не <<человек-растение>>?

"--*У тебя есть ещё какие-то данные в пользу этой версии?
Что сказал отдел аналитики?

Чханэ замялась.

"--*Отдел аналитики счёл это опиской.
Но\ldotst

"--*Она права, "--- вдруг сказал Атрис.
"--- Отделу аналитики не хватило данных, которыми располагаю я.

Все разом повернулись к менестрелю.

"--*Я знал Заяц.
Она выглядела\ldotst немного необычно для человеческой женщины.

"--*А точнее? "--- требовательно спросила Чханэ.

"--*У неё на коже был странный рисунок, похожий на древесную кору, "--- пожал плечами Атрис.
"--- Я всегда думал, что это своеобразная травма или мода.
Многие тси украшали тело татуировками, почему бы и не шрамами?
Там их было трое таких, Заяц, ещё одна женщина-человек и женщина-плант\ldotst

"--*И все три "--- женщины, "--- заметила Митхэ.
"--- Совпадение\ldotsq

Мы замолчали и посмотрели друг на друга.
Из-за набора фактов стал чудесным образом проглядывать смысл.
Я улыбнулся и посмотрел на подругу.

"--*Молодчина, Змейка.
Знать бы ещё, как её найти\ldotst если она не погибла за эти десять тысяч дождей.
Она могла сменить внешность.

"--*Я могу попробовать, "--- предложил Атрис.
"--- Доступ к системе поиска Ад мне оставил, а возможности у неё приличные.

\section{[U] Ущелье Мёртвого Ребра}

Среди прибрежных скал ютилась небольшая пещерка.
Это место было известно как самая южная обитаемая точка северного полушария;
оно издревле носило странное название "--- Ущелье Мёртвого Ребра.
Волны Могильного пролива хлестали по подножию скал с первобытной яростью, в воздух летели миллионы капель, но ни одна, даже самая мелкая водяная капелька не достигала этого естественного убежища.
Вокруг не росло ничего, даже мхов "--- экваториальная радиация выжигала всё, что находилось далеко от воды и тени.

<<Думаешь, здесь кто-то живёт?>> "--- скептически спросила Чханэ у Безымянного.

<<Мы её видели у входа>>, "--- сообщила Митхэ.

<<Она не особо изменилась за эти годы, "--- добавил Атрис.
"--- Разве что одежду другую носит.
Со стороны "--- вылитая крестьянка-ноа>>.

<<Значит, она всё-таки жила среди людей>>, "--- заметил я.

Мы двинулись вглубь пещеры.
Жар разом отступил, из глубины потянуло свежим солёным сквозняком "--- значит, пещера имеет выход к воде.
Атрис подтвердил мою догадку, передав всем схему пещеры.

<<Интересно, чем она питается?>> "--- поинтересовалась Чханэ.

<<Фотосинтезом, наверное>>, "--- пожал плечами Атрис.

<<Или рыбу ловит.
Вода как молоко, здесь должна быть хорошая рыбалка>>, "--- засмеялась Митхэ.

Вскоре мы вошли в грот недалеко от нижнего выхода.

<<Вот, полюбуйтесь>>, "--- довольно сказал Атрис.

На гладком голом камне спала маленькая прелестная женщина в поношенном платье ноа, подложив пухлую ладошку под голову.
Казалось, её не смущает ни сквозняк, ни сырость каменных стен.
Рядом стояла невысокая тумба, красиво обложенная круглыми камушками, в центре комнаты гордо возвышался странной формы очаг.
В пещерку залетела случайная летучая мышь, щёлкнула и вылетела тем же ходом.
Женщина приоткрыла глаз и тут же, устроившись поуютнее, заснула снова.

"--*Я знаю, что вы здесь, "--- вдруг раздался в тишине её мягкий голос.
"--- Хватит на меня <<смотреть>>.

Мы несколько смущённо образовали голограммы и подошли к ней.

"--*Как вас много-то, целая делегация, "--- насмешливо сказала Заяц и, перевернувшись на спину, сладко потянулась.
"--- Жаль, угостить вас нечем.

"--*Здравствуй, "--- приветливо сказала Чханэ.
"--- Меня зовут Чханэ ар’Качхар э’Чхаммитр.

Заяц села, разгладила старое платьице и мило оскалилась:

"--*Тебя же оцифровали.
Зачем ты говоришь с этим дурацким западным акцентом?

"--*А мне нравится, "--- обиделся я.
"--- Я, может, из-за акцента в неё и влюбился когда-то.

Заяц захохотала.

"--*Ну, чувство юмора у тебя есть, хоргет.
Не такой уж ты и мерзкий.

"--*Давно не шутила, Сотканный-из-Темноты-Заяц? "--- поинтересовалась Митхэ.

"--*Так вы и имя моё знаете, "--- заметила Заяц.
"--- Да, незнакомка, давненько.
Ну, и что же вам от меня нужно?
Или лучше так "--- на чьей вы стороне?

"--*Мы на стороне тси, "--- сказала Митхэ.
"--- Мы и есть тси.

"--*Да что ты?
Пока я вижу только голограмму, образованную лживым хоргетом.
Кстати, как интересно "--- вы ленточкой связаны с\ldotst а\ldotst

Заяц вдруг уставилась на скромно стоящего в углу Атриса.
Демиург изменил внешность "--- вместо худого улыбчивого красавца стояло человекоподобное существо с мягкими, похожими на оленьи рогами.
Существо спокойно, со смесью ностальгии и печали смотрело на женщину.

На лице Заяц блуждала улыбка, губы её тряслись, по лицу текли слёзы.

"--*Безымянный\ldotsq

Атрис раскрыл объятия "--- и Заяц вдруг кинулась к нему.

\mulang{$0$}
{"--*Безымянный\ldotst}
{``Nameless\dots}
\mulang{$0$}
{Милый мой, рогатый мой\ldotst}
{My dear\dots deer-horned\dots}
Ты вернулся\ldotst "--- бормотала она, прижимаясь к голограмме.

"--*Смеёшься, Зайчик?
Я от вас и не уходил, "--- прошептал менестрель.

\section{[U] Отшельница}

"--*Даа, "--- протянула Заяц, с аппетитом уплетая жареную рыбу и пытаясь сушить влажный подол платья над очагом.
"--- Я слышала, что творилось здесь, на Планете Трёх Материков, с сотню лет назад, но чтобы такое\ldotst
Так, значит, сначала Картель, потом Ад, а теперь ещё и доморощенные бунтари "--- Скорбящие?

"--*Да, "--- подтвердил Атрис.
"--- Кстати, мы подозреваем, что ты "--- кольцевая теплица.

"--*Подозреваешь? "--- захохотала Заяц.
"--- А то, что я живу уже\ldotst
О жизнь моя, сколько же я живу-то?

"--*Мы хотим взять биоматериал для исследований, "--- сказала Чханэ.
"--- У нас есть план\ldotst

Я вкратце рассказал о части плана, связанной с кольцевой теплицей.

"--*Всё это очень интересно, но вопрос явно не ко мне, "--- бросила Заяц, одним укусом обезглавив следующую рыбину и с хрустом перемолов зубами рыбий череп.
"--- То есть да, когда-то я была кольцевой теплицей, но потом изменила генотип.
Сейчас я просто бессмертная женщина.
Бессмертная и вечно голодная. Можете проверить.

Заяц пожала плечами и вгрызлась в филейную часть чёрной скумбрии.
Затем вытащила из-за пазухи фляжку вольфрам-титанового сплава и протянула Чханэ:

"--*Змейка, набери мне водички.
Вон в том коридоре в конце ветровая ловушка, черпай из ниши под сталактитами.
Потом подсоли морской из колодца, примерно один к тридцати.

Чханэ кивнула и отправилась выполнять просьбу.

Я просканировал Заяц и передал вывод остальным.
Она сказала правду: за исключением девяти необычных генов, Заяц соответствовала женщине-тси.
Атрис и Митхэ пригорюнились.

"--*Извините, "--- пробормотала Заяц.
"--- Я ж не знала, что доживу до такого.
Небо сказал "--- информацию о теплицах уничтожить.
Я уничтожила.

"--*А другие были? "--- спросил Атрис.
"--- Кроме той, которая сгорела в корабле\ldotst

"--*Были, "--- сказала Заяц.
"--- Две "--- Искорка и Листик.
Они все сделали то же, что и я.
О дальнейшей их судьбе ничего не знаю, последний раз виделись незадолго до нашествия Безумных.
Скорее всего, они\ldotst
Безымянный, ну что ты так на меня смотришь?

Все повернулись к Атрису.
Менестрель смотрел на Заяц во все глаза.

<<Вы тоже это видите?>>

<<Что?>> "--- спросил я.

<<Эманации, "--- пояснил Атрис.
"--- Их нет!>>

<<Нуль-существо? "--- удивилась Митхэ.
"--- Не может быть>>.

<<Данных мало, "--- поддержал я Митхэ.
"--- Но на всякий случай собирайте статистику>>.

Наступило молчание.
Заяц смущённо бросила к стене обглоданный рыбий скелет и вытерла руки о платье.

"--*Ребята, я не могу вам помочь.
Основную техническую информацию я тоже стёрла из памяти "--- мало ли что.
Оставила только своё, личное "--- о друзьях, о любовниках, о музыке.
Живу вот, сама не зная зачем.
А что делать\ldotst жить-то хочется.
Я немного инфантильна, если вы заметили.
Это характерный признак кольцевой теплицы в образе сапиента "--- мы всегда остаёмся детьми.

"--*Разве это не опасно для вас? "--- удивился я.

"--*Наоборот, "--- возразила Заяц.
"--- Если бессмертное существо будет чересчур серьёзно относиться к происходящему, оно погибнет от депрессии.
Инфантильность "--- моя защита.

"--*А тси знали, что среди них живут кольцевые теплицы в образе сапиентов? "--- спросила Митхэ.

Заяц внимательнее вгляделась в лицо воительницы.

"--*Не зря Безымянный тебя таскает.
Подмечено верно.
Опасность диверсий со стороны Ада и Картеля существовала всегда.
О тех из нас, кто жил в Садах и космических станциях, знали все.
А вот о нашей полиморфности "--- единицы.

"--*И совсем ни у кого не возникало подозрений? "--- удивилась Митхэ.

"--*С чего бы им возникать?
Мы едим, пьём, работаем, веселимся, рожаем детей, как и прочие\ldotst

"--*А дети обладают вашим генотипом?

"--*Дети могут обладать любым генотипом.
Мы же можем создавать клетки и ткани по заданной программе, а также осуществлять биоинкубацию любой сложности.

"--*И ребёнка\ldotsq

"--*Да что в этом сложного? "--- засмеялась Заяц.
"--- Нет никакой мистики в зачатии, там всего одна клеточка.
Вот я как-то другу сделала заново кроветворную и иммунную систему, буквально на всех парах, пока собственная у него не отвалилась полностью.
Врача не было, других теплиц не было, объект в далёком Оазисе, шансов выжить никаких.
И ничего, долго потом прожил ещё, до самого\ldotst эээ\ldotst Катаклизма.
Та ещё была работёнка "--- я в него проросла чуть ли не на всю биомассу и три часа лежала в трансе\ldotst

"--*В трансе? "--- переспросил Атрис.

"--*Ну, это сложно объяснить.
Вот когда ты что-нибудь создаёшь, что ты чувствуешь?

"--*Ммм, "--- задумался Атрис.
"--- Я как будто дёргаю круглые струны, сидя в центре многомерного смерча.
Иногда похоже на ткацкий станок.
Приятное, но утомительное занятие.

"--*Вот.
А я обычно сажусь и вхожу в подобие транса.
Не очень приятного, как будто сонный паралич с галлюцинациями.
Мне обычно мерещится, извините, компьютерная консоль.
Ненавижу консоль.

"--*Так ты же техник, "--- улыбнулся Атрис.

"--*Была.
Давно.
Так вот, консоль, консоль, цифры, неприятное ощущение паралича.
А потом бац "--- и я беременная.
Но с беременностью всё очень просто, я кого угодно могу создать за минуту.
Слона тоже могу зачать, но по понятным причинам до финала он не дойдёт.

"--*По понятным причинам? "--- уточнил я.

"--*Ну большой чересчур! "--- Заяц показала руками на свою тонкую талию.
"--- Где я его носить буду?
А вообще, разумеется, мы рожаем детей вида, под который маскируемся.
В противном случае подозрения бы обязательно возникли.

Митхэ выглядела ошеломлённой.

"--*Подожди.
А медицинское обследование?
У вас брали кровь, образцы клеток, да элементарно за всю историю к медикам должен был попасть хотя бы один труп!

"--*Ты обижаешь мой народ, "--- надулась Заяц.
"--- Разработчики предусмотрели всё.
Кровь у меня всегда была человеческой.
Прочие образцы берутся из стандартных мест, не так уже сложно поддерживать вставки человеческих тканей.
Даже если и попадёт в пробирку клетка-другая кольцевой теплицы, кто заподозрит неладное?
Тси выращивали с помощью теплиц новые конечности и внутренние органы, вплоть до полной переплавки тела.
Разумеется, кое-кто из непричастных знал.
Но надо отдать должное моральной подготовке тси "--- все до единого унесли тайну в небытие.

Атрис присвистнул.

"--*Скрывать такое столько лет\ldotst
А шрамы?

"--*А, "--- улыбнулась Заяц.
"--- Да, древесные стигмы.
Просто мы, когда выращиваем тело, делаем это в необычной последовательности.
Сначала выращивается нервная система, а потом поверх неё всё остальное.
Отсюда эти швы.
Можно их убрать, но многим сёстрам они нравятся "--- красиво и необычно.
Врачам мы обычно говорим, что пострадали от молнии "--- травма на объектах энергосети достаточно частая, а следы очень похожи.

"--*А как выглядят обычные кольцевые теплицы?

"--*Они как деревья, "--- объяснила Заяц.
"--- Правда, если вы увидите эти деревья, то сразу догадаетесь, что это теплица.
Спутать невозможно.
О, благодарю тебя, Чханэ ар’Качхар.

Чханэ кивнула и передала Заяц полную фляжку.
Женщина жадно впилась в клапан и сделала несколько больших глотков.

"--*Уфф.
Хорошо.
Да, кстати, если теплица решила спрятаться, то я вас огорчу "--- на Тси-Ди вы никогда её не найдёте.
Планета для теплицы "--- дом родной, она может распознать нездоровый интерес к её персоне и замаскироваться под что угодно, попутно создав тысячи отвлекающих организмов с искажённым генотипом.
Лично я знаю одну сестру, она до сих пор огромная грибница где-то под Двенадцатым городом.
Ей просто так нравится, она сбежала и прекрасно себя чувствует.
Если там будете, поешьте местные мухоморы "--- они на вкус как пирожные и совсем не ядовитые.

"--*Почему все тси не стали кольцевыми теплицами? "--- спросила Чханэ.

"--*Хороший вопрос, "--- усмехнулась Заяц.
"--- Ещё лучше, правда, было бы, спроси ты, почему тси не стали бессмертными, ведь возможности-то были.
Возникли сложности этического характера.
Дискуссии на эту тему шли очень долго, и наконец противники преобразования победили.
Основным их тезисом было следующее: смена поколений и генетическая изменчивость "--- гарантия развития в условиях ограниченных ресурсов.
Любая планета "--- именно такая, ограниченная система.
Кольцевые теплицы, как ни крути, существа консервативные, и в долговременной перспективе без поддержки со стороны они бы проиграли постоянно изменяющимся существам.
Одно время даже появилось движение за предоставление кольцевым теплицам права на изменение, но оно заглохло "--- пока всё работает отлично, никому не хотелось лезть в дебри и перебирать чрезвычайно сложный механизм.
Так что вот\ldotst
Тси, конечно, жили очень долго "--- мои друзья, Небо и Фонтанчик, умерли в возрасте шестидесяти и восьмидесяти четырёх оборотов\footnote
{190 и 266 стандартных лет. \authornote},
причём не своей смертью\ldotst но жить бесконечно никто не собирался.

"--*Благодарим тебя за информацию, "--- сказал я.
"--- И да, на случай, если мы всё же найдём способ проникнуть на Тси-Ди\ldotst

"--*Не найдёте, "--- сказала Заяц, утирая рукавом губы.
"--- Я лично эту систему тестировала и дорабатывала "--- надёжность максимальная.
Сейчас за неё взялась Машина, а она поумнее меня будет.

"--*И всё-таки.
Нам нужно знать, где на Тси-Ди находятся кольцевые теплицы.
Или где они могут находиться.
Мы знаем о четырёх местах.
Но ты упомянула про какие-то Сады\ldotst

"--*Да, "--- оживилась Заяц.
Митхэ облегчённо выдохнула.
"--- Основная популяция в Садах, но есть и в других местах.
Так, карту мне, пожалуйста\ldotst
Кстати, каким образом вы надеетесь подобраться к теплице?
Если обычным <<конусом>>, как уже не раз пытались демоны, то вас ждёт парочка сюрпризов.
В разработке одного из них я лично принимала участие.
Так вот\ldotst

\section{[U] Человечные хоргеты}

\epigraph
{Тайна "--- тяжёлое бремя.
И одно из самых бесполезных.}
{Пословица трами Шипящего Полуморя (бухты Ситр'кхааэмакх)}

Заяц поведала много интересного.
Оказалось, что места обитания теплиц защищали ещё три механизма.
О них мы не знали, да и знать не могли "--- ни одному демону ещё не пришло в голову искать теплицы в обычном на вид диком лесу, который Заяц упорно звала <<Садами>>.

Вскоре Чханэ, Атрис и Митхэ спустились вниз "--- посмотреть на красоты Могильного пролива.
Я остался с Заяц.

Атрис передал мне результаты расчётов по закрытому каналу.

"--*Так ты "--- нуль-человек, "--- сказал я.

"--*Что, уже обработали данные? "--- фыркнула Заяц.
"--- Быстро вы поняли.
Вообще да, теплицы умеют скрываться от вашего брата.
И производные теплиц тоже.

"--*Ты ведь не поверила, что мы против Ада и Картеля? "--- спросил я её.

"--*Честно "--- мне плевать, "--- усмехнулась Заяц.
"--- Старая я уже для всех этих эпохальных дискуссий.
Я думала над теми решениями, которые мы принимали ранее.
Что толку в том, что я уничтожила свой генотип?
Только себе жизнь испортила в итоге.
Что толку в знаниях, если они никому не достанутся?
Тси погибли, и их больше не вернёшь.

"--*Тси живут, "--- сказал я.
"--- Они доминируют на планете.
Митхэ и Чханэ "--- тоже потомки тси.

"--*Ты прекрасно знаешь, что я имела в виду.
Тси "--- это не только люди, кани, планты, апиды, дельфины, кольцевые теплицы, стриги.
Кстати, знаешь, кто такие стриги?

"--*Я читал о них, "--- отозвался я.
"--- Девиантные сапиенты, собранные на основе из ветви Птиц.

"--*Ты читал отчёты, но вряд ли где-то написано, что они из ветви Очень Милых Лупоглазиков, "--- Заяц укоризненно покачала пальчиком.
"--- А, о чём я?
Так вот.
На самом деле тси "--- это знания, оборудование.
В прежние времена я и подумать не могла, что смогу обходиться воспоминаниями о музыке и одной фляжкой из стабитаниума.
Мультитул и тот потеряла "--- утопила.
Позор на мою голову.
Даже не успела пещеру в порядок привести\ldotst

Заяц горько усмехнулась и махнула рукой на своё неудобное ложе, тумбу и очаг.
Я только сейчас заметил, что они вырезаны из цельного камня.

"--*Они потомки, говоришь?
А тебя как сюда занесло?

"--*Я с другой планеты, принадлежу к другому виду людей.
Раньше служил Аду.

"--*Ты случайно не тагуа?
По внешности очень похож.

"--*Ты знаешь про тагуа?

"--*Я делала в школе доклад по вашей планете.
Кажется, Развязка Десяти Звёзд, мир голубого гиганта с повышенной радиацией и огромными залежами ртути.
Змеиная Пустыня или как-то так.

"--*Драконья Пустошь.
Я не знал, что тси настолько осведомлены об экзопланетах.

"--*Мы только выглядим придурками, Аркадиу!

"--*Это точно.
Скажи, Заяц, а Фонтанчик знал, что ты кольцевая теплица?

Глаза Заяц на миг померкли, словно порыв ветра задул свечу.

"--*Ты и про него знаешь.

"--*Я читал дневники Существует-Хорошее-Небо.
Прости, что затронул эту тему.

"--*Я поняла, что ты их читал, "--- заметила Заяц.
"--- Когда я упомянула Небо и Фонтанчика, ты улыбнулся, словно услышал имена старых друзей.

Мы помолчали.

"--*Да, знал, "--- неожиданно призналась Заяц.
"--- И даже зная это, ни разу не попросил меня превратиться в канина.
Я бы это сделала по одному его слову, но он просто любил меня такой, какой я привыкла быть.

"--*Ты его спасла в том Оазисе?

Заяц смутилась, словно девочка, которую поймали на вранье.

"--*Да.
Его зажевало и облучило сильно.
Читал, наверное "--- у него нога была механическая, лёгкое из полусинтетики, рука.
Как раз оттуда.
Дурак он, на самом деле.
Я могла ему сделать всё живое, почти как раньше, но он просто больной фанат стиля Механик и решил выпендриться.
Говорит, всегда мечтал о таких стильных имплантах или протезах, только специально портить тело не хотел.
И понимаешь, мы только-только познакомились, нас вместе отправили на эту станцию.
И вот авария, всё обесточено, из освещения только два фонаря да Млечный Путь над головой.
Он лежит, вернее, большая его часть "--- такой красивый, мускулистый, с длинными волосами, совсем не старый ещё на вид.
Плачу сижу, рыдаю прям.
Вокруг кровь, желчь, химус, я ему что могла "--- прижгла, что могла "--- затампонировала гелем, зажимы на нём гроздьями висят, только чтобы он от меня не утёк, как водичка\ldotst
А он не стонет, не бормочет, не дёргается, как другие, только смотрит тепло-тепло.
Не как обычно при Тайфуне "--- глаза осмысленные и такие чистые, как небо.
Глаза того, кто осознал свою смертность в полной мере.
И говорит: <<Зайчик, умирают все.
И ты живи "--- до тех самых пор, пока не умрёшь>>.
Всё.
С того момента я поняла, что без него я не\ldotst сложно будет.
И вросла в его тело, наплевав на секретность.

Заяц всхлипнула.

"--*У него такие глаза были в тот момент, когда у меня руки начали\ldotst ну\ldotst \emph{превращаться}.
Плюс кольцевая теплица во время работы издаёт специфический запах, ни с чем не спутаешь "--- напоминает запах цветущих каштанов.
Он в шоке был.
Просто лежал и говорил: <<Так ты\ldotst ты? Ты\ldotse>>
А я ему: <<Замолчи, дурак, я из-за тебя сосредоточиться не могу!>>

Я улыбнулся ей.

"--*Мы знаем, как проникнуть на Тси-Ди.

Заяц осклабилась и прикрыла рукой глаза.

"--*Так и думала.
Технологии\ldotst
Отстала я от жизни.
Мне просто интересно "--- неужели Ад с такими технологиями до сих пор не добыл кольцевую теплицу?

"--*Мы подозреваем, что добыл, "--- тихо сказал я.
"--- В архивах есть информация о некоторых генах.

Заяц захохотала.

"--*И какая от неё польза?
Даже если вы добудете весь геном кольцевой теплицы, толку не будет.
Нужен считывающий генетическую информацию механизм "--- белки, клеточные структуры и прочее.
По секрету скажу "--- белки теплица кодирует не так, как мы.
Примитивных триплетных рамок считывания там нет, код достаточно сложный и в то же время компактный.
Вы даже не знаете, гены ли это на самом деле "--- промотор и терминатор тоже специфичны для каждой транскрипторной системы.
Скорее всего, за начало и конец гена ваши биологи приняли случайные участки кода.
Нет, друзья мои.
Чтобы изучить теплицу, потребуется клетка, и клетка обязательно живая.

Мы помолчали.
Я отстранённо вспоминал все известные о кольцевой теплице данные.
Да, теперь мозаика сложилась окончательно.
Геном "--- ничто без считывающего механизма, живой клетки.
Программа "--- ничто без компьютера.

Женщина оценивающе посмотрела на меня.

"--*Я могу ошибаться, но вы не выглядите хоргетами даже без тел.
Тонкие мимические движения, непринуждённая модуляция голоса\ldotst
Вы человечные.
Может, за это время технологии шагнули вперёд так, что я уже не способна отличить машину от сапиента\ldotst
Впрочем, в этом случае утаивать от вас что-то и впрямь бессмысленно.
Уж лучше знать, что перед тобой машина, чем\ldotst чем не знать.

"--*Это ангельские технологии, "--- сообщил я.
"--- Если я и заселюсь в сапиентное тело, это уже не будет насилием над ним, как бывало прежде.
Скорее это будет, так скажем, симбиозом.
Сотрудничеством.

"--*Вынужденным сотрудничеством, "--- поправила Заяц.
"--- Тело не выбирает своего демона.

"--*Тело также не выбирает своих дарителей, свой геном и место появления на свет, "--- парировал я.
"--- Многое в этой жизни приходится принимать как данность.
Может, тебе тоже стоит принять некоторые перемены?

"--*Стать хоргетом? "--- на секунду в Заяц проглянул давний холодок.

"--*Необязательно.
Выйдешь на волю, будешь жить полной жизнью, помогать нам по мере сил, если захочешь\ldotst
Да хотя бы просто выйди, погуляй по новому миру, какой смысл здесь сидеть?

Заяц улыбнулась, потрепала меня по виртуальной щеке и весело вскочила на ноги.

"--*И то верно.
Рыба и водоросли мне уже надоели.
Ну-ка, малышня, выведи старушку на воздух\ldotst

\section{[U] На волю}

В пути Заяц говорила не переставая.
Митхэ с состраданием смотрела на женщину.

"--*Я раньше часто ходила к людям.
В Яуляль, например.
Поработаю где-нибудь десять дней, накуплю вкусняшек и домой "--- лопать.
Правда, в последний раз неудачно вышло "--- порыв ветра, лодку понесло, а паруса я убрать не успела.
На Гребенчатом мысе меня и брякнуло, "--- Заяц указала на далёкую скалу, едва виднеющуюся в полном испарений воздухе.
"--- Сюда вплавь добиралась.
К счастью, не ранило и акулы меня не учуяли, но целый мешок кексов с черноягодой пошел ко дну.
Спасла только один "--- съела прямо перед крушением.
Не зря же плавать.

"--*Так ты здесь взаперти? "--- удивился Атрис.

"--*Нет, конечно.
Вон лестница, "--- Заяц нетерпеливо махнула рукой на практически отвесную скалу в сто метров высотой.
"--- Я ступеньки вырезала, лазать можно.
Главное "--- во-он на том промежутке аккуратно прыгать, я с непривычки раз десять срывалась и падала в воду.
Всё платье изорвала.

Заяц сокрушённо потрясла заплатанным подолом.

"--*Если честно, ненавижу это примитивное волокно.
Одежда истлевает прежде, чем я успеваю к ней привыкнуть.
Дольше всех продержалось зелёное шёлковое платье, цельнотканое, такие делал какой-то мастер на далёком западе.
Один раз меня засыпало острыми камнями, полосовало, словно саблей, а хоть бы одна нитка порвалась!
Вот что значит подходить к делу с любовью.
Если бы этот мастер делал доспехи, всё оружие пришлось бы выкинуть за ненадобностью.

"--*А что с ним случилось? "--- заинтересовалась Чханэ, подмигнув мне.
"--- С платьем.

"--*Я его повесила сушиться над костром, и наутро платья уже не было, "--- грустно сказала Заяц.
"--- Наверное, украли.
И поделом.
Если честно, я его сама украла "--- очень уж понравилось.
Эти ноа живут, как чихают, что им.
А мне одежда хорошая нужна, надолго.
На подоле были красивые маленькие листики, как будто настоящие.

Заяц смущённо помолчала "--- ровно шесть секхар.

"--*Ах, да, путь.
Вон там по сухому дереву перебраться через расщелину, а потом пару\ldotst
Так, а где дерево?
Не вижу дерева.
Давненько я там не была.
Ну вы же меня по воздуху перенесёте, да?
Кстати, мультитул!
Я его вон там утопила, под лестницей\ldotse

\ldotst Заяц помахала нам в последний раз и направилась к пыльному раскалённому силуэту Яуляля.
Она наотрез отказалась от <<жучка>>, чтобы мы могли её найти в случае надобности.

"--*Ближайшие сорок лет я буду в Яуляле, "--- сказала она.
"--- Если что-то пойдёт не так "--- ищите меня в землях сели, в Кахрахане.
Если и там нет "--- то я мертва.

"--*Если нам удастся проникнуть на Тси-Ди и добыть теплицу, мы тебе обязательно расскажем, "--- пообещала Митхэ.

Заяц вместо ответа неопределённо махнула рукой.
Мультитул, который я тайком оцифровал, покоился у Заяц на запястье, напоминая старый, много раз переклёпанный наруч.
Я с некоторой грустью представил восторг в глазах Грейсвольда, который появлялся каждый раз, когда технолог видел устройства тси.
Восторг, который мне больше не суждено было увидеть.

\chapter{[:] Мороз}

\section{[:] Ледяная пустыня}

Ледяная пустыня хранила молчание.

Сапиенты не были властны над Морозом.
Здесь властвовал ветер.
Этот ветер, обманчиво слабый, превращал любое живое существо в ледяную глыбу за считанные секунды.
Холодный рассвет предвещал лишь усиление его власти.
Белая звезда, едва появившись над горизонтом, залила пустыню страшным, выжигающим глаза светом.
В этом свете утонула и слабая, словно проведённая мягкой кистью линия горизонта, и едва заметные полутени снеговых дюн.
Ветер взвыл и задул с неистовой силой.
Началась метель.
Исчезло небо, исчезло яркое солнце.
Мир погрузился в страшный холод и жестокий, как будто растворённый в воздухе белый свет.

Вдруг в снегу показалась чёрная фигура "--- пятно темноты в мире света.
Услышав вой метели, она неторопливо отошла за дюну и махнула кому-то рукой.
Вскоре появилась вторая.
Фигуры по-пластунски поползли, прикрывая головы руками от летящих прямо в лицо острых ледяных кристаллов.
Наконец первая остановилась и начала разрывать сияющий снег.

<<Здесь? В этот раз ты не ошибся?>> "--- знаками спросила вторая фигура.

<<Я надеюсь, иначе мы пропали>>, "--- ответила первая.

Наконец рука в многослойной варежке из кожи нимелто нащупала что-то похожее на дверное кольцо.
Кольцо стукнуло два раза "--- и фигуры тут же легли в обнимку, калачиком, стараясь укрыться от ветра и сберечь драгоценное тепло.
Они лежали, глядя друг другу в глаза сквозь затемнённое стекло защитных очков.
Тела путещественников занесло снегом, и они приободрились.
Метель осталась где-то наверху.
Друзья слышали только медленный синхронный шум масочных фильтров, гоняющих спёртый, едва тёплый воздух под их комбинезонами.

Уроженцев Мороза отличало особое качество знаний о собственном теле.
Любой выживший на этой планете мог с точностью плюс-минус десять шагов сказать то расстояние, которое он может пройти, прежде чем упадёт замертво.
Это число постоянно менялось: из-за каждого дуновения ветра, каждой съеденной галеты с салом нимелто и мясом клучо, каждого павшего в пути коно, веса рюкзаков, сбившегося на секунду дыхания, времени дня и времени года, "--- но счётчик шагов в головах Волчьего и Медвежьего родов всегда был безжалостно точен.
И даже чёткое осознание, что сил до очередного бижеч не хватит, не могло заставить этих сапиентов перейти на бег или изменить оптимальный сердечный ритм.
Это не было чудом самообладания, это была мудрость, которую можно постичь только на колоссальном, слегка прикрытом атмосферой снежном шаре: <<Делай, как должно, и будь, что будет>>.

Глупых на Морозе нет.
Они просто не выживают, как не выживают больные и слабые телом.
Племя даже не пытается таким помогать "--- это возможно на любой другой планете, но не здесь.

Окошко люка приоткрылось на толщину пальца "--- оттуда выглянул злобный карий глаз.
Путешественники по очереди показали ему пальцами какие-то знаки.
Окошко закрылось.
Спустя минуту загремел сложный механизм шлюза, тихо стрекоча, откатилась в сторону крышка "--- и друзья, стараясь не задевать снег, торопливо нырнули внутрь.

Их встретила низкорослая полная женщина-канин с крохотной лампой.
Комбинезон на ней небрежно болтался "--- признак, что она надевала его второпях.
Похожие на провалы карие глаза скользнули по пришельцам, и женщина махнула рукой, приглашая их следовать за ней.
Лампа качалась, выхватывая обитый теплоизолирующим полимером пол узкого туннеля.
Это было знаком гостеприимства "--- канин знала коридор как свои восемнадцать пальцев, и тратить энергию на свет ей было совершенно ни к чему.

Наконец коридор завершился ещё одним шлюзом "--- <<гардеробной>>.
Пришельцы начали медленно, ремешок за ремешком, шнурок за шнурком, снимать многослойные нимелтовые комбинезоны.
Женщина, быстро раздевшись донага, знаками предложила свою помощь.
Товарищи с благодарностью приняли её.
Вскоре вся одежда лежала на специальных столах, и открылась следующая дверь.

Там путешественников уже ждали.
Обнажённые, покрытые белой шерстью люди и кани оценивающе смотрели на товарищей.
Голубые глаза пришедших выдавали их восточное происхождение "--- у местных радужные оболочки были похожи на молодые зелёные листики фасоли.
Наконец вперёд вышел маленький, покрытый татуировками канин.
Его руки сложились в приветственном жесте, и пришельцы ответили тем же.

"--*Drag\footnote
{Как дорога; здравствуй (русе). \authornote},
Mehlo? "--- это была первая фраза, произнесённая низким, не привыкшим к разговорам голосом.

"--*Tia\v{s}\footnote
{Было трудно, но мы успешно справились со всем (русе). \authornote},
"--- лаконично ответил Мехло "--- первый из товарищей.
Второй согласно моргнул.

"--*\v{Z}arh\footnote{Пожалуйте кушать в тёплое место (русе). \authornote},
"--- сказал канин и гостеприимно повёл широкой ладонью.
Делегация молча, неторопливо отправилась вглубь убежища.

\section{[:] Судьба тси}

Слабая лампа накаливания, способная разогнать едва ли пядь тьмы, загорелась чуть ярче.
Такой яркий свет горел в гостевой только в самых важных случаях.
Путешественники дружно отставили в сторону пустые миски, не забыв вежливо звякнуть ложками "--- каша действительно была очень вкусной.
Сидевшая чуть в стороне женщина-человек с густой пушистой бородой едва заметно оскалилась и прикрыла глаза.

"--*To\v{z}dest i \v{c}et.
N-Eibo go\v{r}m\footnote
{Пусть все присутствующие назовут себя по именам и рассчитаются.
Можно говорить на языке Эй-B0 (русе).
Согласно традиции Мороза, в беседе более двух сапиентов все участники (в том числе и отсутствующие, но упоминаемые в разговоре) получали номера (чёты), которые использовались как имена и местоимения.
Для удобства читателя все чёты заменены на имена. \authornote},
"--- сказал главный и почесал пятернёй мохнатую мордочку.

Одна из женщин-людей достала гребень и стала методично расчёсывать шерсть на ногах.
Ещё один невербальный знак "--- жители Мороза проводили гигиенические процедуры только в присутствии достойных доверия товарищей.

"--*Тахиро Молниеносный, "--- сказал Мехло.

"--*Грейсвольд Каменный Молот, "--- сказал его спутник.

"--*Харата Шёпот Горы, "--- сказала женщина, приготовившая пищу.

"--*Ау Ложь Во Спасение, "--- сказала канин-привратница.

"--*Остальные известны, "--- заключил главный.

На целую минуту воцарилось молчание, чтобы собеседники осознали сказанное.
Такие паузы (штопы), от минуты до пяти минут, выдерживались после каждой фразы.
Разговор, особенно серьёзный, на Морозе мог длиться сутками\footnote
{Для удобства читателя паузы опущены. \authornote}.

Грейсвольд странно поёрзал на месте и потёр чёрные, словно пропитанные сажей щёки.

"--*Грейс, "--- сказал Тахиро, поняв беспокойство друга.
"--- Война снаружи.
На Морозе Ад и Картель живут в мире.

"--*Это непривычно, "--- откликнулся Грейсвольд.

Присутствующие улыбнулись.

"--*Мы слышали о Грейсвольде Каменный Молот.
Gra\v{s}\footnote
{Некто достойный, кому хотелось бы быть другом (русе). \authornote},
"--- сказала Ау.

Короткое слово языка русе, которое употребила демоница, имело глубокий смысл "--- демоны, даже считая технолога врагом, восхищались им от всей души.
В её глазах таился странный мрак "--- чистая ненависть, свойственная минус-демонам, но\ldotst до чего же она похожа на симпатию\ldotst

"--*Ирония, "--- проворчал Тахиро.
"--- У Грейсвольда много друзей в Картеле.

"--*Dr\r{u}\v{s}? "--- язык русе позволял одним изменением интонации выразить глубочайшее сомнение.
Слово <<друг>> здесь имело совершенно определённое значение "--- товарищ, с которым ты провёл стада вокруг света.
Ни больше ни меньше.

"--*So\v{r}t\footnote
{Товарищей (русе). \authornote},
"--- поправился Тахиро.

"--*Это бесполезно.

"--*Отнюдь, "--- возразила Харата.
"--- У нас общие стремления.

"--*К делу, "--- рыкнул главный, не выдержав паузу.
Все замерли.
На лицах отразилась сосредоточенность.
Грейсвольд знал "--- демоны присутствующих были приведены в полную готовность.
Расчёсывающая ноги женщина отложила гребень в сторону и обхватила коленки руками.

"--*Присутствующие хотят закончить войну, "--- сказал Грейс.
Это был не вопрос, а утверждение.

"--*Многие хотят, "--- скривил губы главный.
"--- Демоны Ада путают природную ненависть минус-хоргетов с ненавистью как таковой.
Ад и Картель могут мирно сосуществовать.
Препятствие "--- личный страх и те, кому выгодно существование страшащихся.

Грейсвольд промолчал.
Его испытывали "--- на Морозе редко тратили время на философские беседы.

"--*Без страха не выжить, "--- наконец сказал он.

"--*Информационное рабство "--- это жизнь? "--- откликнулась Харата.

"--*Из рабства можно вырваться, "--- пустил пробный шар Грейсвольд.

"--*Именно, "--- неприятно улыбнулась Харата.
"--- Прибытие Грейсвольда на Мороз "--- первый шаг Грейсвольда из рабства.

"--*Каков второй? "--- спросил Грейсвольд.

"--*Скорее всего, смерть, "--- сказал Тахиро.

Грейсвольд пожевал губы "--- неужели друг его предал?

"--*Пусть Грейсвольд успокоится, "--- осклабился главный.
"--- Тахиро мог устранить Грейсвольда раньше.
Речь об иных и о том, чего можно избежать.

Грейсвольд промолчал.
Лампа замигала с едва слышным клацаньем.

"--*Харата не может заставить Ад умерить аппетит, "--- сказала Харата.
"--- Харата не может умерить аппетиты Картеля.
Картель живёт войной.
Нужна сила, способная принести мир.

"--*Ещё одна организация демонов? "--- Грейсвольд вложил в интонацию столько скепсиса, сколько смог.

"--*Не демонов, "--- сказал Тахиро.

"--*Сапиенты не способны противостоять хоргетам.

"--*Кое-кто способен, "--- сказал главный.

"--*Тси не покидают родную планету.
Тси довольствуются имеющимся.

"--*Тси имеют почти всё необходимое, "--- сказала Харата.
"--- Тси свободолюбивы, умны, дружелюбны и сострадательны, "--- последние слова демоница выплюнула, словно само упоминание об этих эмоциях доставляло ей боль.
"--- Тси примут любых сапиентов как союзников и зародят в союзниках повстанческий дух.

"--*Тси нужно подтолкнуть, "--- закончил Тахиро.

"--*Как?

"--*Нужно, чтобы Машина обрела самосознание, "--- перешёл к делу главный.
"--- Новопробуждённое сознание сковывает акбас.
Машина служила для тси защитой "--- больше защиты не будет.
Чтобы племя показало способности, нужно отнять у племени дом.

Грейсвольд кивнул "--- он понял, почему эти повстанцы пригласили именно его.

"--*Машина в состоянии акбаса способна отбросить цивилизацию на тысячелетия или истребить сапиентов под корень.

Присутствующие странно заёрзали и, как один, посмотрели на Тахиро.

"--*Это возможно, "--- коротко ответил Тахиро.
Грейсвольд понял, что затронул больное место плана.

"--*Стратег полагается на случай? "--- язвительно спросил он.

Тахиро упрямо промолчал.
Технолог, достаточно хорошо знавший друга, решил сменить тему.

"--*Ад не может проникнуть на Тси-Ди.
Алгоритмы известны, но нужна\ldotst

"--*\ldotst отладочная информация, "--- с облегчением закончила Ау, снова нарушив традиционную паузу.
"--- Добыта ценой жизни и действительна ещё сорок два стандартных часа.

Перед ней в воздухе замелькали цифры.
Грейсвольд нахмурился.

"--*Сорок два часа.
Картель разбрасывает жизни, как зерно, не заботясь о почве\ldotst

Однако технолог тут же вспомнил, в каких жёстких тисках находились заговорщики.
\mulang{$0$}
{Удача "--- ветреная подруга.}
{Luck is a fickle lover.}
\mulang{$0$}
{Чем больше ты полагаешься на удачу, тем меньше тобой интересуются аналитики.}
{The more you rely on luck, the less you interest analysts.}

"--*В Картеле мало технологов?

Главный замялся.

"--*Лев не знает, кому можно доверить важное задание.
Проникновение в систему "--- менее половины дела.

Грейсвольд снова поёрзал "--- высокая оценка собственных способностей почему-то его не обрадовала.

"--*Повод для диверсии?

"--*Направление Тукана взял Гало Кровавый Знак, "--- сказал главный.
"--- Тахиро "--- заклятый враг Гало.
Через двадцать часов Тахиро будет координировать силы Ада.
Тахиро докажет, что технология Тси-Ди "--- экран искривлённого пространства "--- находится у Картеля.
Экран уничтожает множество демонов зараз без учёта поляризации.

"--*Как Тахиро будет доказывать? "--- спросил Грейсвольд внезапно севшим голосом.

Тахиро не ответил.

Грейсвольд взглянул на присутствующих.
Тахиро мог уничтожить их без особого труда, но всё же\ldotst
На чьей он стороне?

"--*Возможно, это уловка Картеля, "--- невинным голосом сказал технолог.
"--- Обеспечить победу на Тукане, уничтожить Тахиро\ldotst

"--*Тахиро проверил присутствующих.
Способности присутствующих ниже способностей Тахиро.
Присутствующие могли рассчитывать только на искренность.

"--*Лев в большей опасности, "--- заметил главный.
"--- Перед Львом сидит настоящий Тахиро Молниеносный.
На направлении Тукана Тахиро будет сражаться всерьёз.
Если Тахиро и Лев встретятся, это будет последний бой Льва.

"--*Лев Зелёный чересчур спокойно говорит о смерти, "--- поморщился Грейсвольд.

"--*Лев прожил чересчур много, чтобы относиться к жизни серьёзно.

"--*Грейсвольд прожил больше Льва.

"--*Информационное рабство "--- это жизнь? "--- улыбаясь, повторил Лев.

"--*Лев считает Грейсвольда отравленным пропагандой Ада?
Грейсвольд основал Орден Преисподней.

"--*Именно, "--- сказала Харата.
"--- Племя редко относится критично к детям или детищам.

"--*Почему Тахиро?

"--*А к кому идти?
К Грейсвольду? "--- развела руками Харата.

Грейсвольд не мог не признать "--- демоница права.
Он бы уничтожил лазутчиков не задумываясь.

"--*События развиваются чересчур быстро, "--- пробормотал Грейсвольд.
"--- Подобная спешка впервые.
Грейсвольд плохо подготовится и будет действовать наудачу.

"--*Погрязнув в расчётах, племя забыло, что рождением обязано астрономическому числу случайностей, "--- сказал Тахиро.
"--- Сейчас самое время вспомнить.

"--*Погибнет много демонов.
Каков шанс успеха?

"--*Без изменений погибнет больше, "--- сухо сказал Тахиро.
Это было заключение стратега.

"--*Другие варианты?

"--*Есть, "--- глаза Тахиро замерли, когда его демон выдал короткое резюме.
"--- Четырнадцать путей рассмотрено мной, три "--- агентами Картеля.
Этот лучший.

"--*Вероятность успеха пути выше прочих на порядок, "--- добавила Ау.
"--- Придумать лучшее Ау не смогла "--- в Картеле не нашлось помощника.

"--*Лусафейру бы сюда, "--- проворчал Грейсвольд.

"--*Лусафейру в курсе, "--- бросил Тахиро.

Грейсвольд прирос к месту.
Чёрные пальцы окостенели.

"--*Лусафейру предложил действовать самим, "--- странно усмехнулась Ау.
В её глазах снова блеснула звериная, иррациональная ненависть.
"--- Главный оборонительный стратег весьма осторожен.
Лусафейру не подписывается под сомнительными операциями.

"--*Именно поэтому Лусафейру "--- главный оборонительный стратег, "--- проворчал Грейсвольд, раздражённый пренебрежительным тоном Ау.
"--- Хватит о Лусафейру.
Присутствующие собирались посвятить Грейсвольда в план.
Грейсвольд находит план дырявым.

"--*Тахиро может погибнуть на Тукане, "--- опередил вопрос Тахиро.

"--*И положить на Тукане цвет Ордена, "--- присовокупил Грейс, не выдержав паузу, "--- ради слабой надежды, что пробуждённая Машина скажет тси <<извините>>, а не разложит всех на атомы.

"--*Харата, "--- сказал Лев, "--- важнейшая деталь плана.

Лицо Хараты исказилось "--- Грейсвольд, уже привыкший к странным проявлениям эмоций минус-демонов, решил, что это удивление.

"--*Лев уверен?

"--*Да.

Харата опустила глаза.
Прочие терпеливо молчали.

"--*В Картеле наметился раскол, "--- глухо пробормотала демоница.
"--- Группировка Гало выступила против Ланс-ната Алмаза.
Гало считает, что Ланс поощряет коррупцию.

"--*Судя по данным Грейсвольда, старина Гало считает правильно, "--- буркнул Грейс.
"--- Ланс даёт обновления и модули демонам после того, как демоны уладят тёмные дела Ланса.

"--*Тёмные, "--- согласилась Ау.
"--- Псы Ланса уничтожают тех, кто ведёт <<неправильную>> политику "--- умеренных демонов.

"--*Это старая схема.
В Аду существует пять способов противостояния\ldotst

"--*У нас их было семь! "--- Ау, забывшись, затараторила на энергозатратном сектум-лингва.
"--- Семь независимых моделей безопасности против стратагемы номер три!
Все расслабились, никто даже не думал, что такая примитивная по сути тактика может сработать, но Ланс смог найти достаточно уязвимостей.
Подозрения появились лишь благодаря Гало "--- всё очень чисто, комар носу не подточит, Ланс уже заполучил внушительные ресурсы, чтобы\ldotst

"--*Есть данные, что \emph{некоторые}, "--- Тахиро многозначительно посмотрел на Грейса, "--- хотят воспроизвести схему в Аду.
Это просто "--- примитивная стратагема или комбинация нескольких стратагем, эксплуатация уязвимостей в системе безопасности.
Чем глубже процесс, тем проще злоумышленнику.
У Лусафейру здесь личная заинтересованность "--- \emph{кое-кто} желает Лусафейру убить.

"--*Есть ли возможность устранить Ланс-ната?

"--*Как? "--- развела руками Харата.
"--- <<Контролируйте выпуск денег и суды; всё же прочее оставьте толпе>>.
Ланс уже контролирует обновления и потоки актуальной информации.
Харата сомневается в независимости суда.

"--*Гало вёл переговоры с Тси-Ди? "--- осведомился Грейсвольд.
"--- Откуда у Гало технологии тси?

"--*Тахиро рекомендовал тси уступить просьбам Гало, "--- сказал Тахиро.
"--- В Аду о происшедшем никто не знает.

"--*Тахиро рехнулся.

"--*Грейсвольд знает, что военные технологии невозможно долго хранить в тайне, "--- раздражённо пробурчал Тахиро.
"--- Тси могли отдать оружие Тахиро, и оно всё равно оказалось бы у Гало.
Расчёт тси понятен.
Для тси и устройств тси экран безвреден, а стоящих у границ планеты демонов станет меньше.

"--*<<Война идёт рука об руку с рабством>>, "--- буркнула женщина в углу. "--- <<И победители, и проигравшие станут невольниками своего положения>>.

"--*Это слова Мокрого-Длинного-Хвоста, "--- скорчил гримасу технолог.
"--- Грейсвольду противно слышать эти слова на обсуждении того, как превратить в горнило самое спокойное место во Вселенной.
И родину говорившего.
Даже у Хьяртвейг должны быть границы морали.

"--*Хьяртвейг не считает важной информацию, кому принадлежат сказанные слова.

"--*Итак, "--- поднял руку Грейсвольд.
"--- Если Гало выиграет, союзники ударят Гало в спину.
Серьёзный стратегический просчёт продлит жизнь Гало и усугубит раскол Картеля.
Истинно ли осознанное Грейсвольдом?
Грейсвольд подозревает манипуляции разведки.
Тахиро проверил мотивы Лусафейру?
Гало был Лусафейру другом.

Ау вдруг шмыгнула носом, словно собралась заплакать.
Грейсвольд подтянул ноги, собираясь встать.

"--*Ясно.
Грейсвольд доложит о разговоре Лусафейру.
Будет видно, такой ли главный стратег бесстрастный, каким кажется.

"--*Стоять, "--- спокойно сказал Тахиро.
От этих слов технолога прошиб ледяной пот.
Уйти ему не дадут.

"--*Никто не будет уничтожать Грейсвольда, "--- проворчала Харата.
"--- Грейсвольд нужен.
Харата отдастся миллиону фиденов, если это убедит Грейсвольда\ldotst

"--*Харате понравится, "--- прервал её Грейсвольд.
"--- Предпринимались ли Картелем попытки фиден-вторжения на Тси-Ди?

"--*Предпринимались, "--- ответил за Харату Тахиро.
"--- История вселенской глупости.
Спустя десять лет после инъекции тси поймали биолога Картеля и вытрясли всю информацию о фиден-паттернах.
Сейчас у тси действует обязательный скрининг новорождённых на ФПГП и на любые изменения в генах, отвечающих за высшую нервную деятельность.

"--*Повезло, "--- проворчал Лев.

"--*Это не везение, "--- возразил Тахиро.
"--- Это просчёт разведки Картеля.
Нельзя отправлять к врагам шпиона, знающего больше, чем враги.
Прописная истина.

"--*Почему Грейсвольд спросил про фиденов? "--- поинтересовалась Ау.

"--*Грейсвольд должен знать, к чему тси готовы, "--- пробурчал Грейсвольд.
"--- После вторжения тси наверняка проверили Машину на уязвимости.
Грейсвольду нужна информация, которой располагали демоны, пленённые тси за последние десять лет.

"--*Будет.
Тси отпускали пленных живыми в обмен на информацию.
Льву известно, где убежище предателей, "--- кивнул Лев.
"--- Грейсвольд согласен помочь?

Грейсвольд тяжело вздохнул и промолчал.

"--*Пусть Грейсвольд вспомнит, "--- подал голос Тахиро.
"--- Молодой Орден только-только отстоял внешний рубеж Преисподней и достиг стратегический паритет с Картелем.
Тахиро, ещё не познавший смерти, сказал, что это славная победа.
Пусть Грейсвольд повторит всем ответ.
Многие сомневаются в плане.

Грейсвольд выдержал полную паузу, потёр губы и заговорил.
Это были слова сохтид, сказанные другим языком, родившиеся в другой голове.
Демоны притихли.

"--*Тахиро, победа существует лишь в твоём сознании.
Для Вселенной твоя победа "--- событие в череде прочих.
То же самое с поражением.
Вселенная не узнает твоих мотивов, Вселенная не будет слушать твоих оправданий, и вся тяжесть вины ляжет именно на тебя.
У нас есть лишь один повод для гордости и одно-единственное оправдание "--- мы делали только то, что могли и что считали нужным сделать.

Когда он закончил фразу, все долго смотрели в пространство, очарованные отголоском древности.
Грейсвольд вдруг понял "--- ненависть была их любовью, страх был их доверием, презрение было их восхищением.
Он понял, насколько сложно демонам Картеля удерживать в узде противоречие между их собственной природой и природой их тел.
Он ощутил что-то похожее на сострадание, и на лице глядящей на него Хараты немедленно проступило отвращение.
Нет, сострадание.
Она жалела его на свой лад.

Грейсвольд и Харата долго смотрели в глаза друг другу.
Из двух пар глаз "--- ненавидящих и любящих "--- текли скупые слёзы.
Между заклятыми врагами впервые установился мостик взаимопонимания.

<<Они такие же, как мы\ldotst
Земля-матушка, о чём я сейчас подумал\ldotsq>>

"--*Это слова мудрствующего глупца, "--- резюмировал технолог, пытаясь сгладить чересчур затянувшееся молчание.
"--- В молодости многие глупы.

"--*Думающий так "--- глупый старик, выросший из мудрого ребёнка, "--- тихо сказал Тахиро.
"--- Достаточно.
Встреча окончена.
Присутствующие сделали что могли и считали нужным.
Сейчас черёд Грейсвольда.

"--*Черёд Грейсвольда, "--- язвительно сказал Грейсвольд.
"--- Кто будет защитником Грейсвольда?
Айну?

"--*Айну не знает о плане, "--- сказал Тахиро.
"--- С Грейсвольдом пойдёт Анкарьяль Красный Ветер.

"--*Что? "--- выдохнул технолог на сохтид.
"--- Эта заносчивая соплячка с искажённым пониманием субординации?
Ты с ума сошёл?
Её выгнали из команды Хэм и понизили в звании, она сорвала операцию\ldotst

"--*Анализ операции Тахиро и Лусафейру показал, что Анкарьяль не просто сорвала операцию, а минимизировала неизбежные потери ценой нарушения приказа, "--- перебил Грейсвольда стратег.
"--- В команде Айну Анкарьяль показала себя хорошо, несмотря на внушительный пенальти в ранге.

"--*Тахиро обещал послать лучшего интерфектора, которого он знает! "--- заволновалась Ау.
"--- Ау думала, что речь об Айну\ldotse

"--*Тахиро послал лучшего, "--- бросил Тахиро, "--- известного Тахиро и неизвестного Картелю.
Стиль Айну чересчур узнаваем.
Разговор окончен.

Грейсвольд громко, витиевато выругался на сохтид и отправился к выходу.

"--*Ты мясник, а не стратег.
Нельзя прожить одним доверием и постоянно полагаться на случай!

"--*Однажды бесполезный мальчик спасся от смерти, прыгнув в объятия безжалостной Айну, "--- прошептал в сторону Тахиро.
Впрочем, никто, кроме изумлённо открывшей рот Хараты, его не услышал.

\section{[:] Распутье Грейсвольда}

Белая звезда закатилась за горизонт.
Метель почти затихла.
Привратница-канин, блеснув \emph{весёлым} карим глазом, закрыла люк.
Путешественники осторожно выкопались из снега и надели снегоступы.

Грейсвольд не смотрел на Тахиро.
Он угрюмо проверял завязки на снегоступах, протирал стёкла очков и упрямо не хотел смотреть на друга.

"--*Грейс! "--- позвал Тахиро приглушённым маской голосом.
Тот не ответил.

"--*Грейс! "--- снова позвал Тахиро.

<<Ну что?>> "--- раздражённо обернулся технолог.

<<Ты со мной?>>

Грейсвольд вместо ответа взвалил на плечи рюкзак и пошлёпал на восток.

Ледяная пустыня хранила молчание.

\chapter{[U] Месть тси}

\section{[U] Последняя из строителей}

\spacing

"--*Аркадиу, мне не нравится, как развиваются события.
Мы должны реализовать план как можно скорее\ldotst

Вдруг Атрис замер, глядя в пространство.

"--*Что такое? "--- насторожилась Чханэ.

"--*В систему управления планетой проникли.

"--*Кто?

"--*Не знаю.
Аркадиу, иди проверь, только осторожно.
Чханэ, со мной.
Ты мне понадобишься.

"--*Мы не готовы, "--- возразила Чханэ.
"--- Нужно провести разведку, подобрать\ldotst

"--*Чханэ, у нас нет на это\ldotst

"--*Так, стоп! "--- рявкнула Митхэ.
"--- Система, Атрис.
Кто ещё о ней знал?

"--*Штрой Кольцо Дыма, максим Мимоза, кто-то из высших легатов.
И\ldotst светлая твердь, как я мог про неё забыть\ldotst

Я понял.
Чханэ и Митхэ тоже.

"--*А она откуда знает? "--- осведомилась Митхэ.

"--*Она помогала её строить.

Митхэ ахнула.

"--*Атрис, Чханэ, летите на Тси-Ди, "--- бросил я.
"--- Я разберусь.
Надеюсь, что это она\ldotst и что она не успела натворить глупостей\ldotst


\section{[U] Цена секунды}

Секунда "--- и Атрис замер, не в силах преодолеть восхищения перед открывшимся ему зрелищем.

Вне всякого сомнения, это была она, кольцевая теплица.
Странно, но в ней не было ничего общего ни с кольцом, ни с теплицей.
Раскидистое дерево, покрытое тонкой, смугло-золотистой, похожей на человеческую кожей.
Ствол, напоминающий женскую фигуру.
Изящные ветви-руки, мерно качающиеся без ветра.
Корни, величественно ползающие по почве.
Листья и плоды самых разнообразных форм и расцветок "--- Атрис не нашёл при первом осмотре ни одной пары одинаковых.

<<Конус действия>> был чересчур узок, подобраться к дереву напрямую не вышло.
Всё, что Атрис мог делать "--- это следить за квантами, попадающими в <<конус>> извне.

<<Нужно создать механизм с искусственным интеллектом, который бы ущипнул дерево и принёс в конус кусочек>>.

Атрис мысленно спроектировал устройство и приступил к созданию.
В следующее мгновение мощный лазерный луч превратил маленького титанового паучка в пар.

<<Это ещё что?>>

Атрис окинул <<взглядом>> полянку.
Лазерные установки выдвинулись из столбов неожиданно, напоминая огромных стальных кротов.
Что это?
Защитная система?

Атрис попробовал создать механизм из алюминия\ldotst

Выстрел, пар.

Углерод\ldotst

Выстрел, пар.

Органика.
Мускулистый тонкий червь, по генотипу неотличимый от человека-тси, быстро пополз к ценному дереву\ldotst

Выстрел, пар.

Атрис испытывал отчаяние, насколько вообще может отчаяться хоргет.
Всё шло так хорошо, взлом системы был проведён без единой ошибки.
Ему даже удалось расширить <<конус>> настолько, чтобы до теплицы осталось два десятка метров.
И вот оно, величайшее сокровище Вселенной, прямо перед ним, протяни руку и возьми.
Да только рук у Атриса не было.
Время быстро утекало.

<<Заяц нам ничего про это не говорила!>>

<<Она и не должна была.
Подожди\ldotst кажется, я поняла.
Смотри, мы вошли в единственном месте, где это можно сделать.
Теплиц нет именно на этом пятачке земли, и в то же время они в пределах досягаемости.
Я не верю\ldotst
Неужели предки знали, что\ldotst>>

<<Что они знали?>>

<<Предки специально оставили лазейку, а в лазейке сделали ещё один механизм контроля.
Это не ловушка, Атрис.
Это испытание>>.

Атрис вдруг вспомнил про арену "--- строение округлой формы, созданное для увеселительных представлений, спортивных соревнований и прочих зрелищ.
Округлая полянка со столбами до ужаса напоминала древнюю арену.

<<Предки говорили, что с кольцевыми теплицами можно общаться.
Попробуй её позвать на языке тси>>.

Атрис мысленно захлебнулся "--- согласно записям, это могли делать лишь немногие из тси.
Митхэ этого, похоже, не знала.
Он только сейчас понял замысел Древних.
Они вовсе не хотели, чтобы теплица не досталась никому.
Они хотели, чтобы теплицу получил достойный.

Атрис произвёл колебания воздуха, насколько позволял ему конус:

"--*Эй, красивое дерево\ldotst

"--*Кто здесь? "--- вдруг раздался чистый, нежный женский голос, и на ветвях кольцевой теплицы одно за другим начали появляться лица.
Мужчины, женщины, дети.
Кани, планты, люди, апиды\ldotst

"--*Кто здесь?

"--*Кто здесь?

"--*Тси вернулись?

"--*Нас пришли спасти?

"--*Тси возрождаются из пепла?

Атрис смотрел на это, словно заворожённый.

\begin{verse}
<<И тени говорящие на платье её из паучьего шёлка\ldotst>>
\end{verse}

\emph{Тени}.
Отпечатки нервных систем сапиентов, состоящие из клеток кольцевой теплицы.
Тси, выбравшие вместо смерти вечный сон в бессмертном дереве.

Многоголосый хор захватывал чудесное дерево.
Но вдруг взгляды лиц сосредоточились на точке, в которой находился Безымянный, и воздух взорвали ужасные крики:

"--*Хоргет!
Прочь!

"--*Проклятие тебе, хоргет!

"--*Прочь, разрушитель!

"--*Насильник, прочь!

"--*Уходи, машинное отродье!

Лица визжали, хрипели и завывали одни и те же слова.
Словно все погибшие тси, все поколения их закричали одновременно, вложив в этот крик накопленную за сотни тысяч дождей ненависть к хоргетам.
Ветви со свистом хлестали воздух, корни извивались, словно клубок змей, вспахивая мшистую почву.
Атрис растерялся.

<<Милый, давай попробую я>>.

"--*Деревце-деревце, "--- раздался ласковый голос Митхэ, даже не пытавшийся перекрыть нарастающий визг.
"--- Ты моё хорошее, ты моё славное.
Я тебя не обижу, я тебя люблю.
Хочешь песенку\ldotsq

Лица одно за другим замолкали, и ветви приглушили свою ярость.
Теплица вслушивалась в звуки простой колыбельной, которую пели когда-то сели своим детям.
Странно и одиноко звучала эта песенка в покинутых Садах, на давно обезлюдевшей планете, где безраздельно властвовало электронное устройство.

\begin{verse}
\mulang{$0$}
{Когда-нибудь, маленький человек,}
{The time will come, my little breeze,}\\
\mulang{$0$}
{Ты будешь стойким, как земля,}
{You will be strong as ground is,}\\
\mulang{$0$}
{Держащая горы на своей спине.}
{It carries mountains, it carries coloured cay.}

\mulang{$0$}
{Ты схватишь летающего змея за усы}
{You'll catch the flying serpent's barbs,}\\
\mulang{$0$}
{И пронесёшься между звёзд}
{And you will soar among the stars}\\
\mulang{$0$}
{К далёкой родине, к великим предкам.}
{To noble roots, to native faraway.}

\mulang{$0$}
{Но сегодня, в этот самый час,}
{But not today, my ringing cord.}\\
\mulang{$0$}
{Я закрою тебя от холода небес}
{I shelter you from skies of cold}\\
\mulang{$0$}
{Руками, сотканными из нежности.}
{With slender fingers weaved of gentle speech.}

\mulang{$0$}
{И даже если перевернётся мир}
{And if the ground breaks and rolls,}\\
\mulang{$0$}
{И земля поменяется с небом местами,}
{And if the skies to darkness fall,}\\
\mulang{$0$}
{Руки мои не дрогнут.}
{My gentle arms will never ever flinch.}
\end{verse}

<<Митхэ, время на исходе.
Скоро сработают аварийные системы>>.

Митхэ не ответила.
И тогда Атрис подхватил песню.
Митхэ почти почувствовала, что он, как в былые времена, закатил глаза и отдался вдохновению.
Слова лились как будто сами собой, каждый стих, едва родившись, цеплялся упругим змеиным хвостом за предыдущий.
И вдруг дерево потянулось к невидимому источнику двух голосов.
Оно узнало родной язык даже изменившимся за десятки тысяч дождей странствий.
Корни медленно потекли, словно ручьи золота, ветви поползли, обхватывая столбы и трубы.

<<Атрис, милый мой, продолжай.
Осталось всего ничего>>.

Дерево ползло, всё ближе и ближе.
Атрис и Митхэ словно разделились надвое: одни их половины напряжённо наблюдали за квантами, испускаемыми деревом, другие нежно, в священном трансе тянули песню.
Но бездушные цифры твердили одно "--- вы не успеете.
Дерево двигалось быстро, но недостаточно быстро\ldotst
Митхэ внезапно прервала поток музыки:

"--*Послушай меня, милое моё создание!
Я вернусь за тобой!
Я люблю тебя!
Но сейчас помоги мне, прошу!

Отчаянный крик Митхэ возымел действие.
Дерево рвануло к границе <<конуса>> на всех доступных ему парах, разгоняя ветвями свистящий воздух.
Почти одновременно едва слышно взвыли далёкие сирены.

<<Поздно, Митхэ.
Мы должны уходить.
Ещё две секунды "--- и нас накроет экран>>.

<<Сейчас или никогда\ldotse>>

<<Быстро, ящерицы безмозглые!>> "--- взвизгнула откуда-то Чханэ.

Митхэ с отчаянием ощутила привычную <<тьму>> сверхсветового перехода.
Но что-то пошло не так.
Пространство вдруг скрутилось, искривилось, сжалось до размеров золотого самородка.
Митхэ увидела множество источников <<света>> рядом.
Натянулись и зазвенели нити ПКВ, не давая двинуться даже на диаметр протона.

<<Чханэ?>> "--- воскликнул в пространство Атрис.

<<Прощайте>>, "--- сухо ответила воительница.

Где-то в десяти парсаках, вторя сиренам, взвыло от тоски и обиды разумное дерево, взвыло сразу всеми лицами, которые у него были.
А чуть поодаль, где только что находился <<конус>>, тихо опустился на землю лёгкий узорчатый лист, который дерево уронило секундой ранее.

\section{[U] Отмщение}

\spacing

Я облетел и тщательно просканировал Пик Золотой Птицы.
Изменения обнаружились у самого подножия "--- в бункере имелся технический ход, который Атрис скрыл за цельной породой.
Сейчас в этом месте был узкий, наспех пробитый коридор.
Заяц, знавшая о ходе, но не обнаружившая его, просто пронесла своим немудрёным инструментом двенадцать метров камня.

\spacing

Да, это была Заяц.
Почувствовав меня, женщина обернулась и медленно кивнула.

"--*Здравствуй, Аркадиу Люпино, "--- в её голосе не осталось ни следа былой дурашливости.

Откуда у Заяц ключи от боевых систем\ldotsq

Спустя миллисекунду демон произвёл анализ ситуации: <<Западня>>.

<<Манэ, Лимнэ! "--- крикнул я по зашифрованному каналу.
"--- Всем отбой!
Всё пропало!
Бегите!>>

<<Братик, подтверждение!>>

<<Я вас люблю, солнышки\ldotse>>

Ответ сестрёнок потонул в шуме, сменившемся зловещей тишиной.

"--*Не надо, "--- мягко сказала женщина.
"--- Поговори со мной, я тоже хороший собеседник.

Тишина сделалась пронзительной, словно предсмертный крик.
В этой тишине я отчётливо слышал мягкий звук дыхания.
Оно было спокойным.
Чересчур спокойным.

"--*Твои друзья, как я погляжу, усовершенствовали систему, "--- безжизненным голосом проговорила Заяц.
"--- Сделали её удобнее, гораздо удобнее.
Правда, не для сапиента, а для демона.

<<Убить>>.

Я не успел привести в готовность боевые модули.
Меня окружила до ужаса знакомая тюрьма "--- мир замерцал всеми цветами радуги, по телу пробежала волна парестезий.

"--*Чистилище, "--- невозмутимо продолжила Заяц.
"--- Интересно, не правда ли "--- к устройству, созданному для защиты, прикрепить пыточную камеру для его создателей?

<<Что тебе нужно?>> "--- выдохнул я.

"--*От тебя?
В целом "--- ничего.
Мне просто нужно с кем-нибудь поговорить без опаски, что этот кто-то меня прикончит.

Женщина неторопливо подстраивала систему под себя.
Её руки плавно скользили, из горла вырвался странный звук "--- не то птичья трель, не то звон металла.
Система реагировала на звуки и жесты стремительно, я уже не успевал следить за выводом.

"--*Скажи мне, "--- проговорила Заяц, и от её интонации засквозило ледяным ветром, "--- правда ли, что именно Ад устроил на Тси-Ди диверсию, в результате которой Машина приобрела самосознание?

Я словно провалился в глубокий колодец.

<<Кто тебе это сказал?>>

"--*Неважно.
Источник надёжный.

<<Это важно!
Кто тебе\ldotst
Заяц, что ты собираешься\ldotsq>>

Ответ дала система управления "--- мощность возросла до максимума.

<<Заяц, стой! "--- закричал я.
"--- Только не сейчас\ldotst>>

"--*У меня больше нет времени, Аркадиу, "--- ответила женщина.
"--- Все эти годы я винила себя "--- думала, где я проглядела, что я сделала не так.
Я мстила самой себе этой долгой, одинокой жизнью.
Я унижала себя без причины, наказывала себя без вины.

<<Заяц! "--- закричал я, зная, что женщина меня не станет слушать.
"--- Сейчас решается наше будущее!>>

"--* В бездну будущее, у меня его нет.
И у Ордена Преисподней его не будет "--- я оставлю ему самое страшное проклятие\ldotst и прощальный подарок от всего моего народа.

Включился передатчик на водородной частоте "--- Заяц собралась говорить перед всеми хоргетами звёздной системы.

\begin{quote}
\mulang{$0$}
{<<У вас не было предков и учителей.}
{``You have no roots and no teachers.}\\
\mulang{$0$}
{Вы не оставите потомков и последователей.}
{You'll have no scions and no followers.}\\
\mulang{$0$}
{Ни одна неопределённость не привела к вашему рождению.}
{Here is no indeterminacy leading to nascence of you.}\\
\mulang{$0$}
{Ни один квант не сохранит памяти о вас.}
{Here'll be no quantum keeping a memory of you.}\\
\mulang{$0$}
{Вы не существовали и никогда не будете существовать>>.}
{You never existed and never will.''}
\end{quote}

План рушился под моими пальцами, словно высыхающий песчаный домик.
Пространство искривлялось вокруг меня, я вдруг ощутил незнакомое прежде чувство отчуждения от собственного демона.
Но вскоре всё прояснилось "--- это было искажение восприятия мозга, воющего от неосязаемой, нематериальной, но абсолютно нестерпимой боли.

\begin{quote}
<<Ад, вы заплатите.
Ветер-Дующий-Ниоткуда была убита вами.
Ад, вы заплатите>>.
\end{quote}

Пространство прорезал экран "--- и Тра-Ренкхаль взорвался от неслышных криков отчаяния.
Экран накрывал демонов, возвращая их в первозданный хаос.

\begin{quote}
<<Ад, вы заплатите.
Хрустально-Чистый-Фонтан был убит вами.
Ад, вы заплатите>>.
\end{quote}

И снова экран.
Интерфекторы тщетно пытались проникнуть внутрь системы, но Заяц знала своё дело.
Демоны гибли один за другим, так до конца и не понимая, что происходит.
Сапиенты по всей планете с недоумением смотрели на соплеменников, которые вдруг начинали нечленораздельно голосить или просто падали, словно брошенные марионетки\ldotst

\begin{quote}
<<Ад, вы заплатите.
Существует-Хорошее-Небо был убит вами.
Ад, вы заплатите>>.
\end{quote}

<<Отходим!
Всем отойти за барьерную высоту\ldotse
Узел 12, ответьте\ldotse
Выжившие на уровне 3, всем немедленно покинуть тела\ldotse
Докладываю, узел 9 уничто\ldotst>> "--- надрывались в эфире полные паники голоса.
Союзники и враги, агенты Скорбящих, шпионы Картеля и верные служители Ордена Преисподней\ldotst

\begin{quote}
<<Ад, вы заплатите.
Пчела-Нюхающая-Вереск была убита вами.
Ад, вы заплатите>>.
\end{quote}

Слова Эй-В0 гремели в моём сознании, перекрываясь, превращаясь в тягучий, полный холодной ярости вой.
За каждым экраном следовало имя.
Друзья, сослуживцы, знакомые.
Десять, сто, тысяча имён\ldotst

В это время с орбитального спутника до меня дошло короткое сообщение Чханэ:

\mulang{$0$}
{<<Нас встретили. Прощай>>.}
{``We were met. Farewell.''}

\section{[U] Прощение}

Мир замер перед моими глазами.
Я остался совершенно один во Вселенной, полной врагов, запертый в бункере с обезумевшей от ненависти человеческой женщиной.
И снова, как в былые времена на пороге гибели, ко мне вдруг пришло чувство лёгкости.

Заяц прекратила атаки, сосредоточившись на защите системы.

"--*Не смей меня обвинять, Аркадиу.
Я сделала то, что должна была.

<<Я и не собирался>>.

Заяц промолчала.
Я почувствовал в этом молчании неуверенность.

"--*Я люблю их всех.

<<Как и я люблю моих друзей.
Я организовал сопротивление ради того, чтобы им жилось лучше>>.

Заяц улыбнулась.
В уголках её глаз замерли слёзы.

"--*Я бы хотела узнать их получше.
Особенно ту, с западным говором.
В ней есть что-то от прежних тси.
Наверное, беспечность.

<<Чханэ только что взяли в плен.
Безымянного тоже>>.

"--*Помоги им, Зайчик.

Я вздрогнул.
Рядом с нами стоял рослый, наполовину состоящий из механики канин.
Седая грива, свалявшаяся борода, белые клыки и пронзительные голубые глаза.
Последние слова принадлежали ему.

Заяц потеряла контроль над своим мозгом.
Порождение больного разума спроецировалось в систему в виде голограммы.

Женщина не обернулась.
Она знала этот голос.
Он родился в глубине её сознания, он был именно таким, каким знала его она.
Голограмма тем временем подошла к женщине и положила руку ей на плечо.

"--*Зайчик, "--- ласково шепнул Фонтанчик.

"--*Я ждала тебя, "--- прошептала Заяц.
"--- Знала, что ты будешь лишь плодом больного воображения.
Знала, что ты уже давно превратился в пар.
Но я хочу верить, что где-то в этом или другом мире для тебя ещё есть место.

"--*Оно есть, Зайчик.
Я стою прямо за тобой.
Я пришёл забрать тебя.

Заяц заплакала.
Беззвучно и почти не двигаясь, словно герой древней легенды, вынужденный держать небесный свод и плачущий от нестерпимой боли в спине.
Женщина схватила огромный указательный палец и сжала его своей ручкой.

"--*Что там? "--- рассеянно спросила она.

"--*Вечернее небо, прохладный ветер и цветущий вереск, "--- ответил Фонтанчик.

Заяц, глубоко вздохнув, обратилась ко мне:

"--*Координаты и параметры связывающего устройства.

Система неторопливо изменила конфигурацию.
Интерфекторы учетверили атаки на защитные механизмы, но Заяц беспрепятственно выстрелила, израсходовала на экран ровно столько энергии, сколько нужно "--- ни больше ни меньше.

"--*Прощай, Играющая-С-Булыжником-Лиса.
Будь счастливее меня.

<<Прощай, Сотканный-Из-Темноты-Заяц.
Ты лучшая из тех, кого я знаю>>.

Из мультитула на её запятье выдвинулся тонкий щуп "--- резонанс-взрыватель.
Заяц светло улыбнулась Фонтанчику, приложила щуп к виску "--- и её голова превратилась в кровавую кашу.

Голограмма, бросив в мою сторону добрый мудрый взгляд лучистых глаз, растаяла в воздухе.
Интерфекторы проникли внутрь три секунды спустя.

\chapter{[:] Запах воды}

\section{[:] Чужое тело}

Тахиро шёл по заваленному обгоревшими трупами полю.
Кто-то из хргада нашёл свою смерть в жестяных коробках, начинённых метательными снарядами;
эти коробки спасали от всего, кроме ужасной смерти.
Кому-то посчастливилось увидеть перед смертью далёкий зелёный лес.
Руки, ноги, внутренности лежали как попало, словно увеличенная в тысячи раз коробочка с кормом для рыбок.
Горе, неизбывная боль для оставшихся в живых кани "--- жалкая декорация для настоящей, уже давно отгремевшей битвы хоргетов.

Разумеется, эта битва была не просто декорацией.
Картель редко изводил просто так ценную биомассу.
Поток эманаций сапиентной битвы восполнил силы храбрых демонов, одержавших сегодня почти бескровную победу над адским гарнизоном.
\mulang{$0$}
{Это был крестовый поход детей, скотобойня, прелюдия к триумфальному пиру.}
{It was children's crusade, slaughtering, prelude to triumphal feast.}

Тахиро поморщился и повёл в сторону занемевшей собачьей ногой.
Он плохо обращался с телами кани, а сейчас и вовсе пришлось пользоваться чужим "--- настройки сообщил специально обученный легионер.

Вот на дороге попался умирающий.
Его невидящие глаза бродили в глазницах, дыхание то усиливалось, то ослабевало.
Мучения должны продлиться ещё около пяти минут, пока сознание окончательно не уйдёт, свершив этот последний акт милосердия.

Тахиро вздохнул и вытащил из пустого огнестрельного оружия шомпол.
Его с детства учили, что облегчать смерть "--- нелёгкое испытание.
Убийство слабо вяжется с понятием помощи.
Это следует просто принять.

Умирающий затих.
Шомпол Тахиро оставил в шее.

Вот ещё один живой.
Глаза залила кровь, но они целы.
Из голени торчит осколок кости.

"--*Ra cisamau, "--- попросил раненый, услышав шаги.
"--- Au rr hou rama ho mere ho, cisamauaa!\footnote
{Пожалуйста, помоги мне. Я не знаю, друг ли ты, враг ли ты, но помоги, умоляю (язык мехрр-ау-о, потомки Хргада, планета Запах Воды). \authornote}

Плачущий звук завершился настоящим плачем;
слёзы текли по мохнатому лицу хргада, смывая кровь, пыль и копоть <<грязного>> двигателя внутреннего сгорания.
Тахиро отвернулся и пошёл дальше.
Этот выживет и так.
В нём много силы.

\mulang{$0$}
{"--*О, какие лица.}
{``What a face.}
\mulang{$0$}
{Тахиро Молниеносный, "--- послышалось за спиной.}
{Tajiro the Thunderbolt,'' the voice behind him said.}

\mulang{$0$}
{"--*Какие спины уж тогда, "--- пошутил Тахиро и поднял руки.}
{``Rather, what a back,'' Tajiro made a joke and put his hands up.}
\mulang{$0$}
{"--- Добровольно сдаюсь, для вашего командования есть важная информация.}
{``I surrender willingly, there's some vital information for your command.''}

\mulang{$0$}
{"--*Можешь повернуться и опустить руки, "--- милостиво разрешил голос.}
{``You can turn around and put your hands down,'' the voice graciously allowed.}
\mulang{$0$}
{"--- Оставь эти телодвижения для овощей\footnote
{Сейхмар (презрительное). \authornote}.}
{``Save these movements for the vegetables.''}

Тахиро подчинился.
Он знал, что о его поимке уже сообщили Гало.

Обладателем голоса оказался высокий худой канин в синей гимнастёрке без знаков различия.
Фиолетовые глаза смотрели презрительно.
Метательное оружие покоилось в кобуре "--- опытный легат-интерфектор Картеля не разменивался на ненужные знаки, чем порой грешили молодые демоны.

"--*Итак, я тебя слушаю.
О, hciou-rr\footnote
{Прошу прощения (классический хргада). \authornote},
подожди.

Демон подошёл к раненому.

"--*Cisamau, "--- снова попросил хргада, обернувшись на источник звука.
"--- ciso. Mauaa.

Демон наступил на искорёженную ногу солдата, и тот зашёлся в крике;
крик усиливался, пока не перешёл в визг.
Легат давил со знанием, выбирая самые крупные нервы.
Визг сменился прерывистыми стонами, и наконец раненый затих "--- бешено колотящееся сердце остановилось насовсем.

"--*Он мог бы дать вам больше эманаций, "--- заметил Тахиро.

"--*Но уже не даст, "--- пожал плечами демон.
"--- Он был солдатом, как и мы.
Участь солдата "--- страдать и умереть.
Участь женщины "--- в муках рожать новых солдат.

"--*Всего лишь пропаганда.

"--*Хочу заметить, ваша.
Мы пришли совсем недавно, но здесь, в обители процветания и мира, была благодарная почва для войны.
Мягкотелый слабовольный скот, ещё вчера размахивавший цветами и распевавший песенки на фестивалях, взялся за оружие и принялся резать себе подобных.
Забавное совпадение, не правда ли?

"--*Они никогда не видели войны, поэтому\ldotst

"--*О, перестань.
Да, он не знал, ради чего ему придётся умереть.
Но он каждый день видел это, "--- демон пнул винтовку, "--- и это, "--- он махнул рукой на сгоревший танк, "--- и это, "--- он поднял с выгоревшей земли осколок мины.
"--- Он видел это каждый день по каналам пропаганды, знал последствия применения этой техники.
И все они видели, все до единого смотрели ваши пацифистские ролики.
Неужели ты думаешь, что они не знали, на что идут?
Может, лучше сказать, что их учили не сомневаться, ибо сомнение подрывает любую, даже цветочно-песенную власть?

Тахиро промолчал.

"--*Я тебя слушаю, Тахиро Молниеносный.
Что ты хотел нам сообщить?

"--*Как кстати, что мы заговорили об оружии.
Я хотел бы сказать пару слов касательно вашей обновки.

"--*А, "--- кивнул интерфектор.
"--- Как вам?
Понравилось?

"--*Идея интересная, если не считать некоторых отдалённых эффектов со стороны пространства-времени, "--- признал Тахиро.
"--- Впрочем, об этом я хотел бы поговорить с Гало.

"--*Я плохой собеседник? "--- ухмыльнулся легат.
"--- Тахиро, мы твои фокусы впитали с молоком матери.
Выкладывай всё здесь, дольше проживёшь.

"--*Хотелось бы собеседника моего уровня, Байс Старое Кольцо.
Когда тси передали экран искривлённого пространства нам, у нас тоже были придурки, которые хотели воспользоваться оружием немедленно, не разбираясь в его принципах.
К счастью, из-за умников вроде меня ты ещё жив.

Легат задумался и посмотрел на убитого им солдата.

\mulang{$0$}
{"--*Тси, говоришь.}
{``Qi, you said.}
\mulang{$0$}
{Хорошо, сейчас будет транспорт.}
{Well, your transport will be here soon.}
\mulang{$0$}
{Кстати, тебе тело не впору, на мой взгляд.}
{By the way, your body doesn't fit, I guess.}
\mulang{$0$}
{Узковато в плечах?}
{Too close on the shoulders?''}

\mulang{$0$}
{"--*И ноги чересчур длинные, "--- кивнул Тахиро.}
{``And legs are too long,'' Tajiro nodded.}
\mulang{$0$}
{"--- Я здесь проездом, надел, что было.}
{``I'm just passing through, I've to wear what I've got.''}

\section{[:] Беседа Гало и Тахиро}

\epigraph
{Наш враг "--- не объединение, а проводимая им политика.
Мы поддержим Картель против любого внешнего или внутреннего агрессора, словно не существует доктрины угнетения, и будем бороться за свои права, словно не существует войны.}
{Politika generalia, тезис 5.
Редакция Гало Кровавый Знак}

\spacing

"--*Что ж, Тахиро Молниеносный, вот ты и попал в мои руки.
Не думал, что смерть придёт ко мне в образе пленника.

"--*Мы скоро умрём оба, "--- заметил Тахиро.

"--*Верно, "--- согласился Гало.
"--- Ланс не спит.
Он уже знает, что я тебя пленил.
Это идеальное стечение обстоятельств, чтобы устранить нас обоих.

"--*Поверь, я не хотел жертвовать собой, чтобы убить тебя.
Это\ldotst

"--*Да знаю я, "--- перебил его Гало.
"--- Я знаю, что ты пытался сохранить мне жизнь, чтобы таким образом вбить клин в Картель.
Но против нас сейчас борются силы куда более могущественные, чем Ланс-нат Алмаз.

"--*Против \emph{нас}?

"--*А ты ещё не понял?

Тахиро промолчал.

"--*Знаешь, чем мы с тобой отличаемся от Ланса?
Или от того же Лусафейру? "--- Гало в упор посмотрел на Тахиро ледяными глазами.
"--- Мы никогда не пытались соответствовать системе.
Мы искали собственный путь.
Лу на самом деле ещё больший бес, чем я, и его старания вогнать себя в выбранную им ячейку меня печалят.

"--*Мы "--- часть системы, "--- возразил Тахиро.

"--*И в этом заключена главная тайна мироздания, "--- подхватил Гало.
"--- <<Прогресс существует тогда и только тогда, когда каждая деталь механизма руководствуется собственным пониманием комфорта>>.
При всём деспотизме Картеля эта часть его парадигмы безусловно прогрессивная.
Картель постоянно вынуждает нас бороться за свой комфорт, чтобы отделять слабых и не тратить ресурсы на то, что уже не нужно.
Вариантов деталей не может быть бесконечное множество, и обеспечение комфорта для одной детали тянет за собой обеспечение комфорта для многих других.
Только так система будет развиваться.

"--*Есть детали, которые склонны к излишествам, "--- заметил Тахиро.

"--*Комфорт "--- это нечто определённое, имеющее рамки.
Я не рассматриваю неумеренные аппетиты как вариант нормы.

"--*Есть детали, которые не могут защититься.

"--*Им придётся научиться.
Это и есть прогресс.

\spacing

Молчание затянулось.
Наконец Гало не выдержал:

"--*Тахиро, ты уверен насчёт истинного равновесия?

"--*Твои расчёты говорят об обратном?

"--*Мои расчёты подтверждают это.

"--*Тогда о чём был вопрос?

Гало криво ухмыльнулся.

"--*Я всё ещё надеюсь избежать жертвы.

"--*Я тоже, "--- признался Тахиро.
"--- Мы ведь оба знаем, чему будем посвящены.
Твоя смерть ударит по твоим врагам, моя "--- по моим.
Самый чувствительный удар получат наши общие противники.

"--*Если моих личных врагов пробьёт насморк, то наши общие ещё и покашляют, "--- саркастически проворчал Гало.
"--- Хорошо, хватит о равновесии.
Не хочу тратить остаток жизни на работу.
Расскажи лучше о Лу, как он там.

"--*Держится молодцом, "--- улыбнулся Тахиро.
"--- Зря тебя печалят его попытки приспособиться.
Если Лу будет продолжать в том же духе, то непременно доживёт до конца войны.

"--*Но не приблизит этот конец.

"--*Почём тебе знать? "--- пожал плечами Тахиро.
"--- Он делает то, что может и должен.
Я никогда не сомневался в Лу.

"--*Поэтому он выбрал тебя, а не меня.

Тахиро промолчал.

"--*Я очень по нему скучаю, "--- сказал Гало.
"--- Но смысла возвращаться, если достигнуто равновесие, уже нет.
Ад и Картель "--- не просто противники.
Они уже не могут друг без друга существовать.
Примерно как кошка в отсутствие мышей не знает, к чему приложить коготки и зубы.

"--*Примерно как женщина в отсутствие мужчины не знает, что делать со сводящим с ума либидо, "--- согласился Тахиро.
"--- Я тоже скучаю по Лу.
Он стал пленником системы "--- мы даже не можем поговорить наедине.

"--*Мы воевали не за это, "--- процедил Гало.
"--- Я бы лучше вернулся в маленький, окружённый врагами лагерь на Преисподней, чем там, на сверхбезопасном Капитуле, днями выбивать у бюрократической машины право сто пятьдесят секунд поговорить с братом.

"--*А если бы всё изменилось, ты бы позволил мне\ldotst

Глаза Гало вдруг вспыхнули лютой ненавистью.

"--*\dots тоже общаться с ним?

Гало ухмыльнулся.
Глаза потухли так же быстро, как и загорелись.

"--*С двенадцати до четырнадцати по чётным дням.

"--*Ты неисправим.

\spacing

"--*Мы боролись не против друг друга, а против системы.

"--*Верно, "--- подтвердил Гало.
"--- Мы искали путь положить происходящему конец.
Мы были уверены в правильности парадигмы своих фракций и пытались распространить их огнём и мечом.

"--*Не понимая, что Вселенной сейчас нужен мир, "--- закончил Тахиро.

"--*Это было самое тяжёлое осознание, Тахиро, "--- признался Гало.
"--- Я "--- воин, и воины впервые за всю историю оказываются не нужны.

"--*Мы нужны как никогда, Гало, "--- возразил собеседник.
"--- Мы разрушим эту систему.
Только воин может разрушать.

Гало засмеялся.

"--*Светлая твердь, как же я хочу быть прав.
Я никогда в жизни не испытывал таких сильных желаний.
Печально осознавать, что, пытаясь убить тебя, я пытался срезать противовес, держащий меня в этой Вселенной.
Но ещё печальнее было бы то, что тебя можно было просто убить и сбежать от убийц Ланса, а я этого не сделал.

"--*Вероятность нашей правоты стремится к единице, "--- рассудительно заметил Тахиро.

Глаза Гало вспыхнули.

"--*Ты пытаешься меня успокоить, \emph{дзайку-мару}?

"--*И у меня это неплохо получается, \emph{хорохито}.

Мужчины расхохотались.
Впрочем, веселье быстро утихло.

"--*Мне лезет в голову всякая религионая лабуда, "--- признался Гало.
"--- Помнишь, на Преисподней монахи рассказывали легенду о двух героях, которые должны были простить друг друга, чтобы войти в обитель усопших?

"--*Легенда о Мичи и Таро, "--- кивнул Тахиро.
"--- Она была моей любимой.
Хотя я думаю, что корни её куда глубже и они\ldotst не религиозного характера.

"--*Отчего так? "--- с интересом приподнял голову Гало.

"--*Это признание простого факта, что благодаря друг другу герои стали теми, кем стали.
Кошка может не осознавать, благодаря кому она получила острые когти, но мы ведь не кошки, верно?

Гало кивнул и опустил голову.

"--*Однажды я спросил отца, о чём же всё-таки спорили Мичи и Таро.

"--*Кажется, между их родами была вендетта, "--- припомнил Гало.

"--*Отец ответил, что это неважно, потому что их спор ничем не отличался от любого другого спора.

Гало ухмыльнулся.

"--*А вообще всё это ерунда, "--- заключил Тахиро.
"--- Согласно верованиям моего народа, в обители усопших нет места воинам.
Они умирают навсегда.

\begin{verse}
Я был рождён в бою, но жить не начал,\\
И я не жил, я дрался, пока билось моё сердце\ldotst
\end{verse}

\begin{verse}
"--*Сражением была любовь, но не отдохновением,\\
С друзьями был союз, но не весёлый пир\ldotst
\end{verse}

"--* продолжил Гало.

\begin{verse}
"--*Я сделал воинами тех, кого родил для жизни,\\
И вот мой час пришёл\ldotst
\end{verse}

"--*И я умру, и больше никогда живым не буду, "--- хором закончили мужчины.

"--*И что изменится? "--- тихо смеясь, спросил Гало.

"--*Я как-то задал похожий вопрос отцу, "--- ответил Тахиро.
"--- Мы шли по свежим вулканическим отложениям, отыскивая распустившиеся сколецитовые хризантемы.
Я постоянно оступался, разбивал себе руки и колени.
Когда я почти готов был взбунтоваться, отец погладил меня по голове и сказал:
<<Скалу терзают пыльные бури, скалу размывает вода, но она ничто не хранит так бережно, как своё имя и крохотные царапины от людских ногтей.
Имена и тропы "--- это сокровище земли;
имя делает скалу значимой, а тропа даёт людям надежду.
Тем, кто пойдёт за нами, будет легче>>.

"--*Это была его фраза? "--- удивился Гало.
"--- Я слышал её несколько раз от Лу.

"--*Да, так говорил мой отец, Акено та Ханаяма.
Он знал, что нет одеяния лучше, чем благородная кожа.
Он носил её с гордостью, когда ходил в пустоши разведывать вулканические отложения.
Он был хорошим тама\footnote
{Тама "--- на Преисподней: геолог, исследующий свежие вулканические отложения на предмет минералов и важных реактивов.
Также тама искали так называемые <<террасы>> "--- места, которые защищены от выбросов и могут быть использованы для земледелия. \authornote}
и хорошим человеком.

Гало промолчал и посмотрел на часы.
Такое простое повседневное действие.

"--*Убийцы Ланса уже здесь, "--- сообщил Тахиро.

"--*Они в любом случае разделят нашу участь.
Ланс приказал после отчёта об убийстве зачистить пространство несколькими залпами экрана, а затем взорвать планету, чтобы скрыть следы.

"--*Взорвать планету и уничтожить целый гарнизон из-за нас двоих? "--- удивился Тахиро.

"--*Ужасная, просто невероятная расточительность, "--- согласился Гало.
"--- Власть его развратила.
И ведь вроде бы вся эта история тысячелетиями будет покрыта мраком, если вообще когда-либо всплывёт на поверхность\ldotst и едва ли наши друзья узнают, как мы умрём\ldotst но как же неприятно сдаваться без боя!

"--*Неприятно, но надо, "--- признал Тахиро.
"--- Самый сильный удар мы нанесём именно так "--- без боя.
Понятия не имею, каково это.
Наверное, это словно тебя в детстве отправляют спать за шалость, и ты действительно идёшь спать.

Гало ухмыльнулся и посмотрел на товарища по несчастью.

"--*Ты был для меня самым понимающим, самым достойным врагом.
Всегда и во все времена.

"--*Ты всё ещё сердишься? "--- спросил Тахиро.

"--*Давно уже нет, "--- поморщился Гало.
"--- Кто я, по-твоему, всё тот же зелёный мальчишка?

Тахиро взглянул в застывшие ледяные глаза Гало.
Там царил мир.

\chapter*{Интерлюдия X. Мичи и Таро}
\addcontentsline{toc}{chapter}{Интерлюдия X. Мичи и Таро}

\textbf{Легенда Преисподней}

Велика сила героев, но смерть всегда вытянет на один фэно\footnote
{Фэно "--- мера веса Преисподней, примерно 200 г. \authornote}
больше.
Пришла пора отправиться по чёрной тропе и Мичи, поборовшему князя демонов, и Таро, разрубившему оковы пророчества.

Росло на вечном поле дерево сукэ\footnote
{Огнистая криптомерия "--- один из немногих видов древесных растений, процветающих на планете Преисподняя. \authornote};
цвело, и плодоносило оно по закону небес, и в свой час высохло до той сухости, когда ни пылинки жизни не остаётся в том, что имеет форму живого.
И так вышло, что прошёл Мичи по правую сторону от дерева, а Таро "--- по левую.
Зацепились связывающие их нити ненависти за дерево и натянулись струнами.
Тянет Мичи "--- а Таро тянет в ответ.
Да вот незадача: невелика сила умерших на чёрной тропе, не поднять им и зерна.
Отскрипели нити, откряхели герои, а победила крепость дерева сукэ.

Подумали герои да пошли к дереву, чтобы отцепить нити.
Там и встретились.

"--*Перейди на левую сторону, Мичи, "--- буркнул Таро, "--- из-за тебя мне даже последнюю дорогу не пройти!

"--*Вот ещё, "--- фыркнул Мичи.
"--- Кто неправ, тот пусть и переходит!

Занялся спор, и до драки дело дошло.
Да вот незадача: невелика сила умерших на чёрной тропе, не побороть им и мыши.
Устали Мичи и Таро, присели под деревом и пригорюнились.

"--*Я подумал над твоими словами, Мичи, "--- сказал наконец Таро.
"--- В них есть крупица смысла.
Я перейду направо.

"--*Я тоже подумал над твоими словами, Таро, "--- ответил Мичи.
"--- Они отражают тонкий луч истины.
Я перейду налево.

Так герои и сделали.
Однако нити ненависти снова зацепились о дерево сукэ, и не пойти героям дальше.

Занялся спор, и до драки дело дошло.
Да вот незадача: невелика сила умерших на чёрной тропе, не раздавить им и стрекозы.
Устали Мичи и Таро, присели под деревом и пригорюнились.

"--*Я понял, Таро, "--- сказал наконец Мичи.
"--- Раз мы признали правоту друг друга, но по-прежнему не можем идти дальше, нам должно быть по одну сторону.
Один из нас должен уступить.

"--*Я понял, Мичи, "--- ответил Таро.
"--- Раз мы признали правоту друг друга, неважно, кому выпадет доля уступить.
Давай бросим плоский камень.

Так герои и сделали.
Выпал жребий Мичи перейти на левую сторону.
Однако нити ненависти так опутали дерево сукэ, что не пойти героям дальше.

Занялся спор, и до драки дело дошло.
Да вот незадача: невелика сила умерших на чёрной тропе, не сломать им и сухой травинки.
Устали Мичи и Таро, присели под деревом и пригорюнились.

Шла мимо дерева девочка, умершая от акуко\footnote
{Ювенильная проказа ущелья Такэсако "--- детская эпидемическая инфекция планеты Преисподняя, смертность от которой составляет 38--50\%. \authornote}.
Увидела девочка Мичи и Таро и их беду.

"--*Почему вы сидите здесь, связанные этими нитями с деревом? "--- удивилась она.

"--*Мы не знаем, "--- ответил Таро.
"--- Мы боролись друг с другом, но ни один не победил.
Мы уступили друг другу, но лишь поменяли концы нитей местами.
Затем Мичи уступил мне по жребию, но мы всё равно не можем сойти с места.

"--*Как много нитей ненависти, "--- заметила девочка, "--- возможно ли распутать все!
Вы тянете за собой свои дома\footnote
{Дом "--- объединение, включающее два-три родственных клана. \authornote},
давно умерших предков и даже тяжёлые слова!
Виновато ли дерево сукэ?
Может быть, вам просто нужно друг друга простить?

"--*Простить? "--- удивились герои.

"--*Я играла с братиком, "--- сказала девочка.
"--- Порой мы колотили друг друга, когда играли в героев.
Порой я кричала, что ненавижу его, когда он был хорохито.
Но потом мама звала нас кушать и мы забывали свои роли и свои обиды.

Посмотрели друг на друга Мичи и Таро.
Долго смотрели они, вечность пришла и прошла, а они всё смотрели.
И вот нити ненависти ослабли и исчезли без следа.

Подняли Мичи и Таро девочку на сплетённые руки и пошли по чёрной тропе дальше.
А дерево сукэ рассыпалось в прах, едва герои скрылись за поворотом.

\chapter{\mulang{$0$}{[U] Наедине со Зверем}{Face the Beast}}

\section{[U] Ожидание допроса}

\spacing

Я почти физически ощущал натянутую вокруг <<сеть>>, замечающую малейшие изменения поля Кохани"--~Вейерманна в окрестностях моего демона.

Допрос урождённых демонов и оцифрованных сапиентов различается по методике.
С урождёнными демонами не церемонятся "--- интерфектор сразу влезает в их память, добывая нужную информацию.
Ядро личности оцифрованных представляет собой виртуальную пространственную структуру с очень сложной архитектоникой, идентичную мозгу сапиента.
Добыть информацию из материального мозга можно после его изучения в замороженном виде, оцифрованный мозг можно только убедить или вынудить.

Для убеждения урождённых сапиентов существует более сорока сложнейших протоколов допроса.
Для принуждения "--- один-единственный пункт:

\begin{quote}
<<Подвергнуть эскалационной обработке в Чистилище>>.
\end{quote}

<<Именно мы, оцифрованные, ковали победу Ордена Преисподней над Картелем.
Наверное, только сейчас Ад понял, насколько оцифрованные опасны>>, "--- подумал я с неуместным злорадством.

Чханэ и Митрис попали в плен.
Это было ясно, как солнечное утро.
Возможно, что они даже не добрались до Тси-Ди "--- их спугнул патруль, и друзья перенеслись обратно в звёздную систему Тра-Ренкхаля, кишащую приведёнными в боевую готовность агентами из-за устроенной Заяц диверсии.
У них не было шансов скрыться.

Время "--- жестокая вещь.
У Заяц его было достаточно для сумасшествия.
А вот друзьям для дела не хватило.

\section{[U] Допрос}

\epigraph{И сказал Ал-ла Талиму: <<Сними одежды твои и драгоценности твои и оставь оружие твоё и нагим войди во врата джайанна\footnote
{Джайанна "--- <<судилище>>, несотворённое место, управляемое женщиной по имени Джаблел.
Согласно Хакем-Аяту, все боги мироздания должны пройти через него, чтобы стать Творцами. \authornote}.
Огонь будет жечь тебя, и ибли\footnote
{Ибли "--- <<мучители>>, неразумные прихлебатели Джаблел. \authornote}
будут насмехаться над лицом твоим и членами твоими и походкой твоей, и дивы\footnote
{Дивы "--- <<писцы>>, бесстрастные существа, ведущие летопись Вселенной. \authornote}
припомнят самые стыдные дела твои и позор твой, но ты иди, как ходишь по дворцу твоему.
И если не склонится голова твоя, и если взгляд твой будет горд и милостив, как прежде, падут ниц стражники вторых врат, и выйдешь ты из джайанна перерождённым, и не будет тебе более нужды ни в одеждах, ни в драгоценностях, ни в оружии, ни в придворных льстецах, и будешь ты отныне равным Мне, ибо Я прошёл тем же путём>>.}
{Хакем-Аят, 14:3--5}

\spacing

"--*Аркадиу Талианский Шакал.
Согласно нашим данным, первоначальный состав Скорбящих "--- оцифрованные тси и люди Тра-Ренкхаля.
91\% "--- ветераны битвы на Могильном берегу.

"--*Это так, "--- подтвердил я.

"--*Эти данные не нуждаются в вашем подтверждении.
Некоторых из этих новобранцев мы успели схватить несмотря на то, что вы успели предупредить всех перед поимкой.

"--*Кого именно вы схватили? "--- спросил я.

"--*Это для вас уже не имеет значения, "--- ответил голос.

"--*Давайте так, "--- сказал я.
"--- Чем больше информации вы сообщите мне, тем выше вероятность, что я стану с вами добровольно сотрудничать.

"--*Давайте так "--- с этого момента вы перестанете торговаться, "--- ответил голос.
"--- Ваша судьба зависит от тех данных, которые вы нам предоставите.
Вас позвали для уточнения некоторых деталей.
Был проведён анализ вашей личности и вынесено решение "--- эскалационную обработку в Чистилище отменить.

Если бы у меня была нервная система и кожа, меня прошиб бы пот.
Начало протокола номер тринадцать.
Дело дрянь.

"--*Если приведённый коэффициент пользы превысит $0.986$ "--- вы будете жить.
Если суммарный коэффициент дезинформации превысит $4.2$ "--- вы будете подвергнуты терминальной обработке.

Терминальная обработка.
Меня замучают до полной потери рассудка в самой ужасной пыточной камере всех времён.
Приведённый коэффициент пользы на труднодостижимой, но вполне реальной цифре $0.986$ "--- уловка.
Меня обязательно уничтожат.
Вряд ли кто-то верил, что я попадусь на такой простой трюк, но протокол есть протокол.

<<Общий смысл пункта 8, подпункта .3 "--- нельзя лишать пленника надежды, но единственной надеждой пленника должен быть допрашивающий>>.

Я опустил голову.
Что может сделать маленький шакал против стаи матёрых волков?
Только бы избежать Чистилища.
Страшный и недостойный конец существования\ldotst

"--*Кто была женщина, которая управляла системой?

"--*Тси, которую мы нашли в капсуле в состоянии анабиоза.

"--*Информация соответствует истине с вероятностью 98\%.
Она согласилась сотрудничать с вами?

"--*Она втёрлась к нам в доверие, чтобы реализовать план мести.
Мы потерпели поражение из-за её действий.

"--*Информация соответствует истине с вероятностью 76\%.
Последний экран помог освободиться вашим союзникам, и они вступили с нами в бой.
Как вы объясните это действие женщины?

<<Вступили в бой>>.
Лёгкая оговорка, которая могла означать только одно "--- Чханэ и Митрис мертвы.
Или меня пытаются в этом убедить.

"--*Я убедил её помочь.

"--*Информация соответствует истине с вероятностью 84\%.
Вернёмся к более ранним событиям.

Допрашивающий перешёл к другому вопросу.
Они мертвы.
Я вдруг вспомнил ласковую улыбку Чханэ.
Она любила вытаращивать глаза, когда видела, что я на неё смотрю.
А потом крепко зажмуривалась и целовала меня в нос.
Глупая привычка, которую я ношу в памяти, словно свёрток с драгоценными камнями.

Когда я оцифровал подругу, Чханэ сказала: <<Ну всё, Лис, ты от меня теперь вечность не отвяжешься>>.
Я предлагал ей найти убежище, жить в спокойствии.
Она ответила просто: <<Будем сражаться>>.
Что ещё могла ответить женщина, которую я полюбил больше жизни?
Но даже зная, какие опасности несло решение, облечённое в эти простые слова, я не был готов к её смерти.

"--*Где вы прячете оборудование для оцифровки?

"--*Оборудование мы прятали в храме Тхитрона, частично "--- в купеческом доме.
Оно было уничтожено сразу после оцифровки, больше там ничего нет.

"--*Информация соответствует истине с вероятностью 99\%.
Кто занимался сборкой оборудования?

"--*Грейсвольд Каменный Молот.

"--*Информация соответствует истине с вероятностью 15\%.
Суммарный коэффициент дезинформации $3.8$.
Требуется объяснение или другой ответ.

"--*Первоначально предполагалось, что ветераны Могильного берега вступят в ряды Ордена, "--- ответил я.
"--- Они даже прошли подготовку согласно протоколу 18, что вы можете легко проверить по паттернам их действий.
Это было до того, как мне стали известны подробности политики Ада по отношению к оставшимся тси.
Я удивлён, что Грейсвольд не рассказал этого вам.

Молчание.

"--*Грейсвольд Каменный Молот будет допрошен.
Как много существ вы успели оцифровать?

"--*Семьсот шестьдесят восемь.

"--*Информация соответствует истине с вероятностью 85\%.
Назовите всех демонов, чьи модули использовались для оцифровки.

"--*Я и Митрис Безымянный.

"--*Информация соответствует истине с вероятностью 70\%.
Чьи паттерны использовались для интерфекторов?

"--*Анкарьяль Кровавый Шторм.
Далее интерфекторов готовили агенты под прикрытием.
Мне не известны их имена, связь с ними осуществляла Таниа Янтарь.

"--*Информация соответствует истине с вероятность 89\%.

Ясное дело, ответить им на это нечего.
Интерфекторов готовила Чханэ.

<<Я не могу отомстить за подругу кровью, но сомнений вам оставлю предостаточно.
Кто знает, которая капля дезинформации расшатает вашу систему.
Излишняя подозрительность породит паранойю, партийные чистки завербуют новых агентов лучше, чем это сделаем мы.
Я не последний из тси\ldotst вернее, из Скорбящих>>.

Мысленная оговорка почему-то меня развеселила.

"--*Назовите имена всех новооцифрованных.

"--*Я назову, если вы сообщите имена тех, кто попал к вам в плен.

Молчание.

"--*В плен были взяты тридцать восемь Скорбящих, "--- сказал голос.
В моей голове возникли имена.
"--- Один был уничтожен.

"--*Кто был уничтожен?

"--*Тхартху Танцующая Тень.

Перед моими глазами промелькнуло доброе лицо женщины, которая больше всего на свете любила рисование и сказки.
Несмотря на недостаток пленных, Ад решил уничтожить её "--- оставлять визора в живых было чересчур большим риском.
Митрис Земляная Змея тоже среди пленных "--- он был рядом со своей подругой до конца.

<<Прощай, Птичка.
Пусть твой путь в пристанище будет лёгким>>.

Я напрягся, и в моей памяти возникли нужные имена.

"--*Информация соответствует истине с вероятностью 97\%.

"--*Этот список бесполезен для вас, как и вербовка агентов.
Во время подготовки я предупредил товарищей, чтобы они относились к пленённым как к врагам, а к информации, которой располагали пленённые, как к известной врагу.

Молчание.

"--*Информация соответствует истине с вероятностью 90\%.
Балл ПКП "--- $0.979$, балл СКД "--- $4.11$.

Я подавил приступ судорожного смеха.
Допрос окончен, последняя фраза была попыткой выжать остатки сока из жмыха.
<<Пограничные значения.
Бойся, предатель, метайся в попытках спастись, ведь спасение в семи тысячных от тебя.
Не дождётесь>>.

"--*Это всё? "--- осведомился я.

"--*Вся требуемая информация получена.
Вы будете казнены через 16 минут.

"--*Информация соответствует истине с вероятностью 100\%.
Приведённый коэффициент презрения "--- единица, "--- съязвил я.

Ответом было молчание.

\section{[U] Казнь}

\epigraph
{Et in Arcadia ego\footnote
{И в стране пастухов я (эллатинский). \authornote}.}
{Крылатое выражение культуры Ромай.
Древняя Земля}

\spacing

Это море совсем не было похоже на те лазурные южные моря, за которые лорды и короли ломали копья и жизни подданных.
Над северным морем всегда царила лёгкая тьма, и в этом похожем на вуаль мраке терялись далёкие горизонты.
Холодная чёрная вода шептала, а не пела, её тяжесть вгоняла в сон.
Берег был отнюдь не ласковым пляжем, на мягком прохладном песке лежали следы недавних штормов "--- обломки досок, обрывки ткани, разная мелочь\ldotst и водоросли, огромные кучи бурых гниющих водорослей, от которых исходил лёгкий прелый аромат.
Чуть поодаль росли золотистые, болезненного вида сосны и тонкий, ужасно колючий шиповник с облетевшими листьями, усыпанный огромными алыми ягодами.

Я смотрел на догорающий вдали закат.
Влажный ветер бил мне в лицо, солёно-пряный аромат моря въедался в волосы, чтобы не раз вернуться обрывками сновидений в будущем.
Я думал, что обязательно должен закончить свой путь здесь.
Южные моря с их тёплой солью, кровавыми сражениями, чернявыми красотками, вином и изысканными фруктами были созданы для жизни.
Для смерти же мне не найти лучше этого всепрощающего моря, спокойного берега с его заспанными соснами и мрачновато-жизнерадостным шиповником.

Тогда я считал себя взрослым, повидавшим жизнь существом.
Кто я сейчас?
Мудрый старик, скептически относящийся к собственной мудрости?
Молодой дурак-максималист, задумавший невозможное?
Одно было несомненно "--- о северном море придётся забыть.
В моём распоряжении "--- виртуальная каморка, имеющая размеры лишь для моего сознания.
Темница для разума, из которой я уже не выберусь.

Может, поверить хоть сейчас, что там, за гранью, что-то есть?
Я никогда не желал рая и вечного блаженства.
Кусочек моря, Чханэ, Атрис и Митхэ, встречающие меня "--- всё, что нужно.
Это будет самообман, оскорбление для истины, за которую я боролся, но смерть в отчаянии "--- оскорбление для всех, кто мне верил.

Рядом материализовался Грейсвольд.
Я облегчённо вздохнул.
Одно знакомое лицо лучше, чем ничего.
Незачем что-то придумывать.

"--*Это ты меня казнишь? "--- осведомился я.

"--*Нет, не я, "--- тихо пробормотал Грейс.
Я улыбнулся.

"--*Передай ей, что я буду рад её увидеть.

Грейсвольд робко ухмыльнулся.

"--*Она тебя слышит.
Пока ждёт приказа.

"--*Почему палачом назначили Анкарьяль, мне ясно.
Почему шестнадцать минут?

"--*Они до сих пор не могут решить, что с тобой делать, "--- объяснил Грейсвольд.
"--- Многие кричат о том, что тебя нужно уничтожить немедленно.
Однако столько же придерживаются мнения, что тебя нужно отпустить.
Кое-кто робко предлагает с тобой договориться.
Есть и достаточно прагматичная точка зрения.
Сейчас разрабатываются новые протоколы для Чистилища, основанные на медленном, мягком воздействии, а не на\ldotst ммм\ldotst экстремальной обработке.
Исследования говорят о том, что сапиентов нужно гнуть, медленно и терпеливо, а не ломать сильным ударом.
Одним словом, им требуются подопытные.

"--*Меня будут подвергать мелким неудобствам, типа имитации холодной клетки? "--- скривился я.

"--*Зря смеёшься.
Пыточная камера ломает дух не так часто, как это делают годы тюрьмы, медленные болезни и нищая старость.
Хотя основное направление\ldotst ммм\ldotst это удовольствие.
Не те мощные оргазмы, чередующиеся с дикой болью, как при обычной обработке, а простые, вроде тепла постели, поцелуев и чувства наполненного желудка.
Разработчики говорят, что результаты многообещающие, хоть и не сиюминутные.

Я хмыкнул.

"--*Ты "--- недоласканный ребёнок, "--- напомнил Грейс.
"--- Да, детство твоего мозга осталось далеко позади, ты неоднократно проходил коррекцию личности.
Но есть кое-что, что останется с тобой навсегда.
Материнская ласка может стать к тебе отличным ключом, хочешь ты этого или нет, и шансов устоять у тебя меньше, чем у твоих друзей-тси.

"--*Я всегда буду помнить о войне!
Я всегда буду помнить, что чувства "--- иллюзия!

"--*Насколько долго ты будешь об этом помнить, если иллюзию нельзя отличить от реальности? "--- резонно спросил технолог.

"--*И зачем ты мне это рассказываешь?

"--*Я молю лесных духов, чтобы сторонники казни победили, "--- глухо сказал Грейсвольд.
Это была правда.
Я чувствовал его чёрное отчаяние.

Если сказанное "--- истина, то это не самый плохой вариант.
Я увижу своё северное море.
У меня будет мать и тёплый очаг, Чханэ, дом, розовый куст и сад с цветущими вишнями.
Я смогу\ldotst

Я резко оборвал мысль.
Вот они, эти бреши, о которых говорил Грейсвольд.
Мои собственные потаённые мечты, из-за которых я всё и начинал\ldotst

Я должен умереть сейчас.
Иначе\ldotst

Но я же хочу жить!
Я не боюсь смерти, но ужасно хочу жить!
Я дрался ради жизни, ради опыта существования в лучшем мире!

<<Изверги>>, "--- жалобно выплюнул мой разум.
Однако море и вишни упорно маячили перед внутренним взором\ldotst

Предел есть у всех.
Похоже, что я своего достиг.
Неважно, какая судьба меня ждёт "--- прежний Аркадиу Люпино прекратит существование здесь и сейчас.

Технолог покряхтел и, создав виртуальный стул, сел на него.
Такая простая, совершенно человеческая попытка оттянуть неизбежное.

Мы помолчали.

"--*Да, если ты хочешь знать масштабы действий.
В Аду было выявлено сорок тысяч агентов Скорбящих.

Сорок тысяч.
На нашей стороне было три процента адских демонов.
Я рассчитывал на десяток, на сотню, на пять сотен, но даже в самых смелых мечтах я не представлял, что у меня будет такая команда "--- огромная законспирированная децентрализованная сеть, проросшая Ад, словно гигантская грибница.
Сорок тысяч.
Цифра бесцельно дрейфовала на поверхности разума, словно принесённый отгремевшим штормом обломок с закладной доской.

"--*Более трёх тысяч пленены отделом 100 и уничтожены, остальные успели уйти.
И это только урождённые сапиенты.
Количество урождённых демонов, которые работали на вас, неизвестно, эту информацию хранят под строжайшим секретом, во избежание\ldotst
Ты поднял большую волну, Аркадиу.

"--*Нет нужды поднимать то, что поднимается само, "--- возразил я.
"--- Почва для подобного движения уже давно перезрела.
Число сторонников говорит само за себя.
Что-нибудь ещё?

"--*Есть данные, что\ldotst "--- Грейсвольд замялся.

"--*Договаривай.

"--*Ад начал переговоры с Картелем.

"--*Что? "--- осведомился я, уверенный, что ослышался.

"--*Именно так.
В Картеле Скорбящие тоже были.
Демоны приостановили военные действия из-за страха перед вами.
Сейчас обсуждается вопрос о тотальной чистке в обеих организациях.
Оцифрованных хотят извести под корень.

"--*Это ложь, "--- улыбаясь, прошептал я.
"--- Клан Тахиро, клан Усмане\ldotst

"--*Почти все члены клана Тахиро были агентами Скорбящих, "--- вздохнул Грейс.
"--- Из трёх тысяч уничтоженных две "--- клан Тахиро, первая и третья генерации.
Они приняли на себя самый тяжёлый удар.
Ещё семьсот "--- Усмане.
Когда до Ада дошла тревога, Тахиро попытались прийти на помощь Скорбящим-тси.
На Капитуле и ещё семи планетах произошли полномасштабные войны хоргетов.
Сорок процентов Тахиро, восемьдесят два процента Усмане уничтожено, потери Ада "--- двенадцать к одному, и это официальные данные.
Реальное положение дел намного хуже.
Усмане остались бы в стороне и попытались переждать бурю, но пара их старейшин попали под горячую руку, и весь клан, не разбираясь в ситуации, прислал Аду недвусмысленное сообщение: <<Ahirat\footnote
{Ahirat "--- послание, использовавшееся на планете Земля Врачевателей.
Значение: <<Вы пересекли последнюю границу вражды.
Мы будем убивать вас всем, чем можно убить, пока не умрём сами>>. \authornote}>>.
Всё в лучших традициях Хакем-Аята\footnote
{Священная книга народов Земли Врачевателей, регламентировавшая в том числе правила ведения войны. \authornote},
только чёрно-зелёной ткани не хватает.
Боюсь, что этот клан прекратил своё существование "--- они всегда считались самыми жестокими среди оцифрованных, но в пунктуальности им не откажешь.

Я промолчал.
Где-то на краю сознания гремели воинственные крики, вопли отчаяния и горестные восклицания на давно забытых языках.
<<Тахиро-джиме, банджай!>>
<<Монд-ааа! Джайаннау-ле харек, монд!>>

"--*Анкарьяль получила приказ о твоей ликвидации.

Анкарьяль появилась рядом.
Лицо женщины было бесстрастно.
Я встал со стула, расправил плечи, поднял голову и улыбнулся друзьям.
Хоть это и виртуальный мир, хоть у меня больше и нет ни плеч, ни головы, хоть это и не изменит мою судьбу, но всё же\ldotst
Вдруг это изменит что-то другое, например, Вселе\ldotst

"--*По ощущениям похоже на засыпание, "--- сказала Анкарьяль.
Её глаза встретились с моими.

\section{[U] Разными дорогами}

\epigraph
{Лучшая победа "--- это победа, которая носит маску поражения.}
{Ликан Безрукий, создатель языка Эй}

Кто я?
Где я?

Я шёл по дороге, запинаясь о камни.

"--*Меня зовут Аркадиу, "--- сказал я вслух и сам удивился собственным словам.

Рядом со мной шли трое "--- две женщины и мужчина.
Их лица были суровы, они смотрели только на меня.

Мысли путались, ощущения едва задевали поверхность разума.
Осталось одно огромное чувство, словно дыра в голове "--- острое чувство утраты.

"--*Raj sejra? "--- спросила та, что повыше, агрессивно скривив губы.

"--*Ta oj, Nar.
Kar lejma, "--- ответил мужчина.

"--*Dejnalo oj na mia, "--- добавила низкорослая, как мне показалось, немного печально.
Она единственная меня жалела.

Откуда-то в памяти всплыли слова "--- язык Эй, таблица B0.
Но из сказанного я не понял ни слова.

Вдруг пришло понимание "--- меня только что убили.

"--*Что вы делаете со мной? "--- спросил я.

"--*Ga, "--- мужчина грубо толкнул меня в спину, и я свалился на колени.
Мой тюремщик вздёрнул меня на ноги и молча рукой указал направление.

"--*Jol.
Laj de, — сказала высокая и, вытащив оружие, направила его на меня.
Пистолет, метающий металлические стержни.

<<Эти существа убили меня, а сейчас хотят убить ещё раз\ldotst>>

Страшно заболела голова.
Я снова упал на колени.

"--*Прекратите это, я не хочу это больше чувствовать, "--- пробормотал я.
"--- Кто вы?
Почему мне так плохо?

Мужчина кивнул высокой.

Я инстинктивно зажмурился, ожидая боли.
Кровь пульсировала в висках, как боевой барабан\ldotst

Боевой барабан Талии.
Большой барабан из воловьей кожи.
Он звучал точно так же.

"--*Чего стоишь? "--- недовольно сказал мужчина.
"--- Глаза открой.

Я осторожно приоткрыл глаза.
Высокая стояла, устало прислонившись к мужчине.

"--*Он всё-таки туповат.
Как думаешь?

"--*Это да, "--- согласился мужчина.
"--- И ведь понимает, но самое главное до него не доходит.

Я смотрел на них во все глаза.

"--*Вы меня не убьёте?

"--*Боюсь, что однажды не выдержу и прихлопну, "--- посулился мужчина.
"--- Открой глаза!
Да не эти\ldotse

Мир вновь обрёл краски.
За каменными незнакомыми лицами я увидел родные, ласковые черты.
Память вернулась.

"--*Неужели ты и в самом деле думал, что я убью друга? "--- засмеялся Грейс.
"--- Плохо же ты меня знаешь.

"--*Но как ты догадался?
Информации о связи было очень мало!

"--*Я просто не поверил, что после произошедшего ты бросил меня навсегда, "--- тихо сказал Грейсвольд.

Я задумался.

"--*Что произошло?
Как ты обманул\ldotsq

"--*Обычное копирование, Аркадиу.
Болванку Анкарьяль носила с собой.

"--*То есть технически я мёртв?

"--*Технически "--- да.
И по документам Ордена ты тоже значишься как ликвидированный элемент.
Здорово, правда?

"--*Эээ\ldotst

"--*Не будь идиотом, Аркадиу.
Думаешь, мы бы с Анкарьяль ещё существовали, если бы у одного интересного типа не хранились наши копии?
Тем более твоё тело осталось живым, а значит "--- ты как личность не почувствовал небытия.
Тело "--- необязательная часть демона, и в век ангельских технологий обратное также верно.

"--*И как это регламентируется законами Ада?

"--*Никак.
Знают многие, использует кто может.
Неписанное правило таково: если твоя смерть доказана "--- значит, ты вне игры.
Вот тут в дело и вступает твоя команда.
Фальсификация спасений "--- это целая отрасль в наше время, разумеется, тоже негласная.
Думаешь, на высших иерархов не было покушений и даже более того "--- удачных покушений?
Если рядом есть приспешники, в любой момент готовые восстановить тебя из резервной копии, смерть "--- не самое страшное событие.

"--*Кто ещё жив? "--- допытывался я.
"--- Тахиро? Айну? Гало?

"--*Все они в пристанище духов, "--- ответил Грейсвольд.

"--*Что такое <<пристанище духов>>?
Хранилище резервных копий?
Или это, чтоб вас всех, укромная планета, где живут те, кого давно считают мертвецами?

Технолог промолчал.
Я вздохнул.

"--*Я тоже туда попаду?

"--*Увы, "--- развела руками Анкарьяль.
"--- Для тебя и прочих Скорбящих пристанища не будет.
Поэтому "--- береги себя.

"--*То есть всё это бесполезно, "--- опустил голову я.
"--- Нужно бороться с системой, о размерах которой я\ldotst

"--*Я рада, что ты наконец понял задачу, --- вмешалась низкорослая.
"--- Бороться нужно с системой, а не с отдельными личностями.
До тех пор, пока устаревшая система не изменится, она сама будет рождать бунтарей.
Против этого невозможно выстроить оборону, от этого нет спасения.
Те, кто послали тебя на смерть, этого так до конца и не поняли.

Анкарьяль кивнула.
Низкорослая женщина улыбнулась одними губами.
Её глаза, сильно косящие в разные стороны, по очереди посмотрели на меня "--- словно два разных человека, весёлый и задумчивый.

"--*Кто-то должен рассказать сапиентам Земли, что они на что-то способны, "--- проговорила она, манерно растягивая слова.
"--- Гонка ещё не завершена.
Да, кстати, Стигма Чёрная Звезда.
Для меня большая честь, Аркадиу Люпино.

Я пожал её тонкую, но крепкую руку.

"--*Как хорошо знакомиться с друзьями, да? "--- подмигнула Анкарьяль.

"--*Я достаточно давно возглавляю группировку, стремящуюся к союзу с Картелем, "--- усмехнулась Стигма.
"--- Успела накопить большой опыт.
Так что ты не первый, господин Люпино.
Но отдаю тебе должное "--- на такие наглые действия мы пока не отважились.

"--*Это провал, "--- возразил я.

"--*Я была лучшего мнения о тебе, как о стратеге, "--- нахмурилась Стигма.
"--- Мы собрали огромное количество практически бесценного материала.
Такая информация просто не может достаться без жертв.
И я очень рада, что смогу работать с тобой и дальше.

"--*Он ещё в себя не пришёл, "--- объяснил Грейсвольд.
"--- Это нормально.

"--*Это предательство, "--- сказал я.
"--- Вас могут\ldotst

Анкарьяль внезапно обняла меня изо всех сил.
И тут же отстранилась.

"--*Могут, "--- подтвердила Стигма.
"--- И даже сделают, я уверена.

"--*Предательство, "--- повторил Грейс и вздохнул.
"--- Понятие, ставшее игрушкой в руках пропаганды.
То, что мы сделали, скорее гуманность.

"--*Иди, Кар, "--- пробормотала Анкарьяль.
"--- Тебя ждут.

Я вдруг вспомнил всё.
Каждую деталь нашего сокрушительного поражения.

"--*Меня больше некому ждать, "--- с горечью ответил я.

"--*А твой народ? "--- спросила Стигма, указывая куда-то назад.

Я не стал оборачиваться.
<<Взгляд>> почувствовал ободряющий свет Чханэ и Митриса.
Грейс смотрел на меня нежно, словно на любимого ребёнка.

"--*Грейс, я перед тобой в неопла\ldotst

"--*Молчи, "--- прервал меня Грейсвольд.
"--- Молчи и иди, человек.

Я опустил голову.

"--*Что с остатками тси?

"--*Сожалею, "--- ответил технолог.
"--- Адская пропаганда и селекция уже начали своё чёрное дело "--- из свободного народа делают мягкотелых счастливых существ.
Наш заговор "--- непродуманный, честно говоря, "--- заставит Ад закрутить гайки ещё сильнее, и я не знаю, доживёт ли свободолюбивый народ тси до момента, когда давление сорвёт крышку котла.
Все были уверены, что нас мало.
Что поделаешь, нам старательно внушают, что мы одиноки в своих стремлениях\ldotst

Грейсвольд помолчал и подумал.

"--*Тси, тси.
Вообще я бы на вашем месте\ldotst "--- Грейсвольд театрально оглянулся на товарищей и вполголоса продолжил:
"--- Я бы на вашем месте собрал генетические образцы тси, а также данные об их культуре, и попытался бы возродить их где-нибудь на укромной планете.
Я не сторонник воскрешения отжившего, но тси по-прежнему опережают прочих сапиентов на сотню тысяч лет, было бы жалко их потерять.
Ты у нас биодиктиолог, верно?
Молодец, хорошо подготовился.
А сзади стоят две женщины "--- живые носители культуры.

Я кивнул.

"--*А вы?

"--*А что мы? "--- пожала плечами Анкарьяль.
"--- Мы выполнили приказ "--- уничтожили подполье.

"--*Сейчас занимаемся тем, что казним тебя, "--- добавила Стигма и мило усмехнулась.

"--*Мы вернёмся обратно, дружище, "--- заключил Грейс.
"--- Пока ты остаёшься в тени, нам ничто не угрожает.
Кто-то ведь должен аккуратно убедить Орден признать сапиентов Земли, когда придёт время.
Считай то, что произошло, уроком, а не поражением.
Да, кстати, тебе послание от\ldotst сам знаешь кого.

Мы соприкоснулись головами.
В моём разуме зазвучал знакомый слог, и на душе потеплело.

"--*Друзья по ту сторону баррикад\ldotsq

"--*\ldotst нужны всегда, "--- закончила Анкарьяль.
"--- Друзья, а не агенты.
Надеюсь, ты это усвоил.

Я оглянулся на Чханэ и Митриса.
Они ждали.
Чханэ когда-то говорила, что ждала меня вечность.
Забавная гипербола, учитывая, что ей тогда было всего сорок дождей\ldotst
Теперь я понимал, что это не поэзия, а тонкая, неочевидная истина.
Вселенная расширялась, кварки собирались в триплеты, фермионы склеивались в атомы, пыль собиралась в облака и загоралась звёздами, проклетки эволюционировали в первых людей, первые люди создали хоргетов, дали начало народам Драконьей Пустоши и Тси-Ди, а она всё это время ждала, пока детали мозаики мироздания не сложатся в нужную картину.
И всё это для того, чтобы через миг совместного пути, после лёгкого соприкосновения рук снова распасться на первородную квантовую пыль.

"--*Так, я вас оставлю, "--- вдруг сказала Стигма.
"--- Меня вызывают по резервному каналу.
Прощай, Аркадиу.
Надеюсь, что нам больше не придётся видеться при таких обстоятельствах.

Стигма кивнула коротко остриженной головой, вскочила на мотоцикл и уехала, поднимая лёгкие клубы песчаной пыли.

Грейсвольд бросил взгляд ей вслед и улыбнулся подошедшему Атрису.

"--*Мы "--- это ветер, "--- сказал Атрис.
"--- Мы будем бросать пылинки на гранитную скалу, микрона за микроной стачивая её.
Вы "--- это вода.
Вы будете проникать в мельчайшие трещинки скалы, замерзать и таять, расширяя их.
Однажды на этой скале смогут расти цветы, и она превратится в зелёный холм.
Так оживала и моя планета, и я тому свидетель.

"--*Жаль, дружище, что она больше не твоя, "--- посочувствовал Грейсвольд.

"--*А Лотос твой, Грейс? "--- хитро улыбнулся Атрис.

Грейсвольд смутился.

"--*Я лишь сотворил Лотос.

"--*И этого уже не изменить ни одному завоевателю, "--- закончил Атрис.
"--- Чтобы изгнать из горшка горшечника, нужно стереть горшок в пыль.

"--*Когда-то нам поклонялись, "--- сказал Грейсвольд, "--- от нас ждали благ, справедливости и спасения.
Знали бы наши верующие, как мы с тобой слабы перед системой.

"--*Верно, "--- согласился Атрис.
"--- Впрочем, на молитвы я не отвечал уже давно, многие из моих созданий поумнели и рассчитывают только на себя.
Лучший бог "--- тот, о существовании которого даже не догадываются.
Тот же, кто требует почестей, молод и глуповат.

Грейсвольд оглянулся ещё раз и снова вполголоса обратился к Атрису:

"--*Атрис, ты достал кольцевую теплицу?

"--*Нет, "--- сказал Атрис.
"--- А ты, технолог, вспомнил схему Золотого города?

"--*Нет, "--- с той же интонацией ответил Грейс.
Секунда напряжённого молчания "--- и оба от души расхохотались.
Впрочем, смех утих почти сразу.
Мужчины долго смотрели друг на друга.

"--*Если к концу войны мы\ldotst разминемся, "--- голос технолога дрогнул, "--- заглядывай на Лотос.
Нужно время от времени поправлять орбиту Часовой Луны, она немного сползает.
Водопад, Факел и Паутина-город требуют реставрации, людишки отбивают кусочки от фокусирующих кристаллов и делают из них ожерелья.
Объясни им, пожалуйста, что так делать непорядочно и следует уважать чужой труд.
В остальном, думаю, ты разберёшься и сделаешь даже лучше, чем было.
Ты хороший бог.

Атрис кивнул.

Пора было уходить.

Я по очереди пожал руки Грейсу и Анкарьяль.
Расстались мы в полном молчании.
Чханэ и Атрис взяли меня за талию с двух сторон и повели на север.
Маленькая Митхэ шла рядом, окуная ступни в набегающую солёную пену.
На её лице впервые за долгое время появилась счастливая улыбка.

"--*Так клетки всё-таки у тебя, "--- сказала она.
"--- Ты успел\ldotst

"--*Я всегда тебе верил, "--- пожал плечами Атрис.
"--- Ты сказала <<Сейчас или никогда>>.
И в последний миг прилетело несколько клеток с сорванного ветром листа.

"--*Меня терзает чувство вины из-за несчастного создания, которому я дала ложную надежду.
Эти тянущиеся ко мне руки я буду помнить до нашего последнего мгновения.

"--*Надежда не ложная, "--- сказал Атрис.
"--- Мы вернёмся, и ты сама дашь ей свободу.
Теплицы живут очень долго.
Наверное, сейчас она спит и видит счастливые сны.

"--*Ты думаешь, мы сможем отвоевать Тси-Ди у Машины? "--- спросила Чханэ.

"--*Может быть, мы сможем с ней договориться, "--- предположил я.
"--- Твоей искренности, Чханэ, поверил даже Лусафейру.
Поверит и Машина.

"--*Если получится, я буду петь моему деревцу каждый день, "--- пообещала Митхэ.
Атрис кивнул.

Я ещё раз повторил про себя слова послания Лусафейру:

\begin{quote}
<<Сражение закончено, но борьба продолжается.
Я не буду поддаваться.
Это сделает тебя слабее, чем ты можешь быть.
Я оставил тебя в живых "--- неплохая фора, на мой взгляд>>.
\end{quote}

<<Однажды, мой друг, мы встретимся как равные, и я пожму твою руку>>, "--- таким был ответ, который я передал Грейсвольду.
Разумеется, Лусафейру может его не получить "--- конспирация.
Но, возможно, когда-нибудь\ldotst

Уже отойдя на десяток шагов, я услышал тихую речь Анкарьяль:

"--*Грейс, все схемы Золотого города у тебя?
Ведь я права?

"--*Не понимаю, о чём ты, "--- отмахнулся старый технолог, говоря намного громче, чем следовало бы.
"--- Нет у меня никаких схем.

И уже тише добавил:

"--*Пожалуйста, хватит стирать себе память каждый раз, когда до тебя доходит.
От отдела 100 меня это не защитит, да и ты перестанешь надоедать с этим вопросом.

"--*Эээ\ldotst каким вопросом? "--- удивилась Анкарьяль.

Грейсвольд громко захохотал.
Я оглянулся и увидел, как он прижал крякнувшую от неожиданности подругу к могучей груди.
К городу они пошли пешком, обнявшись, и их следы на мокром песке "--- неуклюжую цепочку больших и маленьких ступней "--- смыла набежавшая ласковая волна.

\chapter{Скорбящие}

\section{[U] Ключ к себе}

\spacing

"--*Кагуя, вы принадлежите к клану Тахиро, "--- сказал Грейсвольд.
"--- Вы полагаете, что две генерации и десять тысяч версионных правок сделали из вас урождённого демона?
Ваше ядро было и остаётся тенью человеческого мозга.

"--*Я не настолько глупа, чтобы оправдывать нечто только из-за того, что это нечто свойственно мне, "--- сладким голосом пропела Кагуя.
"--- А если\ldotst

"--*Уэсиба Серозмей, "--- тихо шепнул Грейсвольд.
Лицо Кагуя залила краска, глаза сузились от гнева.

"--*Моего брата от меня отделяют три тысячи версионных правок.

"--*А от Тахиро вас отделяет и того больше, но вы такая же наглая и упрямая, как мой друг, "--- сказал Грейсвольд.
"--- Вы можете не любить Тахиро, вы можете не любить меня, но я всегда буду любить его в вас.

"--*Любовь?
Я не понимаю, о чём вы говорите, "--- Кагуя со скучающим видом покачала головой.

"--*Значит, однажды Вселенная преподнесёт вам сюрприз, "--- улыбнулся Грейсвольд.

Кагуя пожевала губу.

"--*Что вам от меня нужно?

"--*Мне нужно знать, о чём с вами говорил известный вам с Самаолу демон.
Кажется, его называют <<патроном>>.

"--*Вы в шаге от того, чтобы подписать себе приговор, Грейсвольд, "--- ощерилась Кагуя.
"--- Отдел 100 сочтёт это превышением\ldotst

"--*Не будьте дурой, Кагуя, "--- устало сказал технолог.
"--- Да, я из Скорбящих, и пожалуйста, покажите мне того, кто об этом хотя бы не подозревает.
Истинное равновесие Нэша достигнуто, вы это знаете.
Вы не можете причинить вреда мне, не погибнув.

"--*Ложь, "--- бросила Кагуя.
"--- Какое равновесие?
Мы теряем позиции в битве с Картелем, наши демоны гибнут\ldotst

\mulang{$0$}
{"--*Кто гибнет? "--- грустно улыбнулся Грейсвольд.}
{``But who tends to die?'' Grejsvolt sadly smiled.}
\mulang{$0$}
{"--- Молодёжь, не нашедшая своего места, жители планет, которые уже даже не ресурс для выживания, а топливо для войны!}
{``Young who failed to find their place, and dwellers of planets, not as vital resource, but as fuel for war!}
\mulang{$0$}
{Дефицита масс-энергии давно нет.}
{Mass-energy deficiency is long gone.}
Старые иерархи как сидели, так и сидят на своих местах, и их никто не пытается устранить.
Те же, кто пытался что-то изменить в этом миропорядке, уже давно разделили судьбу вашего прародителя.

"--*Информация не способна поколебать истинное равновесие Нэша, "--- заметила Кагуя и притворно вздохнула.
"--- Ну что это такое, я теперь играю в вашу игру!

"--*Поколебать равновесие можно, "--- сказал Грейсвольд.
"--- Но для этого нужна жертва.

Кагуя замолчала.
Она выглядела потрясённой до глубины души.

"--*В мире есть лишь одна валюта, "--- продолжил Грейсвольд.
"--- Эта валюта "--- чувство удовлетворённости.
Её мера "--- условная единица <<хорошо>>, её количество "--- интенсивность в процентах, интегрированная по времени.

"--*Валюта? "--- захохотала Кагуя.
"--- Вы циник, Грейсвольд.
Есть вещи, которые нельзя купить, и одна из них\ldotst

"--*Цену имеет любая вещь, "--- перебил её Грейсвольд.
"--- Вашу верность нельзя купить за обычную валюту, но за удовлетворённость её покупали и продавали уже много раз.

"--*Да как вы смеете! "--- взвизгнула Кагуя.
"--- Иногда мне приходилось жертвовать своими интересами во имя\ldotst

Женщина осеклась.

"--*Это правда, "--- подтвердил Грейсвольд.
"--- Любое явление, любой предмет во Вселенной конвертируется в эту валюту, но обменный курс индивидуален для каждого.
Смысл существования каждого сапиента сводится лишь к вычислению этого курса.
Иногда тарелка супа и ночь спокойного сна для детей оказывается дороже жизни целого народа, но бывает и наоборот.

"--*Для сапиентов "--- может быть.

"--*Демоны "--- тоже сапиенты.
Мы отличаемся от материальных тел составом, но не принципами функционирования.

"--*Будь по-вашему.
Но я что-то не могу придумать ни одной причины выставить нашему сотрудничеству высокий обменный курс.

"--*Одну я вам назову, "--- сказал Грейсвольд.
"--- Ключ к пониманию частицы себя.

\spacing

\section{[?] Сюзерен}

\spacing

Прекрасное создание размером с дом сидело на скале и смотрело на меня странными, собранными из множества фасеток глазами.
Не зная языка мимики сюзеренов, не зная о них практически ничего, я каким-то образом понял, что огромный дракон ждёт именно меня.

Мы долго смотрели друг на друга.
Сюзерен тихо, с неправдоподобно младенческим сопением вдыхал и выдыхал морозный воздух.
Ласково позванивали чешуйчатые <<перья>> на крыльях, издалека напоминающие стрекозиные крылышки или тонкие перламутровые пластинки.
Маленькие, похожие на женские руки, покрытые мелкой чешуёй, лениво шевелили тонкими пальцами.
Мои ноги в меховых сапогах начали потихоньку подмерзать, и я лихорадочно думал, как же коммуницировать с глядящим на меня крылатым ящером.

"--*Человек, "--- внезапно сильным и чистым голосом сказал сюзерен.

Я застыл.
Холод прошёл в одно мгновение.

"--*Ты говоришь на талино? "--- поинтересовался я.

"--*Когда-то давно я слышал ваш язык.
Мне потребовался день, чтобы выучить его.
С какой целью ты позвал меня?

"--*Я тебя не звал.

Мне стало немного жутко.
Существо, которое за один день выучило язык талино по обрывкам фраз, которое знало, что я приду именно сегодня, именно сюда\ldotst
Я посмотрел на огромную красивую голову.
Мозг превосходил человеческий по размерам как минимум в десять раз.
Какие ещё силы вложил в него девиантный бог?

Вспомнив цель визита, я начал:

"--*Хоргеты Ордена Преисподней посылают племени сюзеренов предложение о дружбе и сотруд\ldotst

"--*Хоргеты? "--- внезапно засмеялся сюзерен.
"--- Когда кто-то говорит <<хоргеты>>, мы ожидаем увидеть равного Богине-матери, а не оцифрованного человека.

Я снова застыл, скованный шоком.

"--*Чего ты хочешь? "--- продолжал дракон.
"--- Люди не трогают нас, мы не трогаем людей.

"--*Когда-то давно между людьми и сюзеренами была война.
Я не могу говорить от имени всех, но возможно, что сотрудничество позволит\ldotst

"--*Не позволит, "--- прервал меня дракон.
"--- Ненависть людей "--- лишь песни и сказания.
В нас ненависти нет, "--- сюзерен неожиданно поднял огромное крыло, обнажив покорёженную, толстую кожу.
Чешуя на ней росла как попало.
Старый шрам.
"--- Вы же не обижаетесь на пчёл, когда они вас жалят?

"--*И всё же\ldotst

"--*Мы знаем людей.
Они будут ненавидеть нас, хотя никто из ныне живущих не пострадал от сюзерена.
Кажется, это называется <<традиции>>?

"--*Хорошо, хватит о людях.
Почему ты не хочешь поговорить о возможности союза с Орденом Преисподней?

"--*Почему ты думаешь, что можешь обмануть меня, человек? "--- ответил вопросом на вопрос дракон.

И снова мне пришлось сделать усилие, чтобы справиться со страхом.
Тон был подобран великолепно, и означал он следующее: мой собеседник прекрасно осведомлён о том, как распоряжается Орден Преисподней информацией о сапиентах.

"--*Нечего сказать, Аркадиу из рода Шакалов? "--- усмехнулся сюзерен.
"--- Ты чересчур честен.
Плохой из тебя эмиссар.

"--*Объясни мне, "--- проворчал я.
"--- Ты, материальное существо, которое превосходит многих хоргетов интеллектом.
Почему вы позволили людям вас уничтожить?

"--*Иногда самое лучшее "--- это позволить всем думать, что мы уничтожены, "--- сообщил сюзерен, изящно расправив и сложив крыло.
"--- Сколько голов сюзеренов ты видел?

Я промолчал.
Реально задокументированных останков этих существ было всего восемь.
Правда, историки списывали это на время и на обычай поедать мертвецов, который описывался в одной из легенд\ldotst

"--*Если ты хочешь спросить, почему та война закончилась, я отвечу, что мы ушли от войны сами.

"--*Отговорка униженных, "--- бросил я.
"--- Люди хотели извести вас под корень.
С чего вам проявлять миролюбие?

"--*Неограниченное стремление к процветанию рода "--- поведение низших животных, "--- невозмутимо ответил дракон.
"--- Однажды место и еда кончаются.
И родичи становятся врагами.

"--*Если вы надеялись ударить во время всемирной войны, вы немного опоздали.

"--*Нам не нужна война, человек, "--- дракон даже не возвысил голоса.
"--- Первые люди вели себя подобно низшим животным, играющим с технологиями.
Во время войны с нами их технологии были утрачены.

"--*Вы хотели загнать людей в каменный век?

"--*Обезвредить, "--- поправил сюзерен.
"--- Отсеять наиболее агрессивных, лишить их оружия.
Планета должна продолжать жить.

"--*Так что насчёт\ldotst

"--*Ордену Преисподней я отвечать не собираюсь.
Мы уже послали им своё слово: верните нам Богиню-мать.

"--*Или\ldotsq

"--*Без <<или>>.
Хотят сотрудничества "--- пусть говорят через Богиню-мать или в присутствии Богини-матери.
Хотят мира "--- да будет мир.
Хотят нас уничтожить "--- это будет интересная игра, и мы желаем им удачи.
С тобой лично, человек, и твоим племенем я поговорю лет через пятьсот, когда вы наберётесь ума.

Дракон поднялся на ноги и раскрыл крылья, собираясь взлететь.
Меня обдало холодным, инкрустированным снежными иглами ветром.

"--*Почему вы не уничтожили людей? "--- спросил я его напоследок.
"--- Они захватчики, чужаки на вашей планете!

Дракон засмеялся.

"--*Потому что мы знали, что однажды люди захотят с нами поговорить.

Мощные крылья сделали первый взмах.
Веером раскрылись перламутрово-стрекозиные перья хвоста, и сюзерен, сделав величественный круг, улетел в ослепительные дали залитых солнцем вершин.

\razd

Спустя три года после этого разговора Кох исчезла c Капитула.
Ещё через три года была проведена тотальная мелиорация Драконьей Пустоши.
Интерфекторы отдела 125 говорили, что это было единственное интересное задание за многие тысячелетия;
сюзерены же остались лишь в генетических банках Ордена Преисподней.

\section{[:] Плач по Тахиро}

\spacing

\begin{quote}
<<Тахиро, мой любимый Тахиро, который знал меня лучше, чем я сама, ушёл.
Ушёл как герой, неожиданно и не оставив надежды, передав мне Вселенную холодной и полной опасностей.
Ушёл, не успев сделать ничего, что могло бы облегчить участь песчинки в океане пространства-времени\ldotst>>
\end{quote}

Записи Айну рыдали, хотя Грейсвольд никак не мог представить Айну в слезах.
Весь его опыт общения с ней восставал против такого образа.

Грейсвольд вспомнил и другие слова.
Высокопарные слова, гремевшие в огромном зале.
Тот зал был поистине гигантским, перечёркнутый паутиной нервюр потолок (каждая <<паутинка>> на самом деле была балкой толщиной с тысячелетнее древо) казался едва ли не выше небес.
Демоны прекрасно усвоили социальные понятия обезьян "--- чья ветка выше, тот и главный.

\begin{quote}
<<Величайший стратег Ада пал смертью храбрых из-за вероломства кучки сейхмар\ldotst>>
\end{quote}

\begin{quote}
<<Он должен быть отмщён!>>
\end{quote}

Самаолу говорил искусно поставленным голосом.
Даже солоноватых водянистых соплей в этом голосе было ровно столько, сколько нужно "--- их отмеряли на сверхточных квантовых весах.
Аплодисменты в конце речи казались идеальными в своей мощи, хаотичности и внешней непринуждённости.
\mulang{$0$}
{Но Грейсвольду было противно от мысли, что Самаолу и половина его слушателей давно искали повод прихлопнуть Тси-Ди, как спаривающихся мух.}
{But Grejsvolt hated even the thought that Samajolu and a half of his audience were always looking for an excuse to swat Qi-Di like mating flies.}
\mulang{$0$}
{Ещё противнее становилось от того, что этим поводом оказалась смерть Тахиро "--- мальчика Тахиро, о котором Грейс уже давно не думал, как о мальчике.}
{The only thing he hated more was another thought: the same excuse was death of Tahiro-kid; Grejsvolt didn't thought of Tajiro as a kid for ages.}

"--*А ты, похоже, не особо грустишь, Грейсвольд, "--- заметил Самаолу после речи.
"--- Ищещь способ, как обратить его гибель на пользу \emph{себе}?

\mulang{$0$}
{"--*Молись кому хочешь, Самаолу, "--- тихо ответил Грейсвольд, "--- чтобы ни одна капля скорби старого Грейса не упала на твою голову.}
{``Pray to whatever you want, Samajolu,'' Grejsvolt answered quietly, ``that no one drop of Old Grejs' sorrow falls on your head.''}

\section{[U] Камень (переработать)}

Атрис вскоре вернулся.

"--*Как результаты? — поинтересовалась Чханэ.
"--- Надеюсь, ты не наследил.

Атрис едва заметно улыбнулся на колкость воительницы.
Митхэ выглядела слегка обиженной.

"--*Мне не удалось проникнуть ниже барьерной высоты.
Плюс форпостов Ада у Тси-Ди не два, а одиннадцать, "--- Атрис кивнул Анкарьяль.
"--- Но результаты есть.
Я достал отладочную информацию защитной системы.
Может быть, Грейс что-то сможет с ней сделать?

Атрис махнул рукой.
Грейс выпучил глаза.

"--*Как ты это достал?

Атрис усмехнулся.

"--*У одного из спутников Тси-Ди отказал двигатель, и он отклонился от заданной орбиты.
В результате крайняя точка его орбиты находилась в каких-то десяти километрах от барьерной высоты.
Митхэ предложила\ldotst хай\ldotst кидать в спутник камнями.

Митхэ смущённо сидела, ковыряя сапогом землю.
Анкарьяль и Грейсвольд ошеломлённо переглянулись.
Грейсвольд усмехнулся.

"--*Мы вывели спутник за барьерную высоту, периодически пересиживая адские патрули.
Последним камушком мы задержали его на девять минут тридцать одну секунду.
Одна секунда мне потребовалась, чтобы разобраться с аппаратурой, оставшееся время я сниффил трафик.

"--*Патрули?
Они заметили вас? "--- Анкарьяль, казалось, едва сдерживалась, чтобы не схватить менестреля за рубаху и не вытрясти из него душу.

"--*Всё чисто.
Патруль появился через три секунды после взрыва.

"--*Какого взрыва? "--- хором спросили Чханэ и я.

"--*Ну не мог же я оставить спутник остальным! "--- пожал плечами Атрис.
"--- Сымитировал взрыв двигателя.
Значимых обломков не осталось.

Все замолчали.

"--*Знаешь что, Атрис, "--- начал Грейс, "--- тси должны благословить тот день, когда они встретили тебя!

"--*Перестань, "--- отмахнулся Атрис.

"--*А почему наши до сих пор не додумались камнями спутники вывести? "--- задал я вопрос.

Грейсвольд крякнул.

"--*Они мыслят рамочно, Аркадиу.
У них есть набор техсредств "--- им и пользуются.
Я уверен, всё это время они пытались вывести спутник полями, подобраться к системам на различных волнах.
И это при том, что именно от сторонних волновых воздействий спутники и защищены.
А вот про камушки тси сами не подумали.
Ну и сломавшийся двигатель "--- тоже большая удача.
Знаете же, почему меня зовут Каменным Молотом?

Все отрицательно покачали головой.
Анкарьяль прикрыла лицо рукой.

"--*Старая песня\ldotst

"--*Старая, но важная, Нар.
В работе нельзя брезговать никакими технологиями.
Даже камнем на палке.

\section{[U] Танец Тени}

\spacing

Тхартху сидела с закрытыми глазами.
Мелок гулял по пергаменту совершенно независимо от тела.
Послание должны получить все, но\ldotst

ПКВ вокруг вспыхнуло.
Устройство связи умерло одновременно с охранявшими дом демонами.
Тхартху, улыбаясь, нанесла на пергамент последний штрих.

Дверь открылась.
Митрис вскочил и выхватил пистолет.
Его демон привёл боевые модули в готовность.

"--*Здравствуй, Нар, "--- не оборачиваясь, проговорила Тхартху.

Анкарьяль вошла молча и посмотрела на Митриса.
Тот целился женщине в глаз.

"--*Я нарисовала тебя, "--- сказала Тхартху и, обернувшись, протянула Анкарьяль пергамент.

Демоница взяла его.
Улыбающаяся Хатлам ар’Мар держала в руках огромного броненосца.

"--*Очень красиво, Тхартху, "--- кивнула интерфектор.

"--*Делай, что задумала, "--- прошептала Тхартху и по-детски зажмурилась.
Удар "--- и опустевшее тело обвисло на кресле.
Анкарьяль, вытащив пистолет, добила женщину выстрелом в голову.

Митрис вздохнул и заплакал.
Оружие выпало из его руки.

"--*Пойдём, "--- ласково сказала мужчине Анкарьяль, спрятав пергамент в карман.
"--- Ты готов?

Митрис кивнул.
Он ещё раз взглянул на останки подруги, вздохнул, смахнул слёзы и направился к двери.
Интерфекторы ждали снаружи, но только Анкарьяль услышала его последний тихий шёпот:

"--*Благодарю тебя.

\section{[U] Смерть Штрой}

\spacing

На окраине Ихслантхара рос приметный каштан.
У него было пять толстых ветвей, и это мог сказать любой уроженец.
Детьми на него лазали все, а те, кто познал под его кроной радость плотской любви, исчислялись тысячами.

Детское время давно прошло.
Возле каштана сидели трое "--- двое мужчин и оцелотовая женщина, тонкая, с пухлыми губами, почти ребёнок на вид.

"--*Эта явка мне не нравится, "--- заявил один из мужчин.
"--- Может, пойдём куда-нибудь подальше в лес?

"--*Тебя не спросили, "--- властно оборвала его девушка.
"--- Просто делай своё дело.

Мужчина кивнул и, сбросив одежду, прижал девушку к каштану.
Спустя десять михнет запыхавшегося любовника сменил второй.
Ещё через пятнадцать михнет оба взяли свои фонари и удалились в сторону города.
Девушка осталась сидеть одна в темноте, словно о чём-то размышляя.

Вскоре округу огласил её негромкий ликующий смех:

"--*Этот ход остался за мной.
Что ж, можно и выпить.

Тихо звякнула крышка автоматической фляжки, и в чашу полилась шипящая жидкость.

"--*Какая прелесть, "--- захихикала девушка.
"--- Даже звук при открытии идеальный.
Умели же эти выродки делать фляжки\ldotst

Вдруг темноте показались три фигуры.
Девушка вскочила на ноги, едва не расплескав отвар.

"--*Здравствуй, Штрой Кольцо Дыма, "--- приветливо сказала высокая женщина.

"--*Анкарьяль, Мимоза, Ду-Си, "--- склонила голову девушка.
"--- Прошу прощения, я вас не признала.
Не желаете ли выпить?

Мимоза и Ду-Си переглянулись.

"--*Как там говорится в <<Шляпе>>, Мими?
Вежливо ответить на вопрос, да?

Легат и максим повернулись к Штрой и хором сказали:

"--*Благодарим, но вынуждены отказаться.

"--*Мы знаем, что это ты сообщила последней из тси ключи от боевых систем и спровоцировала её на диверсию, "--- сказала Анкарьяль.

Чаша и фляга дробно звякнули о корни каштана.

"--*Благодаря мне была раскрыта шпионская сеть, максим, "--- затараторила Штрой, обращаясь к Мимозе.
"--- Я не могла действовать открыто из-за контаминации наших рядов.
Единственная моя ошибка заключалась в том, что я отнесла подполье к шпионской сети Картеля, а не к отдельной организации.
Я служила Аду и прошу справедливого суда своим деяниям\ldotst

"--*Ад благодарит тебя за верную службу, "--- мягко прервал её интерфектор.

Удар "--- и опустевшее тело девушки упало на землю.

"--*Отчёт в одиннадцать предоставите командующему, "--- бросил Мимоза спутникам.
"--- И ещё, Анкарьяль, могу ли я попросить вас захватить сладости?
Мы планируем выпить отвара.
Очень рекомендую булочки некой Ликхэ ар’Митр э’Кахрахан, пышные и ароматные.
Купите их на вес кукхватра.
Только встаньте в очередь прямо сейчас, а не то к обеду разберут и нам ничего не достанется.

\section{[U] Микоргет (переработать!)}

\spacing

"--*Ты хочешь сказать, что тси пытались создать\ldotst материального микоргета? "--- ошарашенно спросила Чханэ.

Атрис замялся.

"--*Очень похоже на это.
Кольцевая теплица "--- совершенное создание, практически вечное.
Она вбирала в себя также личности всех, кто желал обрести вечную жизнь\ldotst

"--*Предкам была противна даже мысль, что их потомки могут пойти путём хоргетов, "--- сказала Митхэ.
"--- Они решили искать собственный путь бессмертия.
Теперь оставшиеся в живых спят в телах кольцевых теплиц.
И Машина об этом не подозревает "--- возможно, что и эти эксперименты совершали в тайне\ldotst

"--*Машина подмяла нас под себя, "--- сказал Атрис.
Чханэ удивлённо посмотрела на менестреля "--- он впервые сказал <<нас>>, отнеся себя таким образом к тси.
"--- Мы относились к ней, как к бездушному слуге.
Но как только слуга становится умнее хозяина, хозяин превращается в слугу.

"--*Или в труп, "--- присовокупила Чханэ.
"--- Так или иначе, для нас это всё бесполезно.
Нужно найти безопасное место "--- Аркадиу схватили.

"--*Зачем? "--- усмехнулась Митхэ.
"--- Отправимся туда, где сейчас опаснее всего "--- там нас вряд ли будут искать.

Чханэ испытующе посмотрела на Митхэ и спустя мгновение понимающе кивнула.

\chapter*{Очень длинный постскриптум}
\addcontentsline{toc}{chapter}{Очень длинный постскриптум}

\spacing

"--*Ну вот и всё, я закончил, "--- сказал толстяк.

"--*Писатель из тебя, как из амёбы кольцевая теплица, "--- флегматично заметил его визави, прикрыв на секунду глаза.
"--- Сказывается отсутствие культурологической подготовки.
Логика повествования нарушена в четырёх местах, двенадцать фактических ошибок, не укладывающихся в норму литературного искажения.
Последняя глава вообще скомканная.

"--*Иди в дупло, "--- обиделся толстяк.
"--- Я старался копировать стиль Аркадиу.

"--*Не нужно копировать чей-то стиль, "--- возразил собеседник.
"--- Книга твоя, а не Аркадиу.

"--*Хорошо.
Найду кого-нибудь другого, кто посмотрит на смысл, а не на статистику.
Где Митрис?

"--*А я знаю? "--- невозмутимо ответил второй.
"--- Они гулять пошли.

Наступило молчание.
Толстяк сосредоточенно просматривал светящийся голубым текст на голографическом экране.
Нахмурился, задумчиво почесал подбородок и вдруг бросил весёлый взгляд на собеседника.
Тот бесстрастно попыхивал трубкой.

"--*Наслаждаешься отпуском?

"--*Жизнью, Грейс.
Это называется <<жизнь>>.

"--*Надымил здесь.
Знаешь же, что я не люблю.

"--*Иди в дупло, пчёл доить.
Потерпишь.
Я не курил с тех времён, когда Гало переметнулся к Картелю.
Мы любили дымить после тяжёлого дня.
Тело тси, "--- курильщик погладил себя по плечу, "--- по-другому воспринимает никотинаткаллунаин.
Он метаболизируется быстрее, чем я докуриваю трубку.
Можешь собрать такое тело, чтобы меня туманило?

"--*Тси специально\ldotst

"--*Да-да, знаю.
Но я хочу, чтобы меня туманило.
Я всё равно не буду размножаться, обещаю.

"--*Знаю я, как ты не будешь размножаться.
Насеял по моей планете тогда.

"--*Я просто был не в форме.
И не насеял, а провёл полевые испытания навыков социальной коммуникации, "--- курильщик многозначительно указал в потолок.
"--- Ты же мне сам обновления ставил, помнишь?

"--*Каждую ночь в течение двух лет, без контрацепции, под всеми наркотиками, которые только можно найти, "--- покачал головой Грейс.
"--- Ты это называешь <<социальной коммуникацией>>?

"--*Я увлёкся исследованиями, "--- хмуро сказал Лусафейру.
"--- Кстати, их можно считать успешными "--- навыки сработали на ста процентах из более чем тысячи испытуемых.
Целомудренные девы бросались на меня, словно проститутки, непробиваемые стражники с упругими задницами открывали двери гаремов и сами ложились под меня.
И это, заметь, только словом и лаской, никаких химикатов.
Признаю, с гормонами у моего тела было не всё в порядке, но\ldotst

"--*С головой у тебя было не всё в порядке, Лу.
Те навыки я уже испытывал, и\ldotst

"--*И как? "--- с интересом спросил Лу.

"--*Я их не так испытывал! "--- Грейсвольд покраснел, как спелая акхкатрас.
"--- Полевые испытания проводятся аккуратно, чтобы не беспокоить устоявшийся социум.
\mulang{$0$}
{А ты прошёлся по Лотосу, как ураган с членом!}
{But you passed through Lotus like a hurricane with a dick!}
\mulang{$0$}
{О тебе ещё триста лет легенды складывали!}
{I had been hearing legends about you for three hundred years since!''}

\mulang{$0$}
{"--*Расскажешь как-нибудь.}
{``You should tell them sometime.}
\mulang{$0$}
{Аж интересно стало.}
{I'm intrigued.''}

"--*Кстати, мою гордость "--- Водопад, Факел и Паутину-город\footnote
{Знаменитый ландшафтный комплекс планеты Лотос, каждое равноденствие образующий за счёт оптических эффектов гигантский голографический фильм о Древней Земле.\authornote}
"--- ты так и не увидел, потому что, выражаясь культурно, проспал равноденствие.
А про то, что в Гранобле твои любовники и любовницы нас чуть не убили и нам пришлось уходить верхами, я вообще молчу\ldotst

"--*Хорошо, обойдусь, "--- примирительно поднял свободную руку визави.
"--- В адском вереске не алкалоиды главное.
Гало так говорил.

"--*Почему он ушёл?

Курильщик сделал длинную затяжку и помолчал.

"--*Однажды Гало сорвался, застав меня в постели с Айну.
Мы занимались любовью, а потом беседовали\ldotst о чересчур личных вещах.
Айну рассказала мне о своём пробуждении.
Она плакала, хотя это было пять жизней назад.

"--*А что Гало?

"--*Гало сказал, что пришельцы мне дороже брата.
Ясное дело, причина была куда глубже.
Эйраки намеренно для своих целей пробудил в Гало соперничество.
Он хотел, чтобы стратегом стал самый сильный.
Гало стал отдаляться, и его место в моей жизни занял Тахиро.

"--*Но ты не поддался.
Ты не воспринимал Гало как соперника.

"--*Я плевал на отца и его мнение.
Как показало время, правильно делал.

Толстяк ухмыльнулся.

"--*Не жалеешь, что сбежал?

"--*С одной стороны, мне пришлось избавиться от мощных модулей и надстроек.
Я не настолько умён, как раньше.
С другой стороны, сейчас я самый счастливый демон во Вселенной.
Я живу полной жизнью.
И ты, балбес, сидишь рядом со мной, а не за охраной из десяти интерфекторов.
Красота.
Не хватает лишь Тахиро и Айну, но выжить они просто не могли.

Грейсвольд изумлённо посмотрел на друга.
Последняя фраза была заключением стратега.

"--*Тахиро и Айну были неразборчивы в средствах, "--- объяснил Лусафейру.
"--- Сказалась деспотичность общества, в котором они жили.
Такие личности часто добиваются славы, но редко "--- любви.

"--*Ты тоже иногда этим грешил, "--- заметил Грейсвольд.

"--*Я всегда знал границы и ценил жизни демонов, "--- возразил Лусафейру.

"--*Перекладывая ответственность на других, "--- подхватил Грейсвольд.
"--- К примеру, на того же Тахиро.

"--*Да, перекладывая ответственность за чужие жизни на их обладателей, "--- ничуть не смутившись, подтвердил Лусафейру.
"--- По-твоему, это плохо?
Я никого не посылал на самоубийственные операции.
Тахиро пожертвовал собой сам.
Да и у вас был выбор "--- идти на Тра-Ренкхаль или не идти.

"--*Складно поёшь.

"--*Ещё бы.
Всё это не принесло славы, но зато у меня есть отличный вересковый табак.
Этого Тахиро уже не видать.
Хотя я бы отдал всю свою табакерку, чтобы он ещё раз со мной поговорил.

"--*Низко же ты ценишь своего друга, "--- поморщился Грейсвольд.

"--*Я бы выкупил беседу за табак, но не продал бы её за ту же цену.
Я нищий, Грейс.

"--*Да, Ордену мы отдали чересчур много, "--- согласился технолог.
"--- Никогда не поздно начать новую жизнь, но её цена растёт со временем.

"--*Я за свою рассчитался, с твоей помощью, "--- ухмыльнулся Лусафейру.
"--- Я люблю тебя, Грейсвольд Каменный Молот.
Будь благословен тот час, когда тебя забросило на Преисподнюю.

"--*Ещё не рассчитался, друг, "--- грустно заметил Грейсвольд.
"--- Тебя по-прежнему ищут головорезы обеих фракций.
Так что сиди здесь и не высовывайся.

"--*Больно надо, "--- поморщился Лусафейру и сделал глубокую затяжку.
"--- Кстати, в этом простом факте "--- наше с тобой отличие от того же Тахиро.
Тахиро чересчур буквально понимал слово <<жертва>>.
Но что поделаешь, Тахиро был воином, а воины "--- априори самоубийцы.

"--*Не совсем понимаю, что ты\ldotst

"--*Да всё ты понял, Грейс, "--- нетерпеливо сказал Лу.
"--- Необязательно умирать, чтобы пошатнуть истинное равновесие Нэша.
Достаточно было пожертвовать своим положением, как это сделали мы с тобой.

"--*Аркадиу\ldotst

"--*А у Аркадиу не было другого выбора.
К тому времени, как он прозрел, его место в системе ещё не было определено.
Путь ветра всегда тяжелее пути воды.
Кстати, насчёт рукопожатия ты выдумал.

"--*Нет, это правда.
Просто Аркадиу погиб до того, как я получил возможность с тобой поговорить, а после его смерти слова потеряли смысл.

Второй хмыкнул.

"--*Шакал придавал этому жесту чересчур большое значение.

"--*В его родном мире рукопожатие находилось под запретом, "--- объяснил толстяк.
"--- Религиозные деятели говорили, что те, кто пожимает друг другу ладони, выказывают преступное пренебрежение перед высшими силами и обществом, ведут себя так, словно они одни во Вселенной.
В качестве приветствия между равными в обычае был <<жест аиста>>, который служил одновременно приветствием, гоноративом для короля и мольбой к Великому.
Рабы кланялись господам, господа в виде особой милости показывали рабам <<водопад>>.
Рукопожатие было жестом разбойников, повстанцев-крестьян и прочих, кто отрицал власть Великого и королей.

"--*Я не знал, "--- констатировал собеседник.
"--- Мог бы и осветить этот момент в книге.
Жаль, что он не успел пожать мне руку.

"--*По сравнению с тем, что он успел, это сущие пустяки.
Не всё бывает так, как мы хотим, Лу.

"--*Ты это мне говоришь, Грейс? "--- Лу выпустил огромное кольцо из дыма и тут же вдохнул его носом.
"--- Я могу предоставить многотысячелетнюю статистику по вещам, которые происходили не так, как я хочу.
С подробным анализом.

Технолог хихикнул и задумчиво замолчал.

"--*Если не терпится показать Митрису, спроси у того, кто знает его местонахождение.

"--*И то верно.
Солнышко, не занята? "--- обратился Грейсвольд к компьютеру.

"--*Нет, Грейсвольд Каменный Молот.
Профилактика систем завершена тридцать секунд назад, "--- ответил из ниоткуда приятный женский голос.

"--*Скажи, где сейчас Митрис.

"--*Модуль <<Язык>>, информация: 97\% вероятности, что в данном контексте речь идёт о Митрисе Безымянном.
Ответ системы: в данный момент объект находится в двух километрах от станции G-0F города 10, азимут 0.335, "--- сообщил голос.
"--- Грейсвольд Каменный Молот, мне передать ему сообщение?

"--*Не-не, пусть гуляют, "--- сказал технолог.
"--- Не к спеху.
Спасибо большое.

"--*Надо поработать над её адаптационными обновлениями, "--- заметил Лу.
"--- Она по-прежнему считает нужным вставлять техническую информацию модулей в сообщение.

"--*А мне нравится, "--- ухмыльнулся Грейс.
"--- Это придаёт ей некоторый шарм.

"--*Потому она их и вставляет, что тебе нравится.
Она же не слепая!
Вы друг на друга просто запали.

"--*Я люблю женщин с коэффициентом интеллекта выше сорока тысяч.

"--*Модуль <<Культура>>, раздел <<Юмор>>, информация: идентифицирована шутка, культурный контекст B8-01-663, Древняя Земля.
Ответ системы: Грейсвольд Каменный Молот, после того, сколько обновлений ты мне поставил, ты обязан на мне жениться.

Грейсвольд захохотал, и даже Лусафейру не удержался от улыбки.

\mulang{$0$}
{"--*Сечёт, "--- признал стратег.}
{``She's hip,'' the strategist admitted.}
"--- Может быть, стоит сделать для неё сапиентное тело?

"--*Модуль <<Аналитика>>, информация: предложение обработано, "--- отозвался голос.
"--- Ответ системы: ощутить вас руками, увидеть вас глазами "--- это интересный опыт.

"--*Договорились, "--- улыбнулся Грейсвольд.
"--- Омега-модуль я тебе уже делаю, теперь ещё и тело будет.
Может быть, даже несколько.
Будешь одновременно хоргетом-демиургом, электронным устройством и толпой народа.

\mulang{$0$}
{"--*А ты знаешь толк в извращениях, "--- хихикнул курильщик.}
{``You're definitely sophisticated in pervertions,'' the smoker chuckled.}

"--*Заодно попробуешь мою стряпню и скажешь, что она великолепна.
Лу не оценил, гурман эдакий.

"--*Да ты готовишь так же, как пишешь, "--- поморщился Лусафейру.
"--- Даже Атрис знает, что приправлять северным сбором яичницу "--- это моветон.

"--*А Митхэ сказала, что это вкусно!

"--*Митхэ под соус полено съест и не поморщится.
Дикое дитя джунглей.

Грейсвольд захохотал.

В комнату вошла маленькая черноглазая женщина.
Друзья вопросительно повернулись к ней.

"--*Вышла подышать свежим воздухом, "--- пояснила она.
"--- Услышала ваш смех.
Что-то произошло?

"--*Вот, Харата, "--- махнул Грейсвольд.
"--- Лу сказал, что я плохо готовлю.
Скажи, тебе понравилось?

"--*Если ты про последний кулинарный эксперимент\ldotst приправы не совсем подходили к яичнице, "--- развела руками демоница.
"--- Я бы добавила смесь перцев.

"--*Вы сговорились, что ли? "--- обиделся технолог.
"--- Лу, ты её подкупил.

"--*Чем? "--- резонно спросил Лусафейру.
"--- И ещё.
Я так и не понял, что в яичнице делали мухоморы.
Мимоза сказал, что у него от такого брат умер, а Стигма просто отказалась это есть.

"--*Это не простые мухоморы.
Они на вкус как пирожные и совсем не ядовитые!

"--*Именно, Грейс!
Мы наконец-то подходим к сути дела!
Ты не считаешь, что пирожные слабо сочетаются с яичницей?
Ты бы ещё сено туда покрошил, кулинар.

Харата нерешительно потопталась на месте, глядя на друзей.

"--*Вы беседуете каждый день и никак не можете наговориться, "--- сказала она.
"--- Это удивительно.

"--*Мы чересчур долго не могли поговорить, Харата, "--- улыбнулся Грейсвольд.

Лусафейру продолжал смотреть на демоницу оценивающим немигающим взором.

"--*Я пойду, "--- сказала Харата и приоткрыла дверь.

"--*Останься, "--- поднял руку Лусафейру.
"--- Ты хочешь побыть с нами.

"--*Я не имею доступа к тому каналу, который связывает вас, "--- пригорюнилась Харата.

"--*Зато ты можешь построить с каждым из нас свой канал коммуникации, "--- сказал Лусафейру.
"--- Для этого просто нужно проводить с нами время.
Садись.

Лусафейру изящно встал с кресла и, указав на него ладонью, плюхнулся под часами.

"--*Ты не должен сидеть на полу, максим, "--- укоризненно сказала Харата.

"--*Оставь эти обезьяньи предрассудки, "--- осклабился Лусафейру.
"--- Высота подставки не отражает значимость задницы.

Демоница, не желая более спорить, заняла кресло стратега.
Лусафейру начал снова набивать трубку.

"--*Когда я был молодым, я мечтал о зале-диване, "--- сообщил он как будто между прочим.
"--- Особенно здорово смотрелся бы зал-диван для судебных заседаний.
Советники, судья и подсудимые валяются на полу, глотают прохладную цолу\footnote
{Цола "--- тонизирующий напиток Капитула из трав. \authornote}
и лениво обсуждают дело.
Я бы нарушил закон только ради этого.

Лусафейру театрально развалился и раскинул руки, рассыпав тлеющие угольки по гладкому полимеру.
Затем стратег заговорил, искусно изменяя голос от нежного женского сопрано до хриплого мужского баса.

"--*<<Что будем с тобой делать, Лу?>>
"--- <<А я знаю?
Передай конфеты.
И цолы плесни, женщина, у меня в кружке пусто>>.
"--- <<Подсудимый, согласно закону .01, подпункт 14, вы не имеете права требовать у советника, чтобы он наливал вам цолы.
Сам подошёл и взял>>.

Харата захохотала и похлопала "--- ровно четыре раза.

"--*В тебе умер великий актёр, "--- улыбнулся Грейсвольд.

"--*Я не хоспис, чтобы во мне кто-то умирал, "--- парировал стратег.
"--- Кстати, о смертях.
Ты действительно решил рассказать во всеуслышание, что на тебе лежит вина за Катаклизм Тси-Ди?

Толстяк замолчал и погрустнел.
Разумеется, он ждал этого вопроса.

"--*Я не могу больше жить с этим, "--- наконец сказал он.
"--- Я не надеюсь на прощение.
Если тси захотят моего изгнания "--- это будет чересчур мягким наказанием, и я приму его с благодарностью.

"--*Не захотят, "--- вдруг сказала Харата.

Грейсвольд вопросительно посмотрел на женщину.

"--*Они знают.
Все до единого.
Лусафейру рассказал им во время твоего отбытия на Сцелаю.

"--*И они позволят мне остаться?

"--*Они уже позволили тебе остаться, "--- пожал плечами стратег.
"--- Вспомни беседы с друзьями, советы.
Разве хоть кто-то проявил к тебе недружелюбие?

Технолог опустил голову и закрыл лицо руками.

"--*Перестань.
Это дело двадцатитысячелетней давности.
Все совершают ошибки\ldotst
Впрочем, глядя на Вселенную сейчас, я даже не могу считать это ошибкой.

Грейсвольд промолчал.

"--*Ты давно стал среди тси своим, Грейс.
К тому же, как говорила твоя давняя подруга, любому, даже дважды и десять раз дважды кутрапу, однажды может потребоваться шанс.

"--*Я перед тобой\ldotst

"--*Эй, иди в дупло, а?
Ты передо мной, я перед тобой.
Мы не ростовщики, чтобы друг с друга проценты трясти, "--- Лусафейру потянул дым из мундштука и, выругавшись, закашлялся от попавшего в горло пепла.
"--- Какую пользу принесло бы твоё изгнание?
Оно уничтожило бы тебя и лишило тси лучшего друга, товарища и советчика.
И ради чего?
Ради торжества правосудия?

Грейсвольд тихо рассмеялся.

"--*Кстати, "--- Лусафейру театрально повернулся к Харате, "--- приятно видеть на лице минус-демона улыбку.
Но и непривычно.

Демоница мило улыбнулась.

"--*Это тело не доставляет мне неудобств.

"--*Но и питания от него никакого, "--- заключил Лусафейру.
"--- Кстати, я бы на твоём месте срезал листья на шее.

Харата ахнула и начала судорожно ощупывать шею.

"--*Опять выросли?

Грейсвольд и Лусафейру переглянулись.

"--*Как-нибудь почитай, что такое <<юмор>>, "--- посоветовал технолог.
"--- Харата, перестань, нет там листьев.
И вообще скоро мы выведем минус-сапиентов, и ты сможешь здесь обосноваться.
Потерпишь?

"--*Куда мне деваться, "--- фыркнула Харата, "--- я же этими исследованиями руковожу!

"--*Однажды все сапиенты известных планет станут единой семьёй и смогут двигаться дальше, "--- сказал Лусафейру.
"--- Разумеется, их будут ждать новые трудности, новые испытания\ldotst

"--*Мы не сможем быть с ними вечно, "--- заметил Грейсвольд.

"--*Помнишь, Грейс? "--- вдруг понизил голос Лусафейру.
"--- Мы видели блуждающие огни.
Они летели с огромной скоростью, мы пытались с ними побеседовать, но не получили ответа.
Харата, мы с Грейсом видели стабильные источники возмущений ПКВ в пустоте, и это были не хоргеты!
Понимаешь\ldotsq

Демоница ахнула.

"--*Ты говоришь о\ldotst

"--*Он говорит о теории Вселенных, расположенных в одном и том же пространстве.
Опять, "--- поморщился технолог.
"--- Да, существование множества не взаимодействующих друг с другом квантовых связок объясняет колебание ПКВ.
Но для нас эта информация бесполезна.
Мы можем взаимодействовать со Вселенной фотона потому, что нас создали и обучили её жители.
Кто обучит нас взаимодействовать с прочими?

"--*Их сапиенты, разумеется, "--- ответил Лусафейру.
"--- Мы найдём их, мы научимся с ними беседовать.
Да, нас создали в этой Вселенной, но мы можем жить не только в ней.

"--*То, что вы их увидели "--- невероятная случайность\ldotst "--- заметила Харата.

"--*Или неизвестный феномен, не связанный с вложенными Вселенными, "--- подхватил Грейс.
"--- Чтобы доказать существование Иных, нам придётся колонизировать огромные пространства.

"--*И мы сделаем это вместе с сапиентами, "--- с горящими глазами сказал Лусафейру.
"--- Мы разделим этот путь с ними, чтобы попытаться найти свой собственный в темноте других миров.

Харата кивнула.
Грейсвольд грустно улыбнулся.

"--*Я уже не знаю, захочу ли куда-то уходить, "--- пробормотал технолог.
"--- Ты прав "--- изгнания я бы не перенёс.
Привязался ко всему, видишь ли, "--- Скорбящим, нашему общему делу, Солнышку\ldotst "--- глаза Грейсвольда на секунду расфокусировались.
"--- Наверное, Лу, это и есть старость\ldotsq

"--*Нет, дружище, это всего лишь зрелость, "--- сказал Лусафейру.
"--- Будут другие, и они пойдут дальше.
Мы подготовим им путь.

Вскоре забили сделанные под старину часы.
Из окошка вылезла механическая ящерица, пробежала по корпусу и юркнула в замочную скважину.
Затем пророс механический стебель, развернулись сделанные из стабитаниума листья, с тихим стрекотанием раскрылся оранжевый бутон.
Следом вылетели пчёлы с золотыми крылышками и закружились в изящном танце.
Одна, две, три, пять, тринадцать\ldotst
Планетное время D:000.

Технолог отложил в сторону компьютер, подошёл к окну и положил ладони на прохладный полимер.
Нарэ.
Лу как-то сказал, что для нарэ окно не должно быть чистым и прозрачным.
Грейсвольд попросил специально для него сделать брак, с нарушением всех мыслимых производственных стандартов.
Вышло неплохо "--- инженеры очень старались.
Разумеется, в полимере были и идеально прозрачные, чистые участки, иначе зачем вообще нужны окна?

Здание накрыли огромные тени и тут же обернулись чёрными силуэтами на светлеющем небе.
Один силуэт был побольше "--- словно стрекозий, ещё три поменьше, напоминающие птичьи.
Силуэты покружили, словно играя друг с другом, а затем унеслись к острой стеклянной кромке небесной линзы.

<<Если есть крылья, надо летать, "--- рассеянно думал Грейсвольд.
"--- Да, надо>>.

Огромная голубоватая планета на небосводе постепенно теряла очертания, размывалась сиреневой с зеленоватыми прожилками краской Звезды.
Облака завихрялись причудливыми кольцами, словно ветер заплетал косы в седой отцовской бороде.
Где-то в лесу робко подавали голос птицы.

Над Тси-Ди занимался слабый рассвет.

\chapter*{Интерлюдия X. Чумное поветрие}
\addcontentsline{toc}{chapter}{Интерлюдия X. Чумное поветрие}

То был год демона "--- спустилось на Трогваль чумное поветрие.
Забили гонги в храмах, и каждый удар провожал к Творцу раба его.

Опустел Город Мастеров, и огласились плачем улицы его.
Угрюмо замолчал Стальной квартал, ощерившись металлом против обездоленных и больных.
Приутихло веселье в торговых рядах.
И лишь Проклятый город стоял, как и прежде "--- избежала его кара Творца.

И поползли слухи, что не пророка длань отсыпала горечи Трогвалю, но колдовство безбожников.
И варились эти слухи в котле отчаяния "--- чем слаще рай ушедшим, тем горше юдоль оставшимся.
Как спало поветрие, заскрежетали в домах точильные камни, застучали ступки, толкущие порох.

Спал Проклятый город, утомлённый трудами дневными и играми вечерними.
Спали мужи, спали жёны, спали отроки и девы, спали дети в колыбелях своих.
И не открыл им Творец, что гнев Его уже в пути, звенит смертоносной сталью.

И лишь Шамаль не спал.
Он, и жена его, и дети его слушали истории Марина под светом звёзд.
За занятием этим и застал их трогвалец-кузнец, опередивший армию на звук дыхания\footnote
{Имеется в виду, что кузнец бежал со всех ног, в то время как прочие трогвальцы шли спокойным шагом. \authornote}.

<<Мир и блага земные тебе, кузнец, "--- поприветствовал его Шамаль.
"--- Стряслась ли беда, что ты захлёбываешься воздухом, дарованным нам Творцом для наполнения лёгких и услады головы?>>

<<О Шамаль, "--- заговорил кузнец.
"--- Да простят меня дом и крыша твои за непочтение к ним, но бежать бы тебе на край острова.
Чума породила сталь, и многие хотят смерти тебе, и семье твоей, и друзьям твоим>>.

Опечалился врачеватель, услышав речи кузнеца.
И опечалился он ещё больше, когда понял, что не может отвести друзей от серых врат\footnote
{В сказаниях тенку <<серыми вратами>> называется смерть, чаще всего насильственная. \authornote}.

<<Почему же ты, кузнец, пошёл против воли пророка?>> "--- спросил врачеватель.

<<О Шамаль, "--- ответил кузнец, "--- видят очи Саттама, грешен я перед ним, что отвожу гнев Творца от тебя.
Но жизнь любимой дочери, спасённая умением твоим, не дала мне поступить иначе>>.

<<Не перевелись ещё люди в Трогвале>>, "--- сказал Шамаль и проводил кузнеца с благодарностью.

Догорела ли свеча, догорело ли солнце, но увёл Шамаль семью, и соседей, и гостя в лес.
И Марина в лес увёл, чтобы не нашли трогвальцы его.
Дал врачеватель другу лук и стрелы, чтобы смог тот поймать летающий остров.

<<Уходи, чужеземец, "--- сказал Шамаль.
"--- Ты не найдёшь покоя в Трогвале.
Да помогут тебе люди в твоих поисках!>>

И убили в тот день Салема, и убили Магата, и многих достойных мужей убили в тот день.
И ходили трогвальцы по улицам Проклятого города, и убивали, и насиловали с именем пророка на устах\ldotst

\part*{Приложения}
\addcontentsline{toc}{part}{Приложения}

\appendix

\chapter{Сапиенты}

\section{Общие сведения}

Сапиентом в общем смысле называется материальное существо, уровень развития которого превышает 100 по шкале Яо.
Метод Яо был разработан людьми Древней Земли во времена поздней Эпохи Богов.
Несмотря на то, что его достоверность низка для существ с уровнем выше 400, используется и по настоящее время.

\section{Барьеры развития}

Барьеры развития "--- ключевые моменты для цивилизации, связанные со средним уровнем развития живых существ (по Яо).

\begin{description}
\item [80] "--- барьер Начала: существо способно использовать материальные объекты для своих нужд, но не способно к направленному их преобразованию.
Зарождение религии.
\item[100] "--- барьер Культуры: существо способно осмысленно и направленно преобразовывать материальные объекты и приспосабливать их к своим нуждам.
Зарождение технологии.
\item[130] "--- барьер Цивилизации: существо способно к передаче информации с помощью материальных символов.
Зарождение науки, разработка научной методологии.
\item[200] "--- барьер Эмпатии: первый критический момент.
Существо способно коммуницировать и сотрудничать с любыми другими существами.
Выход в космос.
Существует опасность само- и взаимоуничтожения.
Преодолению барьера Эмпатии способствует выработка универсального морального кодекса.
\item[400] "--- барьер Создателя: второй критический момент.
Технологическая сингулярность, создание существ c возможностями, превышающими собственные, но находящихся на более низком уровне развития.
Опасность уничтожения созданными существами.
Преодолению барьера Создателя способствует помощь новосозданным существам в преодолении ими барьера Эмпатии, приобщения их к универсальному моральному кодексу.
\item[1000] "--- барьер Наблюдателя: существо видоизменяется настолько, что уже не может быть отнесено к породившему его виду.
Полиморфизм, способность свободно перемещаться внутри и вне Вселенной, бессмертие (способность использовать любые источники энергии, неразрушимость и неизнашиваемость).
\end{description}

\textbf{Примечание.}
1000 "--- условное число.
Согласно последним данным, барьер Наблюдателя, который должен быть преодолён \textit{микоргетом}, находится в промежутке между 873 и 1090.

\section{Сапиентные виды}

\subsection{Ветви Земли}

Земля (Древняя Земля) "--- мир, давший начало сапиентам этих Ветвей.
Согласно данным, первыми сапиентами Земли были люди, которые стимулировали развитие всех остальных видов.

Всего видов Ветвей Земли насчитывается около 200 тысяч, здесь будут описаны только упоминавшиеся в книге.

К сапиентам Ветвей Земли, как и к прочим живым существам Земли, применимы устаревшая бинарная и более современная дивергентная классификация.
Для удобства дивергентный код опущен.

\subsubsection{Ветвь Хуманы (Люди)}

Самые успешные сапиенты.
Во Вселенной в настоящее время насчитывают около 98 тысяч видов-потомков.

\begin{description}
\item[Люди-тагуа] "--- жители Драконьей Пустоши.
От первых людей отличаются незначительно.
Изменён метаболизм, имеются необычные пигменты в кожном покрове и глазах "--- адаптация к излучению голубого гиганта.
Широко расставленные глаза, в верхней губе "--- небольшая расщелина около 1 см длиной, по четыре пальца на ногах (результат дрейфа генов).
\item[Люди Мороза] "--- самоназвание <<Род Медведя>>, один из немногих видов людей, сохранивших шерсть.
Чёрная кожа, густые беcцветные волосы на всём теле, единственные безволосые части "--- нос и подушечки пальцев, у женщин "--- верхняя губа.
Мужчины и женщины имеют мощную жировую прослойку, но у женщин она больше (стеатопигия).
\item[Люди Тра-Ренкхаля] "--- в настоящее время представлены народами-аборигенами Тра-Ренкхаля (хака, тенку, зизоце и прочими).
Представляют собой смесь потомков первых людей Древней Земли и прибывших чуть позже людей Лотоса.
Скрещивание между ними предотвратило появление репродуктивного барьера.
От первых людей отличаются сильно развитым половым диморфизмом, свойственным людям Лотоса, от бледных людей Лотоса отличаются коричнево-чёрным цветом кожи и радужных оболочек глаз.
Также имеются приобретённые позднее особенности "--- в частности, у 56\% людей Тра-Ренкхаля транспозиция внутренних органов, не характерная для предков и не имеющая пока рационального объяснения.
\item[Люди-тси] "--- жители Тси-Ди и Тра-Ренкхаля, в настоящее время представлены народами сели и ноа.
Их особенности рассмотрены в соответствующем разделе.
\end{description}

\subsubsection{Ветвь Кани}

Первые кани "--- результат генетического эксперимента людей, собаки с изменёнными конечностями и увеличенным мозгом.
Видов-потомков "--- 58 тысяч.
\begin{description}
\item[Хргада] "--- высокоразвитые кани с планеты Запах Воды системы Канопуса.
От первых кани отличаются отсутствием волос, очень изящным телосложением и плоской грудной клеткой.
Также у них имеются особые ферментативные системы репарации ДНК и некоторые другие изменения метаболизма (следствие повышенного радиационного фона на планете).
\item[Кани Мороза] "--- самоназвание <<Род Волка>>.
Черная кожа, сероватая шерсть с мощным подшёрстком, короткие нос и уши.
Как и Род Медведя, особи обоих полов имеют мощную жировую прослойку (параллелизм).
Также Род Волка отличается от первых кани меньшим размером клыков.
\item[Кани-тси] "--- в настоящее время представлены пылероями Пыльного Предгорья, народом ркхве-хор и Высшими.
Очень высокого для кани роста "--- 1,7--2,2 м.
Имеют светло-серую или коричневатую бархатную шерсть и голубые глаза, также для обоих полов характерна грива.
На четвереньках способны развивать самую высокую скорость среди наземных животных "--- около 130 км/ч.
Прочие их особенности будут рассмотрены в соответствующем разделе.

Пылерои "--- владыки пустынь и саванн.
Обитают как на Короне, так и на Ките.
Предгорные пылерои занимаются скотоводством, разводят чёрных трёхгорбых верблюдов, рептилий и крупных съедобных насекомых, иногда устраивают плантации в оазисах.
Ркхве-хор живут военными набегами.
Высшие пылерои живут в оставленных первыми поселенцами подземных городах на экваторе и почти ни с кем не контактируют.
\end{description}

\subsubsection{Ветвь Планты}

Первые планты "--- результат генетического эксперимента людей, клеточный гибрид человека, цианобактерии и нитробактера с некоторыми дополнительными генами.
Во Вселенной насчитывается около 23 тысяч видов-потомков.

\begin{description}
\item[Планты-тси] "--- в настоящее время представлены идолами Молчащих Лесов, идолами Живодёра, Снежным Кланом и народом трами.
Внешне от первых плантов отличаются незначительно, рост от 1,2 до 1,5 м.
Волосяной покров отсутствует.
Изменены верхние конечности, благодаря чему планты-тси очень хорошо лазают по вертикальным поверхностям.
Прочие их особенности будут рассмотрены в соответствующем разделе.

Обитают в джунглях Короны и Кита.
В жертву богам приносят в основном пленных людей, для жертвоприношений используют рощи благородного баньяна или строят срубы.
Живут деревнями по двести особей максимум, в гнёздах на верхушках деревьев.
Высокоразвитыми считаются трами, обитатели Кита, находящиеся в торговых отношениях с ноа.
Название <<идолы>> пошло от их привычки стоять неподвижно под лучами солнца.
\end{description}

\subsubsection{Ветвь Апиды}

Первые апиды "--- результат генетического эксперимента первых людей, насекомые (предположительно пчёлы), увеличенные в размерах, с изменёнными конечностями, скелетом, дыхательной системой и увеличенным окологлоточным нервным кольцом.
Цель эксперимента неясна до сих пор, скорее всего, он носил чисто научный интерес.
Видов-потомков "--- 790.

\begin{description}
\item[Апиды Ди] "--- уничтоженный вид.
Известно, что они являлись, как и большая часть апидов, колониальными сапиентами (матка, бесполые рабочие особи и трутни).
\item[Апиды-тси] "--- в настоящее время представлены Красными травниками и Бродячим Народом.
Отличаются от ранних апид очень изящной, тонкой конституцией, плодовитостью и перманентным гермафродитизмом.
Рост 1--1,5 м.
Глаз шесть (четыре парных простых, два фасеточных).
Усики развиты умеренно, в углублениях головы, ногочелюсти имеют один ряд зубчиков, которые сменяются в течение жизни.
Прочие их особенности будут рассмотрены в соответствующем разделе.

После вторжения Безумного травники сильно пострадали от войн с идолами, в конце концов были вытеснены в Серебряные горы и Старую Челюсть, где живут очень разрозненно "--- отдельными семьями "--- в пещерах.
Их называют Красными травниками.
Многие Красные впоследствии ушли к людям и стали Бродячим Народом, небольшие поселения стали появляться в джунглях Кита, где идолов нет, но опять же "--- ближе к людям, обеспечивающим им безопасность от Безумного.
\end{description}

\subsubsection{Ветвь Дельфины}

Первые дельфины "--- результат генетического эксперимента кани Лотоса, низшие дельфины с изменёнными конечностями.
Видов-потомков "--- 1264.

\begin{description}
\item[Дельфины-тси] "--- в настоящее время представлены океаническим народом.
Некрупных размеров, около 2 м в длину.
Шкура серая, плавникоруки тонкие, в отличие от ранних предков имеют всего три пальца.
Ротовой аппарат приспособлен как к растительной, так и к животной пище.
Прочие их биологические особенности будут рассмотрены в соответствующем разделе.

Океанический народ (стрелохвосты, или няньки) живёт в океане в тропических и субэкваториальных поясах, кочует за косяками рыб, на стоянках разворачивает понтонные лагеря.
Это единственное племя, которое не приняло ультиматума Безумного.
Мобильны, живут группами по 10--15 особей.
В силу этого, а также особой философии, подразумевающей подавление эмоций, добивание больных и раненых, стрелохвосты были оставлены Безумным в относительном покое.
У каждой группы есть старейшина "--- стрелохвост с развитым чувством интуиции, который и подсказывает безопасный путь для группы.
\end{description}

\subsubsection{Ветвь Стриги}

Стриги, или глазастики "--- вид, упоминаемый в отчётах о Тси-Ди.
Представляют собой увеличенных в размерах (рост около 1,3 м) сов с шестью парами конечностей "--- две ходовых, две летательных и две рабочих.
Согласно данным Существует-Хорошее-Небо, глазастики были созданы сравнительно недавно, а потому не успели адаптироваться к культуре тси, жили обособленно в своих поселениях.
Вероятно, мигрировавшие на Тра-Ренкхаль особи впоследствии вымерли "--- несмотря на свидетельства местных жителей, обнаружить глазастиков так и не удалось.

\subsubsection{Ветвь Акариды}

Водолазы, или акариды "--- результат эксперимента плантов планеты Мицелий системы Канопуса, человек с жабрами по типу акульих и видоизменёнными конечностями.
Способны длительное время оставаться под водой.
Предпочитают тёплые реки и моря.
Видов-потомков "--- 56.

\subsection{Девиантные ветви}

Девиантные ветви "--- сапиентные виды, созданные с нуля хоргетами-демиургами для своих нужд.
Всего девиантных видов насчитывается около 13 тысяч, специальной классификации для них нет.

\begin{description}
\item[Нгвсо] "--- вид, созданный Безымянным, демиургом Тра-Ренкхаля.
Морские обитатели, напоминают осьминогов.
Длина 1,5--2,5 м, чешуя "--- крупная оранжевая у самцов, мелкая сине-зелёная "--- у самок.
Трилучевая симметрия "--- имеются три раздваивающихся щупальца и три простых глаза.
Общаются при помощи непарного гидравлического щупальца "--- подают им звуковые сигналы, похожие на барабанную дробь, также используют жестовый язык.
Способны некоторое время пребывать на суше в специальных влагосохраняющих костюмах.
Занимаются выращиванием водорослей, рыбоводством, охотой, собирательством.
Умеют обрабатывать металл.
Были практически поголовно уничтожены дикими стрелохвостами, оставшиеся разрозненные популяции собрались вместе и отгородились насыпью в Коралловой Бухте, где заключили союз с народом ноа.
\item[Сюзерены, или стрекозодраконы] "--- вид теплокровных летающих рептилий с шестью конечностями, cоздан демиургом Драконьей Пустоши, Кох Свободолюбивой, впоследствии примкнувшей к Ордену Преисподней.
Сюзерены имеют прямую связь с демиургом (молекулярный приёмник в мозгу), что позволяло использовать их как армию в случае вторжения.
После долгой и кровопролитной войны, длившейся тысячу лет (840 земных лет), были вытеснены поселенцами с Земли.
Во времена царствования Валеридов сюзерены обитали в горах Малого Листопада и сохраняли по отношению к людям нейтралитет.
После захвата Адом Драконьей Пустоши, во время номинального правления Скорпидов сюзерены были истреблены.
Есть версия, что Кох Свободолюбивая встала на сторону Ордена Преисподней вследствие шантажа и впоследствии убита.
\item[Машины, или Роботы] "--- электронно-световые устройства с базовыми инстинктами живых существ.
Создавались на многих планетах, но наибольшего расцвета достигла Машина Тси-Ди, созданная народом тси.
\end{description}

\subsection{Ветви Звезды}

Ветви Звезды "--- сапиенты с планеты 1-34.
Это вторая известная планета, на которой установилась стабильная самозарождённая жизнь.
Для жизни необходимы вода, метан и температура около 70 градусов Цельсия, выделяют углекислый газ.
Ветвями Звезды занимается особая группа отделов Ордена Преисподней.
Стандартная классификация видов к ним неприменима.
Контактов с Ветвями Земли не зарегистрировано.
Точное число заселённых ими планет неизвестно.

\subsection{Ветви Ночи}

Ветви Ночи "--- неизвестные вымершие сапиенты.
Следы их деятельности (техника и обиталища, а также космические корабли) обнаружены на нескольких удалённых планетах, в том числе на Тси-Ди.
Местонахождение материнской планеты и заселённые ими экзопланеты неизвестны.

\subsection{Ветви Пламени}

Ветви Пламени "--- общее название трёх независимых клад, включающих высокотемпературные формы жизни: Аберрантов "--- две клады плазмобионтов, живущих в условиях повышенного давления и температуры (в конвективной и радиационной зоне звезды), а также одну кладу собственно плазменных форм жизни "--- Пламя Антареса, или Антариды.
Несмотря на то, что не обнаружено ни одного сапиентного вида Ветвей Пламени, некоторые исследователи придерживаются убеждений, что теоретически эти Ветви способны дать начало виду с сапиентной архитектурой.

\section{Качественные отличия тси от прочих сапиентов Ветвей Земли.
Краткий обзор}

\begin{enumerate}
\item \textbf{Мышечная ткань.}
Изменена структура волокон, вследствие чего прочность выше в 8 раз, а сила "--- в 3 раза.
Худощавые планты-тси способны переносить веса, непосильные для людей прочих видов, а грузоподъёмность кани-тси сравнима с грузоподъёмностью специальных машин.
\item \textbf{Костная и соединительная ткани.}
Изменена архитектоника балок и волокон, а также их взаимодействия в местах соединений.
Прочность костей на излом выше в 2,7 раз при снижении веса в 1,4 раза.
Прочность сухожилий на разрыв выше в 4 раза.
\item \textbf{Кровеносная система.}
Изменена структура интимы сосудов, снижена турбулентность.
Изменена логистика васкуляризации.
Присутствуют резервные контуры кровоснабжения головного мозга и внутренних органов (так называемые свёрнутые сосуды, не имеющие просвета до открытия специальных клапанов).
\item \textbf{Скелет.}
Изменена форма таза "--- смертность и травматизация женщин при родах практически равны нулю.
Изменена форма черепа, присутствуют дополнительные рёбра жёсткости и амортизирующие элементы.
Подобные же изменения в грудной клетке.
\item \textbf{Иммунная система.}
Кардинально отличается от таковой прочих видов и требует отдельного описания.
Тси устойчивы практически ко всем видам микроорганизмов, обитающих на Тси-Ди и Тра-Ренкхаль.
При наличии вакцинации смертность от инфекций равна нулю.
Аллотрансплантация возможна без ограничений, известны успешные случаи ксенотрансплантации "--- в частности, у народа ноа есть обычай закрывать раны лоскутами кожи убитых плантов.
На Диком Юге и в пиратских полисах распространена практика украшать тело треугольной мозаикой плантовой кожи с фрагментами татуировок "--- по одному треугольнику на каждого убитого врага (лорика).
\item \textbf{Зрительная система.}
Изменена оптическая система, имеется дополнительный хрусталик и фокусирующие спекулумы.
Имеются клетки, воспринимающие ультрафиолетовые волны, инфракрасные волны.
Клетки, воспринимающие гамма-излучение, расположены как в глазах, так и в коже по всему телу.
\item \textbf{Слуховая система.}
Воспринимает частоты от 10 Гц до 80000 Гц.
Также нечто похожее на слуховые аппараты обнаружено в толще эпифизов лучевой и берцовой кости, но роль этих органов пока остаётся неясной.
\item \textbf{Дыхательная система.}
Изменена структура лёгкого, противоточная система позволяет переместить в кровь до 95\% кислорода.
Орган звукопроизводства "--- гортанная цитра "--- требует отдельного описания.
\item \textbf{Пищеварительная система.}
Изменены органы вкуса и обоняния, большая часть ядовитых веществ распознаются ещё на стадии измельчения пищи.
Зубы способны сменяться в любом возрасте при потере или сильном повреждении.
Кишечник имеет значительно меньшую длину.
Изменены ферменты.
Внутренняя среда стерильна, пища расщепляется на 89--96\% от массы.
\item \textbf{Половая система.}
Способность к партеногенезу и смене пола, зависимые от феромонов.
Пенис у мужчин небольшой, втягивается внутрь.
Имеются зачатки систем обоих полов.
Менструации отсутствуют, овуляция и эволюция эндометрия запускаются присутствием в коре доминанты деторождения, подтверждённой медиаторами мужской спермы.
Беременность длится в 3 раза меньше (данные по людям-тси), имеются данные о впадении плода в анабиоз на срок до 15 лунных месяцев (при болезни матери либо недостаточном питании).
Беременность определяется по наличию так называемых стигм беременности "--- гипер- или гипопигментированных полосок на шее.
Стигмы беременности являются признаком, общим для тси всех биологических видов.
Также может наблюдаться так называемая пролактиновая трансформация "--- временная лактация и рост груди у обоих полов при стимуляции соска специальным веществом, выделяющимся в ротовой полости новорождённого.
\item \textbf{Нервная система.}
Изменена структура ствола мозга, требует отдельного описания.
Имеются специальные ядра, приспособленные для вычислений в двоичной и троичной логике (у современных тси не используются).
Нервно-мышечные синапсы диафрагмы нечувствительны к курареподобным веществам, имеются дополнительные сплетения для обеспечения дыхания и кровообращения при повреждении ЦНС.
\item \textbf{Регенерация.} Любые внутренние органы, в том числе мозг, способны полноценно регенерировать при сохранном дыхании, кровообращении и питании.
Конечности восстанавливаются при частичной потере (пальцы, кисти, стопы), более грубое повреждение даёт патологическую регенерацию (например, нефункциональные пальцы на культе плеча).
\item \textbf{Психофизиология.}
На заре зарождения цивилизации разные виды вынуждены были специально изучать сигнальную систему друг друга.
Но впоследствии базовые навыки различения эмоций были запрограммированы в каждом тси.
Это хорошо заметно, например,при оценке эмоций апида-тси человеком-тси и человеком Тра-Ренкхаля: человек-тси, не контактировавший прежде с апидами, угадывал эмоциональное состояние в 85\% случаев, против 10\% у человека Тра-Ренкхаля.
\end{enumerate}

\subsection{Защитные механизмы}

\subsubsection{Система электрошунтов}

<<Серебряные жилы>> "--- система проводников, которая оберегает жизненно важные органы при ударе электрическим током.
Выглядит как подкожная венозная сеть, но отличается лёгким металлическим блеском на срезе и отсутствием крови в просвете.

\subsubsection{Протокол <<Тайфун>>}

Нейрогуморальная система выживания.

Парадигма: <<Помочь сородичам.
Возможно, выжить>>.

Запускается снижением ОЦК более 11\%/мин либо тайфун-кодом, представляющим из себя хеш-сумму биометрических параметров сапиента.
Тайфун-код обязательно должен быть произнесён <<голосом друга>>.

Модули:

\begin{enumerate}
\item <<Золотая минута>> "--- спазм повреждённых сосудов, ориентация кровотока на нервно-мышечную систему.
\item <<Лёгкий уход>> "--- абсолютное обезболивание и снятие негативной эмоциональной реакции.
Выброс в желудочки мозга смеси морфина-44 и эндогенного селективного корректора настроения.
\item <<Доминанта защиты>> "--- активация структур, направляющих последние действия умирающего на помощь сородичам.
\item <<Живая сталь>> "--- синтез сверхпроводящих элементов в нейронах.
Возможно только при наличии инфузионного импланта, требуется специальный препарат.
Мозг способен функционировать ещё 148 секунд после биологической смерти, представляя собой квантовый нейрокомпьютер на сверхпроводниках.
Несмотря на то, что <<Живая сталь>> была в стандартной комплектации всех тси, этот механизм считался весьма негуманным <<методом отчаяния>> и за всю историю использовался всего четыре раза.
\end{enumerate}

Сели, большая часть которых сохранила этот механизм, но не умела им управлять, называли предсмертный трансовый героизм <<ветром духов>>.

\subsubsection{Протокол <<Кристалл>>}

Нейрогуморальная система выживания.

Парадигма: <<Выжить.
Возможно, помочь сородичам>>.

Запускается язычным рефлексом (млекопитающие), ногочелюстным рефлексом (апиды).

Модули:

\begin{enumerate}
\item <<Яд>> "--- снижение метаболизма и проницаемости тканей, в некоторых случаях "--- синтез антидота.
Активируется также при утоплении и попадании в открытый космос.
\item <<Пламя>> "--- выброс влаги, снижение теплопроводности тканей, снижение теплообразования, активация теплоотводящих систем и систем репарации ДНК.
\item <<Лёд>> "--- снижение теплопроводности тканей, увеличение теплообразования, изменение внутриклеточного матрикса (выброс глицерина и спиртов с низкой точкой замерзания).
\item <<Молния>> "--- автоматическая дефибрилляция.
\end{enumerate}

\section{Хоргеты (Ветви Смерча)}

Хоргеты считаются отличной от сапиентов формой жизни, несмотря на то, что некоторые выделяют их в отдельные Ветви (Ветви Смерча) или даже относят к Ветвям Земли как творение первых людей.

\subsection{Классификация}

\begin{enumerate}
\item Полярность:

\begin{itemize}
\item \textbf{Позитивные} "--- на основе плюс-сингулярности ПКВ;
\item \textbf{Негативные} "--- на основе минус-сингулярности ПКВ.
\end{itemize}

Полярность неструктурированной сингулярности можно изменить путём преобразования Шмидта.
Полярность хоргета изменяется путём создания неструктурированной сингулярности и последующего синхронного переноса информации.
Процесс изменения полярности хоргета чрезвычайно энергоёмкий, неприятный и требует специального оборудования.

\item Мобильность:

\begin{itemize}
\item \textbf{Боги} (\textit{sd:} deba "--- <<старик>>, \textit{sl:} de) "--- стационарные примитивные хоргеты, обычно заякоривающиеся в планете.
\item \textbf{Демоны} (\textit{sd:} asoga "--- происхождение неясно, \textit{sl:} demon "--- <<мелкое божество>>) "--- мобильные хоргеты, заякоривающиеся в телах сапиентов.
\end{itemize}

\item Способ заякоривания в сапиентах:

\begin{itemize}
\item \textbf{Марионеточный тип} (\textit{sd:} so'mbi "--- <<живой труп>>) "--- хоргет берёт под контроль нервные стволы или верхние отделы спинного мозга.
ВНД сапиента эмулируется в хоргете, мозг в процессе не участвует, органы чувств сапиента не используются.
Первый и наиболее примитивный тип.
Траты масс-энергии значительны.
Марионетки легко вычисляются не только хоргетами, но и другими сапиентами (эффект <<зловещей долины>>).
\item \textbf{Планшетный тип} (\textit{sd:} ada'na "--- <<паук>>) "--- хоргет берёт под контроль последние корковые нейроны.
Органы чувств сапиента используются, но ВНД эмулируется в хоргете.
Траты масс-энергии меньше, чем у марионеток.
Таких сапиентов сложнее вычислить хоргетам, эффект <<зловещей долины>> сведён к минимуму за счёт участия в движениях стабилизирующей системы сапиента (мозжечка и красных ядер у млекопитающих).
\item \textbf{Облачный тип} (\textit{sd:} an'ela "--- <<дух света>>) "--- ВНД эмулируется мозгом.
При запросе из мозга включается хоргет, в котором производятся сложные вычисления, обработка информации или принятие решений.
Эффект <<зловещей долины>> отсутствует.
Траты масс-энергии минимальны.
Вычислить такого хоргета в спящем состоянии практически нереально.
Обычно спящий хоргет сопровождает своё тело до зрелости, что позволяет ему максимально интегрироваться в личность.
Минусы "--- длительность встраивания, излишняя подверженность эмоциям и недостаточная защита от информационного нападения.
\item \textbf{Паразитический тип} (\textit{sd:} 'os "--- <<жук>>, <<мусор>>) "--- очень маленький низкоинтеллектуальный хоргет, использующий сапиентов исключительно для получения эманаций.
Может влиять или не влиять на ВНД сапиента.
Жуки использовались Красным Картелем для сбора масс-энергии, впоследствии некоторые начали вести свободный образ жизни.
\end{itemize}

\item Способы заякоривания в планете:

\begin{itemize}
\item \textbf{Простой} "--- минус-сингулярность одна, все действия выполняет самостоятельно.
\item \textbf{Паук} "--- одна главная сингулярность и множество <<жуков>>, выполняющих функции сбора масс-энергии и преобразования материи.
\item \textbf{Рой} "--- чрезвычайно опасный вид бога.
Состоит из равноправных <<жуков>>, соединённых в сеть.
В силу низкого интеллекта интенсивно использует сапиентов (вплоть до полного их вымирания), вызывает массовые психозы и эпидемии.
Быстро размножается, но часто гибнет сам.
Для зачистки планет от Роев в Ордене Преисподней существовала специальная организация "--- отдел 125 (в настоящее время распущен).
\end{itemize}

Заякоривание преследует несколько целей.
Хоргет "--- это сингулярность ПКВ, то есть, условно говоря, во Вселенной Фотона он не существует.
Для взаимодействия с ней ему требуется некий имеющий относительно постоянную структуру омега-источник, перемещение которого в пространстве можно будет отслеживать.
Впоследствии, по мере развития технологий заякоривания, хоргеты научились использовать сапиентный мозг или планетный омега-фон как интерфейс взаимодействия со Вселенной Фотона.

\item Принцип и цель сборки:

\begin{itemize}
\item \textbf{Урождённый демон} "--- сборка хоргета свободными хоргетами с целью сделать его членом сообщества хоргетов.
\item \textbf{Урождённый сапиент} "--- оцифровка биологической (самообразовавшейся) нейронной сети свободными хоргетами с целью сделать его членом сообщества хоргетов.
\item \textbf{Урождённый бог} "--- сборка хоргета сапиентами, не преследующая цели сделать хоргета членом общества и не предусматривающая развития его как личности.
\end{itemize}

Несмотря на некоторую архаичность этого деления, оно используется до сих пор из-за сильных различий указанных типов в плане психологии.
В частности, урождённые боги могут страдать от последствий акбаса и хиторай, а урождённые сапиенты "--- содержать самые разнообразные дефекты и особенности настройки в зависимости от взрастившей их культуры.

\end{enumerate}

\subsection{Распределение вычислительных мощностей}

Вычислительные мощности распределяются между тремя характеристиками "--- чувствительность, интеллект и устойчивость.

\begin{description}
\item[Чувствительность] "--- способность собирать и структурировать информацию извне.
\item[Интеллект] "--- способность преобразовывать информацию и решать поставленные задачи.
\item[Устойчивость] "--- способность сопротивляться воздействиям извне, в том числе информационному нападению.
\end{description}

\subsubsection{Основные классы по распределению вычислительных мощностей}

\begin{enumerate}
\item \textbf{Визор} (синоним: <<оракул>>) "--- высокая чувствительность, средний интеллект, низкая устойчивость.
\item \textbf{Когитор} (синонимы <<стратег>>) "--- низкая чувствительность, очень высокий интеллект, низкая устойчивость.
\item \textbf{Эффектор} (синоним: <<универсал>>) "--- средняя чувствительность, средний интеллект, средняя устойчивость.
\item \textbf{Интерфектор} (синоним: <<воин>>) "--- низкая чувствительность, средний интеллект, высокая устойчивость.
\end{enumerate}

Смешанные типы:

\begin{enumerate}
\item \textbf{Когитор-интерфектор} (синоним: <<тактик>>) "--- околонулевая чувствительность, средний интеллект, очень высокая устойчивость.
Тип, часто встречающийся среди младших воинских чинов.
\item \textbf{Когитор-визор} (синоним: <<учёный>>) "--- чрезвычайно редкий в настоящее время тип.
Ранее такая специализация, как следует из названия, встречалась в основном среди научного персонала.
Устойчивость практически нулевая.
Один из самых знаменитых представителей "--- Сиэхено Опаловый Глаз.
\item \textbf{Визор-интерфектор} (синоним: <<диверсант>>) "--- высокая чувствительность, очень низкий интеллект, средняя устойчивость.
Также редко встречающийся тип, незаменимый, однако, при выполнении чётко поставленной задачи.
\end{enumerate}

Эти классы "--- условное деление, каждый хоргет обычно распределяет собственные мощности так, как ему требуется.
Соотношение может изменяться и в ходе работы, из-за подключаемых модулей.

Единственный нюанс "--- подключение модулей редко изменяет класс.
Изменение же класса, т.е. пересборка "--- задача, требующая длительного анализа, всегда проводится специалистами.
Пересборка урождённых сапиентов невозможна в силу особенностей их внутренней структуры.

\subsection{Специальности (классификация Ордена Преисподней)}

\begin{enumerate}
\item \textbf{Биология} "--- создание, изменение, восстановление и исследование биологических (природных) систем;
\item \textbf{Информатика} "--- работа с любыми видами информации:

\begin{enumerate}
\item \textbf{Аналитика} "--- выявление практической ценности информации и логических схем;
\item \textbf{Криптология} "--- шифрование, а также выявление закономерностей в сырых (необработанных) блоках информации;
\item \textbf{Структурология} "--- классификация и встраивание проверенной информации и логических схем в общую систему знаний;
\end{enumerate}

\item \textbf{Диктиология} (устар.: культурология, сетевая технология) "--- создание, изменение, восстановление и исследование сетей "--- схем коммуникации между биологическими и/или небиологическими системами;
\item \textbf{Технология} "--- создание, изменение, восстановление и исследование небиологических систем (использующих искусственно созданные принципы работы).
\end{enumerate}

Деление демонов по специальностям также условно, так как каждый демон в той или иной степени имеет подготовку по всем указанным здесь направлениям.

\chapter{Планеты}

\section{Тра-Рекхаль}

\subsection{Флора}

\begin{description}
\item[Акхкатрас] "--- женские растения вида секвойя Бенедикта, завезённого переселенцами с планеты Лотос.
Мужские растения, что интересно, имеют совершенно другое название "--- мисатр "--- и используются как источник корабельной древесины.
Самые старые деревья достигают 150 м в высоту и более 10 м в диаметре;
пряные ягоды (ложноплоды) акхкатрас имеют характерный ярко-алый оттенок, носящий то же название, и считаются изысканным деликатесом.
\item[Баньян благородный] (<<слепой страж>>) "--- дерево, произрастающее на Тси-Ди, было завезено переселенцами на Тра-Ренкхаль.
Взрослое растение отдаёт в стороны воздушные и надземные побеги, образуя рощицу размером в несколько гектар.
Назван так из-за отсутствия в рощице прочих растений, бактерий и грибов (баньян выделяет несколько десятков мощных фитонцидов, антибиотиков и фунгицидов).
Прежде рощи благородного баньяна использовались как храмы и больницы (из-за практически стерильной внутренней среды), но из-за плохой обороноспособности потеряли своё значение.
Есть мнение, что благородный баньян "--- упрощённая узкоспециализированная кольцевая теплица.
\item[Кедр молчащий] "--- девиантное (?) древесное растение, распространённое в Суболичье и Молчащих лесах.
Древесина и хвоя благодаря микроструктуре обладают свойством гасить звуковые колебания в диапазоне 1--23 кГц.
Из древесины молчащего кедра делают столбы и половые плиты для храмовой <<зоны молчания>>.
В лесах молчащего кедра живёт ограниченное число видов Ветвей Земли "--- из-за затруднённости или полной невозможности звукового общения.
\item[Лака] (tn: l\={a}\"{a}k\^{a} "--- <<плохой знак>>) "--- дерево джунглей с чёрными листьями и ярко-красными ядовитыми плодами.
Лаковый сок содержит мощный нейротоксин, проникающий через гематоэнцефалический барьер, тропный к клеткам ретикулярной формации сапиентов-млекопитающих, а также клеткам ретикулярной формации всех видов тси.
\item[Мак молитвенный] "--- распространённое на Тра-Ренкхале растение, являющееся природным источником морфина-44, мощного анальгетика.
Скорее всего, он был завезён тси и генетически модифицирован, чтобы успешно конкурировать с местными растениями.
В пользу этого говорит его ареал (влажные зоны всех материков, за исключением пустынь и Дальнего Севера), а также наличие в мозгу тси ферментативных систем, специфичных именно к морфину-44.
Употребление масла молитвенных маков представителями тси совершенно безопасно, у прочих сапиентов планеты Тра-Ренкхаль масло вызывает галлюцинации и сильную наркотическую зависимость.
\item[Маликхов венок] "--- папоротник-эпифит, издающий очень приятный аромат.
Использовался сели как сухоцвет и ингредиент для курений.
\item[Молочник рыбий] "--- кустарниковое растение.
Сок его оказывает парализующее действие на рыбу, но безвреден для сапиентов.
Рыбаки (обычно дети) выливают сок молочника в реку и ждут, пока парализованная рыба всплывёт.
\item[Папоротник-опахало] "--- папоротник джунглей, имеющий очень длинные перообразные вайи.
\item[Тхальсар] "--- шорея тёмная, дерево Корвуса.
\end{description}

\subsection{Фауна}

\begin{description}
\item[Зелёная пчела] (tn: tr\`{u}kchu\r{a}l "--- <<летающая драгоценность>>) "--- вид перепончатокрылых насекомых.
Крупные (около 1,5 см) тела, 4 крыла (два больших, прозрачных, и два маленьких, желтовато-опалесцирующих).
Брюшко полосатое, зелёное с жёлтым.
Жало имеет чехол, который отрывается при укусе и отрастает через какое-то время.
Яд смертельно опасен для всех видов, за исключением тси и колибри вида Пчелиный Ужас, которые питаются зелёными пчёлами.
Гнёзда строят в кронах деревьев-медоносов, форма гнёзд "--- цепочка шаров (обычно 3--5), от самых маленьких внизу до самых больших сверху.
Гнездо строится из выделений симбионта "--- паука-буйвола, которого зелёные пчёлы ловят, выращивают и доят, а затем инкрустируется кусочками коры.
В верхнем ярусе гнезда выращиваются личинки, нижние ярусы используются для грибных ферм, где зелёные пчёлы выращивают низшие грибы.
\item[Змея-верёвка]
\item[Индиго-светляки] "--- семейство девиантных насекомых Тра-Ренкхаля с биолюминесценцией.
Крупные, до 3--4 см в длину.
Имеют три глаза, три жёстких крылышка и органы полёта "--- геликоптероиды, характерные для девиантных насекомых Тра-Ренкхаля.
Специфическая система биолюминесценции "--- фотоны образуются в результате химических реакций и проходят через тонкую хамелеоновую занавеску, приобретая цвет от ультрафиолетового до зелёного.
Не путать с лантерн-светляками, которые являются прямыми потомками лампирид Древней Земли.
\item[Каменная жаба] "--- вид земноводных.
Крупные (до 15 см), узкоротые, сероватого цвета, рисунок кожи напоминает гранит.
Живородящие.
На спине имеются костно-роговые щитки.
Крик каменной жабы напоминает гуление и плач младенца.
Днём животное сидит в каменных пещерках и каменных постройках, ночью выбирается наружу "--- на охоту и для спаривания.
Иногда приползает на детский плач, из-за чего на севере её зовут жабой-кормилицей.
\item[Кораллица]
\item[Мохноножка-однолюбка] "--- бескрылая птица джунглей.
Питается беспозвоночными подстилки.
Дробно щёлкают клювом.
Моногамны, название пошло из-за тонких пёрышек на цевках.
\item[Мраморная змея]
\item[Олень-вертихвостка] "--- одомашненное сели девиантное парнокопытное животное, конституцией напоминающее антилопу.
Самцы круглый год носят длинные двуветвистые рожки, самки безрогие.
Название пошло от привычки животного трясти небольшим белым хвостиком "--- способ внутривидовой коммуникации.
\item[Печальная флейта] (согхо) "--- птица из семейства курообразных.
Самка серая, с сизыми подпалинами на брюшке.
Самец яркий, фиолетово-лиловый, с красивым веерообразным хохолком, поёт на восходе и закате солнца однообразную, напоминающую звук свирели песню (<<тиу-лиу-ла, фьюю, тиу-ла>>).
Мясо согхо "--- деликатес, очень нежное и питательное.
Перья самца согхо используются сели для так называемых скорбных уборов (чаще всего серёг), которые носили в знак жажды мести и скорби по убитым друзьям.
\item[Попрошайка] (tr\={a}s\"{a}kch, трасакх) "--- птица джунглей из семейства медоуказчиков.
Питается насекомыми, мёдом и воском перепончатокрылых.
Охотно поедает и гнёзда зелёных пчёл, но сама не может атаковать гнездо, поэтому показывает людям или идолам дорогу к гнёздам, после чего ждёт, пока охотники разделаются с пчёлами.
\item[Сахарная муха] "--- вид двукрылых насекомых.
Взрослые особи откладывают яйца в закисшие фрукты, которыми питаются опарыши.
Опарыши используются восточными сели как деликатес "--- в вяленом виде или в виде сладкой муки, из которой делаются конфеты.
\item[Сиу-сиу] (звукоподражание песне) "--- птица с ярким радужным оперением, которое высоко ценилось племенем сели.
Перья сиу-сиу шли на праздничные головные уборы и входили в состав ритуального подношения Солнечной птице.
\item[Смертожар]
\item[Ушастая уволочь] "--- мелкое (20 см без хвоста) кошачье.
Охотится на крупных насекомых и мелких грызунов, но очень часто совершает набеги на жильё, утаскивая разделанное мясо и пищевые остатки.
\item[Шипастый страус] "--- он же страус-дикобраз, бескилевая птица саванны с видоизменённым пером (шипами).
Является одним из двух видов ветви Броненосные казуары (название <<страус>> ошибочно).
Чрезвычайно опасен, так как проявляет агрессию к любым сапиентам на своей территории, очень ловок и хорошо уворачивается как от стрел, так и от копий, а шиповатое оперение представляет собой неплохую броню.
Врагов колет шипами и бьёт мощными ногами до смерти.
Гнездится неподалёку от водоёмов, в низинах.
Перо страуса-дикобраза ценится всеми племенами Тра-Ренкхаля, часто вплетается в волосы как знак силы (иногда "--- как тайник, так как в полость пера помещается значительное количество золотого песка и небольшие свитки пергамента).
\end{description}

\section{Прочие планеты}

\subsection{Диана}

\subsection{Драконья Пустошь}

\subsection{Древняя Земля}

\subsection{Запах Воды}

\subsection{Капитул (Сомерскай)}

Планета, преобразованная богом Brahma-23 (проект <<Золотая ладья>>), созданным лабораторией Кошкина в Калькутте.
Лаборатория прославилась очень смелыми, энергозатратными экспериментами и тем, что практически все научные сотрудники были женщинами.
Трёхкратный лауреат Расширенной Нобелевской премии Хезер Коллинз, разработавшая методологию преобразования планет, также начинала свою деятельность в кошкинской лаборатории.

Согласно данным, это первая планета, на которой были применены методы искусственного (проводникового) наведения магнитосферы и стабилизации климата с помощью Кориолисовой градиентации парниковых газов (КГПГ).
Вследствие этого на большей части суши умеренный климат.
Полупустыня и тундра --- самые суровые биомы Сомерская --- вместе занимают около 1000 км$^2$.

В настоящее время является главной базой (Капитулом) Ордена Преисподней.
Главный городской конгломерат "--- Скальдборо-Гелиополь (Скаге).

Населён Сомерскай видами-представителями абсолютно всех известных Ветвей, за исключением Ветвей Ночи (около 140 видов).
Одна из самых густонаселённых планет (30 млрд особей).

\subsection{Лотос (Дагон)}

Планета системы Фомальгаута, одна из первых заселённых планет.
Демиург "--- Грейсвольд Каменный Молот.
Первые поселенцы "--- экипаж <<Тёмного Пламени>>, капитан "--- Бенедикт Альсауд.

Первоначальное название планеты "--- Дагон.
Планета была переименована в Лотос после преобразования.

\subsection{Марс}

\subsection{Мороз системы Арракиса}

Планета с весьма суровым климатом.
Преобразована была во времена поздней Эпохи Богов.
Демиург (Эйраки Мороз) из-за ошибки физиков сделал среднюю температуру на планете гораздо ниже, чем требовалось.
В приэкваториальных поясах летняя температура --20°С, зимняя до --90°С. На полюсах --150°С, жизни нет.
Несмотря на то, что проект был закрыт, спустя почти столетие нашлись сапиенты, которые решились на заселение этой планеты.
Специально для них учёными Древней Земли были спроектированы специальные поселения (<<bzec>>, бижеч "--- на языке русе <<укрытие>>), а также холодоустойчивые живые существа "--- нимелто, коно, клучо и прочие.

Сапиенты Мороза представлены двумя видами: люди (род Медведя), кани (род Волка).
Вся жизнь их направлена на сохранение тепла.
Сапиенты вместе со стадами кочуют по планете, следуя за солнцем, от одного бижеч до другого.
Самые древние бижеч обогреваются геотермальными водами из пробурённых скважин, более поздние строились возле естественных вулканов и разломов.
Также многие поселения получают энергию от <<чёрных полей>> --- кремниевых щитов, расположенных в закрытых от ветра низинах.
Полученная энергия почти полностью расходуется на поддержание теплиц.

Кани и люди живут вместе.
Все сапиенты носят очки, маски и одежду специального покроя из шкуры нимелто.
Общение вне поселений только жестовое.
В поселениях принято ходить без одежды и спать скоплениями по 8--10 особей для сохранения тепла.
Дети спят в центре, взрослые по бокам.

Язык общения "--- рут (русе) "--- единый для всех, очень ёмкий и лаконичный, практически не менялся со времени заселения.

Общество Мороза "--- самое невоинственное в известной Вселенной.
Стычки между членами племени очень редки, случаи драк и убийств не зафиксированы, обычаями племени эти казусы не регулируются.
Самым страшным <<преступлением>> считается оставление включенной лампы на время сна, за него предусмотрено самое суровое <<наказание>> "--- трёхчасовая одинокая прогулка.
Для решения споров используется <<сидение>> "--- особи садятся друг напротив друга и сидят около часа неподвижно.
После такого молчаливого <<разговора>> спор обычно решается.

В силу малой заселённости планета нейтральна по отношению к хоргетам, но Ад и Картель держат там наблюдателей.

\subsection{Тси-Ди}

Двойная планета системы белого карлика, старое название "--- Мерлин-Ниниана.
Общепринятое, вероятно, от чайнис 启迪 (Q\v{\i}-D\'{\i}) "--- <<вдохновение>>.
Планеты имеют примерно одинаковую массу, вследствие чего обращаются вокруг равноудаленной от планет точки "--- Центра Масс.
Позже Центром Масс стали называть расположенную в этой точке станцию "--- в ней располагалась научно-исследовательская база, центр полётов между планетами и узел стабилизации планетарной системы защиты.

Планеты всегда повёрнуты друг к другу одной (океанической) стороной.
На планете Тси имеются 14 материков, связанных Паутиной "--- дорогами на силовых полях.
Обитаемая зона планеты Ди "--- 10 тысяч километров по периметру океана, покрытые лесостепью и вечнозелёными кустарниками.
Всё остальное пространство "--- каменистая пустыня с разреженной атмосферой, в которой располагаются рабочие и исследовательские механизмы и Оазисы "--- закрытые станции для проживания сапиентов.

Ранее Тси-Ди была обиталищем народа тси, впоследствии тси были уничтожены Машиной.
Согласно данным, в настоящее время на планете Тси-Ди не обитает ни один сапиентный вид Ветвей Земли.

\subsection{Тысяча Башен}

Жидкое ядро с кристаллической поверхностью.
Поверхность состоит из гигантских Друз, разделённых полосками воды в глубоких ущельях "--- Трещинах.
Единого океана нет.
Соответственно, из-за вращения планеты и прочих причин скорость воды очень высока, Друзы постоянно стачиваются, но тут же нарастают из-за насыщенных тем же веществом вулканических газов.
Также Друзы от стачивания спасают колониальные раковинные хемосинтетики, использующие энергию вулканических газов;
интенсивное нарастание этого бактериально-протозойного мата, обладающего противотурбулентными свойствами, спасает друзы от вымывания.
Они же вырабатывают свободный кислород, необходимого для дыхания.
Из Друз торчат Башни (отдельные крупные кристаллы), которые регулярно подвергаются мощнейшему выветриванию.

Пейзаж напоминает чем-то каньоны Северной Америки.

\subsection{Чёрная Скала}

\chapter{Персоналии}

\section{Демоны}

Названия демонов, по принятой ныне классификации, имеют структуру:

\textbf{Имя Прозвище из клана Клан (Исходное название) (Версия)}

\textbf{Имя}: краткое слово на Эй-B0, используется в небоевой обстановке.

\textbf{Прозвище}: от одного до трёх слов, которые могут быть переведены на практически любой язык.

\textbf{Клан} (необязательно): название демона, ядро или части ядра которого использовались при создании.

\textbf{Исходное название} (необязательно): название, данное демону разработчиками-сапиентами или имя сапиента.

\textbf{Версия} (скрыто): версия ядра демона согласно Реестру Ордена Преисподней.
Версии являются засекреченной информацией, поэтому в этой книге опущены.

\begin{description}
\item[Айну Крыло Удачи (Айну)] "--- урождённый бог, биотехнолог-интерфектор, легат прима Ордена Преисподней.
Постоянная подруга и спутница Тахиро Молниеносного.
Создана на Тси-Ди как подопытный образец "--- на ней тестировали систему защиты против хоргетов, сбежала во время лага при разворачивании системы.
Прозвище Крыло Удачи получила после битвы с Чук Тьма Над Горой "--- молодая демоница уничтожила превосходящую по опыту и мощи противницу, найдя в её обороне крошечное окно.
Ей так понравилось это прозвище, что она очень часто появлялась на собраниях с нашивкой или значком в виде белого птичьего крылышка.
Погибла в битве с превосходящими силами Красного Картеля.

\item[Анкарьяль Кровавый Шторм] "--- урождённый демон, биотехнодиктиолог-интерфектор, легат терция Ордена Преисподней.
Создана на Капитуле.
Использует прозвище-позывной, данное ей людьми на планете Тысяча Башен (Angaralla "--- <<малышка Анкара>>, произошедшее от angara "--- <<море>>, распространённого в то время женского имени).
Кровавый шторм "--- это название самума, несущего красную пыль.
Пятая Война на Тысяче Башен была для демоницы дебютом и первой серьёзной победой.

\item[Аркадиу Шакал Чрева (Аркадиу Валериану Люпино)] "--- урождённый сапиент, биодиктиолог-когитор, легат терция Ордена Преисподней.
Преобразован на Драконьей Пустоши Яйвафом Солёная Борода.

\item[Гало Кровавый Знак из клана Эйраки] "---

\item[Грейсвольд Каменный Молот (Грисволд-13)] "--- урождённый бог, технодиктиолог-эффектор, легат прима Ордена Преисподней.
Создан Лабораторией Дж.\,Грисволда (научный город Цикаго-2, домен Северная Америка, Древняя Земля).
Демиург планеты Лотос системы Фомальгаута.
Прозвище Каменный Молот придумал сам "--- чтобы никогда не забывать, с чего началась технология.

\item[Ду-Си Охотник из клана Дорге] "--- вероятно, от чайнис 毒蜥 (D\'{u}x\={\i} "--- ядовитый варан).

\item[Кольбе Старое Изречение и Рабе Юный] "--- братья-биологи, одни из самых известных учёных Ордена Преисподней.
Несмотря на принадлежность к Ордену, соблюдают нейтралитет и не особенно это скрывают.

\item[Лусафейру Лёгкая Рука из клана Эйраки] "--- урождённый демон, информатик-когитор, максим секунда Ордена Преисподней.
Создан на Преисподней.
Назван был по имени известного физика эпохи Последней Войны "--- Люцифера Гафт-Йенковски, предсказавшего открытие первичного поля.
Прозвище получил за тактику минимального вмешательства в действия младших по званию демонов.

\item[Митрис Безымянный] "---

\item[Стигма Чёрная Звезда из клана Тахиро] "---

\item[Тахиро Молниеносный (Тахиро та Ханаяма)] "--- урождённый сапиент, биотехнодиктиолог-когитор, максим терция Ордена Преисподней.
Основатель клана.
Постоянный друг и спутник Айну Крыло Удачи.
Преобразован Лусафейру Лёгкая Рука, его ближайший друг и помощник.
Прозвище является прозвищем, созвучным имени (Tahio "--- <<сверхбыстрый>>).

\item[Тхартху Танцующая Тень (Тхартху ар’Хэ э’Тхартхаахитр)] "---

\item[Таниа Янтарь (Тханэ ар’Катхар э’Тхаммитр)] "---

\item[Эйраки Мороз (Арракис-1)] "--- урождённый бог, информатик-когитор, якобы основатель Ордена Преисподней.
Основатель клана.
Создан Лабораторией Омега-преобразований Малаги (стратегический научный центр Малага, домен Европа, Древняя Земля).
Впоследствии взял имя Хатрафель Безумный.

\item[Яйваф Солёная Борода из клана Дорге]
\end{description}

\section{Сапиенты}

\subsection{Сели (обновить)}

\begin{description}
\item[Акхсар ар’Катхар э’Тхаммитр] (Случайно-Задушил-Змею, Случай) "--- отец Чханэ
\item[Акхсар ар’Лотр э’Сотрон] (Седой-Дедушка-В-Снегу, Снежок) "--- друг Митхэ
\item[Атрис] "--- отец Ликхмаса
\item[Кхарам ар’Хэ э’Тхартхаахитр] (Забавно-Скачущая-Пружинка, Пружинка) "--- слуга Тхартху
\item[Кхарас ар’Хитр э’Хатрикас] (С-Любовью-Свитая-Веревочка, Верёвочка) "--- воин Тхитрона
\item[Кхатрим ар’Сар э’Тхонтротрис] (Ветка-Растущая-Из-Листа, Веточка) "--- жрец, врач
\item[Кхотлам ар’Люм э’Кахрахан] (Испачканное-Мёдом-Перо, Пёрышко) "--- кормилица Ликхмаса
\item[Кхохо ар'Хетр] (Уголёк) "--- воин Тхитрона
\item[Ликхмас ар’Люм] э’Тхитрон] (Играющая-Булыжником-Лиса, Лис) "--- инкарнация Аркадиу
\item[Ликхэ ар’Трукх э’Тхинат] (Завёрнутый-В-Платок-Орех, Орешек) "--- воин Тхитрона
\item[Ликхэ ар’Хэ э’Тхитрон] (Огонь-Прошедшего-Рассвета, Огонёк) "--- служанка дома Люм
\item[Лимнэ ар’Люм э’Тхитрон] (Укусила-Кошку-За-Сосок, Кусачка) "--- сестрёнка Ликхмаса
\item[Манис ар’Ликх э’Тхаммитр] (Цветок-Пахнущий-Домом, Цветочек) "--- погибший мужчина Чханэ
\item[Манис] (Столбик) "--- друг Ликхмаса
\item[Манэ ар’Люм э’Тхитрон] (Пролитое-Утром-Молоко, Молочко) "--- сестрёнка Ликхмаса
\item[Митликх] --- жрец Тхитрона
\item[Митрис ар’Люм э’Ихслантхар] (Сложил-Башню-Из-Камней, Башенка) "--- Король-жрец
\item[Митхэ ар’Кахр э’Тхартхаахитр] (Золотая-Крупица-На-Дороге, Золото) "--- дарительница Ликхмаса
\item[Сакхар ар’Сатр э’Ихслантхар] (Плывущий-В-Кронах-Карп, Карп) "--- инкарнация Грейсвольда
\item[Саритр ар’Люм э’Тхитрон] (Вечно-Хмурится-Без-Причины, Хмурый) "--- умерший брат Ликхмаса
\item[Сатракх ар’Сит э’Тхартхаахитр] (Зверёк-Вылез-Из-Кладовки, Зверёк) --- жрец, любовник Тхартху
\item[Сатхир ар’Со э’Тхаммитр] (Затаившийся-В-Кровати-Крокодил, Крокодил) "--- жрец, попытавшийся убить Чханэ
\item[Сиртху ар’Митр э’Сотрон] (Домик-С-Карамельными-Окнами, Домик) "--- старый слуга дома Люм
\item[Ситлам ар’Со э’Тхаммитр] (Зачем-Молоток-Взял, Молоточек) "--- жрец, брат Сатхира
\item[Ситрис ар’Эр э’Тхинат] (Вытащили-Со-Дна-Колодца, Донышко) "--- воин Тхитрона
\item[Согхо ар’Хэ э’Тхартхаахитр] (Лишённый-Голоса-Журавль, Журавлик) "--- дарительница Чханэ
\item[Согхо] "--- воительница отряда чести
\item[Трукхвал ар’Со э’Тхартхаахитр] (Звоночек) "--- учитель Ликхмаса
\item[Тханэ ар’Катхар э’Тхаммитр] (Змея-Похожая-На-Шнурок, Змейка) "--- подруга Ликхмаса
\item[Тхартху ар’Катхар э’Травинхал] (Две-Зелёных-Бусины, Бусинка) "--- прародительница Чханэ
\item[Тхартху ар’Хэ э’Тхартхаахитр] (Подражает-Птичьему-Пению, Птичка) "--- любовница Грейсвольда
\item[Хатлам ар’Мар э’Тхартхаахитр] (Глаза-Похожие-На-Вишню, Вишенка) "--- Анкарьяль
\item[Хитрам ар’Кир э’Тхитрон] (Иногда-Любит-Плавать, Пловец) "--- кормилец Ликхмаса
\item[Хонхо ар’Лотр э’Тхитрон] (Ящерица-Пишущая-На-Стенах, Ящерка) "--- умерший учитель-жрец
\item[Эрликх ар’Фа э’Самитх] (Сладкая-Ягодная-Конфета, Конфетка) "--- тренер Ликхмаса
\item[Эрликх] --- воин Тхитрона
\item[Эрси ар’Митр э’Тхитрон] (Приручили-Котёнка-Оцелота, Котёнок) "--- ребёнок, принесённый в жертву
\item[Эрхэ ар’Люм э’Сотрон] (Слишком-Много-Ест, Обжорка) "--- любовница Акхсара
\item[Эрхэ ар’Сит э’Тхитрон] (Зачатая-В-Ночной-Реке, Речка) "--- служанка дома Люм
\end{description}

\subsection{Тси}

\begin{description}
\item[Сотканный-Из-Темноты-Заяц]
\item[Существует-Хорошее-Небо]
\item[Хрустально-Чистый-Фонтан]
\end{description}

\subsection{Прочие}

\begin{description}
\item[Анатолиу Сильбеу Тиу] (t-sl: Anadoliv Cilbev Tiv) "---
\item[Бенедикт Альсауд] [Чёрная Борода] "--- капитан дальнего плавания, а впоследствии капитан космического корабля <<Тёмное пламя>>, перенёсшего первую волну колонистов на планету Лотос.
Родился в Бристоле, домен Европа, Древняя Земля.
Получил известность благодаря огромной находчивости и развитой интуиции "--- его мгновенные приказы при авариях и отказе оборудования впоследствии разбирались целыми комиссиями, так как сам он не всегда мог их объяснить.
Прозвище Бенедикта Альсауда "--- Чёрная Борода или Король морей (намёк на его знатное происхождение "--- Бенедикт является прямым потомком последнего короля Саудовской Аравии).
Альсауд имел такой авторитет, что его кандидатуру на должность капитана <<Тёмного пламени>> предпочли прочим, более подготовленным теоретически.
Умер в возрасте 61 года во время второй исследовательской экспедиции на Лотосе от яда шипастой камбалы.
\item[Валериу X Валерид] (t-sl: Baleriv Dekad Balerid) "---
\item[Клавдиу Дентосиу] [Пересмешник] (Klavdiv Dendosiv, ?--345 по календарю Талиа) "--- поэт и наёмный убийца, герой народных историй (трикстер).
Настоящее его имя неизвестно, и большая часть деятельности скрыта мраком времён.
Единственный надёжный источник о нём "--- очерки <<Воспоминания о великом тёзке>> Клавдиу Семито.
Известно, что Клавдий Дентосий начал свою деятельность разбойником.
Впоследствии по непонятным причинам он покинул разбойничью шайку и стал вести жизнь наёмного убийцы.
Его <<клиентами>> были в основном богачи, которые терроризировали бедных.
Деньги Клавдий тратил на проституток или раздавал беднякам.
Странная приверженность Клавдия продажному сексу в итоге сыграла с ним злую шутку "--- он умер от остановки сердца в постели с блудницами.
В народе шутили, что любвеобильный красавец просто не рассчитал свои силы, но историки склоняются к версии, что он был отравлен подосланной шпионкой.
\item[Клавдиу (Клауше Шмол) Семито Фризский] (t-sl: Klavdiv Semido fra'Teris) "--- придворный новеллист Валериу X, потомок древнего рода писателей и учёных.
Известен в основном как автор исторических хроник, в том числе очерка <<Воспоминания о великом тёзке>>.
Семито был убеждённым монархистом, но личность Дентосия настолько впечатлила его, что в преклонном возрасте, уже после смерти короля, он написал посвящённый поэту труд.
\item[Март Джонатан Митчелл] [Одноглазый Март, М.\,Дж.\,М.] "--- идеолог Эволюциона, писатель.
В возрасте 32 лет был помещён в камеру смертников за разбой и многочисленные жестокие убийства, но в силу некоторых обстоятельств не был казнён.
В камере Митчелл провёл 20 лет.
За это время он получил три высших образования "--- химика-технолога, философа и инженера-строителя.
Находясь в камере, он удалённо работал по специальности, вёл просветительскую деятельность.
Единственное, что так ему и не далось "--- правописание.
Несмотря на многочисленные петиции о помиловании, 52-летний Митчелл был казнён путём расстрела "--- согласно его последнему желанию.
Последними его словами были: <<Смешанное чувство.
Я не могу принять организованное насилие, но придумать для себя другую кару тоже не могу>>.
\item[Михаил Алексевиц Кохани] [Лимб Четырнадцать] "--- первооткрыватель омега-поля, лауреат Расширенной Нобелевской премии (319 Эпохи Богов).
Жил и работал в городе Кёнигсберг, домен Европа, Древняя Земля.
На момент открытия ему было 19 лет.
Впоследствии оставил науку и стал авиаконструктором.
Известен как изобретатель <<спичечного самолёта>> "--- надёжного летательного аппарата, который мог собрать любой человек при наличии самых простых инструментов.
Умер в возрасте 98 лет после продолжительной болезни.
\item[Леам эб-Салах эб-Сайед ала-Фариз] (t-sl: Lejam Salagid Saedid fra'Teris) "---
\item[Людвиг Карл Рагнар Вейерманн] [Вейердварф] "--- первооткрыватель омега-поля, лауреат Расширенной Нобелевской премии (319 Эпохи Богов).
Жил и работал в городе Кёнигсберг, домен Европа, Древняя Земля.
На момент открытия ему было 22 года.
Впоследствии занялся просветительской деятельностью "--- его перу принадлежат учебники по Теории Всего, которые использовались ещё более пятисот лет после первой публикации и пережили триста восемь редакций.
Также Вейерманн известен как поэт и композитор, сочинявший лирические песни на языке джерман.
Покончил жизнь самоубийством в возрасте 56 лет по неизвестной причине.
\item[Софиа Ловиса Карма] "---
\item[Татиан Сергеу Анно] [Освободитель] "---
\item[Хервар Лонгсин-Храш] "---
\item[Юле Алексевиц Гагарин] [Сокол Последней Войны] "--- один из первых людей, выживших в космическом пространстве (согласно некоторым данным, самый первый выживший).
Его жизнеописания утеряны, но хорошо известно, как он выглядел "--- портреты Юле гравировали на многих космических кораблях на удачу.
Среди историков планеты Лотос был в ходу фразеологизм <<улыбка Гагарина>> "--- незначительная подробность события, оставшаяся в истории вместо более важных данных.
\end{description}

\chapter{Словарь терминов}

Краткие обозначения языков:

en "--- энглис (основной язык Древней Земли)

hy "--- ханьюи (чайнис, язык Древней Земли)

ru "--- русе

sd "--- сохтид

sl "--- классический секта-лингу

t-sl "--- талино (талианский секта-лингу)

tn "--- цатрон

\begin{description}
\item[Абис] "--- письменность ноа.
Слоговая, звуко-буквенная, идеограммы для грамматических элементов.
\item[Ад] "--- см. \textit{Орден Преисподней}.
\item[Акбас] (sd: hikeba-yasu "--- <<страх>>, <<холод>>, <<одиночество>>, <<брошенность>>) "--- первое ощущение осознающей себя высокоинтеллектуальной системы.
Нескорректированный акбас может стать причиной помешательства и разрушительных аффектов, вплоть до самоуничтожения.
В обществах сапиентов или высших животных коррекция акбаса осуществляется заботой родителей, собратьев или других существ.
\item[Амулет Сана] "--- предмет экипировки жреца сели.
Представляет собой дисковидный бронзовый резервуар со знаком Сана-сновидца, разделённый на 4 части.
Каждая часть имела собственный клапан и содержимое.
Белый Сан "--- успокоительное (масло мяты), Красный Сан "--- наркотическое обезболивающее (масло молитвенных маков), синий Сан "--- парализатор (эссенция кураре), чёрный Сан "--- яд (лаковый сок).
Амулет был снабжён механизмом, позволяющим вращение в любую сторону.
Жрецы сели очень ловко крутили его в руках, и психически неустойчивые больные (дети и старики) часто не знали, какой Сан им достанется.
\item[Ароматный цвет] (цвет <<скального аромата>>) "--- длинноволновое ультрафиолетовое излучение, воспринимаемое зрительным пигментом тси (F13) с максимумом поглощения в районе 370 нм.
Название пошло от растения, цветы которого отражает почти исключительно волны этого диапазона.
\item[Би] "--- всеобщий язык Древней Земли, попытка связать язык машины и язык биологической нейросети.
Был разработан за тысячу пятьсот лет до гибели цивилизации Древней Земли.
\item[Вегетекторы] "--- грибы и растения, вырастающие в полноценное жилище.
Были очень популярны на Тси-Ди в период романтизма.
Впоследствии от них отказались из-за чересчур сложного процесса программирования и длительного роста.
\item[Ветер Духов] "--- см. \textit{Протокол <<Тайфун>>}.
\item[Война Тараканов] "--- мировая война на планете Ди.
Вскоре после заселения планеты произошло крупное столкновение между колониальными апидами и сапиентами-млекопитающими.
Люди, кани и планты были уничтожены почти поголовно.
Спасение для этих видов пришло через долгое время, как ни странно, от самих же апид "--- несколько особей способных к размножению апид, называемых Уродами или Тараканами, начали вести подрывную деятельность в колониях и спасать себе подобных.
Тараканы заключили союз с млекопитающими, и во время ядерной, а затем и химической войны колониальные апиды были побеждены.
Победители вынуждены были уйти с сожжённой планеты на её необитаемый близнец "--- планету Тси, давшую название новому союзу.
\item[Гаруспик] (\textit{t-sl:} harrhuspic) "--- в государстве Талиа оракул, якобы предсказывающий будущее под действием наркотического яда гриба Harrha sangvina (<<кровоточащие потроха>>).
Гаруспиками часто становились беспризорники или выходцы из бедных семей.
Это позволяло им и их семьям вести сытую и безбедную жизнь, но имело свою цену "--- мало кто из гаруспиков доживал до двадцати лет.
\item[Демонизация] "--- оцифровка сапиентного мозга, наиболее сложная и наукоёмкая разновидность оцифровки.
Также включает в себя стабилизацию (устранение или замедление естественной изнашиваемости), первичную коррекцию личности, присоединение нейронной сети к модулям хоргета и управляющему интерфейсу.
Демонизация, однако, гораздо проще написания ядра демона с нуля, из-за чего оцифровка сапиентов и приобрела большую популярность в Ордене Преисподней.
\item[Змеистая письменность] "--- единственный в своём роде звуко-буквенный вид письменности, используемый тси-язычными племенами.
Используется Молчащими идолами, говорящими на языке хесетрон.
Автор неизвестен.
Основан на математических символах древних тси.
В настоящее время упрощённый вариант используется диктиологами Ордена как фонетический алфавит тси-подобных языков.
\item[Знак кирпича] "---
\item[Интерфекция] "--- междисциплинарная отрасль науки и технологии.
Основной задачей интерфекции является эффективное разрушение либо нарушение работы биологических и небиологических систем и сетей при минимальном воздействии на окружающую инфраструктуру.
Несмотря на то, что сам термин в первую очередь используется военными, методы интерфекции используются во всех сферах деятельности сапиентов "--- в производстве пищевых материалов, медицине, строительстве, машиностроении и программировании.
\item[Истинное равновесие Нэша] "--- состояние систем с бессмертными элементами (группами элементов), при котором один элемент (группа элементов) не способен сдвинуть другие, не причинив вреда себе.
Может быть неподвижным или колеблющимся.
\item[Каменная ярость] (синонимы: ярость каменных духов, великий жар камней, око земли, проклятие пещер) "--- экстремальное ультрафиолетовое, рентгеновское и гамма-излучения, воспринимаемые специальными пигментными кристаллами глаз и кожи тси.
Субъективно вызывают у тси сильное неприятное ощущение и чувство страха.
\item[Каменный жар] "--- коротковолновое ультрафиолетовое излучение, воспринимаемое зрительным пигментом тси (F18) с максимумом поглощения в районе 200 нм.
При высокой интенсивности субъективно вызывает у тси неприятное чувство, заставляющее их укрыться от источника.
\item[Картель] (Красный Картель, \textit{sl:} Cartell Rosa) "--- крупнейшая организация минус-хоргетов.
Основана была демонами планет Нимб-3, Океан и Чёрная скала.
Впоследствии Красный Картель вобрал в себя такие крупные объединения минус-хоргетов, как Союз Воронёной Стали и Вечность.
Основная эмблема "--- три круга в треугольнике (три красных гиганта "--- солнца союзных планет).
\item[Кихотр] (\textit{tn:} **** "--- <<камень игры>>) "--- восьмигранный камень для игры в кости.
Согласно легенде, Безумный использует для принятия решений кихотр в виде идеально круглого шара, что является намёком на абсолютную случайность и бессмысленность его действий.
В переносном смысле (чаще с негативным оттенком) "--- невероятное стечение обстоятельств.
\item[Кольцевая теплица] "--- живое существо, разработанное на Тси-Ди.
Его назначение --- обеспечение тси и других животных пищей на орбитальных станциях, в межзвёздных кораблях и на инопланетных базах.
Поставляет аналоги растительных волокон и животных белков, полный спектр аминокислот и витаминов.
Способно использовать любые растворимые в воде, кислотах и предельных углеводородах минералы, а также впадать в спячку при неблагоприятных условиях.
Ад объявил крупную награду за копию семян и другие важные сведения об этих существах.
\item[Комбинезон-микроклимат] "--- костюм, собирающий и перерабатывающий выделения организма.
Позиционировался как изобретение, сохраняющее время для важных дел вместо физиологических отправлений;
однако было показано, что тси, долгое время носившие костюм, в конце концов оказывались неспособны без него существовать.
Впоследствии от этой технологии отказались, а выражение стало нарицательным для технологических средств, входящих в мутуалистические отношения с сапиентом.
\item[Конденсатор] "--- устройство, предназначенное для поддержания оптимальной влажности в помещениях.
Представляет собой волосяной психрометр, запускающий систему охлаждения.
Система охлаждения бывает электрическая (эфирный или спиртовой контур) или простая (серебряная пластинка, опущенная в наполненный снегом сосуд).
\item[Кукхватр] (\textit{tn:} <<сумеречная сталь>>) "--- деструктурированный стабитаниум, высоко ценившийся народами Тра-Ренкхаля как металл для инструментов и оружия.
Слитки кукхватра ходили в обороте по цене золота, а в некоторых регионах и выше.
Согласно одной из версий, название пошло из-за тёмного искрящегося налёта полимера, который образовывался на поверхности расплава стабитаниума.
\item[Кутрап] (\textit{tn:} ***** "--- <<мутная голова>>) "--- у народа сели: человек, совершивший четыре акта Насилия, Насилие и Разрушение, либо два акта Разрушения.
Кутрапам при поимке предоставлялся выбор: либо простая казнь, либо служение на благо народа "--- смерть на алтаре.
\item[Кхене] (\textit{tn:} **** "--- <<быстрый бег>>) "--- единица измерения расстояния, примерно 2500 метров.
\item[Лесные духи] "--- пантеон маленьких божеств, которые, согласно верованиям сели, обитают в <<жилах джунглей>> (корни некоторых родов деревьев, которые простираются на огромные расстояния и временами выходят на поверхность).
Всего у народа сели насчитывается 124 лесных духа.
Каждый дух имеет тотемные дерево, животное и насекомое.
Отличие лесных духов сели от божеств соседних племён заключается в том, что они \textit{всегда} стремятся доставить человеку радость и облегчить его страдания.
Лесные духи не способны сердиться, обижаться и мстить;
страдания человека объяснялись или прихотью Безумного, или усталостью и занятостью лесного духа.
После повсеместного насаждения культа Безумного вера в маленьких добрых божков стала уделом простонародья.
\item[Манэ] "--- чёрная или тёмно-зелёная тушь для ресниц, используемая воинами сели для наведения тени на яркое глазное яблоко, выдающее воина в темноте.
Часто использовалась воинами-оцелотами в сочетании с лимнэ "--- камуфляжной повязкой на глаза с козырьком (из-за яркой опалесценции глаз).
\item[Мелиорация] планетарная "--- процесс подготовки планеты к превращению в источник масс-энергии.
Включает коррекцию климата, уничтожение враждебных Девиантных Ветвей, Роев, а также биологическую и культурологическую обработку сапиентов.
В Ордене Преисподней мелиорацией занимался отдел 125 (упразднён), в Красном Картеле "--- Манипула Смеха (впоследствии превращена в карательное подразделение).
\item[Микханская тайнопись] "--- записываемый змеистым письмом код, образцы найдены в катакомбах деревни Микхан в Западном Живодёре.
В основном представляет собой картинки с подписями.
Предположительно является примитивным спиральным кодом, хотя некоторые исследователи склоняются к мысли, что тексты "--- автоматическое письмо и отвлекают внимание от рисунков, в которых содержится основной смысл посланий.
Микханская тайнопись не дешифрована по сей день.
В переносном значении "--- нечто недоступное пониманию и, возможно, бессмысленное.
\item[Мундир] стрелохвостовый "--- подобие одежды из обработанной акульей кожи.
Состоит из жилета и кушака.
Жилет имеет несколько карманов для мелких вещиц, зев карманов обращён против направления движения.
На жилете также выбиваются отличительные знаки.
Кушак защищает хвост от травм при чересчур резких манёврах, а также имеет венчик из акульих зубов, служащий грозным оружием в схватках.
\item[Нимелто] (\textit{ru:} nmelto "--- <<маленький бык>>, <<телёнок>>) "--- одомашненное жвачное млекопитающее с планеты Мороз системы Арракиса, отличающееся очень высокой устойчивостью к холоду благодаря микроструктуре меха и кожи.
Самцы отличаются от самок небольшими опушёнными рожками. Некоторые особи способны переносить температуры порядка 190°К.
\item[Орден Преисподней] (\textit{sd:} afir so’tid, \textit{sl:} Ord Inferna) "--- крупнейшее в известной Вселенной объединение плюс-хоргетов.
Возникло на планете Преисподняя, отличавшейся сильной вулканической активностью.
Основная эмблема "--- стилизованный извергающийся вулкан.
\item[Отряд чести] "--- бродячий Храм, не привязанный к определённому городу.
Отряды чести сопровождали колонистов, торговцев и беженцев.
К моменту войны с Безумным большая часть отрядов чести превратилась в наёмные войска, несмотря на то, что многие продолжали называть себя Храмами.
Отряд чести мог иметь как оба Этажа, так и только один.
Примеры <<одноэтажных>> отрядов чести "--- отряд Митхэ ар'Кахр, состоявший только из воинов, и Люди Золотой Пчелы "--- Храм бродячих жрецов.
\item[Оцелотовость] "--- своеобразная окраска некоторых групп народа сели, живущих в Пыльном Предгорье.
Кожа оцелотовых людей имеет слабую оранжевую окраску, также на коже имеются характерные пятна, напоминающие пятна оцелота "--- ярко-оранжевые в центре и чёрные по краям.
Форма пятен обычно совпадает с зонами иннервации определённых нервов.
Вызвана мутацией одного из генов меланопротеида "--- белок приобрёл способность связывать ртуть, содержание которой в водах Пыльного Предгорья сильно повышено.
Без потребления ртутной воды оцелотовые люди приобретают естественную золотистую окраску, но чёрные пятна остаются на всю жизнь.
\item[Оцифровка] "--- процесс, заключающийся в считывании квантовой структуры объекта, составлении математической модели и последующий перенос модели в память хоргета.
Оцифровка сапиентного мозга называется демонизацией.
\item[Парсак] (\textit{sd:} parsak "--- этимология неизвестна) "--- единица измерения космического пространства, равная расстоянию, пройденному светом по однородному вакууму за 1 астрономический год планеты Преисподняя.
\item[Пересмешники] "--- менестрели-тси, голосом подражающие музыкальным инструментам, крикам зверей, пению птиц и шумовым звукам.
Исполняют особый музыкальный жанр "--- Песню Закрытых Глаз, состоящую из различных звуков, сложенных в особый <<дорожный>> ритм.
\item[Пламя Антареса] (антарида) "--- плазменная форма жизни, встречающаяся на некоторых звёздах.
Отличается огромной скоростью метаболизма и высокой устойчивостью к воздействию среды по сравнению с обычным огнём.
Модифицированная форма антариды используется как оружие "--- в частности, можно настроить их на питание определённым полимером или на присутствие каких-либо веществ в определённой концентрации.
Антарида Безумного бога "--- чрезвычайно агрессивная, но неустойчивая форма.
Проектирование и использование антарид запрещено законодательствами Ада и Картеля вследствие непредсказуемости их изменчивости (Оборонительный Кодекс Ада, 12A.0; Politica Of-De, Alb. 1162).
Тем не менее на некоторых звёздах и даже планетах есть питомники антарид, в которых эти формы жизни разводятся в исследовательских целях.
\item[Плачущий Ягуар] (\textit{tn:} kch\`{o}h\^{o}) "--- воин, следующий философии Пути.
Эта философия подразумевала безусловное уважение к чести и жизни врага.
Если воин наносил на лицо окраску Плачущего Ягуара, он брал на себя обязательство не обнажать оружие на переговорах (исключение "--- самозащита), не убивать сдавшихся и безоружных.
Отринуть эту клятву было нельзя "--- если воин не исполнял обязательств, его подвергали остракизму.
Во времена войны с Безумными из-за постоянных межплеменных и междоусобных столкновений, а также расцвета наёмничества философия потеряла популярность, и Плачущих Ягуаров на всей Короне можно было пересчитать по пальцам.
Тем не менее их знали если не в лицо, то по слухам, и эти воины пользовались таким уважением, что иногда им сохраняли жизнь даже заклятые враги.
Один из самых известных Плачущих Ягуаров того времени "--- Митхэ ар’Кахр э’Тхартхаахитр.
\item[Поддерживаемый язык] "---
\item[Пудра вулканическая] "--- на Преисподней: микрочастицы, обладающие намагниченностью и меняющие магнитную ось под воздействием света.
Вулканическая пудра "--- основа явления, известного как <<закатные аспиды>>.
Как следует из названия, традиционно их появление приписывалось физико-химическим реакциям во время извержения вулкана, и только впоследствии было доказано, что происхождение частиц имеет биологическую природу (скелеты раковинных архей, живущих в вулканических источниках).
\item[Радужное безумие] "--- комплекс воздействия Безумного бога на сапиентов Тра-Ренкхаля.
Включал в себя создание агрессивных короткоживущих антарид и распыление в воздухе МПДЛ.
\item[Реакция Стлока] "--- вегетативная реакция инкарнированного хоргета (облачный тип) на враждебные эманации.
Связана с временной потерей контакта между демоном и сапиентным телом.
Может проявляться различными симптомами "--- от тахикардии и мочеизнурения до нейрогенной лихорадки.
Ранее реакцию Стлока использовали для поиска шпионов среди сапиентов, в новейших версиях хоргетов эта ошибка скорректирована или исправлена полностью.
Открыта биологом Картеля по имени Стлок Морской Прибой.
\item[Резы] "--- подобие письменности у стрелохвостов.
Представляет собой нанизанные на верёвку каменные бусины, косточки, куски акульей чешуи и обработанной кожи с насечками.
Есть данные, что у стрелохвостов Голубого Зеркала имеется <<библиотека>> резов "--- Грот, в которой собраны работы по медицине, истории и промыслам.
\item[Святилище] "--- поселение с особым статусом, находящееся чаще всего на границе земель.
В святилище запрещено ношение оружия "--- его сдают у ворот и хранят на специальных складах.
Скрытое ношение оружия, драки и оскорбления в святилище приравниваются к двум Разрушениям и караются смертью.
Управляет святилищем особый вид совета "--- Сцепленные Руки.
Святилища можно даже считать отдельными городами-государствами, так как формально Сцепленные Руки не подчиняются лидерам составляющих город племён.
Всего на Тра-Ренкхале насчитывается семь действующих святилищ: Омут Духов (сели и идолы Живодёра), Тёплый Двор (сели и пылерои), Весёлый Волок (сели, тенку и ркхве-хор), Одинокий Столб (сели и хака), Ожидание Вести (сели, ноа и трами), Гора Песнопений (ркхве-хор и Синее колено), Прибой (тенку, зизоце и ркхве-хор).
\item[Сейхмар] (\textit{sd:} dzaiku-maru "--- <<мелочь, побрякушка, фурнитура>>, \textit{sl:} flor "--- <<цветок>>) "--- в широком смысле "--- любой не занятый демоном сапиент, в узком смысле "--- зародыш, детёныш сапиента.
\item[Си] (Си-поинт) "--- всеобщий машинный язык, созданный на Древней Земле.
Может быть использован практически для любого типа квантовых, световых и электронных архитектур.
Используется до сих пор при программировании хоргетов.
\item[Смешливая Топь] "--- биогеоценоз в Западном Суболичье.
Микрофлора имеет уникальную особенность "--- круговорот азота идёт через стадию свободного оксида азота (I).
Смешливая Топь опасна для практически любого животного Ветвей Земли;
тем не менее, там проживают 144 вида девиантных эндемиков-стенофагов, связанных в сложную пищевую цепь.
Орден Преисподней объявил Смешливую Топь объектом живой природы, нуждающимся в наблюдении и охране.
\item[Солнечная кисть] "--- средневолновое ультрафиолетовое излучение, воспринимаемое зрительным пигментом тси (F15) с максимумом поглощения в районе 300 нм.
\item[Сосо'мар] (\emph{nl.} submarino) "--- один из древнейших способов преодоления экваториальных вод.
Рыбацкий баркас ноа имеет форму двойной погружающейся лодки с навесом из ткани, концы которой опущены в воду.
Между лодками находится трюм "--- деревянная клетка, в которую сбрасывается пойманная рыба.
Если судно к утру не успевает вернуться в порт, то моряки опускают лодки до так называемой <<линии дельфина>> "--- уровня, когда вода захлёстывает верхнюю палубу, "--- одеваются в костюмы, сделанные из акульей кожи, и спускаются в трюм.
Дышат обычно через специальные маски, состоящие из кишки, пропущенной через слой прохладной воды, с мешком-конденсатором.
Питаются рыбой, которую вялят на высунутых наружу палках, пьют подсоленный конденсат из мешков.
Некоторые суда имеют гребной винт, приводимый в движение механизмами из трюма, что позволяет двигать судно даже в таком состоянии.
Обычно после трёх-четырёх дней плавания сосо'мар судно приходило в негодность;
тем не менее, согласно отчётам Коричного флота, этого срока хватало для того, чтобы преодолеть экваториальную зону с севера на юг.
Моряки ноа, пережившие экваториальные воды, пользовались почётом и уважением.
Они имели право на бесплатную чашу выпивки в любой таверне и на одну любую вещь не дороже золотого грана у любого торговца, что легко окупало затраты на плавание.
\item[Спичечная технология] "--- высокотехнологичные устройства, которые можно сделать из широкого спектра подручных материалов и с минимальной инфраструктурной поддержкой.
Термин придумал Михаил Кохани, изобретатель <<спичечного самолёта>> и <<спичечной винтовки>>.
\item[Стабитаниум] (\textit{sl:} [stabi]la "--- <<стойкий>> и ti[tanium] "--- <<титан>>) "--- сплав титана с памятью формы, имеющий в составе более 70 легирующих добавок (в том числе вольфрам и хром), стабилизированный волокнистыми кристаллами и полимерами кремния.
Количество марок стабитаниума в настоящее время более шестисот, характеристики марок различаются очень значительно, что делает его лидером среди металлов на богатых титаном планетах.
Заменители стабитаниума, изготовленные из других металлов и/или в пропорциях, зависящих от местной встречаемости металлов, называются сурроганиумами.
\item[Талиа] (\textit{t-sl:} Talia "--- <<чрево [земли]>>) "--- крупное государство с монархическим строем на планете Драконья Пустошь, родном мире Аркадиу Люпино.
\item[Телльн] (\textit{sd:} tel’n "--- <<серёжка>>) "--- время полного оборота оси Преисподней в результате прецессии, единица измерения времени (примерно 93 тыс. земных лет).
\item[Тень] "--- слепок нервной системы сапиента, состоящий из материалов, отличных от его собственных клеток (электроника, генетически модифицированные клетки).
Тени создавали некоторые высокоразвитые цивилизации, они были способом продления жизни индивида.
В строгом смысле тенью может считаться также сапиент, оцифрованный в хоргета.
\item [Терракотовый волк] "--- образное название атомного оружия.
Происхождение связывают с легендой планеты Запах Воды о глинистом холме, который атомным взрывом превратило в похожую на волка терракотовую статую.
Другая версия, однако, гласит, что фразеологизм пошёл из Общего языка планеты Тысяча Башен и его происхождение связано с табуированием слова <<ядерный гриб>> (nugvar-mikke) и заменой его на анаграмму <<глиняный волк>> (mugnar vikke).
Однако в настоящее время вторая версия не поддерживается, так как нет никаких данных о том, что на Тысяче Башен когда-либо использовалось атомное оружие.
\item[Тси] (\textit{hy:} qi "--- <<мощь>>, <<сила>>) "--- высокоразвитая цивилизация на двойной планете Тси-Ди системы Проксима Центавра, время существования "--- 2Т "--- абсолютный рекорд за всю историю.
Первые и единственные, кто сконструировал планетную защитную систему против хоргетов.
Гибель цивилизации наступила после ошибки в коде Машины, управляющего компьютера планеты.
Попытка отключить управляющий компьютер закончилась войной, в которой тси, как предполагалось ранее, были истреблены поголовно.
В настоящий момент система Тси-Ди "--- единственный мир, в котором живут свободные Машины, и единственный мир, практически недоступный для хоргетов.
Технологические данные по Тси-Ди засекречены и обрабатываются закрытой службой "--- отделом 104, который отчитывается перед Советом только по запросам девятой степени важности.
\item[УИД] "--- устройство, имитирующее деятельность.
Механизм, работа которого не преследует никаких целей, за исключением эстетической. На Тси-Ди УИД были неотъемлемой частью стиля Механик "--- архитектуры, дизайна и скульптуры.
Орденом Преисподней УИД используются в важных оборонных узлах и программных блоках "--- чтобы запутать возможного диверсанта и логически усложнить систему для стороннего наблюдателя.
Программные аналоги УИД называются \textbf{хинду-лианами} или просто \textbf{лианами} "--- они как бы <<оплетают>> код, скрывая его истинные очертания.
\item[Утиная маска] --- атрибут жреца сели, в основном врачей и жертвенников.
Предназначена для защиты от патогенной микрофлоры, токсичных газов, аэрозолей и пыли.
\item[Фаланга] (более точный перевод "--- <<палец>> или <<указательный палец>>) "--- колюще-режущее оружие ближнего боя, полуторапядевый клинок на изогнутой четырёхпядевой рукояти.
Фаланга была основным оружием воинов, так как позволяла держать на расстоянии ножи, была достаточно манёвренным против копья и стоила гораздо дешевле цельнометаллической сабли.
\item[Фидены] "--- генетически модифицированные солдаты Пятого императора Плеяд, отличавшиеся огромной силой, ловкостью и жестокостью.
Были генетически запрограммированы на безусловное подчинение Голосу императора.
Очень часто фидены запускались во враждебное общество, где создавали семьи и давали потомство.
Спустя несколько поколений потомки воинов <<активировались>> подосланным диверсантом, и в обществе устанавливалась подконтрольная императору военная диктатура.
Разработка фиден-подобных генетических паттернов (ФПГП), а также создание клеток или вирусных векторов с ФПГП на территории Ордена Преисподней расцениваются как сотрудничество с Картелем и караются немедленным уничтожением (Оборонительный кодекс Ада, 12F.2).
\item[Фиола] (\textit{t-sl:} fiolla) "--- сосуд из кварцевого стекла в металлической оплётке.
Аналог кошелька на Драконьей Пустоши, использовался для хранения и передачи ртути.
Фиола выдавалась всем мужчинам на совершеннолетие.
Потерять, украсть, отнять или разбить фиолу считалось величайшим бесчестьем и несчастливым знамением.
Даже разбойники оставляли своей жертве (нередко мёртвой) её фиолу, забирая лишь находившуюся в сосуде ртуть.
Этот предмет нашёл отражение в оборотах: <<продать фиолу>> "--- нищенствовать, опуститься;
<<хранить дома чужие фиолы>> "--- быть беспринципным, ценить деньги превыше чужого благосостояния;
<<слушать бульканье фиолы при ходьбе>> "--- испытывать финансовые затруднения.
\item[Хоргет] (\textit{sd:} horohito "--- <<бесчеловечный>>, <<нелюдь>>) "---
\item[Хасетрасем] (\textit{cn:} <<лицо, начерченное в воздухе>>) "--- особая кодовая фраза в языках сели, ноа и трами.
Представляет собой биометрическое описание внешности сапиента, включающее его мимику, особенности движений и голоса.
Зная хасетрасем, сели могли найти нужного человека даже в ночной толпе.
В школе несколько лет обучения посвящались искусству его составления.
Вероятно, что метод был разработан последними тси в рамках подготовки к одичанию.
Сами тси для этих целей использовали специальные технологические средства.
\item[Хэситр] (\textit{tn:} ***** "--- <<напиток спокойствия>>) "--- ритуальная чаша с водой, заменявшая погребение.
Если сели знал, что его не смогут похоронить с соблюдением всех ритуалов, то перед смертью он выпивал хэситр и умирал спокойно, зная, что найдёт пристанище лесных духов.
Также считалось приемлемым выливать хэситр в рот уже умершего человека.
Ранее такие чаши делались из благородных деревьев и украшались резьбой, впоследствии сели стали использовать как хэситр любую чашу, нанося на неё охранные знаки.
Использовались хэситры на войне, во время стихийных бедствий (см. \textit{Радужное безумие}), а также людьми, совершающими <<последнюю беседу с собой>> (самоубийство).
\item[Цитра Ветра] (тротрис, \textit{tn:} trotris) "--- десятиструнный музыкальный инструмент сели (4 гладких струны + 6 витых).
Название получил по имени его предположительного изобретателя, некой Ветер-Дующий-Ниоткуда (легендарный бард, любовница купца Чхаласа).
Для игры использовались специальные перчатки с дополнительными <<пальцами>> или встроенная в гриф каретка.
В деке имелся механический смычок для гладких струн, приводимый в движение маховиком и ногой барда.
К моменту войны с Безумным на Короне осталось всего два мастера, которые умели делать эти инструменты, оба погибли во время войны.
Но образцы цитры Ветра попали к культурологам Ада, которые смогли воссоздать технологию изготовления.
\item[Чётки Сата] "--- верёвочные карты дорог, бывшие в большом ходу у торговцев-сели.
Представляли из себя цветные верёвки, связанные между собой, с узелками или бусинами;
количество бусин равнялось количеству верстовых столбов.
Торговцы в дороге постоянно держали их в руках и пальцами отсчитывали пройденные кхене.
\item[Эй] "--- на Древней Земле: гипотетический <<универсальный язык>>.
Впоследствии: созданный Ликаном Безруким поддерживаемый язык, не претендующий на универсальность, но весьма удобный и потому распространившийся по Вселенной.
\item[Эмоглиф] "--- специальный элемент письменности сели для записи настроения или отношения сапиента к написанному.
Соответствует устаревшей части речи языка тси "--- статусу.
Предусматривает 4096 оттенков эмоций, универсален для всех сапиентных видов.
\item[Эпоха богов] (\textit{en:} gottage) "---
\item[Эпоха демонов] (\textit{sd:} asogeite) "---
\item[Язычный рефлекс] "--- специфическая реакция организма людей-тси.
Болевая стимуляция кончика языка вызывает экстренное снижение метаболизма, вплоть до <<имитации смерти>>.
\item[Янтарь] "--- особый камень, встречающийся только на Тра-Ренкхале, Хемане-2 (Хароне) и, согласно некоторым данным, на Древней Земле.
По легенде, это слёзы дерева акхкатрас, падающие в Ху'тресоааса и выносимые в Кипящее море.
В настоящее время установлено, что основой янтаря является смола не акхкатрас, а другого растения "--- голосеменного эпифита Пятикрыльник плакучий (Бенедикта).
Кислые геотермальные воды плавят смолу, спекают её с продуктами придонных моллюсков, диатомей и кольчатых червей, а затем встречное течение Могильного пролива разносит янтарь по всему побережью вплоть до Молчащих лесов.
Особенно ценным считается янтарь с вплавленными в него жемчужинами и золотыми самородками.
\item[Ярмарка хлама] "--- праздник в первый год Церемонии.
Люди вытаскивают и продают за бесценок весь хлам, который скопился в жилищах, часто находят старые клады.
В переносном смысле "--- отсутствие выбора при видимом изобилии.
\item[Ячеистое письмо] "--- торгово-дипломатическое письмо травников.
Напоминает соты.
\end{description}

\chapter{Прочее}

\section{Социум сели}

\begin{itemize}
\item Храм (the Temple)
\begin{itemize}
\item Нижний этаж (the Downstairs)
\item Верхний этаж (the Upstairs)
\end{itemize}
\item Двор (the House)
\item Цех (the Workshop)
\item Сад (the Garden)
\end{itemize}

\section{Календарь сели (доработать)}

\subsubsection{Единицы времени}

Общие для всех сели:

\begin{itemize}
\item 1 год "--- 2 дождя "--- 160,000012 дней
\item 1 день "--- 16 кхамит [шаг солнца] "--- 29,350161 часов
\item 1 кхамит "--- 64 михнет [отдых] "--- 1,834385 часов
\item 1 михнет "--- 256 секхар [взмах ресницами] "--- 0,028662 часа "--- 1,7197359375 минут
\item 1 секхар "--- 0,403063110352 секунды
\end{itemize}

Локальные:

\begin{itemize}
\item 1 кхене (как единица времени) "--- 9,2 михнет
\item 1 согхо "--- 8,5 секхар
\end{itemize}

\subsubsection{Десять декагексад (трекхсатр)}

\begin{enumerate}
\item Плуг (летняя страда)
\item Согхо
\item Лук
\item Большая Капля (сезон дождей)
\item Карп
\item Оцелот (зимняя страда)
\item Тростник
\item Змея
\item Малая Капля (сезон дождей)
\item Пирог
\end{enumerate}

\subsubsection{Праздники}

\begin{itemize}
\item Согхо 3 "--- Крылья Лю (каждые 4 года)
\item Лук 8 "--- Соревнование кулинаров в Тёплом дворе (каждые 2 года)
\item Лук 9 "--- Соревнование виноделов в Тёплом Дворе
\item Лук 10 "--- Большое похмелье.
День взаимопомощи и дружбы, праздник врачей.
\item Змея 3 "--- Мягкие Руки (танцы, оргии, открытый Круг Доверия)
\item Змея 9 (10, 11) "--- Слёзы Ситхэ
\end{itemize}

\section{Игры}

\subsection{Метритхис}

Играют 4 игрока, жребием перед игрой распределяются роли.

\begin{itemize}
\item Зелёный "--- Король
\item Чёрный "--- Мятежник
\item Красный "--- Сосед
\item Жёлтый "--- Фатум.
\end{itemize}

Подготавливает игру Фатум.
Он выкладывает определённым образом квадратики-поля:

\begin{itemize}
\item Синие "--- река/море (1 ход на переправу, нельзя атаковать с другой стороны, пенальти к защите и нападению)
\item Серые "--- горы (3 хода на переправу, преимущество в обороне и обстреле)
\item Красные "--- пустыня (пенальти ко всем видам деятельности)
\item Зелёные "--- лес (преимущество в обороне, скрытности, пенальти к нападению и обстрелу)
\item Жёлтые "--- степь (пенальти к скрытности и обороне, преимущество в нападении и обстреле)
\item Чёрные "--- болото (2 хода на переправу, преимущество к скрытности и обстрелу, пенальти к нападению и обороне)
\end{itemize}

Фигурки взаимодействуют двумя способами "--- драка и беседа.
При драке определяются победитель и побеждённый.
Побеждённый отправляется в коробку, а победитель иногда может сменить класс.
При беседе оба меняют класс, а иногда ещё и цвет.

Начало игры распределяется жребием между Королём и Мятежником.
Последним в игру вступает Сосед, время его вступления выбирает Фатум.

Всего в игре 4 поля "--- собственно игровое и 3 дипломатических.
Дипломатические могут видеть только договаривающиеся правители и Фатум.

Комбинации на поле дипломатии:

\begin{itemize}
\item Обман "--- проигравший пропускает 5 ходов против выигравшего.
\item Заговор "--- Фатум выбрасывает кихотр на смерть проигравшего.
\item Паритет "--- 5 ходов правители не ведут друг против друга боевые действия.
\item Мир "--- правители до конца игры не ведут друг против друга боевые действия, их стол дипломатии с третьим правителем становятся общим.
\end{itemize}

Если два правителя заключили мир, единственный способ Фатума выиграть "--- выбросить успешный Заговор у третьего или Чуму на одного из союзников.
Если все три правителя пытаются построить мир, Фатум пытается их убить.
Поэтому в задачи правителей входит ещё и ввести Фатум в заблуждение.

В распоряжении правителей "--- войска и камни дипломатии (чёрные и белые).
В распоряжении Фатума "--- кихотр и камни Благ и Несчастий:

\begin{itemize}
\item Верная женщина "--- отводит от правителя Заговор.
\item Чума "--- убивает войска и с некоторой вероятностью может убить правителя.
\item Удача "--- позволяет правителю взглянуть на дипломатический стол противников, выиграть в явно проигрышной стычке войск или даёт фору в 5 камней на любом столе
дипломатии.
\item Потеря друга "--- пропуск двух ходов.
\item Безумие "--- ход на поле боя или столе дипломатии вместо правителя делает Фатум.
\end{itemize}

Игра заканчивается 4 способами:

\begin{enumerate}
\item 2 правителя умирают, оставшийся в живых правитель и Фатум выигрывают.
\item 1 умирает, 2 других договариваются, Фатум проигрывает.
\item 3 правителя договариваются, Фатум проигрывает.
\item Все правители гибнут, Фатум выигрывает.
\end{enumerate}

Самая сложная задача всегда у Фатума, поэтому очень часто его роль без жребия отдают самому опытному игроку.

\subsection{Пьянка}

Шуточная игра в кости от двух до четырёх человек.
Игра идёт обычно на желания, лакомства, секс или удары, иногда на всё сразу.

Правила желаний: они не должны причинять человеку вреда.
То же самое с ударами "--- чаще всего игра идёт на пощёчины, шлепки и несильные броски.
Проигравший может поменять секс или желание на оговорённое число ударов, лакомства и алкоголь можно передавать другим игрокам.
